%!TEX root = std.tex
\setcounter{chapter}{19}
\rSec0[utilities]{General utilities library}

\setcounter{section}{1}
\rSec1[utility]{Utility components}

{\color{addclr}
\pnum
\synopsis{Header \tcode{<experimental/ranges/utility>} synopsis}
}

\begin{codeblock}
@\added{namespace std \{ namespace experimental \{ namespace ranges \{ inline namespace v1 \{}@
  // \ref{utility.swap}, swap:
  @\removed{template <class T> void swap(T\& a, T\& b) noexcept(\seebelow);}@
  @\removed{template <class T, size_t N> void swap(T (\&a)[N], T (\&b)[N]) noexcept(noexcept(swap(*a, *b)));}@
  @\added{namespace \{}@
    @\added{constexpr \unspec swap = \unspec;}@
  @\added{\}}@

  // \ref{utility.exchange}, exchange:
  template <@\changed{class}{MoveConstructible}@ T, class U=T>
    @\added{requires Assignable<T\&, U>()}@
  T exchange(T& obj, U&& new_val);
\end{codeblock}
\begin{addedblock}
\begin{codeblock}
  // \ref{taggedtup.tagged}, struct with named accessors
  template <class T>
  concept bool TagSpecifier() {
    return @\seebelow@;
  }

  template <class F>
  concept bool TaggedType() {
    return @\seebelow@;
  }

  template <class Base, TagSpecifier... Tags>
    requires sizeof...(Tags) <= tuple_size<Base>::value
  struct tagged;

  // \ref{tagged.pairs}, tagged pairs
  template <TaggedType T1, TaggedType T2> using tagged_pair = @\seebelow@;

  template <TagSpecifier Tag1, TagSpecifier Tag2, class T1, class T2>
  constexpr @\seebelow@ make_tagged_pair(T1&& x, T2&& y);
}}}}

namespace std {
  // \ref{tagged.astuple}, tuple-like access to tagged
  template <class Base, class... Tags>
  struct tuple_size<experimental::ranges::tagged<Base, Tags...>>;

  template <size_t N, class Base, class... Tags>
  struct tuple_element<N, experimental::ranges::tagged<Base, Tags...>>;
}
\end{codeblock}
\end{addedblock}

\begin{addedblock}
\pnum
Any entities declared or defined directly in namespace \tcode{std} in header \tcode{<utility>}
that are not already defined in namespace \tcode{std::experimental::ranges::v1} in header
\tcode{<experimental/ranges/utility>} are imported with
\grammarterm{using-declaration}{s}~(\cxxref{namespace.udecl}). \enterexample
\begin{codeblock}
namespace std { namespace experimental { namespace ranges { inline namespace v1 {
  using std::pair;
  using std::make_pair;
  // ... others
}}}}
\end{codeblock}
\exitexample
\end{addedblock}

\setcounter{subsection}{1}
\rSec2[utility.swap]{swap}

\begin{addedblock}
\indexlibrary{\idxcode{swap}}%
\pnum The name \tcode{swap} denotes a customization point
object~(\ref{customization.point.object}). The effect of the expression
\tcode{ranges::swap(E1, E2)} for some expressions \tcode{E1}
and \tcode{E2} is equivalent to:

\begin{itemize}
\item
  \tcode{(void)swap(E1, E2)}, with overload resolution
  performed in a context that includes the declarations
\begin{codeblock}
  template <class T>
  void swap(T&, T&) = delete;
  template <class T, size_t N>
  void swap(T(&)[N], T(&)[N]) = delete;
\end{codeblock}
  and does not include a declaration of \tcode{ranges::swap}.
  If the function selected by overload resolution does not
  exchange the values denoted by \tcode{E1} and \tcode{E2},
  the program is ill-formed with no diagnostic required.

\item
  Otherwise, \tcode{(void)swap_ranges(E1, E2)} if \tcode{E1} and
  \tcode{E2} are lvalues of array types~(\cxxref{basic.compound})
  of equal extent and \tcode{ranges::swap(*(E1), *(E2))}
  is a valid expression, except that
  \tcode{noexcept(ranges::swap(E1, E2))} is equal to
  \tcode{noexcept(ranges::swap(*(E1), *(E2)))}.

  \ednote{This formulation intentionally allows swapping arrays
  with identical extent and differing element types, but only
  when swapping the element types is well-defined. Swapping
  arrays of \tcode{int} and \tcode{double} continues to be
  unsound, but \tcode{Swappable<T\&, U\&>()} implies
  \tcode{Swappable<T(\&)[N], U(\&)[N]>()}.}

\item
  Otherwise, if \tcode{E1} and \tcode{E2} are lvalues of the
  same type \tcode{T} which meets the syntactic requirements of
  \tcode{MoveConstructible<T>()} and
  \tcode{Assignable<T\&, T>()}, exchanges the denoted values.
  \tcode{ranges::swap(E1, E2)} is a constant expression if
  the constructor selected by overload resolution for
  \tcode{T\{std::move(E1)\}} is a constexpr constructor and
  the expression \tcode{E1 = std::move(E2)} can appear in a
  constexpr function. \tcode{noexcept(ranges::swap(E1, E2))}
  is equal to \tcode{is_nothrow_move_constructible<T>::value
  \&\& is_nothrow_move_assignable<T>::value}. If either
  \tcode{MoveConstructible<T>()} or
  \tcode{Assignable<T\&, T>()} is not satisfied, the program
  is ill-formed with no diagnostic required.
  \tcode{ranges::swap(E1, E2)} has type \tcode{void}.

\item
  Otherwise, \tcode{ranges::swap(E1, E2)} is ill-formed.
\end{itemize}

\pnum
\remark Whenever \tcode{ranges::swap(E1, E2)} is a valid
expression, it exchanges the values denoted by \tcode{E1}
and \tcode{E2} and has type \tcode{void}.
\end{addedblock}

\begin{removedblock}
\begin{itemdecl}
template <class T>
void swap(T& a, T& b) noexcept(@\seebelow@);
\end{itemdecl}

\begin{itemdescr}
\pnum
\remarks The expression inside \tcode{noexcept} is equivalent to:

\begin{codeblock}
is_nothrow_move_constructible<T>::value &&
is_nothrow_move_assignable<T>::value
\end{codeblock}

\pnum
\remarks
A library implementor is free to omit the \tcode{requires} clause so long as
this function does not participate in overload resolution if the following
is false
\begin{codeblock}
is_move_constructible<T>::value &&
is_move_assignable<T>::value &&
is_destructible<T>::value
\end{codeblock}

\pnum
\requires
Type
\tcode{T}
shall be
\tcode{MoveConstructible} (Table~\cxxref{moveconstructible})
and
\tcode{MoveAssignable} (Table~\cxxref{moveassignable}).

\pnum
\effects
Exchanges values stored in two locations.
\end{itemdescr}
\end{removedblock}

\begin{removedblock}
\begin{itemdecl}
template <class T, size_t N>
void swap(T (&a)[N], T (&b)[N]) noexcept(noexcept(swap(*a, *b)));}
\end{itemdecl}

\begin{itemdescr}
\pnum
\requires
\tcode{a[i]} shall be swappable with~(\ref{concepts.lib.corelang.swappable}) \tcode{b[i]}
for all \tcode{i} in the range \range{0}{N}.

\pnum
\effects \tcode{swap_ranges(a, a + N, b)}
\end{itemdescr}
\end{removedblock}

\rSec2[utility.exchange]{exchange}

\begin{itemdecl}
template <@\changed{class}{MoveConstructible}@ T, class U=T>
  @\added{requires Assignable<T\&, U>()}@
T exchange(T& obj, U&& new_val);
\end{itemdecl}

\begin{itemdescr}
\pnum
\effects
Equivalent to:

\begin{codeblock}
T old_val = std::move(obj);
obj = std::forward<U>(new_val);
return old_val;
\end{codeblock}
\end{itemdescr}

\setcounter{section}{6}
\rSec1[memory]{Memory}
\setcounter{Paras}{0}
\pnum
\synopsis{Header \tcode{<\newtxt{experimental/ranges/}memory>} synopsis}

\begin{codeblock}
namespace std { namespace experimental { namespace ranges { inline namespace v1 {
  template <InputIterator I, Sentinel<I> S, ForwardIterator O>
    requires Constructible<value_type_t<O>, reference_t<O>>() \&\&
             Same<value_type_t<O>\&, reference_t<O>>
  tagged_pair<tag::in(I), tag::out(O)>
    uninitialized_copy(I first, S last, O result);

  template <InputRange Rng, class O>
    requires ForwardIterator<O>() \&\&
             Constructible<value_type_t<O>, reference_t<iterator_t<Rng>>>() \&\&
             Same<value_type_t<O>\&, reference_t<O>>
  tagged_pair<tag::in(safe_iterator_t<Rng>), tag::out(O)>
    uninitialized_copy(Rng\&\& rng, O\&\& result);
\end{codeblock}

\setcounter{Paras}{0}
\pnum
\effects Copies elements in the range \tcode{[first, last)} into the range
\tcode{[result, (result + (last - first))} starting from \tcode{first} and proceeding
to \tcode{last}. For each non-negative integer \tcode{n < (last - first)}, performs
\tcode{::new(static_cast<void*>(\&*(result + n)) T(*(first + n))}.

\pnum
\returns \tcode{\{last, result + (last - first)\}}.

\pnum
\requires \tcode{result} shall not be in the range \tcode{[first, last)}.

\pnum
\complexity Exactly \tcode{last - first} copy constructions.

\begin{codeblock}
  template <InputIterator I, OutputIterator O>
    requires Constructible<value_type_t<O>, reference_t<I>>() &&
             Same<value_type_t<O>&, reference_t<O>>
  tagged_pair<tag::in(I), tag::out(O)>
    uninitialized_copy_n(I&& first, difference_type_t<I> n, O&& out)
\end{codeblock}

\pnum
\effects For each non-negative integer $i < n$, performs
\tcode{::new(static_cast<void*>(\&*(result + i)) T(*(first + i))}.

\pnum
\returns \tcode{\{first + n, result + n\}}.

\pnum Exactly \tcode{n} copy constructions.

\setcounter{section}{8}
\rSec1[function.objects]{Function objects}

\setcounter{Paras}{1}
\pnum
\synopsis{Header \tcode{<\added{experimental/ranges/}functional>} synopsis}

\begin{codeblock}
@\added{namespace std \{ namespace experimental \{ namespace ranges \{ inline namespace v1 \{}@
  @\added{// \ref{func.invoke}, invoke:}@
  @\added{template <class F, class... Args>}@
  @\added{result_of_t<F\&\&(Args\&\&...)> invoke(F\&\& f, Args\&\&... args);}@

  @\added{// \ref{comparisons}, comparisons:}@
  template <class T = void> struct equal_to; @\added{// not defined}@
  template <class T = void> struct not_equal_to; @\added{// not defined}@
  template <class T = void> struct greater; @\added{// not defined}@
  template <class T = void> struct less; @\added{// not defined}@
  template <class T = void> struct greater_equal; @\added{// not defined}@
  template <class T = void> struct less_equal; @\added{// not defined}@
  @\added{template <EqualityComparable T> struct equal_to<T>;}@
  @\added{template <EqualityComparable T> struct not_equal_to<T>;}@
  @\added{template <StrictTotallyOrdered T> struct greater<T>;}@
  @\added{template <StrictTotallyOrdered T> struct less<T>;}@
  @\added{template <StrictTotallyOrdered T> struct greater_equal<T>;}@
  @\added{template <StrictTotallyOrdered T> struct less_equal<T>;}@
  template <> struct equal_to<void>;
  template <> struct not_equal_to<void>;
  template <> struct greater<void>;
  template <> struct less<void>;
  template <> struct greater_equal<void>;
  template <> struct less_equal<void>;
  @\added{template <class T> struct greater<T*>;}@
  @\added{template <class T> struct less<T*>;}@
  @\added{template <class T> struct greater_equal<T*>;}@
  @\added{template <class T> struct less_equal<T*>;}@

  @\added{// \ref{func.identity}, identity:}@
  @\added{struct identity;}@
@\added{\}\}\}\}}@
\end{codeblock}

\begin{addedblock}
\pnum
Any entities declared or defined directly in namespace \tcode{std} in header \tcode{<functional>}
that are not already defined in namespace \tcode{std::experimental::ranges} in header
\tcode{<experimental/ranges/functional>} are imported with
\grammarterm{using-declaration}{s}~(\cxxref{namespace.udecl}). \enterexample
\begin{codeblock}
namespace std { namespace experimental { namespace ranges { inline namespace v1 {
  using std::reference_wrapper;
  using std::ref;
  // ... others
}}}}
\end{codeblock}
\exitexample

\pnum
Any nested namespaces defined directly in namespace \tcode{std} in header \tcode{<functional>}
that are not already defined in namespace \tcode{std::experimental::ranges} in header
\tcode{<experimental/ranges/functional>} are aliased with a
\grammarterm{namespace-alias-definition}~(\cxxref{namespace.alias}). \enterexample
\begin{codeblock}
namespace std { namespace experimental { namespace ranges { inline namespace v1 {
  namespace placeholders = std::placeholders;
}}}}
\end{codeblock}
\exitexample
\end{addedblock}

\ednote{Before [refwrap], insert the following section and renumber subsequent sections
as appropriate. (Renumbering hasn't been performed herein to ease review.)}

\begin{addedblock}
\setcounter{subsection}{2}
\rSec2[func.invoke]{Function template invoke}
\begin{itemdecl}
template <class F, class... Args>
result_of_t<F&&(Args&&...)> invoke(F&& f, Args&&... args);
\end{itemdecl}
\begin{itemdescr}
\pnum
\effects Equivalent to \tcode{\textit{INVOKE}(std::forward<F>(f), std::forward<Args>(args)...)}~(\cxxref{func.require}).
\end{itemdescr}
\end{addedblock}

\setcounter{subsection}{4}
\rSec2[comparisons]{Comparisons}

\pnum
The library provides basic function object classes for all of the comparison
operators in the language~(\cxxref{expr.rel}, \cxxref{expr.eq}).

\indexlibrary{\idxcode{equal_to}}%
\begin{itemdecl}
template <@\changed{class}{EqualityComparable} T\removed{ = void}@>
struct equal_to@\added{<T>}@ {
  constexpr bool operator()(const T& x, const T& y) const;
  @\removed{typedef T first_argument_type;}@
  @\removed{typedef T second_argument_type;}@
  @\removed{typedef bool result_type;}@
};
\end{itemdecl}

\begin{itemdescr}
\pnum
\tcode{operator()} returns \tcode{x == y}.
\end{itemdescr}

\indexlibrary{\idxcode{not_equal_to}}%
\begin{itemdecl}
template <@\changed{class}{EqualityComparable} T\removed{ = void}@>
struct not_equal_to@\added{<T>}@ {
  constexpr bool operator()(const T& x, const T& y) const;
  @\removed{typedef T first_argument_type;}@
  @\removed{typedef T second_argument_type;}@
  @\removed{typedef bool result_type;}@
};
\end{itemdecl}

\begin{itemdescr}
\pnum
\tcode{operator()} returns \tcode{x != y}.
\end{itemdescr}

\indexlibrary{\idxcode{greater}}%
\begin{itemdecl}
template <@\changed{class}{StrictTotallyOrdered} T\removed{ = void}@>
struct greater@\added{<T>}@ {
  constexpr bool operator()(const T& x, const T& y) const;
  @\removed{typedef T first_argument_type;}@
  @\removed{typedef T second_argument_type;}@
  @\removed{typedef bool result_type;}@
};
@\added{template <class T>}@
@\added{struct greater<T*> \{}@
  @\added{constexpr bool operator()(T* x, T* y) const;}@
@\added{\};}@
\end{itemdecl}

\begin{itemdescr}
\pnum
\tcode{operator()} returns \tcode{x > y}.
\end{itemdescr}

\indexlibrary{\idxcode{less}}%
\begin{itemdecl}
template <@\changed{class}{StrictTotallyOrdered} T\removed{ = void}@>
struct less@\added{<T>}@ {
  constexpr bool operator()(const T& x, const T& y) const;
  @\removed{typedef T first_argument_type;}@
  @\removed{typedef T second_argument_type;}@
  @\removed{typedef bool result_type;}@
};
@\added{template <class T>}@
@\added{struct less<T*> \{}@
  @\added{constexpr bool operator()(T* x, T* y) const;}@
@\added{\};}@
\end{itemdecl}

\begin{itemdescr}
\pnum
\tcode{operator()} returns \tcode{x < y}.
\end{itemdescr}

\indexlibrary{\idxcode{greater_equal}}%
\begin{itemdecl}
template <@\changed{class}{StrictTotallyOrdered} T\removed{ = void}@>
struct greater_equal@\added{<T>}@ {
  constexpr bool operator()(const T& x, const T& y) const;
  @\removed{typedef T first_argument_type;}@
  @\removed{typedef T second_argument_type;}@
  @\removed{typedef bool result_type;}@
};
@\added{template <class T>}@
@\added{struct greater_equal<T*> \{}@
  @\added{constexpr bool operator()(T* x, T* y) const;}@
@\added{\};}@
\end{itemdecl}

\begin{itemdescr}
\pnum
\tcode{operator()} returns \tcode{x >= y}.
\end{itemdescr}

\indexlibrary{\idxcode{less_equal}}%
\begin{itemdecl}
template <@\changed{class}{StrictTotallyOrdered} T\removed{ = void}@>
struct less_equal@\added{<T>}@ {
  constexpr bool operator()(const T& x, const T& y) const;
  @\removed{typedef T first_argument_type;}@
  @\removed{typedef T second_argument_type;}@
  @\removed{typedef bool result_type;}@
};
@\added{template <class T>}@
@\added{struct less_equal<T*> \{}@
  @\added{constexpr bool operator()(T* x, T* y) const;}@
@\added{\};}@
\end{itemdecl}

\begin{itemdescr}
\pnum
\tcode{operator()} returns \tcode{x <= y}.
\end{itemdescr}

\indexlibrary{\idxcode{equal_to<>}}%
\begin{itemdecl}
template <> struct equal_to<void> {
  template <class T, class U>
    @\added{requires EqualityComparable<T, U>()}@
  constexpr auto operator()(T&& t, U&& u) const
    -> decltype(std::forward<T>(t) == std::forward<U>(u));

  typedef @\unspec@ is_transparent;
};
\end{itemdecl}

\begin{itemdescr}
\pnum
\tcode{operator()} returns \tcode{std::forward<T>(t) == std::forward<U>(u)}.
\end{itemdescr}

\indexlibrary{\idxcode{not_equal_to<>}}%
\begin{itemdecl}
template <> struct not_equal_to<void> {
  template <class T, class U>
    @\added{requires EqualityComparable<T, U>()}@
  constexpr auto operator()(T&& t, U&& u) const
    -> decltype(std::forward<T>(t) != std::forward<U>(u));

  typedef @\unspec@ is_transparent;
};
\end{itemdecl}

\begin{itemdescr}
\pnum
\tcode{operator()} returns \tcode{std::forward<T>(t) != std::forward<U>(u)}.
\end{itemdescr}

\indexlibrary{\idxcode{greater<>}}%
\begin{itemdecl}
template <> struct greater<void> {
  template <class T, class U>
    @\added{requires StrictTotallyOrdered<T, U>()}@
      @\added{|| \textit{BUILTIN_PTR_CMP}(T, >, U) // \expos, \seebelow}@
  constexpr auto operator()(T&& t, U&& u) const
    -> decltype(std::forward<T>(t) > std::forward<U>(u));

  typedef @\unspec@ is_transparent;
};
\end{itemdecl}

\begin{itemdescr}
\pnum
\tcode{operator()} returns \tcode{std::forward<T>(t) > std::forward<U>(u)}.
\end{itemdescr}

\indexlibrary{\idxcode{less<>}}%
\begin{itemdecl}
template <> struct less<void> {
  template <class T, class U>
    @\added{requires StrictTotallyOrdered<T, U>()}@
      @\added{|| \textit{BUILTIN_PTR_CMP}(T, <, U) // \expos, \seebelow}@
  constexpr auto operator()(T&& t, U&& u) const
    -> decltype(std::forward<T>(t) < std::forward<U>(u));

  typedef @\unspec@ is_transparent;
};
\end{itemdecl}

\begin{itemdescr}
\pnum
\tcode{operator()} returns \tcode{std::forward<T>(t) < std::forward<U>(u)}.
\end{itemdescr}

\indexlibrary{\idxcode{greater_equal<>}}%
\begin{itemdecl}
template <> struct greater_equal<void> {
  template <class T, class U>
    @\added{requires StrictTotallyOrdered<T, U>()}@
      @\added{|| \textit{BUILTIN_PTR_CMP}(T, >=, U) // \expos, \seebelow}@
  constexpr auto operator()(T&& t, U&& u) const
    -> decltype(std::forward<T>(t) >= std::forward<U>(u));

  typedef @\unspec@ is_transparent;
};
\end{itemdecl}

\begin{itemdescr}
\pnum
\tcode{operator()} returns \tcode{std::forward<T>(t) >= std::forward<U>(u)}.
\end{itemdescr}

\indexlibrary{\idxcode{less_equal<>}}%
\begin{itemdecl}
template <> struct less_equal<void> {
  template <class T, class U>
    @\added{requires StrictTotallyOrdered<T, U>()}@
      @\added{|| \textit{BUILTIN_PTR_CMP}(T, <=, U) // \expos, \seebelow}@
  constexpr auto operator()(T&& t, U&& u) const
    -> decltype(std::forward<T>(t) <= std::forward<U>(u));

  typedef @\unspec@ is_transparent;
};
\end{itemdecl}

\begin{itemdescr}
\pnum
\tcode{operator()} returns \tcode{std::forward<T>(t) <= std::forward<U>(u)}.
\end{itemdescr}

\pnum
For templates \tcode{greater}, \tcode{less}, \tcode{greater_equal}, and
\tcode{less_equal}, the specializations for any pointer type yield a total order,
even if the built-in operators \tcode{<}, \tcode{>}, \tcode{<=}, \tcode{>=}
do not. \ednote{The following sentence is taken from the proposed resolution of
\href{https://cplusplus.github.io/LWG/lwg-active.html\#2450}{LWG \#2450}.}
\added{For template specializations \tcode{greater<void>}, \tcode{less<void>},
\tcode{greater_equal<void>}, and \tcode{less_equal<void>}, if the call operator calls a
built-in operator comparing pointers, the call operator yields a total order.}

\begin{addedblock}
\pnum
If \tcode{X} is an lvalue reference type, let \tcode{x} be an lvalue of type \tcode{X}, or an rvalue otherwise.
If \tcode{Y} is an lvalue reference type, let \tcode{y} be an lvalue of type \tcode{Y}, or an rvalue otherwise.
Given a relational operator \tcode{\textit{OP}}, \tcode{\textit{BUILTIN_PTR_CMP}(X, \textit{OP}, Y)}
shall be \tcode{true} if an only if \tcode{\textit{OP}} in the expression \tcode{(X\&\&)x \textit{OP} (Y\&\&)y}
resolves to a built-in operator comparing pointers.

\pnum
All specializations of \tcode{equal_to}, \tcode{not_equal_to}, \tcode{greater}, \tcode{less},
\tcode{greater_equal}, and \tcode{less_equal} shall satisfy \tcode{DefaultConstructible}~(\ref{concepts.lib.object.defaultconstructible}).

\pnum
For all object types \tcode{T} for which there exists a specialization \tcode{less<T>},
the instantiation \tcode{less<T>} shall satisfy \tcode{StrictWeakOrder<less<T>, T>()}~(\ref{concepts.lib.callables.strictweakorder}).

\pnum
For all object types \tcode{T} for which there exists a specialization \tcode{equal_to<T>},
the instantiation \tcode{equal_to<T>} shall satisfy \tcode{Relation<equal_to<T>, T>()}~(\ref{concepts.lib.callables.relation}),
and \tcode{equal_to<T>} shall induce an equivalence relation on its arguments.

\pnum
For any (possibly \tcode{const}) lvalues \tcode{x} and \tcode{y} of types \tcode{T}, the following
shall be true

\begin{itemize}
\item If there exists a specialization \tcode{not_equal_to<T>}, then the instantiation \tcode{not_equal_to<T>}
      shall satisfy \tcode{Relation<not_equal_to<T>, T>()}, and \tcode{not_equal_to<T>\{\}(x, y)} shall
      equal \tcode{!equal_to<T>\{\}(x, y)}.
\item If there exists a specialization \tcode{greater<T>}, then the instantiation \tcode{greater<T>}
      shall satisfy \tcode{Strict\-Weak\-Order<greater<T>, T>()}, and \tcode{greater<T>\{\}(x, y)} shall
      equal \tcode{less<T>\{\}(y, x)}.
\item If there exists a specialization \tcode{greater_equal<T>}, then the instantiation \tcode{greater_equal<T>}
      shall satisfy \tcode{Relation<greater_equal<T>, T>()}, and \tcode{greater_equal<T>\{\}(x, y)} shall
      equal \tcode{!less<T>\{\}(x, y)}.
\item If there exists a specialization \tcode{less_equal<T>}, then the instantiation \tcode{less_equal<T>}
      shall satisfy \tcode{Relation<greater_equal<T>, T>()}, and \tcode{less_equal<T>\{\}(x, y)} shall
      equal \tcode{!less<T>\{\}(y, x)}.
\end{itemize}

\pnum
For any pointer type \tcode{T}, the specializations \tcode{equal_to<T>}, \tcode{not_equal_to<T>}, \tcode{greater<T>},
\tcode{less<T>}, \tcode{greater_equal<T>}, \tcode{less_equal<T>} shall yield the same results as
\tcode{equal_to<void*>}, \tcode{not_equal_to<void*>}, \tcode{greater<void*>},
\tcode{less<void*>}, \tcode{greater_equal<void*>}, \tcode{less_equal<void*>}, respectively.
\end{addedblock}

\ednote{After subsection 20.9.12 [unord.hash] add the following subsection:}

\setcounter{subsection}{12}
\begin{addedblock}
\rSec2[func.identity]{Class identity}

\indexlibrary{\idxcode{identity}}%
\begin{itemdecl}
struct identity {
  template <class T>
  constexpr T&& operator()(T&& t) const noexcept;

  typedef @\unspec@ is_transparent;
};
\end{itemdecl}

\begin{itemdescr}
\pnum
\tcode{operator()} returns \tcode{std::forward<T>(t)}.

\ednote{REVIEW: From Stephan T. Lavavej: "[This] \tcode{identity} functor, being
a non-template, clashes with any attempt to provide \tcode{identity<T>::type}." <Insert
bikeshed naming discussion here>.}
\end{itemdescr}
\end{addedblock}

\setcounter{section}{14}

\begin{addedblock}
\rSec1[taggedtup]{Tagged tuple-like types}

\rSec2[taggedtup.general]{In general}

\pnum The library provides a template for augmenting a tuple-like type with named element accessor
member functions. The library also provides several templates that provide access to \tcode{tagged}
objects as if they were \tcode{tuple} objects (see~\cxxref{tuple.elem}).

\ednote{This type exists so that the algorithms can return pair- and tuple-like objects with named
accessors, as requested by LEWG. Rather than create a bunch of one-off class types with no relation
to pair and tuple, I opted instead to create a general utility. I'll note that this functionality
can be merged into \tcode{pair} and \tcode{tuple} directly with minimal breakage, but I opted for
now to keep it separate.}

\rSec2[taggedtup.tagged]{Class template \tcode{tagged}}

\pnum
Class template \tcode{tagged} augments a tuple-like class type (e.g., \tcode{pair}~(\cxxref{pairs}),
\tcode{tuple}~(\cxxref{tuple})) by giving it named accessors. It is used to define the alias
templates \tcode{tagged_pair}~(\ref{tagged.pairs}) and
\tcode{tagged_tuple}~(\ref{tagged.tuple}).

\pnum In the class synopsis below, let $i$ be in the range
\range{0}{sizeof...(Tags)} and $T_i$ be the $i^{th}$ type in \tcode{Tags}, where indexing
is zero-based.

\indexlibrary{\idxcode{tagged}}%
\begin{codeblock}
// defined in header <experimental/ranges/utility>

namespace std { namespace experimental { namespace ranges { inline namespace v1 {
  template <class T>
  concept bool TagSpecifier() {
    return @\impdef@;
  }

  template <class F>
  concept bool TaggedType() {
    return @\impdef@;
  }

  template <class Base, TagSpecifier... Tags>
    requires sizeof...(Tags) <= tuple_size<Base>::value
  struct tagged :
    Base, @\textit{TAGGET}@(tagged<Base, Tags...>, @$T_i$@, @$i$@)... { // \seebelow
    using Base::Base;
    tagged() = default;
    tagged(tagged&&) = default;
    tagged(const tagged&) = default;
    tagged &operator=(tagged&&) = default;
    tagged &operator=(const tagged&) = default;
    template <class Other>
      requires Constructible<Base, Other>()
    tagged(tagged<Other, Tags...> &&that) noexcept(@\seebelow@);
    template <class Other>
      requires Constructible<Base, const Other&>()
    tagged(const tagged<Other, Tags...> &that);
    template <class Other>
      requires Assignable<Base&, Other>()
    tagged& operator=(tagged<Other, Tags...>&& that) noexcept(@\seebelow@);
    template <class Other>
      requires Assignable<Base&, const Other&>()
    tagged& operator=(const tagged<Other, Tags...>& that);
    template <class U>
      requires Assignable<Base&, U>() && !Same<decay_t<U>, tagged>()
    tagged& operator=(U&& u) noexcept(@\seebelow@);
    void swap(tagged& that) noexcept(@\seebelow@)
      requires Swappable<Base&>();
    friend void swap(tagged&, tagged&) noexcept(@\seebelow@)
      requires Swappable<Base&>();
  };
}}}}
\end{codeblock}

\pnum A \techterm{tagged getter} is an empty trivial class type that has a named member function that
returns a reference to a member of a tuple-like object that is assumed to be derived from the getter
class. The tuple-like type of a tagged getter is called its \techterm{DerivedCharacteristic}.
The index of the tuple element returned from the getter's member functions is called its
\techterm{ElementIndex}. The name of the getter's member function is called its
\techterm{ElementName}

\pnum A tagged getter class with DerivedCharacteristic \tcode{\textit{D}}, ElementIndex
\tcode{\textit{N}}, and ElementName \tcode{\textit{name}} shall provide the following interface:

\begin{codeblock}
struct @\xname{\textit{TAGGED_GETTER}}@ {
  constexpr decltype(auto) @$name$@() &       { return get<@$N$@>(static_cast<@$D$@&>(*this)); }
  constexpr decltype(auto) @$name$@() &&      { return get<@$N$@>(static_cast<@$D$@&&>(*this)); }
  constexpr decltype(auto) @$name$@() const & { return get<@$N$@>(static_cast<const @$D$@&>(*this)); }
};
\end{codeblock}

\pnum
A \techterm{tag specifier} is a type that facilitates a mapping from a tuple-like type and an
element index into a \textit{tagged getter} that gives named access to the element at that index.
\tcode{TagSpecifier<T>()} is satisfied if and only if \tcode{T} is a tag specifier. The tag specifiers in the
\tcode{Tags} parameter pack shall be unique. \enternote The mapping mechanism from tag specifier to
tagged getter is unspecified.\exitnote

\pnum Let \tcode{\textit{TAGGET}(D, T, $N$)} name a tagged getter type that gives named
access to the $N$-th element of the tuple-like type \tcode{D}.

\pnum It shall not be possible to delete an instance of class template \tcode{tagged} through a
pointer to any base other than \tcode{Base}.

\pnum
\tcode{TaggedType<F>()} is satisfied if and only if \tcode{F} is a unary function
type with return type \tcode{T} which satisfies \tcode{TagSpecifier<T>()}. Let
\tcode{\textit{TAGSPEC}(F)} name the tag specifier of the \tcode{TaggedType} \tcode{F}, and let
\tcode{\textit{TAGELEM}(F)} name the argument type of the \tcode{TaggedType} \tcode{F}.

\indexlibrary{\idxcode{tagged}!\idxcode{tagged}}
\begin{itemdecl}
template <class Other>
  requires Constructible<Base, Other>()
tagged(tagged<Other, Tags...> &&that) noexcept(@\seebelow@);
\end{itemdecl}

\begin{itemdescr}
\pnum
\remarks The expression in the \tcode{noexcept} is equivalent to:

\begin{codeblock}
is_nothrow_constructible<Base, Other>::value
\end{codeblock}

\pnum
\effects Initializes \tcode{Base} with \tcode{static_cast<Other\&\&>(that)}.
\end{itemdescr}

\indexlibrary{\idxcode{tagged}!\idxcode{tagged}}
\begin{itemdecl}
template <class Other>
  requires Constructible<Base, const Other&>()
tagged(const tagged<Other, Tags...>& that);
\end{itemdecl}

\begin{itemdescr}
\pnum
\effects Initializes \tcode{Base} with \tcode{static_cast<const Other\&>(that)}.
\end{itemdescr}

\indexlibrary{\idxcode{operator=}!\idxcode{tagged}}
\indexlibrary{\idxcode{tagged}!\idxcode{operator=}}
\begin{itemdecl}
template <class Other>
  requires Assignable<Base&, Other>()
tagged& operator=(tagged<Other, Tags...>&& that) noexcept(@\seebelow@);
\end{itemdecl}

\begin{itemdescr}
\pnum
\remarks The expression in the \tcode{noexcept} is equivalent to:

\begin{codeblock}
is_nothrow_assignable<Base&, Other>::value
\end{codeblock}

\pnum
\effects Assigns \tcode{static_cast<Other\&\&>(that)} to \tcode{static_cast<Base\&>(*this)}.

\pnum
\returns \tcode{*this}.
\end{itemdescr}

\indexlibrary{\idxcode{operator=}!\idxcode{tagged}}
\indexlibrary{\idxcode{tagged}!\idxcode{operator=}}
\begin{itemdecl}
template <class Other>
  requires Assignable<Base&, const Other&>()
tagged& operator=(const tagged<Other, Tags...>& that);
\end{itemdecl}

\begin{itemdescr}
\pnum
\effects Assigns \tcode{static_cast<const Other\&>(that)} to \tcode{static_cast<Base\&>(*this)}.

\pnum
\returns \tcode{*this}.
\end{itemdescr}

\indexlibrary{\idxcode{operator=}!\idxcode{tagged}}
\indexlibrary{\idxcode{tagged}!\idxcode{operator=}}
\begin{itemdecl}
template <class U>
  requires Assignable<Base&, U>() && !Same<decay_t<U>, tagged>()
tagged& operator=(U&& u) noexcept(@\seebelow@);
\end{itemdecl}

\begin{itemdescr}
\pnum
\remarks The expression in the \tcode{noexcept} is equivalent to:

\begin{codeblock}
is_nothrow_assignable<Base&, U>::value
\end{codeblock}

\pnum
\effects Assigns \tcode{std::forward<U>(u)} to \tcode{static_cast<Base\&>(*this)}.

\pnum
\returns \tcode{*this}.
\end{itemdescr}

\indexlibrary{\idxcode{swap}!\idxcode{tagged}}
\indexlibrary{\idxcode{tagged}!\idxcode{swap}}
\begin{itemdecl}
void swap(tagged& rhs) noexcept(@\seebelow@)
  requires Swappable<Base&>();
\end{itemdecl}

\begin{itemdescr}
\pnum
\remarks The expression in the \tcode{noexcept} is equivalent to:

\begin{codeblock}
noexcept(swap(declval<Base&>(), declval<Base&>()))
\end{codeblock}

\pnum
\effects Calls \tcode{swap} on the result of applying \tcode{static_cast} to \tcode{*this} and
\tcode{that}.

\pnum
\throws Nothing unless the call to \tcode{swap} on the \tcode{Base} sub-objects throws.
\end{itemdescr}

\indexlibrary{\idxcode{swap}!\tcode{tagged}}%
\begin{itemdecl}
friend void swap(tagged& lhs, tagged& rhs) noexcept(@\seebelow@)
  requires Swappable<Base&>();
\end{itemdecl}

\begin{itemdescr}
\pnum
\remarks The expression in the \tcode{noexcept} is equivalent to:

\begin{codeblock}
noexcept(lhs.swap(rhs))
\end{codeblock}

\pnum
\effects Equivalent to: \tcode{lhs.swap(rhs)}.

\pnum
\throws Nothing unless the call to \tcode{lhs.swap(rhs)} throws.
\end{itemdescr}

\rSec2[tagged.astuple]{Tuple-like access to \tcode{tagged}}

\indexlibrary{\idxcode{tuple_size}}%
\indexlibrary{\idxcode{tuple_element}}%
\begin{itemdecl}
namespace std {
  template <class Base, class... Tags>
  struct tuple_size<experimental::ranges::tagged<Base, Tags...>>
    : tuple_size<Base> { };

  template <size_t N, class Base, class... Tags>
  struct tuple_element<N, experimental::ranges::tagged<Base, Tags...>>
    : tuple_element<N, Base> { };
}
\end{itemdecl}

\rSec2[tagged.pairs]{Alias template \tcode{tagged_pair}}

\begin{codeblock}
// defined in header <experimental/ranges/utility>

namespace std { namespace experimental { namespace ranges { inline namespace v1 {
  // ...
  template <TaggedType T1, TaggedType T2>
  using tagged_pair = tagged<pair<@\textit{TAGELEM}@(T1), @\textit{TAGELEM}@(T2)>,
                             @\textit{TAGSPEC}@(T1), @\textit{TAGSPEC}@(T2)>;
}}}}
\end{codeblock}

\pnum \enterexample
\begin{codeblock}
// See \ref{alg.tagspec}:
tagged_pair<tag::min(int), tag::max(int)> p{0, 1};
assert(&p.min() == &p.first);
assert(&p.max() == &p.second);
\end{codeblock}
\exitexample

\rSec3[tagged.pairs.creation]{Tagged pair creation functions}

\indexlibrary{\idxcode{make_tagged_pair}}%
\begin{itemdecl}
// defined in header <experimental/ranges/utility>

namespace std { namespace experimental { namespace ranges { inline namespace v1 {
  template <TagSpecifier Tag1, TagSpecifier Tag2, class T1, class T2>
    constexpr @\seebelow@ make_tagged_pair(T1&& x, T2&& y);
}}}}
\end{itemdecl}

\begin{itemdescr}
\pnum
Let \tcode{P} be the type of \tcode{make_pair(std::forward<T1>(x), std::forward<T2>(y))}.
Then the return type is \tcode{tagged<P, Tag1, Tag2>}.

\pnum
\returns \tcode{\{std::forward<T1>(x), std::forward<T2>(y)\}}.

\pnum
\enterexample
In place of:

\begin{codeblock}
  return tagged_pair<tag::min(int), tag::max(double)>(5, 3.1415926);   // explicit types
\end{codeblock}

a \Cpp program may contain:

\begin{codeblock}
  return make_tagged_pair<tag::min, tag::max>(5, 3.1415926);           // types are deduced
\end{codeblock}
\exitexample
\end{itemdescr}

\rSec2[tagged.tuple]{Alias template \tcode{tagged_tuple}}

\synopsis{Header \tcode{<experimental/ranges/tuple>} synopsis}

\begin{codeblock}
namespace std { namespace experimental { namespace ranges { inline namespace v1 {
  template <TaggedType... Types>
  using tagged_tuple = tagged<tuple<@\textit{TAGELEM}@(Types)...>,
                              @\textit{TAGSPEC}@(Types)...>;

  template <TagSpecifier... Tags, class... Types>
    requires sizeof...(Tags) == sizeof...(Types)
      constexpr @\seebelow@ make_tagged_tuple(Types&&... t);
}}}}
\end{codeblock}

\pnum
Any entities declared or defined in namespace \tcode{std} in header \tcode{<tuple>}
that are not already defined in namespace \tcode{std::experimental::ranges} in header
\tcode{<experimental/ranges/tuple>} are imported with
\grammarterm{using-declaration}{s}~(\cxxref{namespace.udecl}). \enterexample
\begin{codeblock}
namespace std { namespace experimental { namespace ranges { inline namespace v1 {
  using std::tuple;
  using std::make_tuple;
  // ... others
}}}}
\end{codeblock}
\exitexample

\begin{codeblock}
template <TaggedType... Types>
using tagged_tuple = tagged<tuple<@\textit{TAGELEM}@(Types)...>,
                            @\textit{TAGSPEC}@(Types)...>;
\end{codeblock}

\pnum \enterexample
\begin{codeblock}
// See \ref{alg.tagspec}:
tagged_tuple<tag::in(char*), tag::out(char*)> t{0, 0};
assert(&t.in() == &get<0>(t));
assert(&t.out() == &get<1>(t));
\end{codeblock}
\exitexample

\rSec3[tagged.tuple.creation]{Tagged tuple creation functions}

\indexlibrary{\idxcode{make_tagged_tuple}}%
\indexlibrary{\idxcode{tagged_tuple}!\idxcode{make_tagged_tuple}}%
\begin{itemdecl}
template <TagSpecifier... Tags, class... Types>
  requires sizeof...(Tags) == sizeof...(Types)
    constexpr @\seebelow@ make_tagged_tuple(Types&&... t);
\end{itemdecl}

\begin{itemdescr}
\pnum
Let \tcode{T} be the type of \tcode{make_tuple(std::forward<Types>(t)...)}.
Then the return type is \tcode{tagged<T, Tags...>}.

\pnum
\returns \tcode{tagged<T, Tags...>(std::forward<Types>(t)...)}.

\pnum
\enterexample

\begin{codeblock}
int i; float j;
make_tagged_tuple<tag::in1, tag::in2, tag::out>(1, ref(i), cref(j))
\end{codeblock}

creates a tagged tuple of type

\begin{codeblock}
tagged_tuple<tag::in1(int), tag::in2(int&), tag::out(const float&)>
\end{codeblock}
\exitexample
\end{itemdescr}
\end{addedblock}
