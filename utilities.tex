%!TEX root = std.tex
\setcounter{chapter}{19}
\rSec0[utilities]{General utilities library}

\setcounter{section}{1}
\rSec1[utility]{Utility components}

{\color{addclr}
\pnum
\synopsis{Header \tcode{<experimental/ranges/utility>} synopsis}
}

\begin{codeblock}
@\added{namespace std \{ namespace experimental \{ namespace ranges \{ inline namespace v1 \{}@
  // \ref{utility.swap}, swap:
  @\removed{template <class T> void swap(T\& a, T\& b) noexcept(\seebelow);}@
  @\removed{template <class T, size_t N> void swap(T (\&a)[N], T (\&b)[N]) noexcept(noexcept(swap(*a, *b)));}@
  @\added{namespace \{}@
    @\added{constexpr \unspec swap = \unspec;}@
  @\added{\}}@

  // \ref{utility.exchange}, exchange:
  template <@\changed{class}{MoveConstructible}@ T, class U=T>
    @\added{requires Assignable<T\&, U>()}@
  T exchange(T& obj, U&& new_val);
\end{codeblock}
\begin{addedblock}
\begin{codeblock}
  // \ref{taggedtup.tagged}, struct with named accessors
  template <class T>
  concept bool TagSpecifier() {
    return @\seebelow@;
  }

  template <class F>
  concept bool TaggedType() {
    return @\seebelow@;
  }

  template <class Base, TagSpecifier... Tags>
    requires sizeof...(Tags) <= tuple_size<Base>::value
  struct tagged;

  // \ref{tagged.pairs}, tagged pairs
  template <TaggedType T1, TaggedType T2> using tagged_pair = @\seebelow@;

  template <TagSpecifier Tag1, TagSpecifier Tag2, class T1, class T2>
  constexpr @\seebelow@ make_tagged_pair(T1&& x, T2&& y);
}}}}

namespace std {
  // \ref{tagged.astuple}, tuple-like access to tagged
  template <class Base, class... Tags>
  struct tuple_size<experimental::ranges::tagged<Base, Tags...>>;

  template <size_t N, class Base, class... Tags>
  struct tuple_element<N, experimental::ranges::tagged<Base, Tags...>>;
}
\end{codeblock}
\end{addedblock}

\begin{addedblock}
\pnum
Any entities declared or defined directly in namespace \tcode{std} in header \tcode{<utility>}
that are not already defined in namespace \tcode{std::experimental::ranges::v1} in header
\tcode{<experimental/ranges/utility>} are imported with
\grammarterm{using-declaration}{s}~(\cxxref{namespace.udecl}). \enterexample
\begin{codeblock}
namespace std { namespace experimental { namespace ranges { inline namespace v1 {
  using std::pair;
  using std::make_pair;
  // ... others
}}}}
\end{codeblock}
\exitexample
\end{addedblock}

\setcounter{subsection}{1}
\rSec2[utility.swap]{\tcode{swap}}

\begin{addedblock}
\indexlibrary{\idxcode{swap}}%
\pnum The name \tcode{swap} denotes a customization point
object~(\ref{customization.point.object}). The effect of the expression
\tcode{ranges::swap(E1, E2)} for some expressions \tcode{E1}
and \tcode{E2} is equivalent to:

\begin{itemize}
\item
  \tcode{(void)swap(E1, E2)}, with overload resolution
  performed in a context that includes the declarations
\begin{codeblock}
  template <class T>
  void swap(T&, T&) = delete;
  template <class T, size_t N>
  void swap(T(&)[N], T(&)[N]) = delete;
\end{codeblock}
  and does not include a declaration of \tcode{ranges::swap}.
  If the function selected by overload resolution does not
  exchange the values denoted by \tcode{E1} and \tcode{E2},
  the program is ill-formed with no diagnostic required.

\item
  Otherwise, \tcode{(void)swap_ranges(E1, E2)} if \tcode{E1} and
  \tcode{E2} are lvalues of array types~(\cxxref{basic.compound})
  of equal extent and \tcode{ranges::swap(*(E1), *(E2))}
  is a valid expression, except that
  \tcode{noexcept(ranges::swap(E1, E2))} is equal to
  \tcode{noexcept(ranges::swap(*(E1), *(E2)))}.

  \ednote{This formulation intentionally allows swapping arrays
  with identical extent and differing element types, but only
  when swapping the element types is well-defined. Swapping
  arrays of \tcode{int} and \tcode{double} continues to be
  unsound, but \tcode{Swappable<T\&, U\&>()} implies
  \tcode{Swappable<T(\&)[N], U(\&)[N]>()}.}

\item
  Otherwise, if \tcode{E1} and \tcode{E2} are lvalues of the
  same type \tcode{T} which meets the syntactic requirements of
  \tcode{MoveConstructible<T>()} and
  \tcode{Assignable<T\&, T>()}, exchanges the denoted values.
  \tcode{ranges::swap(E1, E2)} is a constant expression if
  the constructor selected by overload resolution for
  \tcode{T\{std::move(E1)\}} is a constexpr constructor and
  the expression \tcode{E1 = std::move(E2)} can appear in a
  constexpr function. \tcode{noexcept(ranges::swap(E1, E2))}
  is equal to \tcode{is_nothrow_move_constructible<T>::value
  \&\& is_nothrow_move_assignable<T>::value}. If either
  \tcode{MoveConstructible<T>()} or
  \tcode{Assignable<T\&, T>()} is not satisfied, the program
  is ill-formed with no diagnostic required.
  \tcode{ranges::swap(E1, E2)} has type \tcode{void}.

\item
  Otherwise, \tcode{ranges::swap(E1, E2)} is ill-formed.
\end{itemize}

\pnum
\remark Whenever \tcode{ranges::swap(E1, E2)} is a valid
expression, it exchanges the values denoted by \tcode{E1}
and \tcode{E2} and has type \tcode{void}.
\end{addedblock}

\begin{removedblock}
\begin{itemdecl}
template <class T>
void swap(T& a, T& b) noexcept(@\seebelow@);
\end{itemdecl}

\begin{itemdescr}
\pnum
\remarks The expression inside \tcode{noexcept} is equivalent to:

\begin{codeblock}
is_nothrow_move_constructible<T>::value &&
is_nothrow_move_assignable<T>::value
\end{codeblock}

\pnum
\remarks
A library implementor is free to omit the \tcode{requires} clause so long as
this function does not participate in overload resolution if the following
is false
\begin{codeblock}
is_move_constructible<T>::value &&
is_move_assignable<T>::value &&
is_destructible<T>::value
\end{codeblock}

\pnum
\requires
Type
\tcode{T}
shall be
\tcode{MoveConstructible} (Table~\cxxref{moveconstructible})
and
\tcode{MoveAssignable} (Table~\cxxref{moveassignable}).

\pnum
\effects
Exchanges values stored in two locations.
\end{itemdescr}
\end{removedblock}

\begin{removedblock}
\begin{itemdecl}
template <class T, size_t N>
void swap(T (&a)[N], T (&b)[N]) noexcept(noexcept(swap(*a, *b)));}
\end{itemdecl}

\begin{itemdescr}
\pnum
\requires
\tcode{a[i]} shall be swappable with~(\ref{concepts.lib.corelang.swappable}) \tcode{b[i]}
for all \tcode{i} in the range \range{0}{N}.

\pnum
\effects \tcode{swap_ranges(a, a + N, b)}
\end{itemdescr}
\end{removedblock}

\rSec2[utility.exchange]{\tcode{exchange}}

\begin{itemdecl}
template <@\changed{class}{MoveConstructible}@ T, class U=T>
  @\added{requires Assignable<T\&, U>()}@
T exchange(T& obj, U&& new_val);
\end{itemdecl}

\begin{itemdescr}
\pnum
\effects
Equivalent to:

\begin{codeblock}
T old_val = std::move(obj);
obj = std::forward<U>(new_val);
return old_val;
\end{codeblock}
\end{itemdescr}

\setcounter{section}{8}
\rSec1[function.objects]{Function objects}

\setcounter{Paras}{1}
\pnum
\synopsis{Header \tcode{<\added{experimental/ranges/}functional>} synopsis}

\begin{codeblock}
@\added{namespace std \{ namespace experimental \{ namespace ranges \{ inline namespace v1 \{}@
  @\added{// \ref{func.invoke}, invoke:}@
  @\added{template <class F, class... Args>}@
  @\added{result_of_t<F\&\&(Args\&\&...)> invoke(F\&\& f, Args\&\&... args);}@

  @\added{// \ref{comparisons}, comparisons:}@
  template <class T = void>
    @\added{requires EqualityComparable<T>() || Same<T, void>()}@
  struct equal_to;

  template <class T = void>
    @\added{requires EqualityComparable<T>() || Same<T, void>()}@
  struct not_equal_to;

  template <class T = void>
    @\added{requires StrictTotallyOrdered<T>() || Same<T, void>()}@
  struct greater;

  template <class T = void>
    @\added{requires StrictTotallyOrdered<T>() || Same<T, void>()}@
  struct less;

  template <class T = void>
    @\added{requires StrictTotallyOrdered<T>() || Same<T, void>()}@
  struct greater_equal;

  template <class T = void>
    @\added{requires StrictTotallyOrdered<T>() || Same<T, void>()}@
  struct less_equal;

  template <> struct equal_to<void>;
  template <> struct not_equal_to<void>;
  template <> struct greater<void>;
  template <> struct less<void>;
  template <> struct greater_equal<void>;
  template <> struct less_equal<void>;

  @\added{// \ref{func.identity}, identity:}@
  @\added{struct identity;}@
@\added{\}\}\}\}}@
\end{codeblock}

\begin{addedblock}
\pnum
Any entities declared or defined directly in namespace \tcode{std} in header \tcode{<functional>}
that are not already defined in namespace \tcode{std::experimental::ranges} in header
\tcode{<experimental/ranges/functional>} are imported with
\grammarterm{using-declaration}{s}~(\cxxref{namespace.udecl}). \enterexample
\begin{codeblock}
namespace std { namespace experimental { namespace ranges { inline namespace v1 {
  using std::reference_wrapper;
  using std::ref;
  // ... others
}}}}
\end{codeblock}
\exitexample

\pnum
Any nested namespaces defined directly in namespace \tcode{std} in header \tcode{<functional>}
that are not already defined in namespace \tcode{std::experimental::ranges} in header
\tcode{<experimental/ranges/functional>} are aliased with a
\grammarterm{namespace-alias-definition}~(\cxxref{namespace.alias}). \enterexample
\begin{codeblock}
namespace std { namespace experimental { namespace ranges { inline namespace v1 {
  namespace placeholders = std::placeholders;
}}}}
\end{codeblock}
\exitexample
\end{addedblock}

\ednote{Before [refwrap], insert the following section and renumber subsequent sections
as appropriate. (Renumbering hasn't been performed herein to ease review.)}

\begin{addedblock}
\setcounter{subsection}{2}
\rSec2[func.invoke]{Function template \tcode{invoke}}
\begin{itemdecl}
template <class F, class... Args>
result_of_t<F&&(Args&&...)> invoke(F&& f, Args&&... args);
\end{itemdecl}
\begin{itemdescr}
\pnum
\effects Equivalent to \tcode{\textit{INVOKE}(std::forward<F>(f), std::forward<Args>(args)...)}~(\cxxref{func.require}).
\end{itemdescr}
\end{addedblock}

\setcounter{subsection}{4}
\rSec2[comparisons]{Comparisons}

\pnum
The library provides basic function object classes for all of the comparison
operators in the language~(\cxxref{expr.rel}, \cxxref{expr.eq}).

\indexlibrary{\idxcode{equal_to}}%
\begin{itemdecl}
template <class T = void>
  @\added{requires EqualityComparable<T>() || Same<T, void>()}@
struct equal_to {
  constexpr bool operator()(const T& x, const T& y) const;
  @\removed{typedef T first_argument_type;}@
  @\removed{typedef T second_argument_type;}@
  @\removed{typedef bool result_type;}@
};
\end{itemdecl}

\begin{itemdescr}
\pnum
\tcode{operator()} returns \tcode{x == y}.
\end{itemdescr}

\indexlibrary{\idxcode{not_equal_to}}%
\begin{itemdecl}
template <class T = void>
  @\added{requires EqualityComparable<T>() || Same<T, void>()}@
struct not_equal_to {
  constexpr bool operator()(const T& x, const T& y) const;
  @\removed{typedef T first_argument_type;}@
  @\removed{typedef T second_argument_type;}@
  @\removed{typedef bool result_type;}@
};
\end{itemdecl}

\begin{itemdescr}
\pnum
\tcode{operator()} returns \tcode{x != y}.
\end{itemdescr}

\indexlibrary{\idxcode{greater}}%
\begin{itemdecl}
template <class T = void>
  @\added{requires StrictTotallyOrdered<T>() || Same<T, void>()}@
struct greater {
  constexpr bool operator()(const T& x, const T& y) const;
  @\removed{typedef T first_argument_type;}@
  @\removed{typedef T second_argument_type;}@
  @\removed{typedef bool result_type;}@
};
\end{itemdecl}

\begin{itemdescr}
\pnum
\tcode{operator()} returns \tcode{x > y}.
\end{itemdescr}

\indexlibrary{\idxcode{less}}%
\begin{itemdecl}
template <class T = void>
  @\added{requires StrictTotallyOrdered<T>() || Same<T, void>()}@
struct less {
  constexpr bool operator()(const T& x, const T& y) const;
  @\removed{typedef T first_argument_type;}@
  @\removed{typedef T second_argument_type;}@
  @\removed{typedef bool result_type;}@
};
\end{itemdecl}

\begin{itemdescr}
\pnum
\tcode{operator()} returns \tcode{x < y}.
\end{itemdescr}

\indexlibrary{\idxcode{greater_equal}}%
\begin{itemdecl}
template <class T = void>
  @\added{requires StrictTotallyOrdered<T>() || Same<T, void>()}@
struct greater_equal {
  constexpr bool operator()(const T& x, const T& y) const;
  @\removed{typedef T first_argument_type;}@
  @\removed{typedef T second_argument_type;}@
  @\removed{typedef bool result_type;}@
};
\end{itemdecl}

\begin{itemdescr}
\pnum
\tcode{operator()} returns \tcode{x >= y}.
\end{itemdescr}

\indexlibrary{\idxcode{less_equal}}%
\begin{itemdecl}
template <class T = void>
  @\added{requires StrictTotallyOrdered<T>() || Same<T, void>()}@
struct less_equal {
  constexpr bool operator()(const T& x, const T& y) const;
  @\removed{typedef T first_argument_type;}@
  @\removed{typedef T second_argument_type;}@
  @\removed{typedef bool result_type;}@
};
\end{itemdecl}

\begin{itemdescr}
\pnum
\tcode{operator()} returns \tcode{x <= y}.
\end{itemdescr}

\indexlibrary{\idxcode{equal_to<>}}%
\begin{itemdecl}
template <> struct equal_to<void> {
  template <class T, class U>
    @\added{requires EqualityComparable<T, U>()}@
  constexpr auto operator()(T&& t, U&& u) const
    -> decltype(std::forward<T>(t) == std::forward<U>(u));

  typedef @\unspec@ is_transparent;
};
\end{itemdecl}

\begin{itemdescr}
\pnum
\tcode{operator()} returns \tcode{std::forward<T>(t) == std::forward<U>(u)}.
\end{itemdescr}

\indexlibrary{\idxcode{not_equal_to<>}}%
\begin{itemdecl}
template <> struct not_equal_to<void> {
  template <class T, class U>
    @\added{requires EqualityComparable<T, U>()}@
  constexpr auto operator()(T&& t, U&& u) const
    -> decltype(std::forward<T>(t) != std::forward<U>(u));

  typedef @\unspec@ is_transparent;
};
\end{itemdecl}

\begin{itemdescr}
\pnum
\tcode{operator()} returns \tcode{std::forward<T>(t) != std::forward<U>(u)}.
\end{itemdescr}

\indexlibrary{\idxcode{greater<>}}%
\begin{itemdecl}
template <> struct greater<void> {
  template <class T, class U>
    @\added{requires StrictTotallyOrdered<T, U>()}@
  constexpr auto operator()(T&& t, U&& u) const
    -> decltype(std::forward<T>(t) > std::forward<U>(u));

  typedef @\unspec@ is_transparent;
};
\end{itemdecl}

\begin{itemdescr}
\pnum
\tcode{operator()} returns \tcode{std::forward<T>(t) > std::forward<U>(u)}.
\end{itemdescr}

\indexlibrary{\idxcode{less<>}}%
\begin{itemdecl}
template <> struct less<void> {
  template <class T, class U>
    @\added{requires StrictTotallyOrdered<T, U>()}@
  constexpr auto operator()(T&& t, U&& u) const
    -> decltype(std::forward<T>(t) < std::forward<U>(u));

  typedef @\unspec@ is_transparent;
};
\end{itemdecl}

\begin{itemdescr}
\pnum
\tcode{operator()} returns \tcode{std::forward<T>(t) < std::forward<U>(u)}.
\end{itemdescr}

\indexlibrary{\idxcode{greater_equal<>}}%
\begin{itemdecl}
template <> struct greater_equal<void> {
  template <class T, class U>
    @\added{requires StrictTotallyOrdered<T, U>()}@
  constexpr auto operator()(T&& t, U&& u) const
    -> decltype(std::forward<T>(t) >= std::forward<U>(u));

  typedef @\unspec@ is_transparent;
};
\end{itemdecl}

\begin{itemdescr}
\pnum
\tcode{operator()} returns \tcode{std::forward<T>(t) >= std::forward<U>(u)}.
\end{itemdescr}

\indexlibrary{\idxcode{less_equal<>}}%
\begin{itemdecl}
template <> struct less_equal<void> {
  template <class T, class U>
    @\added{requires StrictTotallyOrdered<T, U>()}@
  constexpr auto operator()(T&& t, U&& u) const
    -> decltype(std::forward<T>(t) <= std::forward<U>(u));

  typedef @\unspec@ is_transparent;
};
\end{itemdecl}

\begin{itemdescr}
\pnum
\tcode{operator()} returns \tcode{std::forward<T>(t) <= std::forward<U>(u)}.
\end{itemdescr}

\pnum
For templates \tcode{greater}, \tcode{less}, \tcode{greater_equal}, and
\tcode{less_equal}, the specializations for any pointer type yield a total order,
even if the built-in operators \tcode{<}, \tcode{>}, \tcode{<=}, \tcode{>=}
do not.

\ednote{After subsection 20.9.12 [unord.hash] add the following subsection:}

\setcounter{subsection}{12}
\begin{addedblock}
\rSec2[func.identity]{Class \tcode{identity}}

\indexlibrary{\idxcode{identity}}%
\begin{itemdecl}
struct identity {
  template <class T>
  constexpr T&& operator()(T&& t) const noexcept;

  typedef @\unspec@ is_transparent;
};
\end{itemdecl}

\begin{itemdescr}
\pnum
\tcode{operator()} returns \tcode{std::forward<T>(t)}.

\ednote{REVIEW: From Stephan T. Lavavej: "[This] \tcode{identity} functor, being
a non-template, clashes with any attempt to provide \tcode{identity<T>::type}." <Insert
bikeshed naming discussion here>.}
\end{itemdescr}
\end{addedblock}

\rSec1[meta]{Metaprogramming and type traits}
\setcounter{subsection}{1}
\rSec2[meta.type.synop]{Header \tcode{<type_traits>} synopsis}

\ednote{Change the \tcode{<type_traits>} synopsis as follows. Note: this change is
intended to be made in \tcode{namespace std}.}

\begin{codeblock}
namespace std {
  [@...@]
  // 20.10.4.3, type properties:
  [@...@]
  template <class T> struct is_move_assignable;

  @\newtxt{template <class T, class U> struct is_swappable_with;}@
  @\newtxt{template <class T> struct is_swappable;}@

  template <class T> struct is_destructible;
  [@...@]
  template <class T> struct is_nothrow_move_assignable;

  @\newtxt{template <class T, class U> struct is_nothrow_swappable_with;}@
  @\newtxt{template <class T> struct is_nothrow_swappable;}@

  template <class T> struct is_nothrow_destructible;
  [@...@]

  // 20.10.7.6, other transformations:
  [@...@]
  template <class... T> struct common_type;
  @\newtxt{template <class T, class U, template <class> class TQual, template <class> class UQual>}@
    @\newtxt{struct basic_common_reference \{ \};}@
  @\newtxt{template <class... T> struct common_reference;}@
  template <class T> struct underlying_type;
  [@...@]
  template <class... T>
    using common_type_t = typename common_type<T...>::type;
  @\newtxt{template <class... T>}@
    @\newtxt{using common_reference_t = typename common_reference<T...>::type;}@
  template <class T>
    using underlying_type_t = typename underlying_type<T>::type;
  [@...@]

  // 20.15.4.3, type properties
  [@...@]
  @\newtxt{template <class T, class U> constexpr bool is_swappable_with_v}@
    @\newtxt{= is_swappable_with<T, U>::value;}@
  @\newtxt{template <class T> constexpr bool is_swappable_v}@
    @\newtxt{= is_swappable<T>::value;}@
  [@...@]
  @\newtxt{template <class T, class U> constexpr bool is_nothrow_swappable_with_v}@
    @\newtxt{= is_nothrow_swappable_with<T, U>::value;}@
  @\newtxt{template <class T> constexpr bool is_nothrow_swappable_v}@
    @\newtxt{= is_nothrow_swappable<T>::value;}@
  [@...@]
}
\end{codeblock}

\ednote{Change [meta.unary.prop], Table 49 -- ``Type property predicates'', as indicated.
The following is taken from the current Working Draft of C++17.}

\setcounter{table}{48}
\begin{libreqtab3b}{Type property predicates}{tab:type-traits.properties}
\\ \topline
\lhdr{Template} &   \chdr{Condition}    &   \rhdr{Preconditions}    \\ \capsep
\endfirsthead
\continuedcaption\\
\topline
\lhdr{Template} &   \chdr{Condition}    &   \rhdr{Preconditions}    \\ \capsep
\endhead

...  \\ \rowsep

\indexlibrary{\idxcode{is_swappable_with}}%
\newtxt{\tcode{template <class T, class U>}}\br
  \newtxt{\tcode{struct is_swappable_with;}} &
  \newtxt{The expressions \tcode{swap(declval<T>(),}}\brk\newtxt{\tcode{ declval<U>())}
  and \tcode{swap(}}\brk\newtxt{\tcode{declval<U>(),}}\brk\newtxt{\tcode{ declval<T>())}
  are each well-formed when treated as an unevaluated operand (Clause~\cxxref{expr})
  in an overload-resolution context for swappable
  values~(\ref{concepts.lib.corelang.swappable}).
  Access checking is performed as if in a context unrelated to \tcode{T} and \tcode{U}.
  Only the validity of the immediate context of the \tcode{swap} expressions is
  considered.}
  \begin{note}
  \newtxt{The compilation of the expressions can result in side effects such as the
  instantiation of class template specializations and function template specializations,
  the generation of implicitly-defined functions, and so on. Such side effects are not
  in the ``immediate context'' and can result in the program being ill-formed.}
  \end{note} &
  \newtxt{\tcode{T} and \tcode{U} shall be complete types,
  (possibly cv-qualified) \tcode{void}, or
  arrays of unknown bound.}  \\ \rowsep

\indexlibrary{\idxcode{is_swappable}}%
\newtxt{\tcode{template <class T>}}\br
  \newtxt{\tcode{struct is_swappable;}} &
  \newtxt{For a referenceable type \tcode{T},
  the same result as \tcode{is_\brk\-swappable_\brk\-with_\brk\-v<}}\brk\newtxt{\tcode{T\&,\brk T\&>},
  otherwise \tcode{false}.} &
  \newtxt{\tcode{T} shall be a complete type,
  (possibly cv-qualified) \tcode{void}, or
  an array of unknown bound.} \\ \rowsep

... \\ \rowsep

\indexlibrary{\idxcode{is_nothrow_swappable_with}}%
\tcode{\newtxt{template <class T, class U>}}\br
  \newtxt{\tcode{struct is_nothrow_swappable_with;}} &
  \newtxt{\tcode{is_swappable_with_v<T, }}\brk\newtxt{\tcode{U>} is \tcode{true} and
  each \tcode{swap} expression of the definition of
  \tcode{is_swappable_with<T, U>} is known not to throw
  any exceptions~(\cxxref{expr.unary.noexcept}).} &
  \newtxt{\tcode{T} and \tcode{U} shall be complete types,
  (possibly cv-qualified) \tcode{void}, or
  arrays of unknown bound.} \\ \rowsep

\indexlibrary{\idxcode{is_nothrow_swappable}}%
\tcode{\newtxt{template <class T>}}\br
  \newtxt{\tcode{struct is_nothrow_swappable;}} &
  \newtxt{For a referenceable type \tcode{T},
  the same result as \tcode{is_nothrow_swappable_}}\brk\newtxt{\tcode{with_v<T\&, T\&>},
  otherwise \tcode{false}.} &
  \newtxt{\tcode{T} shall be a complete type,
  (possibly cv-qualified) \tcode{void}, or
  an array of unknown bound.} \\ \rowsep

... \\ \rowsep

\end{libreqtab3b}

\ednote{Change Table 57 -- Other Transformations as follows:}

\setcounter{table}{56}
\begin{libreqtab3d}{Other transformations}{tab:type-traits.other}
\\ \topline
\lhdr{Template} &   \chdr{Condition}    &   \rhdr{Comments} \\ \capsep
\endfirsthead
\continuedcaption\\
\topline
\lhdr{Template} &   \chdr{Condition}    &   \rhdr{Comments} \\ \capsep
\endhead

... \\ \rowsep

 \tcode{template <class... T>} \tcode{struct common_type;}
 & &
 The member typedef \tcode{type} shall be defined or omitted as specified below.
 If it is omitted, there shall be no member \tcode{type}. \oldoldtxt{All types}
 \newnewtxt{Each type} in the parameter pack \tcode{T} shall be complete or
 (possibly \cv) \tcode{void}. A program may specialize this trait if at least one
 template parameter in the specialization \oldoldtxt{is}\newnewtxt{depends on} a
 user-defined type\newnewtxt{ and \tcode{sizeof...(T) == 2}}. \begin{note} Such
 specializations are needed when only explicit conversions are desired among the
 template arguments.
 \end{note} \\ \rowsep

\newtxt{\tcode{template <class T, class U,}}\br
\newtxt{\tcode{  template <class> class TQual,}}
\newtxt{\tcode{  template <class> class UQual>}}
\newtxt{\tcode{struct basic_common_reference;}} & &
\newtxt{The primary template shall have no member typedef
\tcode{type}. A program may specialize this trait if at least one
template parameter in the specialization depends on a user-defined
type. In such a specialization, a member typedef \tcode{type} may be
defined or omitted. If it is omitted, there shall be no member
\tcode{type}.}
\begin{note} \newtxt{Such specializations may be used to influence
the result of \tcode{common_reference}.}\end{note} \\ \rowsep

\newtxt{\tcode{template <class... T>}}
\newtxt{\tcode{struct common_reference;}} & &
\newtxt{The member typedef \tcode{type} shall be defined or omitted
as specified below. If it is omitted, there shall be no member
\tcode{type}. Each type in the parameter pack \tcode{T} shall be
complete or (possibly \cv) \tcode{void}. A program may specialize
this trait if at least one template parameter in the specialization
depends on a user-defined type and \tcode{sizeof...(T) == 2}.}
\begin{note} \newtxt{Such specializations are needed to properly
handle proxy reference types in generic code.} \end{note} \\ \rowsep

... \\ \rowsep

\end{libreqtab3d}

\ednote{Delete [meta.trans.other]/p3 and replace it with the following:}

\setcounter{Paras}{2}

{\color{newclr}
\pnum
Let \tcode{CREF(A)} be \tcode{add_lvalue_reference_t<const
remove_reference_t<A>{}>}. Let \tcode{UNCVREF(A)} be
\tcode{remove_cv_t<remove_reference_t<A>{}>}. Let \tcode{XREF(A)}
denote a unary template \tcode{T} such that \tcode{T<UNCVREF(A)>}
denotes the same type as \tcode{A}. Let \tcode{COPYCV(FROM, TO)} be
an alias for type \tcode{TO} with the addition of \tcode{FROM}'s
top-level \cv-qualifiers. \enterexample \tcode{COPYCV(const int,
volatile short)} is an alias for \tcode{const volatile short}.
\exitexample Let \tcode{RREF_RES(Z)} be
\tcode{remove_reference_t<Z>\&\&} if \tcode{Z} is a reference type
or \tcode{Z} otherwise. Let \tcode{COND_RES(X, Y)} be
\tcode{decltype(declval<bool>() ? declval<X>() : declval<Y>())}.
Given types \tcode{A} and \tcode{B}, let \tcode{X} be
\tcode{remove_reference_t<A>}, let \tcode{Y} be
\tcode{remove_reference_t<B>}, and let \tcode{COMMON_REF(A, B)} be:
\begin{itemize}
\item If \tcode{A} and \tcode{B} are both lvalue reference types,
  \tcode{COMMON_REF(A, B)} is
  \tcode{COND_RES(COPYCV(X, Y) \&, COPYCV(Y, X) \&)}.
\item Otherwise, let \tcode{C} be
  \tcode{RREF_RES(COMMON_REF(X\&, Y\&))}. If \tcode{A} and \tcode{B}
  are both rvalue reference types, and \tcode{C} is well-formed,
  and \tcode{is_convertible<A, C>::value} and
  \tcode{is_convertible<B, C>::value} are \tcode{true}, then
  \tcode{COMMON_REF(A, B)} is \tcode{C}.
\item Otherwise, let \tcode{D} be
  \tcode{COMMON_REF(const X\&, Y\&)}. If \tcode{A} is an rvalue
  reference and \tcode{B} is an lvalue reference and \tcode{D} is
  well-formed and \tcode{is_convertible<A, D>::value} is
  \tcode{true}, then \tcode{COMMON_REF(A, B)} is \tcode{D}.
\item Otherwise, if \tcode{A} is an lvalue reference and \tcode{B}
  is an rvalue reference, then \tcode{COMMON_REF(A, B)} is
  \tcode{COMMON_REF(B, A)}.
\item Otherwise, \tcode{COMMON_REF(A, B)} is
  \tcode{decay_t<COND_RES(CREF(A), CREF(B))>}.
\end{itemize}

If any of the types computed above are ill-formed, then
\tcode{COMMON_REF(A, B)} is ill-formed.

\pnum
\ednote{The following text in black is taken from the current C++17 draft.}
\color{black} For the \tcode{common_type} trait applied to a parameter pack
\tcode{T} of types, the member \tcode{type} shall be either defined or not
present as follows:
\begin{itemize}
\item If \tcode{sizeof...(T)} is zero, there shall be no member \tcode{type}.
\item Otherwise, if \tcode{sizeof...(T)} is one, let \oldoldtxt{T0}\newnewtxt{T1}
  denote the sole type in the pack \tcode{T}. The member typedef \tcode{type}
  shall denote the same type as \tcode{decay_t<\oldoldtxt{T0}\newnewtxt{T1}>}.
  \color{newclr}
\item Otherwise, if \tcode{sizeof...(T)} is two, let \tcode{T1} and \tcode{T2}
  denote the two types in the pack \tcode{T}, and let \tcode{D1} and \tcode{D2}
  be \tcode{decay_t<T1>} and \tcode{decay_t<T2>} respectively. Then
\begin{itemize}
\item If \tcode{D1} and \tcode{T1} denote the same type and \tcode{D2} and
  \tcode{T2} denote the same type, then
\begin{itemize}
\item If \tcode{COMMON_REF(T1, T2)} is well-formed, then the member typedef
  \tcode{type} denotes that type.
\item Otherwise, there shall be no member \tcode{type}.
\end{itemize}
\item Otherwise, if \tcode{common_type_t<D1, D2>} is well-formed, then the
  member typedef \tcode{type} denotes that type.
\item Otherwise, there shall be no member \tcode{type}.
\end{itemize}\color{black}
\item Otherwise, if \tcode{sizeof...(T)} is greater than \oldoldtxt{one}
  \newnewtxt{two}, let \tcode{T1}, \tcode{T2}, and \tcode{R\newnewtxt{est}},
  respectively, denote the first, second, and (pack of) remaining types
  comprising \tcode{T}. \oldoldtxt{\enternote \tcode{sizeof...(R)} may be zero.
  \exitnote} Let \tcode{C} \oldoldtxt{denote the type, if any, of an unevaluated
  conditional expression (5.16) whose first operand is an arbitrary value of
  type \tcode{bool}, whose second operand is an xvalue of type \tcode{T1}, and
  whose third operand is an xvalue of type \tcode{T2}.}\color{newclr}be the type
  \tcode{common_type_t<T1, T2>}. Then:
\begin{itemize}
\item {\color{black}If there is such a type \tcode{C}, the member typedef
  \tcode{type} shall denote the same type, if any, as
  \tcode{common_type_t<C, R\newnewtxt{est}...>}.}
\item {\color{black}Otherwise, there shall be no member \tcode{type}.}
\end{itemize}
\end{itemize}

\color{newclr}
\pnum
For the \tcode{common_reference} trait applied to a parameter pack \tcode{T} of
types, the member \tcode{type} shall be either defined or not present as follows:
\begin{itemize}
\item If \tcode{sizeof...(T)} is zero, there shall be no member \tcode{type}.
\item Otherwise, if \tcode{sizeof...(T)} is one, let \tcode{T1} denote the sole
  type in the pack \tcode{T}. The member typedef \tcode{type} shall denote the
  same type as \tcode{T1}.
\item Otherwise, if \tcode{sizeof...(T)} is two, let \tcode{T1} and \tcode{T2}
  denote the two types in the pack \tcode{T}. Then
\begin{itemize}
\item If \tcode{COMMON_REF(T1, T2)} is well-formed and denotes a reference type
  then the member typedef \tcode{type} denotes that type.
\item Otherwise, if \tcode{basic_common_reference<UNCVREF(T1), UNCVREF(T2),
  XREF(T1), XREF(T2)>::type} is well-formed, then the member typedef
  \tcode{type} denotes that type.
\item Otherwise, if \tcode{common_type_t<T1, T2>} is well-formed, then the
  member typedef \tcode{type} denotes that type.
\item Otherwise, there shall be no member \tcode{type}.
\end{itemize}
\item Otherwise, if \tcode{sizeof...(T)} is greater than two, let \tcode{T1},
  \tcode{T2}, and \tcode{Rest}, respectively, denote the first, second, and
  (pack of) remaining types comprising \tcode{T}. Let \tcode{C} be the type
  \tcode{common_reference_t<T1, T2>}. Then:
\begin{itemize}
\item If there is such a type \tcode{C}, the member typedef \tcode{type} shall
  denote the same type, if any, as \tcode{common_reference_t<C, Rest...>}.
\item Otherwise, there shall be no member \tcode{type}.
\end{itemize}
\end{itemize}
} %\color{newclr}

\setcounter{section}{14}

\begin{addedblock}
\rSec1[taggedtup]{Tagged tuple-like types}

\rSec2[taggedtup.general]{In general}

\pnum The library provides a template for augmenting a tuple-like type with named element accessor
member functions. The library also provides several templates that provide access to \tcode{tagged}
objects as if they were \tcode{tuple} objects (see~\cxxref{tuple.elem}).

\ednote{This type exists so that the algorithms can return pair- and tuple-like objects with named
accessors, as requested by LEWG. Rather than create a bunch of one-off class types with no relation
to pair and tuple, I opted instead to create a general utility. I'll note that this functionality
can be merged into \tcode{pair} and \tcode{tuple} directly with minimal breakage, but I opted for
now to keep it separate.}

\rSec2[taggedtup.tagged]{Class template \tcode{tagged}}

\pnum
Class template \tcode{tagged} augments a tuple-like class type (e.g., \tcode{pair}~(\cxxref{pairs}),
\tcode{tuple}~(\cxxref{tuple})) by giving it named accessors. It is used to define the alias
templates \tcode{tagged_pair}~(\ref{tagged.pairs}) and
\tcode{tagged_tuple}~(\ref{tagged.tuple}).

\pnum In the class synopsis below, let $i$ be in the range
\range{0}{sizeof...(Tags)} and $T_i$ be the $i^{th}$ type in \tcode{Tags}, where indexing
is zero-based.

\indexlibrary{\idxcode{tagged}}%
\begin{codeblock}
// defined in header <experimental/ranges/utility>

namespace std { namespace experimental { namespace ranges { inline namespace v1 {
  template <class T>
  concept bool TagSpecifier() {
    return @\impdef@;
  }

  template <class F>
  concept bool TaggedType() {
    return @\impdef@;
  }

  template <class Base, TagSpecifier... Tags>
    requires sizeof...(Tags) <= tuple_size<Base>::value
  struct tagged :
    Base, @\textit{TAGGET}@(tagged<Base, Tags...>, @$T_i$@, @$i$@)... { // \seebelow
    using Base::Base;
    tagged() = default;
    tagged(tagged&&) = default;
    tagged(const tagged&) = default;
    tagged &operator=(tagged&&) = default;
    tagged &operator=(const tagged&) = default;
    template <class Other>
      requires Constructible<Base, Other>()
    tagged(tagged<Other, Tags...> &&that) noexcept(@\seebelow@);
    template <class Other>
      requires Constructible<Base, const Other&>()
    tagged(const tagged<Other, Tags...> &that);
    template <class Other>
      requires Assignable<Base&, Other>()
    tagged& operator=(tagged<Other, Tags...>&& that) noexcept(@\seebelow@);
    template <class Other>
      requires Assignable<Base&, const Other&>()
    tagged& operator=(const tagged<Other, Tags...>& that);
    template <class U>
      requires Assignable<Base&, U>() && !Same<decay_t<U>, tagged>()
    tagged& operator=(U&& u) noexcept(@\seebelow@);
    void swap(tagged& that) noexcept(@\seebelow@)
      requires Swappable<Base&>();
    friend void swap(tagged&, tagged&) noexcept(@\seebelow@)
      requires Swappable<Base&>();
  };
}}}}
\end{codeblock}

\pnum A \techterm{tagged getter} is an empty trivial class type that has a named member function that
returns a reference to a member of a tuple-like object that is assumed to be derived from the getter
class. The tuple-like type of a tagged getter is called its \techterm{DerivedCharacteristic}.
The index of the tuple element returned from the getter's member functions is called its
\techterm{ElementIndex}. The name of the getter's member function is called its
\techterm{ElementName}

\pnum A tagged getter class with DerivedCharacteristic \tcode{\textit{D}}, ElementIndex
\tcode{\textit{N}}, and ElementName \tcode{\textit{name}} shall provide the following interface:

\begin{codeblock}
struct @\xname{\textit{TAGGED_GETTER}}@ {
  constexpr decltype(auto) @$name$@() &       { return get<@$N$@>(static_cast<@$D$@&>(*this)); }
  constexpr decltype(auto) @$name$@() &&      { return get<@$N$@>(static_cast<@$D$@&&>(*this)); }
  constexpr decltype(auto) @$name$@() const & { return get<@$N$@>(static_cast<const @$D$@&>(*this)); }
};
\end{codeblock}

\pnum
A \techterm{tag specifier} is a type that facilitates a mapping from a tuple-like type and an
element index into a \textit{tagged getter} that gives named access to the element at that index.
\tcode{TagSpecifier<T>()} is satisfied if and only if \tcode{T} is a tag specifier. The tag specifiers in the
\tcode{Tags} parameter pack shall be unique. \enternote The mapping mechanism from tag specifier to
tagged getter is unspecified.\exitnote

\pnum Let \tcode{\textit{TAGGET}(D, T, $N$)} name a tagged getter type that gives named
access to the $N$-th element of the tuple-like type \tcode{D}.

\pnum It shall not be possible to delete an instance of class template \tcode{tagged} through a
pointer to any base other than \tcode{Base}.

\pnum
\tcode{TaggedType<F>()} is satisfied if and only if \tcode{F} is a unary function
type with return type \tcode{T} which satisfies \tcode{TagSpecifier<T>()}. Let
\tcode{\textit{TAGSPEC}(F)} name the tag specifier of the \tcode{TaggedType} \tcode{F}, and let
\tcode{\textit{TAGELEM}(F)} name the argument type of the \tcode{TaggedType} \tcode{F}.

\indexlibrary{\idxcode{tagged}!\idxcode{tagged}}
\begin{itemdecl}
template <class Other>
  requires Constructible<Base, Other>()
tagged(tagged<Other, Tags...> &&that) noexcept(@\seebelow@);
\end{itemdecl}

\begin{itemdescr}
\pnum
\remarks The expression in the \tcode{noexcept} is equivalent to:

\begin{codeblock}
is_nothrow_constructible<Base, Other>::value
\end{codeblock}

\pnum
\effects Initializes \tcode{Base} with \tcode{static_cast<Other\&\&>(that)}.
\end{itemdescr}

\indexlibrary{\idxcode{tagged}!\idxcode{tagged}}
\begin{itemdecl}
template <class Other>
  requires Constructible<Base, const Other&>()
tagged(const tagged<Other, Tags...>& that);
\end{itemdecl}

\begin{itemdescr}
\pnum
\effects Initializes \tcode{Base} with \tcode{static_cast<const Other\&>(that)}.
\end{itemdescr}

\indexlibrary{\idxcode{operator=}!\idxcode{tagged}}
\indexlibrary{\idxcode{tagged}!\idxcode{operator=}}
\begin{itemdecl}
template <class Other>
  requires Assignable<Base&, Other>()
tagged& operator=(tagged<Other, Tags...>&& that) noexcept(@\seebelow@);
\end{itemdecl}

\begin{itemdescr}
\pnum
\remarks The expression in the \tcode{noexcept} is equivalent to:

\begin{codeblock}
is_nothrow_assignable<Base&, Other>::value
\end{codeblock}

\pnum
\effects Assigns \tcode{static_cast<Other\&\&>(that)} to \tcode{static_cast<Base\&>(*this)}.

\pnum
\returns \tcode{*this}.
\end{itemdescr}

\indexlibrary{\idxcode{operator=}!\idxcode{tagged}}
\indexlibrary{\idxcode{tagged}!\idxcode{operator=}}
\begin{itemdecl}
template <class Other>
  requires Assignable<Base&, const Other&>()
tagged& operator=(const tagged<Other, Tags...>& that);
\end{itemdecl}

\begin{itemdescr}
\pnum
\effects Assigns \tcode{static_cast<const Other\&>(that)} to \tcode{static_cast<Base\&>(*this)}.

\pnum
\returns \tcode{*this}.
\end{itemdescr}

\indexlibrary{\idxcode{operator=}!\idxcode{tagged}}
\indexlibrary{\idxcode{tagged}!\idxcode{operator=}}
\begin{itemdecl}
template <class U>
  requires Assignable<Base&, U>() && !Same<decay_t<U>, tagged>()
tagged& operator=(U&& u) noexcept(@\seebelow@);
\end{itemdecl}

\begin{itemdescr}
\pnum
\remarks The expression in the \tcode{noexcept} is equivalent to:

\begin{codeblock}
is_nothrow_assignable<Base&, U>::value
\end{codeblock}

\pnum
\effects Assigns \tcode{std::forward<U>(u)} to \tcode{static_cast<Base\&>(*this)}.

\pnum
\returns \tcode{*this}.
\end{itemdescr}

\indexlibrary{\idxcode{swap}!\idxcode{tagged}}
\indexlibrary{\idxcode{tagged}!\idxcode{swap}}
\begin{itemdecl}
void swap(tagged& rhs) noexcept(@\seebelow@)
  requires Swappable<Base&>();
\end{itemdecl}

\begin{itemdescr}
\pnum
\remarks The expression in the \tcode{noexcept} is equivalent to:

\begin{codeblock}
noexcept(swap(declval<Base&>(), declval<Base&>()))
\end{codeblock}

\pnum
\effects Calls \tcode{swap} on the result of applying \tcode{static_cast} to \tcode{*this} and
\tcode{that}.

\pnum
\throws Nothing unless the call to \tcode{swap} on the \tcode{Base} sub-objects throws.
\end{itemdescr}

\indexlibrary{\idxcode{swap}!\tcode{tagged}}%
\begin{itemdecl}
friend void swap(tagged& lhs, tagged& rhs) noexcept(@\seebelow@)
  requires Swappable<Base&>();
\end{itemdecl}

\begin{itemdescr}
\pnum
\remarks The expression in the \tcode{noexcept} is equivalent to:

\begin{codeblock}
noexcept(lhs.swap(rhs))
\end{codeblock}

\pnum
\effects Equivalent to: \tcode{lhs.swap(rhs)}.

\pnum
\throws Nothing unless the call to \tcode{lhs.swap(rhs)} throws.
\end{itemdescr}

\rSec2[tagged.astuple]{Tuple-like access to \tcode{tagged}}

\indexlibrary{\idxcode{tuple_size}}%
\indexlibrary{\idxcode{tuple_element}}%
\begin{itemdecl}
namespace std {
  template <class Base, class... Tags>
  struct tuple_size<experimental::ranges::tagged<Base, Tags...>>
    : tuple_size<Base> { };

  template <size_t N, class Base, class... Tags>
  struct tuple_element<N, experimental::ranges::tagged<Base, Tags...>>
    : tuple_element<N, Base> { };
}
\end{itemdecl}

\rSec2[tagged.pairs]{Alias template \tcode{tagged_pair}}

\begin{codeblock}
// defined in header <experimental/ranges/utility>

namespace std { namespace experimental { namespace ranges { inline namespace v1 {
  // ...
  template <TaggedType T1, TaggedType T2>
  using tagged_pair = tagged<pair<@\textit{TAGELEM}@(T1), @\textit{TAGELEM}@(T2)>,
                             @\textit{TAGSPEC}@(T1), @\textit{TAGSPEC}@(T2)>;
}}}}
\end{codeblock}

\pnum \enterexample
\begin{codeblock}
// See \ref{alg.tagspec}:
tagged_pair<tag::min(int), tag::max(int)> p{0, 1};
assert(&p.min() == &p.first);
assert(&p.max() == &p.second);
\end{codeblock}
\exitexample

\rSec3[tagged.pairs.creation]{Tagged pair creation functions}

\indexlibrary{\idxcode{make_tagged_pair}}%
\begin{itemdecl}
// defined in header <experimental/ranges/utility>

namespace std { namespace experimental { namespace ranges { inline namespace v1 {
  template <TagSpecifier Tag1, TagSpecifier Tag2, class T1, class T2>
    constexpr @\seebelow@ make_tagged_pair(T1&& x, T2&& y);
}}}}
\end{itemdecl}

\begin{itemdescr}
\pnum
Let \tcode{P} be the type of \tcode{make_pair(std::forward<T1>(x), std::forward<T2>(y))}.
Then the return type is \tcode{tagged<P, Tag1, Tag2>}.

\pnum
\returns \tcode{\{std::forward<T1>(x), std::forward<T2>(y)\}}.

\pnum
\enterexample
In place of:

\begin{codeblock}
  return tagged_pair<tag::min(int), tag::max(double)>(5, 3.1415926);   // explicit types
\end{codeblock}

a \Cpp program may contain:

\begin{codeblock}
  return make_tagged_pair<tag::min, tag::max>(5, 3.1415926);           // types are deduced
\end{codeblock}
\exitexample
\end{itemdescr}

\rSec2[tagged.tuple]{Alias template \tcode{tagged_tuple}}

\synopsis{Header \tcode{<experimental/ranges/tuple>} synopsis}

\begin{codeblock}
namespace std { namespace experimental { namespace ranges { inline namespace v1 {
  template <TaggedType... Types>
  using tagged_tuple = tagged<tuple<@\textit{TAGELEM}@(Types)...>,
                              @\textit{TAGSPEC}@(Types)...>;

  template <TagSpecifier... Tags, class... Types>
    requires sizeof...(Tags) == sizeof...(Types)
      constexpr @\seebelow@ make_tagged_tuple(Types&&... t);
}}}}
\end{codeblock}

\pnum
Any entities declared or defined in namespace \tcode{std} in header \tcode{<tuple>}
that are not already defined in namespace \tcode{std::experimental::ranges} in header
\tcode{<experimental/ranges/tuple>} are imported with
\grammarterm{using-declaration}{s}~(\cxxref{namespace.udecl}). \enterexample
\begin{codeblock}
namespace std { namespace experimental { namespace ranges { inline namespace v1 {
  using std::tuple;
  using std::make_tuple;
  // ... others
}}}}
\end{codeblock}
\exitexample

\begin{codeblock}
template <TaggedType... Types>
using tagged_tuple = tagged<tuple<@\textit{TAGELEM}@(Types)...>,
                            @\textit{TAGSPEC}@(Types)...>;
\end{codeblock}

\pnum \enterexample
\begin{codeblock}
// See \ref{alg.tagspec}:
tagged_tuple<tag::in(char*), tag::out(char*)> t{0, 0};
assert(&t.in() == &get<0>(t));
assert(&t.out() == &get<1>(t));
\end{codeblock}
\exitexample

\rSec3[tagged.tuple.creation]{Tagged tuple creation functions}

\indexlibrary{\idxcode{make_tagged_tuple}}%
\indexlibrary{\idxcode{tagged_tuple}!\idxcode{make_tagged_tuple}}%
\begin{itemdecl}
template <TagSpecifier... Tags, class... Types>
  requires sizeof...(Tags) == sizeof...(Types)
    constexpr @\seebelow@ make_tagged_tuple(Types&&... t);
\end{itemdecl}

\begin{itemdescr}
\pnum
Let \tcode{T} be the type of \tcode{make_tuple(std::forward<Types>(t)...)}.
Then the return type is \tcode{tagged<T, Tags...>}.

\pnum
\returns \tcode{tagged<T, Tags...>(std::forward<Types>(t)...)}.

\pnum
\enterexample

\begin{codeblock}
int i; float j;
make_tagged_tuple<tag::in1, tag::in2, tag::out>(1, ref(i), cref(j))
\end{codeblock}

creates a tagged tuple of type

\begin{codeblock}
tagged_tuple<tag::in1(int), tag::in2(int&), tag::out(const float&)>
\end{codeblock}
\exitexample
\end{itemdescr}
\end{addedblock}
