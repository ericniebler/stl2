%!TEX root = std.tex
\rSec0[utilities]{General utilities library}

\rSec1[utilities.general]{General}

\pnum
This Clause describes utilities that are generally useful in \Cpp programs; some
of these utilities are used by other elements of the Ranges library.
These utilities are summarized in Table~\ref{tab:util.lib.summary}.

\begin{libsumtab}{General utilities library summary}{tab:util.lib.summary}
\ref{utility}               & Utility components                & \tcode{<experimental/ranges/utility>}     \\ \rowsep
\ref{function.objects}      & Function objects                  & \tcode{<experimental/ranges/functional>}  \\ \rowsep
\ref{meta}                  & Type traits                       & \tcode{<type_traits>}                     \\ \rowsep
\ref{taggedtup}             & Tagged tuple-like types           & \tcode{<experimental/ranges/utility>} \&  \\
                            &                                   & \tcode{<experimental/ranges/tuple>}       \\
\end{libsumtab}

\rSec1[utility]{Utility components}

\pnum
This subclause contains some basic function and class templates that are used
throughout the rest of the library.

\indexlibrary{\idxhdr{experimental/ranges/utility}}%
\synopsis{Header \tcode{<experimental/ranges/utility>} synopsis}

\pnum
The header \tcode{<experimental/ranges/utility>} defines several types,
function templates, and concepts that are described in this Clause. It also
defines the templates \tcode{tagged} and \tcode{tagged_pair} and various
function templates that operate on \tcode{tagged_pair} objects.

\begin{codeblock}
namespace std { namespace experimental { namespace ranges { inline namespace v1 {
  // \ref{utility.swap}, swap:
  namespace {
    constexpr @\unspec@ swap = @\unspec@;
  }

  // \ref{utility.exchange}, exchange:
  template <MoveConstructible T, class U=T>
    requires Assignable<T&, U>
  constexpr T exchange(T& obj, U&& new_val) noexcept(@\seebelow@);

  // \ref{taggedtup.tagged}, struct with named accessors
  template <class T>
  concept bool TagSpecifier = @\seebelow@;

  template <class F>
  concept bool TaggedType = @\seebelow@;

  template <class Base, TagSpecifier... Tags>
    requires sizeof...(Tags) <= tuple_size<Base>::value
  struct tagged;

  // \ref{tagged.pairs}, tagged pairs
  template <TaggedType T1, TaggedType T2> using tagged_pair = @\seebelow@;

  template <TagSpecifier Tag1, TagSpecifier Tag2, class T1, class T2>
  constexpr @\seebelow@ make_tagged_pair(T1&& x, T2&& y);
}}}}

namespace std {
  // \ref{tagged.astuple}, tuple-like access to tagged
  template <class Base, class... Tags>
  struct tuple_size<experimental::ranges::tagged<Base, Tags...>>;

  template <size_t N, class Base, class... Tags>
  struct tuple_element<N, experimental::ranges::tagged<Base, Tags...>>;
}
\end{codeblock}

\rSec2[utility.swap]{\tcode{swap}}

\indexlibrary{\idxcode{swap}}%
\pnum The name \tcode{swap} denotes a customization point
object~(\ref{customization.point.object}). The effect of the expression
\tcode{ranges::swap(E1, E2)} for some expressions \tcode{E1}
and \tcode{E2} is equivalent to:

\begin{itemize}
\item
  \tcode{(void)swap(E1, E2)}, if that expression is valid, with overload resolution
  performed in a context that includes the declarations
\begin{codeblock}
  template <class T>
  void swap(T&, T&) = delete;
  template <class T, size_t N>
  void swap(T(&)[N], T(&)[N]) = delete;
\end{codeblock}
  and does not include a declaration of \tcode{ranges::swap}.
  If the function selected by overload resolution does not
  exchange the values referenced by \tcode{E1} and \tcode{E2},
  the program is ill-formed with no diagnostic required.

\item
  Otherwise, \tcode{(void)swap_ranges(E1, E2)} if \tcode{E1} and
  \tcode{E2} are lvalues of array types~(\cxxref{basic.compound})
  of equal extent and \tcode{ranges::swap(*(E1), *(E2))}
  is a valid expression, except that
  \tcode{noexcept(\brk{}ranges::swap(E1, E2))} is equal to
  \tcode{noexcept(\brk{}ranges::swap(*(E1), *(E2)))}.

\item
  Otherwise, if \tcode{E1} and \tcode{E2} are lvalues of the
  same type \tcode{T} which meets the syntactic requirements of
  \tcode{MoveConstructible<T>} and
  \tcode{Assignable<T\&, T>}, exchanges the referenced values.
  \tcode{ranges::swap(E1, E2)} is a constant expression if
  the constructor selected by overload resolution for
  \tcode{T\{std::move(E1)\}} is a constexpr constructor and
  the expression \tcode{E1 = std::move(E2)} can appear in a
  constexpr function. \tcode{noexcept(ranges::swap(E1, E2))}
  is equal to \tcode{is_nothrow_move_constructible<T>::value
  \&\& is_nothrow_move_assignable<T>::value}. If either
  \tcode{MoveConstructible} or
  \tcode{Assignable} is not satisfied, the program
  is ill-formed with no diagnostic required.

\item
  Otherwise, \tcode{ranges::swap(E1, E2)} is ill-formed.
\end{itemize}

\pnum
\remark Whenever \tcode{ranges::swap(E1, E2)} is a valid
expression, it exchanges the values referenced by \tcode{E1}
and \tcode{E2} and has type \tcode{void}.

\rSec2[utility.exchange]{\tcode{exchange}}

\begin{itemdecl}
template <MoveConstructible T, class U=T>
  requires Assignable<T&, U>
constexpr T exchange(T& obj, U&& new_val) noexcept(@\seebelow@);
\end{itemdecl}

\begin{itemdescr}
\pnum
\effects
Equivalent to:

\begin{codeblock}
T old_val = std::move(obj);
obj = std::forward<U>(new_val);
return old_val;
\end{codeblock}

\remarks
The expression in \tcode{noexcept} is equivalent to:
\begin{codeblock}
is_nothrow_move_constructible<T>::value &&
is_nothrow_assignable<T&, U>::value
\end{codeblock}
\end{itemdescr}

\rSec1[function.objects]{Function objects}

\pnum
\synopsis{Header \tcode{<experimental/ranges/functional>} synopsis}

\begin{codeblock}
namespace std { namespace experimental { namespace ranges { inline namespace v1 {
  // \ref{func.invoke}, invoke:
  template <class F, class... Args>
  result_of_t<F&&(Args&&...)> invoke(F&& f, Args&&... args);

  // \ref{comparisons}, comparisons:
  template <class T = void>
    requires @\seebelow@
  struct equal_to;

  template <class T = void>
    requires @\seebelow@
  struct not_equal_to;

  template <class T = void>
    requires @\seebelow@
  struct greater;

  template <class T = void>
    requires @\seebelow@
  struct less;

  template <class T = void>
    requires @\seebelow@
  struct greater_equal;

  template <class T = void>
    requires @\seebelow@
  struct less_equal;

  template <> struct equal_to<void>;
  template <> struct not_equal_to<void>;
  template <> struct greater<void>;
  template <> struct less<void>;
  template <> struct greater_equal<void>;
  template <> struct less_equal<void>;

  // \ref{func.identity}, identity:
  struct identity;
}}}}
\end{codeblock}

\rSec2[func.invoke]{Function template \tcode{invoke}}
\begin{itemdecl}
template <class F, class... Args>
result_of_t<F&&(Args&&...)> invoke(F&& f, Args&&... args);
\end{itemdecl}
\begin{itemdescr}
\pnum
\effects Equivalent to: \\ \tcode{return \textit{INVOKE}(std::forward<F>(f), std::forward<Args>(args)...);} (\cxxref{func.require}).
\end{itemdescr}

\rSec2[comparisons]{Comparisons}

\pnum
The library provides basic function object classes for all of the comparison
operators in the language~(\cxxref{expr.rel}, \cxxref{expr.eq}).

\pnum
In this section, \tcode{\textit{BUILTIN_PTR_CMP}(T, $op$, U)} for types \tcode{T}
and \tcode{U} and where $op$ is an equality~(\cxxref{expr.eq}) or relational
operator~(\cxxref{expr.rel}) is a boolean constant expression.
\tcode{\textit{BUILTIN_PTR_CMP}(T, $op$, U)} is \tcode{true} if and only if $op$
in the expression \tcode{declval<T>() $op$ declval<U>()} resolves to a built-in
operator comparing pointers.

\pnum
There is an implementation-defined strict total ordering over all pointer values
of a given type. This total ordering is consistent with the partial order imposed
by the builtin operators \tcode{<}, \tcode{>}, \tcode{<=}, and \tcode{>=}.

\indexlibrary{\idxcode{equal_to}}%
\begin{itemdecl}
template <class T = void>
  requires EqualityComparable<T> || Same<T, void> || @\textit{\tcode{BUILTIN_PTR_CMP}}@(const T&, ==, const T&)
struct equal_to {
  constexpr bool operator()(const T& x, const T& y) const;
};
\end{itemdecl}

\begin{itemdescr}
\pnum
\tcode{operator()} has effects equivalent to: \tcode{return equal_to<>{}(x, y);}
\end{itemdescr}

\indexlibrary{\idxcode{not_equal_to}}%
\begin{itemdecl}
template <class T = void>
  requires EqualityComparable<T> || Same<T, void> || @\textit{\tcode{BUILTIN_PTR_CMP}}@(const T&, ==, const T&)
struct not_equal_to {
  constexpr bool operator()(const T& x, const T& y) const;
};
\end{itemdecl}

\begin{itemdescr}
\pnum
\tcode{operator()} has effects equivalent to: \tcode{return !equal_to<>{}(x, y);}
\end{itemdescr}

\indexlibrary{\idxcode{greater}}%
\begin{itemdecl}
template <class T = void>
  requires StrictTotallyOrdered<T> || Same<T, void> || @\textit{\tcode{BUILTIN_PTR_CMP}}@(const T&, <, const T&)
struct greater {
  constexpr bool operator()(const T& x, const T& y) const;
};
\end{itemdecl}

\begin{itemdescr}
\pnum
\tcode{operator()} has effects equivalent to: \tcode{return less<>{}(y, x);}
\end{itemdescr}

\indexlibrary{\idxcode{less}}%
\begin{itemdecl}
template <class T = void>
  requires StrictTotallyOrdered<T> || Same<T, void> || @\textit{\tcode{BUILTIN_PTR_CMP}}@(const T&, <, const T&)
struct less {
  constexpr bool operator()(const T& x, const T& y) const;
};
\end{itemdecl}

\begin{itemdescr}
\pnum
\tcode{operator()} has effects equivalent to: \tcode{return less<>{}(x, y);}
\end{itemdescr}

\indexlibrary{\idxcode{greater_equal}}%
\begin{itemdecl}
template <class T = void>
  requires StrictTotallyOrdered<T> || Same<T, void> || @\textit{\tcode{BUILTIN_PTR_CMP}}@(const T&, <, const T&)
struct greater_equal {
  constexpr bool operator()(const T& x, const T& y) const;
};
\end{itemdecl}

\begin{itemdescr}
\pnum
\tcode{operator()} has effects equivalent to: \tcode{return !less<>{}(x, y);}
\end{itemdescr}

\indexlibrary{\idxcode{less_equal}}%
\begin{itemdecl}
template <class T = void>
  requires StrictTotallyOrdered<T> || Same<T, void> || @\textit{\tcode{BUILTIN_PTR_CMP}}@(const T&, <, const T&)
struct less_equal {
  constexpr bool operator()(const T& x, const T& y) const;
};
\end{itemdecl}

\begin{itemdescr}
\pnum
\tcode{operator()} has effects equivalent to: \tcode{return !less<>{}(y, x);}
\end{itemdescr}

\indexlibrary{\idxcode{equal_to<>}}%
\begin{itemdecl}
template <> struct equal_to<void> {
  template <class T, class U>
    requires EqualityComparableWith<T, U> || @\textit{\tcode{BUILTIN_PTR_CMP}}@(T, ==, U)
  constexpr bool operator()(T&& t, U&& u) const;

  typedef @\unspec@ is_transparent;
};
\end{itemdecl}

\begin{itemdescr}
\pnum
\requires If the expression \tcode{std::forward<T>(t) == std::forward<U>(u)}
results in a call to a built-in operator \tcode{==} comparing pointers of type
\tcode{P}, the conversion sequences from both \tcode{T} and \tcode{U} to \tcode{P}
shall be equality-preserving~(\ref{concepts.lib.general.equality}).

\pnum
\effects
\begin{itemize}
\item
If the expression \tcode{std::forward<T>(t) == std::forward<U>(u)} results in a
call to a built-in operator \tcode{==} comparing pointers of type \tcode{P}:
returns \tcode{false} if either (the converted value of) \tcode{t} precedes
\tcode{u} or \tcode{u} precedes \tcode{t} in the implementation-defined strict
total order over pointers of type \tcode{P} and otherwise \tcode{true}.

\item
Otherwise, equivalent to: \tcode{return std::forward<T>(t) == std::forward<U>(u);}
\end{itemize}
\end{itemdescr}

\indexlibrary{\idxcode{not_equal_to<>}}%
\begin{itemdecl}
template <> struct not_equal_to<void> {
  template <class T, class U>
    requires EqualityComparableWith<T, U> || @\textit{\tcode{BUILTIN_PTR_CMP}}@(T, ==, U)
  constexpr bool operator()(T&& t, U&& u) const;

  typedef @\unspec@ is_transparent;
};
\end{itemdecl}

\begin{itemdescr}
\pnum
\tcode{operator()} has effects equivalent to:
\begin{codeblock}
return !equal_to<>{}(std::forward<T>(t), std::forward<U>(u));
\end{codeblock}
\end{itemdescr}

\indexlibrary{\idxcode{greater<>}}%
\begin{itemdecl}
template <> struct greater<void> {
  template <class T, class U>
    requires StrictTotallyOrderedWith<T, U> || @\textit{\tcode{BUILTIN_PTR_CMP}}@(U, <, T)
  constexpr bool operator()(T&& t, U&& u) const;

  typedef @\unspec@ is_transparent;
};
\end{itemdecl}

\begin{itemdescr}
\pnum
\tcode{operator()} has effects equivalent to:
\begin{codeblock}
return less<>{}(std::forward<U>(u), std::forward<T>(t));
\end{codeblock}
\end{itemdescr}

\indexlibrary{\idxcode{less<>}}%
\begin{itemdecl}
template <> struct less<void> {
  template <class T, class U>
    requires StrictTotallyOrderedWith<T, U> || @\textit{\tcode{BUILTIN_PTR_CMP}}@(T, <, U)
  constexpr bool operator()(T&& t, U&& u) const;

  typedef @\unspec@ is_transparent;
};
\end{itemdecl}

\begin{itemdescr}
\pnum
\requires
If the expression \tcode{std::forward<T>(t) < std::forward<U>(u)} results in a
call to a built-in operator \tcode{<} comparing pointers of type \tcode{P}, the
conversion sequences from both \tcode{T} and \tcode{U} to \tcode{P} shall be
equality-preserving~(\ref{concepts.lib.general.equality}). For any expressions
\tcode{ET} and \tcode{EU} such that \tcode{decltype((ET))} is \tcode{T} and
\tcode{decltype((EU))} is \tcode{U}, exactly one of \tcode{less<>\{\}(ET, EU)},
\tcode{less<>\{\}(EU, ET)} or \tcode{equal_to<>\{\}(ET, EU)} shall be
\tcode{true}.

\pnum
\effects
\begin{itemize}
\item
If the expression \tcode{std::forward<T>(t) < std::forward<U>(u)} results in a
call to a built-in operator \tcode{<} comparing pointers of type \tcode{P}:
returns \tcode{true} if (the converted value of) \tcode{t} precedes \tcode{u} in
the implementation-defined strict total order over pointers of type \tcode{P}
and otherwise \tcode{false}.

\item
Otherwise, equivalent to: \tcode{return std::forward<T>(t) < std::forward<U>(u);}
\end{itemize}
\end{itemdescr}

\indexlibrary{\idxcode{greater_equal<>}}%
\begin{itemdecl}
template <> struct greater_equal<void> {
  template <class T, class U>
    requires StrictTotallyOrderedWith<T, U> || @\textit{\tcode{BUILTIN_PTR_CMP}}@(T, <, U)
  constexpr bool operator()(T&& t, U&& u) const;

  typedef @\unspec@ is_transparent;
};
\end{itemdecl}

\begin{itemdescr}
\pnum
\tcode{operator()} has effects equivalent to:
\begin{codeblock}
return !less<>{}(std::forward<T>(t), std::forward<U>(u));
\end{codeblock}
\end{itemdescr}

\indexlibrary{\idxcode{less_equal<>}}%
\begin{itemdecl}
template <> struct less_equal<void> {
  template <class T, class U>
    requires StrictTotallyOrderedWith<T, U> || @\textit{\tcode{BUILTIN_PTR_CMP}}@(U, <, T)
  constexpr bool operator()(T&& t, U&& u) const;

  typedef @\unspec@ is_transparent;
};
\end{itemdecl}

\begin{itemdescr}
\pnum
\tcode{operator()} has effects equivalent to:
\begin{codeblock}
return !less<>{}(std::forward<U>(u), std::forward<T>(t));
\end{codeblock}
\end{itemdescr}

\rSec2[func.identity]{Class \tcode{identity}}

\indexlibrary{\idxcode{identity}}%
\begin{itemdecl}
struct identity {
  template <class T>
  constexpr T&& operator()(T&& t) const noexcept;

  typedef @\unspec@ is_transparent;
};
\end{itemdecl}

\begin{itemdescr}
\pnum
\tcode{operator()} returns \tcode{std::forward<T>(t)}.

\end{itemdescr}

\rSec1[meta]{Metaprogramming and type traits}

\rSec2[meta.type.synop]{Header \tcode{<experimental/ranges/type_traits>} synopsis}
\begin{codeblock}
namespace std { namespace experimental { namespace ranges { inline namespace v1 {
  // \ref{meta.unary.prop}, type properties:
  template <class T, class U> struct is_swappable_with;
  template <class T> struct is_swappable;

  template <class T, class U> struct is_nothrow_swappable_with;
  template <class T> struct is_nothrow_swappable;

  template <class T, class U> constexpr bool is_swappable_with_v
    = is_swappable_with<T, U>::value;
  template <class T> constexpr bool is_swappable_v
    = is_swappable<T>::value;

  template <class T, class U> constexpr bool is_nothrow_swappable_with_v
    = is_nothrow_swappable_with<T, U>::value;
  template <class T> constexpr bool is_nothrow_swappable_v
    = is_nothrow_swappable<T>::value;

  // \ref{meta.trans.other}, other transformations:
  template <class... T> struct common_type;
  template <class T, class U, template <class> class TQual, template <class> class UQual>
    struct basic_common_reference { };
  template <class... T> struct common_reference;
  
  template <class... T>
    using common_type_t = typename common_type<T...>::type;
  template <class... T>
    using common_reference_t = typename common_reference<T...>::type;
}}}}
\end{codeblock}

\rSec2[meta.unary.prop]{Type properties}

\pnum
These templates provide access to some of the more important properties of types.

\pnum
It is unspecified whether the library defines any full or partial specializations
of any of these templates.

\pnum
For all of the class templates \tcode{X} declared in this subclause, instantiating
that template with a template argument that is a class template specialization may
result in the implicit instantiation of the template argument if and only if the
semantics of \tcode{X} require that the argument must be a complete type.

\pnum
For the purpose of defining the templates in this subclause, a function call
expression \tcode{declval<T>()} for any type \tcode{T} is considered to be a
trivial~(\cxxref{basic.types}, \cxxref{special}) function call that is not an
odr-use~(\cxxref{basic.def.odr}) of \tcode{declval} in the context of the
corresponding definition notwithstanding the restrictions of~(\cxxref{declval}).

\begin{libreqtab3b}{Additional type property predicates}{tab:type-traits.properties}
\\ \topline
\lhdr{Template} &   \chdr{Condition}    &   \rhdr{Precondition}    \\ \capsep
\endfirsthead
\continuedcaption\\
\topline
\lhdr{Template} &   \chdr{Condition}    &   \rhdr{Precondition}    \\ \capsep
\endhead

\indexlibrary{\idxcode{is_swappable_with}}%
\tcode{template <class T, class U>}\br
  \tcode{struct is_swappable_with;} &
  The expressions \tcode{ranges::swap(}\brk{}\tcode{declval<T>(),}\brk\tcode{ declval<U>())}
  and \tcode{ranges::swap(}\brk\tcode{declval<U>(),}\brk\tcode{ declval<T>())}
  are each well-formed when treated as an unevaluated operand (Clause~\cxxref{expr}).
  Access checking is performed as if in a context unrelated to \tcode{T} and \tcode{U}.
  Only the validity of the immediate context of the \tcode{swap} expressions is
  considered.
  \enternote
  The compilation of the expressions can result in side effects such as the
  instantiation of class template specializations and function template specializations,
  the generation of implicitly-defined functions, and so on. Such side effects are not
  in the ``immediate context'' and can result in the program being ill-formed.
  \exitnote &
  \tcode{T} and \tcode{U} shall be complete types,
  (possibly \cv-qualified) \tcode{void}, or
  arrays of unknown bound.  \\ \rowsep

\indexlibrary{\idxcode{is_swappable}}%
\tcode{template <class T>}\br
  \tcode{struct is_swappable;} &
  For a referenceable type \tcode{T},
  the same result as \tcode{is_\brk\-swappable_\brk\-with_\brk\-v<}\brk\tcode{T\&,\brk T\&>},
  otherwise \tcode{false}. &
  \tcode{T} shall be a complete type,
  (possibly \cv-qualified) \tcode{void}, or
  an array of unknown bound. \\ \rowsep

\indexlibrary{\idxcode{is_nothrow_swappable_with}}%
\tcode{template <class T, class U>}\br
  \tcode{struct is_nothrow_swappable_with;} &
  \tcode{is_swappable_with_v<T, }\brk\tcode{U>} is \tcode{true} and
  each \tcode{swap} expression of the definition of
  \tcode{is_swappable_with<T, U>} is known not to throw
  any exceptions~(\cxxref{expr.unary.noexcept}). &
  \tcode{T} and \tcode{U} shall be complete types,
  (possibly \cv-qualified) \tcode{void}, or
  arrays of unknown bound. \\ \rowsep

\indexlibrary{\idxcode{is_nothrow_swappable}}%
\tcode{template <class T>}\br
  \tcode{struct is_nothrow_swappable;} &
  For a referenceable type \tcode{T},
  the same result as \tcode{is_nothrow_swappable_}\brk\tcode{with_v<T\&, T\&>},
  otherwise \tcode{false}. &
  \tcode{T} shall be a complete type,
  (possibly \cv-qualified) \tcode{void}, or
  an array of unknown bound. \\ \rowsep

\end{libreqtab3b}

\rSec2[meta.trans.other]{Other transformations}

\begin{libreqtab2b}{Other transformations}{tab:type-traits.other}
\\ \topline
\lhdr{Template} &   \rhdr{Comments} \\ \capsep
\endfirsthead
\continuedcaption\\
\topline
\lhdr{Template} &   \rhdr{Comments} \\ \capsep
\endhead

 \tcode{template <class... T>}\br
 \tcode{  struct common_type;}
 &
 The member typedef \tcode{type} shall be defined or omitted as specified below.
 If it is omitted, there shall be no member \tcode{type}.
 Each type in the parameter pack \tcode{T} shall be complete or
 (possibly \cv) \tcode{void}. A program may specialize this trait if at least one
 template parameter in the specialization depends on a
 user-defined type and \tcode{sizeof...(T) == 2}. \enternote Such
 specializations are needed when only explicit conversions are desired among the
 template arguments.
 \exitnote \\ \rowsep

\tcode{template <class T, class U,}\br
\tcode{    template <class> class TQual,}\br
\tcode{    template <class> class UQual>}\br
\tcode{  struct basic_common_reference;} &
The primary template shall have no member typedef
\tcode{type}. A program may specialize this trait if at least one
template parameter in the specialization depends on a user-defined
type. In such a specialization, a member typedef \tcode{type} may be
defined or omitted. If it is omitted, there shall be no member
\tcode{type}.
\enternote Such specializations may be used to influence
the result of \tcode{common_reference}.\exitnote \\ \rowsep

\tcode{template <class... T>}\br
\tcode{  struct common_reference;} &
The member typedef \tcode{type} shall be defined or omitted
as specified below. If it is omitted, there shall be no member
\tcode{type}. Each type in the parameter pack \tcode{T} shall be
complete or (possibly \cv) \tcode{void}. \\ \rowsep

\end{libreqtab2b}

\pnum
Let \tcode{CREF(A)} be \tcode{add_lvalue_reference_t<const
remove_reference_t<A>{}>}. Let \tcode{UNCVREF(A)} be
\tcode{remove_cv_t<remove_reference_t<A>{}>}. Let \tcode{XREF(A)}
denote a unary template \tcode{T} such that \tcode{T<UNCVREF(A)>}
denotes the same type as \tcode{A}. Let \tcode{COPYCV(FROM, TO)} be
an alias for type \tcode{TO} with the addition of \tcode{FROM}'s
top-level \cv-qualifiers. \enterexample \tcode{COPYCV(const int,
volatile short)} is an alias for \tcode{const volatile short}.
\exitexample Let \tcode{RREF_RES(Z)} be
\tcode{remove_reference_t<Z>\&\&} if \tcode{Z} is a reference type
or \tcode{Z} otherwise. Let \tcode{COND_RES(X, Y)} be
\tcode{decltype(declval<bool>() ? declval<X>() : declval<Y>())}.
Given types \tcode{A} and \tcode{B}, let \tcode{X} be
\tcode{remove_reference_t<A>}, let \tcode{Y} be
\tcode{remove_reference_t<B>}, and let \tcode{COMMON_REF(A, B)} be:
\begin{itemize}
\item If \tcode{A} and \tcode{B} are both lvalue reference types,
  \tcode{COMMON_REF(A, B)} is
  \tcode{COND_RES(COPYCV(X, Y) \&, COPYCV(Y, X) \&)}.
\item Otherwise, let \tcode{C} be
  \tcode{RREF_RES(COMMON_REF(X\&, Y\&))}. If \tcode{A} and \tcode{B}
  are both rvalue reference types, and \tcode{C} is well-formed,
  and \tcode{is_convertible<A, C>::value} and
  \tcode{is_convertible<B, C>::value} are \tcode{true}, then
  \tcode{COMMON_REF(A, B)} is \tcode{C}.
\item Otherwise, let \tcode{D} be
  \tcode{COMMON_REF(const X\&, Y\&)}. If \tcode{A} is an rvalue
  reference and \tcode{B} is an lvalue reference and \tcode{D} is
  well-formed and \tcode{is_convertible<A, D>::value} is
  \tcode{true}, then \tcode{COMMON_REF(A, B)} is \tcode{D}.
\item Otherwise, if \tcode{A} is an lvalue reference and \tcode{B}
  is an rvalue reference, then \tcode{COMMON_REF(A, B)} is
  \tcode{COMMON_REF(B, A)}.
\item Otherwise, \tcode{COMMON_REF(A, B)} is
  \tcode{decay_t<COND_RES(CREF(A), CREF(B))>}.
\end{itemize}

If any of the types computed above are ill-formed, then
\tcode{COMMON_REF(A, B)} is ill-formed.

\pnum
Note A: For the \tcode{common_type} trait applied to a parameter pack
\tcode{T} of types, the member \tcode{type} shall be either defined or not
present as follows:
\begin{itemize}
\item If \tcode{sizeof...(T)} is zero, there shall be no member \tcode{type}.
\item Otherwise, if \tcode{sizeof...(T)} is one, let T1
  denote the sole type in the pack \tcode{T}. The member typedef \tcode{type}
  shall denote the same type as \tcode{decay_t<T1>}.
\item Otherwise, if \tcode{sizeof...(T)} is two, let \tcode{T1} and \tcode{T2}
  denote the two types in the pack \tcode{T}, and let \tcode{D1} and \tcode{D2}
  be \tcode{decay_t<T1>} and \tcode{decay_t<T2>} respectively. Then
\begin{itemize}
\item If \tcode{D1} and \tcode{T1} denote the same type and \tcode{D2} and
  \tcode{T2} denote the same type, then
\begin{itemize}
\item If \tcode{std::common_type_t<T1, T2>} is well-formed, then the member typedef
  \tcode{type} denotes \tcode{std::common_type_t<T1, T2>}.
\item If \tcode{COMMON_REF(T1, T2)} is well-formed, then the member typedef
  \tcode{type} denotes that type.
\item Otherwise, there shall be no member \tcode{type}.
\end{itemize}
\item Otherwise, if \tcode{common_type_t<D1, D2>} is well-formed, then the
  member typedef \tcode{type} denotes that type.
\item Otherwise, there shall be no member \tcode{type}.
\end{itemize}
\item Otherwise, if \tcode{sizeof...(T)} is greater than two,
  let \tcode{T1}, \tcode{T2}, and \tcode{Rest},
  respectively, denote the first, second, and (pack of) remaining types
  comprising \tcode{T}. Let \tcode{C} be the type
  \tcode{common_type_t<T1, T2>}. Then:
\begin{itemize}
\item If there is such a type \tcode{C}, the member typedef
  \tcode{type} shall denote the same type, if any, as
  \tcode{common_type_t<C, Rest...>}.
\item Otherwise, there shall be no member \tcode{type}.
\end{itemize}
\end{itemize}

\pnum
Note B: A program may specialize the \tcode{common_type} trait for two
\cv-unqualified non-reference types if at least one of them depends on a
user-defined type. \enternote Such specializations are needed when only
explicit conversions are desired among the template arguments. \exitnote
Such a specialization need not have a member named \tcode{type}, but if it does,
that member shall be a \grammarterm{typedef-name} for a \cv-unqualified
non-reference type that need not otherwise meet the specification set forth in
Note A, above.

\pnum
For the \tcode{common_reference} trait applied to a parameter pack \tcode{T} of
types, the member \tcode{type} shall be either defined or not present as follows:
\begin{itemize}
\item If \tcode{sizeof...(T)} is zero, there shall be no member \tcode{type}.
\item Otherwise, if \tcode{sizeof...(T)} is one, let \tcode{T1} denote the sole
  type in the pack \tcode{T}. The member typedef \tcode{type} shall denote the
  same type as \tcode{T1}.
\item Otherwise, if \tcode{sizeof...(T)} is two, let \tcode{T1} and \tcode{T2}
  denote the two types in the pack \tcode{T}. Then
\begin{itemize}
\item If \tcode{COMMON_REF(T1, T2)} is well-formed and denotes a reference type
  then the member typedef \tcode{type} denotes that type.
\item Otherwise, if \tcode{basic_common_reference<UNCVREF(T1), UNCVREF(T2),
  XREF(T1), XREF(T2)>::type} is well-formed, then the member typedef
  \tcode{type} denotes that type.
\item Otherwise, if \tcode{common_type_t<T1, T2>} is well-formed, then the
  member typedef \tcode{type} denotes that type.
\item Otherwise, there shall be no member \tcode{type}.
\end{itemize}
\item Otherwise, if \tcode{sizeof...(T)} is greater than two, let \tcode{T1},
  \tcode{T2}, and \tcode{Rest}, respectively, denote the first, second, and
  (pack of) remaining types comprising \tcode{T}. Let \tcode{C} be the type
  \tcode{common_reference_t<T1, T2>}. Then:
\begin{itemize}
\item If there is such a type \tcode{C}, the member typedef \tcode{type} shall
  denote the same type, if any, as \tcode{common_reference_t<C, Rest...>}.
\item Otherwise, there shall be no member \tcode{type}.
\end{itemize}
\end{itemize}

\pnum
A program may specialize the \tcode{basic_common_reference} trait for
two \cv-unqualified non-reference types if at least one of them depends on a
user-defined type. Such a specialization need not have a member named
\tcode{type}.

\rSec1[taggedtup]{Tagged tuple-like types}

\rSec2[taggedtup.general]{In general}

\pnum The library provides a template for augmenting a tuple-like type with named element accessor
member functions. The library also provides several templates that provide access to \tcode{tagged}
objects as if they were \tcode{tuple} objects (see~\cxxref{tuple.elem}).

\rSec2[taggedtup.tagged]{Class template \tcode{tagged}}

\pnum
Class template \tcode{tagged} augments a tuple-like class type (e.g., \tcode{pair}~(\cxxref{pairs}),
\tcode{tuple}~(\cxxref{tuple})) by giving it named accessors. It is used to define the alias
templates \tcode{tagged_pair}~(\ref{tagged.pairs}) and
\tcode{tagged_tuple}~(\ref{tagged.tuple}).

\pnum In the class synopsis below, let $i$ be in the range
\range{0}{sizeof...(Tags)} and $T_i$ be the $i^{th}$ type in \tcode{Tags}, where indexing
is zero-based.

\indexlibrary{\idxcode{tagged}}%
\begin{codeblock}
// defined in header <experimental/ranges/utility>

namespace std { namespace experimental { namespace ranges { inline namespace v1 {
  template <class T>
  concept bool TagSpecifier = @\impdef@;

  template <class F>
  concept bool TaggedType = @\impdef@;

  template <class Base, TagSpecifier... Tags>
    requires sizeof...(Tags) <= tuple_size<Base>::value
  struct tagged :
    Base, @\textit{TAGGET}@(tagged<Base, Tags...>, @$T_i$@, @$i$@)... { // \seebelow
    using Base::Base;
    tagged() = default;
    tagged(tagged&&) = default;
    tagged(const tagged&) = default;
    tagged &operator=(tagged&&) = default;
    tagged &operator=(const tagged&) = default;
    tagged(Base&&) noexcept(@\seebelow@)
      requires MoveConstructible<Base>;
    tagged(const Base&) noexcept(@\seebelow@)
      requires CopyConstructible<Base>;
    template <class Other>
      requires Constructible<Base, Other>
    constexpr tagged(tagged<Other, Tags...> &&that) noexcept(@\seebelow@);
    template <class Other>
      requires Constructible<Base, const Other&>
    constexpr tagged(const tagged<Other, Tags...> &that);
    template <class Other>
      requires Assignable<Base&, Other>
    constexpr tagged& operator=(tagged<Other, Tags...>&& that) noexcept(@\seebelow@);
    template <class Other>
      requires Assignable<Base&, const Other&>
    constexpr tagged& operator=(const tagged<Other, Tags...>& that);
    template <class U>
      requires Assignable<Base&, U> && !Same<decay_t<U>, tagged>
    constexpr tagged& operator=(U&& u) noexcept(@\seebelow@);
    constexpr void swap(tagged& that) noexcept(@\seebelow@)
      requires Swappable<Base>;
    friend constexpr void swap(tagged&, tagged&) noexcept(@\seebelow@)
      requires Swappable<Base>;
  };
}}}}
\end{codeblock}

\pnum A \techterm{tagged getter} is an empty trivial class type that has a named member function that
returns a reference to a member of a tuple-like object that is assumed to be derived from the getter
class. The tuple-like type of a tagged getter is called its \techterm{DerivedCharacteristic}.
The index of the tuple element returned from the getter's member functions is called its
\techterm{ElementIndex}. The name of the getter's member function is called its
\techterm{ElementName}

\pnum A tagged getter class with DerivedCharacteristic \tcode{\textit{D}}, ElementIndex
\tcode{\textit{N}}, and ElementName \tcode{\textit{name}} shall provide the following interface:

\begin{codeblock}
struct @\xname{\textit{TAGGED_GETTER}}@ {
  constexpr decltype(auto) @$name$@() &       { return get<@$N$@>(static_cast<@$D$@&>(*this)); }
  constexpr decltype(auto) @$name$@() &&      { return get<@$N$@>(static_cast<@$D$@&&>(*this)); }
  constexpr decltype(auto) @$name$@() const & { return get<@$N$@>(static_cast<const @$D$@&>(*this)); }
};
\end{codeblock}

\pnum
A \techterm{tag specifier} is a type that facilitates a mapping from a tuple-like type and an
element index into a \textit{tagged getter} that gives named access to the element at that index.
\tcode{TagSpecifier<T>} is satisfied if and only if \tcode{T} is a tag specifier. The tag specifiers in the
\tcode{Tags} parameter pack shall be unique. \enternote The mapping mechanism from tag specifier to
tagged getter is unspecified.\exitnote

\pnum Let \tcode{\textit{TAGGET}(D, T, $N$)} name a tagged getter type that gives named
access to the $N$-th element of the tuple-like type \tcode{D}.

\pnum It shall not be possible to delete an instance of class template \tcode{tagged} through a
pointer to any base other than \tcode{Base}.

\pnum
\tcode{TaggedType<F>} is satisfied if and only if \tcode{F} is a unary function
type with return type \tcode{T} which satisfies \tcode{TagSpecifier<T>}. Let
\tcode{\textit{TAGSPEC}(F)} name the tag specifier of the \tcode{TaggedType} \tcode{F}, and let
\tcode{\textit{TAGELEM}(F)} name the argument type of the \tcode{TaggedType} \tcode{F}.

\indexlibrary{\idxcode{tagged}!\idxcode{tagged}}

\begin{itemdecl}
tagged(Base&& that) noexcept(@\seebelow@)
  requires MoveConstructible<Base>;
\end{itemdecl}

\begin{itemdescr}
\pnum
\effects Initializes \tcode{Base} with \tcode{std::move(that)}.

\pnum
\remarks The expression in the \tcode{noexcept} is equivalent to:

\begin{codeblock}
is_nothrow_move_constructible<Base>::value
\end{codeblock}
\end{itemdescr}

\begin{itemdecl}
tagged(const Base& that) noexcept(@\seebelow@)
  requires CopyConstructible<Base>;
\end{itemdecl}

\begin{itemdescr}
\pnum
\effects Initializes \tcode{Base} with \tcode{that}.

\pnum
\remarks The expression in the \tcode{noexcept} is equivalent to:

\begin{codeblock}
is_nothrow_copy_constructible<Base>::value
\end{codeblock}
\end{itemdescr}

\begin{itemdecl}
template <class Other>
  requires Constructible<Base, Other>
constexpr tagged(tagged<Other, Tags...> &&that) noexcept(@\seebelow@);
\end{itemdecl}

\begin{itemdescr}
\pnum
\effects Initializes \tcode{Base} with \tcode{static_cast<Other\&\&>(that)}.

\pnum
\remarks The expression in the \tcode{noexcept} is equivalent to:

\begin{codeblock}
is_nothrow_constructible<Base, Other>::value
\end{codeblock}
\end{itemdescr}

\indexlibrary{\idxcode{tagged}!\idxcode{tagged}}
\begin{itemdecl}
template <class Other>
  requires Constructible<Base, const Other&>
constexpr tagged(const tagged<Other, Tags...>& that);
\end{itemdecl}

\begin{itemdescr}
\pnum
\effects Initializes \tcode{Base} with \tcode{static_cast<const Other\&>(that)}.
\end{itemdescr}

\indexlibrary{\idxcode{operator=}!\idxcode{tagged}}
\indexlibrary{\idxcode{tagged}!\idxcode{operator=}}
\begin{itemdecl}
template <class Other>
  requires Assignable<Base&, Other>
constexpr tagged& operator=(tagged<Other, Tags...>&& that) noexcept(@\seebelow@);
\end{itemdecl}

\begin{itemdescr}
\pnum
\effects Assigns \tcode{static_cast<Other\&\&>(that)} to \tcode{static_cast<Base\&>(*this)}.

\pnum
\returns \tcode{*this}.

\pnum
\remarks The expression in the \tcode{noexcept} is equivalent to:

\begin{codeblock}
is_nothrow_assignable<Base&, Other>::value
\end{codeblock}
\end{itemdescr}

\indexlibrary{\idxcode{operator=}!\idxcode{tagged}}
\indexlibrary{\idxcode{tagged}!\idxcode{operator=}}
\begin{itemdecl}
template <class Other>
  requires Assignable<Base&, const Other&>
constexpr tagged& operator=(const tagged<Other, Tags...>& that);
\end{itemdecl}

\begin{itemdescr}
\pnum
\effects Assigns \tcode{static_cast<const Other\&>(that)} to \tcode{static_cast<Base\&>(*this)}.

\pnum
\returns \tcode{*this}.
\end{itemdescr}

\indexlibrary{\idxcode{operator=}!\idxcode{tagged}}
\indexlibrary{\idxcode{tagged}!\idxcode{operator=}}
\begin{itemdecl}
template <class U>
  requires Assignable<Base&, U> && !Same<decay_t<U>, tagged>
constexpr tagged& operator=(U&& u) noexcept(@\seebelow@);
\end{itemdecl}

\begin{itemdescr}
\pnum
\effects Assigns \tcode{std::forward<U>(u)} to \tcode{static_cast<Base\&>(*this)}.

\pnum
\returns \tcode{*this}.

\pnum
\remarks The expression in the \tcode{noexcept} is equivalent to:

\begin{codeblock}
is_nothrow_assignable<Base&, U>::value
\end{codeblock}
\end{itemdescr}

\indexlibrary{\idxcode{swap}!\idxcode{tagged}}
\indexlibrary{\idxcode{tagged}!\idxcode{swap}}
\begin{itemdecl}
constexpr void swap(tagged& rhs) noexcept(@\seebelow@)
  requires Swappable<Base>;
\end{itemdecl}

\begin{itemdescr}
\pnum
\effects Calls \tcode{swap} on the result of applying \tcode{static_cast} to \tcode{*this} and
\tcode{that}.

\pnum
\throws Nothing unless the call to \tcode{swap} on the \tcode{Base} sub-objects throws.

\pnum
\remarks The expression in the \tcode{noexcept} is equivalent to:

\begin{codeblock}
noexcept(swap(declval<Base&>(), declval<Base&>()))
\end{codeblock}
\end{itemdescr}

\indexlibrary{\idxcode{swap}!\tcode{tagged}}%
\begin{itemdecl}
friend constexpr void swap(tagged& lhs, tagged& rhs) noexcept(@\seebelow@)
  requires Swappable<Base>;
\end{itemdecl}

\begin{itemdescr}
\pnum
\effects Equivalent to \tcode{lhs.swap(rhs)}.

\pnum
\remarks The expression in the \tcode{noexcept} is equivalent to:

\begin{codeblock}
noexcept(lhs.swap(rhs))
\end{codeblock}
\end{itemdescr}

\rSec2[tagged.astuple]{Tuple-like access to \tcode{tagged}}

\indexlibrary{\idxcode{tuple_size}}%
\indexlibrary{\idxcode{tuple_element}}%
\begin{itemdecl}
namespace std {
  template <class Base, class... Tags>
  struct tuple_size<experimental::ranges::tagged<Base, Tags...>>
    : tuple_size<Base> { };

  template <size_t N, class Base, class... Tags>
  struct tuple_element<N, experimental::ranges::tagged<Base, Tags...>>
    : tuple_element<N, Base> { };
}
\end{itemdecl}

\rSec2[tagged.pairs]{Alias template \tcode{tagged_pair}}

\begin{codeblock}
// defined in header <experimental/ranges/utility>

namespace std { namespace experimental { namespace ranges { inline namespace v1 {
  // ...
  template <TaggedType T1, TaggedType T2>
  using tagged_pair = tagged<pair<@\textit{TAGELEM}@(T1), @\textit{TAGELEM}@(T2)>,
                             @\textit{TAGSPEC}@(T1), @\textit{TAGSPEC}@(T2)>;
}}}}
\end{codeblock}

\pnum \enterexample
\begin{codeblock}
// See \ref{alg.tagspec}:
tagged_pair<tag::min(int), tag::max(int)> p{0, 1};
assert(&p.min() == &p.first);
assert(&p.max() == &p.second);
\end{codeblock}
\exitexample

\rSec3[tagged.pairs.creation]{Tagged pair creation functions}

\indexlibrary{\idxcode{make_tagged_pair}}%
\begin{itemdecl}
// defined in header <experimental/ranges/utility>

namespace std { namespace experimental { namespace ranges { inline namespace v1 {
  template <TagSpecifier Tag1, TagSpecifier Tag2, class T1, class T2>
    constexpr @\seebelow@ make_tagged_pair(T1&& x, T2&& y);
}}}}
\end{itemdecl}

\begin{itemdescr}
\pnum
Let \tcode{P} be the type of \tcode{make_pair(std::forward<T1>(x), std::forward<T2>(y))}.
Then the return type is \tcode{tagged<P, Tag1, Tag2>}.

\pnum
\returns \tcode{\{std::forward<T1>(x), std::forward<T2>(y)\}}.

\pnum
\enterexample
In place of:

\begin{codeblock}
  return tagged_pair<tag::min(int), tag::max(double)>(5, 3.1415926);   // explicit types
\end{codeblock}

a \Cpp program may contain:

\begin{codeblock}
  return make_tagged_pair<tag::min, tag::max>(5, 3.1415926);           // types are deduced
\end{codeblock}
\exitexample
\end{itemdescr}

\rSec2[tagged.tuple]{Alias template \tcode{tagged_tuple}}

\pnum
\synopsis{Header \tcode{<experimental/ranges/tuple>} synopsis}

\begin{codeblock}
namespace std { namespace experimental { namespace ranges { inline namespace v1 {
  template <TaggedType... Types>
  using tagged_tuple = tagged<tuple<@\textit{TAGELEM}@(Types)...>,
                              @\textit{TAGSPEC}@(Types)...>;

  template <TagSpecifier... Tags, class... Types>
    requires sizeof...(Tags) == sizeof...(Types)
      constexpr @\seebelow@ make_tagged_tuple(Types&&... t);
}}}}
\end{codeblock}

\pnum
\begin{codeblock}
template <TaggedType... Types>
using tagged_tuple = tagged<tuple<@\textit{TAGELEM}@(Types)...>,
                            @\textit{TAGSPEC}@(Types)...>;
\end{codeblock}

\pnum \enterexample
\begin{codeblock}
// See \ref{alg.tagspec}:
tagged_tuple<tag::in(char*), tag::out(char*)> t{0, 0};
assert(&t.in() == &get<0>(t));
assert(&t.out() == &get<1>(t));
\end{codeblock}
\exitexample

\rSec3[tagged.tuple.creation]{Tagged tuple creation functions}

\indexlibrary{\idxcode{make_tagged_tuple}}%
\indexlibrary{\idxcode{tagged_tuple}!\idxcode{make_tagged_tuple}}%
\begin{itemdecl}
template <TagSpecifier... Tags, class... Types>
  requires sizeof...(Tags) == sizeof...(Types)
    constexpr @\seebelow@ make_tagged_tuple(Types&&... t);
\end{itemdecl}

\begin{itemdescr}
\pnum
Let \tcode{T} be the type of \tcode{make_tuple(std::forward<Types>(t)...)}.
Then the return type is \tcode{tagged<T, Tags...>}.

\pnum
\returns \tcode{tagged<T, Tags...>(std::forward<Types>(t)...)}.

\pnum
\enterexample

\begin{codeblock}
int i; float j;
make_tagged_tuple<tag::in1, tag::in2, tag::out>(1, ref(i), cref(j))
\end{codeblock}

creates a tagged tuple of type

\begin{codeblock}
tagged_tuple<tag::in1(int), tag::in2(int&), tag::out(const float&)>
\end{codeblock}
\exitexample
\end{itemdescr}
