%!TEX root = std.tex

\rSec0[stmt]{Statements}

\rSec1[stmt.iter]{Iteration statements}

\rSec2[stmt.ranged]{The range-based \tcode{for} statement}

\pnum
\enternote This clause is presented as a set of differences to apply to
\cxxref{stmt.ranged} to allow differently typed begin and end
iterators, as in C++17. \exitnote

\begin{quote}
\pnum
\begin{removedblock}
For a range-based \tcode{for} statement of the form

\begin{bnf}
\terminal{for} \terminal{(} for-range-declaration \terminal{:} expression \terminal{)} statement
\end{bnf}
%
let \grammarterm{range-init} be equivalent to the \grammarterm{expression} surrounded by parentheses

\begin{bnf}
\terminal{(} expression \terminal{)}
\end{bnf}
%
and for a range-based \tcode{for} statement of the form

\begin{bnf}
\terminal{for} \terminal{(} for-range-declaration \terminal{:} braced-init-list \terminal{)} statement
\end{bnf}
%
let \grammarterm{range-init} be equivalent to the \grammarterm{braced-init-list}.
%
In each case, a range-based \tcode{for} statement is equivalent to
\begin{codeblock}
{
  auto && __range = range-init;
  for ( auto __begin = begin-expr,
             __end = end-expr;
        __begin != __end;
        ++__begin ) {
    @\placeholder{for-range-declaration}@= *__begin;
    @\placeholder{statement}@
  }
}
\end{codeblock}
\end{removedblock}

\begin{addedblock}
The range-based \tcode{for} statement

\begin{bnf}
\terminal{for} \terminal{(} for-range-declaration \terminal{:} for-range-initializer \terminal{)} statement
\end{bnf}
%
is equivalent to
\begin{codeblock}
{
  auto &&__range = @\grammarterm{for-range-initializer}@;
  auto __begin = @\grammarterm{begin-expr}@;
  auto __end = @\grammarterm{end-expr}@;
  for ( ; __begin != __end; ++__begin ) {
    @\grammarterm{for-range-declaration}@ = *__begin;
    @\grammarterm{statement}@
  }
}
\end{codeblock}
\end{addedblock}
%
where
\begin{itemize}
\item
\added{if the \grammarterm{for-range-initializer} is an \grammarterm{expression}, it is
regarded as if it were surrounded by parentheses (so that a comma operator cannot
be reinterpreted as delimiting two \grammarterm{init-declarators});}

\item
\tcode{__range}, \tcode{__begin}, and \tcode{__end} are variables defined for
exposition only; and \tcode{\removed{_RangeT}}\removed{ is the type of the
\grammarterm{}{expression}, and \textit{begin-expr} and \textit{end-expr} are
determined as follows:}

\item
\added{\grammarterm{begin-expr} and \grammarterm{end-expr} are determined as follows:}

\begin{itemize}
\item
if \tcode{\removed{_RangeT}}\added{the \grammarterm{for-range-initializer}} is an
\added{expression of} array type \added{\tcode{R}}, \textit{begin-expr} and \textit{end-expr} are
\tcode{__range} and \tcode{__range + __bound}, respectively, where \tcode{__bound} is
the array bound. If \tcode{\removed{_RangeT}}\tcode{\added{R}} is an array of unknown
\changed{size}{bound} or an array of incomplete type, the program is ill-formed;

\item
if \tcode{\removed{_RangeT}}\added{the \grammarterm{for-range-initializer}} is
\changed{a}{an expression of} class type \added{\tcode{C}}, the
\grammarterm{unqualified-id}{s} \tcode{begin} and \tcode{end} are looked up in
the scope of \changed{class \tcode{\mbox{_RangeT}}}{\tcode{C}} as if by class
member access lookup~(\stdcxxref{basic.lookup.classref}), and if either
(or both) finds at least one declaration, \grammarterm{begin-expr} and
\grammarterm{end-expr} are \tcode{__range.begin()} and \tcode{__range.end()},
respectively;

\item
otherwise, \textit{begin-expr} and \textit{end-expr} are \tcode{begin(__range)}
and \tcode{end(__range)}, respectively, where \tcode{begin} and \tcode{end} are looked
up in the associated namespaces~(\stdcxxref{basic.lookup.argdep}).
\enternote Ordinary unqualified lookup~(\stdcxxref{basic.lookup.unqual}) is not
performed. \exitnote
\end{itemize}
\end{itemize}

\enterexample
\begin{codeblock}
int array[5] = { 1, 2, 3, 4, 5 };
for (int& x : array)
  x *= 2;
\end{codeblock}
\exitexample%
\indextext{statement!iteration|)}

\pnum
In the \grammarterm{decl-specifier-seq} of a \grammarterm{for-range-declaration},
each \grammarterm{decl-specifier} shall be either a \grammarterm{type-specifier}
or \tcode{constexpr}. The \grammarterm{decl-specifier-seq} shall not define a
class or enumeration.
\end{quote}
