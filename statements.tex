%!TEX root = std.tex

\rSec0[stmt]{Statements}

\rSec1[stmt.iter]{Iteration statements}

\rSec2[stmt.ranged]{The range-based \tcode{for} statement}

\ednote{Modify \cxxref{stmt.ranged}/p1 to allow differently typed begin and end
iterators, like in C++17.}

\begin{quote}
\pnum
For a range-based for statement of the form

\begin{bnf}
\terminal{for} \terminal{(} for-range-declaration \terminal{:} expression \terminal{)} statement
\end{bnf}
%
let \grammarterm{range-init} be equivalent to the expression surrounded by parentheses

\begin{bnf}
\terminal{(} expression \terminal{)}
\end{bnf}
%
and for a range-based for statement of the form

\begin{bnf}
\terminal{for} \terminal{(} for-range-declaration \terminal{:} braced-init-list \terminal{)} statement
\end{bnf}
%
let \grammarterm{range-init} be equivalent to the \grammarterm{braced-init-list}.
%
In each case, a range-based for statement is equivalent to
\begin{removedblock}
\begin{codeblock}
{
  auto && __range = range-init;
  for ( auto __begin = begin-expr,
             __end = end-expr;
        __begin != __end;
        ++__begin ) {
    @\grammarterm{for-range-declaration}@ = *__begin;
    @\grammarterm{statement}@
  }
}
\end{codeblock}
\end{removedblock}
\begin{addedblock}
\begin{codeblock}
{
  auto && __range = @\grammarterm{range-init}@;
  auto __begin = @\grammarterm{begin-expr}@;
  auto __end = @\grammarterm{end-expr}@;
  for ( ; __begin != __end; ++__begin ) {
    @\grammarterm{for-range-declaration}@ = *__begin;
    @\grammarterm{statement}@
  }
}
\end{codeblock}
\end{addedblock}
%
where \tcode{__range}, \tcode{__begin}, and \tcode{__end} are variables defined for
exposition only, and \tcode{_RangeT} is the type of the
\grammarterm{}{expression}, and \textit{begin-expr} and \textit{end-expr} are
determined as follows:

\begin{itemize}
\item if \tcode{_RangeT} is an array type, \textit{begin-expr} and \textit{end-expr} are
\tcode{__range} and \tcode{__range + __bound}, respectively, where \tcode{__bound} is
the array bound. If \tcode{_RangeT} is an array of unknown size or an array of
incomplete type, the program is ill-formed;

\item if \tcode{_RangeT} is a class type, the \grammarterm{unqualified-id}{s}
\tcode{begin} and \tcode{end} are looked up in the scope of class \tcode{\mbox{_RangeT}}
as if by class member access lookup~(\stdcxxref{basic.lookup.classref}), and if either
(or both) finds at least one declaration, \grammarterm{begin-expr} and
\grammarterm{end-expr} are \tcode{__range.begin()} and \tcode{__range.end()},
respectively;

\item otherwise, \textit{begin-expr} and \textit{end-expr} are \tcode{begin(__range)}
and \tcode{end(__range)}, respectively, where \tcode{begin} and \tcode{end} are looked
up in the associated namespaces~(\stdcxxref{basic.lookup.argdep}).
\enternote Ordinary unqualified lookup~(\stdcxxref{basic.lookup.unqual}) is not
performed. \exitnote
\end{itemize}

\enterexample
\begin{codeblock}
int array[5] = { 1, 2, 3, 4, 5 };
for (int& x : array)
  x *= 2;
\end{codeblock}
\exitexample%
\indextext{statement!iteration|)}

\pnum
In the \grammarterm{decl-specifier-seq} of a \grammarterm{for-range-declaration},
each \grammarterm{decl-specifier} shall be either a \grammarterm{type-specifier}
or \tcode{constexpr}. The \grammarterm{decl-specifier-seq} shall not define a
class or enumeration.
\end{quote}
