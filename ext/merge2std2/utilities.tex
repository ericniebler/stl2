%!TEX root = std.tex
\rSec0[std2.utilities]{General utilities library}

\rSec1[std2.utilities.general]{General}

\pnum
This \changed{C}{Subc}lause describes utilities that are generally useful in \Cpp programs; some
of these utilities are used by other elements of the \changed{Ranges library}{Standard Library, Version 2}.
These utilities are summarized in Table~\ref{tab:util.lib.summary}.

\begin{libsumtab}{General utilities library summary}{tab:util.lib.summary}
\ref{std2.utility}          & Utility components      & \tcode{<\changed{experimental/ranges}{std2}/utility>}     \\ \rowsep
\ref{std2.function.objects} & Function objects        & \tcode{<\changed{experimental/ranges}{std2}/functional>}  \\ \rowsep
\removed{\ref{std2.meta}}   & \removed{Type traits}   & \removed{\tcode{<type_traits>}}                           \\ \rowsep
\ref{std2.taggedtup}        & Tagged tuple-like types & \tcode{<\changed{experimental/ranges}{std2}/utility>} \&  \\
                            &                         & \tcode{<\changed{experimental/ranges}{std2}/tuple>}       \\
\end{libsumtab}

\rSec1[std2.utility]{Utility components}

\pnum
This subclause contains some basic function and class templates that are used
throughout the rest of the library.

\indexlibrary{\idxhdr{std2/utility}}%
\synopsis{Header \tcode{<\changed{experimental/ranges}{std2}/utility>} synopsis}

\pnum
The header \tcode{<\changed{experimental/ranges}{std2}/utility>} defines several types,
function templates, and concepts that are described in this Clause. It also
defines the templates \tcode{tagged} and \tcode{tagged_pair} and various
function templates that operate on \tcode{tagged_pair} objects.

\begin{codeblock}
namespace @\changed{std \{ namespace experimental \{ namespace ranges}{std2}@ { inline namespace v1 {
  @\removed{// \ref{std2.utility.swap}, swap:}@
  @\removed{namespace \{}@
    @\removed{constexpr \unspec swap = \unspec;}@
  @\removed{\}}@

  // \ref{std2.utility.exchange}, exchange:
  template <MoveConstructible T, class U=T>
    requires Assignable<T&, U>
  constexpr T exchange(T& obj, U&& new_val) noexcept(@\seebelow@);

  // \ref{std2.taggedtup.tagged}, struct with named accessors
  template <class T>
  concept @\removed{bool}@ TagSpecifier = @\seebelow@;

  template <class F>
  concept @\removed{bool}@ TaggedType = @\seebelow@;

  template <class Base, TagSpecifier... Tags>
    requires sizeof...(Tags) <= tuple_size<Base>::value
  struct tagged;

  // \ref{std2.tagged.pairs}, tagged pairs
  template <TaggedType T1, TaggedType T2> using tagged_pair = @\seebelow@;

  template <TagSpecifier Tag1, TagSpecifier Tag2, class T1, class T2>
  constexpr @\seebelow@ make_tagged_pair(T1&& x, T2&& y);
}}@\removed{\}\}}@

namespace std {
  // \ref{std2.tagged.astuple}, tuple-like access to tagged
  template <class Base, class... Tags>
  struct tuple_size<@\changed{experimental::ranges}{::std2}@::tagged<Base, Tags...>>;

  template <size_t N, class Base, class... Tags>
  struct tuple_element<N, @\changed{experimental::ranges}{::std2}@::tagged<Base, Tags...>>;
}
\end{codeblock}

\ednote{Section [utility.swap] ``swap'' is moved to P0898.}

\rSec2[std2.utility.exchange]{\tcode{exchange}}

\begin{itemdecl}
template <MoveConstructible T, class U=T>
  requires Assignable<T&, U>
constexpr T exchange(T& obj, U&& new_val) noexcept(@\seebelow@);
\end{itemdecl}

\begin{itemdescr}
\pnum
\effects
Equivalent to:

\begin{codeblock}
T old_val = std::move(obj);
obj = std::forward<U>(new_val);
return old_val;
\end{codeblock}

\remarks
The expression in \tcode{noexcept} is equivalent to:
\begin{codeblock}
is_nothrow_move_constructible<T>::value &&
is_nothrow_assignable<T&, U>::value
\end{codeblock}
\end{itemdescr}

\rSec1[std2.function.objects]{Function objects}

\pnum
\synopsis{Header \tcode{<\changed{experimental/ranges}{std2}/functional>} synopsis}

\begin{codeblock}
namespace @\changed{std \{ namespace experimental \{ namespace ranges}{std2}@ { inline namespace v1 {
  @\removed{// \ref{std2.func.invoke}, invoke:}@
  @\removed{template <class F, class... Args>}@
  @\removed{result_of_t<F\&\&(Args\&\&...)> invoke(F\&\& f, Args\&\&... args);}@

  // \ref{std2.comparisons}, comparisons:
  template <class T = void>
    requires @\seebelow@
  struct equal_to;

  template <class T = void>
    requires @\seebelow@
  struct not_equal_to;

  template <class T = void>
    requires @\seebelow@
  struct greater;

  template <class T = void>
    requires @\seebelow@
  struct less;

  template <class T = void>
    requires @\seebelow@
  struct greater_equal;

  template <class T = void>
    requires @\seebelow@
  struct less_equal;

  template <> struct equal_to<void>;
  template <> struct not_equal_to<void>;
  template <> struct greater<void>;
  template <> struct less<void>;
  template <> struct greater_equal<void>;
  template <> struct less_equal<void>;

  // \ref{std2.func.identity}, identity:
  struct identity;
}}@\removed{\}\}}@
\end{codeblock}

\ednote{Section [std2.func.invoke] ``Function template \tcode{invoke}'' is
intentionally omitted.}

\rSec2[std2.comparisons]{Comparisons}

\pnum
The library provides basic function object classes for all of the comparison
operators in the language~(\cxxref{expr.rel}, \cxxref{expr.eq}).

\pnum
In this section, \tcode{\textit{BUILTIN_PTR_CMP}(T, $op$, U)} for types \tcode{T}
and \tcode{U} and where $op$ is an equality~(\cxxref{expr.eq}) or relational
operator~(\cxxref{expr.rel}) is a boolean constant expression.
\tcode{\textit{BUILTIN_PTR_CMP}(T, $op$, U)} is \tcode{true} if and only if $op$
in the expression \tcode{declval<T>() $op$ declval<U>()} resolves to a built-in
operator comparing pointers.

\pnum
There is an implementation-defined strict total ordering over all pointer values
of a given type. This total ordering is consistent with the partial order imposed
by the builtin operators \tcode{<}, \tcode{>}, \tcode{<=}, and \tcode{>=}.

\indexlibrary{\idxcode{equal_to}}%
\begin{itemdecl}
template <class T = void>
  requires EqualityComparable<T> || Same<T, void> || @\textit{\tcode{BUILTIN_PTR_CMP}}@(const T&, ==, const T&)
struct equal_to {
  constexpr bool operator()(const T& x, const T& y) const;
};
\end{itemdecl}

\begin{itemdescr}
\pnum
\tcode{operator()} has effects equivalent to: \tcode{return equal_to<>{}(x, y);}
\end{itemdescr}

\indexlibrary{\idxcode{not_equal_to}}%
\begin{itemdecl}
template <class T = void>
  requires EqualityComparable<T> || Same<T, void> || @\textit{\tcode{BUILTIN_PTR_CMP}}@(const T&, ==, const T&)
struct not_equal_to {
  constexpr bool operator()(const T& x, const T& y) const;
};
\end{itemdecl}

\begin{itemdescr}
\pnum
\tcode{operator()} has effects equivalent to: \tcode{return !equal_to<>{}(x, y);}
\end{itemdescr}

\indexlibrary{\idxcode{greater}}%
\begin{itemdecl}
template <class T = void>
  requires StrictTotallyOrdered<T> || Same<T, void> || @\textit{\tcode{BUILTIN_PTR_CMP}}@(const T&, <, const T&)
struct greater {
  constexpr bool operator()(const T& x, const T& y) const;
};
\end{itemdecl}

\begin{itemdescr}
\pnum
\tcode{operator()} has effects equivalent to: \tcode{return less<>{}(y, x);}
\end{itemdescr}

\indexlibrary{\idxcode{less}}%
\begin{itemdecl}
template <class T = void>
  requires StrictTotallyOrdered<T> || Same<T, void> || @\textit{\tcode{BUILTIN_PTR_CMP}}@(const T&, <, const T&)
struct less {
  constexpr bool operator()(const T& x, const T& y) const;
};
\end{itemdecl}

\begin{itemdescr}
\pnum
\tcode{operator()} has effects equivalent to: \tcode{return less<>{}(x, y);}
\end{itemdescr}

\indexlibrary{\idxcode{greater_equal}}%
\begin{itemdecl}
template <class T = void>
  requires StrictTotallyOrdered<T> || Same<T, void> || @\textit{\tcode{BUILTIN_PTR_CMP}}@(const T&, <, const T&)
struct greater_equal {
  constexpr bool operator()(const T& x, const T& y) const;
};
\end{itemdecl}

\begin{itemdescr}
\pnum
\tcode{operator()} has effects equivalent to: \tcode{return !less<>{}(x, y);}
\end{itemdescr}

\indexlibrary{\idxcode{less_equal}}%
\begin{itemdecl}
template <class T = void>
  requires StrictTotallyOrdered<T> || Same<T, void> || @\textit{\tcode{BUILTIN_PTR_CMP}}@(const T&, <, const T&)
struct less_equal {
  constexpr bool operator()(const T& x, const T& y) const;
};
\end{itemdecl}

\begin{itemdescr}
\pnum
\tcode{operator()} has effects equivalent to: \tcode{return !less<>{}(y, x);}
\end{itemdescr}

\indexlibrary{\idxcode{equal_to<>}}%
\begin{itemdecl}
template <> struct equal_to<void> {
  template <class T, class U>
    requires EqualityComparableWith<T, U> || @\textit{\tcode{BUILTIN_PTR_CMP}}@(T, ==, U)
  constexpr bool operator()(T&& t, U&& u) const;

  typedef @\unspec@ is_transparent;
};
\end{itemdecl}

\begin{itemdescr}
\pnum
\requires If the expression \tcode{std::forward<T>(t) == std::forward<U>(u)}
results in a call to a built-in operator \tcode{==} comparing pointers of type
\tcode{P}, the conversion sequences from both \tcode{T} and \tcode{U} to \tcode{P}
shall be equality-preserving~(\ref{concepts.lib.general.equality}).

\pnum
\effects
\begin{itemize}
\item
If the expression \tcode{std::forward<T>(t) == std::forward<U>(u)} results in a
call to a built-in operator \tcode{==} comparing pointers of type \tcode{P}:
returns \tcode{false} if either (the converted value of) \tcode{t} precedes
\tcode{u} or \tcode{u} precedes \tcode{t} in the implementation-defined strict
total order over pointers of type \tcode{P} and otherwise \tcode{true}.

\item
Otherwise, equivalent to: \tcode{return std::forward<T>(t) == std::forward<U>(u);}
\end{itemize}
\end{itemdescr}

\indexlibrary{\idxcode{not_equal_to<>}}%
\begin{itemdecl}
template <> struct not_equal_to<void> {
  template <class T, class U>
    requires EqualityComparableWith<T, U> || @\textit{\tcode{BUILTIN_PTR_CMP}}@(T, ==, U)
  constexpr bool operator()(T&& t, U&& u) const;

  typedef @\unspec@ is_transparent;
};
\end{itemdecl}

\begin{itemdescr}
\pnum
\tcode{operator()} has effects equivalent to:
\begin{codeblock}
return !equal_to<>{}(std::forward<T>(t), std::forward<U>(u));
\end{codeblock}
\end{itemdescr}

\indexlibrary{\idxcode{greater<>}}%
\begin{itemdecl}
template <> struct greater<void> {
  template <class T, class U>
    requires StrictTotallyOrderedWith<T, U> || @\textit{\tcode{BUILTIN_PTR_CMP}}@(U, <, T)
  constexpr bool operator()(T&& t, U&& u) const;

  typedef @\unspec@ is_transparent;
};
\end{itemdecl}

\begin{itemdescr}
\pnum
\tcode{operator()} has effects equivalent to:
\begin{codeblock}
return less<>{}(std::forward<U>(u), std::forward<T>(t));
\end{codeblock}
\end{itemdescr}

\indexlibrary{\idxcode{less<>}}%
\begin{itemdecl}
template <> struct less<void> {
  template <class T, class U>
    requires StrictTotallyOrderedWith<T, U> || @\textit{\tcode{BUILTIN_PTR_CMP}}@(T, <, U)
  constexpr bool operator()(T&& t, U&& u) const;

  typedef @\unspec@ is_transparent;
};
\end{itemdecl}

\begin{itemdescr}
\pnum
\requires
If the expression \tcode{std::forward<T>(t) < std::forward<U>(u)} results in a
call to a built-in operator \tcode{<} comparing pointers of type \tcode{P}, the
conversion sequences from both \tcode{T} and \tcode{U} to \tcode{P} shall be
equality-preserving~(\ref{concepts.lib.general.equality}). For any expressions
\tcode{ET} and \tcode{EU} such that \tcode{decltype((ET))} is \tcode{T} and
\tcode{decltype((EU))} is \tcode{U}, exactly one of \tcode{less<>\{\}(ET, EU)},
\tcode{less<>\{\}(EU, ET)} or \tcode{equal_to<>\{\}(ET, EU)} shall be
\tcode{true}.

\pnum
\effects
\begin{itemize}
\item
If the expression \tcode{std::forward<T>(t) < std::forward<U>(u)} results in a
call to a built-in operator \tcode{<} comparing pointers of type \tcode{P}:
returns \tcode{true} if (the converted value of) \tcode{t} precedes \tcode{u} in
the implementation-defined strict total order over pointers of type \tcode{P}
and otherwise \tcode{false}.

\item
Otherwise, equivalent to: \tcode{return std::forward<T>(t) < std::forward<U>(u);}
\end{itemize}
\end{itemdescr}

\indexlibrary{\idxcode{greater_equal<>}}%
\begin{itemdecl}
template <> struct greater_equal<void> {
  template <class T, class U>
    requires StrictTotallyOrderedWith<T, U> || @\textit{\tcode{BUILTIN_PTR_CMP}}@(T, <, U)
  constexpr bool operator()(T&& t, U&& u) const;

  typedef @\unspec@ is_transparent;
};
\end{itemdecl}

\begin{itemdescr}
\pnum
\tcode{operator()} has effects equivalent to:
\begin{codeblock}
return !less<>{}(std::forward<T>(t), std::forward<U>(u));
\end{codeblock}
\end{itemdescr}

\indexlibrary{\idxcode{less_equal<>}}%
\begin{itemdecl}
template <> struct less_equal<void> {
  template <class T, class U>
    requires StrictTotallyOrderedWith<T, U> || @\textit{\tcode{BUILTIN_PTR_CMP}}@(U, <, T)
  constexpr bool operator()(T&& t, U&& u) const;

  typedef @\unspec@ is_transparent;
};
\end{itemdecl}

\begin{itemdescr}
\pnum
\tcode{operator()} has effects equivalent to:
\begin{codeblock}
return !less<>{}(std::forward<U>(u), std::forward<T>(t));
\end{codeblock}
\end{itemdescr}

\rSec2[std2.func.identity]{Class \tcode{identity}}

\indexlibrary{\idxcode{identity}}%
\begin{itemdecl}
struct identity {
  template <class T>
  constexpr T&& operator()(T&& t) const noexcept;

  typedef @\unspec@ is_transparent;
};
\end{itemdecl}

\begin{itemdescr}
\pnum
\tcode{operator()} returns \tcode{std::forward<T>(t)}.

\end{itemdescr}

\ednote{Subsection [meta] ``Metaprogramming and type traits'' is intentionally omitted.}

\rSec1[std2.taggedtup]{Tagged tuple-like types}

\rSec2[std2.taggedtup.general]{General}

\pnum The library provides a template for augmenting a tuple-like type with named element accessor
member functions. The library also provides several templates that provide access to \tcode{tagged}
objects as if they were \tcode{tuple} objects (see~\cxxref{tuple.elem}).

\rSec2[std2.taggedtup.tagged]{Class template \tcode{tagged}}

\pnum
Class template \tcode{tagged} augments a tuple-like class type (e.g., \tcode{pair}~(\cxxref{pairs}),
\tcode{tuple} (\cxxref{tuple})) by giving it named accessors. It is used to define the alias
templates \tcode{tagged_pair}~(\ref{std2.tagged.pairs}) and
\tcode{tagged_tuple}~(\ref{std2.tagged.tuple}).

\pnum In the class synopsis below, let $i$ be in the range
\range{0}{sizeof...(Tags)} and $T_i$ be the $i^{th}$ type in \tcode{Tags}, where indexing
is zero-based.

\indexlibrary{\idxcode{tagged}}%
\begin{codeblock}
// defined in header <\changed{experimental/ranges}{std2}/utility>

namespace @\changed{std \{ namespace experimental \{ namespace ranges}{std2}@ { inline namespace v1 {
  template <class T>
  concept @\removed{bool}@ TagSpecifier = @\impdef@;

  template <class F>
  concept @\removed{bool}@ TaggedType = @\impdef@;

  template <class Base, TagSpecifier... Tags>
    requires sizeof...(Tags) <= tuple_size<Base>::value
  struct tagged :
    Base, @\textit{TAGGET}@(tagged<Base, Tags...>, @$T_i$@, @$i$@)... { // \seebelow
    using Base::Base;
    tagged() = default;
    tagged(tagged&&) = default;
    tagged(const tagged&) = default;
    tagged &operator=(tagged&&) = default;
    tagged &operator=(const tagged&) = default;
    tagged(Base&&) noexcept(@\seebelow@)
      requires MoveConstructible<Base>;
    tagged(const Base&) noexcept(@\seebelow@)
      requires CopyConstructible<Base>;
    template <class Other>
      requires Constructible<Base, Other>
    constexpr tagged(tagged<Other, Tags...> &&that) noexcept(@\seebelow@);
    template <class Other>
      requires Constructible<Base, const Other&>
    constexpr tagged(const tagged<Other, Tags...> &that);
    template <class Other>
      requires Assignable<Base&, Other>
    constexpr tagged& operator=(tagged<Other, Tags...>&& that) noexcept(@\seebelow@);
    template <class Other>
      requires Assignable<Base&, const Other&>
    constexpr tagged& operator=(const tagged<Other, Tags...>& that);
    template <class U>
      requires Assignable<Base&, U> && !Same<decay_t<U>, tagged>
    constexpr tagged& operator=(U&& u) noexcept(@\seebelow@);
    constexpr void swap(tagged& that) noexcept(@\seebelow@)
      requires Swappable<Base>;
    friend constexpr void swap(tagged&, tagged&) noexcept(@\seebelow@)
      requires Swappable<Base>;
  };
}}@\removed{\}\}}@
\end{codeblock}

\pnum A \techterm{tagged getter} is an empty trivial class type that has a named member function that
returns a reference to a member of a tuple-like object that is assumed to be derived from the getter
class. The tuple-like type of a tagged getter is called its \techterm{DerivedCharacteristic}.
The index of the tuple element returned from the getter's member functions is called its
\techterm{ElementIndex}. The name of the getter's member function is called its
\techterm{ElementName}

\pnum A tagged getter class with DerivedCharacteristic \tcode{\textit{D}}, ElementIndex
\tcode{\textit{N}}, and ElementName \tcode{\textit{name}} shall provide the following interface:

\begin{codeblock}
struct @\xname{\textit{TAGGED_GETTER}}@ {
  constexpr decltype(auto) @$name$@() &       { return get<@$N$@>(static_cast<@$D$@&>(*this)); }
  constexpr decltype(auto) @$name$@() &&      { return get<@$N$@>(static_cast<@$D$@&&>(*this)); }
  constexpr decltype(auto) @$name$@() const & { return get<@$N$@>(static_cast<const @$D$@&>(*this)); }
};
\end{codeblock}

\pnum
A \techterm{tag specifier} is a type that facilitates a mapping from a tuple-like type and an
element index into a \textit{tagged getter} that gives named access to the element at that index.
\tcode{TagSpecifier<T>} is satisfied if and only if \tcode{T} is a tag specifier. The tag specifiers in the
\tcode{Tags} parameter pack shall be unique. \enternote The mapping mechanism from tag specifier to
tagged getter is unspecified.\exitnote

\pnum Let \tcode{\textit{TAGGET}(D, T, $N$)} name a tagged getter type that gives named
access to the $N$-th element of the tuple-like type \tcode{D}.

\pnum It shall not be possible to delete an instance of class template \tcode{tagged} through a
pointer to any base other than \tcode{Base}.

\pnum
\tcode{TaggedType<F>} is satisfied if and only if \tcode{F} is a unary function
type with return type \tcode{T} which satisfies \tcode{TagSpecifier<T>}. Let
\tcode{\textit{TAGSPEC}(F)} name the tag specifier of the \tcode{TaggedType} \tcode{F}, and let
\tcode{\textit{TAGELEM}(F)} name the argument type of the \tcode{TaggedType} \tcode{F}.

\indexlibrary{\idxcode{tagged}!\idxcode{tagged}}

\begin{itemdecl}
tagged(Base&& that) noexcept(@\seebelow@)
  requires MoveConstructible<Base>;
\end{itemdecl}

\begin{itemdescr}
\pnum
\effects Initializes \tcode{Base} with \tcode{std::move(that)}.

\pnum
\remarks The expression in the \tcode{noexcept} is equivalent to:

\begin{codeblock}
is_nothrow_move_constructible<Base>::value
\end{codeblock}
\end{itemdescr}

\begin{itemdecl}
tagged(const Base& that) noexcept(@\seebelow@)
  requires CopyConstructible<Base>;
\end{itemdecl}

\begin{itemdescr}
\pnum
\effects Initializes \tcode{Base} with \tcode{that}.

\pnum
\remarks The expression in the \tcode{noexcept} is equivalent to:

\begin{codeblock}
is_nothrow_copy_constructible<Base>::value
\end{codeblock}
\end{itemdescr}

\begin{itemdecl}
template <class Other>
  requires Constructible<Base, Other>
constexpr tagged(tagged<Other, Tags...> &&that) noexcept(@\seebelow@);
\end{itemdecl}

\begin{itemdescr}
\pnum
\effects Initializes \tcode{Base} with \tcode{static_cast<Other\&\&>(that)}.

\pnum
\remarks The expression in the \tcode{noexcept} is equivalent to:

\begin{codeblock}
is_nothrow_constructible<Base, Other>::value
\end{codeblock}
\end{itemdescr}

\indexlibrary{\idxcode{tagged}!\idxcode{tagged}}
\begin{itemdecl}
template <class Other>
  requires Constructible<Base, const Other&>
constexpr tagged(const tagged<Other, Tags...>& that);
\end{itemdecl}

\begin{itemdescr}
\pnum
\effects Initializes \tcode{Base} with \tcode{static_cast<const Other\&>(that)}.
\end{itemdescr}

\indexlibrary{\idxcode{operator=}!\idxcode{tagged}}
\indexlibrary{\idxcode{tagged}!\idxcode{operator=}}
\begin{itemdecl}
template <class Other>
  requires Assignable<Base&, Other>
constexpr tagged& operator=(tagged<Other, Tags...>&& that) noexcept(@\seebelow@);
\end{itemdecl}

\begin{itemdescr}
\pnum
\effects Assigns \tcode{static_cast<Other\&\&>(that)} to \tcode{static_cast<Base\&>(*this)}.

\pnum
\returns \tcode{*this}.

\pnum
\remarks The expression in the \tcode{noexcept} is equivalent to:

\begin{codeblock}
is_nothrow_assignable<Base&, Other>::value
\end{codeblock}
\end{itemdescr}

\indexlibrary{\idxcode{operator=}!\idxcode{tagged}}
\indexlibrary{\idxcode{tagged}!\idxcode{operator=}}
\begin{itemdecl}
template <class Other>
  requires Assignable<Base&, const Other&>
constexpr tagged& operator=(const tagged<Other, Tags...>& that);
\end{itemdecl}

\begin{itemdescr}
\pnum
\effects Assigns \tcode{static_cast<const Other\&>(that)} to \tcode{static_cast<Base\&>(*this)}.

\pnum
\returns \tcode{*this}.
\end{itemdescr}

\indexlibrary{\idxcode{operator=}!\idxcode{tagged}}
\indexlibrary{\idxcode{tagged}!\idxcode{operator=}}
\begin{itemdecl}
template <class U>
  requires Assignable<Base&, U> && !Same<decay_t<U>, tagged>
constexpr tagged& operator=(U&& u) noexcept(@\seebelow@);
\end{itemdecl}

\begin{itemdescr}
\pnum
\effects Assigns \tcode{std::forward<U>(u)} to \tcode{static_cast<Base\&>(*this)}.

\pnum
\returns \tcode{*this}.

\pnum
\remarks The expression in the \tcode{noexcept} is equivalent to:

\begin{codeblock}
is_nothrow_assignable<Base&, U>::value
\end{codeblock}
\end{itemdescr}

\indexlibrary{\idxcode{swap}!\idxcode{tagged}}
\indexlibrary{\idxcode{tagged}!\idxcode{swap}}
\begin{itemdecl}
constexpr void swap(tagged& rhs) noexcept(@\seebelow@)
  requires Swappable<Base>;
\end{itemdecl}

\begin{itemdescr}
\pnum
\effects Calls \tcode{swap} on the result of applying \tcode{static_cast} to \tcode{*this} and
\tcode{that}.

\pnum
\throws Nothing unless the call to \tcode{swap} on the \tcode{Base} sub-objects throws.

\pnum
\remarks The expression in the \tcode{noexcept} is equivalent to:

\begin{codeblock}
noexcept(swap(declval<Base&>(), declval<Base&>()))
\end{codeblock}
\end{itemdescr}

\indexlibrary{\idxcode{swap}!\tcode{tagged}}%
\begin{itemdecl}
friend constexpr void swap(tagged& lhs, tagged& rhs) noexcept(@\seebelow@)
  requires Swappable<Base>;
\end{itemdecl}

\begin{itemdescr}
\pnum
\effects Equivalent to \tcode{lhs.swap(rhs)}.

\pnum
\remarks The expression in the \tcode{noexcept} is equivalent to:

\begin{codeblock}
noexcept(lhs.swap(rhs))
\end{codeblock}
\end{itemdescr}

\rSec2[std2.tagged.astuple]{Tuple-like access to \tcode{tagged}}

\indexlibrary{\idxcode{tuple_size}}%
\indexlibrary{\idxcode{tuple_element}}%
\begin{itemdecl}
namespace std {
  template <class Base, class... Tags>
  struct tuple_size<@\changed{experimental::ranges}{::std2}@::tagged<Base, Tags...>>
    : tuple_size<Base> { };

  template <size_t N, class Base, class... Tags>
  struct tuple_element<N, @\changed{experimental::ranges}{::std2}@::tagged<Base, Tags...>>
    : tuple_element<N, Base> { };
}
\end{itemdecl}

\rSec2[std2.tagged.pairs]{Alias template \tcode{tagged_pair}}

\begin{codeblock}
// defined in header <\changed{experimental/ranges}{std2}/utility>

namespace @\changed{std \{ namespace experimental \{ namespace ranges}{std2}@ { inline namespace v1 {
  // ...
  template <TaggedType T1, TaggedType T2>
  using tagged_pair = tagged<pair<@\textit{TAGELEM}@(T1), @\textit{TAGELEM}@(T2)>,
                             @\textit{TAGSPEC}@(T1), @\textit{TAGSPEC}@(T2)>;
}}@\removed{\}\}}@

\end{codeblock}

\pnum \enterexample
\begin{codeblock}
// See \ref{std2.alg.tagspec}:
tagged_pair<tag::min(int), tag::max(int)> p{0, 1};
assert(&p.min() == &p.first);
assert(&p.max() == &p.second);
\end{codeblock}
\exitexample

\rSec3[std2.tagged.pairs.creation]{Tagged pair creation functions}

\indexlibrary{\idxcode{make_tagged_pair}}%
\begin{itemdecl}
// defined in header <\changed{experimental/ranges}{std2}/utility>

namespace @\changed{std \{ namespace experimental \{ namespace ranges}{std2}@ { inline namespace v1 {
  template <TagSpecifier Tag1, TagSpecifier Tag2, class T1, class T2>
    constexpr @\seebelow@ make_tagged_pair(T1&& x, T2&& y);
}}@\removed{\}\}}@
\end{itemdecl}

\begin{itemdescr}
\pnum
Let \tcode{P} be the type of \tcode{make_pair(std::forward<T1>(x), std::forward<T2>(y))}.
Then the return type is \tcode{tagged<P, Tag1, Tag2>}.

\pnum
\returns \tcode{\{std::forward<T1>(x), std::forward<T2>(y)\}}.

\pnum
\enterexample
In place of:

\begin{codeblock}
  return tagged_pair<tag::min(int), tag::max(double)>(5, 3.1415926);   // explicit types
\end{codeblock}

a \Cpp program may contain:

\begin{codeblock}
  return make_tagged_pair<tag::min, tag::max>(5, 3.1415926);           // types are deduced
\end{codeblock}
\exitexample
\end{itemdescr}

\rSec2[std2.tagged.tuple]{Alias template \tcode{tagged_tuple}}

\pnum
\synopsis{Header \tcode{<\changed{experimental/ranges}{std2}/tuple>} synopsis}

\begin{codeblock}
namespace @\changed{std \{ namespace experimental \{ namespace ranges}{std2}@ { inline namespace v1 {
  template <TaggedType... Types>
  using tagged_tuple = tagged<tuple<@\textit{TAGELEM}@(Types)...>,
                              @\textit{TAGSPEC}@(Types)...>;

  template <TagSpecifier... Tags, class... Types>
    requires sizeof...(Tags) == sizeof...(Types)
      constexpr @\seebelow@ make_tagged_tuple(Types&&... t);
}}@\removed{\}\}}@
\end{codeblock}

\pnum
\begin{codeblock}
template <TaggedType... Types>
using tagged_tuple = tagged<tuple<@\textit{TAGELEM}@(Types)...>,
                            @\textit{TAGSPEC}@(Types)...>;
\end{codeblock}

\pnum \enterexample
\begin{codeblock}
// See \ref{std2.alg.tagspec}:
tagged_tuple<tag::in(char*), tag::out(char*)> t{0, 0};
assert(&t.in() == &get<0>(t));
assert(&t.out() == &get<1>(t));
\end{codeblock}
\exitexample

\rSec3[std2.tagged.tuple.creation]{Tagged tuple creation functions}

\indexlibrary{\idxcode{make_tagged_tuple}}%
\indexlibrary{\idxcode{tagged_tuple}!\idxcode{make_tagged_tuple}}%
\begin{itemdecl}
template <TagSpecifier... Tags, class... Types>
  requires sizeof...(Tags) == sizeof...(Types)
    constexpr @\seebelow@ make_tagged_tuple(Types&&... t);
\end{itemdecl}

\begin{itemdescr}
\pnum
Let \tcode{T} be the type of \tcode{make_tuple(std::forward<Types>(t)...)}.
Then the return type is \tcode{tagged<\brk{}T, Tags...>}.

\pnum
\returns \tcode{tagged<T, Tags...>(std::forward<Types>(t)...)}.

\pnum
\enterexample

\begin{codeblock}
int i; float j;
make_tagged_tuple<tag::in1, tag::in2, tag::out>(1, ref(i), cref(j))
\end{codeblock}

creates a tagged tuple of type

\begin{codeblock}
tagged_tuple<tag::in1(int), tag::in2(int&), tag::out(const float&)>
\end{codeblock}
\exitexample
\end{itemdescr}
