%!TEX root = P0896.tex

\setcounter{chapter}{16}
\rSec0[concepts]{Concepts library}

\setcounter{section}{2}
\rSec1[concepts.lang]{Language-related concepts}

\setcounter{subsection}{10}
\rSec2[concept.swappable]{Concept \libconcept{Swappable}}

\ednote{Modify the definitions of the \libconcept{Swappable} and \libconcept{SwappableWith}
concepts as follows (This restores the Ranges TS design for these concepts from
which P0898 had to deviate due to the absence of the \tcode{ranges::swap}
customization point):}

\indexlibrary{\idxcode{Swappable}}%
\begin{itemdecl}
template<class T>
@\removed{concept Swappable = is_swappable_v<T>; // \seebelow}@
@\added{concept Swappable = requires(T\& a, T\& b) \{ ranges::swap(a, b); \};}@
\end{itemdecl}

{\color{remclr}
\begin{itemdescr}
\pnum
Let \tcode{a1} and \tcode{a2} denote distinct equal objects of type \tcode{T},
and let \tcode{b1} and \tcode{b2} similarly denote distinct equal objects of type
\tcode{T}. \tcode{Swappable<T>} is satisfied only if:
\begin{itemize}
\item After evaluating either \tcode{swap(a1, b1)} or \tcode{swap(b1, a1)} in the
  context described below, \tcode{a1} is equal to \tcode{b2} and \tcode{b1} is
  equal to \tcode{a2}.
\end{itemize}

\pnum
The context in which \tcode{swap(a1, b1)} or \tcode{swap(b1, a1)} are evaluated shall ensure that a binary non-member
function named "swap" is selected via overload resolution\cxxiref{over.match} on a candidate set that includes:
\begin{itemize}
\item the two \tcode{swap} function templates defined in \tcode{<utility>}\iref{utility} and
\item the lookup set produced by argument-dependent lookup\cxxiref{basic.lookup.argdep}.
\end{itemize}
\end{itemdescr}
} %% \color{remclr}

\indexlibrary{\idxcode{SwappableWith}}%
\begin{itemdecl}
template<class T, class U>
concept SwappableWith =
  @\removed{is_swappable_with_v<T, T> \&\& is_swappable_with_v<U, U> \&\& // \seebelow}@
  CommonReference<const remove_reference_t<T>&, const remove_reference_t<U>&> &&
  @\removed{is_swappable_with_v<T, U> \&\& is_swappable_with_v<U, T>; // \seebelow}@
  @\added{requires(T\&\& t, U\&\& u) \{}@
    @\added{ranges::swap(std::forward<T>(t), std::forward<T>(t));}@
    @\added{ranges::swap(std::forward<U>(u), std::forward<U>(u));}@
    @\added{ranges::swap(std::forward<T>(t), std::forward<U>(u));}@
    @\added{ranges::swap(std::forward<U>(u), std::forward<T>(t));}@
  @\added{\};}@
\end{itemdecl}

{\color{remclr}
\begin{itemdescr}
\pnum
Let \tcode{t1} and \tcode{t2} denote distinct equal objects of type
\tcode{remove_cvref_t<T>}, and $E_t$ be an expression that denotes \tcode{t1}
such that \tcode{decltype(($E_t$))} is \tcode{T}.
Let \tcode{u1} and \tcode{u2} similarly denote distinct equal objects of type
\tcode{remove_cvref_t<U>}, and $E_u$ be an expression that denotes \tcode{u1}
such that \tcode{decltype(($E_u$))} is \tcode{U}. Let \tcode{C} be
\tcode{common_reference_t<const remove_reference_t<T>\&, const remove_reference_t<U>\&>}.
\tcode{SwappableWith<T, U>} is satisfied only if:

\begin{itemize}
\item After evaluating either \tcode{swap($E_t$, $E_u$)} or \tcode{swap($E_u$, $E_t$)} in the
  context described above, \tcode{C(t1)} is equal to \tcode{C(u2)} and \tcode{C(u1)} is
  equal to \tcode{C(t2)}.
\end{itemize}

\pnum
The context in which \tcode{swap($E_t$, $E_u$)} or \tcode{swap($E_u$, $E_t$)} are evaluated shall ensure that a binary non-member
function named "swap" is selected via overload resolution\cxxiref{over.match} on a candidate set that includes:
\begin{itemize}
\item the two \tcode{swap} function templates defined in \tcode{<utility>}\iref{utility} and
\item the lookup set produced by argument-dependent lookup\cxxiref{basic.lookup.argdep}.
\end{itemize}
\end{itemdescr}
} %% \color{remclr}

{\color{addclr}
\begin{itemdescr}
\pnum
This subclause provides definitions for swappable types and expressions. In these
definitions, let \tcode{t} denote an expression of type \tcode{T}, and let \tcode{u}
denote an expression of type \tcode{U}.

\pnum
An object \tcode{t} is \defn{swappable with} an object \tcode{u} if and only if
\tcode{SwappableWith<T, U>} is satisfied. \tcode{Swappable\-With<T, U>} is satisfied
only if given distinct objects \tcode{t2} equal to \tcode{t}
and \tcode{u2} equal to \tcode{u}, after evaluating either
\tcode{ranges::swap(t, u)} or \tcode{ranges::swap(u, t)}, \tcode{t2} is equal to
\tcode{u} and \tcode{u2} is equal to \tcode{t}.

\pnum
An rvalue or lvalue \tcode{t} is \defn{swappable} if and only if \tcode{t} is
swappable with any rvalue or lvalue, respectively, of type \tcode{T}.

\begin{example}
User code can ensure that the evaluation of \tcode{swap} calls
is performed in an appropriate context under the various conditions as follows:
\begin{codeblock}
#include <concepts>
#include <utility>

template<class T, @\oldtxt{class}\newtxt{std::SwappableWith<T>}@ U>
void value_swap(T&& t, U&& u) {
  ranges::swap(std::forward<T>(t), std::forward<U>(u)); // OK: uses ``swappable with'' conditions
                                                        // for rvalues and lvalues
}

// Requires: lvalues of \tcode{T} shall be swappable.
template<@\oldtxt{class}\newtxt{std::Swappable}@ T>
void lv_swap(T& t1, T& t2) {
  ranges::swap(t1, t2);                                 // OK: uses swappable conditions for
}                                                       // lvalues of type \tcode{T}

namespace N {
  struct A { int m; };
  struct Proxy { A* a; };
  Proxy proxy(A& a) { return Proxy{ &a }; }

  void swap(A& x, Proxy p) {
    ranges::swap(x.m, p.a->m);                // OK: uses context equivalent to swappable
                                              // conditions for fundamental types
  }
  void swap(Proxy p, A& x) { swap(x, p); }  // satisfy symmetry constraint
}

int main() {
  int i = 1, j = 2;
  lv_swap(i, j);
  assert(i == 2 && j == 1);

  N::A a1 = { 5 }, a2 = { -5 };
  value_swap(a1, proxy(a2));
  assert(a1.m == -5 && a2.m == 5);
}
\end{codeblock}
\end{example}
\end{itemdescr}
} %% \color{addclr}
