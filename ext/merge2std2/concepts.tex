%!TEX root = P0896.tex

\setcounter{chapter}{16}
\rSec0[concepts]{Concepts library}

\setcounter{section}{2}
\rSec1[concepts.syn]{Header \tcode{<concepts>} synopsis}

\indexlibrary{\idxhdr{concepts}}%
\begin{codeblock}
namespace std {
  [...]

  // \cxxref{concept.assignable}, concept \libconcept{Assignable}
  template<class LHS, class RHS>
    concept Assignable = @\seebelow@;

  // \ref{concept.swappable}, concept \libconcept{Swappable}
\end{codeblock}
\begin{addedblock}
\begin{codeblock}
  namespace ranges {
    inline namespace @\unspec@ {
      inline constexpr @\unspec@ swap = @\unspecnc@;
    }
  }
\end{codeblock}
\end{addedblock}
\begin{codeblock}
  template<class T>
    concept Swappable = @\seebelow@;
  template<class T, class U>
    concept SwappableWith = @\seebelow@;

  [...]
}
\end{codeblock}

\rSec1[concepts.lang]{Language-related concepts}

\setcounter{subsection}{10}
\rSec2[concept.swappable]{Concept \libconcept{Swappable}}

\begin{addedblock}
\pnum
Let \tcode{t1} and \tcode{t2} be equality-preserving expressions that denote
distinct equal objects of type \tcode{T}, and let \tcode{u1} and \tcode{u2}
similarly denote distinct equal objects of type \tcode{U}. An operation
\term{exchanges the values} denoted by \tcode{t1} and \tcode{u1} if and only
if the operation modifies neither \tcode{t2} nor \tcode{u2} and:
\begin{itemize}
\item If \tcode{T} and \tcode{U} are the same type, the result of the operation
  is that \tcode{t1} equals \tcode{u2} and \tcode{u1} equals \tcode{t2}.

\item If \tcode{T} and \tcode{U} are different types that model
  \libconcept{CommonReference<const T\&, const U\&>},
  the result of the operation is that
  \tcode{C(t1)} equals \tcode{C(u2)}
  and
  \tcode{C(u1)} equals \tcode{C(t2)}
  where \tcode{C} is \tcode{common_reference_t<const T\&, const U\&>}.
\end{itemize}

\indexlibrary{\idxcode{ranges::swap}}%
\pnum The name \tcode{ranges::swap} denotes a customization point
object\cxxiref{customization.point.object}. The expression
\tcode{ranges::swap(E1, E2)} for some subexpressions \tcode{E1}
and \tcode{E2} is expression-equivalent to:

\begin{itemize}
\item
  \tcode{(void)swap(E1, E2)}\footnote{The name \tcode{swap} is used
  here unqualified.}, if \tcode{E1} or \tcode{E2}
  has class type\cxxiref{basic.compound} and that expression is valid, with
  overload resolution performed in a context that includes the declarations
\begin{codeblock}
  template<class T>
    void swap(T&, T&) = delete;
  template<class T, size_t N>
    void swap(T(&)[N], T(&)[N]) = delete;
\end{codeblock}
  and does not include a declaration of \tcode{ranges::swap}.
  If the function selected by overload resolution does not
  exchange the values denoted by
  \tcode{E1} and \tcode{E2},
  the program is ill-formed with no diagnostic required.

\item
  Otherwise, \tcode{(void)ranges::swap_ranges(E1, E2)} if \tcode{E1} and
  \tcode{E2} are lvalues of array types\cxxiref{basic.compound}
  with equal extent and \tcode{ranges::swap(*E1, *E2)}
  is a valid expression, except that
  \tcode{noexcept(\brk{}ranges::swap(E1, E2))} is equal to
  \tcode{noexcept(\brk{}ranges::swap(*E1, *E2))}.

\item
  Otherwise, if \tcode{E1} and \tcode{E2} are lvalues of the
  same type \tcode{T} that models \libconcept{MoveConstructible<T>} and
  \libconcept{Assignable<T\&, T>}, exchanges the denoted values.
  \tcode{ranges::swap(\brk{}E1, E2)} is a constant expression if
  \begin{itemize}
  \item \tcode{T} is a literal type\cxxiref{basic.types},
  \item both \tcode{E1 = std::move(E2)} and \tcode{E2 = std::move(E1)} are
    constant subexpressions\cxxiref{defns.const.subexpr}, and
  \item the full-expressions of the initializers in the declarations
    \begin{codeblock}
    T t1(std::move(E1));
    T t2(std::move(E2));
    \end{codeblock}
    are constant subexpressions.
  \end{itemize}
  \tcode{noexcept(ranges::swap(E1, E2))} is equal to
  \tcode{is_nothrow_move_constructible_v<T> \&\&
  is_nothrow_move_assignable_v<T>}.

\item
  Otherwise, \tcode{ranges::swap(E1, E2)} is ill-formed.
  \begin{note}
  This case can result in substitution failure when \tcode{ranges::swap(E1, E2)}
  appears in the immediate context of a template instantiation.
  \end{note}
\end{itemize}

\pnum
\begin{note}
Whenever \tcode{ranges::swap(E1, E2)} is a valid expression, it
exchanges the values denoted by
\tcode{E1} and \tcode{E2} and has type \tcode{void}.
\end{note}
\end{addedblock}

\indexlibrary{\idxcode{Swappable}}%
\begin{itemdecl}
template<class T>
  @\removed{concept Swappable = is_swappable_v<T>;}@
  @\added{concept Swappable = requires(T\& a, T\& b) \{ ranges::swap(a, b); \};}@
\end{itemdecl}

\begin{removedblock}
\begin{itemdescr}
\pnum
Let \tcode{a1} and \tcode{a2} denote distinct equal objects of type \tcode{T},
and let \tcode{b1} and \tcode{b2} similarly denote distinct equal objects of
type \tcode{T}. \tcode{\libconcept{Swappable}<T>} is satisfied only if after
evaluating either \tcode{swap(a1, b1)} or \tcode{swap(b1, a1)} in the context
described below, \tcode{a1} is equal to \tcode{b2} and \tcode{b1} is equal to
\tcode{a2}.

\pnum
The context in which \tcode{swap(a1, b1)} or \tcode{swap(b1, a1)} is evaluated
shall ensure that a binary non-member function named \tcode{swap} is selected via
overload resolution\cxxiref{over.match} on a candidate set that includes:
\begin{itemize}
\item the two \tcode{swap} function templates defined in
  \tcode{<utility>}\cxxiref{utility} and
\item the lookup set produced by argument-dependent
  lookup\cxxiref{basic.lookup.argdep}.
\end{itemize}
\end{itemdescr}
\end{removedblock}

\indexlibrary{\idxcode{SwappableWith}}%
\begin{itemdecl}
template<class T, class U>
  concept SwappableWith =
    @\removed{is_swappable_with_v<T, T> \&\& is_swappable_with_v<U, U> \&\&}@
    CommonReference<const remove_reference_t<T>&, const remove_reference_t<U>&> &&
    @\removed{is_swappable_with_v<T, U> \&\& is_swappable_with_v<U, T>;}@
    @\added{requires(T\&\& t, U\&\& u) \{}@
      @\added{ranges::swap(std::forward<T>(t), std::forward<T>(t));}@
      @\added{ranges::swap(std::forward<U>(u), std::forward<U>(u));}@
      @\added{ranges::swap(std::forward<T>(t), std::forward<U>(u));}@
      @\added{ranges::swap(std::forward<U>(u), std::forward<T>(t));}@
    @\added{\};}@
\end{itemdecl}

\begin{removedblock}
\begin{itemdescr}
\pnum
Let:
\begin{itemize}
\item \tcode{t1} and \tcode{t2} denote distinct equal objects of type
  \tcode{remove_cvref_t<T>},
\item $E_t$ be an expression that denotes \tcode{t1} such that
  \tcode{decltype(($E_t$))} is \tcode{T},
\item \tcode{u1} and \tcode{u2} similarly denote distinct equal objects of type
  \tcode{remove_cvref_t<U>},
\item $E_u$ be an expression that denotes \tcode{u1} such that
  \tcode{decltype(($E_u$))} is \tcode{U}, and
\item \tcode{C} be
  % This comfortably fits on one line as a codeblock, but overfills the page as
  % tcode. :(
  \begin{codeblock}
    common_reference_t<const remove_reference_t<T>&, const remove_reference_t<U>&>
  \end{codeblock}
\end{itemize}
\tcode{\libconcept{SwappableWith}<T, U>} is satisfied only if after evaluating
either \tcode{swap($E_t$, $E_u$)} or \tcode{swap($E_u$, $E_t$)} in the context
described above, \tcode{C(t1)} is equal to \tcode{C(u2)} and \tcode{C(u1)} is
equal to \tcode{C(t2)}.

\pnum
The context in which \tcode{swap($E_t$, $E_u$)} or \tcode{swap($E_u$, $E_t$)}
is evaluated shall ensure that a binary non-member function named \tcode{swap} is
selected via overload resolution\cxxiref{over.match} on a candidate set that
includes:
\begin{itemize}
\item the two \tcode{swap} function templates defined in
  \tcode{<utility>}\cxxiref{utility} and
\item the lookup set produced by argument-dependent
  lookup\cxxiref{basic.lookup.argdep}.
\end{itemize}
\end{itemdescr}
\end{removedblock}

\begin{addedblock}
\pnum
\begin{note}
The semantics of the \libconcept{Swappable} and \libconcept{SwappableWith}
concepts are fully defined by the \tcode{ranges::swap} customization point.
\end{note}
\end{addedblock}

\pnum
\begin{example}
User code can ensure that the evaluation of \tcode{swap} calls
is performed in an appropriate context under the various conditions as follows:
\begin{codeblock}
#include <cassert>
#include <concepts>
#include <utility>

@\added{namespace ranges = std::ranges;}@

template<class T, std::SwappableWith<T> U>
void value_swap(T&& t, U&& u) {
  @\removed{using std::swap;}@
  @\added{ranges::}@swap(std::forward<T>(t), std::forward<U>(u)); @\removed{// OK: uses ``swappable with'' conditions}@
                                                        @\removed{// for rvalues and lvalues}@
}

template<std::Swappable T>
void lv_swap(T& t1, T& t2) {
  @\removed{using std::swap;}@
  @\added{ranges::}@swap(t1, t2);                                 @\removed{// OK: uses swappable conditions for}@
}                                                       @\removed{// lvalues of type \tcode{T}}@

namespace N {
  struct A { int m; };
  struct Proxy { A* a; };
  Proxy proxy(A& a) { return Proxy{ &a }; }

  void swap(A& x, Proxy p) {
    @\changed{std}{ranges}@::swap(x.m, p.a->m);                       @\removed{// OK: uses context equivalent to swappable}@
                                                        @\removed{// conditions for fundamental types}@
  }
  void swap(Proxy p, A& x) { swap(x, p); }              // satisfy symmetry \changed{constraint}{requirement}
}

int main() {
  int i = 1, j = 2;
  lv_swap(i, j);
  assert(i == 2 && j == 1);

  N::A a1 = { 5 }, a2 = { -5 };
  value_swap(a1, proxy(a2));
  assert(a1.m == -5 && a2.m == 5);
}
\end{codeblock}
\end{example}
