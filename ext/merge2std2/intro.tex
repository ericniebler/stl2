\rSec0[intro.scope]{Scope}

\begin{quote}
``Eventually, all things merge into one, and a river runs through it.''
\begin{flushright}
\textemdash \textit{Norman Maclean}
\end{flushright}
\end{quote}

\pnum
This document proposes to merge the ISO/IEC TS 21425:2017, aka the Ranges TS,
into the working draft.

\rSec1[intro.history]{Revision History}
\rSec2[intro.history.r4]{Revision 4}
\begin{itemize}
\item (Throughout) Audit \grammarterm{requires-clause}s for
  non-primary-expressions. For example, \tcode{requires !C<T>} was a valid
  \grammarterm{requires-clause} per the Concepts TS, but C++20 requires
  (pun intended) \tcode{requires (!C<T>)}.
\item (Throughout) Order iterator trait aliases consistently.
\item (Throughout) Changed all \tcode{T foo \{\};} member declarations to
  \tcode{T foo = T();} to avoid \href{https://wg21.link/LWG3149}{LWG 3149}.
\item \ref{support.limits.general} Add \tcode{__cpp_lib_ranges}
\item \ref{special.mem.concepts} require nothrow SMFs for no-throw-sentinel;
  qualify references to \tcode{ranges::begin} and \tcode{::end}
\item \ref{iterators.general} Strike subclauses from Table 73
\item \ref{iterator.requirements.general} Several editorial clarifications from
  LWG review.
\item \ref{incrementable.traits} Replace use of \tcode{decay_t} with
  \tcode{remove_const_t}.
\item \ref{incrementable.traits} \& \ref{readable.traits} Strike noise paragraph
  ``XXX is implemented as if:''.
\item \ref{iterator.traits} Fix broken wording in P1. Rephrase p2 without
  ``non-program-defined''. Strike unneeded second sentence of P4
  (There's no reason to require that program-defined \tcode{iterator_concept}
  has any relation to a \tcode{std} type, or that it has special member
  functions needed to implement tag dispatch). Rename \tcode{BidirectionalIterator}
  to \tcode{BI} in example to avoid confusion with the concept of the same name.
  Revert all ``program-defined specialization of \tcode{iterator_traits}'' wording
  to ``instantiation of the primary template'' wording.
  \tcode{\placeholder{cpp17-forward-iterator}} must require an lvalue reference
  type.
\item \ref{iterator.custpoints.iter_move} Don't italicize ``customization point
  object'', we're not defining it here
  (or in \ref{iterator.custpoints.iter_swap}). Rephrase in the style of
  \tcode{ranges::swap}\iref{concept.swappable} (and in
  \ref{iterator.custpoints.iter_swap}).
\item \ref{iterator.custpoints.iter_swap} Much rephrasing for clarification.
\item \ref{iterator.concepts.general} Strike p1. Clarify p2.
  Add a note explaining \tcode{ITER_TRAITS}.
\item \ref{iterator.concept.writable} Add a note explaining the weird
  \tcode{const_cast} expressions.
\item \ref{iterator.concept.sizedsentinel} editorial clarification;
  missing semicolon
\item \ref{iterator.concept.forward} Add a void cast to avoid (pun intended) ADL
  comma hijacking
\item \ref{iterator.concept.random.access} Clarify p1; strike note and add a
  \tcode{// not defined} comment in the class definition. Harmonize wording with
  \tcode{\placeholder{Advanceable}}\iref{range.iota.view}.
\item \ref{projected} Clarify p1
\item \ref{commonalgoreq.general} Clarify note
\item \ref{range.iterator.operations.advance} Rephrase all the things.
\item \ref{range.iterator.operations.distance} Rephrase p1.
\item \ref{reverse.iter.requirements} Clarify p2
\item \ref{reverse.iter.cmp} Explode p1 into \constraints elements.
\item \ref{reverse.iter.nonmember} Expand the effects for \tcode{iter_move} and
  \tcode{iter_swap} to the intended implementation to avoid confusion over the
  conditional \tcode{noexcept} specifications. Simplify said \tcode{noexcept}s
  by using an lvalue \grammarterm{id-expression} instead of \tcode{declval}.
\item \ref{move.iterator} and \ref{move.iter.nav} Change return type of
  \tcode{operator++} from \tcode{decltype(auto)} to \tcode{auto}
\item \ref{move.iter.op.comp} Explode p1 into \constraints elements
\item \ref{move.iter.nonmember} Explode p1 into \constraints elements
\item \ref{move.sentinel}, \ref{move.sent.ops}, \ref{common.iterator},
  \ref{common.iter.const}, \ref{counted.iterator},
  and \ref{counted.iter.const} Fix misuses of \tcode{ConvertibleTo}.
\item \ref{iterators.common} Push hanging paragraphs down
  into \ref{common.iterator}.
\item \ref{common.iterator} Coalesce tiny subclauses. Strike unnecessary
  specialization of \tcode{readable_traits} (given that \tcode{iterator_traits}
  is specialized). Replace nested \tcode{difference_type} with a specialization
  of \tcode{incrementable_traits}.
\item \ref{common.iter.access} Move conditional \tcode{noexcept} paragraph
  immediately after declaration of \tcode{operator->}.
  Mark \tcode{count()} as \tcode{noexcept}.
\item \ref{common.iter.cmp} Replace \tcode{I1} and \tcode{S1} with
  \tcode{I} and \tcode{S}.
\item \ref{common.iter.cust} Get \tcode{I2} from \tcode{y}, not \tcode{I}.
\item \ref{default.sentinels} Merge [default.sent].
\item \ref{iterators.counted} Rearrange subclauses / titles / stable names
  in the style of the working draft. Merge all \returns elements into \effects.
\item \ref{counted.iterator} A counted iterator knows the distance to the
  \textit{end} of its range. Strike ``possibly differing''.
  Strike unnecessary specialization of \tcode{readable_traits}
  (given that \tcode{iterator_traits} is specialized).
  Rename \tcode{cnt} to \tcode{length}. Replace nested \tcode{difference_type}
  with a specialization of \tcode{incrementable_traits}.
  Add default member initializers and \tcode{default} the default constructor.
\item \ref{counted.iter.const} Merge [counted.iter.op.conv].
\item \ref{counted.iter.nav} Strike unnecessary \expects element from
  \tcode{operator++(int) requires ForwardIterator<I>}, \tcode{operator+}, and
  \tcode{operator-}.
\item \ref{unreachable.sentinel} Strike "is a placeholder type that".
  Strike p1s2.
\item \ref{range} Consistently name
  \libconcept{View} template parameters \tcode{V},
  and rename \libconcept{Range} template parameters from \tcode{O} to \tcode{R}.
\item \ref{range.general} Import [thread.decaycopy], add \tcode{noexcept} and
  \tcode{constexpr}, and reference it for uses of \tcode{DECAY_COPY} in [range].
\item \ref{ranges.syn} Change stable name from "range.syn" to "ranges.syn" to
  reflect the header name \tcode{<ranges>}.
  Rearrange declarations to clarify namespaces.
  Add textual descriptions to "link" comments.
\item \ref{range.access.end} Clarify exactly which types model \tcode{Sentinel}
  in the note. (Also in \ref{range.access.cend}, \ref{range.access.rend},
  and \ref{range.access.crend}.)
\item \ref{range.primitives.size} Change \tcode{remove_cvref_t} to
  \tcode{remove_cv_t} since the type of an expression is never a reference type.
  Extract \tcode{disable_sized_range} from 1.2 and 1.3 and turn them into
  sub-bullets. Simplify 1.3 by naming the expression.
\item \ref{range.primitives.empty} Add \tcode{()} around expression to clarify.
   Simplify 1.3 by naming the expression.
\item \ref{range.primitives.data}
  Remove "return \tcode{begin()} if it's a pointer" behavior.
\item \ref{range.requirements.general} Heavy rephrasing.
\item \ref{range.range} Much rephrasing for clarity.
  \tcode{begin} is \textit{sometimes} required to be equality-preserving.
  A note is insufficient to cause \tcode{begin} and \tcode{end}
  to not generate implicit expression variations.
  \tcode{\placeholder{forwarding-range}} is missing the most important
  semantic requirement! It also needs some explication.
\item \ref{range.sized} Strike extraneous deduction constraint; rephrase.
\item \ref{range.refinements} Merge tiny range subclauses into one.
\item \ref{range.utility.helpers} Relocate [range.adaptors.helpers] here
  under [range.utility].
\item \ref{range.view_interface} Require that the template parameter \tcode{D}
  is cv-unqualified. Apply post-completion requirements to \tcode{D}.
  Strike conversion operator in expectation that container constructors are a
  superior solution.
  Merge all function specifications into the class synopsis.
\item \ref{range.subrange} Reorganize and retitle subclauses.
  Clarify namespaces in class synopsis.
  Cleanup exposition-only concept definitions.
\item \ref{range.subrange.ctor} Use constructor delegation to save some
  paragraphs.
\item \ref{range.semi.wrap} Change stable name from
  [range.adaptor.semiregular_wrapper]; editorial rephrasing.
\item \ref{range.all} Change stable name from [range.adaptor.all].
\item \ref{range.view.ref} Strike excessive conditional \tcode{noexcept}.
  Merge [range.view.ref.ops] into [range.view.ref].
  Apply the LWG 2993 P/R to \tcode{\placeholder{ref-view}}'s constructor, so
  \tcode{\placeholder{ref-view}<const R>} isn't constructible from rvalues.
  Inline silly one-liner "\effects Equivalent to" members into the class.
  Non-member \tcode{begin} and \tcode{end} take their argument by-value.
\item \ref{range.filter} Change stable names, combine subclauses.
  Merge \tcode{end} overloads with deduced return type,
  and specify directly in the class body.
  Add missing precondition to \tcode{begin}.
\item \ref{range.filter.iterator}
  Changing the value of an element in a filter view is ok, changing it to
  something the predicate would reject is UB.
  Negate the result of the predicate in \tcode{operator--}.
\item \ref{range.transform} Rearrange subclauses. Inline effects of functions
  with deduced return types into class bodies.
\item \ref{range.transform.view}
  Constraints should use \tcode{RegularInvocable} instead of \tcode{Invocable}
  (unlike the algorithm, the view can visit each element multiple times).
\item \ref{range.transform.iterator} Fix definition of \tcode{Base}.
\item \ref{range.iota} Use \tcode{W} as the name of \tcode{WeaklyIncrementable}
  template parameters.
\item \ref{range.iota.view} Harmonize wording for
  \tcode{\placeholder{Advanceable}} with \libconcept{RandomAccessIterator}.
  Inline \tcode{size} into the class body since it has a deduced return type.
\item \ref{range.iota.iterator} Use \tcode{i.value_} consistently in friends.
\item \ref{range.take.overview} Change "the first N elements" to
  "at most N elements".
\item \ref{range.join.iterator} Pass an lvalue to
  \tcode{ranges::end} even when \tcode{\placeholder{inner-range}} is an xvalue.
\item \ref{range.single} Remove extraneous \tcode{single_view}
  deduction guide.
\item \ref{range.split.view} Ensure that \tcode{remove_reference_t<R>::size()}
  is a constant expression before evaluating it in
  \tcode{\placeholder{tiny-range}}.
\item \ref{range.split.inner} Rename \tcode{zero_} to \tcode{incremented_}.
\end{itemize}

\rSec2[intro.history.r3]{Revision 3}
\begin{itemize}
\item (Throughout) Use \tcode{R} (\tcode{r}) instead of \tcode{Rng} (\tcode{rng}) for
  \tcode{Range} template (function) parameter names.
\item (Throughout) In addition to changing the cross references, replace
  ``is a contiguous iterator'' in container requirements with ``models
  \libconcept{ContiguousIterator}''.
\item \ref{defns.projection} Use \tcode{string_view} instead of \tcode{char*}
  in example.
\item \ref{allocator.requirements} Rephrase final sentence
\item \ref{concept.swappable} Define semantics of the swap concepts completely in
  the \tcode{ranges::swap} customization point, which moves here from \cxxref{utility}
\item \ref{iterator.synopsis} Replace bogus \tcode{-> auto\&\&}
  deduction constraints with a legit requirement that the expression has
  a referenceable type.
\item \ref{iterator.custpoints.iter_swap} Vastly simplify conditional
  \tcode{noexcept} for the exposition-only \placeholder{iter-exchange-move}.
\item \ref{range.subrange} Merge \tcode{subrange} deduction guides with identical
  parameter-declaration-clauses that otherwise conflict per
  \cxxref{temp.deduct.guide}/3.
\item \ref{algorithm.syn} Simplify algorithm declarations with the form:
\begin{codeblock}
template <ForwardIterator I, [...]>
  requires Permutable<I> && [...]
[...]
\end{codeblock}
to the equivalent (since \tcode{Permutable} subsumes \tcode{ForwardIterator}):
\begin{codeblock}
template <Permutable I, [...]>
  requires [...]
[...]
\end{codeblock}
which favors smaller declarations over consistency in form between
iterator-sentinel and range overloads of algorithms that require
\tcode{Permutable}.
\end{itemize}

\rSec2[intro.history.r2]{Revision 2}
\begin{itemize}
\item Merge P0789R3.

\item Merge P1033R1.

\item Reformulate non-member operators of \tcode{common_iterator} and
  \tcode{counted_iterator} as members or hidden friends.

\item Merge P1037R0.

\item Merge P0970R1: Drop \tcode{dangling} per LEWG request, and make calls that
  would have returned a \tcode{dangling} iterator ill-formed instead by
  redefining \tcode{safe_iterator_t<R>} to be ill-formed when iterators from
  \tcode{R} may dangle.

\item Merge P0944R0.

\item Drop \tcode{tagged} and related machinery. Algorithm \tcode{foo} that did
  return a \tcode{tagged} \tcode{tuple} or \tcode{pair} now instead returns a
  named type \tcode{foo_result} with public data members whose names are the
  same as the previous set of tag names. Exceptions:
  \begin{itemize}
  \item The single-range \tcode{transform} overload returns
    \tcode{unary_transform_result}.
  \item The dual-range \tcode{transform} overload returns
    \tcode{binary_transform_result}.
  \end{itemize}

\item LEWG was disturbed by the use of \tcode{enable_if} to define
  \tcode{difference_type} and \tcode{value_type}; use \tcode{requires} clauses
  instead.

\item Per LEWG direction, rename header \tcode{<range>} to \tcode{<ranges>} to
  agree with the namespace name \tcode{std::ranges}.

\item Remove inappropriate usage of \tcode{value_type_t} in the insert iterators:
  design intent of \tcode{value_type_t} is to be only an associated type trait
  for \tcode{Readable}, and the \tcode{Container} type parameter of the insert
  iterators is not \tcode{Readable}.

\item Use \tcode{remove_cvref_t} where appropriate.

\item Restore the design intent that neither \tcode{Writable} types nor
  non-\tcode{Readable} \tcode{Iterator} types are required to have an
  equality-preserving \tcode{*} operator.

\item Require semantics for the \libconcept{Constructible} requirements of
  \libconcept{IndirectlyMovableStorable} and
  \libconcept{Indirect\-lyCopyableStorable}.

\item Declare \tcode{constexpr} the algorithms that are so declared in the
  working draft.

\item \tcode{constexpr} some \tcode{move_sentinel} members that we apparently
  missed in P0579.
\end{itemize}

\rSec2[intro.history.r1]{Revision 1}
\begin{itemize}
\item Remove section [std2.numerics] which is incorporated into P0898.

\item Do not propose \tcode{ranges::exchange}: it is not used in the Ranges TS.

\item Rewrite nearly everything to merge into \tcode{std::ranges}\footnote{\tcode{std::two}
was another popular suggestion.} rather than into \tcode{std2}:
  \begin{itemize}
  \item Occurrences of ``std2.'' in stable names are either removed, or replaced with
    ``range'' when the name resulting from removal would conflict with an existing
    stable name.
  \end{itemize}

\item Incorporate the \tcode{std2::swap} customization point from P0898R0 as
\tcode{ranges::swap}. (This was necessarily dropped from P0898R1.) Perform the
necessary surgery on the \libconcept{Swappable} concept from P0898R1 to restore the
intended design that uses the renamed customization point.

\end{itemize}

\rSec0[intro]{General Principles}

\rSec1[intro.goals]{Goals}

\pnum
The primary goal of this proposal is to deliver high-quality, constrained generic
Standard Library components at the same time that the language gets support for
such components.

\rSec1[intro.rationale]{Rationale}

\pnum
The best, and arguably only practical way to achieve the goal stated above is by
incorporating the Ranges TS into the working paper. The sooner we can agree on
what we want ``\tcode{Iterator}'' and ``\libconcept{Range}'' to mean going forward
(for instance), and the sooner users are able to rely on them, the sooner we can
start building and delivering functionality on top of those fundamental
abstractions. (For example, see ``P0789: Range Adaptors and
Utilities''~(\cite{P0789}).)

\pnum
The cost of not delivering such a set of Standard Library concepts
and algorithms is that users will either do without or create a babel of mutually
incompatible concepts and algorithms, often without the rigour followed by the
Ranges TS. The experience of the authors and implementors of the Ranges TS is that
getting concept definitions and algorithm constraints right is \textit{hard}. The
Standard Library should save its users from needless heartache.

\rSec1[intro.risks]{Risks}

\pnum
Shipping constrained components from the Ranges TS in the C++20 timeframe is not
without risk. As of the time of writing (February 1, 2018), no major Standard
Library vendor has shipped an implementation of the Ranges TS. Two of the three
major compiler vendors have not even shipped an implementation of the concepts
language feature. Arguably, we have not yet gotten the usage experience for which
all Technical Specifications are intended.

\pnum
On the other hand, the components of Ranges TS have been vetted very thoroughly
by the range-v3~(\cite{range-v3}) project, on which the Ranges TS is based. There is
no part of the Ranges TS -- concepts included -- that has not seen extensive use
via range-v3. (The concepts in range-v3 are emulated with high fidelity through
the use of generalized SFINAE for expressions.) As an Open Source project, usage
statistics are hard to come by, but the following may be indicitive:

\begin{itemize}
\item The range-v3 GitHub project has over 1,400 stars, over 120 watchers, and
145 forks.
\item It is cloned on average about 6,000 times a month.
\item A GitHub search, restricted to C++ files, for the string
``\tcode{range/v3}'' (a path component of all of range-v3's header files), turns
up over 7,000 hits.
\end{itemize}

\pnum
Lacking true concepts, range-v3 cannot emulate concept-based function
overloading, or the sorts of constraints-checking short-circuit evaluation
required by true concepts. For that reason, the authors of the Ranges TS have
created a reference implementation: CMCSTL2~(\cite{cmcstl2}) using true concepts. To
this reference implementation, the authors ported all of range-v3's tests. These
exposed only a handful of concepts-specific bugs in the components of the Ranges
TS (and a great many more bugs in compilers). Those improvements were back-ported
to range-v3 where they have been thoroughly vetted over the past 2 years.

\pnum
In short, concern about lack of implementation experience should not be a reason
to withhold this important Standard Library advance from users.

\rSec1[intro.methedology]{Methodology} %FIXME

\pnum
The contents of the Ranges TS, Clause 7 (``Concepts library'') are proposed
for namespace \tcode{std} by P0898, ``Standard Library Concepts''~(\cite{P0898}).
Additionally, P0898 proposes the \tcode{identity} function
object~(\tsref{func.identity}) and the \tcode{common_reference} type
trait~(\tsref{meta.trans.other}) for namespace \tcode{std}. The changes proposed by
the Ranges TS to \tcode{common_type} are merged into the working paper (also
by P0898). The ``\tcode{invoke}'' function and the ``\tcode{swappable}'' type
traits (e.g., \tcode{is_swappable_with}) already exist in the text of the
working paper, so they are omitted here.

\pnum
The salient, high-level features of this proposal are as follows:

\begin{itemize}
\item The remaining library components in the Ranges TS are proposed for
namespace \tcode{::std::ranges}.

\item The text of the Ranges TS is rebased on the latest working draft.

\item Structurally, this paper proposes to specify each piece of \tcode{std::ranges}
alongside the content of \tcode{std} from the same header. Since some Ranges
TS components reuse names that previously had meaning in the C++ Standard, we
sometimes rename old content to avoid name collisions.

\item The content of headers from the Ranges TS with the same base name as a
standard header are merged into that standard header. For example, the content
of \tcode{<experimental/ranges/iterator>} will be merged into \tcode{<iterator>}.
The new header \tcode{<experimental/ranges/range>} will be added under the name
\tcode{<ranges>}.
\end{itemize}

\rSec1[intro.style]{Style of presentation}

\pnum
The remainder of this document is a technical specification in the form of
editorial instructions directing that changes be made to the text of the C++
working draft. The formatting of the text suggests the origin of each portion of
the wording.

Existing wording from the C++ working draft - included to provide context - is
presented without decoration.

\begin{addedblock}
Entire clauses / subclauses / paragraphs incorporated from the Ranges TS are
presented in a distinct cyan color.
\end{addedblock}

\added{In-line additions of wording from the Ranges TS to the C++ working draft
are presented in cyan with underline.}

\removed{In-line bits of wording that the Ranges TS strikes from the C++ working
draft are presented in red with strike-through.}

\newtxt{Wording to be added which is original to this document appears in gold
with underline.}

\oldtxt{Wording which this document strikes is presented in magenta with
strikethrough. (Hopefully context makes it clear whether the wording is currently
in the C++ working draft, or wording that is not being added from the Ranges TS.)}

Ideally, these formatting conventions make it clear which wording comes from
which document in this three-way merge.
