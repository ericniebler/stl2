\rSec0[intro.scope]{Scope}

\begin{quote}
``Eventually, all things merge into one, and a river runs through it.''
\begin{flushright}
\textemdash \textit{Norman Maclean}
\end{flushright}
\end{quote}

\pnum
This document proposes to merge the ISO/IEC TS 21425:2017, aka the Ranges TS,
into the working draft. This document is intended to be taken in conjunction with
P0898, a paper which proposes importing the definitions of the Ranges TS's Concepts
library (Clause 7) into namespace \tcode{std}.

\rSec0[intro.refs]{Normative References}

\pnum
The following referenced documents are indispensable for the
application of this document. For dated references, only the
edition cited applies. For undated references, the latest edition
of the referenced document (including any amendments) applies.

\begin{itemize}
\item ISO/IEC 14882, \doccite{Programming Languages - \Cpp}
\item ISO/IEC TS 21425:2017, \doccite{Technical Specification - \Cpp Extensions for Ranges}
\end{itemize}

ISO/IEC 14882 is herein called the \defn{C\Rplus\Rplus\xspace Standard} and
ISO/IEC TS 21425:2017 is called the \defn{Ranges TS}.

\rSec0[intro.defs]{Terms and Definitions}

\ednote{The following definitions are hereby proposed for subclause [definitions]
of ISO/IEC 14882.}

For the purposes of this document, the terms and definitions given in
ISO/IEC 14882 and the following apply.

ISO and IEC maintain terminological databases for use in standardization at
the following addresses:
\begin{itemize}
\item
ISO Online browsing platform: available at http://www.iso.org/obp
\item
IEC Electropedia: available at http://www.electropedia.org/
\end{itemize}

\def\definition{\definitionx{\section}}%

\indexdefn{projection}%
\definition{projection}{std2.defns.projection}
\defncontext{function object argument} transformation which an algorithm applies
before inspecting the values of elements

\enterexample
\begin{codeblock}
std::pair<int, const char*> pairs[] = {{2, "foo"}, {1, "bar"}, {0, "baz"}};
std2::sort(pairs, std::less<>{}, [](auto const& p) { return p.first; });
\end{codeblock}
sorts the pairs in increasing order of their \tcode{first} members:
\begin{codeblock}
{{0, "baz"}, {1, "bar"}, {2, "foo"}}
\end{codeblock}
\exitexample

\rSec0[intro]{General Principles}

\rSec1[intro.goals]{Goals}

\pnum
The primary goal of this proposal is to deliver high-quality, constrained generic
Standard Library components at the same time that the language gets support for
such components. 

\rSec1[intro.rationale]{Rationale}

\pnum
The best, and arguably only practical way to achieve the goal stated above is by
incorporating the Ranges TS into the working paper. The sooner we can agree on
what we want ``\tcode{Iterator}'' and ``\tcode{Range}'' to mean going forward
(for instance), and the sooner users are able to rely on them, the sooner we can
start building and delivering functionality on top of those fundamental
abstractions. (For example, see ``P0789: Range Adaptors and
Utilities''~(\cite{P0789}).)

\pnum
The cost of not delivering such a set of Standard Library concepts
and algorithms is that users will either do without or create a babel of mutually
incompatible concepts and algorithms, often without the rigour followed by the
Ranges TS. The experience of the authors and implementors of the Ranges TS is that
getting concept definitions and algorithm constraints right is \textit{hard}. The
Standard Library should save its users from needless heartache.

\rSec1[intro.risks]{Risks}

\pnum
Shipping constrained components from the Ranges TS in the C++20 timeframe is not
without risk. As of the time of writing (February 1, 2018), no major Standard
Library vendor has shipped an implementation of the Ranges TS. Two of the three
major compiler vendors have not even shipped an implementation of the concepts
language feature. Arguably, we have not yet gotten the usage experience for which
all Technical Specifications are intended.

\pnum
On the other hand, the components of Ranges TS have been vetted very thoroughly
by the range-v3~(\cite{range-v3}) project, on which the Ranges TS is based. There is
no part of the Ranges TS -- concepts included -- that has not seen extensive use
via range-v3. (The concepts in range-v3 are emulated with high fidelity through
the use of generalized SFINAE for expressions.) As an Open Source project, usage
statistics are hard to come by, but the following may be indicitive:

\begin{itemize}
\item The range-v3 GitHub project has over 1,400 stars, over 120 watchers, and
145 forks.
\item It is cloned on average about 6,000 times a month.
\item A GitHub search, restricted to C++ files, for the string
``\tcode{range/v3}'' (a path component of all of range-v3's header files), turns
up over 7,000 hits.
\end{itemize}

\pnum
Lacking true concepts, range-v3 cannot emulate concept-based function
overloading, or the sorts of constraints-checking short-circuit evaluation
required by true concepts. For that reason, the authors of the Ranges TS have
created a reference implementation: CMCSTL2~(\cite{cmcstl2}) using true concepts. To
this reference implementation, the authors ported all of range-v3's tests. These
exposed only a handful of concepts-specific bugs in the components of the Ranges
TS (and a great many more bugs in compilers). Those improvements were back-ported
to range-v3 where they have been thoroughly vetted over the past 2 years.

\pnum
In short, concern about lack of implementation experience should not be a reason
to withhold this important Standard Library advance from users.

\rSec1[intro.methedology]{Methodology}

\pnum
The contents of the Ranges TS, Clause 7 (``Concepts library'') are proposed
for namespace \tcode{std} by P0898, ``Standard Library Concepts''~(\cite{P0898}).
Additionally, P0898 proposes the \tcode{identity} function
object~(\tsref{func.identity}) and the \tcode{common_reference} type
trait~(\tsref{meta.trans.other}) for namespace \tcode{std}, and the \tcode{swap}
customization point object for namespace \tcode{std2}. The changes proposed by
the Ranges TS to \tcode{common_type} are merged into the working paper (also
by P0898). The ``\tcode{invoke}'' function and the ``\tcode{swappable}'' type
traits (e.g., \tcode{is_swappable_with}) already exist in the text of the
working paper, so they are omitted here.

\pnum
The salient, high-level features of this proposal are as follows:

\begin{itemize}
\item The remaining library components in the Ranges TS are proposed for
namespace \tcode{::std2::}.

\item The text of the Ranges TS is rebased on the latest working draft.

\item Structurally, this paper proposes to relocate the existing library clauses
of the working draft (20-33) down one level under a new clause 20, provisionally
titled ``Standard Library, Version 1''. No stable names are changed.

\item The Concepts Library clause, proposed by P0898, is located in that paper
between the ``Language Support Library'' and the ``Diagnostics library''. In the
organization proposed by this paper, that places it as subclause 20.3. This paper
refers to it as such.

\item We additionally propose that a new clause 21 be created, provisionally
titled ``Standard Library, Version 2'', and the following clauses of the Ranges
TS should be made subclauses of this clause: 6, 8-12.

\item Where the text of the Ranges TS needs to be updated, the text is presented
with change markings: \removed{red strikethrough} for removed text and
\added{blue underline} for added text.

\item The stable names of everything in the Ranges TS, clauses 6, 8-12 are
changed by preprending ``\tcode{std2.}''. References are updated accordingly.

\item The headers of the Ranges TS are renamed from
\tcode{<experimental/ranges/\textit{foo}>} to \tcode{<std2/\textit{foo}>}.
\end{itemize}

\rSec1[intro.compliance]{Implementation compliance}

\pnum
Conformance requirements for this specification are the same as those
defined in \cxxref{intro.compliance} in the \Cpp Standard.
\enternote
Conformance is defined in terms of the behavior of programs.
\exitnote

\rSec1[intro.namespaces]{Namespaces, headers, and modifications to standard classes}

\pnum
Since the library components described in this document are constrained versions
of facilities already found within namespace \tcode{std}, we propose to define
everything within namespace \tcode{::std2::v1}, where \tcode{v1} is an inline
namespace.

\pnum
Unless otherwise specified, references to entities described in this
document are assumed to be qualified with \tcode{::std2::}, and
references to entities described in the current working draft of the
International Standard, including the entities of the ``Concepts library''
proposed by P0898, are assumed to be qualified with \tcode{::std::}.
