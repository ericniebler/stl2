%!TEX root = std.tex
\rSec0[std2.iterators]{Iterators library}

\rSec1[std2.iterators.general]{General}

\pnum
This \changed{C}{Subc}lause describes components that \Cpp programs may use to perform
iterations over containers (\changed{C}{Subc}lause \cxxref{containers}),
streams~(\cxxref{iostream.format}),
and stream buffers~(\cxxref{stream.buffers}).

\pnum
The following subclauses describe
iterator requirements, and
components for
iterator primitives,
predefined iterators,
and stream iterators,
as summarized in Table~\ref{tab:iterators.lib.summary}.

\begin{libsumtab}{Iterators library summary}{tab:iterators.lib.summary}
\ref{std2.iterator.requirements} & Iterator requirements          &                           \\
\ref{std2.indirectcallable}      & Indirect callable requirements &                           \\
\ref{std2.commonalgoreq}         & Common algorithm requirements  &                           \\ \rowsep
\ref{std2.iterator.primitives}   & Iterator primitives            & \tcode{<\changed{experimental/ranges}{std2}/iterator>} \\
\ref{std2.iterators.predef}      & Predefined iterators           &                           \\
\ref{std2.iterators.stream}      & Stream iterators               &                           \\
\end{libsumtab}

\rSec1[std2.iterator.synopsis]{Header \tcode{<std2/iterator>} synopsis}

\indexlibrary{\idxhdr{std2/iterator}}%
\begin{codeblock}
namespace @\changed{std \{ namespace experimental \{ namespace ranges}{std2}@ { inline namespace v1 {
  template <class T> concept @\removed{bool}@ @\placeholder{dereferenceable}@ // \expos
    = requires(T& t) { {*t} -> auto&&; };

  // \ref{std2.iterator.requirements}, iterator requirements:
  // \ref{std2.iterator.custpoints}, customization points:
  @\removed{namespace \{}@
    // \ref{std2.iterator.custpoints.iter_move}, iter_move:
    @\added{inline}@ constexpr @\unspec@ iter_move = @\unspec@;

    // \ref{std2.iterator.custpoints.iter_swap}, iter_swap:
    @\added{inline}@ constexpr @\unspec@ iter_swap = @\unspec@;
  @\removed{\}}@

  // \ref{std2.iterator.assoc.types}, associated types:
  // \ref{std2.iterator.assoc.types.difference_type}, difference_type:
  template <class> struct difference_type;
  template <class T> using difference_type_t
    = typename difference_type<T>::type;

  // \ref{std2.iterator.assoc.types.value_type}, value_type:
  template <class> struct value_type;
  template <class T> using value_type_t
    = typename value_type<T>::type;

  // \ref{std2.iterator.assoc.types.iterator_category}, iterator_category:
  template <class> struct iterator_category;
  template <class T> using iterator_category_t
    = typename iterator_category<T>::type;

  template <@\placeholder{dereferenceable}@ T> using reference_t
    = decltype(*declval<T&>());

  template <@\placeholder{dereferenceable}@ T>
      requires @\seebelow@ using rvalue_reference_t
    = decltype(@\changed{ranges}{::std2}@::iter_move(declval<T&>()));

  // \ref{std2.iterators.readable}, Readable:
  template <class In>
  concept @\removed{bool}@ Readable = @\seebelow@;

  // \ref{std2.iterators.writable}, Writable:
  template <class Out, class T>
  concept @\removed{bool}@ Writable = @\seebelow@;

  // \ref{std2.iterators.weaklyincrementable}, WeaklyIncrementable:
  template <class I>
  concept @\removed{bool}@ WeaklyIncrementable = @\seebelow@;

  // \ref{std2.iterators.incrementable}, Incrementable:
  template <class I>
  concept @\removed{bool}@ Incrementable = @\seebelow@;

  // \ref{std2.iterators.iterator}, Iterator:
  template <class I>
  concept @\removed{bool}@ Iterator = @\seebelow@;

  // \ref{std2.iterators.sentinel}, Sentinel:
  template <class S, class I>
  concept @\removed{bool}@ Sentinel = @\seebelow@;

  // \ref{std2.iterators.sizedsentinel}, SizedSentinel:
  template <class S, class I>
    constexpr bool disable_sized_sentinel = false;

  template <class S, class I>
  concept @\removed{bool}@ SizedSentinel = @\seebelow@;

  // \ref{std2.iterators.input}, InputIterator:
  template <class I>
  concept @\removed{bool}@ InputIterator = @\seebelow@;

  // \ref{std2.iterators.output}, OutputIterator:
  template <class I>
  concept @\removed{bool}@ OutputIterator = @\seebelow@;

  // \ref{std2.iterators.forward}, ForwardIterator:
  template <class I>
  concept @\removed{bool}@ ForwardIterator = @\seebelow@;

  // \ref{std2.iterators.bidirectional}, BidirectionalIterator:
  template <class I>
  concept @\removed{bool}@ BidirectionalIterator = @\seebelow@;

  // \ref{std2.iterators.random.access}, RandomAccessIterator:
  template <class I>
  concept @\removed{bool}@ RandomAccessIterator = @\seebelow@;

  // \ref{std2.indirectcallable}, indirect callable requirements:
  // \ref{std2.indirectcallable.indirectinvocable}, indirect callables:
  template <class F, class I>
  concept @\removed{bool}@ IndirectUnaryInvocable = @\seebelow@;

  template <class F, class I>
  concept @\removed{bool}@ IndirectRegularUnaryInvocable = @\seebelow@;

  template <class F, class I>
  concept @\removed{bool}@ IndirectUnaryPredicate = @\seebelow@;

  template <class F, class I1, class I2 = I1>
  concept @\removed{bool}@ IndirectRelation = @\seebelow@;

  template <class F, class I1, class I2 = I1>
  concept @\removed{bool}@ IndirectStrictWeakOrder = @\seebelow@;

  template <class> struct indirect_result_of;

  template <class F, class... Is>
    requires Invocable<F, reference_t<Is>...>
  struct indirect_result_of<F(Is...)>;

  template <class F>
  using indirect_result_of_t
    = typename indirect_result_of<F>::type;

  // \ref{std2.projected}, projected:
  template <Readable I, IndirectRegularUnaryInvocable<I> Proj>
  struct projected;

  template <WeaklyIncrementable I, class Proj>
  struct difference_type<projected<I, Proj>>;

  // \ref{std2.commonalgoreq}, common algorithm requirements:
  // \ref{std2.commonalgoreq.indirectlymovable} IndirectlyMovable:
  template <class In, class Out>
  concept @\removed{bool}@ IndirectlyMovable = @\seebelow@;

  template <class In, class Out>
  concept @\removed{bool}@ IndirectlyMovableStorable = @\seebelow@;

  // \ref{std2.commonalgoreq.indirectlycopyable} IndirectlyCopyable:
  template <class In, class Out>
  concept @\removed{bool}@ IndirectlyCopyable = @\seebelow@;

  template <class In, class Out>
  concept @\removed{bool}@ IndirectlyCopyableStorable = @\seebelow@;

  // \ref{std2.commonalgoreq.indirectlyswappable} IndirectlySwappable:
  template <class I1, class I2 = I1>
  concept @\removed{bool}@ IndirectlySwappable = @\seebelow@;

  // \ref{std2.commonalgoreq.indirectlycomparable} IndirectlyComparable:
  template <class I1, class I2, class R = equal_to<>, class P1 = identity,
      class P2 = identity>
  concept @\removed{bool}@ IndirectlyComparable = @\seebelow@;

  // \ref{std2.commonalgoreq.permutable} Permutable:
  template <class I>
  concept @\removed{bool}@ Permutable = @\seebelow@;

  // \ref{std2.commonalgoreq.mergeable} Mergeable:
  template <class I1, class I2, class Out,
      class R = less<>, class P1 = identity, class P2 = identity>
  concept @\removed{bool}@ Mergeable = @\seebelow@;

  template <class I, class R = less<>, class P = identity>
  concept @\removed{bool}@ Sortable = @\seebelow@;

  // \ref{std2.iterator.primitives}, primitives:
  // \ref{std2.iterator.traits}, traits:
  template <class Iterator> using iterator_traits = @\seebelow@;

  template <Readable T> using iter_common_reference_t
    = common_reference_t<reference_t<T>, value_type_t<T>&>;

  // \ref{std2.iterator.tags}, iterator tags:
  struct output_iterator_tag { };
  struct input_iterator_tag { };
  struct forward_iterator_tag : input_iterator_tag { };
  struct bidirectional_iterator_tag : forward_iterator_tag { };
  struct random_access_iterator_tag : bidirectional_iterator_tag { };

  // \ref{std2.iterator.operations}, iterator operations:
  @\removed{namespace \{}@
    @\added{inline}@ constexpr @\unspec@ advance = @\unspec@;
    @\added{inline}@ constexpr @\unspec@ distance = @\unspec@;
    @\added{inline}@ constexpr @\unspec@ next = @\unspec@;
    @\added{inline}@ constexpr @\unspec@ prev = @\unspec@;
  @\removed{\}}@

  // \ref{std2.iterators.predef}, predefined iterators and sentinels:

  // \ref{std2.iterators.reverse}, reverse iterators:
  template <BidirectionalIterator I> class reverse_iterator;

  template <class I1, class I2>
      requires EqualityComparableWith<I1, I2>
    constexpr bool operator==(
      const reverse_iterator<I1>& x,
      const reverse_iterator<I2>& y);
  template <class I1, class I2>
      requires EqualityComparableWith<I1, I2>
    constexpr bool operator!=(
      const reverse_iterator<I1>& x,
      const reverse_iterator<I2>& y);
  template <class I1, class I2>
      requires StrictTotallyOrderedWith<I1, I2>
    constexpr bool operator<(
      const reverse_iterator<I1>& x,
      const reverse_iterator<I2>& y);
  template <class I1, class I2>
      requires StrictTotallyOrderedWith<I1, I2>
    constexpr bool operator>(
      const reverse_iterator<I1>& x,
      const reverse_iterator<I2>& y);
  template <class I1, class I2>
      requires StrictTotallyOrderedWith<I1, I2>
    constexpr bool operator>=(
      const reverse_iterator<I1>& x,
      const reverse_iterator<I2>& y);
  template <class I1, class I2>
      requires StrictTotallyOrderedWith<I1, I2>
    constexpr bool operator<=(
      const reverse_iterator<I1>& x,
      const reverse_iterator<I2>& y);

  template <class I1, class I2>
      requires SizedSentinel<I1, I2>
    constexpr difference_type_t<I2> operator-(
      const reverse_iterator<I1>& x,
      const reverse_iterator<I2>& y);
  template <RandomAccessIterator I>
    constexpr reverse_iterator<I> operator+(
      difference_type_t<I> n,
      const reverse_iterator<I>& x);

  template <BidirectionalIterator I>
    constexpr reverse_iterator<I> make_reverse_iterator(I i);

  // \ref{std2.iterators.insert}, insert iterators:
  template <class Container> class back_insert_iterator;
  template <class Container>
    back_insert_iterator<Container> back_inserter(Container& x);

  template <class Container> class front_insert_iterator;
  template <class Container>
    front_insert_iterator<Container> front_inserter(Container& x);

  template <class Container> class insert_iterator;
  template <class Container>
    insert_iterator<Container> inserter(Container& x, iterator_t<Container> i);

  // \ref{std2.iterators.move}, move iterators and sentinels:
  template <InputIterator I> class move_iterator;
  template <class I1, class I2>
      requires EqualityComparableWith<I1, I2>
    constexpr bool operator==(
      const move_iterator<I1>& x, const move_iterator<I2>& y);
  template <class I1, class I2>
      requires EqualityComparableWith<I1, I2>
    constexpr bool operator!=(
      const move_iterator<I1>& x, const move_iterator<I2>& y);
  template <class I1, class I2>
      requires StrictTotallyOrderedWith<I1, I2>
    constexpr bool operator<(
      const move_iterator<I1>& x, const move_iterator<I2>& y);
  template <class I1, class I2>
      requires StrictTotallyOrderedWith<I1, I2>
    constexpr bool operator<=(
      const move_iterator<I1>& x, const move_iterator<I2>& y);
  template <class I1, class I2>
      requires StrictTotallyOrderedWith<I1, I2>
    constexpr bool operator>(
      const move_iterator<I1>& x, const move_iterator<I2>& y);
  template <class I1, class I2>
      requires StrictTotallyOrderedWith<I1, I2>
    constexpr bool operator>=(
      const move_iterator<I1>& x, const move_iterator<I2>& y);

  template <class I1, class I2>
      requires SizedSentinel<I1, I2>
    constexpr difference_type_t<I2> operator-(
      const move_iterator<I1>& x,
      const move_iterator<I2>& y);
  template <RandomAccessIterator I>
    constexpr move_iterator<I> operator+(
      difference_type_t<I> n,
      const move_iterator<I>& x);
  template <InputIterator I>
    constexpr move_iterator<I> make_move_iterator(I i);

  template <Semiregular S> class move_sentinel;

  template <class I, Sentinel<I> S>
    constexpr bool operator==(
      const move_iterator<I>& i, const move_sentinel<S>& s);
  template <class I, Sentinel<I> S>
    constexpr bool operator==(
      const move_sentinel<S>& s, const move_iterator<I>& i);
  template <class I, Sentinel<I> S>
    constexpr bool operator!=(
      const move_iterator<I>& i, const move_sentinel<S>& s);
  template <class I, Sentinel<I> S>
    constexpr bool operator!=(
      const move_sentinel<S>& s, const move_iterator<I>& i);

  template <class I, SizedSentinel<I> S>
    constexpr difference_type_t<I> operator-(
      const move_sentinel<S>& s, const move_iterator<I>& i);
  template <class I, SizedSentinel<I> S>
    constexpr difference_type_t<I> operator-(
      const move_iterator<I>& i, const move_sentinel<S>& s);

  template <Semiregular S>
    constexpr move_sentinel<S> make_move_sentinel(S s);

  // \ref{std2.iterators.common}, common iterators:
  template <Iterator I, Sentinel<I> S>
    requires !Same<I, S>
  class common_iterator;

  template <Readable I, class S>
  struct value_type<common_iterator<I, S>>;

  template <InputIterator I, class S>
  struct iterator_category<common_iterator<I, S>>;

  template <ForwardIterator I, class S>
  struct iterator_category<common_iterator<I, S>>;

  template <class I1, class I2, Sentinel<I2> S1, Sentinel<I1> S2>
  bool operator==(
    const common_iterator<I1, S1>& x, const common_iterator<I2, S2>& y);
  template <class I1, class I2, Sentinel<I2> S1, Sentinel<I1> S2>
    requires EqualityComparableWith<I1, I2>
  bool operator==(
    const common_iterator<I1, S1>& x, const common_iterator<I2, S2>& y);
  template <class I1, class I2, Sentinel<I2> S1, Sentinel<I1> S2>
  bool operator!=(
    const common_iterator<I1, S1>& x, const common_iterator<I2, S2>& y);

  template <class I2, SizedSentinel<I2> I1, SizedSentinel<I2> S1, SizedSentinel<I1> S2>
  difference_type_t<I2> operator-(
    const common_iterator<I1, S1>& x, const common_iterator<I2, S2>& y);

  // \ref{std2.default.sentinels}, default sentinels:
  class default_sentinel;

  // \ref{std2.iterators.counted}, counted iterators:
  template <Iterator I> class counted_iterator;

  template <class I1, class I2>
      requires Common<I1, I2>
    constexpr bool operator==(
      const counted_iterator<I1>& x, const counted_iterator<I2>& y);
  @\added{template <class I>}@
    constexpr bool operator==(
      const counted_iterator<@\changed{auto}{I}@>& x, default_sentinel);
  @\added{template <class I>}@
    constexpr bool operator==(
      default_sentinel, const counted_iterator<@\changed{auto}{I}@>& x);
  template <class I1, class I2>
      requires Common<I1, I2>
    constexpr bool operator!=(
      const counted_iterator<I1>& x, const counted_iterator<I2>& y);
  @\added{template <class I>}@
    constexpr bool operator!=(
      const counted_iterator<@\changed{auto}{I}@>& x, default_sentinel y);
  @\added{template <class I>}@
    constexpr bool operator!=(
      default_sentinel x, const counted_iterator<@\changed{auto}{I}@>& y);
  template <class I1, class I2>
      requires Common<I1, I2>
    constexpr bool operator<(
      const counted_iterator<I1>& x, const counted_iterator<I2>& y);
  template <class I1, class I2>
      requires Common<I1, I2>
    constexpr bool operator<=(
      const counted_iterator<I1>& x, const counted_iterator<I2>& y);
  template <class I1, class I2>
      requires Common<I1, I2>
    constexpr bool operator>(
      const counted_iterator<I1>& x, const counted_iterator<I2>& y);
  template <class I1, class I2>
      requires Common<I1, I2>
    constexpr bool operator>=(
      const counted_iterator<I1>& x, const counted_iterator<I2>& y);
  template <class I1, class I2>
      requires Common<I1, I2>
    constexpr difference_type_t<I2> operator-(
      const counted_iterator<I1>& x, const counted_iterator<I2>& y);
  template <class I>
    constexpr difference_type_t<I> operator-(
      const counted_iterator<I>& x, default_sentinel y);
  template <class I>
    constexpr difference_type_t<I> operator-(
      default_sentinel x, const counted_iterator<I>& y);
  template <RandomAccessIterator I>
    constexpr counted_iterator<I>
      operator+(difference_type_t<I> n, const counted_iterator<I>& x);
  template <Iterator I>
    constexpr counted_iterator<I> make_counted_iterator(I i, difference_type_t<I> n);

  // \ref{std2.unreachable.sentinels}, unreachable sentinels:
  class unreachable;
  template <Iterator I>
    constexpr bool operator==(const I&, unreachable) noexcept;
  template <Iterator I>
    constexpr bool operator==(unreachable, const I&) noexcept;
  template <Iterator I>
    constexpr bool operator!=(const I&, unreachable) noexcept;
  template <Iterator I>
    constexpr bool operator!=(unreachable, const I&) noexcept;

  // \ref{std2.dangling.wrappers}, dangling wrapper:
  template <class T> class dangling;

  // \ref{std2.iterators.stream}, stream iterators:
  template <class T, class charT = char, class traits = char_traits<charT>,
      class Distance = ptrdiff_t>
  class istream_iterator;
  template <class T, class charT, class traits, class Distance>
    bool operator==(const istream_iterator<T, charT, traits, Distance>& x,
            const istream_iterator<T, charT, traits, Distance>& y);
  template <class T, class charT, class traits, class Distance>
    bool operator==(default_sentinel x,
            const istream_iterator<T, charT, traits, Distance>& y);
  template <class T, class charT, class traits, class Distance>
    bool operator==(const istream_iterator<T, charT, traits, Distance>& x,
            default_sentinel y);
  template <class T, class charT, class traits, class Distance>
    bool operator!=(const istream_iterator<T, charT, traits, Distance>& x,
            const istream_iterator<T, charT, traits, Distance>& y);
  template <class T, class charT, class traits, class Distance>
   bool operator!=(default_sentinel x,
            const istream_iterator<T, charT, traits, Distance>& y);
  template <class T, class charT, class traits, class Distance>
    bool operator!=(const istream_iterator<T, charT, traits, Distance>& x,
            default_sentinel y);

  template <class T, class charT = char, class traits = char_traits<charT>>
      class ostream_iterator;

  template <class charT, class traits = char_traits<charT> >
    class istreambuf_iterator;
  template <class charT, class traits>
    bool operator==(const istreambuf_iterator<charT, traits>& a,
            const istreambuf_iterator<charT, traits>& b);
  template <class charT, class traits>
    bool operator==(default_sentinel a,
            const istreambuf_iterator<charT, traits>& b);
  template <class charT, class traits>
    bool operator==(const istreambuf_iterator<charT, traits>& a,
            default_sentinel b);
  template <class charT, class traits>
    bool operator!=(const istreambuf_iterator<charT, traits>& a,
            const istreambuf_iterator<charT, traits>& b);
  template <class charT, class traits>
    bool operator!=(default_sentinel a,
            const istreambuf_iterator<charT, traits>& b);
  template <class charT, class traits>
    bool operator!=(const istreambuf_iterator<charT, traits>& a,
            default_sentinel b);

  template <class charT, class traits = char_traits<charT> >
    class ostreambuf_iterator;
}}@\removed{\}\}}@

namespace std {
  // \ref{std2.iterator.stdtraits}, iterator traits:
  template <@\changed{experimental::ranges}{::std2}@::Iterator Out>
    struct iterator_traits<Out>;
  template <@\changed{experimental::ranges}{::std2}@::InputIterator In>
    struct iterator_traits<In>;
  template <@\changed{experimental::ranges}{::std2}@::InputIterator In>
      requires @\changed{experimental::ranges}{::std2}@::Sentinel<In, In>
    struct iterator_traits;
}
\end{codeblock}

\rSec1[std2.iterator.requirements]{Iterator requirements}

\rSec2[std2.iterator.requirements.general]{General}

\pnum
\indextext{requirements!iterator}%
Iterators are a generalization of pointers that allow a \Cpp program to work with different data structures
(for example, containers and ranges) in a uniform manner.
To be able to construct template algorithms that work correctly and
efficiently on different types of data structures, the library formalizes not just the interfaces but also the
semantics and complexity assumptions of iterators.
All input iterators
\tcode{i}
support the expression
\tcode{*i},
resulting in a value of some object type
\tcode{T},
called the
\term{value type}
of the iterator.
All output iterators support the expression
\tcode{*i = o}
where
\tcode{o}
is a value of some type that is in the set of types that are
\term{writable}
to the particular iterator type of
\tcode{i}.
For every iterator type
\tcode{X}
there is a corresponding signed integer type called the
\term{difference type}
of the iterator.

\pnum
Since iterators are an abstraction of pointers, their semantics are
a generalization of most of the semantics of pointers in \Cpp.
This ensures that every
function template
that takes iterators
works as well with regular pointers.
This document defines
five categories of iterators, according to the operations
defined on them:
\techterm{input iterators},
\techterm{output iterators},
\techterm{forward iterators},
\techterm{bidirectional iterators}
and
\techterm{random access iterators},
as shown in Table~\ref{tab:iterators.relations}.

\begin{floattable}{Relations among iterator categories}{tab:iterators.relations}
{llll}
\topline
\textbf{Random Access}          &   $\rightarrow$ \textbf{Bidirectional}    &
$\rightarrow$ \textbf{Forward}  &   $\rightarrow$ \textbf{Input}            \\
                        &   &   &   $\rightarrow$ \textbf{Output}           \\
\end{floattable}

\pnum
The five categories of iterators correspond to the iterator concepts
\tcode{Input\-Iterator},
\tcode{Output\-Iterator},
\tcode{Forward\-Iterator},
\tcode{Bidirectional\-Iterator}, and
\tcode{RandomAccess\-Iterator}, respectively. The generic term \techterm{iterator} refers to
any type that satisfies \tcode{Iterator}.

\pnum
Forward iterators satisfy all the requirements of input
iterators and can be used whenever an input iterator is specified;
Bidirectional iterators also satisfy all the requirements of
forward iterators and can be used whenever a forward iterator is specified;
Random access iterators also satisfy all the requirements of bidirectional
iterators and can be used whenever a bidirectional iterator is specified.

\pnum
Iterators that further satisfy the requirements of output iterators are
called \defn{mutable iterator}{s}. Nonmutable iterators are referred to
as \defn{constant iterator}{s}.

\pnum
Just as a regular pointer to an array guarantees that there is a pointer value pointing past the last element
of the array, so for any iterator type there is an iterator value that points past the last element of a
corresponding sequence.
These values are called
\term{past-the-end}
values.
Values of an iterator
\tcode{i}
for which the expression
\tcode{*i}
is defined are called
\term{dereferenceable}.
The library never assumes that past-the-end values are dereferenceable.
Iterators can also have singular values that are not associated with any
sequence.
\enterexample
After the declaration of an uninitialized pointer
\tcode{x}
(as with
\tcode{int* x;}),
\tcode{x}
must always be assumed to have a singular value of a pointer.
\exitexample
Results of most expressions are undefined for singular values;
the only exceptions are destroying an iterator that holds a singular value,
the assignment of a non-singular value to
an iterator that holds a singular value, and using a value-initialized iterator
as the source of a copy or move operation. \enternote This guarantee is not
offered for default initialization, although the distinction only matters for types
with trivial default constructors such as pointers or aggregates holding pointers.
\exitnote
In these cases the singular
value is overwritten the same way as any other value.
Dereferenceable
values are always non-singular.

\pnum
Most of the library's algorithmic templates that operate on data structures have
interfaces that use ranges. A range is an iterator and a \term{sentinel} that designate
the beginning and end of the computation, or an iterator and a count that designate
the beginning and the number of elements to which the computation is to be applied.

\pnum
An iterator and a sentinel denoting a range are comparable. The types of a sentinel
and an iterator that denote a range must satisfy
\tcode{Sentinel}~(\ref{std2.iterators.sentinel}).
A range \range{i}{s}
is empty if \tcode{i == s};
otherwise, \range{i}{s}
refers to the elements in the data structure starting with the element
pointed to by
\tcode{i}
and up to but not including the element pointed to by
the first iterator \tcode{j} such that \tcode{j == s}.

\pnum
A sentinel
\tcode{s}
is called
\term{reachable}
from an iterator
\tcode{i}
if and only if there is a finite sequence of applications of
the expression
\tcode{++i}
that makes
\tcode{i == s}.
If
\tcode{s}
is reachable from
\tcode{i},
\range{i}{s} denotes a range.

\pnum
A counted range \range{i}{n} is empty if \tcode{n == 0}; otherwise, \range{i}{n}
refers to the \tcode{n} elements in the data structure starting with the element
pointed to by \tcode{i} and up to but not including the element pointed to by the
result of incrementing \tcode{i} \tcode{n} times.

\pnum
A range \range{i}{s}
is valid if and only if
\tcode{s}
is reachable from
\tcode{i}.
A counted range \range{i}{n} is valid if and only if \tcode{n == 0}; or \tcode{n}
is positive, \tcode{i} is dereferenceable, and \range{++i}{-{-}n} is valid.
The result of the application of functions in the library to invalid ranges is
undefined.

\pnum
All the categories of iterators require only those functions that are realizable for a given category in
constant time (amortized).

\pnum
Destruction of an iterator may invalidate pointers and references
previously obtained from that iterator.

\pnum
An
\techterm{invalid}
iterator is an iterator that may be singular.\footnote{This definition applies to pointers, since pointers are iterators.
The effect of dereferencing an iterator that has been invalidated
is undefined.
}

\rSec2[std2.iterator.custpoints]{Customization points}

\rSec3[std2.iterator.custpoints.iter_move]{\tcode{iter_move}}

\pnum
The name \tcode{iter_move} denotes a \techterm{customization point
object}~(\cxxref{customization.point.object}). The expression
\tcode{\changed{ranges}{::std2}::iter_move(E)} for some subexpression \tcode{E} is expression-equivalent to the
following:

\begin{itemize}
\item \tcode{static_cast<decltype(iter_move(E))>(iter_move(E))}, if that expression is well-formed when evaluated in
a context that does not include \tcode{\changed{ranges}{::std2}::iter_move} but does include the
lookup set produced by argument-dependent lookup~(\cxxref{basic.lookup.argdep}).

\item Otherwise, if the expression \tcode{*E} is well-formed:
\begin{itemize}
\item if \tcode{*E} is an lvalue, \tcode{std::move(*E)};

\item otherwise, \tcode{static_cast<decltype(*E)>(*E)}.
\end{itemize}

\item Otherwise, \tcode{\changed{ranges}{::std2}::iter_move(E)} is ill-formed.
\end{itemize}

\pnum
If \tcode{\changed{ranges}{::std2}::iter_move(E)} does not equal \tcode{*E}, the program is
ill-formed with no diagnostic required.

\rSec3[std2.iterator.custpoints.iter_swap]{\tcode{iter_swap}}

\pnum
The name \tcode{iter_swap} denotes a \techterm{customization point
object}~(\cxxref{customization.point.object}). The expression
\tcode{\changed{ranges}{::std2}::iter_swap(E1, E2)} for some subexpressions \tcode{E1} and \tcode{E2}
is expression-equivalent to the following:

\begin{itemize}
\item \tcode{(void)iter_swap(E1, E2)}, if that expression is well-formed when
evaluated in a context that does not include \tcode{\changed{ranges}{::std2}::iter_swap} but does
include the lookup set produced by argument-dependent
lookup~(\cxxref{basic.lookup.argdep}) and the following declaration:
\begin{codeblock}
@\added{template <class I1, class I2>}@
void iter_swap(@\changed{auto, auto}{I1, I2}@) = delete;
\end{codeblock}

\item Otherwise, if the types of \tcode{E1} and \tcode{E2} both satisfy
\tcode{Readable}, and if the reference type of \tcode{E1} is swappable
with~(\cxxref{concepts.lib.corelang.swappable}) the reference type of \tcode{E2},
then \tcode{\changed{ranges}{::std2}::swap(*E1, *E2)}

\item Otherwise, if the types \tcode{T1} and \tcode{T2} of \tcode{E1} and
\tcode{E2} satisfy \tcode{IndirectlyMovableStorable<T1, T2> \&\&
IndirectlyMovableStorable<T2, T1>}, \tcode{(void)(*E1 = iter_exchange_move(E2, E1))},
except that \tcode{E1} is evaluated only once.

\item Otherwise, \tcode{\changed{ranges}{::std2}::iter_swap(E1, E2)} is ill-formed.
\end{itemize}

\pnum
If \tcode{\changed{ranges}{::std2}::iter_swap(E1, E2)} does not swap the values denoted by the
expressions \tcode{E1} and \tcode{E2}, the program is ill-formed with no
diagnostic required.

\pnum
\tcode{iter_exchange_move} is an exposition-only function specified as:
\begin{itemdecl}
template <class X, class Y>
  constexpr value_type_t<remove_reference_t<X>> iter_exchange_move(X&& x, Y&& y)
    noexcept(@\seebelow@);
\end{itemdecl}

\begin{itemdescr}
\pnum
\effects Equivalent to:
\begin{codeblock}
value_type_t<remove_reference_t<X>> old_value(iter_move(x));
*x = iter_move(y);
return old_value;
\end{codeblock}

\pnum
\remarks The expression in the \tcode{noexcept} is equivalent to:
\begin{codeblock}
NE(remove_reference_t<X>, remove_reference_t<Y>) &&
NE(remove_reference_t<Y>, remove_reference_t<X>)
\end{codeblock}
Where \tcode{NE(T1, T2)} is the expression:
\begin{codeblock}
is_nothrow_constructible@\added{_v}@<value_type_t<T1>, rvalue_reference_t<T1>>@\removed{::value}@ &&
is_nothrow_assignable@\added{_v}@<value_type_t<T1>&, rvalue_reference_t<T1>>@\removed{::value}@ &&
is_nothrow_assignable@\added{_v}@<reference_t<T1>, rvalue_reference_t<T2>>@\removed{::value}@ &&
is_nothrow_assignable@\added{_v}@<reference_t<T1>, value_type_t<T2>>@\removed{::value}@> &&
is_nothrow_move_constructible@\added{_v}@<value_type_t<T1>>@\removed{::value}@ &&
noexcept(@\changed{ranges}{::std2}@::iter_move(declval<T1&>()))
\end{codeblock}
\end{itemdescr}

\rSec2[std2.iterator.assoc.types]{Iterator associated types}

\pnum
To implement algorithms only in terms of iterators, it is often necessary to
determine the value and
difference types that correspond to a particular iterator type.
Accordingly, it is required that if
\tcode{WI} is the name of a type that
satisfies the \tcode{WeaklyIncrementable} concept~(\ref{std2.iterators.weaklyincrementable}),
\tcode{R} is the name of a type that
satisfies the \tcode{Readable} concept~(\ref{std2.iterators.readable}), and
\tcode{II} is the name of a type that satisfies the
\tcode{InputIterator} concept~(\ref{std2.iterators.input}) concept, the types

\begin{codeblock}
difference_type_t<WI>
value_type_t<R>
iterator_category_t<II>
\end{codeblock}

be defined as the iterator's difference type, value type and iterator category, respectively.

\rSec3[std2.iterator.assoc.types.difference_type]{\tcode{difference_type}}

\pnum
\indexlibrary{\idxcode{difference_type_t}}%
\tcode{difference_type_t<T>} is implemented as if:

\indexlibrary{\idxcode{difference_type}}%
\begin{codeblock}
  template <class> struct difference_type { };

  template <class T>
  struct difference_type<T*>
    : enable_if<is_object@\added{_v}@<T>@\removed{::value}@, ptrdiff_t> { };

  template <class I>
  struct difference_type<const I> : difference_type<decay_t<I>> { };

  template <class T>
    requires requires { typename T::difference_type; }
  struct difference_type<T> {
    using type = typename T::difference_type;
  };

  template <class T>
    requires !requires { typename T::difference_type; } &&
      requires(const T& a, const T& b) { { a - b } -> Integral; }
  struct difference_type<T>
    : make_signed< decltype(declval<T>() - declval<T>()) > {
  };

  template <class T> using difference_type_t
    = typename difference_type<T>::type;
\end{codeblock}

\pnum
Users may specialize \tcode{difference_type} on user-defined types.

\rSec3[std2.iterator.assoc.types.value_type]{\tcode{value_type}}

\pnum
A \tcode{Readable} type has an associated value type that can be accessed with the
\tcode{value_type_t} alias template.

\indexlibrary{\idxcode{value_type}}%
\begin{codeblock}
  template <class> struct value_type { };

  template <class T>
  struct value_type<T*>
    : enable_if<is_object<T>@\removed{::value}@, remove_cv_t<T>> { };

  template <class I>
    requires is_array@\added{_v}@<I>@\removed{::value}@
  struct value_type<I> : value_type<decay_t<I>> { };

  template <class I>
  struct value_type<const I> : value_type<decay_t<I>> { };

  template <class T>
    requires requires { typename T::value_type; }
  struct value_type<T>
    : enable_if<is_object@\added{_v}@<typename T::value_type>@\removed{::value}@, typename T::value_type> { };

  template <class T>
    requires requires { typename T::element_type; }
  struct value_type<T>
    : enable_if<
        is_object@\added{_v}@<typename T::element_type>@\removed{::value}@,
        remove_cv_t<typename T::element_type>>
    { };

  template <class T> using value_type_t
    = typename value_type<T>::type;
\end{codeblock}

\pnum
If a type \tcode{I} has an associated value type, then \tcode{value_type<I>::type} shall name the
value type. Otherwise, there shall be no nested type \tcode{type}.

\pnum
The \tcode{value_type} class template may be specialized on user-defined types.

\pnum
When instantiated with a type \tcode{I}
such that \tcode{I::value_type} is valid and denotes a type,
\tcode{value_type<I>::type} names that type, unless it is not an object type~(\cxxref{basic.types}) in which case
\tcode{value_type<I>} shall have no nested type \tcode{type}. \enternote Some legacy output
iterators define a nested type named \tcode{value_type} that is an alias for \tcode{void}. These
types are not \tcode{Readable} and have no associated value types.\exitnote

\pnum
When instantiated with a type \tcode{I}
such that \tcode{I::element_type} is valid and denotes a type,
\tcode{value_type<I>::\brk{}type} names the type \tcode{remove_cv_t<I::element_type>}, unless it is
not an object type~(\cxxref{basic.types}) in which case
\tcode{value_type<I>} shall have no nested type \tcode{type}. \enternote Smart pointers like
\tcode{shared_ptr<int>} are \tcode{Readable} and have an associated value type. But a smart pointer
like \tcode{shared_ptr<void>} is not \tcode{Readable} and has no associated value type.\exitnote

\rSec3[std2.iterator.assoc.types.iterator_category]{\tcode{iterator_category}}

\pnum
\indexlibrary{\idxcode{iterator_category_t}}%
\tcode{iterator_category_t<T>}
is implemented as if:

\indexlibrary{\idxcode{iterator_category}}%
\begin{codeblock}
  template <class> struct iterator_category { };

  template <class T>
  struct iterator_category<T*>
    : enable_if<is_object@\added{_v}@<T>@\removed{::value}@, random_access_iterator_tag> { };

  template <class T>
  struct iterator_category<T const> : iterator_category<T> { };

  template <class T>
    requires requires { typename T::iterator_category; }
  struct iterator_category<T> {
    using type = @\seebelow@;
  };

  template <class T> using iterator_category_t
    = typename iterator_category<T>::type;
\end{codeblock}

\pnum
Users may specialize \tcode{iterator_category} on user-defined types.

\pnum
If
\tcode{T::iterator_category} is valid and denotes a type, then the
type \tcode{iterator_category<T>::type} is computed as follows:
\begin{itemize}
\item If \tcode{T::iterator_category} is the same as or derives from \tcode{std::random_access_iterator_tag},
      \tcode{iter\-ator_category<T>::type} is \tcode{\changed{ranges}{::std2}::random_access_iterator_tag}.
\item Otherwise, if \tcode{T::iterator_category} is the same as or derives from \tcode{std::bidirectional_iterator_tag},
      \tcode{iterator_category<T>::type} is \tcode{\changed{ranges}{::std2}::bidirectional_iterator_tag}.
\item Otherwise, if \tcode{T::iterator_category} is the same as or derives from \tcode{std::forward_iterator_tag},
      \tcode{iterator_category<T>::type} is \tcode{\changed{ranges}{::std2}::forward_iterator_tag}.
\item Otherwise, if \tcode{T::iterator_category} is the same as or derives from \tcode{std::input_iterator_tag},
      \tcode{iterator_category<T>::type} is \tcode{\changed{ranges}{::std2}::input_iterator_tag}.
\item Otherwise, if \tcode{T::iterator_category} is the same as or derives from \tcode{std::output_iterator_tag},
      \tcode{iterator_category<T>} has no nested \tcode{type}.
\item Otherwise, \tcode{iterator_category<T>::type} is \tcode{T::iterator_category}
\end{itemize}

\pnum
\indexlibrary{\idxcode{rvalue_reference_t}}%
\tcode{rvalue_reference_t<T>} is implemented as if:

\begin{itemdecl}
  template <@\placeholder{dereferenceable}@ T>
      requires @\seebelow{ }@using rvalue_reference_t
    = decltype(@\changed{ranges}{::std2}@::iter_move(declval<T&>()));
\end{itemdecl}

\begin{itemdescr}
\pnum
The expression in the \tcode{requires} clause is equivalent to:
\begin{codeblock}
requires(T& t) { { @\changed{ranges}{::std2}@::iter_move(t) } -> auto&&; }
\end{codeblock}
\end{itemdescr}

\rSec2[std2.iterators.readable]{Concept \tcode{Readable}}

\pnum
The \tcode{Readable} concept is satisfied by types that are readable by
applying \tcode{operator*} including pointers, smart pointers, and iterators.

\indexlibrary{\idxcode{Readable}}%
\begin{codeblock}
  template <class In>
  concept @\removed{bool}@ Readable =
    requires {
      typename value_type_t<In>;
      typename reference_t<In>;
      typename rvalue_reference_t<In>;
    } &&
    CommonReference<reference_t<In>&&, value_type_t<In>&> &&
    CommonReference<reference_t<In>&&, rvalue_reference_t<In>&&> &&
    CommonReference<rvalue_reference_t<In>&&, const value_type_t<In>&>;
\end{codeblock}

\rSec2[std2.iterators.writable]{Concept \tcode{Writable}}

\pnum
The \tcode{Writable} concept specifies the requirements for writing a value into an iterator's
referenced object.

\indexlibrary{\idxcode{Writable}}%
\begin{codeblock}
  template <class Out, class T>
  concept @\removed{bool}@ Writable =
    requires(Out&& o, T&& t) {
      *o = std::forward<T>(t); // not required to be equality preserving
      *std::forward<Out>(o) = std::forward<T>(t); // not required to be equality preserving
      const_cast<const reference_t<Out>&&>(*o) =
        std::forward<T>(t); // not required to be equality preserving
      const_cast<const reference_t<Out>&&>(*std::forward<Out>(o)) =
        std::forward<T>(t); // not required to be equality preserving
    };
\end{codeblock}

\pnum
Let \tcode{E} be an an expression such that \tcode{decltype((E))} is \tcode{T}, and let \tcode{o}
be a dereferenceable object of type \tcode{Out}. \tcode{Writable<Out, T>} is satisfied only if

\begin{itemize}
\item If \tcode{Readable<Out> \&\& Same<value_type_t<Out>, decay_t<T>{>}} is satisfied,
then \tcode{*o} after any above assignment is equal
to the value of \tcode{E} before the assignment.
\end{itemize}

\pnum
After evaluating any above assignment expression, \tcode{o} is not required to be dereferenceable.

\pnum
If \tcode{E} is an xvalue~(\cxxref{basic.lval}), the resulting
state of the object it denotes is valid but unspecified~(\cxxref{lib.types.movedfrom}).

\pnum
\enternote
The only valid use of an \tcode{operator*} is on the left side of the assignment statement.
\textit{Assignment through the same value of the writable type happens only once.}
\exitnote

\rSec2[std2.iterators.weaklyincrementable]{Concept \tcode{WeaklyIncrementable}}

\pnum
The \tcode{WeaklyIncrementable} concept specifies the requirements on
types that can be incremented with the pre- and post-increment operators.
The increment operations are not required to be equality-preserving,
nor is the type required to be \tcode{EqualityComparable}.

\indexlibrary{\idxcode{WeaklyIncrementable}}%
\begin{codeblock}
  template <class I>
  concept @\removed{bool}@ WeaklyIncrementable =
    Semiregular<I> &&
    requires(I i) {
      typename difference_type_t<I>;
      requires SignedIntegral<difference_type_t<I>>;
      { ++i } -> Same<I>&; // not required to be equality preserving
      i++; // not required to be equality preserving
    };
\end{codeblock}

\pnum
Let \tcode{i} be an object of type \tcode{I}. When \tcode{i} is in the domain of
both pre- and post-increment, \tcode{i} is said to be \techterm{incrementable}.
\tcode{WeaklyIncrementable<I>} is satisfied only if

\begin{itemize}
\item The expressions \tcode{++i} and \tcode{i++} have the same domain.
\item If \tcode{i} is incrementable, then both \tcode{++i}
  and \tcode{i++} advance \tcode{i} to the next element.
\item If \tcode{i} is incrementable, then \tcode{\&++i} is equal to \tcode{\&i}.
\end{itemize}

\pnum
\enternote For \tcode{WeaklyIncrementable} types, \tcode{a} equals \tcode{b} does not imply that \tcode{++a}
equals \tcode{++b}. (Equality does not guarantee the substitution property or referential
transparency.) Algorithms on weakly incrementable types should never attempt to pass
through the same incrementable value twice. They should be single pass algorithms. These algorithms
can be used with istreams as the source of the input data through the \tcode{istream_iterator} class
template.\exitnote

\rSec2[std2.iterators.incrementable]{Concept \tcode{Incrementable}}

\pnum
The \tcode{Incrementable} concept specifies requirements on types that can be incremented with the pre-
and post-increment operators. The increment operations are required to be equality-preserving,
and the type is required to be \tcode{EqualityComparable}. \enternote This requirement
supersedes the annotations on the increment expressions in the definition of
\tcode{WeaklyIncrementable}. \exitnote

\indexlibrary{\idxcode{Incrementable}}%
\begin{codeblock}
  template <class I>
  concept @\removed{bool}@ Incrementable =
    Regular<I> &&
    WeaklyIncrementable<I> &&
    requires(I i) {
      { i++ } -> Same<I>&&;
    };
\end{codeblock}

\pnum
Let \tcode{a} and \tcode{b} be incrementable objects of type \tcode{I}.
\tcode{Incrementable<I>} is satisfied only if

\begin{itemize}
\item If \tcode{bool(a == b)} then \tcode{bool(a++ == b)}.
\item If \tcode{bool(a == b)} then \tcode{bool((a++, a) == ++b)}.
\end{itemize}

\pnum
\enternote The requirement that \tcode{a} equals \tcode{b} implies \tcode{++a} equals \tcode{++b}
(which is not true for weakly incrementable types) allows the use of multi-pass one-directional
algorithms with types that satisfy \tcode{Increment\-able}.\exitnote

\rSec2[std2.iterators.iterator]{Concept \tcode{Iterator}}

\pnum
The \tcode{Iterator} concept forms
the basis of the iterator concept taxonomy; every iterator satisfies the
\tcode{Iterator} requirements. This
concept specifies operations for dereferencing and incrementing
an iterator. Most algorithms will require additional operations
to compare iterators with sentinels~(\ref{std2.iterators.sentinel}), to
read~(\ref{std2.iterators.input}) or write~(\ref{std2.iterators.output}) values, or
to provide a richer set of iterator movements~(\ref{std2.iterators.forward},
\ref{std2.iterators.bidirectional}, \ref{std2.iterators.random.access}).)

\indexlibrary{\idxcode{Iterator}}%
\begin{codeblock}
  template <class I>
  concept @\removed{bool}@ Iterator =
    requires(I i) {
      { *i } -> auto&&; // Requires: i is dereferenceable
    } &&
    WeaklyIncrementable<I>;
\end{codeblock}

\pnum
\enternote The requirement that the result of dereferencing the iterator is deducible from
\tcode{auto\&\&} means that it cannot be \tcode{void}.\exitnote

\rSec2[std2.iterators.sentinel]{Concept \tcode{Sentinel}}
\pnum
The \tcode{Sentinel} concept
specifies the relationship
between an \tcode{Iterator} type and a \tcode{Semiregular} type whose values
denote a range.

\indexlibrary{\idxcode{Sentinel}}%
\begin{itemdecl}
  template <class S, class I>
  concept @\removed{bool}@ Sentinel =
    Semiregular<S> &&
    Iterator<I> &&
    WeaklyEqualityComparableWith<S, I>;
\end{itemdecl}

\begin{itemdescr}
\pnum
Let \tcode{s} and \tcode{i} be values of type \tcode{S} and
\tcode{I} such that \range{i}{s} denotes a range. Types
\tcode{S} and \tcode{I} satisfy \tcode{Sentinel<S, I>} only if:

\begin{itemize}
\item \tcode{i == s} is well-defined.

\item If \tcode{bool(i != s)} then \tcode{i} is dereferenceable and
      \range{++i}{s} denotes a range.
\end{itemize}
\end{itemdescr}

\pnum
The domain of \tcode{==} can change over time.
Given an iterator \tcode{i} and sentinel \tcode{s} such that \range{i}{s}
denotes a range and \tcode{i != s}, \range{i}{s} is not required to continue to
denote a range after incrementing any iterator equal to \tcode{i}. Consequently,
\tcode{i == s} is no longer required to be well-defined.

\rSec2[std2.iterators.sizedsentinel]{Concept \tcode{SizedSentinel}}
\pnum
The \tcode{SizedSentinel} concept specifies
requirements on an \tcode{Iterator} and a \tcode{Sentinel}
that allow the use of the \tcode{-} operator to compute the distance
between them in constant time.

\indexlibrary{\idxcode{SizedSentinel}}%

\begin{itemdecl}
  template <class S, class I>
  concept @\removed{bool}@ SizedSentinel =
    Sentinel<S, I> &&
    !disable_sized_sentinel<remove_cv_t<S>, remove_cv_t<I>> &&
    requires(const I& i, const S& s) {
      { s - i } -> Same<difference_type_t<I>>&&;
      { i - s } -> Same<difference_type_t<I>>&&;
    };
\end{itemdecl}

\begin{itemdescr}
\pnum
Let \tcode{i} be an iterator of type \tcode{I}, and \tcode{s}
a sentinel of type \tcode{S} such that \range{i}{s} denotes a range.
Let $N$ be the smallest number of applications of \tcode{++i}
necessary to make \tcode{bool(i == s)} be \tcode{true}.
\tcode{SizedSentinel<S, I>} is satisfied only if:

\begin{itemize}
\item If $N$ is representable by \tcode{difference_type_t<I>},
      then \tcode{s - i} is well-defined and equals $N$.

\item If $-N$ is representable by \tcode{difference_type_t<I>},
      then \tcode{i - s} is well-defined and equals $-N$.
\end{itemize}
\end{itemdescr}

\pnum
\enternote \tcode{disable_sized_sentinel} provides a mechanism to
enable use of sentinels and iterators with the library that meet the
syntactic requirements but do not in fact satisfy \tcode{SizedSentinel}.
A program that instantiates a library template that requires
\tcode{SizedSentinel} with an iterator type \tcode{I} and sentinel type
\tcode{S} that meet the syntactic requirements of \tcode{SizedSentinel<S, I>}
but do not satisfy \tcode{SizedSentinel} is ill-formed with no diagnostic required
unless \tcode{disable_sized_sentinel<S, I>} evaluates to
\tcode{true}~(\ref{std2.structure.requirements}). \exitnote

\pnum
\enternote The \tcode{SizedSentinel}
concept is satisfied by pairs of
\tcode{RandomAccessIterator}s~(\ref{std2.iterators.random.access}) and by
counted iterators and their sentinels~(\ref{std2.counted.iterator}).\exitnote

\rSec2[std2.iterators.input]{Concept \tcode{InputIterator}}

\pnum
The \tcode{InputIterator} concept is a refinement of
\tcode{Iterator}~(\ref{std2.iterators.iterator}). It
defines requirements for a type whose referenced values can be read (from the requirement for
\tcode{Readable}~(\ref{std2.iterators.readable})) and which can be both pre- and post-incremented.
\enternote Unlike in ISO/IEC 14882, input iterators are not required to satisfy
\tcode{EqualityComparable}~(\cxxref{concepts.lib.compare.equalitycomparable}).\exitnote

\indexlibrary{\idxcode{InputIterator}}%
\begin{codeblock}
  template <class I>
  concept @\removed{bool}@ InputIterator =
    Iterator<I> &&
    Readable<I> &&
    requires { typename iterator_category_t<I>; } &&
    DerivedFrom<iterator_category_t<I>, input_iterator_tag>;
\end{codeblock}

\rSec2[std2.iterators.output]{Concept \tcode{OutputIterator}}

\pnum
The \tcode{OutputIterator} concept is a refinement of
\tcode{Iterator}~(\ref{std2.iterators.iterator}). It defines requirements for a type that
can be used to write values (from the requirement for
\tcode{Writable}~(\ref{std2.iterators.writable})) and which can be both pre- and post-incremented.
However, output iterators are not required to
satisfy \tcode{EqualityComparable}.

\indexlibrary{\idxcode{OutputIterator}}%
\begin{codeblock}
  template <class I, class T>
  concept @\removed{bool}@ OutputIterator =
    Iterator<I> &&
    Writable<I, T> &&
    requires(I i, T&& t) {
      *i++ = std::forward<T>(t); // not required to be equality preserving
    };
\end{codeblock}

\pnum
Let \tcode{E} be an expression such that \tcode{decltype((E))} is \tcode{T}, and let \tcode{i} be a
dereferenceable object of type \tcode{I}. \tcode{OutputIterator<I, T>} is satisfied only if
\tcode{*i++ = E;} has effects equivalent to:
\begin{codeblock}
  *i = E;
  ++i;
\end{codeblock}

\pnum
\enternote
Algorithms on output iterators should never attempt to pass through the same iterator twice.
They should be
\term{single pass}
algorithms.
Algorithms that take output iterators can be used with ostreams as the destination
for placing data through the
\tcode{ostream_iterator}
class as well as with insert iterators and insert pointers.
\exitnote

\rSec2[std2.iterators.forward]{Concept \tcode{ForwardIterator}}

\pnum
The \tcode{ForwardIterator} concept refines \tcode{InputIterator}~(\ref{std2.iterators.input}),
adding equality comparison and the multi-pass guarantee, specified below.

\indexlibrary{\idxcode{ForwardIterator}}%
\begin{codeblock}
  template <class I>
  concept @\removed{bool}@ ForwardIterator =
    InputIterator<I> &&
    DerivedFrom<iterator_category_t<I>, forward_iterator_tag> &&
    Incrementable<I> &&
    Sentinel<I, I>;
\end{codeblock}

\pnum
The domain of \tcode{==} for forward iterators is that of iterators over the same
underlying sequence. However, value-initialized iterators of the same type
may be compared and shall compare equal to other value-initialized iterators of the same type.
\enternote Value-initialized iterators behave as if they refer past the end of
the same empty sequence. \exitnote

\pnum
Pointers and references obtained from a forward iterator into a range \range{i}{s}
shall remain valid while \range{i}{s} continues to denote a range.

\pnum
Two dereferenceable iterators \tcode{a} and \tcode{b} of type \tcode{X} offer the
\defn{multi-pass guarantee} if:

\begin{itemize}
\item \tcode{a == b} implies \tcode{++a == ++b} and
\item The expression
\tcode{([](X x)\{++x;\}(a), *a)} is equivalent to the expression \tcode{*a}.
\end{itemize}

\pnum
\enternote
The requirement that
\tcode{a == b}
implies
\tcode{++a == ++b}
(which is not true for weaker iterators)
and the removal of the restrictions on the number of assignments through
a mutable iterator
(which applies to output iterators)
allow the use of multi-pass one-directional algorithms with forward iterators.
\exitnote

\rSec2[std2.iterators.bidirectional]{Concept \tcode{BidirectionalIterator}}

\pnum
The \tcode{BidirectionalIterator} concept refines \tcode{ForwardIterator}~(\ref{std2.iterators.forward}),
and adds the ability to move an iterator backward as well as forward.

\indexlibrary{\idxcode{BidirectionalIterator}}%
\begin{codeblock}
  template <class I>
  concept @\removed{bool}@ BidirectionalIterator =
    ForwardIterator<I> &&
    DerivedFrom<iterator_category_t<I>, bidirectional_iterator_tag> &&
    requires(I i) {
      { --i } -> Same<I>&;
      { i-- } -> Same<I>&&;
    };
\end{codeblock}

\pnum
A bidirectional iterator \tcode{r} is decrementable if and only if there exists some \tcode{s} such that
\tcode{++s == r}. Decrementable iterators \tcode{r} shall be in the domain of the expressions
\tcode{\dcr{}r} and \tcode{r\dcr{}}.

\pnum
Let \tcode{a} and \tcode{b} be decrementable objects of type \tcode{I}.
\tcode{BidirectionalIterator<I>} is satisfied only if:

\begin{itemize}
\item \tcode{\&\dcr{}a == \&a}.
\item If \tcode{bool(a == b)}, then \tcode{bool(a\dcr{} == b)}.
\item If \tcode{bool(a == b)}, then after evaluating both \tcode{a\dcr} and \tcode{\dcr{}b},
\tcode{bool(a == b)} still holds.
\item If \tcode{a} is incrementable and \tcode{bool(a == b)}, then
      \tcode{bool(\dcr{}(++a) == b)}.
\item If \tcode{bool(a == b)}, then \tcode{bool(++(\dcr{}a) == b)}.
\end{itemize}

\rSec2[std2.iterators.random.access]{Concept \tcode{RandomAccessIterator}}

\pnum
The \tcode{RandomAccessIterator} concept refines \tcode{BidirectionalIterator}~(\ref{std2.iterators.bidirectional})
and adds support for constant-time advancement with \tcode{+=}, \tcode{+},  \tcode{-=}, and \tcode{-}, and the
computation of distance in constant time with \tcode{-}. Random access iterators also support array
notation via subscripting.

\indexlibrary{\idxcode{RandomAccessIterator}}%
\begin{codeblock}
  template <class I>
  concept @\removed{bool}@ RandomAccessIterator =
    BidirectionalIterator<I> &&
    DerivedFrom<iterator_category_t<I>, random_access_iterator_tag> &&
    StrictTotallyOrdered<I> &&
    SizedSentinel<I, I> &&
    requires(I i, const I j, const difference_type_t<I> n) {
      { i += n } -> Same<I>&;
      { j + n }  -> Same<I>&&;
      { n + j }  -> Same<I>&&;
      { i -= n } -> Same<I>&;
      { j - n }  -> Same<I>&&;
      j[n];
      requires Same<decltype(j[n]), reference_t<I>>;
    };
\end{codeblock}

\pnum
Let \tcode{a} and \tcode{b} be valid iterators of type \tcode{I} such that \tcode{b} is reachable
from \tcode{a}. Let \tcode{n} be the smallest value of type
\tcode{difference_type_t<I>} such that after
\tcode{n} applications of \tcode{++a}, then \tcode{bool(a == b)}.
\tcode{Random\-Access\-Iterator<\brk{}I>} is satisfied only if:

\begin{itemize}
\item \tcode{(a += n)} is equal to \tcode{b}.
\item \tcode{\&(a += n)} is equal to \tcode{\&a}.
\item \tcode{(a + n)} is equal to \tcode{(a += n)}.
\item For any two positive integers \tcode{x} and \tcode{y}, if \tcode{a + (x + y)} is valid, then
\tcode{a + (x + y)} is equal to \tcode{(a + x) + y}.
\item \tcode{a + 0} is equal to \tcode{a}.
\item If \tcode{(a + (n - 1))} is valid, then \tcode{a + n} is equal to \tcode{++(a + (n - 1))}.
\item \tcode{(b += -n)} is equal to \tcode{a}.
\item \tcode{(b -= n)} is equal to \tcode{a}.
\item \tcode{\&(b -= n)} is equal to \tcode{\&b}.
\item \tcode{(b - n)} is equal to \tcode{(b -= n)}.
\item If \tcode{b} is dereferenceable, then \tcode{a[n]} is valid and is equal to \tcode{*b}.
\end{itemize}

\rSec1[std2.indirectcallable]{Indirect callable requirements}

\rSec2[std2.indirectcallable.general]{General}

\pnum
There are several concepts that group requirements of algorithms that take callable
objects~(\cxxref{func.require}) as arguments.

\rSec2[std2.indirectcallable.indirectinvocable]{Indirect callables}

\pnum
The indirect callable concepts are used to constrain those algorithms that accept
callable objects~(\cxxref{func.def}) as arguments.

\indexlibrary{\idxcode{indirect_result_of}}%
\indexlibrary{\idxcode{IndirectUnaryInvocable}}%
\indexlibrary{\idxcode{IndirectRegularUnaryInvocable}}%
\indexlibrary{\idxcode{IndirectUnaryPredicate}}%
\indexlibrary{\idxcode{IndirectRelation}}%
\indexlibrary{\idxcode{IndirectStrictWeakOrder}}%
\begin{codeblock}
  template <class F, class I>
  concept @\removed{bool}@ IndirectUnaryInvocable =
    Readable<I> &&
    CopyConstructible<F> &&
    Invocable<F&, value_type_t<I>&> &&
    Invocable<F&, reference_t<I>> &&
    Invocable<F&, iter_common_reference_t<I>> &&
    CommonReference<
      result_of_t<F&(value_type_t<I>&)>,
      result_of_t<F&(reference_t<I>&&)>>;

  template <class F, class I>
  concept @\removed{bool}@ IndirectRegularUnaryInvocable =
    Readable<I> &&
    CopyConstructible<F> &&
    RegularInvocable<F&, value_type_t<I>&> &&
    RegularInvocable<F&, reference_t<I>> &&
    RegularInvocable<F&, iter_common_reference_t<I>> &&
    CommonReference<
      result_of_t<F&(value_type_t<I>&)>,
      result_of_t<F&(reference_t<I>&&)>>;

  template <class F, class I>
  concept @\removed{bool}@ IndirectUnaryPredicate =
    Readable<I> &&
    CopyConstructible<F> &&
    Predicate<F&, value_type_t<I>&> &&
    Predicate<F&, reference_t<I>> &&
    Predicate<F&, iter_common_reference_t<I>>;

  template <class F, class I1, class I2 = I1>
  concept @\removed{bool}@ IndirectRelation =
    Readable<I1> && Readable<I2> &&
    CopyConstructible<F> &&
    Relation<F&, value_type_t<I1>&, value_type_t<I2>&> &&
    Relation<F&, value_type_t<I1>&, reference_t<I2>> &&
    Relation<F&, reference_t<I1>, value_type_t<I2>&> &&
    Relation<F&, reference_t<I1>, reference_t<I2>> &&
    Relation<F&, iter_common_reference_t<I1>, iter_common_reference_t<I2>>;

  template <class F, class I1, class I2 = I1>
  concept @\removed{bool}@ IndirectStrictWeakOrder =
    Readable<I1> && Readable<I2> &&
    CopyConstructible<F> &&
    StrictWeakOrder<F&, value_type_t<I1>&, value_type_t<I2>&> &&
    StrictWeakOrder<F&, value_type_t<I1>&, reference_t<I2>> &&
    StrictWeakOrder<F&, reference_t<I1>, value_type_t<I2>&> &&
    StrictWeakOrder<F&, reference_t<I1>, reference_t<I2>> &&
    StrictWeakOrder<F&, iter_common_reference_t<I1>, iter_common_reference_t<I2>>;

  template <class> struct indirect_result_of { };

  template <class F, class... Is>
    requires Invocable<F, reference_t<Is>...>
  struct indirect_result_of<F(Is...)> :
    result_of<F(reference_t<Is>&&...)> { };
\end{codeblock}

\rSec2[std2.projected]{Class template \tcode{projected}}

\pnum
The \tcode{projected} class template is intended for use when specifying the constraints of
algorithms that accept callable objects and projections~(\ref{std2.defns.projection}). It bundles a \tcode{Readable} type
\tcode{I} and a function \tcode{Proj} into a new \tcode{Readable} type whose
\tcode{reference} type is the result of applying \tcode{Proj} to the
\tcode{reference_t} of \tcode{I}.

\indexlibrary{\idxcode{projected}}%
\begin{codeblock}
  template <Readable I, IndirectRegularUnaryInvocable<I> Proj>
  struct projected {
    using value_type = remove_cv_t<remove_reference_t<indirect_result_of_t<Proj&(I)>>>;
    indirect_result_of_t<Proj&(I)> operator*() const;
  };

  template <WeaklyIncrementable I, class Proj>
  struct difference_type<projected<I, Proj>> {
    using type = difference_type_t<I>;
  };
\end{codeblock}

\pnum
\enternote \tcode{projected} is only used to ease constraints specification. Its
member function need not be defined.\exitnote

\rSec1[std2.commonalgoreq]{Common algorithm requirements}

\rSec2[std2.commonalgoreq.general]{General}

\pnum
There are several additional iterator concepts that are commonly applied to families of algorithms.
These group together iterator requirements of algorithm families. There are three relational
concepts that specify how element values are transferred between \tcode{Readable} and \tcode{Writable} types:
\tcode{Indirectly\-Movable}, \tcode{Indir\-ect\-ly\-Copy\-able}, and \tcode{Indirectly\-Swappable}. There are three relational concepts
for rearrangements: \tcode{Permut\-able}, \tcode{Mergeable}, and \tcode{Sortable}.
There is one relational concept for comparing values from different sequences: \tcode{IndirectlyComparable}.

\pnum
\enternote The \tcode{equal_to<>} and \tcode{less<>}~(\ref{std2.comparisons}) function types used in the
concepts below impose additional constraints on their arguments beyond those that appear explicitly in the
concepts' bodies. \tcode{equal_to<>} requires its arguments satisfy \tcode{EqualityComparableWith}~(\cxxref{concepts.lib.compare.equalitycomparable}),
and \tcode{less<>} requires its arguments satisfy \tcode{StrictTotallyOrderedWith}~(\cxxref{concepts.lib.compare.stricttotallyordered}).\exitnote

\rSec2[std2.commonalgoreq.indirectlymovable]{Concept \tcode{IndirectlyMovable}}

\pnum
The \tcode{IndirectlyMovable} concept specifies the relationship between a \tcode{Readable}
type and a \tcode{Writable} type between which values may be moved.

\indexlibrary{\idxcode{IndirectlyMovable}}%
\begin{codeblock}
  template <class In, class Out>
  concept @\removed{bool}@ IndirectlyMovable =
    Readable<In> &&
    Writable<Out, rvalue_reference_t<In>>;
\end{codeblock}

\pnum
The \tcode{IndirectlyMovableStorable} concept augments \tcode{IndirectlyMovable} with additional
requirements enabling the transfer to be performed through an intermediate object of the
\tcode{Readable} type's value type.

\indexlibrary{\idxcode{IndirectlyMovableStorable}}%
\begin{codeblock}
  template <class In, class Out>
  concept @\removed{bool}@ IndirectlyMovableStorable =
    IndirectlyMovable<In, Out> &&
    Writable<Out, value_type_t<In>> &&
    Movable<value_type_t<In>> &&
    Constructible<value_type_t<In>, rvalue_reference_t<In>> &&
    Assignable<value_type_t<In>&, rvalue_reference_t<In>>;
\end{codeblock}

\rSec2[std2.commonalgoreq.indirectlycopyable]{Concept \tcode{IndirectlyCopyable}}

\pnum
The \tcode{IndirectlyCopyable} concept specifies the relationship between a \tcode{Readable}
type and a \tcode{Writable} type between which values may be copied.

\indexlibrary{\idxcode{IndirectlyCopyable}}%
\begin{codeblock}
  template <class In, class Out>
  concept @\removed{bool}@ IndirectlyCopyable =
    Readable<In> &&
    Writable<Out, reference_t<In>>;
\end{codeblock}

\pnum
The \tcode{IndirectlyCopyableStorable} concept augments \tcode{IndirectlyCopyable} with additional
requirements enabling the transfer to be performed through an intermediate object of the
\tcode{Readable} type's value type. It also requires the capability to make copies of values.

\indexlibrary{\idxcode{IndirectlyCopyableStorable}}%
\begin{codeblock}
  template <class In, class Out>
  concept @\removed{bool}@ IndirectlyCopyableStorable =
    IndirectlyCopyable<In, Out> &&
    Writable<Out, const value_type_t<In>&> &&
    Copyable<value_type_t<In>> &&
    Constructible<value_type_t<In>, reference_t<In>> &&
    Assignable<value_type_t<In>&, reference_t<In>>;
\end{codeblock}

\rSec2[std2.commonalgoreq.indirectlyswappable]{Concept \tcode{IndirectlySwappable}}

\pnum
The \tcode{IndirectlySwappable} concept specifies a swappable relationship between the
values referenced by two \tcode{Readable} types.

\indexlibrary{\idxcode{IndirectlySwappable}}%
\begin{codeblock}
  template <class I1, class I2 = I1>
  concept @\removed{bool}@ IndirectlySwappable =
    Readable<I1> && Readable<I2> &&
    requires(I1&& i1, I2&& i2) {
      @\changed{ranges}{::std2}@::iter_swap(std::forward<I1>(i1), std::forward<I2>(i2));
      @\changed{ranges}{::std2}@::iter_swap(std::forward<I2>(i2), std::forward<I1>(i1));
      @\changed{ranges}{::std2}@::iter_swap(std::forward<I1>(i1), std::forward<I1>(i1));
      @\changed{ranges}{::std2}@::iter_swap(std::forward<I2>(i2), std::forward<I2>(i2));
    };
\end{codeblock}

\pnum
Given an object \tcode{i1} of type \tcode{I1} and an object \tcode{i2} of
type \tcode{I2}, \tcode{IndirectlySwappable<I1, I2>} is satisfied if after
\tcode{\changed{ranges}{::std2}::iter_swap(i1, i2)}, the value of \tcode{*i1} is equal to the
value of \tcode{*i2} before the call, and \textit{vice versa}.

\rSec2[std2.commonalgoreq.indirectlycomparable]{Concept \tcode{IndirectlyComparable}}

\pnum
The \tcode{IndirectlyComparable} concept specifies the common requirements of algorithms that
compare values from two different sequences.

\indexlibrary{\idxcode{IndirectlyComparable}}%
\begin{codeblock}
  template <class I1, class I2, class R = equal_to<>, class P1 = identity,
    class P2 = identity>
  concept @\removed{bool}@ IndirectlyComparable =
    IndirectRelation<R, projected<I1, P1>, projected<I2, P2>>;
\end{codeblock}

\rSec2[std2.commonalgoreq.permutable]{Concept \tcode{Permutable}}

\pnum
The \tcode{Permutable} concept specifies the common requirements of algorithms that reorder
elements in place by moving or swapping them.

\indexlibrary{\idxcode{Permutable}}%
\begin{codeblock}
  template <class I>
  concept @\removed{bool}@ Permutable =
    ForwardIterator<I> &&
    IndirectlyMovableStorable<I, I> &&
    IndirectlySwappable<I, I>;
\end{codeblock}

\rSec2[std2.commonalgoreq.mergeable]{Concept \tcode{Mergeable}}

\pnum
The \tcode{Mergeable} concept specifies the requirements of
algorithms that merge sorted sequences into an output sequence by copying elements.

\indexlibrary{\idxcode{Mergeable}}%
\begin{codeblock}
  template <class I1, class I2, class Out,
      class R = less<>, class P1 = identity, class P2 = identity>
  concept @\removed{bool}@ Mergeable =
    InputIterator<I1> &&
    InputIterator<I2> &&
    WeaklyIncrementable<Out> &&
    IndirectlyCopyable<I1, Out> &&
    IndirectlyCopyable<I2, Out> &&
    IndirectStrictWeakOrder<R, projected<I1, P1>, projected<I2, P2>>;
\end{codeblock}

\rSec2[std2.commonalgoreq.sortable]{Concept \tcode{Sortable}}

\pnum
The \tcode{Sortable} concept specifies the common requirements of algorithms that permute
sequences into ordered sequences (e.g., \tcode{sort}).

\indexlibrary{\idxcode{Sortable}}%
\begin{codeblock}
  template <class I, class R = less<>, class P = identity>
  concept @\removed{bool}@ Sortable =
    Permutable<I> &&
    IndirectStrictWeakOrder<R, projected<I, P>>;
\end{codeblock}

\rSec1[std2.iterator.primitives]{Iterator primitives}

\pnum
To simplify the task of defining iterators, the library provides
several classes and functions:

\rSec2[std2.iterator.traits]{Iterator traits}

\pnum
For the sake of backwards compatibility, this document specifies the existence of an \tcode{iterator_traits}
alias that collects an iterator's associated types. It is defined as if:

\indexlibrary{\idxcode{iterator_traits}}%
\begin{codeblock}
  template <InputIterator I> struct @\xname{pointer_type}@ {        // \expos
    using type = add_pointer_t<reference_t<I>>;
  };
  template <InputIterator I>
    requires requires(I i) { { i.operator->() } -> auto&&; }
  struct @\xname{pointer_type}@<I> {                                    // \expos
    using type = decltype(declval<I>().operator->());
  };
  template <class> struct @\xname{iterator_traits}@ { };                // \expos
  template <Iterator I> struct @\xname{iterator_traits}@<I> {
    using difference_type = difference_type_t<I>;
    using value_type = void;
    using reference = void;
    using pointer = void;
    using iterator_category = output_iterator_tag;
  };
  template <InputIterator I> struct @\xname{iterator_traits}@<I> {  // \expos
    using difference_type = difference_type_t<I>;
    using value_type = value_type_t<I>;
    using reference = reference_t<I>;
    using pointer = typename @\xname{pointer_type}@<I>::type;
    using iterator_category = iterator_category_t<I>;
  };
  template <class I>
    using iterator_traits = @\xname{iterator_traits}@<I>;
\end{codeblock}

\pnum
\enternote
\tcode{iterator_traits} is an alias template
to prevent user code from specializing it.
\exitnote

\pnum
\enterexample
To implement a generic
\tcode{reverse}
function, a \Cpp program can do the following:

\begin{codeblock}
template <BidirectionalIterator I>
void reverse(I first, I last) {
  difference_type_t<I> n = distance(first, last);
  --n;
  while(n > 0) {
    value_type_t<I> tmp = *first;
    *first++ = *--last;
    *last = tmp;
    n -= 2;
  }
}
\end{codeblock}
\exitexample

\rSec2[std2.iterator.stdtraits]{Standard iterator traits}

\pnum
To facilitate interoperability between new code using iterators conforming to this document
and older code using iterators that conform to the iterator
requirements specified in ISO/IEC 14882, three specializations of \tcode{std::iterator_traits}
are provided to map the newer iterator categories and associated types to the older ones.

\begin{codeblock}
namespace std {
  template <@\changed{experimental::ranges}{::std2}@::Iterator Out>
  struct iterator_traits<Out> {
    using difference_type   = @\changed{experimental::ranges}{::std2}@::difference_type_t<Out>;
    using value_type        = @\seebelow@;
    using reference         = @\seebelow@;
    using pointer           = @\seebelow@;
    using iterator_category = std::output_iterator_tag;
  };
\end{codeblock}

\pnum
The nested type \tcode{value_type} is computed as follows:
\begin{itemize}
\item If
      \tcode{Out::value_type} is valid and denotes a type, then
      \tcode{std::iterator_traits<Out>::value_type} is \tcode{Out::value_type}.
\item Otherwise, \tcode{std::iterator_traits<Out>::value_type} is \tcode{void}.
\end{itemize}

\pnum
The nested type \tcode{reference} is computed as follows:
\begin{itemize}
\item If
      \tcode{Out::reference} is valid and denotes a type, then
      \tcode{std::iterator_traits<Out>::reference} is \tcode{Out::\brk{}reference}.
\item Otherwise, \tcode{std::iterator_traits<Out>::reference} is \tcode{void}.
\end{itemize}

\pnum
The nested type \tcode{pointer} is computed as follows:
\begin{itemize}
\item If
      \tcode{Out::pointer} is valid and denotes a type, then
      \tcode{std::iterator_traits<Out>::pointer} is \tcode{Out::\brk{}pointer}.
\item Otherwise, \tcode{std::iterator_traits<Out>::pointer} is \tcode{void}.
\end{itemize}

\begin{codeblock}
  template <@\changed{experimental::ranges}{::std2}@::InputIterator In>
  struct iterator_traits<In> { };

  template <@\changed{experimental::ranges}{::std2}@::InputIterator In>
    requires @\changed{experimental::ranges}{::std2}@::Sentinel<In, In>
  struct iterator_traits<In> {
    using difference_type   = @\changed{experimental::ranges}{::std2}@::difference_type_t<In>;
    using value_type        = @\changed{experimental::ranges}{::std2}@::value_type_t<In>;
    using reference         = @\seebelow@;
    using pointer           = @\seebelow@;
    using iterator_category = @\seebelow@;
  };
}
\end{codeblock}

\pnum
The nested type \tcode{reference} is computed as follows:
\begin{itemize}
\item If
      \tcode{In::reference} is valid and denotes a type, then
      \tcode{std::iterator_traits<In>::reference} is \tcode{In::reference}.
\item Otherwise, \tcode{std::iterator_traits<In>::reference} is
      \changed{\tcode{experimental::\-ranges}}{\tcode{::std2}}\tcode{::\-\-reference_t<In>}.
\end{itemize}

\pnum
The nested type \tcode{pointer} is computed as follows:
\begin{itemize}
\item If
      \tcode{In::pointer} is valid and denotes a type, then
      \tcode{std::iterator_traits<In>::pointer} is \tcode{In::\brk{}pointer}.
\item Otherwise, \tcode{std::iterator_traits<In>::pointer} is
      \tcode{\changed{experimental::ranges}{::std2}::\brk{}iterator_\-traits<\brk{}In\brk{}>::\brk{}pointer}.
\end{itemize}

\pnum
Let type \tcode{C} be \tcode{\changed{experimental::ranges}{::std2}::}\tcode{iterator_category_t}\tcode{<In>}.
The nested type \tcode{std::\brk{}iterator_traits<In>::\brk{}iterator_category} is computed as
follows:
\begin{itemize}
\item If \tcode{C} is the same as or inherits from \tcode{std::input_iterator_tag} or
      \tcode{std::output_iterator_tag}, \tcode{std::\brk{}iterator_traits<In>::\brk{}iterator_category}
      is \tcode{C}.
\item Otherwise, if \tcode{\changed{experimental::ranges}{::std2}::reference_t<In>} is not a reference type,
      \tcode{std::\brk{}iterator_traits<In>::\brk{}iterator_category} is \tcode{std::input_iterator_tag}.
\item Otherwise, if \tcode{C} is the same as or inherits from \changed{\tcode{experimental::\brk{}ranges}}{\tcode{::std2}}\tcode{::\brk{}random_access_iter\-at\-or_tag},
      \tcode{std::\brk{}iterator_traits<In>::\brk{}iterator_category} is \tcode{std::random_access_iterator_tag}.
\item Otherwise, if \tcode{C} is the same as or inherits from \changed{\tcode{experimental::\brk{}ranges}}{\tcode{::std2}}\tcode{::\brk{}bidirectional_iter\-at\-or_tag},
      \tcode{std::\brk{}iterator_traits<In>::\brk{}iterator_category} is \tcode{std::bidirectional_iterator_tag}.
\item Otherwise, if \tcode{C} is the same as or inherits from \changed{\tcode{experimental::\brk{}ranges}}{\tcode{::std2}}\tcode{::\brk{}forward_iter\-at\-or_tag},
      \tcode{std::\brk{}iterator_traits<In>::\brk{}iterator_category} is \tcode{std::forward_iterator_tag}.
\item Otherwise, \tcode{std::iterator_traits<In>::iterator_category} is \tcode{std::input_iterator_tag}.
\end{itemize}

\pnum
\enternote Some implementations may find it necessary to add additional constraints to
these partial specializations to prevent them from being considered for types that
conform to the iterator requirements specified in ISO/IEC 14882.\exitnote

\rSec2[std2.iterator.tags]{Standard iterator tags}

\pnum
\indexlibrary{\idxcode{output_iterator_tag}}%
\indexlibrary{\idxcode{input_iterator_tag}}%
\indexlibrary{\idxcode{forward_iterator_tag}}%
\indexlibrary{\idxcode{bidirectional_iterator_tag}}%
\indexlibrary{\idxcode{random_access_iterator_tag}}%
It is often desirable for a
function template specialization
to find out what is the most specific category of its iterator
argument, so that the function can select the most efficient algorithm at compile time.
To facilitate this, the
library introduces
\techterm{category tag}
classes which can be used as compile time tags for algorithm selection.
\enternote The preferred way to dispatch to more specialized algorithm implementations is
with concept-based overloading.\exitnote
The category tags are:
\tcode{input_iterator_tag},
\tcode{output_iterator_tag},
\tcode{forward_iterator_tag},
\tcode{bidirectional_iterator_tag}
and
\tcode{random_access_iterator_tag}.
For every input iterator of type
\tcode{I},
\tcode{it\-er\-a\-tor_\-ca\-te\-go\-ry_t<I>}
shall be defined to be the most specific category tag that describes the
iterator's behavior.

\begin{codeblock}
namespace @\changed{std \{ namespace experimental \{ namespace ranges}{std2}@ { inline namespace v1 {
  struct output_iterator_tag { };
  struct input_iterator_tag { };
  struct forward_iterator_tag : input_iterator_tag { };
  struct bidirectional_iterator_tag : forward_iterator_tag { };
  struct random_access_iterator_tag : bidirectional_iterator_tag { };
}}@\removed{\}\}}@
\end{codeblock}

\pnum
\enternote
The \tcode{output_iterator_tag} is provided for the sake of backward compatibility.
\exitnote

\pnum
\indexlibrary{\idxcode{empty}}%
\indexlibrary{\idxcode{input_iterator_tag}}%
\indexlibrary{\idxcode{output_iterator_tag}}%
\indexlibrary{\idxcode{forward_iterator_tag}}%
\indexlibrary{\idxcode{bidirectional_iterator_tag}}%
\indexlibrary{\idxcode{random_access_iterator_tag}}%
\enterexample
For a program-defined iterator
\tcode{BinaryTreeIterator},
it could be included
into the bidirectional iterator category by specializing the
\tcode{difference_type}, \tcode{value_type}, and
\tcode{iterator_category} templates:

\begin{codeblock}
template <class T> struct difference_type<BinaryTreeIterator<T>> {
  using type = ptrdiff_t;
};
template <class T> struct value_type<BinaryTreeIterator<T>> {
  using type = T;
};
template <class T> struct iterator_category<BinaryTreeIterator<T>> {
  using type = bidirectional_iterator_tag;
};
\end{codeblock}
\exitexample

\rSec2[std2.iterator.operations]{Iterator operations}

\pnum
Since only types that satisfy
\tcode{RandomAccessIterator} provide the \tcode{+} operator, and
types that satisfy \tcode{Sized\-Sent\-inel} provide the \tcode{-}
operator, the library provides customization point objects~(\cxxref{customization.point.object})
\tcode{advance}, \tcode{dist\-ance}, \tcode{next}, and \tcode{prev}.
These
customization point objects
use
\tcode{+}
and
\tcode{-}
for random access iterators and ranges that satisfy \tcode{SizedSentinel} (and are, therefore, constant
time for them); for output, input, forward and bidirectional iterators they use
\tcode{++}
to provide linear time
implementations.

\indexlibrary{\idxcode{advance}}%
\pnum
The name \tcode{advance} denotes a
customization point object~(\cxxref{customization.point.object}).
It has the following function call operators:

\begin{itemdecl}
template <Iterator I>
  constexpr void operator()(I& i, difference_type_t<I> n) const;
\end{itemdecl}

\begin{itemdescr}
\pnum
\requires
\tcode{n}
shall be negative only for bidirectional iterators.

\pnum
\effects
For random access iterators, equivalent to \tcode{i += n}.
Otherwise, increments (or decrements for negative
\tcode{n})
iterator
\tcode{i}
by
\tcode{n}.
\end{itemdescr}

\begin{itemdecl}
template <Iterator I, Sentinel<I> S>
  constexpr void operator()(I& i, S bound) const;
\end{itemdecl}

\begin{itemdescr}
\pnum
\requires
If
\tcode{Assignable<I\&, S>} is not satisfied, \range{i}{bound}
shall denote a range.

\pnum
\effects
\begin{itemize}
\item If \tcode{Assignable<I\&, S>} is satisfied,
      equivalent to \tcode{i = std::move(bound)}.

\item Otherwise, if
      \tcode{SizedSentinel<S, I>} is satisfied, equivalent to \tcode{advance(i, bound - i)}.

\item Otherwise, increments \tcode{i} until \tcode{i == bound}.
\end{itemize}
\end{itemdescr}

\begin{itemdecl}
template <Iterator I, Sentinel<I> S>
  constexpr difference_type_t<I> operator()(I& i, difference_type_t<I> n, S bound) const;
\end{itemdecl}

\begin{itemdescr}
\pnum
\requires
If \tcode{n > 0}, \range{i}{bound} shall denote a range. If
\tcode{n == 0}, \range{i}{bound} or \range{bound}{i} shall denote a range. If \tcode{n < 0},
\range{bound}{i} shall denote a range and \tcode{(BidirectionalIterator<I> \&\& Same<I, S>)}
shall be satisfied.

\pnum
\effects
\begin{itemize}
\item If \tcode{SizedSentinel<S, I>} is satisfied:
      \begin{itemize}
      \item If \brk{}$|\tcode{n}| >= |\tcode{bound - i}|$, equivalent to \tcode{advance(i, bound)}.

      \item Otherwise, equivalent to \tcode{advance(i, n)}.
      \end{itemize}

\item Otherwise, increments (or decrements for negative \tcode{n})
      iterator \tcode{i} either \tcode{n} times or until \tcode{i == bound},
      whichever comes first.
\end{itemize}

\pnum
\returns
\tcode{n - $M$}, where $M$ is the distance from the starting position of
\tcode{i} to the ending position.
\end{itemdescr}

\indexlibrary{\idxcode{distance}}%
\pnum
The name \tcode{distance} denotes a customization point object.
It has the following function call operators:

\begin{itemdecl}
template <Iterator I, Sentinel<I> S>
  constexpr difference_type_t<I> operator()(I first, S last) const;
\end{itemdecl}

\begin{itemdescr}
\pnum
\requires
\range{first}{last} shall denote a range, or \tcode{(Same<S, I> \&\& SizedSentinel<S, I>)} shall be
satisfied and \range{last}{first} shall denote a range.

\pnum
\effects
If \tcode{SizedSentinel<S, I>} is satisfied, returns \tcode{(last - first)}; otherwise,
returns the number of increments needed to get from
\tcode{first}
to
\tcode{last}.
\end{itemdescr}

\begin{itemdecl}
template <Range R>
  constexpr difference_type_t<iterator_t<R>> operator()(R&& r) const;
\end{itemdecl}

\begin{itemdescr}
\effects
Equivalent to:
\tcode{return distance(\changed{ranges}{::std2}::begin(r), \changed{ranges}{::std2}::end(r));}~(\ref{std2.range.access})

\remarks
Instantiations of this member function template may be ill-formed if the
declarations in \tcode{<\changed{experimental/ranges}{std2}/range>} are not in scope at the
point of instantiation~(\cxxref{temp.point}).
\end{itemdescr}


\begin{itemdecl}
template <SizedRange R>
  constexpr difference_type_t<iterator_t<R>> operator()(R&& r) const;
\end{itemdecl}

\begin{itemdescr}
\pnum
\effects
Equivalent to: \tcode{return \changed{ranges}{::std2}::size(r);}~(\ref{std2.range.primitives.size})

\pnum
\remarks
Instantiations of this member function template may be ill-formed if the
declarations in \tcode{<\changed{experimental/ranges}{std2}/range>} are not in scope at the
point of instantiation~(\cxxref{temp.point}).
\end{itemdescr}

\indexlibrary{\idxcode{next}}%
\pnum
The name \tcode{next} denotes a customization point object. It has the following function call operators:

\begin{itemdecl}
template <Iterator I>
  constexpr I operator()(I x) const;
\end{itemdecl}

\begin{itemdescr}
\pnum
\effects Equivalent to: \tcode{++x; return x;}
\end{itemdescr}

\begin{itemdecl}
template <Iterator I>
  constexpr I operator()(I x, difference_type_t<I> n) const;
\end{itemdecl}

\begin{itemdescr}
\pnum
\effects Equivalent to: \tcode{advance(x, n); return x;}
\end{itemdescr}

\begin{itemdecl}
template <Iterator I, Sentinel<I> S>
  constexpr I operator()(I x, S bound) const;
\end{itemdecl}

\begin{itemdescr}
\pnum
\effects Equivalent to: \tcode{advance(x, bound); return x;}
\end{itemdescr}

\begin{itemdecl}
template <Iterator I, Sentinel<I> S>
  constexpr I operator()(I x, difference_type_t<I> n, S bound) const;
\end{itemdecl}

\begin{itemdescr}
\pnum
\effects Equivalent to: \tcode{advance(x, n, bound); return x;}
\end{itemdescr}

\indexlibrary{\idxcode{prev}}%
\pnum
The name \tcode{prev} denotes a customization point object. It has the following function call operators:

\begin{itemdecl}
template <BidirectionalIterator I>
  constexpr I operator()(I x) const;
\end{itemdecl}

\begin{itemdescr}
\pnum
\effects Equivalent to: \tcode{-{-}x; return x;}
\end{itemdescr}

\begin{itemdecl}
template <BidirectionalIterator I>
  constexpr I operator()(I x, difference_type_t<I> n) const;
\end{itemdecl}

\begin{itemdescr}
\pnum
\effects Equivalent to: \tcode{advance(x, -n); return x;}
\end{itemdescr}

\begin{itemdecl}
template <BidirectionalIterator I>
  constexpr I operator()(I x, difference_type_t<I> n, I bound) const;
\end{itemdecl}

\begin{itemdescr}
\pnum
\effects Equivalent to: \tcode{advance(x, -n, bound); return x;}
\end{itemdescr}

\rSec1[std2.iterators.predef]{Iterator adaptors}

\rSec2[std2.iterators.reverse]{Reverse iterators}

\pnum
Class template \tcode{reverse_iterator} is an iterator adaptor that iterates from the end of the sequence defined by its underlying iterator to the beginning of that sequence.
The fundamental relation between a reverse iterator and its corresponding underlying iterator
\tcode{i}
is established by the identity:
\tcode{*make_reverse_iterator(i) == *prev(i)}.

\rSec3[std2.reverse.iterator]{Class template \tcode{reverse_iterator}}

\indexlibrary{\idxcode{reverse_iterator}}%
\begin{codeblock}
namespace @\changed{std \{ namespace experimental \{ namespace ranges}{std2}@ { inline namespace v1 {
  template <BidirectionalIterator I>
  class reverse_iterator {
  public:
    using iterator_type = I;
    using difference_type = difference_type_t<I>;
    using value_type = value_type_t<I>;
    using iterator_category = iterator_category_t<I>;
    using reference = reference_t<I>;
    using pointer = I;

    constexpr reverse_iterator();
    explicit constexpr reverse_iterator(I x);
    @\added{template <ConvertibleTo<I> U>}@
      constexpr reverse_iterator(const reverse_iterator<@\changed{ConvertibleTo<I>}{U}@>& i);
    @\added{template <ConvertibleTo<I> U>}@
      constexpr reverse_iterator& operator=(const reverse_iterator<@\changed{ConvertibleTo<I>}{U}@>& i);

    constexpr I base() const;
    constexpr reference operator*() const;
    constexpr pointer operator->() const;

    constexpr reverse_iterator& operator++();
    constexpr reverse_iterator  operator++(int);
    constexpr reverse_iterator& operator--();
    constexpr reverse_iterator  operator--(int);

    constexpr reverse_iterator  operator+ (difference_type n) const
      requires RandomAccessIterator<I>;
    constexpr reverse_iterator& operator+=(difference_type n)
      requires RandomAccessIterator<I>;
    constexpr reverse_iterator  operator- (difference_type n) const
      requires RandomAccessIterator<I>;
    constexpr reverse_iterator& operator-=(difference_type n)
      requires RandomAccessIterator<I>;
    constexpr reference operator[](difference_type n) const
      requires RandomAccessIterator<I>;

    friend constexpr rvalue_reference_t<I> iter_move(const reverse_iterator& i)
      noexcept(@\seebelow@);
    template <IndirectlySwappable<I> I2>
      friend constexpr void iter_swap(const reverse_iterator& x, const reverse_iterator<I2>& y)
        noexcept(@\seebelow@);

  private:
    I current; // \expos
  };

  template <class I1, class I2>
      requires EqualityComparableWith<I1, I2>
    constexpr bool operator==(
      const reverse_iterator<I1>& x,
      const reverse_iterator<I2>& y);
  template <class I1, class I2>
      requires EqualityComparableWith<I1, I2>
    constexpr bool operator!=(
      const reverse_iterator<I1>& x,
      const reverse_iterator<I2>& y);
  template <class I1, class I2>
      requires StrictTotallyOrderedWith<I1, I2>
    constexpr bool operator<(
      const reverse_iterator<I1>& x,
      const reverse_iterator<I2>& y);
  template <class I1, class I2>
      requires StrictTotallyOrderedWith<I1, I2>
    constexpr bool operator>(
      const reverse_iterator<I1>& x,
      const reverse_iterator<I2>& y);
  template <class I1, class I2>
      requires StrictTotallyOrderedWith<I1, I2>
    constexpr bool operator>=(
      const reverse_iterator<I1>& x,
      const reverse_iterator<I2>& y);
  template <class I1, class I2>
      requires StrictTotallyOrderedWith<I1, I2>
    constexpr bool operator<=(
      const reverse_iterator<I1>& x,
      const reverse_iterator<I2>& y);
  template <class I1, class I2>
      requires SizedSentinel<I1, I2>
    constexpr difference_type_t<I2> operator-(
      const reverse_iterator<I1>& x,
      const reverse_iterator<I2>& y);
  template <RandomAccessIterator I>
    constexpr reverse_iterator<I> operator+(
      difference_type_t<I> n,
      const reverse_iterator<I>& x);

  template <BidirectionalIterator I>
    constexpr reverse_iterator<I> make_reverse_iterator(I i);
}}@\removed{\}\}}@
\end{codeblock}

\rSec3[std2.reverse.iter.ops]{\tcode{reverse_iterator} operations}

\rSec4[std2.reverse.iter.cons]{\tcode{reverse_iterator} constructor}

\indexlibrary{\idxcode{reverse_iterator}!\tcode{reverse_iterator}}%
\begin{itemdecl}
constexpr reverse_iterator();
\end{itemdecl}

\begin{itemdescr}
\pnum
\effects
Value-initializes
\tcode{current}.
Iterator operations applied to the resulting iterator have defined behavior
if and only if the corresponding operations are defined on a
value-initialized iterator of type
\tcode{I}.
\end{itemdescr}

\indexlibrary{\idxcode{reverse_iterator}!constructor}%
\begin{itemdecl}
explicit constexpr reverse_iterator(I x);
\end{itemdecl}

\begin{itemdescr}
\pnum
\effects
Initializes
\tcode{current}
with \tcode{x}.
\end{itemdescr}

\indexlibrary{\idxcode{reverse_iterator}!constructor}%
\begin{itemdecl}
@\added{template<ConvertibleTo<I> U>}@
  constexpr reverse_iterator(const reverse_iterator<@\changed{ConvertibleTo<I>}{U}@>& i);
\end{itemdecl}

\begin{itemdescr}
\pnum
\effects
Initializes
\tcode{current}
with
\tcode{i.current}.
\end{itemdescr}

\rSec4[std2.reverse.iter.op=]{\tcode{reverse_iterator::operator=}}

\indexlibrary{\idxcode{operator=}!\tcode{reverse_iterator}}%
\begin{itemdecl}
@\added{template<ConvertibleTo<I> U>}@
  constexpr reverse_iterator&
    operator=(const reverse_iterator<@\changed{ConvertibleTo<I>}{U}@>& i);
\end{itemdecl}

\begin{itemdescr}
\pnum
\effects
Assigns \tcode{i.current} to \tcode{current}.

\pnum
\returns
\tcode{*this}.
\end{itemdescr}

\rSec4[std2.reverse.iter.conv]{Conversion}

\indexlibrary{\idxcode{base}!\idxcode{reverse_iterator}}%
\indexlibrary{\idxcode{reverse_iterator}!\idxcode{base}}%
\begin{itemdecl}
constexpr I base() const;
\end{itemdecl}

\begin{itemdescr}
\pnum
\returns
\tcode{current}.
\end{itemdescr}

\rSec4[std2.reverse.iter.op.star]{\tcode{operator*}}

\indexlibrary{\idxcode{operator*}!\idxcode{reverse_iterator}}%
\begin{itemdecl}
constexpr reference operator*() const;
\end{itemdecl}

\begin{itemdescr}
\pnum
\effects Equivalent to: \tcode{return *prev(current);}
\end{itemdescr}

\rSec4[std2.reverse.iter.opref]{\tcode{operator->}}

\indexlibrary{\idxcode{operator->}!\idxcode{reverse_iterator}}%
\begin{itemdecl}
constexpr pointer operator->() const;
\end{itemdecl}

\begin{itemdescr}
\pnum
\effects Equivalent to: \tcode{return prev(current);}
\end{itemdescr}

\rSec4[std2.reverse.iter.op++]{\tcode{operator++}}

\indexlibrary{\idxcode{operator++}!\idxcode{reverse_iterator}}%
\begin{itemdecl}
constexpr reverse_iterator& operator++();
\end{itemdecl}

\begin{itemdescr}
\pnum
\effects
\tcode{\dcr current;}

\pnum
\returns
\tcode{*this}.
\end{itemdescr}

\indexlibrary{\idxcode{operator++}!\idxcode{reverse_iterator}}%
\indexlibrary{\idxcode{reverse_iterator}!\idxcode{operator++}}%
\begin{itemdecl}
constexpr reverse_iterator operator++(int);
\end{itemdecl}

\begin{itemdescr}
\pnum
\effects
\begin{codeblock}
reverse_iterator tmp = *this;
--current;
return tmp;
\end{codeblock}
\end{itemdescr}

\rSec4[std2.reverse.iter.op\dcr]{\tcode{operator\dcr}}

\indexlibrary{\idxcode{operator\dcr}!\idxcode{reverse_iterator}}%
\begin{itemdecl}
constexpr reverse_iterator& operator--();
\end{itemdecl}

\begin{itemdescr}
\pnum
\effects
\tcode{++current}

\pnum
\returns
\tcode{*this}.
\end{itemdescr}

\indexlibrary{\idxcode{operator\dcr}!\idxcode{reverse_iterator}}%
\indexlibrary{\idxcode{reverse_iterator}!\idxcode{operator\dcr}}%
\begin{itemdecl}
constexpr reverse_iterator operator--(int);
\end{itemdecl}

\begin{itemdescr}
\pnum
\effects
\begin{codeblock}
reverse_iterator tmp = *this;
++current;
return tmp;
\end{codeblock}
\end{itemdescr}

\rSec4[std2.reverse.iter.op+]{\tcode{operator+}}

\indexlibrary{\idxcode{operator+}!\idxcode{reverse_iterator}}%
\begin{itemdecl}
constexpr reverse_iterator
  operator+(difference_type n) const
    requires RandomAccessIterator<I>;
\end{itemdecl}

\begin{itemdescr}
\pnum
\returns
\tcode{reverse_iterator(current-n)}.
\end{itemdescr}

\rSec4[std2.reverse.iter.op+=]{\tcode{operator+=}}

\indexlibrary{\idxcode{operator+=}!\idxcode{reverse_iterator}}%
\begin{itemdecl}
constexpr reverse_iterator&
  operator+=(difference_type n)
    requires RandomAccessIterator<I>;
\end{itemdecl}

\begin{itemdescr}
\pnum
\effects
\tcode{current -= n;}

\pnum
\returns
\tcode{*this}.
\end{itemdescr}

\rSec4[std2.reverse.iter.op-]{\tcode{operator-}}

\indexlibrary{\idxcode{operator-}!\idxcode{reverse_iterator}}%
\begin{itemdecl}
constexpr reverse_iterator
  operator-(difference_type n) const
    requires RandomAccessIterator<I>;
\end{itemdecl}

\begin{itemdescr}
\pnum
\returns
\tcode{reverse_iterator(current+n)}.
\end{itemdescr}

\rSec4[std2.reverse.iter.op-=]{\tcode{operator-=}}

\indexlibrary{\idxcode{operator-=}!\idxcode{reverse_iterator}}%
\begin{itemdecl}
constexpr reverse_iterator&
  operator-=(difference_type n)
    requires RandomAccessIterator<I>;
\end{itemdecl}

\begin{itemdescr}
\pnum
\effects
\tcode{current += n;}

\pnum
\returns
\tcode{*this}.
\end{itemdescr}

\rSec4[std2.reverse.iter.opindex]{\tcode{operator[]}}

\indexlibrary{\idxcode{operator[]}!\idxcode{reverse_iterator}}%
\begin{itemdecl}
constexpr reference operator[](
  difference_type n) const
    requires RandomAccessIterator<I>;
\end{itemdecl}

\begin{itemdescr}
\pnum
\returns
\tcode{current[-n-1]}.
\end{itemdescr}

\rSec4[std2.reverse.iter.op==]{\tcode{operator==}}

\indexlibrary{\idxcode{operator==}!\idxcode{reverse_iterator}}%
\begin{itemdecl}
template <class I1, class I2>
    requires EqualityComparableWith<I1, I2>
  constexpr bool operator==(
    const reverse_iterator<I1>& x,
    const reverse_iterator<I2>& y);
\end{itemdecl}

\begin{itemdescr}
\pnum
\effects Equivalent to:
\tcode{return x.current == y.current;}
\end{itemdescr}

\rSec4[std2.reverse.iter.op!=]{\tcode{operator!=}}

\indexlibrary{\idxcode{operator"!=}!\idxcode{reverse_iterator}}%
\begin{itemdecl}
template <class I1, class I2>
    requires EqualityComparableWith<I1, I2>
  constexpr bool operator!=(
    const reverse_iterator<I1>& x,
    const reverse_iterator<I2>& y);
\end{itemdecl}

\begin{itemdescr}
\pnum
\effects Equivalent to:
\tcode{return x.current != y.current;}
\end{itemdescr}

\rSec4[std2.reverse.iter.op<]{\tcode{operator<}}

\indexlibrary{\idxcode{operator<}!\idxcode{reverse_iterator}}%
\begin{itemdecl}
template <class I1, class I2>
    requires StrictTotallyOrderedWith<I1, I2>
  constexpr bool operator<(
    const reverse_iterator<I1>& x,
    const reverse_iterator<I2>& y);
\end{itemdecl}

\begin{itemdescr}
\pnum
\effects Equivalent to:
\tcode{return x.current > y.current;}
\end{itemdescr}

\rSec4[std2.reverse.iter.op>]{\tcode{operator>}}

\indexlibrary{\idxcode{operator>}!\idxcode{reverse_iterator}}%
\begin{itemdecl}
template <class I1, class I2>
    requires StrictTotallyOrderedWith<I1, I2>
  constexpr bool operator>(
    const reverse_iterator<I1>& x,
    const reverse_iterator<I2>& y);
\end{itemdecl}

\begin{itemdescr}
\pnum
\effects Equivalent to:
\tcode{return x.current < y.current;}
\end{itemdescr}

\rSec4[std2.reverse.iter.op>=]{\tcode{operator>=}}

\indexlibrary{\idxcode{operator>=}!\idxcode{reverse_iterator}}%
\begin{itemdecl}
template <class I1, class I2>
    requires StrictTotallyOrderedWith<I1, I2>
  constexpr bool operator>=(
    const reverse_iterator<I1>& x,
    const reverse_iterator<I2>& y);
\end{itemdecl}

\begin{itemdescr}
\pnum
\effects Equivalent to:
\tcode{return x.current <= y.current;}
\end{itemdescr}

\rSec4[std2.reverse.iter.op<=]{\tcode{operator<=}}

\indexlibrary{\idxcode{operator<=}!\idxcode{reverse_iterator}}%
\begin{itemdecl}
template <class I1, class I2>
    requires StrictTotallyOrderedWith<I1, I2>
  constexpr bool operator<=(
    const reverse_iterator<I1>& x,
    const reverse_iterator<I2>& y);
\end{itemdecl}

\begin{itemdescr}
\pnum
\effects Equivalent to:
\tcode{return x.current >= y.current;}
\end{itemdescr}

\rSec4[std2.reverse.iter.opdiff]{\tcode{operator-}}

\indexlibrary{\idxcode{operator-}!\idxcode{reverse_iterator}}%
\begin{itemdecl}
template <class I1, class I2>
    requires SizedSentinel<I1, I2>
  constexpr difference_type_t<I2> operator-(
    const reverse_iterator<I1>& x,
    const reverse_iterator<I2>& y);
\end{itemdecl}

\begin{itemdescr}
\pnum
\effects Equivalent to:
\tcode{return y.current - x.current;}
\end{itemdescr}

\rSec4[std2.reverse.iter.opsum]{\tcode{operator+}}

\indexlibrary{\idxcode{operator+}!\idxcode{reverse_iterator}}%
\begin{itemdecl}
template <RandomAccessIterator I>
  constexpr reverse_iterator<I> operator+(
    difference_type_t<I> n,
    const reverse_iterator<I>& x);
\end{itemdecl}

\begin{itemdescr}
\pnum
\effects Equivalent to:
\tcode{return reverse_iterator<I>(x.current - n);}
\end{itemdescr}

\rSec4[std2.reverse.iter.iter_move]{\tcode{iter_move}}
\indexlibrary{\idxcode{reverse_iterator}!\idxcode{iter_move}}%
\indexlibrary{\idxcode{iter_move}!\idxcode{reverse_iterator}}%
\begin{itemdecl}
friend constexpr rvalue_reference_t<I> iter_move(const reverse_iterator& i)
   noexcept(@\seebelow@);
\end{itemdecl}

\begin{itemdescr}
\pnum
\effects Equivalent to: \tcode{return \changed{ranges}{::std2}::iter_move(prev(i.current));}

\pnum
\remarks The expression in \tcode{noexcept} is equivalent to:
\begin{codeblock}
   noexcept(@\changed{ranges}{::std2}@::iter_move(declval<I&>())) && noexcept(--declval<I&>()) &&
     is_nothrow_copy_constructible@\added{_v}@<I>@\removed{::value}@
\end{codeblock}
\end{itemdescr}

\rSec4[std2.reverse.iter.iter_swap]{\tcode{iter_swap}}
\indexlibrary{\idxcode{reverse_iterator}!\idxcode{iter_swap}}%
\indexlibrary{\idxcode{iter_swap}!\idxcode{reverse_iterator}}%
\begin{itemdecl}
template <IndirectlySwappable<I> I2>
  friend constexpr void iter_swap(const reverse_iterator& x, const reverse_iterator<I2>& y)
    noexcept(@\seebelow@);
\end{itemdecl}

\begin{itemdescr}
\pnum
\effects Equivalent to \tcode{\changed{ranges}{::std2}::iter_swap(prev(x.current), prev(y.current))}.

\pnum
\remarks The expression in \tcode{noexcept} is equivalent to:
\begin{codeblock}
  noexcept(@\changed{ranges}{::std2}@::iter_swap(declval<I>(), declval<I>())) && noexcept(--declval<I&>())
\end{codeblock}
\end{itemdescr}

\rSec4[std2.reverse.iter.make]{Non-member function \tcode{make_reverse_iterator()}}

\indexlibrary{\idxcode{reverse_iterator}!\idxcode{make_reverse_iterator}~non-member~function}
\indexlibrary{\idxcode{make_reverse_iterator}}%
\begin{itemdecl}
template <BidirectionalIterator I>
  constexpr reverse_iterator<I> make_reverse_iterator(I i);
\end{itemdecl}

\begin{itemdescr}
\pnum
\returns
\tcode{reverse_iterator<I>(i)}.
\end{itemdescr}

\rSec2[std2.iterators.insert]{Insert iterators}

\pnum
To make it possible to deal with insertion in the same way as writing into an array, a special kind of iterator
adaptors, called
\techterm{insert iterators},
are provided in the library.
With regular iterator classes,

\begin{codeblock}
while (first != last) *result++ = *first++;
\end{codeblock}

causes a range \range{first}{last}
to be copied into a range starting with result.
The same code with
\tcode{result}
being an insert iterator will insert corresponding elements into the container.
This device allows all of the
copying algorithms in the library to work in the
\techterm{insert mode}
instead of the \techterm{regular overwrite} mode.

\pnum
An insert iterator is constructed from a container and possibly one of its iterators pointing to where
insertion takes place if it is neither at the beginning nor at the end of the container.
Insert iterators satisfy \tcode{OutputIterator}.
\tcode{operator*}
returns the insert iterator itself.
The assignment
\tcode{operator=(const T\& x)}
is defined on insert iterators to allow writing into them, it inserts
\tcode{x}
right before where the insert iterator is pointing.
In other words, an insert iterator is like a cursor pointing into the
container where the insertion takes place.
\tcode{back_insert_iterator}
inserts elements at the end of a container,
\tcode{front_insert_iterator}
inserts elements at the beginning of a container, and
\tcode{insert_iterator}
inserts elements where the iterator points to in a container.
\tcode{back_inserter},
\tcode{front_inserter},
and
\tcode{inserter}
are three
functions making the insert iterators out of a container.

\rSec3[std2.back.insert.iterator]{Class template \tcode{back_insert_iterator}}

\indexlibrary{\idxcode{back_insert_iterator}}%
\begin{codeblock}
namespace @\changed{std \{ namespace experimental \{ namespace ranges}{std2}@ { inline namespace v1 {
  template <class Container>
  class back_insert_iterator {
  public:
    using container_type = Container;
    using difference_type = ptrdiff_t;

    constexpr back_insert_iterator();
    explicit back_insert_iterator(Container& x);
    back_insert_iterator&
      operator=(const value_type_t<Container>& value);
    back_insert_iterator&
      operator=(value_type_t<Container>&& value);

    back_insert_iterator& operator*();
    back_insert_iterator& operator++();
    back_insert_iterator operator++(int);

  private:
    Container* container; // \expos
  };

  template <class Container>
    back_insert_iterator<Container> back_inserter(Container& x);
}}@\removed{\}\}}@
\end{codeblock}

\rSec3[std2.back.insert.iter.ops]{\tcode{back_insert_iterator} operations}

\rSec4[std2.back.insert.iter.cons]{\tcode{back_insert_iterator} constructor}

\indexlibrary{\idxcode{back_insert_iterator}!\idxcode{back_insert_iterator}}%
\begin{itemdecl}
constexpr back_insert_iterator();
\end{itemdecl}

\begin{itemdescr}
\pnum
\effects
Value-initializes
\tcode{container}.
\end{itemdescr}

\indexlibrary{\idxcode{back_insert_iterator}!constructor}%

\begin{itemdecl}
explicit back_insert_iterator(Container& x);
\end{itemdecl}

\begin{itemdescr}
\pnum
\effects
Initializes
\tcode{container}
with \tcode{addressof(x)}.
\end{itemdescr}

\rSec4[std2.back.insert.iter.op=]{\tcode{back_insert_iterator::operator=}}

\indexlibrary{\idxcode{operator=}!\idxcode{back_insert_iterator}}%
\begin{itemdecl}
back_insert_iterator&
  operator=(const value_type_t<Container>& value);
\end{itemdecl}

\begin{itemdescr}
\pnum
\effects Equivalent to
\tcode{container->push_back(value)}.

\pnum
\returns
\tcode{*this}.
\end{itemdescr}

\indexlibrary{\idxcode{operator=}!\idxcode{back_insert_iterator}}%
\begin{itemdecl}
back_insert_iterator&
  operator=(value_type_t<Container>&& value);
\end{itemdecl}

\begin{itemdescr}
\pnum
\effects Equivalent to
\tcode{container->push_back(std::move(value))}.

\pnum
\returns
\tcode{*this}.
\end{itemdescr}

\rSec4[std2.back.insert.iter.op*]{\tcode{back_insert_iterator::operator*}}

\indexlibrary{\idxcode{operator*}!\idxcode{back_insert_iterator}}%
\begin{itemdecl}
back_insert_iterator& operator*();
\end{itemdecl}

\begin{itemdescr}
\pnum
\returns
\tcode{*this}.
\end{itemdescr}

\rSec4[std2.back.insert.iter.op++]{\tcode{back_insert_iterator::operator++}}

\indexlibrary{\idxcode{operator++}!\idxcode{back_insert_iterator}}%
\begin{itemdecl}
back_insert_iterator& operator++();
back_insert_iterator operator++(int);
\end{itemdecl}

\begin{itemdescr}
\pnum
\returns
\tcode{*this}.
\end{itemdescr}

\rSec4[std2.back.inserter]{ \tcode{back_inserter}}

\indexlibrary{\idxcode{back_inserter}}%
\begin{itemdecl}
template <class Container>
  back_insert_iterator<Container> back_inserter(Container& x);
\end{itemdecl}

\begin{itemdescr}
\pnum
\returns
\tcode{back_insert_iterator<Container>(x)}.
\end{itemdescr}

\rSec3[std2.front.insert.iterator]{Class template \tcode{front_insert_iterator}}

\indexlibrary{\idxcode{front_insert_iterator}}%
\begin{codeblock}
namespace @\changed{std \{ namespace experimental \{ namespace ranges}{std2}@ { inline namespace v1 {
  template <class Container>
  class front_insert_iterator {
  public:
    using container_type = Container;
    using difference_type = ptrdiff_t;

    constexpr front_insert_iterator();
    explicit front_insert_iterator(Container& x);
    front_insert_iterator&
      operator=(const value_type_t<Container>& value);
    front_insert_iterator&
      operator=(value_type_t<Container>&& value);

    front_insert_iterator& operator*();
    front_insert_iterator& operator++();
    front_insert_iterator operator++(int);

  private:
    Container* container; // \expos
  };

  template <class Container>
    front_insert_iterator<Container> front_inserter(Container& x);
}}@\removed{\}\}}@
\end{codeblock}

\rSec3[std2.front.insert.iter.ops]{\tcode{front_insert_iterator} operations}

\rSec4[std2.front.insert.iter.cons]{\tcode{front_insert_iterator} constructor}

\indexlibrary{\idxcode{front_insert_iterator}!\idxcode{front_insert_iterator}}%
\begin{itemdecl}
constexpr front_insert_iterator();
\end{itemdecl}

\begin{itemdescr}
\pnum
\effects
Value-initializes
\tcode{container}.
\end{itemdescr}

\indexlibrary{\idxcode{front_insert_iterator}!constructor}%
\begin{itemdecl}
explicit front_insert_iterator(Container& x);
\end{itemdecl}

\begin{itemdescr}
\pnum
\effects
Initializes
\tcode{container}
with \tcode{addressof(x)}.
\end{itemdescr}

\rSec4[std2.front.insert.iter.op=]{\tcode{front_insert_iterator::operator=}}

\indexlibrary{\idxcode{operator=}!\idxcode{front_insert_iterator}}%
\begin{itemdecl}
front_insert_iterator&
  operator=(const value_type_t<Container>& value);
\end{itemdecl}

\begin{itemdescr}
\pnum
\effects Equivalent to
\tcode{container->push_front(value)}.

\pnum
\returns
\tcode{*this}.
\end{itemdescr}

\indexlibrary{\idxcode{operator=}!\idxcode{front_insert_iterator}}%
\begin{itemdecl}
front_insert_iterator&
  operator=(value_type_t<Container>&& value);
\end{itemdecl}

\begin{itemdescr}
\pnum
\effects Equivalent to
\tcode{container->push_front(std::move(value))}.

\pnum
\returns
\tcode{*this}.
\end{itemdescr}

\rSec4[std2.front.insert.iter.op*]{\tcode{front_insert_iterator::operator*}}

\indexlibrary{\idxcode{operator*}!\idxcode{front_insert_iterator}}%
\begin{itemdecl}
front_insert_iterator& operator*();
\end{itemdecl}

\begin{itemdescr}
\pnum
\returns
\tcode{*this}.
\end{itemdescr}

\rSec4[std2.front.insert.iter.op++]{\tcode{front_insert_iterator::operator++}}

\indexlibrary{\idxcode{operator++}!\idxcode{front_insert_iterator}}%
\begin{itemdecl}
front_insert_iterator& operator++();
front_insert_iterator operator++(int);
\end{itemdecl}

\begin{itemdescr}
\pnum
\returns
\tcode{*this}.
\end{itemdescr}

\rSec4[std2.front.inserter]{\tcode{front_inserter}}

\indexlibrary{\idxcode{front_inserter}}%
\begin{itemdecl}
template <class Container>
  front_insert_iterator<Container> front_inserter(Container& x);
\end{itemdecl}

\begin{itemdescr}
\pnum
\returns
\tcode{front_insert_iterator<Container>(x)}.
\end{itemdescr}

\rSec3[std2.insert.iterator]{Class template \tcode{insert_iterator}}

\indexlibrary{\idxcode{insert_iterator}}%
\begin{codeblock}
namespace @\changed{std \{ namespace experimental \{ namespace ranges}{std2}@ { inline namespace v1 {
  template <class Container>
  class insert_iterator {
  public:
    using container_type = Container;
    using difference_type = ptrdiff_t;

    insert_iterator();
    insert_iterator(Container& x, iterator_t<Container> i);
    insert_iterator&
      operator=(const value_type_t<Container>& value);
    insert_iterator&
      operator=(value_type_t<Container>&& value);

    insert_iterator& operator*();
    insert_iterator& operator++();
    insert_iterator& operator++(int);

  private:
    Container* container;       // \expos
    iterator_t<Container> iter; // \expos
  };

  template <class Container>
    insert_iterator<Container> inserter(Container& x, iterator_t<Container> i);
}}@\removed{\}\}}@
\end{codeblock}

\rSec3[std2.insert.iter.ops]{\tcode{insert_iterator} operations}

\rSec4[std2.insert.iter.cons]{\tcode{insert_iterator} constructor}

\indexlibrary{\idxcode{insert_iterator}!\idxcode{insert_iterator}}%
\begin{itemdecl}
insert_iterator();
\end{itemdecl}

\begin{itemdescr}
\pnum
\effects
Value-initializes
\tcode{container} and \tcode{iter}.
\end{itemdescr}

\indexlibrary{\idxcode{insert_iterator}!constructor}%
\begin{itemdecl}
insert_iterator(Container& x, iterator_t<Container> i);
\end{itemdecl}

\begin{itemdescr}
\pnum
\requires
\tcode{i} is an iterator into \tcode{x}.

\pnum
\effects
Initializes
\tcode{container}
with \tcode{addressof(x)} and
\tcode{iter}
with \tcode{i}.
\end{itemdescr}

\rSec4[std2.insert.iter.op=]{\tcode{insert_iterator::operator=}}

\indexlibrary{\idxcode{operator=}!\idxcode{insert_iterator}}%
\begin{itemdecl}
insert_iterator&
  operator=(const value_type_t<Container>& value);
\end{itemdecl}

\begin{itemdescr}
\pnum
\effects Equivalent to:
\begin{codeblock}
iter = container->insert(iter, value);
++iter;
\end{codeblock}

\pnum
\returns
\tcode{*this}.
\end{itemdescr}

\indexlibrary{\idxcode{operator=}!\idxcode{insert_iterator}}%
\begin{itemdecl}
insert_iterator&
  operator=(value_type_t<Container>&& value);
\end{itemdecl}

\begin{itemdescr}
\pnum
\effects Equivalent to:
\begin{codeblock}
iter = container->insert(iter, std::move(value));
++iter;
\end{codeblock}

\pnum
\returns
\tcode{*this}.
\end{itemdescr}

\rSec4[std2.insert.iter.op*]{\tcode{insert_iterator::operator*}}

\indexlibrary{\idxcode{operator*}!\idxcode{insert_iterator}}%
\begin{itemdecl}
insert_iterator& operator*();
\end{itemdecl}

\begin{itemdescr}
\pnum
\returns
\tcode{*this}.
\end{itemdescr}

\rSec4[std2.insert.iter.op++]{\tcode{insert_iterator::operator++}}

\indexlibrary{\idxcode{operator++}!\idxcode{insert_iterator}}%
\begin{itemdecl}
insert_iterator& operator++();
insert_iterator& operator++(int);
\end{itemdecl}

\begin{itemdescr}
\pnum
\returns
\tcode{*this}.
\end{itemdescr}

\rSec4[std2.inserter]{\tcode{inserter}}

\indexlibrary{\idxcode{inserter}}%
\begin{itemdecl}
template <class Container>
  insert_iterator<Container> inserter(Container& x, iterator_t<Container> i);
\end{itemdecl}

\begin{itemdescr}
\pnum
\returns
\tcode{insert_iterator<Container>(x, i)}.
\end{itemdescr}

\rSec2[std2.iterators.move]{Move iterators and sentinels}

\rSec3[std2.move.iterator]{Class template \tcode{move_iterator}}

\pnum
Class template \tcode{move_iterator} is an iterator adaptor
with the same behavior as the underlying iterator except that its
indirection operator implicitly converts the value returned by the
underlying iterator's indirection operator to an rvalue
of the value type.
Some generic algorithms can be called with move iterators to replace
copying with moving.

\pnum
\enterexample

\begin{codeblock}
list<string> s;
// populate the list \tcode{s}
vector<string> v1(s.begin(), s.end());          // copies strings into \tcode{v1}
vector<string> v2(make_move_iterator(s.begin()),
                  make_move_iterator(s.end())); // moves strings into \tcode{v2}
\end{codeblock}

\exitexample

\indexlibrary{\idxcode{move_iterator}}%
\begin{codeblock}
namespace @\changed{std \{ namespace experimental \{ namespace ranges}{std2}@ { inline namespace v1 {
  template <InputIterator I>
  class move_iterator {
  public:
    using iterator_type     = I;
    using difference_type   = difference_type_t<I>;
    using value_type        = value_type_t<I>;
    using iterator_category = input_iterator_tag;
    using reference         = rvalue_reference_t<I>;

    constexpr move_iterator();
    explicit constexpr move_iterator(I i);
    @\added{template <ConvertibleTo<I> U>}@
      constexpr move_iterator(const move_iterator<@\changed{ConvertibleTo<I>}{U}@>& i);
    @\added{template <ConvertibleTo<I> U>}@
      constexpr move_iterator& operator=(const move_iterator<@\changed{ConvertibleTo<I>}{U}@>& i);

    constexpr I base() const;
    constexpr reference operator*() const;

    constexpr move_iterator& operator++();
    constexpr void operator++(int);
    constexpr move_iterator operator++(int)
      requires ForwardIterator<I>;
    constexpr move_iterator& operator--()
      requires BidirectionalIterator<I>;
    constexpr move_iterator operator--(int)
      requires BidirectionalIterator<I>;

    constexpr move_iterator operator+(difference_type n) const
      requires RandomAccessIterator<I>;
    constexpr move_iterator& operator+=(difference_type n)
      requires RandomAccessIterator<I>;
    constexpr move_iterator operator-(difference_type n) const
      requires RandomAccessIterator<I>;
    constexpr move_iterator& operator-=(difference_type n)
      requires RandomAccessIterator<I>;
    constexpr reference operator[](difference_type n) const
      requires RandomAccessIterator<I>;

    friend constexpr rvalue_reference_t<I> iter_move(const move_iterator& i)
      noexcept(@\seebelow@);
    template <IndirectlySwappable<I> I2>
      friend constexpr void iter_swap(const move_iterator& x, const move_iterator<I2>& y)
        noexcept(@\seebelow@);

  private:
    I current; // \expos
  };

  template <class I1, class I2>
      requires EqualityComparableWith<I1, I2>
    constexpr bool operator==(
      const move_iterator<I1>& x, const move_iterator<I2>& y);
  template <class I1, class I2>
      requires EqualityComparableWith<I1, I2>
    constexpr bool operator!=(
      const move_iterator<I1>& x, const move_iterator<I2>& y);
  template <class I1, class I2>
      requires StrictTotallyOrderedWith<I1, I2>
    constexpr bool operator<(
      const move_iterator<I1>& x, const move_iterator<I2>& y);
  template <class I1, class I2>
      requires StrictTotallyOrderedWith<I1, I2>
    constexpr bool operator<=(
      const move_iterator<I1>& x, const move_iterator<I2>& y);
  template <class I1, class I2>
      requires StrictTotallyOrderedWith<I1, I2>
    constexpr bool operator>(
      const move_iterator<I1>& x, const move_iterator<I2>& y);
  template <class I1, class I2>
      requires StrictTotallyOrderedWith<I1, I2>
    constexpr bool operator>=(
      const move_iterator<I1>& x, const move_iterator<I2>& y);

  template <class I1, class I2>
      requires SizedSentinel<I1, I2>
    constexpr difference_type_t<I2> operator-(
      const move_iterator<I1>& x,
      const move_iterator<I2>& y);
  template <RandomAccessIterator I>
    constexpr move_iterator<I> operator+(
      difference_type_t<I> n,
      const move_iterator<I>& x);
  template <InputIterator I>
    constexpr move_iterator<I> make_move_iterator(I i);
}}@\removed{\}\}}@
\end{codeblock}

\pnum
\enternote \tcode{move_iterator} does not provide an \tcode{operator->} because the class member access
expression \tcode{\textit{i}->\textit{m}} may have different semantics than the expression
\tcode{(*\textit{i}).\textit{m}} when the expression \tcode{*\textit{i}} is an rvalue.\exitnote

\rSec3[std2.move.iter.ops]{\tcode{move_iterator} operations}

\rSec4[std2.move.iter.op.const]{\tcode{move_iterator} constructors}

\indexlibrary{\idxcode{move_iterator}!\idxcode{move_iterator}}%
\begin{itemdecl}
constexpr move_iterator();
\end{itemdecl}

\begin{itemdescr}
\pnum
\effects Constructs a \tcode{move_iterator}, value-initializing
\tcode{current}. Iterator operations applied to the resulting
iterator have defined behavior if and only if the corresponding operations are defined
on a value-initialized iterator of type \tcode{I}.
\end{itemdescr}


\indexlibrary{\idxcode{move_iterator}!constructor}%
\begin{itemdecl}
explicit constexpr move_iterator(I i);
\end{itemdecl}

\begin{itemdescr}
\pnum
\effects Constructs a \tcode{move_iterator}, initializing
\tcode{current} with \tcode{i}.
\end{itemdescr}


\indexlibrary{\idxcode{move_iterator}!constructor}%
\begin{itemdecl}
@\added{template <ConvertibleTo<I> U>}@
  constexpr move_iterator(const move_iterator<@\changed{ConvertibleTo<I>}{U}@>& i);
\end{itemdecl}

\begin{itemdescr}
\pnum
\effects Constructs a \tcode{move_iterator}, initializing
\tcode{current} with \tcode{i.current}.
\end{itemdescr}

\rSec4[std2.move.iter.op=]{\tcode{move_iterator::operator=}}

\indexlibrary{\idxcode{operator=}!\idxcode{move_iterator}}%
\indexlibrary{\idxcode{move_iterator}!\idxcode{operator=}}%
\begin{itemdecl}
@\added{template <ConvertibleTo<I> U>}@
  constexpr move_iterator& operator=(const move_iterator<@\changed{ConvertibleTo<I>}{U}@>& i);
\end{itemdecl}

\begin{itemdescr}
\pnum
\effects Assigns \tcode{i.current} to
\tcode{current}.
\end{itemdescr}

\rSec4[std2.move.iter.op.conv]{\tcode{move_iterator} conversion}

\indexlibrary{\idxcode{base}!\idxcode{move_iterator}}%
\indexlibrary{\idxcode{move_iterator}!\idxcode{base}}%
\begin{itemdecl}
constexpr I base() const;
\end{itemdecl}

\begin{itemdescr}
\pnum
\returns \tcode{current}.
\end{itemdescr}

\rSec4[std2.move.iter.op.star]{\tcode{move_iterator::operator*}}

\indexlibrary{\idxcode{operator*}!\idxcode{move_iterator}}%
\indexlibrary{\idxcode{move_iterator}!\idxcode{operator*}}%
\begin{itemdecl}
constexpr reference operator*() const;
\end{itemdecl}

\begin{itemdescr}
\pnum
\effects Equivalent to:
\tcode{return iter_move(current);}
\end{itemdescr}

\rSec4[std2.move.iter.op.incr]{\tcode{move_iterator::operator++}}

\indexlibrary{\idxcode{operator++}!\idxcode{move_iterator}}%
\indexlibrary{\idxcode{move_iterator}!\idxcode{operator++}}%
\begin{itemdecl}
constexpr move_iterator& operator++();
\end{itemdecl}

\begin{itemdescr}
\pnum
\effects Equivalent to \tcode{++current}.

\pnum
\returns \tcode{*this}.
\end{itemdescr}

\indexlibrary{\idxcode{operator++}!\idxcode{move_iterator}}%
\indexlibrary{\idxcode{move_iterator}!\idxcode{operator++}}%
\begin{itemdecl}
constexpr void operator++(int);
\end{itemdecl}

\begin{itemdescr}
\pnum
\effects Equivalent to \tcode{++current}.
\end{itemdescr}

\begin{itemdecl}
constexpr move_iterator operator++(int)
  requires ForwardIterator<I>;
\end{itemdecl}

\begin{itemdescr}
\pnum
\effects Equivalent to:
\begin{codeblock}
move_iterator tmp = *this;
++current;
return tmp;
\end{codeblock}
\end{itemdescr}

\rSec4[std2.move.iter.op.decr]{\tcode{move_iterator::operator-{-}}}

\indexlibrary{\idxcode{operator\dcr}!\idxcode{move_iterator}}%
\indexlibrary{\idxcode{move_iterator}!\idxcode{operator\dcr}}%
\begin{itemdecl}
constexpr move_iterator& operator--()
  requires BidirectionalIterator<I>;
\end{itemdecl}

\begin{itemdescr}
\pnum
\effects Equivalent to \tcode{\dcr{}current}.

\pnum
\returns \tcode{*this}.
\end{itemdescr}

\indexlibrary{\idxcode{operator\dcr}!\idxcode{move_iterator}}%
\indexlibrary{\idxcode{move_iterator}!\idxcode{operator\dcr}}%
\begin{itemdecl}
constexpr move_iterator operator--(int)
  requires BidirectionalIterator<I>;
\end{itemdecl}

\begin{itemdescr}
\pnum
\effects Equivalent to:
\begin{codeblock}
move_iterator tmp = *this;
--current;
return tmp;
\end{codeblock}
\end{itemdescr}

\rSec4[std2.move.iter.op.+]{\tcode{move_iterator::operator+}}

\indexlibrary{\idxcode{operator+}!\idxcode{move_iterator}}%
\indexlibrary{\idxcode{move_iterator}!\idxcode{operator+}}%
\begin{itemdecl}
constexpr move_iterator operator+(difference_type n) const
  requires RandomAccessIterator<I>;
\end{itemdecl}

\begin{itemdescr}
\pnum
\effects Equivalent to:
\tcode{return move_iterator(current + n);}
\end{itemdescr}

\rSec4[std2.move.iter.op.+=]{\tcode{move_iterator::operator+=}}

\indexlibrary{\idxcode{operator+=}!\idxcode{move_iterator}}%
\indexlibrary{\idxcode{move_iterator}!\idxcode{operator+=}}%
\begin{itemdecl}
constexpr move_iterator& operator+=(difference_type n)
  requires RandomAccessIterator<I>;
\end{itemdecl}

\begin{itemdescr}
\pnum
\effects Equivalent to \tcode{current += n}.

\pnum
\returns \tcode{*this}.
\end{itemdescr}

\rSec4[std2.move.iter.op.-]{\tcode{move_iterator::operator-}}

\indexlibrary{\idxcode{operator-}!\idxcode{move_iterator}}%
\indexlibrary{\idxcode{move_iterator}!\idxcode{operator-}}%
\begin{itemdecl}
constexpr move_iterator operator-(difference_type n) const
  requires RandomAccessIterator<I>;
\end{itemdecl}

\begin{itemdescr}
\pnum
\effects Equivalent to:
\tcode{return move_iterator(current - n);}
\end{itemdescr}

\rSec4[std2.move.iter.op.-=]{\tcode{move_iterator::operator-=}}

\indexlibrary{\idxcode{operator-=}!\idxcode{move_iterator}}%
\indexlibrary{\idxcode{move_iterator}!\idxcode{operator-=}}%
\begin{itemdecl}
constexpr move_iterator& operator-=(difference_type n)
  requires RandomAccessIterator<I>;
\end{itemdecl}

\begin{itemdescr}
\pnum
\effects Equivalent to \tcode{current -= n}.

\pnum
\returns \tcode{*this}.
\end{itemdescr}

\rSec4[std2.move.iter.op.index]{\tcode{move_iterator::operator[]}}

\indexlibrary{\idxcode{operator[]}!\idxcode{move_iterator}}%
\indexlibrary{\idxcode{move_iterator}!\idxcode{operator[]}}%
\begin{itemdecl}
constexpr reference operator[](difference_type n) const
  requires RandomAccessIterator<I>;
\end{itemdecl}

\begin{itemdescr}
\pnum
\effects Equivalent to:
\tcode{return iter_move(current + n);}
\end{itemdescr}

\rSec4[std2.move.iter.op.comp]{\tcode{move_iterator} comparisons}

\indexlibrary{\idxcode{operator==}!\idxcode{move_iterator}}%
\indexlibrary{\idxcode{move_iterator}!\idxcode{operator==}}%
\begin{itemdecl}
template <class I1, class I2>
    requires EqualityComparableWith<I1, I2>
  constexpr bool operator==(
    const move_iterator<I1>& x, const move_iterator<I2>& y);
\end{itemdecl}

\begin{itemdescr}
\pnum
\effects Equivalent to:
\tcode{return x.current == y.current;}
\end{itemdescr}

\indexlibrary{\idxcode{operator"!=}!\idxcode{move_iterator}}%
\indexlibrary{\idxcode{move_iterator}!\idxcode{operator"!=}}%
\begin{itemdecl}
template <class I1, class I2>
    requires EqualityComparableWith<I1, I2>
  constexpr bool operator!=(
    const move_iterator<I1>& x, const move_iterator<I2>& y);
\end{itemdecl}

\begin{itemdescr}
\pnum
\effects Equivalent to:
\tcode{return !(x == y);}
\end{itemdescr}

\indexlibrary{\idxcode{operator<}!\idxcode{move_iterator}}%
\indexlibrary{\idxcode{move_iterator}!\idxcode{operator<}}%
\begin{itemdecl}
template <class I1, class I2>
    requires StrictTotallyOrderedWith<I1, I2>
  constexpr bool operator<(
    const move_iterator<I1>& x, const move_iterator<I2>& y);
\end{itemdecl}

\begin{itemdescr}
\pnum
\effects Equivalent to:
\tcode{return x.current < y.current;}
\end{itemdescr}

\indexlibrary{\idxcode{operator<=}!\idxcode{move_iterator}}%
\indexlibrary{\idxcode{move_iterator}!\idxcode{operator<=}}%
\begin{itemdecl}
template <class I1, class I2>
    requires StrictTotallyOrderedWith<I1, I2>
  constexpr bool operator<=(
    const move_iterator<I1>& x, const move_iterator<I2>& y);
\end{itemdecl}

\begin{itemdescr}
\pnum
\effects Equivalent to:
\tcode{return !(y < x);}
\end{itemdescr}

\indexlibrary{\idxcode{operator>}!\idxcode{move_iterator}}%
\indexlibrary{\idxcode{move_iterator}!\idxcode{operator>}}%
\begin{itemdecl}
template <class I1, class I2>
    requires StrictTotallyOrderedWith<I1, I2>
  constexpr bool operator>(
    const move_iterator<I1>& x, const move_iterator<I2>& y);
\end{itemdecl}

\begin{itemdescr}
\pnum
\effects Equivalent to:
\tcode{return y < x;}
\end{itemdescr}

\indexlibrary{\idxcode{operator>=}!\idxcode{move_iterator}}%
\indexlibrary{\idxcode{move_iterator}!\idxcode{operator>=}}%
\begin{itemdecl}
template <class I1, class I2>
    requires StrictTotallyOrderedWith<I1, I2>
  constexpr bool operator>=(
    const move_iterator<I1>& x, const move_iterator<I2>& y);
\end{itemdecl}

\begin{itemdescr}
\pnum
\effects Equivalent to:
\tcode{return !(x < y);}.
\end{itemdescr}

\rSec4[std2.move.iter.nonmember]{\tcode{move_iterator} non-member functions}

\indexlibrary{\idxcode{operator-}!\idxcode{move_iterator}}%
\indexlibrary{\idxcode{move_iterator}!\idxcode{operator-}}%
\begin{itemdecl}
template <class I1, class I2>
    requires SizedSentinel<I1, I2>
  constexpr difference_type_t<I2> operator-(
    const move_iterator<I1>& x,
    const move_iterator<I2>& y);
\end{itemdecl}

\begin{itemdescr}
\pnum
\effects Equivalent to:
\tcode{return x.current - y.current;}
\end{itemdescr}

\indexlibrary{\idxcode{operator+}!\idxcode{move_iterator}}%
\indexlibrary{\idxcode{move_iterator}!\idxcode{operator+}}%
\begin{itemdecl}
template <RandomAccessIterator I>
  constexpr move_iterator<I> operator+(
    difference_type_t<I> n,
    const move_iterator<I>& x);
\end{itemdecl}

\begin{itemdescr}
\pnum
\effects Equivalent to:
\tcode{return x + n;}
\end{itemdescr}

\indexlibrary{\idxcode{iter_move}!\idxcode{move_iterator}}%
\indexlibrary{\idxcode{move_iterator}!\idxcode{iter_move}}%
\begin{itemdecl}
friend constexpr rvalue_reference_t<I> iter_move(const move_iterator& i)
  noexcept(@\seebelow@);
\end{itemdecl}

\begin{itemdescr}
\pnum
\effects Equivalent to: \tcode{return \changed{ranges}{::std2}::iter_move(i.current);}

\pnum
\remarks The expression in \tcode{noexcept} is equivalent to:
\begin{codeblock}
noexcept(@\changed{ranges}{::std2}@::iter_move(i.current))
\end{codeblock}
\end{itemdescr}

\indexlibrary{\idxcode{iter_swap}!\idxcode{move_iterator}}%
\indexlibrary{\idxcode{move_iterator}!\idxcode{iter_swap}}%
\begin{itemdecl}
template <IndirectlySwappable<I> I2>
  friend constexpr void iter_swap(const move_iterator& x, const move_iterator<I2>& y)
    noexcept(@\seebelow@);
\end{itemdecl}

\begin{itemdescr}
\pnum
\effects Equivalent to: \tcode{\changed{ranges}{::std2}::iter_swap(x.current, y.current)}.

\pnum
\remarks The expression in \tcode{noexcept} is equivalent to:
\begin{codeblock}
noexcept(@\changed{ranges}{::std2}@::iter_swap(x.current, y.current))
\end{codeblock}
\end{itemdescr}

\indexlibrary{\idxcode{make_move_iterator}}%
\begin{itemdecl}
template <InputIterator I>
  constexpr move_iterator<I> make_move_iterator(I i);
\end{itemdecl}

\begin{itemdescr}
\pnum
\returns \tcode{move_iterator<I>(i)}.
\end{itemdescr}

\rSec3[std2.move.sentinel]{Class template \tcode{move_sentinel}}

\pnum
Class template \tcode{move_sentinel} is a sentinel adaptor useful for denoting
ranges together with \tcode{move_iterator}. When an input iterator type
\tcode{I} and sentinel type \tcode{S} satisfy \tcode{Sentinel<S, I>},
\tcode{Sentinel<move_sentinel<S>, move_iterator<I>{>}} is satisfied as well.

\pnum
\enterexample A \tcode{move_if} algorithm is easily implemented with
\tcode{copy_if} using \tcode{move_iterator} and \tcode{move_sentinel}:

\begin{codeblock}
template <InputIterator I, Sentinel<I> S, WeaklyIncrementable O,
          IndirectUnaryPredicate<I> Pred>
  requires IndirectlyMovable<I, O>
void move_if(I first, S last, O out, Pred pred)
{
  copy_if(move_iterator<I>{first}, move_sentinel<S>{last}, out, pred);
}
\end{codeblock}

\exitexample

\indexlibrary{\idxcode{move_sentinel}}%
\begin{codeblock}
namespace @\changed{std \{ namespace experimental \{ namespace ranges}{std2}@ { inline namespace v1 {
  template <Semiregular S>
  class move_sentinel {
  public:
    constexpr move_sentinel();
    explicit move_sentinel(S s);
    @\added{template <ConvertibleTo<S> U>}@
      move_sentinel(const move_sentinel<@\changed{ConvertibleTo<S>}{U}@>& s);
    @\added{template <ConvertibleTo<S> U>}@
      move_sentinel& operator=(const move_sentinel<@\changed{ConvertibleTo<S>}{U}@>& s);

    S base() const;

  private:
    S last; // \expos
  };

  template <class I, Sentinel<I> S>
    constexpr bool operator==(
      const move_iterator<I>& i, const move_sentinel<S>& s);
  template <class I, Sentinel<I> S>
    constexpr bool operator==(
      const move_sentinel<S>& s, const move_iterator<I>& i);
  template <class I, Sentinel<I> S>
    constexpr bool operator!=(
      const move_iterator<I>& i, const move_sentinel<S>& s);
  template <class I, Sentinel<I> S>
    constexpr bool operator!=(
      const move_sentinel<S>& s, const move_iterator<I>& i);

  template <class I, SizedSentinel<I> S>
    constexpr difference_type_t<I> operator-(
      const move_sentinel<S>& s, const move_iterator<I>& i);
  template <class I, SizedSentinel<I> S>
    constexpr difference_type_t<I> operator-(
      const move_iterator<I>& i, const move_sentinel<S>& s);

  template <Semiregular S>
    constexpr move_sentinel<S> make_move_sentinel(S s);
}}@\removed{\}\}}@
\end{codeblock}

\rSec3[std2.move.sent.ops]{\tcode{move_sentinel} operations}

\rSec4[std2.move.sent.op.const]{\tcode{move_sentinel} constructors}

\indexlibrary{\idxcode{move_sentinel}!\idxcode{move_sentinel}}%
\begin{itemdecl}
constexpr move_sentinel();
\end{itemdecl}

\begin{itemdescr}
\pnum
\effects Constructs a \tcode{move_sentinel}, value-initializing
\tcode{last}. If \tcode{is_\-trivially_\-default_\-constr\-uct\-ible\-\added{_v}<\brk{}S>\removed{::value}} is \tcode{true}, then this constructor
is a \tcode{constexpr} constructor.
\end{itemdescr}

\indexlibrary{\idxcode{move_sentinel}!constructor}%
\begin{itemdecl}
explicit move_sentinel(S s);
\end{itemdecl}

\begin{itemdescr}
\pnum
\effects Constructs a \tcode{move_sentinel}, initializing
\tcode{last} with \tcode{s}.
\end{itemdescr}

\indexlibrary{\idxcode{move_sentinel}!constructor}%
\begin{itemdecl}
@\added{template <ConvertibleTo<S> U>}@
  move_sentinel(const move_sentinel<@\changed{ConvertibleTo<S>}{U}@>& s);
\end{itemdecl}

\begin{itemdescr}
\pnum
\effects Constructs a \tcode{move_sentinel}, initializing
\tcode{last} with \tcode{s.last}.
\end{itemdescr}

\rSec4[std2.move.sent.op=]{\tcode{move_sentinel::operator=}}

\indexlibrary{\idxcode{operator=}!\idxcode{move_sentinel}}%
\indexlibrary{\idxcode{move_sentinel}!\idxcode{operator=}}%
\begin{itemdecl}
@\added{template <ConvertibleTo<S> U>}@
  move_sentinel& operator=(const move_sentinel<@\changed{ConvertibleTo<S>}{U}@>& s);
\end{itemdecl}

\begin{itemdescr}
\pnum
\effects Assigns \tcode{s.last} to \tcode{last}.

\pnum
\returns \tcode{*this}.
\end{itemdescr}

\rSec4[std2.move.sent.op.comp]{\tcode{move_sentinel} comparisons}

\indexlibrary{\idxcode{operator==}!\idxcode{move_sentinel}}%
\indexlibrary{\idxcode{move_sentinel}!\idxcode{operator==}}%
\begin{itemdecl}
template <class I, Sentinel<I> S>
  constexpr bool operator==(
    const move_iterator<I>& i, const move_sentinel<S>& s);
template <class I, Sentinel<I> S>
  constexpr bool operator==(
    const move_sentinel<S>& s, const move_iterator<I>& i);
\end{itemdecl}

\begin{itemdescr}
\pnum
\effects Equivalent to: \tcode{return i.current == s.last;}
\end{itemdescr}

\indexlibrary{\idxcode{operator"!=}!\idxcode{move_sentinel}}%
\indexlibrary{\idxcode{move_sentinel}!\idxcode{operator"!=}}%
\begin{itemdecl}
template <class I, Sentinel<I> S>
  constexpr bool operator!=(
    const move_iterator<I>& i, const move_sentinel<S>& s);
template <class I, Sentinel<I> S>
  constexpr bool operator!=(
    const move_sentinel<S>& s, const move_iterator<I>& i);
\end{itemdecl}

\begin{itemdescr}
\pnum
\effects Equivalent to: \tcode{return !(i == s);}
\end{itemdescr}

\rSec4[std2.move.sent.nonmember]{\tcode{move_sentinel} non-member functions}

\indexlibrary{\idxcode{operator-}!\idxcode{move_sentinel}}%
\indexlibrary{\idxcode{move_sentinel}!\idxcode{operator-}}%
\begin{itemdecl}
template <class I, SizedSentinel<I> S>
  constexpr difference_type_t<I> operator-(
    const move_sentinel<S>& s, const move_iterator<I>& i);
\end{itemdecl}

\begin{itemdescr}
\pnum
\effects Equivalent to: \tcode{return s.last - i.current;}
\end{itemdescr}

\begin{itemdecl}
template <class I, SizedSentinel<I> S>
  constexpr difference_type_t<I> operator-(
    const move_iterator<I>& i, const move_sentinel<S>& s);
\end{itemdecl}

\begin{itemdescr}
\pnum
\effects Equivalent to: \tcode{return i.current - s.last;}
\end{itemdescr}

\indexlibrary{\idxcode{make_move_sentinel}}%
\begin{itemdecl}
template <Semiregular S>
  constexpr move_sentinel<S> make_move_sentinel(S s);
\end{itemdecl}

\begin{itemdescr}
\pnum
\returns \tcode{move_sentinel<S>(s)}.
\end{itemdescr}

\rSec2[std2.iterators.common]{Common iterators}

\ednote{TODO: respecify this in terms of \tcode{std::variant}.}

\pnum
Class template \tcode{common_iterator} is an iterator/sentinel adaptor that is
capable of representing a non-bounded range of elements (where the types of the
iterator and sentinel differ) as a bounded range (where they are the same). It
does this by holding either an iterator or a sentinel, and implementing the
equality comparison operators appropriately.

\pnum
\enternote The \tcode{common_iterator} type is useful for interfacing with legacy
code that expects the begin and end of a range to have the same type.\exitnote

\pnum
\enterexample
\begin{codeblock}
template <class ForwardIterator>
void fun(ForwardIterator begin, ForwardIterator end);

list<int> s;
// populate the list \tcode{s}
using CI =
  common_iterator<counted_iterator<list<int>::iterator>,
                  default_sentinel>;
// call \tcode{fun} on a range of 10 ints
fun(CI(make_counted_iterator(s.begin(), 10)),
    CI(default_sentinel()));
\end{codeblock}
\exitexample

\rSec3[std2.common.iterator]{Class template \tcode{common_iterator}}

\indexlibrary{\idxcode{common_iterator}}%
\begin{codeblock}
namespace @\changed{std \{ namespace experimental \{ namespace ranges}{std2}@ { inline namespace v1 {
  template <Iterator I, Sentinel<I> S>
    requires !Same<I, S>
  class common_iterator {
  public:
    using difference_type = difference_type_t<I>;

    constexpr common_iterator();
    constexpr common_iterator(I i);
    constexpr common_iterator(S s);
    @\removed{constexpr common_iterator(const common_iterator<ConvertibleTo<I>, ConvertibleTo<S>>\& u);}@
    @\added{template <ConvertibleTo<I> II, ConvertibleTo<S> SS>}@
      @\added{constexpr common_iterator(const common_iterator<II, SS>\& u);}@
    @\removed{common_iterator\& operator=(const common_iterator<ConvertibleTo<I>, ConvertibleTo<S>>\& u);}@
    @\added{template <ConvertibleTo<I> II, ConvertibleTo<S> SS>}@
      @\added{common_iterator\& operator=(const common_iterator<II, SS>\& u);}@

    decltype(auto) operator*();
    decltype(auto) operator*() const
      requires @\placeholder{dereferenceable}@<const I>;
    decltype(auto) operator->() const
      requires @\seebelow@;

    common_iterator& operator++();
    decltype(auto) operator++(int);
    common_iterator operator++(int)
      requires ForwardIterator<I>;

    friend rvalue_reference_t<I> iter_move(const common_iterator& i)
      noexcept(@\seebelow@)
        requires InputIterator<I>;
    template <IndirectlySwappable<I> I2, class S2>
      friend void iter_swap(const common_iterator& x, const common_iterator<I2, S2>& y)
        noexcept(@\seebelow@);

  private:
    bool is_sentinel; // \expos
    I iter;           // \expos
    S sentinel;       // \expos
  };

  template <Readable I, class S>
  struct value_type<common_iterator<I, S>> {
    using type = value_type_t<I>;
  };

  template <InputIterator I, class S>
  struct iterator_category<common_iterator<I, S>> {
    using type = input_iterator_tag;
  };

  template <ForwardIterator I, class S>
  struct iterator_category<common_iterator<I, S>> {
    using type = forward_iterator_tag;
  };

  template <class I1, class I2, Sentinel<I2> S1, Sentinel<I1> S2>
  bool operator==(
    const common_iterator<I1, S1>& x, const common_iterator<I2, S2>& y);
  template <class I1, class I2, Sentinel<I2> S1, Sentinel<I1> S2>
    requires EqualityComparableWith<I1, I2>
  bool operator==(
    const common_iterator<I1, S1>& x, const common_iterator<I2, S2>& y);
  template <class I1, class I2, Sentinel<I2> S1, Sentinel<I1> S2>
  bool operator!=(
    const common_iterator<I1, S1>& x, const common_iterator<I2, S2>& y);

  template <class I2, SizedSentinel<I2> I1, SizedSentinel<I2> S1, SizedSentinel<I1> S2>
  difference_type_t<I2> operator-(
    const common_iterator<I1, S1>& x, const common_iterator<I2, S2>& y);
}}@\removed{\}\}}@
\end{codeblock}

\rSec3[std2.common.iter.ops]{\tcode{common_iterator} operations}

\rSec4[std2.common.iter.op.const]{\tcode{common_iterator} constructors}

\indexlibrary{\idxcode{common_iterator}!\idxcode{common_iterator}}%
\begin{itemdecl}
constexpr common_iterator();
\end{itemdecl}

\begin{itemdescr}
\pnum
\effects Constructs a \tcode{common_iterator}, value-initializing
\tcode{is_sentinel}, \tcode{iter}, and \tcode{sentinel}. Iterator operations
applied to the resulting iterator have defined behavior if and only if the
corresponding operations are defined on a value-initialized iterator of type
\tcode{I}.
\end{itemdescr}

\indexlibrary{\idxcode{common_iterator}!constructor}%
\begin{itemdecl}
constexpr common_iterator(I i);
\end{itemdecl}

\begin{itemdescr}
\pnum
\effects Constructs a \tcode{common_iterator}, initializing
\tcode{is_sentinel} with \tcode{false}, \tcode{iter} with \tcode{i}, and
value-initializing \tcode{sentinel}.
\end{itemdescr}

\indexlibrary{\idxcode{common_iterator}!constructor}%
\begin{itemdecl}
constexpr common_iterator(S s);
\end{itemdecl}

\begin{itemdescr}
\pnum
\effects Constructs a \tcode{common_iterator}, initializing
\tcode{is_sentinel} with \tcode{true}, value-initializing \tcode{iter}, and
initializing \tcode{sentinel} with \tcode{s}.
\end{itemdescr}

\indexlibrary{\idxcode{common_iterator}!constructor}%
\begin{itemdecl}
@\removed{constexpr common_iterator(const common_iterator<ConvertibleTo<I>, ConvertibleTo<S>>\& u);}@
@\added{template <ConvertibleTo<I> II, ConvertibleTo<S> SS>}@
  @\added{constexpr common_iterator(const common_iterator<II, SS>\& u);}@
\end{itemdecl}

\begin{itemdescr}
\pnum
\effects Constructs a \tcode{common_iterator}, initializing \tcode{is_sentinel}
with \tcode{u.is_sentinel}, \tcode{iter} with \tcode{u.iter}, and \tcode{sentinel}
with \tcode{u.sentinel}.
\end{itemdescr}

\rSec4[std2.common.iter.op=]{\tcode{common_iterator::operator=}}

\indexlibrary{\idxcode{operator=}!\idxcode{common_iterator}}%
\indexlibrary{\idxcode{common_iterator}!\idxcode{operator=}}%
\begin{itemdecl}
    @\removed{common_iterator\& operator=(const common_iterator<ConvertibleTo<I>, ConvertibleTo<S>>\& u);}@
    @\added{template <ConvertibleTo<I> II, ConvertibleTo<S> SS>}@
      @\added{common_iterator\& operator=(const common_iterator<II, SS>\& u);}@
\end{itemdecl}

\begin{itemdescr}
\pnum
\effects Assigns \tcode{u.is_sentinel} to \tcode{is_sentinel}, \tcode{u.iter} to
\tcode{iter}, and \tcode{u.sentinel} to \tcode{sentinel}.

\pnum
\returns \tcode{*this}
\end{itemdescr}

\rSec4[std2.common.iter.op.star]{\tcode{common_iterator::operator*}}

\indexlibrary{\idxcode{operator*}!\idxcode{common_iterator}}%
\indexlibrary{\idxcode{common_iterator}!\idxcode{operator*}}%
\begin{itemdecl}
decltype(auto) operator*();
decltype(auto) operator*() const
  requires @\placeholder{dereferenceable}@<const I>;
\end{itemdecl}

\begin{itemdescr}
\pnum
\requires \tcode{!is_sentinel}

\pnum
\effects Equivalent to: \tcode{return *iter;}
\end{itemdescr}

\rSec4[std2.common.iter.op.ref]{\tcode{common_iterator::operator->}}

\indexlibrary{\idxcode{operator->}!\idxcode{common_iterator}}%
\indexlibrary{\idxcode{common_iterator}!\idxcode{operator->}}%
\begin{itemdecl}
decltype(auto) operator->() const
  requires @\seebelow@;
\end{itemdecl}

\begin{itemdescr}
\pnum
\requires \tcode{!is_sentinel}

\pnum
\effects Equivalent to:
\begin{itemize}
\item
If \tcode{I} is a pointer type or if the expression \tcode{i.operator->()} is
well-formed, \tcode{return iter;}

\item
Otherwise, if the expression \tcode{*iter} is a glvalue:
\begin{codeblock}
auto&& tmp = *iter;
return addressof(tmp);
\end{codeblock}

\item
Otherwise, \tcode{return proxy(*iter);} where \tcode{proxy} is the exposition-only class:
\begin{codeblock}
class proxy {               // \expos
  value_type_t<I> keep_;
  proxy(reference_t<I>&& x)
    : keep_(std::move(x)) {}
public:
  const value_type_t<I>* operator->() const {
    return addressof(keep_);
  }
};
\end{codeblock}
\end{itemize}

\pnum
The expression in the requires clause is equivalent to:
\begin{codeblock}
Readable<const I> &&
  (requires(const I& i) { i.operator->(); } ||
   is_reference@\added{_v}@<reference_t<I>>@\removed{::value}@ ||
   Constructible<value_type_t<I>, reference_t<I>>)
\end{codeblock}
\end{itemdescr}


\rSec4[std2.common.iter.op.incr]{\tcode{common_iterator::operator++}}

\indexlibrary{\idxcode{operator++}!\idxcode{common_iterator}}%
\indexlibrary{\idxcode{common_iterator}!\idxcode{operator++}}%
\begin{itemdecl}
common_iterator& operator++();
\end{itemdecl}

\begin{itemdescr}
\pnum
\requires \tcode{!is_sentinel}

\pnum
\effects Equivalent to \tcode{++iter}.

\pnum
\returns \tcode{*this}.
\end{itemdescr}

\indexlibrary{\idxcode{operator++}!\idxcode{common_iterator}}%
\indexlibrary{\idxcode{common_iterator}!\idxcode{operator++}}%
\begin{itemdecl}
decltype(auto) operator++(int);
\end{itemdecl}

\begin{itemdescr}
\pnum
\requires \tcode{!is_sentinel}.

\pnum
\effects Equivalent to: \tcode{return iter++;}
\end{itemdescr}

\begin{itemdecl}
common_iterator operator++(int)
  requires ForwardIterator<I>;
\end{itemdecl}

\begin{itemdescr}
\pnum
\requires \tcode{!is_sentinel}

\pnum
\effects Equivalent to:
\begin{codeblock}
common_iterator tmp = *this;
++iter;
return tmp;
\end{codeblock}
\end{itemdescr}

\rSec4[std2.common.iter.op.comp]{\tcode{common_iterator} comparisons}

\indexlibrary{\idxcode{operator==}!\idxcode{common_iterator}}%
\indexlibrary{\idxcode{common_iterator}!\idxcode{operator==}}%
\begin{itemdecl}
template <class I1, class I2, Sentinel<I2> S1, Sentinel<I1> S2>
bool operator==(
  const common_iterator<I1, S1>& x, const common_iterator<I2, S2>& y);
\end{itemdecl}

\begin{itemdescr}
\pnum
\effects Equivalent to:
\begin{codeblock}
  return x.is_sentinel ?
    (y.is_sentinel || y.iter == x.sentinel) :
    (!y.is_sentinel || x.iter == y.sentinel);
\end{codeblock}
\end{itemdescr}

\indexlibrary{\idxcode{operator==}!\idxcode{common_iterator}}%
\indexlibrary{\idxcode{common_iterator}!\idxcode{operator==}}%
\begin{itemdecl}
template <class I1, class I2, Sentinel<I2> S1, Sentinel<I1> S2>
  requires EqualityComparableWith<I1, I2>
bool operator==(
  const common_iterator<I1, S1>& x, const common_iterator<I2, S2>& y);
\end{itemdecl}

\begin{itemdescr}
\pnum
\effects Equivalent to:
\begin{codeblock}
  return x.is_sentinel ?
    (y.is_sentinel || y.iter == x.sentinel) :
    (y.is_sentinel ?
        x.iter == y.sentinel :
        x.iter == y.iter);
\end{codeblock}
\end{itemdescr}

\indexlibrary{\idxcode{operator"!=}!\idxcode{common_iterator}}%
\indexlibrary{\idxcode{common_iterator}!\idxcode{operator"!=}}%
\begin{itemdecl}
template <class I1, class I2, Sentinel<I2> S1, Sentinel<I1> S2>
bool operator!=(
  const common_iterator<I1, S1>& x, const common_iterator<I2, S2>& y);
\end{itemdecl}

\begin{itemdescr}
\pnum
\effects Equivalent to:
\tcode{return !(x == y);}
\end{itemdescr}

\indexlibrary{\idxcode{operator-}!\idxcode{common_iterator}}%
\indexlibrary{\idxcode{common_iterator}!\idxcode{operator-}}%
\begin{itemdecl}
template <class I2, SizedSentinel<I2> I1, SizedSentinel<I2> S1, SizedSentinel<I1> S2>
difference_type_t<I2> operator-(
  const common_iterator<I1, S1>& x, const common_iterator<I2, S2>& y);
\end{itemdecl}

\begin{itemdescr}
\pnum
\effects Equivalent to:
\begin{codeblock}
  return x.is_sentinel ?
    (y.is_sentinel ? 0 : x.sentinel - y.iter) :
    (y.is_sentinel ?
         x.iter - y.sentinel :
         x.iter - y.iter);
\end{codeblock}
\end{itemdescr}

\rSec4[std2.common.iter.op.iter_move]{\tcode{iter_move}}

\indexlibrary{\idxcode{iter_move}!\idxcode{common_iterator}}%
\indexlibrary{\idxcode{common_iterator}!\idxcode{iter_move}}%
\begin{itemdecl}
friend rvalue_reference_t<I> iter_move(const common_iterator& i)
  noexcept(@\seebelow@)
    requires InputIterator<I>;
\end{itemdecl}

\begin{itemdescr}
\pnum
\requires \tcode{!i.is_sentinel}.

\pnum
\effects Equivalent to: \tcode{return \changed{ranges}{::std2}::iter_move(i.iter);}

\pnum
\remarks The expression in \tcode{noexcept} is equivalent to:
\begin{codeblock}
noexcept(@\changed{ranges}{::std2}@::iter_move(i.iter))
\end{codeblock}
\end{itemdescr}

\rSec4[std2.common.iter.op.iter_swap]{\tcode{iter_swap}}

\indexlibrary{\idxcode{iter_swap}!\idxcode{common_iterator}}%
\indexlibrary{\idxcode{common_iterator}!\idxcode{iter_swap}}%
\begin{itemdecl}
template <IndirectlySwappable<I> I2>
  friend void iter_swap(const common_iterator& x, const common_iterator<I2>& y)
    noexcept(@\seebelow@);
\end{itemdecl}

\begin{itemdescr}
\pnum
\requires \tcode{!x.is_sentinel \&\& !y.is_sentinel}.

\pnum
\effects Equivalent to \tcode{\changed{ranges}{::std2}::iter_swap(x.iter, y.iter)}.

\pnum
\remarks The expression in \tcode{noexcept} is equivalent to:
\begin{codeblock}
noexcept(@\changed{ranges}{::std2}@::iter_swap(x.iter, y.iter))
\end{codeblock}
\end{itemdescr}


\rSec2[std2.default.sentinels]{Default sentinels}

\rSec3[std2.default.sent]{Class \tcode{default_sentinel}}

\indexlibrary{\idxcode{default_sentinel}}%
\begin{itemdecl}
namespace @\changed{std \{ namespace experimental \{ namespace ranges}{std2}@ { inline namespace v1 {
  class default_sentinel { };
}}@\removed{\}\}}@
\end{itemdecl}

\pnum
Class \tcode{default_sentinel} is an empty type used to denote the end of a
range. It is intended to be used together with iterator types that know the bound
of their range (e.g., \tcode{counted_iterator}~(\ref{std2.counted.iterator})).

\rSec2[std2.iterators.counted]{Counted iterators}

\rSec3[std2.counted.iterator]{Class template \tcode{counted_iterator}}

\pnum
Class template \tcode{counted_iterator} is an iterator adaptor
with the same behavior as the underlying iterator except that it
keeps track of its distance from its starting position. It can be
used together with class \tcode{default_sentinel} in calls to generic
algorithms to operate on a range of $N$ elements starting at a given
position without needing to know the end position \textit{a priori}.

\pnum
\enterexample

\begin{codeblock}
list<string> s;
// populate the list \tcode{s} with at least 10 strings
vector<string> v(make_counted_iterator(s.begin(), 10),
                 default_sentinel()); // copies 10 strings into \tcode{v}
\end{codeblock}
\exitexample

\pnum
Two values \tcode{i1} and \tcode{i2} of (possibly differing) types
\tcode{counted_iterator<I1>} and \tcode{counted_iterator<I2>} refer to
elements of the same sequence if and only if \tcode{next(i1.base(), i1.count())}
and \tcode{next(\brk{}i2.\brk{}base(), i2.count())} refer to the same (possibly past-the-end) element.

\indexlibrary{\idxcode{counted_iterator}}%
\begin{codeblock}
namespace @\changed{std \{ namespace experimental \{ namespace ranges}{std2}@ { inline namespace v1 {
  template <Iterator I>
  class counted_iterator {
  public:
    using iterator_type = I;
    using difference_type = difference_type_t<I>;

    constexpr counted_iterator();
    constexpr counted_iterator(I x, difference_type_t<I> n);
    @\added{template <ConvertibleTo<I> U>}@
      constexpr counted_iterator(const counted_iterator<@\changed{ConvertibleTo<I>}{U}@>& i);
    @\added{template <ConvertibleTo<I> U>}@
      constexpr counted_iterator& operator=(const counted_iterator<@\changed{ConvertibleTo<I>}{U}@>& i);

    constexpr I base() const;
    constexpr difference_type_t<I> count() const;
    constexpr decltype(auto) operator*();
    constexpr decltype(auto) operator*() const
      requires @\placeholder{dereferenceable}@<const I>;

    constexpr counted_iterator& operator++();
    decltype(auto) operator++(int);
    constexpr counted_iterator operator++(int)
      requires ForwardIterator<I>;
    constexpr counted_iterator& operator--()
      requires BidirectionalIterator<I>;
    constexpr counted_iterator operator--(int)
      requires BidirectionalIterator<I>;

    constexpr counted_iterator  operator+ (difference_type n) const
      requires RandomAccessIterator<I>;
    constexpr counted_iterator& operator+=(difference_type n)
      requires RandomAccessIterator<I>;
    constexpr counted_iterator  operator- (difference_type n) const
      requires RandomAccessIterator<I>;
    constexpr counted_iterator& operator-=(difference_type n)
      requires RandomAccessIterator<I>;
    constexpr decltype(auto) operator[](difference_type n) const
      requires RandomAccessIterator<I>;

    friend constexpr rvalue_reference_t<I> iter_move(const counted_iterator& i)
      noexcept(@\seebelow@)
        requires InputIterator<I>;
    template <IndirectlySwappable<I> I2>
      friend constexpr void iter_swap(const counted_iterator& x, const counted_iterator<I2>& y)
        noexcept(@\seebelow@);

  private:
    I current; // \expos
    difference_type_t<I> cnt; // \expos
  };

  template <Readable I>
  struct value_type<counted_iterator<I>> {
    using type = value_type_t<I>;
  };

  template <InputIterator I>
  struct iterator_category<counted_iterator<I>> {
    using type = iterator_category_t<I>;
  };

  template <class I1, class I2>
      requires Common<I1, I2>
    constexpr bool operator==(
      const counted_iterator<I1>& x, const counted_iterator<I2>& y);
  @\added{template <class I>}@
    constexpr bool operator==(
      const counted_iterator<@\changed{auto}{I}@>& x, default_sentinel);
  @\added{template <class I>}@
    constexpr bool operator==(
      default_sentinel, const counted_iterator<@\changed{auto}{I}@>& x);

  template <class I1, class I2>
      requires Common<I1, I2>
    constexpr bool operator!=(
      const counted_iterator<I1>& x, const counted_iterator<I2>& y);
  @\added{template <class I>}@
      constexpr bool operator!=(
      const counted_iterator<@\changed{auto}{I}@>& x, default_sentinel y);
  @\added{template <class I>}@
    constexpr bool operator!=(
      default_sentinel x, const counted_iterator<@\changed{auto}{I}@>& y);

  template <class I1, class I2>
      requires Common<I1, I2>
    constexpr bool operator<(
      const counted_iterator<I1>& x, const counted_iterator<I2>& y);
  template <class I1, class I2>
      requires Common<I1, I2>
    constexpr bool operator<=(
      const counted_iterator<I1>& x, const counted_iterator<I2>& y);
  template <class I1, class I2>
      requires Common<I1, I2>
    constexpr bool operator>(
      const counted_iterator<I1>& x, const counted_iterator<I2>& y);
  template <class I1, class I2>
      requires Common<I1, I2>
    constexpr bool operator>=(
      const counted_iterator<I1>& x, const counted_iterator<I2>& y);
  template <class I1, class I2>
      requires Common<I1, I2>
    constexpr difference_type_t<I2> operator-(
      const counted_iterator<I1>& x, const counted_iterator<I2>& y);
  template <class I>
    constexpr difference_type_t<I> operator-(
      const counted_iterator<I>& x, default_sentinel y);
  template <class I>
    constexpr difference_type_t<I> operator-(
      default_sentinel x, const counted_iterator<I>& y);

  template <RandomAccessIterator I>
    constexpr counted_iterator<I> operator+(
      difference_type_t<I> n, const counted_iterator<I>& x);

  template <Iterator I>
    constexpr counted_iterator<I> make_counted_iterator(I i, difference_type_t<I> n);
}}@\removed{\}\}}@
\end{codeblock}

\rSec3[std2.counted.iter.ops]{\tcode{counted_iterator} operations}

\rSec4[std2.counted.iter.op.const]{\tcode{counted_iterator} constructors}

\indexlibrary{\idxcode{counted_iterator}!\idxcode{counted_iterator}}%
\begin{itemdecl}
constexpr counted_iterator();
\end{itemdecl}

\begin{itemdescr}
\pnum
\effects Constructs a \tcode{counted_iterator}, value-initializing
\tcode{current} and \tcode{cnt}. Iterator operations applied to the
resulting iterator have defined behavior if and only if the corresponding operations
are defined on a value-initialized iterator of type \tcode{I}.
\end{itemdescr}

\indexlibrary{\idxcode{counted_iterator}!constructor}%
\begin{itemdecl}
constexpr counted_iterator(I i, difference_type_t<I> n);
\end{itemdecl}

\begin{itemdescr}
\pnum
\requires \tcode{n >= 0}

\pnum
\effects Constructs a \tcode{counted_iterator}, initializing
\tcode{current} with \tcode{i} and \tcode{cnt} with \tcode{n}.
\end{itemdescr}

\indexlibrary{\idxcode{counted_iterator}!constructor}%
\begin{itemdecl}
@\added{template <ConvertibleTo<I> U>}@
  constexpr counted_iterator(const counted_iterator<@\changed{ConvertibleTo<I>}{U}@>& i);
\end{itemdecl}

\begin{itemdescr}
\pnum
\effects Constructs a \tcode{counted_iterator}, initializing
\tcode{current} with \tcode{i.current} and \tcode{cnt} with \tcode{i.cnt}.
\end{itemdescr}

\rSec4[std2.counted.iter.op=]{\tcode{counted_iterator::operator=}}

\indexlibrary{\idxcode{operator=}!\idxcode{counted_iterator}}%
\indexlibrary{\idxcode{counted_iterator}!\idxcode{operator=}}%
\begin{itemdecl}
@\added{template <ConvertibleTo<I> U>}@
  constexpr counted_iterator& operator=(const counted_iterator<@\changed{ConvertibleTo<I>}{U}@>& i);
\end{itemdecl}

\begin{itemdescr}
\pnum
\effects Assigns \tcode{i.current} to
\tcode{current} and \tcode{i.cnt} to \tcode{cnt}.

\end{itemdescr}

\rSec4[std2.counted.iter.op.conv]{\tcode{counted_iterator} conversion}

\indexlibrary{\idxcode{base}!\idxcode{counted_iterator}}%
\indexlibrary{\idxcode{counted_iterator}!\idxcode{base}}%
\begin{itemdecl}
constexpr I base() const;
\end{itemdecl}

\begin{itemdescr}
\pnum
\returns \tcode{current}.
\end{itemdescr}

\rSec4[std2.counted.iter.op.cnt]{\tcode{counted_iterator} count}

\indexlibrary{\idxcode{count}!\idxcode{counted_iterator}}%
\indexlibrary{\idxcode{counted_iterator}!\idxcode{count}}%
\begin{itemdecl}
constexpr difference_type_t<I> count() const;
\end{itemdecl}

\begin{itemdescr}
\pnum
\returns \tcode{cnt}.
\end{itemdescr}

\rSec4[std2.counted.iter.op.star]{\tcode{counted_iterator::operator*}}

\indexlibrary{\idxcode{operator*}!\idxcode{counted_iterator}}%
\indexlibrary{\idxcode{counted_iterator}!\idxcode{operator*}}%
\begin{itemdecl}
constexpr decltype(auto) operator*();
constexpr decltype(auto) operator*() const
  requires @\placeholder{dereferenceable}@<const I>;
\end{itemdecl}

\begin{itemdescr}
\pnum
\effects Equivalent to:
\tcode{return *current;}
\end{itemdescr}

\rSec4[std2.counted.iter.op.incr]{\tcode{counted_iterator::operator++}}

\indexlibrary{\idxcode{operator++}!\idxcode{counted_iterator}}%
\indexlibrary{\idxcode{counted_iterator}!\idxcode{operator++}}%
\begin{itemdecl}
constexpr counted_iterator& operator++();
\end{itemdecl}

\begin{itemdescr}
\pnum
\requires \tcode{cnt > 0}

\pnum
\effects Equivalent to:
\begin{codeblock}
++current;
@\dcr@cnt;
\end{codeblock}

\pnum
\returns \tcode{*this}.
\end{itemdescr}

\indexlibrary{\idxcode{operator++}!\idxcode{counted_iterator}}%
\indexlibrary{\idxcode{counted_iterator}!\idxcode{operator++}}%
\begin{itemdecl}
decltype(auto) operator++(int);
\end{itemdecl}

\begin{itemdescr}
\pnum
\requires \tcode{cnt > 0}.

\pnum
\effects Equivalent to:
\begin{codeblock}
--cnt;
try { return current++; }
catch(...) { ++cnt; throw; }
\end{codeblock}
\end{itemdescr}

\begin{itemdecl}
constexpr counted_iterator operator++(int)
  requires ForwardIterator<I>;
\end{itemdecl}

\begin{itemdescr}
\pnum
\requires \tcode{cnt > 0}

\pnum
\effects Equivalent to:
\begin{codeblock}
counted_iterator tmp = *this;
++*this;
return tmp;
\end{codeblock}
\end{itemdescr}

\rSec4[std2.counted.iter.op.decr]{\tcode{counted_iterator::operator-{-}}}

\indexlibrary{\idxcode{operator\dcr}!\idxcode{counted_iterator}}%
\indexlibrary{\idxcode{counted_iterator}!\idxcode{operator\dcr}}%
\begin{itemdecl}
  constexpr counted_iterator& operator--();
    requires BidirectionalIterator<I>
\end{itemdecl}

\begin{itemdescr}
\pnum
\effects Equivalent to:
\begin{codeblock}
--current;
++cnt;
\end{codeblock}

\pnum
\returns \tcode{*this}.
\end{itemdescr}

\indexlibrary{\idxcode{operator\dcr}!\idxcode{counted_iterator}}%
\indexlibrary{\idxcode{counted_iterator}!\idxcode{operator\dcr}}%
\begin{itemdecl}
  constexpr counted_iterator operator--(int)
    requires BidirectionalIterator<I>;
\end{itemdecl}

\begin{itemdescr}
\pnum
\effects Equivalent to:
\begin{codeblock}
counted_iterator tmp = *this;
--*this;
return tmp;
\end{codeblock}
\end{itemdescr}

\rSec4[std2.counted.iter.op.+]{\tcode{counted_iterator::operator+}}

\indexlibrary{\idxcode{operator+}!\idxcode{counted_iterator}}%
\indexlibrary{\idxcode{counted_iterator}!\idxcode{operator+}}%
\begin{itemdecl}
  constexpr counted_iterator operator+(difference_type n) const
    requires RandomAccessIterator<I>;
\end{itemdecl}

\begin{itemdescr}
\pnum
\requires \tcode{n <= cnt}

\pnum
\effects Equivalent to:
\tcode{return counted_iterator(current + n, cnt - n);}
\end{itemdescr}

\rSec4[std2.counted.iter.op.+=]{\tcode{counted_iterator::operator+=}}

\indexlibrary{\idxcode{operator+=}!\idxcode{counted_iterator}}%
\indexlibrary{\idxcode{counted_iterator}!\idxcode{operator+=}}%
\begin{itemdecl}
  constexpr counted_iterator& operator+=(difference_type n)
    requires RandomAccessIterator<I>;
\end{itemdecl}

\begin{itemdescr}
\pnum
\requires \tcode{n <= cnt}

\pnum
\effects
\begin{codeblock}
current += n;
cnt -= n;
\end{codeblock}

\pnum
\returns \tcode{*this}.
\end{itemdescr}

\rSec4[std2.counted.iter.op.-]{\tcode{counted_iterator::operator-}}

\indexlibrary{\idxcode{operator-}!\idxcode{counted_iterator}}%
\indexlibrary{\idxcode{counted_iterator}!\idxcode{operator-}}%
\begin{itemdecl}
  constexpr counted_iterator operator-(difference_type n) const
    requires RandomAccessIterator<I>;
\end{itemdecl}

\begin{itemdescr}
\pnum
\requires \tcode{-n <= cnt}

\pnum
\effects Equivalent to:
\tcode{return counted_iterator(current - n, cnt + n);}
\end{itemdescr}

\rSec4[std2.counted.iter.op.-=]{\tcode{counted_iterator::operator-=}}

\indexlibrary{\idxcode{operator-=}!\idxcode{counted_iterator}}%
\indexlibrary{\idxcode{counted_iterator}!\idxcode{operator-=}}%
\begin{itemdecl}
  constexpr counted_iterator& operator-=(difference_type n)
    requires RandomAccessIterator<I>;
\end{itemdecl}

\begin{itemdescr}
\pnum
\requires \tcode{-n <= cnt}

\pnum
\effects
\begin{codeblock}
current -= n;
cnt += n;
\end{codeblock}

\pnum
\returns \tcode{*this}.
\end{itemdescr}

\rSec4[std2.counted.iter.op.index]{\tcode{counted_iterator::operator[]}}

\indexlibrary{\idxcode{operator[]}!\idxcode{counted_iterator}}%
\indexlibrary{\idxcode{counted_iterator}!\idxcode{operator[]}}%
\begin{itemdecl}
  constexpr decltype(auto) operator[](difference_type n) const
    requires RandomAccessIterator<I>;
\end{itemdecl}

\begin{itemdescr}
\pnum
\requires \tcode{n <= cnt}

\pnum
\effects Equivalent to:
\tcode{return current[n];}
\end{itemdescr}

\rSec4[std2.counted.iter.op.comp]{\tcode{counted_iterator} comparisons}

\indexlibrary{\idxcode{operator==}!\idxcode{counted_iterator}}%
\indexlibrary{\idxcode{counted_iterator}!\idxcode{operator==}}%
\begin{itemdecl}
template <class I1, class I2>
    requires Common<I1, I2>
  constexpr bool operator==(
    const counted_iterator<I1>& x, const counted_iterator<I2>& y);
\end{itemdecl}

\begin{itemdescr}
\pnum
\requires \tcode{x} and {y} shall refer to elements of the same
sequence~(\ref{std2.iterators.counted}).

\pnum
\effects Equivalent to:
\tcode{return x.cnt == y.cnt;}
\end{itemdescr}

\begin{itemdecl}
@\added{template <class I>}@
  constexpr bool operator==(
    const counted_iterator<@\changed{auto}{I}@>& x, default_sentinel);
@\added{template <class I>}@
  constexpr bool operator==(
    default_sentinel, const counted_iterator<@\changed{auto}{I}@>& x);
\end{itemdecl}

\begin{itemdescr}
\pnum
\effects Equivalent to:
\tcode{return x.cnt == 0;}
\end{itemdescr}

\indexlibrary{\idxcode{operator"!=}!\idxcode{counted_iterator}}%
\indexlibrary{\idxcode{counted_iterator}!\idxcode{operator"!=}}%
\begin{itemdecl}
template <class I1, class I2>
    requires Common<I1, I2>
  constexpr bool operator!=(
    const counted_iterator<I1>& x, const counted_iterator<I2>& y);
@\added{template <class I>}@
  constexpr bool operator!=(
    const counted_iterator<@\changed{auto}{I}@>& x, default_sentinel);
@\added{template <class I>}@
  constexpr bool operator!=(
    default_sentinel, const counted_iterator<@\changed{auto}{I}@>& x);
\end{itemdecl}

\begin{itemdescr}
\pnum
\requires For the first overload, \tcode{x} and {y} shall refer to
elements of the same sequence~(\ref{std2.iterators.counted}).

\pnum
\effects Equivalent to:
\tcode{return !(x == y);}
\end{itemdescr}

\indexlibrary{\idxcode{operator<}!\idxcode{counted_iterator}}%
\indexlibrary{\idxcode{counted_iterator}!\idxcode{operator<}}%
\begin{itemdecl}
template <class I1, class I2>
    requires Common<I1, I2>
  constexpr bool operator<(
    const counted_iterator<I1>& x, const counted_iterator<I2>& y);
\end{itemdecl}

\begin{itemdescr}
\pnum
\requires \tcode{x} and {y} shall refer to
elements of the same sequence~(\ref{std2.iterators.counted}).

\pnum
\effects Equivalent to:
\tcode{return y.cnt < x.cnt;}

\pnum
\enternote The argument order in the \textit{Effects} element is reversed because \tcode{cnt}
counts down, not up. \exitnote

\end{itemdescr}

\indexlibrary{\idxcode{operator<=}!\idxcode{counted_iterator}}%
\indexlibrary{\idxcode{counted_iterator}!\idxcode{operator<=}}%
\begin{itemdecl}
template <class I1, class I2>
    requires Common<I1, I2>
  constexpr bool operator<=(
    const counted_iterator<I1>& x, const counted_iterator<I2>& y);
\end{itemdecl}

\begin{itemdescr}
\pnum
\requires \tcode{x} and {y} shall refer to
elements of the same sequence~(\ref{std2.iterators.counted}).

\pnum
\effects Equivalent to:
\tcode{return !(y < x);}
\end{itemdescr}

\indexlibrary{\idxcode{operator>}!\idxcode{counted_iterator}}%
\indexlibrary{\idxcode{counted_iterator}!\idxcode{operator>}}%
\begin{itemdecl}
template <class I1, class I2>
    requires Common<I1, I2>
  constexpr bool operator>(
    const counted_iterator<I1>& x, const counted_iterator<I2>& y);
\end{itemdecl}

\begin{itemdescr}
\pnum
\requires \tcode{x} and {y} shall refer to
elements of the same sequence~(\ref{std2.iterators.counted}).

\pnum
\effects Equivalent to:
\tcode{return y < x;}
\end{itemdescr}

\indexlibrary{\idxcode{operator>=}!\idxcode{counted_iterator}}%
\indexlibrary{\idxcode{counted_iterator}!\idxcode{operator>=}}%
\begin{itemdecl}
template <class I1, class I2>
    requires Common<I1, I2>
  constexpr bool operator>=(
    const counted_iterator<I1>& x, const counted_iterator<I2>& y);
\end{itemdecl}

\begin{itemdescr}
\pnum
\requires \tcode{x} and {y} shall refer to
elements of the same sequence~(\ref{std2.iterators.counted}).

\pnum
\effects Equivalent to:
\tcode{return !(x < y);}
\end{itemdescr}

\rSec4[std2.counted.iter.nonmember]{\tcode{counted_iterator} non-member functions}

\indexlibrary{\idxcode{operator-}!\idxcode{counted_iterator}}%
\indexlibrary{\idxcode{counted_iterator}!\idxcode{operator-}}%
\begin{itemdecl}
  template <class I1, class I2>
      requires Common<I1, I2>
  constexpr difference_type_t<I2> operator-(
    const counted_iterator<I1>& x, const counted_iterator<I2>& y);
\end{itemdecl}

\begin{itemdescr}
\pnum
\requires \tcode{x} and {y} shall refer to
elements of the same sequence~(\ref{std2.iterators.counted}).

\pnum
\effects Equivalent to:
\tcode{return y.cnt - x.cnt;}
\end{itemdescr}

\begin{itemdecl}
template <class I>
  constexpr difference_type_t<I> operator-(
    const counted_iterator<I>& x, default_sentinel y);
\end{itemdecl}

\begin{itemdescr}
\pnum
\effects Equivalent to:
\tcode{return -x.cnt;}
\end{itemdescr}

\begin{itemdecl}
template <class I>
  constexpr difference_type_t<I> operator-(
    default_sentinel x, const counted_iterator<I>& y);
\end{itemdecl}

\begin{itemdescr}
\pnum
\effects Equivalent to:
\tcode{return y.cnt;}
\end{itemdescr}

\indexlibrary{\idxcode{operator+}!\idxcode{counted_iterator}}%
\indexlibrary{\idxcode{counted_iterator}!\idxcode{operator+}}%
\begin{itemdecl}
template <RandomAccessIterator I>
  constexpr counted_iterator<I> operator+(
    difference_type_t<I> n, const counted_iterator<I>& x);
\end{itemdecl}

\begin{itemdescr}
\pnum
\requires \tcode{n <= x.cnt}.

\pnum
\effects Equivalent to:
\tcode{return x + n;}
\end{itemdescr}

\indexlibrary{\idxcode{iter_move}!\idxcode{counted_iterator}}%
\indexlibrary{\idxcode{counted_iterator}!\idxcode{iter_move}}%
\begin{itemdecl}
friend constexpr rvalue_reference_t<I> iter_move(const counted_iterator& i)
  noexcept(@\seebelow@)
    requires InputIterator<I>;
\end{itemdecl}

\begin{itemdescr}
\pnum
\effects Equivalent to: \tcode{return \changed{ranges}{::std2}::iter_move(i.current);}

\pnum
\remarks The expression in \tcode{noexcept} is equivalent to:
\begin{codeblock}
noexcept(@\changed{ranges}{::std2}@::iter_move(i.current))
\end{codeblock}
\end{itemdescr}

\indexlibrary{\idxcode{iter_swap}!\idxcode{counted_iterator}}%
\indexlibrary{\idxcode{counted_iterator}!\idxcode{iter_swap}}%
\begin{itemdecl}
template <IndirectlySwappable<I> I2>
  friend constexpr void iter_swap(const counted_iterator& x, const counted_iterator<I2>& y)
    noexcept(@\seebelow@);
\end{itemdecl}

\begin{itemdescr}
\pnum
\effects Equivalent to \tcode{\changed{ranges}{::std2}::iter_swap(x.current, y.current)}.

\pnum
\remarks The expression in \tcode{noexcept} is equivalent to:
\begin{codeblock}
noexcept(@\changed{ranges}{::std2}@::iter_swap(x.current, y.current))
\end{codeblock}
\end{itemdescr}

\indexlibrary{\idxcode{make_counted_iterator}}%
\begin{itemdecl}
template <Iterator I>
  constexpr counted_iterator<I> make_counted_iterator(I i, difference_type_t<I> n);
\end{itemdecl}

\begin{itemdescr}
\pnum
\requires \tcode{n >= 0}.

\pnum
\returns \tcode{counted_iterator<I>(i, n)}.
\end{itemdescr}

\ednote{\tcode{dangling} has been moved to the ``Ranges library'' subclause.}

\rSec2[std2.unreachable.sentinels]{Unreachable sentinel}

\rSec3[std2.unreachable.sentinel]{Class \tcode{unreachable}}

\pnum
\indexlibrary{\idxcode{unreachable}}%
Class \tcode{unreachable} is a sentinel type that can be used with any
\tcode{Iterator} to denote an infinite range. Comparing an iterator for equality with
an object of type \tcode{unreachable} always returns \tcode{false}.

\enterexample
\begin{codeblock}
char* p;
// set \tcode{p} to point to a character buffer containing newlines
char* nl = find(p, unreachable(), '@\textbackslash@n');
\end{codeblock}

Provided a newline character really exists in the buffer, the use of \tcode{unreachable}
above potentially makes the call to \tcode{find} more efficient since the loop test against
the sentinel does not require a conditional branch.
\exitexample

\begin{codeblock}
namespace @\changed{std \{ namespace experimental \{ namespace ranges}{std2}@ { inline namespace v1 {
  class unreachable { };

  template <Iterator I>
    constexpr bool operator==(const I&, unreachable) noexcept;
  template <Iterator I>
    constexpr bool operator==(unreachable, const I&) noexcept;
  template <Iterator I>
    constexpr bool operator!=(const I&, unreachable) noexcept;
  template <Iterator I>
    constexpr bool operator!=(unreachable, const I&) noexcept;
}}@\removed{\}\}}@
\end{codeblock}

\rSec3[std2.unreachable.sentinel.ops]{\tcode{unreachable} operations}

\rSec4[std2.unreachable.sentinel.op==]{\tcode{operator==}}

\indexlibrary{\idxcode{operator==}!\idxcode{unreachable}}%
\indexlibrary{\idxcode{unreachable}!\idxcode{operator==}}%
\begin{itemdecl}
template <Iterator I>
  constexpr bool operator==(const I&, unreachable) noexcept;
template <Iterator I>
  constexpr bool operator==(unreachable, const I&) noexcept;
\end{itemdecl}

\begin{itemdescr}
\pnum
\returns \tcode{false}.
\end{itemdescr}

\rSec4[std2.unreachable.sentinel.op!=]{\tcode{operator!=}}

\indexlibrary{\idxcode{operator"!=}!\idxcode{unreachable}}%
\indexlibrary{\idxcode{unreachable}!\idxcode{operator"!=}}%
\begin{itemdecl}
template <Iterator I>
  constexpr bool operator!=(const I& x, unreachable y) noexcept;
template <Iterator I>
  constexpr bool operator!=(unreachable x, const I& y) noexcept;
\end{itemdecl}

\begin{itemdescr}
\pnum
\returns
\tcode{true}.
\end{itemdescr}

\rSec1[std2.iterators.stream]{Stream iterators}

\pnum
To make it possible for algorithmic templates to work directly with input/output streams, appropriate
iterator-like
class templates
are provided.

\enterexample
\begin{codeblock}
partial_sum(istream_iterator<double, char>(cin),
  istream_iterator<double, char>(),
  ostream_iterator<double, char>(cout, "@\textbackslash@n"));
\end{codeblock}

reads a file containing floating point numbers from
\tcode{cin},
and prints the partial sums onto
\tcode{cout}.
\exitexample

\rSec2[std2.istream.iterator]{Class template \tcode{istream_iterator}}

\pnum
\indexlibrary{\idxcode{istream_iterator}}%
The class template
\tcode{istream_iterator}
is an input iterator~(\ref{std2.iterators.input}) that
reads (using
\tcode{operator\shr})
successive elements from the input stream for which it was constructed.
After it is constructed, and every time
\tcode{++}
is used, the iterator reads and stores a value of
\tcode{T}.
If the iterator fails to read and store a value of \tcode{T}
(\tcode{fail()}
on the stream returns
\tcode{true}),
the iterator becomes equal to the
\term{end-of-stream}
iterator value.
The constructor with no arguments
\tcode{istream_iterator()}
always constructs
an end-of-stream input iterator object, which is the only legitimate iterator to be used
for the end condition.
The result of
\tcode{operator*}
on an end-of-stream iterator is not defined.
For any other iterator value a
\tcode{const T\&}
is returned.
The result of
\tcode{operator->}
on an end-of-stream iterator is not defined.
For any other iterator value a
\tcode{const T*}
is returned.
The behavior of a program that applies \tcode{operator++()} to an end-of-stream
iterator is undefined.
It is impossible to store things into istream iterators.

\pnum
Two end-of-stream iterators are always equal.
An end-of-stream iterator is not
equal to a non-end-of-stream iterator.
Two non-end-of-stream iterators are equal when they are constructed from the same stream.

\begin{codeblock}
namespace @\changed{std \{ namespace experimental \{ namespace ranges}{std2}@ { inline namespace v1 {
  template <class T, class charT = char, class traits = char_traits<charT>,
      class Distance = ptrdiff_t>
  class istream_iterator {
  public:
    typedef input_iterator_tag iterator_category;
    typedef Distance difference_type;
    typedef T value_type;
    typedef const T& reference;
    typedef const T* pointer;
    typedef charT char_type;
    typedef traits traits_type;
    typedef basic_istream<charT, traits> istream_type;
    constexpr istream_iterator();
    constexpr istream_iterator(default_sentinel);
    istream_iterator(istream_type& s);
    istream_iterator(const istream_iterator& x) = default;
    ~istream_iterator() = default;

    const T& operator*() const;
    const T* operator->() const;
    istream_iterator& operator++();
    istream_iterator  operator++(int);
  private:
    basic_istream<charT, traits>* in_stream; // \expos
    T value;                                 // \expos
  };

  template <class T, class charT, class traits, class Distance>
    bool operator==(const istream_iterator<T, charT, traits, Distance>& x,
            const istream_iterator<T, charT, traits, Distance>& y);
  template <class T, class charT, class traits, class Distance>
    bool operator==(default_sentinel x,
            const istream_iterator<T, charT, traits, Distance>& y);
  template <class T, class charT, class traits, class Distance>
    bool operator==(const istream_iterator<T, charT, traits, Distance>& x,
            default_sentinel y);
  template <class T, class charT, class traits, class Distance>
    bool operator!=(const istream_iterator<T, charT, traits, Distance>& x,
            const istream_iterator<T, charT, traits, Distance>& y);
  template <class T, class charT, class traits, class Distance>
    bool operator!=(default_sentinel x,
            const istream_iterator<T, charT, traits, Distance>& y);
  template <class T, class charT, class traits, class Distance>
    bool operator!=(const istream_iterator<T, charT, traits, Distance>& x,
            default_sentinel y);
}}@\removed{\}\}}@
\end{codeblock}

\rSec3[std2.istream.iterator.cons]{\tcode{istream_iterator} constructors and destructor}

\indexlibrary{\idxcode{istream_iterator}!constructor}%
\begin{itemdecl}
constexpr istream_iterator();
constexpr istream_iterator(default_sentinel);
\end{itemdecl}

\begin{itemdescr}
\pnum
\effects
Constructs the end-of-stream iterator. If \tcode{T} is a literal type, then these
constructors shall be \tcode{constexpr} constructors.

\pnum
\postcondition \tcode{in_stream == nullptr}.
\end{itemdescr}

\indexlibrary{\idxcode{istream_iterator}!constructor}%
\begin{itemdecl}
istream_iterator(istream_type& s);
\end{itemdecl}

\begin{itemdescr}
\pnum
\effects
Initializes \tcode{in_stream} with \tcode{\&s}. \tcode{value} may be initialized during
construction or the first time it is referenced.

\pnum
\postcondition \tcode{in_stream == \&s}.
\end{itemdescr}

\indexlibrary{\idxcode{istream_iterator}!constructor}%
\begin{itemdecl}
istream_iterator(const istream_iterator& x) = default;
\end{itemdecl}

\begin{itemdescr}
\pnum
\effects
Constructs a copy of \tcode{x}. If \tcode{T} is a literal type, then this constructor shall be a trivial copy constructor.

\pnum
\postcondition \tcode{in_stream == x.in_stream}.
\end{itemdescr}

\indexlibrary{\idxcode{istream_iterator}!destructor}%
\begin{itemdecl}
~istream_iterator() = default;
\end{itemdecl}

\begin{itemdescr}
\pnum
\effects
The iterator is destroyed. If \tcode{T} is a literal type, then this destructor shall be a trivial destructor.
\end{itemdescr}

\rSec3[std2.istream.iterator.ops]{\tcode{istream_iterator} operations}

\indexlibrary{\idxcode{operator*}!\idxcode{istream_iterator}}%
\indexlibrary{\idxcode{istream_iterator}!\idxcode{operator*}}%
\begin{itemdecl}
const T& operator*() const;
\end{itemdecl}

\begin{itemdescr}
\pnum
\returns
\tcode{value}.
\end{itemdescr}

\indexlibrary{\idxcode{operator->}!\idxcode{istream_iterator}}%
\indexlibrary{\idxcode{istream_iterator}!\idxcode{operator->}}%
\begin{itemdecl}
const T* operator->() const;
\end{itemdecl}

\begin{itemdescr}
\pnum
\effects Equivalent to:
\tcode{return addressof(operator*())}.
\end{itemdescr}

\indexlibrary{\idxcode{operator++}!\idxcode{istream_iterator}}%
\indexlibrary{\idxcode{istream_iterator}!\idxcode{operator++}}%
\begin{itemdecl}
istream_iterator& operator++();
\end{itemdecl}

\begin{itemdescr}
\pnum
\requires \tcode{in_stream != nullptr}.

\pnum
\effects
\tcode{*in_stream \shr{} value}.

\pnum
\returns
\tcode{*this}.
\end{itemdescr}

\indexlibrary{\idxcode{operator++}!\idxcode{istream_iterator}}%
\indexlibrary{\idxcode{istream_iterator}!\idxcode{operator++}}%
\begin{itemdecl}
istream_iterator operator++(int);
\end{itemdecl}

\begin{itemdescr}
\pnum
\requires \tcode{in_stream != nullptr}.

\pnum
\effects
\begin{codeblock}
istream_iterator tmp = *this;
*in_stream >> value;
return tmp;
\end{codeblock}
\end{itemdescr}

\indexlibrary{\idxcode{operator==}!\idxcode{istream_iterator}}%
\indexlibrary{\idxcode{istream_iterator}!\idxcode{operator==}}%
\begin{itemdecl}
template <class T, class charT, class traits, class Distance>
  bool operator==(const istream_iterator<T, charT, traits, Distance> &x,
                  const istream_iterator<T, charT, traits, Distance> &y);
\end{itemdecl}

\begin{itemdescr}
\pnum
\returns
\tcode{x.in_stream == y.in_stream}.%
\indexlibrary{\idxcode{istream_iterator}!\idxcode{operator==}}
\end{itemdescr}

\begin{itemdecl}
template <class T, class charT, class traits, class Distance>
  bool operator==(default_sentinel x,
                  const istream_iterator<T, charT, traits, Distance> &y);
\end{itemdecl}

\begin{itemdescr}
\pnum
\returns
\tcode{nullptr == y.in_stream}.%
\end{itemdescr}

\begin{itemdecl}
template <class T, class charT, class traits, class Distance>
  bool operator==(const istream_iterator<T, charT, traits, Distance> &x,
                  default_sentinel y);
\end{itemdecl}

\begin{itemdescr}
\pnum
\returns
\tcode{x.in_stream == nullptr}.%
\end{itemdescr}

\indexlibrary{\idxcode{operator"!=}!\idxcode{istream_iterator}}%
\indexlibrary{\idxcode{istream_iterator}!\idxcode{operator"!=}}%
\begin{itemdecl}
template <class T, class charT, class traits, class Distance>
  bool operator!=(const istream_iterator<T, charT, traits, Distance>& x,
                  const istream_iterator<T, charT, traits, Distance>& y);
template <class T, class charT, class traits, class Distance>
  bool operator!=(default_sentinel x,
                  const istream_iterator<T, charT, traits, Distance>& y);
template <class T, class charT, class traits, class Distance>
  bool operator!=(const istream_iterator<T, charT, traits, Distance>& x,
                  default_sentinel y);
\end{itemdecl}

\indexlibrary{\idxcode{istream_iterator}!\idxcode{operator"!=}}%
\begin{itemdescr}
\pnum
\returns
\tcode{!(x == y)}
\end{itemdescr}

\rSec2[std2.ostream.iterator]{Class template \tcode{ostream_iterator}}

\pnum
\indexlibrary{\idxcode{ostream_iterator}}%
\tcode{ostream_iterator}
writes (using
\tcode{operator\shl})
successive elements onto the output stream from which it was constructed.
If it was constructed with
\tcode{charT*}
as a constructor argument, this string, called a
\term{delimiter string},
is written to the stream after every
\tcode{T}
is written.
It is not possible to get a value out of the output iterator.
Its only use is as an output iterator in situations like

\begin{codeblock}
while (first != last)
  *result++ = *first++;
\end{codeblock}

\pnum
\tcode{ostream_iterator}
is defined as:

\begin{codeblock}
namespace @\changed{std \{ namespace experimental \{ namespace ranges}{std2}@ { inline namespace v1 {
  template <class T, class charT = char, class traits = char_traits<charT>>
  class ostream_iterator {
  public:
    typedef ptrdiff_t difference_type;
    typedef charT char_type;
    typedef traits traits_type;
    typedef basic_ostream<charT, traits> ostream_type;
    constexpr ostream_iterator() noexcept;
    ostream_iterator(ostream_type& s) noexcept;
    ostream_iterator(ostream_type& s, const charT* delimiter) noexcept;
    ostream_iterator(const ostream_iterator& x) noexcept;
    ~ostream_iterator();
    ostream_iterator& operator=(const T& value);

    ostream_iterator& operator*();
    ostream_iterator& operator++();
    ostream_iterator& operator++(int);
  private:
    basic_ostream<charT, traits>* out_stream;  // \expos
    const charT* delim;                        // \expos
  };
}}@\removed{\}\}}@
\end{codeblock}

\rSec3[std2.ostream.iterator.cons.des]{\tcode{ostream_iterator} constructors and destructor}

\indexlibrary{\idxcode{ostream_iterator}!constructor}%
\begin{itemdecl}
constexpr ostream_iterator() noexcept;
\end{itemdecl}

\begin{itemdescr}
\pnum
\effects
Initializes \tcode{out_stream} and \tcode{delim} with \tcode{nullptr}.
\end{itemdescr}

\indexlibrary{\idxcode{ostream_iterator}!constructor}%
\begin{itemdecl}
ostream_iterator(ostream_type& s) noexcept;
\end{itemdecl}

\begin{itemdescr}
\pnum
\effects
Initializes \tcode{out_stream} with \tcode{\&s} and \tcode{delim} with \tcode{nullptr}.
\end{itemdescr}

\indexlibrary{\idxcode{ostream_iterator}!constructor}%
\begin{itemdecl}
ostream_iterator(ostream_type& s, const charT* delimiter) noexcept;
\end{itemdecl}

\begin{itemdescr}
\pnum
\effects
Initializes \tcode{out_stream} with \tcode{\&s} and \tcode{delim} with \tcode{delimiter}.
\end{itemdescr}

\indexlibrary{\idxcode{ostream_iterator}!constructor}%
\begin{itemdecl}
ostream_iterator(const ostream_iterator& x) noexcept;
\end{itemdecl}

\begin{itemdescr}
\pnum
\effects
Constructs a copy of \tcode{x}.
\end{itemdescr}

\indexlibrary{\idxcode{ostream_iterator}!destructor}%
\begin{itemdecl}
~ostream_iterator();
\end{itemdecl}

\begin{itemdescr}
\pnum
\effects
The iterator is destroyed.
\end{itemdescr}

\rSec3[std2.ostream.iterator.ops]{\tcode{ostream_iterator} operations}

\indexlibrary{\idxcode{operator=}!\idxcode{ostream_iterator}}%
\indexlibrary{\idxcode{ostream_iterator}!\idxcode{operator=}}%
\begin{itemdecl}
ostream_iterator& operator=(const T& value);
\end{itemdecl}

\begin{itemdescr}
\pnum
\effects Equivalent to:
\begin{codeblock}
*out_stream << value;
if(delim != nullptr)
  *out_stream << delim;
return *this;
\end{codeblock}
\end{itemdescr}

\indexlibrary{\idxcode{operator*}!\idxcode{ostream_iterator}}%
\indexlibrary{\idxcode{ostream_iterator}!\idxcode{operator*}}%
\begin{itemdecl}
ostream_iterator& operator*();
\end{itemdecl}

\begin{itemdescr}
\pnum
\returns
\tcode{*this}.
\end{itemdescr}

\indexlibrary{\idxcode{operator++}!\idxcode{ostream_iterator}}%
\indexlibrary{\idxcode{ostream_iterator}!\idxcode{operator++}}%
\begin{itemdecl}
ostream_iterator& operator++();
ostream_iterator& operator++(int);
\end{itemdecl}

\begin{itemdescr}
\pnum
\returns
\tcode{*this}.
\end{itemdescr}

\rSec2[std2.istreambuf.iterator]{Class template \tcode{istreambuf_iterator}}

\pnum
The
class template
\tcode{istreambuf_iterator}
defines an input iterator~(\ref{std2.iterators.input}) that
reads successive
\textit{characters}
from the streambuf for which it was constructed.
\tcode{operator*}
provides access to the current input character, if any.
Each time
\tcode{operator++}
is evaluated, the iterator advances to the next input character.
If the end of stream is reached (\tcode{streambuf_type::sgetc()} returns
\tcode{traits::eof()}),
the iterator becomes equal to the
\term{end-of-stream}
iterator value.
The default constructor
\tcode{istreambuf_iterator()}
and the constructor
\tcode{istreambuf_iterator(nullptr)}
both construct an end-of-stream iterator object suitable for use
as an end-of-range.
All specializations of \tcode{istreambuf_iterator} shall have a trivial copy
constructor, a \tcode{constexpr} default constructor, and a trivial destructor.

\pnum
The result of
\tcode{operator*()}
on an end-of-stream iterator is undefined.
\indextext{undefined behavior}%
For any other iterator value a
\tcode{char_type}
value is returned.
It is impossible to assign a character via an input iterator.

\indexlibrary{\idxcode{istreambuf_iterator}}%

\begin{codeblock}
namespace @\changed{std \{ namespace experimental \{ namespace ranges}{std2}@ { inline namespace v1 {
  template <class charT, class traits = char_traits<charT>>
  class istreambuf_iterator {
  public:
    typedef input_iterator_tag             iterator_category;
    typedef charT                          value_type;
    typedef typename traits::off_type      difference_type;
    typedef charT                          reference;
    typedef @\unspec@                   pointer;
    typedef charT                          char_type;
    typedef traits                         traits_type;
    typedef typename traits::int_type      int_type;
    typedef basic_streambuf<charT, traits> streambuf_type;
    typedef basic_istream<charT, traits>   istream_type;

    class proxy;                           // \expos

    constexpr istreambuf_iterator() noexcept;
    constexpr istreambuf_iterator(default_sentinel) noexcept;
    istreambuf_iterator(const istreambuf_iterator&) noexcept = default;
    ~istreambuf_iterator() = default;
    istreambuf_iterator(istream_type& s) noexcept;
    istreambuf_iterator(streambuf_type* s) noexcept;
    istreambuf_iterator(const proxy& p) noexcept;
    charT operator*() const;
    istreambuf_iterator& operator++();
    proxy operator++(int);
    bool equal(const istreambuf_iterator& b) const;
  private:
    streambuf_type* sbuf_;                // \expos
  };

  template <class charT, class traits>
    bool operator==(const istreambuf_iterator<charT, traits>& a,
            const istreambuf_iterator<charT, traits>& b);
  template <class charT, class traits>
    bool operator==(default_sentinel a,
            const istreambuf_iterator<charT, traits>& b);
  template <class charT, class traits>
    bool operator==(const istreambuf_iterator<charT, traits>& a,
            default_sentinel b);
  template <class charT, class traits>
    bool operator!=(const istreambuf_iterator<charT, traits>& a,
            const istreambuf_iterator<charT, traits>& b);
  template <class charT, class traits>
    bool operator!=(default_sentinel a,
            const istreambuf_iterator<charT, traits>& b);
  template <class charT, class traits>
    bool operator!=(const istreambuf_iterator<charT, traits>& a,
            default_sentinel b);
}}@\removed{\}\}}@
\end{codeblock}

\rSec3[std2.istreambuf.iterator::proxy]{Class template \tcode{istreambuf_iterator::proxy}}

\indexlibrary{\idxcode{proxy}!\idxcode{istreambuf_iterator}}%
\begin{codeblock}
namespace @\changed{std \{ namespace experimental \{ namespace ranges}{std2}@ { inline namespace v1 {
  template <class charT, class traits = char_traits<charT>>
  class istreambuf_iterator<charT, traits>::proxy { // \expos
    charT keep_;
    basic_streambuf<charT, traits>* sbuf_;
    proxy(charT c, basic_streambuf<charT, traits>* sbuf)
      : keep_(c), sbuf_(sbuf) { }
  public:
    charT operator*() { return keep_; }
  };
}}@\removed{\}\}}@
\end{codeblock}

\pnum
Class
\tcode{istreambuf_iterator<charT, traits>::proxy}
is for exposition only.
An implementation is permitted to provide equivalent functionality without
providing a class with this name.
Class
\tcode{istreambuf_iterator<charT, traits>\colcol{}proxy}
provides a temporary
placeholder as the return value of the post-increment operator
(\tcode{operator++}).
It keeps the character pointed to by the previous value
of the iterator for some possible future access to get the character.

\rSec3[std2.istreambuf.iterator.cons]{\tcode{istreambuf_iterator} constructors}

\indexlibrary{\idxcode{istreambuf_iterator}!constructor}%
\begin{itemdecl}
constexpr istreambuf_iterator() noexcept;
constexpr istreambuf_iterator(default_sentinel) noexcept;
\end{itemdecl}

\begin{itemdescr}
\pnum
\effects
Constructs the end-of-stream iterator.
\end{itemdescr}

\indexlibrary{\idxcode{istreambuf_iterator}!constructor}%
\begin{itemdecl}
istreambuf_iterator(basic_istream<charT, traits>& s) noexcept;
istreambuf_iterator(basic_streambuf<charT, traits>* s) noexcept;
\end{itemdecl}

\begin{itemdescr}
\pnum
\effects
Constructs an
\tcode{istreambuf_iterator}
that uses the
\tcode{basic_streambuf}
object
\tcode{*(s.rdbuf())},
or
\tcode{*s},
respectively.
Constructs an end-of-stream iterator if
\tcode{s.rdbuf()}
is null.
\end{itemdescr}


\indexlibrary{\idxcode{istreambuf_iterator}!constructor}%
\begin{itemdecl}
istreambuf_iterator(const proxy& p) noexcept;
\end{itemdecl}

\begin{itemdescr}
\pnum
\effects
Constructs a
\tcode{istreambuf_iterator}
that uses the
\tcode{basic_streambuf}
object pointed to by the
\tcode{proxy}
object's constructor argument \tcode{p}.
\end{itemdescr}

\rSec3[std2.istreambuf.iterator::op*]{\tcode{istreambuf_iterator::operator*}}

\indexlibrary{\idxcode{operator*}!\idxcode{istreambuf_iterator}}%
\begin{itemdecl}
charT operator*() const
\end{itemdecl}

\begin{itemdescr}
\pnum
\returns
The character obtained via the
\tcode{streambuf}
member
\tcode{sbuf_->sgetc()}.
\end{itemdescr}

\rSec3[std2.istreambuf.iterator::op++]{\tcode{istreambuf_iterator::operator++}}

\indexlibrary{\idxcode{operator++}!\idxcode{istreambuf_iterator}}%
\begin{itemdecl}
istreambuf_iterator&
    istreambuf_iterator<charT, traits>::operator++();
\end{itemdecl}

\begin{itemdescr}
\pnum
\effects Equivalent to
\tcode{sbuf_->sbumpc()}.

\pnum
\returns
\tcode{*this}.
\end{itemdescr}

\indexlibrary{\idxcode{operator++}!\idxcode{istreambuf_iterator}}%
\indexlibrary{\idxcode{istreambuf_iterator}!\idxcode{operator++}}%
\begin{itemdecl}
proxy istreambuf_iterator<charT, traits>::operator++(int);
\end{itemdecl}

\begin{itemdescr}
\pnum
\effects Equivalent to:
\tcode{return proxy(sbuf_->sbumpc(), sbuf_);}
\end{itemdescr}

\rSec3[std2.istreambuf.iterator::equal]{\tcode{istreambuf_iterator::equal}}

\indexlibrary{\idxcode{equal}!\idxcode{istreambuf_iterator}}%
\begin{itemdecl}
bool equal(const istreambuf_iterator& b) const;
\end{itemdecl}

\begin{itemdescr}
\pnum
\returns
\tcode{true}
if and only if both iterators are at end-of-stream,
or neither is at end-of-stream, regardless of what
\tcode{streambuf}
object they use.
\end{itemdescr}

\rSec3[std2.istreambuf.iterator::op==]{\tcode{operator==}}

\indexlibrary{\idxcode{operator==}!\idxcode{istreambuf_iterator}}%
\begin{itemdecl}
template <class charT, class traits>
  bool operator==(const istreambuf_iterator<charT, traits>& a,
                  const istreambuf_iterator<charT, traits>& b);
\end{itemdecl}

\begin{itemdescr}
\pnum
\effects Equivalent to:
\tcode{return a.equal(b);}
\end{itemdescr}

\begin{itemdecl}
template <class charT, class traits>
  bool operator==(default_sentinel a,
                  const istreambuf_iterator<charT, traits>& b);
\end{itemdecl}

\begin{itemdescr}
\pnum
\effects Equivalent to:
\tcode{return istreambuf_iterator<charT, traits>\{\}.equal(b);}
\end{itemdescr}

\begin{itemdecl}
template <class charT, class traits>
  bool operator==(const istreambuf_iterator<charT, traits>& a,
                  default_sentinel b);
\end{itemdecl}

\begin{itemdescr}
\pnum
\effects Equivalent to:
\tcode{return a.equal(istreambuf_iterator<charT, traits>\{\});}
\end{itemdescr}

\rSec3[std2.istreambuf.iterator::op!=]{\tcode{operator!=}}

\indexlibrary{\idxcode{operator"!=}!\idxcode{istreambuf_iterator}}%
\begin{itemdecl}
template <class charT, class traits>
  bool operator!=(const istreambuf_iterator<charT, traits>& a,
                  const istreambuf_iterator<charT, traits>& b);
template <class charT, class traits>
  bool operator!=(default_sentinel a,
                  const istreambuf_iterator<charT, traits>& b);
template <class charT, class traits>
  bool operator!=(const istreambuf_iterator<charT, traits>& a,
                  default_sentinel b);
\end{itemdecl}

\begin{itemdescr}
\pnum
\effects Equivalent to:
\tcode{return !(a == b);}
\end{itemdescr}

\rSec2[std2.ostreambuf.iterator]{Class template \tcode{ostreambuf_iterator}}

\indexlibrary{\idxcode{ostreambuf_iterator}}%
\begin{codeblock}
namespace @\changed{std \{ namespace experimental \{ namespace ranges}{std2}@ { inline namespace v1 {
  template <class charT, class traits = char_traits<charT>>
  class ostreambuf_iterator {
  public:
    typedef ptrdiff_t                      difference_type;
    typedef charT                          char_type;
    typedef traits                         traits_type;
    typedef basic_streambuf<charT, traits> streambuf_type;
    typedef basic_ostream<charT, traits>   ostream_type;

    constexpr ostreambuf_iterator() noexcept;
    ostreambuf_iterator(ostream_type& s) noexcept;
    ostreambuf_iterator(streambuf_type* s) noexcept;
    ostreambuf_iterator& operator=(charT c);

    ostreambuf_iterator& operator*();
    ostreambuf_iterator& operator++();
    ostreambuf_iterator& operator++(int);
    bool failed() const noexcept;

  private:
    streambuf_type* sbuf_;                // \expos
  };
}}@\removed{\}\}}@
\end{codeblock}

\pnum
The
class template
\tcode{ostreambuf_iterator}
writes successive
\textit{characters}
onto the output stream from which it was constructed.
It is not possible to get a character value out of the output iterator.

\rSec3[std2.ostreambuf.iter.cons]{\tcode{ostreambuf_iterator} constructors}

\indexlibrary{\idxcode{ostreambuf_iterator}!constructor}%
\begin{itemdecl}
constexpr ostreambuf_iterator() noexcept;
\end{itemdecl}

\begin{itemdescr}
\pnum
\effects
Initializes \tcode{sbuf_} with \tcode{nullptr}.
\end{itemdescr}

\indexlibrary{\idxcode{ostreambuf_iterator}!constructor}%
\begin{itemdecl}
ostreambuf_iterator(ostream_type& s) noexcept;
\end{itemdecl}

\begin{itemdescr}
\pnum
\requires
\tcode{s.rdbuf() != nullptr}.
\end{itemdescr}

\begin{itemdescr}
\pnum
\effects
Initializes \tcode{sbuf_} with \tcode{s.rdbuf()}.
\end{itemdescr}

\indexlibrary{\idxcode{ostreambuf_iterator}!constructor}%
\begin{itemdecl}
ostreambuf_iterator(streambuf_type* s) noexcept;
\end{itemdecl}

\begin{itemdescr}
\pnum
\requires
\tcode{s != nullptr}.

\pnum
\effects
Initializes \tcode{sbuf_} with \tcode{s}.
\end{itemdescr}

\rSec3[std2.ostreambuf.iter.ops]{\tcode{ostreambuf_iterator} operations}

\indexlibrary{\idxcode{operator=}!\idxcode{ostreambuf_iterator}}%
\begin{itemdecl}
ostreambuf_iterator&
  operator=(charT c);
\end{itemdecl}

\begin{itemdescr}
\pnum
\requires \tcode{sbuf_ != nullptr}.

\pnum
\effects
If
\tcode{failed()}
yields
\tcode{false},
calls
\tcode{sbuf_->sputc(c)};
otherwise has no effect.

\pnum
\returns
\tcode{*this}.
\end{itemdescr}

\indexlibrary{\idxcode{operator*}!\idxcode{ostreambuf_iterator}}%
\begin{itemdecl}
ostreambuf_iterator& operator*();
\end{itemdecl}

\begin{itemdescr}
\pnum
\returns
\tcode{*this}.
\end{itemdescr}

\indexlibrary{\idxcode{operator++}!\idxcode{ostreambuf_iterator}}%
\begin{itemdecl}
ostreambuf_iterator& operator++();
ostreambuf_iterator& operator++(int);
\end{itemdecl}

\begin{itemdescr}
\pnum
\returns
\tcode{*this}.
\end{itemdescr}

\indexlibrary{\idxcode{failed}!\idxcode{ostreambuf_iterator}}%
\begin{itemdecl}
bool failed() const noexcept;
\end{itemdecl}

\begin{itemdescr}
\pnum
\requires \tcode{sbuf_ != nullptr}.

\pnum
\returns
\tcode{true}
if in any prior use of member
\tcode{operator=},
the call to
\tcode{sbuf_->sputc()}
returned
\tcode{traits::eof()};
or
\tcode{false}
otherwise.
\end{itemdescr}
