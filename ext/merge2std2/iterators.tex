%!TEX root = P0896.tex

\setcounter{table}{72}
\rSec0[iterators]{Iterators library}

\rSec1[iterators.general]{General}

\pnum
This Clause describes components that \Cpp{} programs may use to perform
iterations over containers\iref{containers},
streams\cxxiref{iostream.format},
\removed{and} stream buffers\cxxiref{stream.buffers}\added{,
and other ranges\iref{range}}.

\pnum
The following subclauses describe
iterator requirements, and
components for
iterator primitives,
predefined iterators,
and stream iterators,
as summarized in \tref{iterators.lib.summary}.

\begin{libsumtab}{Iterators library summary}{tab:iterators.lib.summary}
\ref{iterator.requirements}     & \changed{R}{Iterator r}equirements      & \added{\tcode{<iterator>}}   \\
\added{\oldtxt{\ref{indirectcallable}}}  & \added{\oldtxt{Indirect callable requirements}}  &                              \\
\added{\oldtxt{\ref{commonalgoreq}}}     & \added{\oldtxt{Common algorithm requirements}}   &                              \\
\ref{iterator.primitives}       & Iterator primitives                     & \removed{\tcode{<iterator>}} \\
\ref{predef.iterators}          & Predefined iterators                    &                              \\
\ref{stream.iterators}          & Stream iterators                        &                              \\
\end{libsumtab}

\ednote{Move [iterator.synopsis] here immediately after [iterators.general]
and modify as follows:}

\rSec1[iterator.synopsis]{Header \tcode{<iterator>} synopsis}

\indexlibrary{\idxhdr{iterator}}%
\begin{codeblock}
@\added{\#include <concepts>}@

namespace std {
\end{codeblock}\begin{addedblock}\begin{codeblock}
  @\newtxt{template<class T> using \placeholder{with-reference} // \expos}@
    @\newtxt{= T\&;}@
  @\newtxt{template<class T> concept \placeholder{can-reference} // \expos}@
    @\newtxt{= requires \{ typename \placeholdernc{with-reference}<T>; \};}@
  template<class T> concept @\placeholder{dereferenceable}@ // \expos
    = requires(T& t) {
      @\oldtxt{\{}@ *t @\oldtxt{\} -> auto\&\&}@; // not required to be equality-preserving
      @\newtxt{requires \placeholdernc{can-reference}<decltype(*t)>;}@
    };

  // \ref{iterator.assoc.types}, associated types
  // \ref{incrementable.traits}, incrementable traits
  template<class> struct incrementable_traits;
  template<class T>
    using iter_difference_t = @\seebelownc@;

  // \ref{readable.traits}, readable traits
  template<class> struct readable_traits;
  template<class T>
    using iter_value_t = @\seebelownc@;
\end{codeblock}\end{addedblock}\begin{codeblock}

  // \changed{\ref{iterator.primitives}, primitives}{\ref{iterator.traits}, Iterator traits}
  template<class @\changed{Iterator}{I}@> struct iterator_traits;
  template<class T> struct iterator_traits<T*>;
\end{codeblock}\begin{addedblock}\begin{codeblock}

  template<@\placeholder{dereferenceable}@ T>
    using iter_reference_t = decltype(*declval<T&>());

  namespace ranges {
    // \ref{iterator.custpoints}, customization points
    inline namespace @\unspec@ {
      // \ref{iterator.custpoints.iter_move}, iter_move
      inline constexpr @\unspec@ iter_move = @\unspecnc@;

      // \ref{iterator.custpoints.iter_swap}, iter_swap
      inline constexpr @\unspec@ iter_swap = @\unspecnc@;
    }
  }

  template<@\placeholder{dereferenceable}@ T>
    requires requires (T& t) {
      @\oldtxt{\{}@ ranges::iter_move(t) @\oldtxt{\} -> auto \&\&}@;
      @\newtxt{requires \placeholder{can-reference}<decltype(ranges::iter_move(t))>;}@
    }
      using iter_rvalue_reference_t
        = decltype(ranges::iter_move(declval<T&>()));

  // \ref{iterator.concepts}, iterator concepts
  // \ref{iterator.concept.readable}, Readable
  template<class In>
    concept Readable = @\seebelownc@;

  template<Readable T>
    using iter_common_reference_t =
      common_reference_t<iter_reference_t<T>, iter_value_t<T>&>;

  // \ref{iterator.concept.writable}, Writable
  template<class Out, class T>
    concept Writable = @\seebelownc@;

  // \ref{iterator.concept.weaklyincrementable}, WeaklyIncrementable
  template<class I>
    concept WeaklyIncrementable = @\seebelownc@;

  // \ref{iterator.concept.incrementable}, Incrementable
  template<class I>
    concept Incrementable = @\seebelownc@;

  // \ref{iterator.concept.iterator}, Iterator
  template<class I>
    concept Iterator = @\seebelownc@;

  // \ref{iterator.concept.incrementable}, Sentinel
  template<class S, class I>
    concept Sentinel = @\seebelownc@;

  // \ref{iterator.concept.sizedsentinel}, SizedSentinel
  template<class S, class I>
    inline constexpr bool disable_sized_sentinel = false;

  template<class S, class I>
    concept SizedSentinel = @\seebelownc@;

  // \ref{iterator.concept.input}, InputIterator
  template<class I>
    concept InputIterator = @\seebelownc@;

  // \ref{iterator.concept.output}, OutputIterator
  template<class I@\newtxt{, class T}@>
    concept OutputIterator = @\seebelownc@;

  // \ref{iterator.concept.forward}, ForwardIterator
  template<class I>
    concept ForwardIterator = @\seebelownc@;

  // \ref{iterator.concept.bidirectional}, BidirectionalIterator
  template<class I>
    concept BidirectionalIterator = @\seebelownc@;

  // \ref{iterator.concept.random.access}, RandomAccessIterator
  template<class I>
    concept RandomAccessIterator = @\seebelownc@;

  // \ref{iterator.concept.contiguous}, ContiguousIterator
  template<class I>
    concept ContiguousIterator = @\seebelownc@;

  // \ref{indirectcallable}, indirect callable requirements
  // \ref{indirectcallable.indirectinvocable}, indirect callables
  template<class F, class I>
    concept IndirectUnaryInvocable = @\seebelownc@;

  template<class F, class I>
    concept IndirectRegularUnaryInvocable = @\seebelownc@;

  template<class F, class I>
    concept IndirectUnaryPredicate = @\seebelownc@;

  template<class F, class I1, class I2 = I1>
    concept IndirectRelation = @\seebelownc@;

  template<class F, class I1, class I2 = I1>
    concept IndirectStrictWeakOrder = @\seebelownc@;

  template<class F, class... Is>
    requires (Readable<Is> && ...) && Invocable<F, iter_reference_t<Is>...>
      using indirect_result_t = invoke_result_t<F, iter_reference_t<Is>...>;

  // \ref{projected}, projected
  template<Readable I, IndirectRegularUnaryInvocable<I> Proj>
    struct projected;

  template<WeaklyIncrementable I, class Proj>
    struct incrementable_traits<projected<I, Proj>>;

  // \ref{commonalgoreq}, common algorithm requirements
  // \ref{commonalgoreq.indirectlymovable} IndirectlyMovable
  template<class In, class Out>
    concept IndirectlyMovable = @\seebelownc@;

  template<class In, class Out>
    concept IndirectlyMovableStorable = @\seebelownc@;

  // \ref{commonalgoreq.indirectlycopyable} IndirectlyCopyable
  template<class In, class Out>
    concept IndirectlyCopyable = @\seebelownc@;

  template<class In, class Out>
    concept IndirectlyCopyableStorable = @\seebelownc@;

  // \ref{commonalgoreq.indirectlyswappable} IndirectlySwappable
  template<class I1, class I2 = I1>
    concept IndirectlySwappable = @\seebelownc@;

  // \ref{commonalgoreq.indirectlycomparable} IndirectlyComparable
  template<class I1, class I2, class R, class P1 = identity, class P2 = identity>
    concept IndirectlyComparable = @\seebelownc@;

  // \ref{commonalgoreq.permutable} Permutable
  template<class I>
    concept Permutable = @\seebelownc@;

  // \ref{commonalgoreq.mergeable} Mergeable
  template<class I1, class I2, class Out,
      class R = ranges::less<>, class P1 = identity, class P2 = identity>
    concept Mergeable = @\seebelownc@;

  template<class I, class R = ranges::less<>, class P = identity>
    concept Sortable = @\seebelownc@;

  // \ref{iterator.primitives}, primitives
  // \ref{std.iterator.tags}, iterator tags
\end{codeblock}\end{addedblock}\begin{codeblock}
  struct input_iterator_tag { };
  struct output_iterator_tag { };
  struct forward_iterator_tag: public input_iterator_tag { };
  struct bidirectional_iterator_tag: public forward_iterator_tag { };
  struct random_access_iterator_tag: public bidirectional_iterator_tag { };
  @\added{struct contiguous_iterator_tag: public random_access_iterator_tag \{ \};}@

  // \ref{iterator.operations}, iterator operations
  template<class InputIterator, class Distance>
    constexpr void
      advance(InputIterator& i, Distance n);
  template<class InputIterator>
    constexpr typename iterator_traits<InputIterator>::difference_type
      distance(InputIterator first, InputIterator last);
  template<class InputIterator>
    constexpr InputIterator
      next(InputIterator x,
           typename iterator_traits<InputIterator>::difference_type n = 1);
  template<class BidirectionalIterator>
    constexpr BidirectionalIterator
      prev(BidirectionalIterator x,
           typename iterator_traits<BidirectionalIterator>::difference_type n = 1);

\end{codeblock}\begin{addedblock}\begin{codeblock}
  // \ref{range.iterator.operations}, range iterator operations
  namespace ranges {
    // \ref{range.iterator.operations.advance}, \tcode{ranges::advance}
    template<Iterator I>
      constexpr void advance(I& i, iter_difference_t<I> n);
    template<Iterator I, Sentinel<I> S>
      constexpr void advance(I& i, S bound);
    template<Iterator I, Sentinel<I> S>
      constexpr iter_difference_t<I> advance(I& i, iter_difference_t<I> n, S bound);

    // \ref{range.iterator.operations.distance}, \tcode{ranges::distance}
    template<Iterator I, Sentinel<I> S>
      constexpr iter_difference_t<I> distance(I first, S last);
    template<Range R>
      constexpr iter_difference_t<iterator_t<R>> distance(R&& r);

    // \ref{range.iterator.operations.next}, \tcode{ranges::next}
    template<Iterator I>
      constexpr I next(I x);
    template<Iterator I>
      constexpr I next(I x, iter_difference_t<I> n);
    template<Iterator I, Sentinel<I> S>
      constexpr I next(I x, S bound);
    template<Iterator I, Sentinel<I> S>
      constexpr I next(I x, iter_difference_t<I> n, S bound);

    // \ref{range.iterator.operations.prev}, \tcode{ranges::prev}
    template<BidirectionalIterator I>
      constexpr I prev(I x);
    template<BidirectionalIterator I>
      constexpr I prev(I x, iter_difference_t<I> n);
    template<BidirectionalIterator I>
      constexpr I prev(I x, iter_difference_t<I> n, I bound);
  }
\end{codeblock}\end{addedblock}\begin{codeblock}

  // \ref{predef.iterators}, predefined iterators \added{and sentinels}
  @\added{// \ref{reverse.iterators}, reverse iterators}@
  template<class Iterator> class reverse_iterator;

  template<class Iterator1, class Iterator2>
    constexpr bool operator==(
      const reverse_iterator<Iterator1>& x,
      const reverse_iterator<Iterator2>& y);
  template<class Iterator1, class Iterator2>
    constexpr bool operator!=(
      const reverse_iterator<Iterator1>& x,
      const reverse_iterator<Iterator2>& y);
  template<class Iterator1, class Iterator2>
    constexpr bool operator<(
      const reverse_iterator<Iterator1>& x,
      const reverse_iterator<Iterator2>& y);
  template<class Iterator1, class Iterator2>
    constexpr bool operator>(
      const reverse_iterator<Iterator1>& x,
      const reverse_iterator<Iterator2>& y);
  template<class Iterator1, class Iterator2>
    constexpr bool operator<=(
      const reverse_iterator<Iterator1>& x,
      const reverse_iterator<Iterator2>& y);
  template<class Iterator1, class Iterator2>
    constexpr bool operator>=(
      const reverse_iterator<Iterator1>& x,
      const reverse_iterator<Iterator2>& y);

  template<class Iterator1, class Iterator2>
    constexpr auto operator-(
      const reverse_iterator<Iterator1>& x,
      const reverse_iterator<Iterator2>& y) -> decltype(y.base() - x.base());
  template<class Iterator>
    constexpr reverse_iterator<Iterator>
      operator+(
    typename reverse_iterator<Iterator>::difference_type n,
    const reverse_iterator<Iterator>& x);

  template<class Iterator>
    constexpr reverse_iterator<Iterator> make_reverse_iterator(Iterator i);

  @\added{// \ref{insert.iterators}, insert iterators}@
  template<class Container> class back_insert_iterator;
  template<class Container>
    back_insert_iterator<Container> back_inserter(Container& x);

  template<class Container> class front_insert_iterator;
  template<class Container>
    front_insert_iterator<Container> front_inserter(Container& x);

  template<class Container> class insert_iterator;
  template<class Container>
    insert_iterator<Container> inserter(Container& x, @\changed{typename Container::iterator}{iterator_t<Container>}@ i);

  @\added{// \ref{move.iterators}, move iterators and sentinels}@
  template<class Iterator> class move_iterator;
  template<class Iterator1, class Iterator2>
    constexpr bool operator==(
      const move_iterator<Iterator1>& x, const move_iterator<Iterator2>& y);
  template<class Iterator1, class Iterator2>
    constexpr bool operator!=(
      const move_iterator<Iterator1>& x, const move_iterator<Iterator2>& y);
  template<class Iterator1, class Iterator2>
    constexpr bool operator<(
      const move_iterator<Iterator1>& x, const move_iterator<Iterator2>& y);
  template<class Iterator1, class Iterator2>
    constexpr bool operator<=(
      const move_iterator<Iterator1>& x, const move_iterator<Iterator2>& y);
  template<class Iterator1, class Iterator2>
    constexpr bool operator>(
      const move_iterator<Iterator1>& x, const move_iterator<Iterator2>& y);
  template<class Iterator1, class Iterator2>
    constexpr bool operator>=(
      const move_iterator<Iterator1>& x, const move_iterator<Iterator2>& y);

  template<class Iterator1, class Iterator2>
    constexpr auto operator-(
    const move_iterator<Iterator1>& x,
    const move_iterator<Iterator2>& y) -> decltype(x.base() - y.base());
  template<class Iterator>
    constexpr move_iterator<Iterator> operator+(
      typename move_iterator<Iterator>::difference_type n, const move_iterator<Iterator>& x);
  template<class Iterator>
    constexpr move_iterator<Iterator> make_move_iterator(Iterator i);

\end{codeblock}\begin{addedblock}\begin{codeblock}
  template<Semiregular S> class move_sentinel;

  // \ref{iterators.common}, common iterators
  template<Iterator I, Sentinel<I> S>
    requires @\newtxt{(}@!Same<I, S>@\newtxt{)}@
      class common_iterator;

  @\oldtxt{template<Readable I, class S>}@
    @\oldtxt{struct readable_traits<common_iterator<I, S>>;}@

  template<InputIterator I, class S>
    struct iterator_traits<common_iterator<I, S>>;

  // \ref{default.sentinels}, default sentinels
  class default_sentinel;

  // \ref{iterators.counted}, counted iterators
  template<Iterator I> class counted_iterator;

  @\oldtxt{template<Readable I>}@
    @\oldtxt{struct readable_traits<counted_iterator<I>>;}@

  template<InputIterator I>
    struct iterator_traits<counted_iterator<I>>;

  // \ref{unreachable.sentinels}, unreachable sentinels
  class unreachable;
\end{codeblock}\end{addedblock}\begin{codeblock}

  // \ref{stream.iterators}, stream iterators
  [...]
}
\end{codeblock}

\rSec1[iterator.requirements]{Iterator requirements}

\rSec2[iterator.requirements.general]{In general}

\pnum
\indextext{requirements!iterator}%
Iterators are a generalization of pointers that allow a \Cpp{} program
to work with different data structures
(\added{for example,} containers \added{and ranges}) in a uniform manner.
To be able to construct template algorithms that work correctly and efficiently
on different types of data structures, the library formalizes  not just
the interfaces but also the semantics and complexity assumptions of iterators.
An input iterator
\tcode{i}
supports the expression
\tcode{*i},
resulting in a value of some object type
\tcode{T},
called the
\term{value type}
of the iterator.
% Think about making this "non-empty set of expression type-and-value-categories"
% what's here now is easy to understand and incorrect.
An output iterator \tcode{i} has a non-empty set of types that are
\defn{writable} to the iterator;
for each such type \tcode{T}, the expression \tcode{*i = o}
is valid where \tcode{o} is a value of type \tcode{T}.
\removed{An iterator
\tcode{i}
for which the expression
\tcode{(*i).m}
is well-defined supports the expression
\tcode{i->m}
with the same semantics as
\tcode{(*i).m}.}
For every iterator type
\tcode{X}
\removed{for which
equality is defined}, there is a corresponding signed integer type called the
\term{difference type}
of the iterator.

\pnum
Since iterators are an abstraction of pointers, their semantics is
a generalization of most of the semantics of pointers in \Cpp{}.
This ensures that every
function template
that takes iterators
works as well with regular pointers.
This document defines
\changed{five}{six} categories of iterators, according to the operations
defined on them:
\term{input iterators},
\term{output iterators},
\term{forward iterators},
\term{bidirectional iterators},
\term{random access iterators},
and
\added{\term{contiguous iterators}},
as shown in \tref{iterators.relations}.

\begin{floattable}{Relations among iterator categories}{tab:iterators.relations}
{lllll}
\topline
              \added{\textbf{Contiguous}}    & \added{$\rightarrow$} \textbf{Random Access} &
$\rightarrow$ \textbf{Bidirectional} & $\rightarrow$ \textbf{Forward}       &
$\rightarrow$ \textbf{Input}                                                \\
   &   &   &   &   $\rightarrow$ \textbf{Output}                            \\
\end{floattable}

\begin{addedblock}
\pnum
The six categories of iterators correspond to the iterator concepts
\libconcept{Input\-Iterator}\iref{iterator.concept.input},
\libconcept{Output\-Iterator}\iref{iterator.concept.output},
\libconcept{Forward\-Iterator}\iref{iterator.concept.forward},
\libconcept{Bidirectional\-Iterator}\iref{iterator.concept.bidirectional}
\libconcept{RandomAccess\-Iterator}\iref{iterator.concept.random.access}, and
\libconcept{Contiguous\-Iterator}\iref{iterator.concept.contiguous}, respectively.
The generic term \defn{iterator} refers to any type that models the
\libconcept{Iterator} concept\iref{iterator.concept.iterator}.
\end{addedblock}

\pnum
Forward iterators satisfy all the requirements of input
iterators and can be used whenever
an input iterator is specified;
Bidirectional iterators also satisfy all the requirements of
forward iterators and can be used whenever a forward iterator is specified;
Random access iterators also satisfy all the requirements of bidirectional
iterators and can be used whenever a bidirectional iterator is specified;
\added{Contiguous iterators also satisfy all the requirements of random access
iterators and can be used whenever a random access iterator is specified}.

\pnum
Iterators that further satisfy the requirements of output iterators are
called \defnx{mutable iterators}{mutable iterator}. Nonmutable iterators are referred to
as \defnx{constant iterators}{constant iterator}.

\pnum
In addition to the requirements in this subclause,
the nested \grammarterm{typedef-name}{s} specified in \ref{iterator.traits}
shall be provided for the iterator type.
\begin{note} Either the iterator type must provide the \grammarterm{typedef-name}{s} directly
(in which case \tcode{iterator_traits} pick\added{s} them up automatically), or
an \tcode{iterator_traits} specialization must provide them. \end{note}

\begin{removedblock}
\pnum
Iterators that further satisfy the requirement that,
for integral values \tcode{n} and
dereferenceable iterator values \tcode{a} and \tcode{(a + n)},
\tcode{*(a + n)} is equivalent to \tcode{*(addressof(*a) + n)},
are called \defn{contiguous iterators}.
\begin{note}
For example, the type ``pointer to \tcode{int}'' is a contiguous iterator,
but \tcode{reverse_iterator<int *>} is not.
For a valid iterator range $[$\tcode{a}$, $\tcode{b}$)$ with dereferenceable \tcode{a},
the corresponding range denoted by pointers is
$[$\tcode{addressof(*a)}$, $\tcode{addressof(*a) + (b - a)}$)$;
\tcode{b} might not be dereferenceable.
\end{note}
\end{removedblock}

\pnum
Just as a regular pointer to an array guarantees that there is a pointer value pointing past the last element
of the array, so for any iterator type there is an iterator value that points past the last element of a
corresponding sequence.
These values are called
\term{past-the-end}
values.
Values of an iterator
\tcode{i}
for which the expression
\tcode{*i}
is defined are called
\term{dereferenceable}.
The library never assumes that past-the-end values are dereferenceable.
Iterators can also have singular values that are not associated with any
sequence.
\begin{example}
After the declaration of an uninitialized pointer
\tcode{x}
(as with
\tcode{int* x;}),
\tcode{x}
must always be assumed to have a singular value of a pointer.
\end{example}
Results of most expressions are undefined for singular values;
the only exceptions are destroying an iterator that holds a singular value,
the assignment of a non-singular value to
an iterator that holds a singular value, and, for iterators that satisfy the
\oldconcept{DefaultConstructible} requirements, using a value-initialized iterator
as the source of a copy or move operation. \begin{note} This guarantee is not
offered for default-initialization, although the distinction only matters for types
with trivial default constructors such as pointers or aggregates holding pointers.
\end{note}
In these cases the singular
value is overwritten the same way as any other value.
Dereferenceable
values are always non-singular.

\begin{removedblock}
\pnum
An iterator
\tcode{j}
is called
\term{reachable}
from an iterator
\tcode{i}
if and only if there is a finite sequence of applications of
the expression
\tcode{++i}
that makes
\tcode{i == j}.
If
\tcode{j}
is reachable from
\tcode{i},
they refer to elements of the same sequence.

\pnum
Most of the library's algorithmic templates that operate on data structures have interfaces that use ranges.
A
\term{range}
is a pair of iterators that designate the beginning and end of the computation.
A range \range{i}{i}
is an empty range;
in general, a range \range{i}{j}
refers to the elements in the data structure starting with the element
pointed to by
\tcode{i}
and up to but not including the element pointed to by
\tcode{j}.
Range \range{i}{j}
is valid if and only if
\tcode{j}
is reachable from
\tcode{i}.
The result of the application of functions in the library to invalid ranges is
undefined.
\end{removedblock}

\begin{addedblock}
\pnum
Most of the library's algorithmic templates that operate on data structures have
interfaces that use ranges. A \term{range} is an iterator and a \term{sentinel}
that designate the beginning and end of the computation, or an iterator and a
count that designate the beginning and the number of elements to which the
computation is to be applied.\footnote{The sentinel denoting the end of a range
may have the same type as the iterator denoting the beginning of the range, or a
different type.}

\pnum
An iterator and a sentinel denoting a range are comparable.
A range \range{i}{s}
is empty if \tcode{i == s};
otherwise, \range{i}{s}
refers to the elements in the data structure starting with the element
pointed to by
\tcode{i}
and up to but not including the element\newtxt{, if any,} pointed to by
the first iterator \tcode{j} such that \tcode{j == s}.

\pnum
A sentinel \tcode{s} is called \term{reachable} from an iterator \tcode{i} if
and only if there is a finite sequence of applications of the expression
\tcode{++i} that makes \tcode{i == s}. If \tcode{s} is reachable from \tcode{i},
\range{i}{s} denotes a \newtxt{valid} range.

\pnum
A counted range \range{i}{n} is empty if \tcode{n == 0}; otherwise, \range{i}{n}
refers to the \tcode{n} elements in the data structure starting with the element
pointed to by \tcode{i} and up to but not including the element\newtxt{, if any,} pointed to by
the result of incrementing \tcode{i} \tcode{n} times. \newtxt{A counted range
\range{i}{n} is valid if and only if \tcode{n == 0}; or \tcode{n} is positive,
\tcode{i} is dereferenceable, and \range{++i}{-{-}n} is valid.}

\pnum
\oldtxt{A range \range{i}{s} is valid if and only if \tcode{s} is reachable from
\tcode{i}. A counted range \range{i}{n} is valid if and only if \tcode{n == 0};
or \tcode{n} is positive, \tcode{i} is dereferenceable, and \range{++i}{-{-}n}
is valid.} The result of the application of \newtxt{library} functions \oldtxt{in the library} to invalid
ranges is undefined.
\end{addedblock}

\pnum
All the categories of iterators require only those functions
that are realizable for a given category in constant time (amortized).
Therefore, requirement tables \added{and concept definitions} for the iterators
do not \removed{have a} \added{specify} complexity \removed{column}.

\pnum
Destruction of a\oldoldtxt{n} \newnewtxt{non-forward} iterator \added{\oldtxt{whose category is weaker than forward}}
may invalidate pointers and references previously obtained from that iterator.

\pnum
An
\term{invalid}
iterator is an iterator that may be singular.\footnote{This definition applies
to pointers, since pointers are iterators. The effect of dereferencing
an iterator that has been invalidated is undefined.}

\pnum
\indextext{iterator!constexpr}%
Iterators are called \defn{constexpr iterators}
if all operations provided to satisfy iterator category operations
are constexpr functions, except for
\begin{itemize}
\item \tcode{swap},
\item a pseudo-destructor call\cxxiref{expr.pseudo}, and
\item the construction of an iterator with a singular value.
\end{itemize}
\begin{note}
For example, the types ``pointer to \tcode{int}'' and
\tcode{reverse_iterator<int*>} are constexpr iterators.
\end{note}

\begin{removedblock}
\pnum
In the following sections,
\tcode{a}
and
\tcode{b}
denote values of type
\tcode{X} or \tcode{const X},
\tcode{difference_type} and \tcode{reference} refer to the
types \tcode{iterator_traits<X>::difference_type} and
\tcode{iterator_traits<X>::reference}, respectively,
\tcode{n}
denotes a value of
\tcode{difference_type},
\tcode{u},
\tcode{tmp},
and
\tcode{m}
denote identifiers,
\tcode{r}
denotes a value of
\tcode{X\&},
\tcode{t}
denotes a value of value type
\tcode{T},
\tcode{o}
denotes a value of some type that is writable to the output iterator.
\begin{note} For an iterator type \tcode{X} there must be an instantiation
of \tcode{iterator_traits<X>}\iref{iterator.traits}. \end{note}
\end{removedblock}

\begin{addedblock}
\rSec2[iterator.assoc.types]{Associated types}
\rSec3[incrementable.traits]{Incrementable traits}

\pnum
To implement algorithms only in terms of incrementable types,
it is often necessary to determine the difference type that
corresponds to a particular incrementable type. Accordingly,
it is required that if \tcode{WI} is the name of a type that models  the
\tcode{WeaklyIncrementable} concept\iref{iterator.concept.weaklyincrementable},
the type

\begin{codeblock}
iter_difference_t<WI>
\end{codeblock}

be defined as the incrementable type's difference type.

{\color{oldclr}
\pnum
\tcode{iter_difference_t} is implemented as if:
} %% \color{oldclr}

\indexlibrary{\idxcode{iter_difference_t}}%
\indexlibrary{\idxcode{incrementable_traits}}%
\begin{codeblock}
namespace std {
  template<class> struct incrementable_traits { };

  template<class T>
    requires is_object_v<T>
  struct incrementable_traits<T*> {
    using difference_type = ptrdiff_t;
  };

  template<class I>
  struct incrementable_traits<const I>
    : incrementable_traits<@\oldtxt{decay_t}\newtxt{remove_const_t}@<I>> { };

  template<class T>
    requires requires { typename T::difference_type; }
  struct incrementable_traits<T> {
    using difference_type = typename T::difference_type;
  };

  template<class T>
    requires @\newtxt{(}@!requires { typename T::difference_type; }@\newtxt{)}@ &&
      requires(const T& a, const T& b) { { a - b } -> Integral; }
  struct incrementable_traits<T> {
    using difference_type = make_signed_t<decltype(declval<T>() - declval<T>())>;
  };

  template<class T>
    using iter_difference_t = @\seebelownc@;
}
\end{codeblock}

\pnum
If \tcode{iterator_traits<I>} \oldtxt{is a program-defined specialization,
then \tcode{iter_difference_t<I>} denotes
\tcode{iterator_traits<I>::difference_type}; otherwise, it denotes
\tcode{incrementable_traits<I>::difference_type}}
\newtxt{names an instantiation of the primary template,
then }\tcode{\newtxt{iter_difference_t<I>}}\newtxt{ denotes
}\tcode{\newtxt{incrementable_traits<I>::difference_type}}\newtxt{;
otherwise, it denotes
}\tcode{\newtxt{iterator_traits<I>::difference_type}}.

\pnum
Users may specialize \tcode{incrementable_traits} on program-defined types.

\rSec3[readable.traits]{Readable traits}

\pnum
To implement algorithms only in terms of readable types, it is often necessary
to determine the value type that corresponds to a particular readable type.
Accordingly, it is required that if \tcode{R} is the name of a type that
models the \tcode{Readable} concept\iref{iterator.concept.readable},
the type

\begin{codeblock}
iter_value_t<R>
\end{codeblock}

be defined as the readable type's value type.

{\color{oldclr}
\pnum
\tcode{iter_value_t} is implemented as if:
} %% \color{oldclr}

\indexlibrary{\idxcode{iter_value_t}}%
\indexlibrary{\idxcode{readable_traits}}%
\begin{codeblock}
  template<class> struct @\placeholder{cond-value-type}@ { }; // \expos
  template<class T>
    requires is_object_v<T>
  struct @\placeholder{cond-value-type}@ {
    using value_type = remove_cv_t<T>;
  };

  template<class> struct readable_traits { };

  template<class T>
  struct readable_traits<T*>
    : @\placeholder{cond-value-type}@<T> { };

  template<class I>
    requires is_array_v<I>
  struct readable_traits<I>
    : readable_traits<decay_t<I>> { };

  template<class I>
  struct readable_traits<const I>
    : readable_traits<remove_const_t<I>> { };

  template<class T>
    requires requires { typename T::value_type; }
  struct readable_traits<T>
    : @\placeholder{cond-value-type}@<typename T::value_type> { };

  template<class T>
    requires requires { typename T::element_type; }
  struct readable_traits<T>
    : @\placeholder{cond-value-type}@<typename T::element_type> { };

  template<class T> using iter_value_t = @\seebelownc@;
\end{codeblock}

\pnum
If \tcode{iterator_traits<I>} \oldtxt{is a program-defined specialization, then
\tcode{iter_value_t<I>} denotes \tcode{iterator_traits<I>::value_type};
otherwise, it denotes \tcode{readable_traits<I>::value_type}}
\newtxt{names an instantiation of the primary template, then
}\tcode{\newtxt{iter_value_t<I>}}\newtxt{ denotes
}\tcode{\newtxt{readable_traits<I>::value_type}}\newtxt{;
otherwise, it denotes }\tcode{\newtxt{iterator_traits<I>::value_type}}.

\pnum
Class template \tcode{readable_traits} may be specialized
on program-defined types.

\pnum
\begin{note}
Some legacy output iterators define a nested type named \tcode{value_type}
that is an alias for \tcode{void}. These types are not \tcode{Readable}
and have no associated value types.
\end{note}

\pnum
\begin{note}
Smart pointers like \tcode{shared_ptr<int>} are \tcode{Readable} and
have an associated value type, but a smart pointer like \tcode{shared_ptr<void>}
is not \tcode{Readable} and has no associated value type.
\end{note}
\end{addedblock}

\ednote{Relocate [iterator.traits] here and modify as follows:}
\rSec3[iterator.traits]{Iterator traits}

\pnum
\indexlibrary{\idxcode{iterator_traits}}%
To implement algorithms only in terms of iterators, it is \changed{often}{sometimes}
necessary to determine the
\changed{value and difference types}{iterator category} that
correspond\added{s} to a particular iterator type. Accordingly,
it is required that if \tcode{\changed{Iterator}{I}} is the type of an iterator,
the type\removed{s}

\indexlibrarymember{iterator_category}{iterator_traits}%
\begin{codeblock}
@\removed{iterator_traits<Iterator>::difference_type}@
@\removed{iterator_traits<Iterator>::value_type}@
iterator_traits<@\changed{Iterator}{I}@>::iterator_category
\end{codeblock}

be defined as the iterator's \removed{difference type, value type
and} iterator category\removed{, respectively}.
In addition, the types

\indexlibrarymember{reference}{iterator_traits}%
\indexlibrarymember{pointer}{iterator_traits}%
\begin{codeblock}
@\oldoldtxt{iterator_traits<Iterator>::reference}@
iterator_traits<@\changed{Iterator}{I}@>::pointer
@\newnewtxt{iterator_traits<I>::reference}@
\end{codeblock}

shall be defined as the iterator's \newnewtxt{pointer and} reference \oldoldtxt{and pointer} types\changed{,}{;} that is, for an
iterator object \tcode{a} \newnewtxt{of class type}, the same type as
\oldoldtxt{the type of \tcode{*a} and \tcode{a->}}
\newnewtxt{\tcode{decltype(a.operator->())} and \tcode{decltype(*a)}},
respectively. \added{The type
}\tcode{\added{iterator_traits<I>::pointer}}\added{ shall be \tcode{void} for
a\newnewtxt{n iterator of class} type \tcode{I} that does not support
\tcode{operator->}. Additionally, i}\removed{I}n the case of an output iterator,
the types

\begin{codeblock}
@\newnewtxt{iterator_traits<I>::value_type}@
iterator_traits<@\changed{Iterator}{I}@>::difference_type
@\oldoldtxt{iterator_traits<Iterator>::value_type}@
iterator_traits<@\changed{Iterator}{I}@>::reference
@\removed{iterator_traits<Iterator>::pointer}@
\end{codeblock}

may be defined as \tcode{void}.

\begin{addedblock}
\pnum
\oldtxt{The member types of non-program-defined specializations of
\tcode{iterator_traits} are computed as defined below. The definition below uses
several exposition-only concepts equivalent to the following:}
\newtxt{The definitions in this subclause make use of the following
exposition-only concepts:}

\begin{codeblock}
template<class I>
concept @\placeholder{\oldtxt{_Cpp17Iterator}} \placeholder{\newtxt{cpp17-iterator}}@ =
  Copyable<I> && requires (I i) {
    @\oldtxt{\{}@ *i @\oldtxt{\} -> auto\&\&;}@
    @\newtxt{requires \placeholdernc{can-reference}<decltype(*i)>;}@
    { ++i } -> Same<I>&;
    @\oldtxt{\{}@ *i++ @\oldtxt{\} -> auto\&\&;}@
    @\newtxt{requires \placeholdernc{can-reference}<decltype(*i++)>;}@
  };

template<class I>
concept @\placeholder{\oldtxt{_Cpp17InputIterator}} \placeholder{\newtxt{cpp17-input-iterator}}@ =
  @\placeholder{\oldtxt{_Cpp17Iterator}} \placeholder{\newtxt{cpp17-iterator}}@<I> && EqualityComparable<I> && requires (I i) {
    typename incrementable_traits<I>::difference_type;
    typename readable_traits<I>::value_type;
    typename common_reference_t<iter_reference_t<I> &&,
                                typename readable_traits<I>::value_type &>;
    typename common_reference_t<decltype(*i++) &&,
                                typename readable_traits<I>::value_type &>;
    requires SignedIntegral<typename incrementable_traits<I>::difference_type>;
  };

template<class I>
concept @\placeholder{\oldtxt{_Cpp17ForwardIterator}} \placeholder{\newtxt{cpp17-forward-iterator}}@ =
  @\placeholder{\oldtxt{_Cpp17InputIterator}} \placeholder{\newtxt{cpp17-input-iterator}}@<I> && Constructible<I> &&
  @\newtxt{is_lvalue_reference_v<iter_reference_t<I>> \&\&}@
  Same<remove_cvref_t<iter_reference_t<I>>, typename readable_traits<I>::value_type> &&
  requires (I i) {
    { i++ } -> const I&;
    requires Same<iter_reference_t<I>, decltype(*i++)>;
  };

template<class I>
concept @\placeholder{\oldtxt{_Cpp17BidirectionalIterator}} \placeholder{\newtxt{cpp17-bidirectional-iterator}}@ =
  @\placeholder{\oldtxt{_Cpp17ForwardIterator}} \placeholder{\newtxt{cpp17-forward-iterator}}@<I> && requires (I i) {
    { --i } -> Same<I>&;
    { i-- } -> const I&;
    requires Same<iter_reference_t<I>, decltype(*i--)>;
  };

template<class I>
concept @\placeholder{\oldtxt{_Cpp17RandomAccessIterator}} \placeholder{\newtxt{cpp17-random-access-iterator}}@ =
  @\placeholder{\oldtxt{_Cpp17BidirectionalIterator}} \placeholder{\newtxt{cpp17-bidirectional-iterator}}@<I> && StrictTotallyOrdered<I> &&
  requires (I i, typename incrementable_traits<I>::difference_type n) {
    { i += n } -> Same<I>&;
    { i -= n } -> Same<I>&;
    requires Same<I, decltype(i + n)>;
    requires Same<I, decltype(n + i)>;
    requires Same<I, decltype(i - n)>;
    requires Same<decltype(n), decltype(i - i)>;
    { i[n] } -> iter_reference_t<I>;
  };
\end{codeblock}

{\color{newclr}
\pnum
The members of an instantiation \tcode{iterator_traits<I>} of the
\tcode{iterator_traits} primary template are computed as follows:
} %% \color{newclr}
\end{addedblock}

\begin{itemize}
\item
If \tcode{\changed{Iterator}{I}} has valid\cxxiref{temp.deduct} member
types \tcode{difference_type}, \tcode{value_type}, \removed{\tcode{pointer},}
\tcode{reference}, and \tcode{iterator_category}, \newnewtxt{then}
\tcode{iterator_traits<\changed{Iterator}{I}>}
\oldoldtxt{shall have} \newnewtxt{has} the following \oldoldtxt{as}
publicly accessible members:
\begin{codeblock}
  @\newnewtxt{using iterator_category = typename I::iterator_category;}@
  @\oldoldtxt{using difference_type{ }{ }{ }= typename Iterator::difference_type;}@
  using value_type        = typename @\changed{Iterator}{I}@::value_type;
  @\newnewtxt{using difference_type{ }{ }{ }= typename I::difference_type;}@
  using pointer           = @\changed{typename Iterator::pointer}{\seebelownc}@;
  using reference         = typename @\changed{Iterator}{I}@::reference;
  @\oldoldtxt{using iterator_category = typename Iterator::iterator_category;}@
\end{codeblock}
\begin{addedblock}
If the \grammarterm{qualified-id} \tcode{I::pointer} is valid and
denotes a type, then \tcode{iterator_traits<I>::pointer} names that type;
otherwise, it names \tcode{void}.

\item
Otherwise, if \tcode{I} satisfies the exposition-only concept
\tcode{\placeholder{\oldtxt{_Cpp17InputIterator}} \placeholder{\newtxt{cpp17-input-iterator}}},
\newtxt{then} \tcode{iterator_traits<I>} \oldtxt{shall have} \newtxt{has} the following
\oldtxt{as} publicly accessible members:
\begin{codeblock}
  using iterator_category = @\seebelownc@;
  using value_type        = typename readable_traits<I>::value_type;
  using difference_type   = typename incrementable_traits<I>::difference_type;
  using pointer           = @\seebelownc@;
  using reference         = @\seebelownc@;
\end{codeblock}
\begin{itemize}
\item If the \grammarterm{qualified-id} \tcode{I::pointer} is valid and denotes a type,
\tcode{pointer} names that type. Otherwise, if
\tcode{decltype(\brk{}declval<I\&>().operator->())} is well-formed, then
\tcode{pointer} names that type. Otherwise, \tcode{pointer} \oldtxt{is}
\newtxt{names} \tcode{void}.

\item If the \grammarterm{qualified-id} \tcode{I::reference} is valid and denotes a
type, \tcode{reference} names that type. Otherwise, \tcode{reference} \oldtxt{is}
\newtxt{names} \tcode{iter_reference_t<I>}.

\item If the \grammarterm{qualified-id} \tcode{I::iterator_category} is valid and
denotes a type, \tcode{iterator_category} names that type. Otherwise, if
\tcode{I} satisfies \tcode{\placeholder{\oldtxt{_Cpp17RandomAccessIterator}}
\placeholder{\newtxt{cpp17-random-access-iterator}}},
\tcode{iterator_category} \oldtxt{is} \newtxt{names}
\tcode{random_access_iterator_tag}. Otherwise, if
\tcode{I} satisfies \tcode{\placeholder{\oldtxt{_Cpp17BidirectionalIterator}}
\placeholder{\newtxt{cpp17-bidirectional-iterator}}},
\tcode{iterator_category} \oldtxt{is} \newtxt{names}
\tcode{bidirectional_iterator_tag}. Otherwise, if
\tcode{I} satisfies \tcode{\placeholder{\oldtxt{_Cpp17ForwardIterator}}
\placeholder{\newtxt{cpp17-forward-iterator}}},
\tcode{iterator_category} \oldtxt{is} \newtxt{names}
\tcode{forward_iterator_tag}.
Otherwise, \tcode{iterator_category} \oldtxt{is} \newtxt{names}
\tcode{input_iterator_tag}.
\end{itemize}

\item
Otherwise, if \tcode{I} satisfies the exposition-only concept
\tcode{\placeholder{\oldtxt{_Cpp17Iterator}}
\placeholder{\newtxt{cpp17-iterator}}}, \newtxt{then} \tcode{iterator_traits<I>}
\oldtxt{shall have} \newtxt{has} the following \oldtxt{as} publicly accessible
members:
\begin{codeblock}
  using iterator_category = output_iterator_tag;
  using value_type        = void;
  using difference_type   = @\seebelownc@;
  using pointer           = void;
  using reference         = void;
\end{codeblock}
If \tcode{incrementable_traits<I>::difference_type} is well-formed
and names a type, then \tcode{difference_type} names that type; otherwise, it
\oldtxt{is} \newtxt{names} \tcode{void}.
\end{addedblock}

\item
Otherwise, \tcode{iterator_traits<\changed{Iterator}{I}>}
\oldoldtxt{shall have} \newnewtxt{has} no members by any of the above names.
\end{itemize}

\begin{addedblock}
\pnum
\oldtxt{Additionally, program-defined} Specializations of \tcode{iterator_traits}
\newtxt{that are not instantiations of the primary template} may
have a member type \tcode{iterator_concept} that is used to opt in or out of
conformance to the iterator concepts defined in~\ref{iterator.concepts}.
\oldtxt{If specified, it should be an alias for one of the standard iterator tag
types\iref{std.iterator.tags}, or an empty, copy- and move-constructible,
trivial class type that is publicly and unambiguously derived from one of the
standard iterator tag types.}
\end{addedblock}

\pnum
\removed{It}\tcode{\added{iterator_traits}} is specialized for pointers as

\begin{codeblock}
namespace std {
  template<class T>
    @\added{requires is_object_v<T>}@
  struct iterator_traits<T*> {
    @\newnewtxt{using iterator_concept}@  @\newnewtxt{= contiguous_iterator_tag;}@
    @\newnewtxt{using iterator_category = random_access_iterator_tag;}@
    @\newnewtxt{using value_type}@        @\newnewtxt{= remove_cv_t<T>;}@
    using difference_type   = ptrdiff_t;
    @\oldoldtxt{using value_type}@        @\oldoldtxt{= remove_cv_t<T>;}@
    using pointer           = T*;
    using reference         = T&;
    @\oldoldtxt{using iterator_category = random_access_iterator_tag;}@
    @\oldtxt{using iterator_concept}@  @\oldtxt{= contiguous_iterator_tag;}@
  };
}
\end{codeblock}

\pnum
\begin{example}
To implement a generic
\tcode{reverse}
function, a \Cpp{} program can do the following:

\begin{codeblock}
template<class B@\newnewtxt{I}\oldoldtxt{idirectionalIterator}@>
void reverse(B@\newnewtxt{I}\oldoldtxt{idirectionalIterator}@ first, B@\newnewtxt{I}\oldoldtxt{idirectionalIterator}@ last) {
  typename iterator_traits<B@\newnewtxt{I}\oldoldtxt{idirectionalIterator}@>::difference_type n =
    distance(first, last);
  --n;
  while(n > 0) {
    typename iterator_traits<B@\newnewtxt{I}\oldoldtxt{idirectionalIterator}@>::value_type
     tmp = *first;
    *first++ = *--last;
    *last = tmp;
    n -= 2;
  }
}
\end{codeblock}
\end{example}

\begin{addedblock}
\rSec2[iterator.custpoints]{Customization points}
\rSec3[iterator.custpoints.iter_move]{\tcode{iter_move}}

\pnum
The name \tcode{iter_move} denotes a \oldtxt{\term{customization point object}}
\newtxt{customization point object}\cxxiref{customization.point.object}.
The expression \tcode{ranges::iter_move(E)} for some subexpression \tcode{E} is
expression-equivalent to the following:

\begin{itemize}
\item \tcode{iter_move(E)}, if that expression is \newtxt{valid, with overload
resolution performed in a context that does not include a declaration of
}\tcode{\newtxt{ranges::iter_move}}\newtxt{.} \oldtxt{well-formed when evaluated in a context
that does not include \tcode{ranges::iter_move}
but does include the lookup set produced by
argument-dependent lookup\cxxiref{basic.lookup.argdep}.}

\item Otherwise, if the expression \tcode{*E} is well-formed:
\begin{itemize}
\item if \tcode{*E} is an lvalue, \tcode{std::move(*E)};

\item otherwise, \tcode{*E}.
\end{itemize}

\item Otherwise, \tcode{ranges::iter_move(E)} is ill-formed.
\end{itemize}

\pnum
If \tcode{ranges::iter_move(E)} is not equal to \tcode{*E}, the program is
ill-formed with no diagnostic required.

\rSec3[iterator.custpoints.iter_swap]{\tcode{iter_swap}}

\pnum
The name \tcode{iter_swap} denotes a \oldtxt{\term{customization point object}}
\newtxt{customization point object}\cxxiref{customization.point.object}
\newtxt{that exchanges the values\iref{concept.swappable} denoted by its
arguments}.

\pnum
\newtxt{Let} \tcode{\placeholder{iter-exchange-move}} \oldtxt{is an} \newtxt{be the} exposition-only function \oldtxt{specified as}:
\begin{itemdecl}
template<class X, class Y>
  constexpr iter_value_t<remove_reference_t<X>> @\placeholdernc{iter-exchange-move}@(X&& x, Y&& y)
    noexcept(@\oldtxt{\seebelow} \newtxt{noexcept(iter_value_t<remove_reference_t<X>>(iter_move(x))) \&\&}@
      @\newtxt{noexcept(*x = iter_move(y))}@);
\end{itemdecl}

\begin{itemdescr}
\pnum
\effects Equivalent to:
\begin{codeblock}
iter_value_t<remove_reference_t<X>> old_value(iter_move(x));
*x = iter_move(y);
return old_value;
\end{codeblock}

\pnum
\oldtxt{\remarks The expression in the \tcode{noexcept} is equivalent to:}
\begin{codeblock}
@\oldtxt{NE(remove_reference_t<X>, remove_reference_t<Y>) \&\&}@
@\oldtxt{NE(remove_reference_t<Y>, remove_reference_t<X>)}@
\end{codeblock}
\oldtxt{Where \tcode{NE(T1, T2)} is the expression:}
\begin{codeblock}
@\oldtxt{is_nothrow_constructible_v<iter_value_t<T1>, iter_rvalue_reference_t<T1>> \&\&}@
@\oldtxt{is_nothrow_assignable_v<iter_value_t<T1>\&, iter_rvalue_reference_t<T1>> \&\&}@
@\oldtxt{is_nothrow_assignable_v<iter_reference_t<T1>, iter_rvalue_reference_t<T2>> \&\&}@
@\oldtxt{is_nothrow_assignable_v<iter_reference_t<T1>, iter_value_t<T2>> \&\&}@
@\oldtxt{is_nothrow_move_constructible_v<iter_value_t<T1>> \&\&}@
@\oldtxt{noexcept(ranges::iter_move(declval<T1\&>()))}@
\end{codeblock}
\end{itemdescr}

\pnum
The expression \tcode{ranges::iter_swap(E1, E2)} for some subexpressions
\tcode{E1} and \tcode{E2} is expression-equivalent to the following:

\begin{itemize}
\item \tcode{(void)iter_swap(E1, E2)}, if that expression is \oldtxt{well-formed when
evaluated in a context that does not include \tcode{ranges::iter_swap} but does
include the lookup set produced by
argument-dependent lookup\cxxiref{basic.lookup.argdep}
and the following declaration:} \newtxt{valid, with overload resolution
performed in a context that includes the declaration}
\begin{codeblock}
template<class I1, class I2>
  void iter_swap(I1, I2) = delete;
\end{codeblock}
\newtxt{and does not include a declaration of
}\tcode{\newtxt{ranges::iter_swap}}\newtxt{.}
If \tcode{\oldtxt{ranges::iter_swap(E1, E2)}} \newtxt{the function selected
by overload resolution} does not \oldtxt{swap} \newtxt{exchange} the values
denoted by \oldtxt{the expressions} \tcode{E1} and \tcode{E2}, the program is
ill-formed with no diagnostic required.

\item Otherwise, if the types of \tcode{E1} and \tcode{E2} \oldtxt{both} \newtxt{each} model
\tcode{Readable}, and if the reference type\newtxt{s} of \tcode{E1} \oldtxt{is swappable
with\iref{concept.swappable} the reference type of} \newtxt{and} \tcode{E2}
\newtxt{model \libconcept{SwappableWith}\iref{concept.swappable}},
then \tcode{ranges::swap(*E1, *E2)}\newtxt{.}

\item Otherwise, if the types \tcode{T1} and \tcode{T2} of \tcode{E1} and
\tcode{E2} model \tcode{IndirectlyMovableStorable<T1, T2>} and
\tcode{IndirectlyMovableStorable<T2, T1>}, \newtxt{then}
\tcode{(void)(*E1 = \placeholdernc{iter-exchange-move}(E2, E1))},
except that \tcode{E1} is evaluated only once.

\item Otherwise, \tcode{ranges::iter_swap(E1, E2)} is ill-formed.
\end{itemize}

\rSec2[iterator.concepts]{Iterator concepts}

\rSec3[iterator.concepts.general]{General}

{\color{oldclr}
\pnum
Many of the concepts defined in this subclause\iref{iterator.concepts} use the
exposition-only type function \tcode{\placeholder{ITER_CONCEPT}} in their
specifications.
} %% \color{oldclr}

\pnum
For a type \tcode{I}, let \tcode{\placeholdernc{ITER_TRAITS}(I)} denote the type
\tcode{I} if \tcode{iterator_traits<I>} names an instantiation of the primary
template. Otherwise, \tcode{\placeholdernc{ITER_TRAITS}(I)} denotes
\tcode{iterator_traits<I>}.
\begin{itemize}
\item If the \grammarterm{qualified-id}
  \tcode{\placeholdernc{ITER_TRAITS}(I)::iterator_concept} is valid
  and names a type, then \tcode{\placeholdernc{ITER_CONCEPT}(I)} denotes that
  type.
\item Otherwise, if \newtxt{the \grammarterm{qualified-id}}
  \tcode{\placeholdernc{ITER_TRAITS}(I)\brk{}::iterator_category}
  is valid and names a type, then \tcode{\placeholdernc{ITER_CONCEPT}(I)}
  denotes that type.
\item Otherwise, if \tcode{iterator_traits<I>} names an instantiation of
  the primary template, then \tcode{\placeholdernc{ITER_CONCEPT}(I)} denotes
  \tcode{random_access_iterator_tag}.
\item Otherwise, \tcode{\placeholdernc{ITER_CONCEPT}(I)} does not denote a type.
\end{itemize}

{\color{newclr}
\pnum
\begin{note}
\tcode{\placeholdernc{ITER_TRAITS}} enables independent syntactic determination
of an iterator's category and concept.
\end{note}
\begin{example}
\begin{codeblock}
struct I {
  using value_type = int;
  using difference_type = int;

  int operator*() const;
  I& operator++();
  I operator++(int);
  I& operator--();
  I operator--(int);

  bool operator==(I) const;
  bool operator!=(I) const;
};
\end{codeblock}
\tcode{iterator_traits<I>::iterator_category} denotes \tcode{input_iterator_tag},
and \tcode{\placeholder{ITER_CONCEPT}(I)} denotes \tcode{random_access_iterator}.
\end{example}
} %% \color{newclor}

\rSec3[iterator.concept.readable]{Concept \libconcept{Readable}}

\pnum
The \libconcept{Readable} concept is satisfied by types that are readable by
applying \tcode{operator*} including pointers, smart pointers, and iterators.

\indexlibrary{\idxcode{Readable}}%
\begin{codeblock}
template<class In>
  concept Readable =
    requires {
      typename iter_value_t<In>;
      typename iter_reference_t<In>;
      typename iter_rvalue_reference_t<In>;
    } &&
    CommonReference<iter_reference_t<In>&&, iter_value_t<In>&> &&
    CommonReference<iter_reference_t<In>&&, iter_rvalue_reference_t<In>&&> &&
    CommonReference<iter_rvalue_reference_t<In>&&, const iter_value_t<In>&>;
\end{codeblock}

\pnum
Given a value \tcode{i} of type \tcode{I}, \tcode{I} models \libconcept{Readable}
only if the expression \tcode{*i} (which is indirectly required to be valid via the
exposition-only \placeholder{dereferenceable} concept\iref{iterator.synopsis}) is
equality-preserving.

\rSec3[iterator.concept.writable]{Concept \tcode{Writable}}

\pnum
The \tcode{Writable} concept specifies the requirements for writing a value
into an iterator's referenced object.

\indexlibrary{\idxcode{Writable}}%
\begin{codeblock}
template<class Out, class T>
  concept Writable =
    requires(Out&& o, T&& t) {
      *o = std::forward<T>(t); // not required to be equality-preserving
      *std::forward<Out>(o) = std::forward<T>(t); // not required to be equality-preserving
      const_cast<const iter_reference_t<Out>&&>(*o) =
        std::forward<T>(t); // not required to be equality-preserving
      const_cast<const iter_reference_t<Out>&&>(*std::forward<Out>(o)) =
        std::forward<T>(t); // not required to be equality-preserving
    };
\end{codeblock}

\pnum
Let \tcode{E} be an an expression such that \tcode{decltype((E))} is \tcode{T},
and let \tcode{o} be a dereferenceable object of type \tcode{Out}.
\tcode{Out} and \tcode{T} model \tcode{Writable<Out, T>} only if

\begin{itemize}
\item If \tcode{Readable<Out> \&\& Same<iter_value_t<Out>, decay_t<T>{>}} is satisfied,
then \tcode{*o} after any above assignment is equal
to the value of \tcode{E} before the assignment.
\end{itemize}

\pnum
After evaluating any above assignment expression, \tcode{o} is not required to be dereferenceable.

\pnum
If \tcode{E} is an xvalue\cxxiref{basic.lval}, the resulting
state of the object it denotes is valid but unspecified\cxxiref{lib.types.movedfrom}.

\pnum
\begin{note}
The only valid use of an \tcode{operator*} is on the left side of the assignment statement.
Assignment through the same value of the writable type happens only once.
\end{note}

{\color{newclr}
\pnum
\begin{note}
\tcode{Writable} requires the awkward \tcode{const_cast} expressions to reject
iterators with prvalue non-proxy reference types that permit rvalue
assignment but do not also permit \tcode{const} rvalue assignment.
Consequently, an iterator type \tcode{I} that returns \tcode{std::string}
by value does not model \libconcept{Writable<I, std::string>}.
\end{note}
} %% \color{newclr}

\rSec3[iterator.concept.weaklyincrementable]{Concept \tcode{WeaklyIncrementable}}

\pnum
The \tcode{WeaklyIncrementable} concept specifies the requirements on
types that can be incremented with the pre- and post-increment operators.
The increment operations are not required to be equality-preserving,
nor is the type required to be \libconcept{EqualityComparable}.

\indexlibrary{\idxcode{WeaklyIncrementable}}%
\begin{codeblock}
template<class I>
  concept WeaklyIncrementable =
    Semiregular<I> &&
    requires(I i) {
      typename iter_difference_t<I>;
      requires SignedIntegral<iter_difference_t<I>>;
      { ++i } -> Same<I>&; // not required to be equality-preserving
      i++; // not required to be equality-preserving
    };
\end{codeblock}

\pnum
Let \tcode{i} be an object of type \tcode{I}. When \tcode{i} is in the domain of
both pre- and post-increment, \tcode{i} is said to be \term{incrementable}.
\tcode{WeaklyIncrementable<I>} is satisfied only if

\begin{itemize}
\item The expressions \tcode{++i} and \tcode{i++} have the same domain.
\item If \tcode{i} is incrementable, then both \tcode{++i}
  and \tcode{i++} advance \tcode{i} to the next element.
\item If \tcode{i} is incrementable, then
  \tcode{addressof(++i)} is equal to
  \tcode{addressof(i)}.
\end{itemize}

\pnum
\begin{note}
For \tcode{WeaklyIncrementable} types, \tcode{a} equals \tcode{b} does not imply that \tcode{++a}
equals \tcode{++b}. (Equality does not guarantee the substitution property or referential
transparency.) Algorithms on weakly incrementable types should never attempt to pass
through the same incrementable value twice. They should be single-pass algorithms. These algorithms
can be used with istreams as the source of the input data through the \tcode{istream_iterator} class
template.
\end{note}

\rSec3[iterator.concept.incrementable]{Concept \tcode{Incrementable}}

\pnum
The \tcode{Incrementable} concept specifies requirements on types that can be incremented with the pre-
and post-increment operators. The increment operations are required to be equality-preserving,
and the type is required to be \libconcept{EqualityComparable}.
\begin{note}
This requirement
supersedes the annotations on the increment expressions in the definition of
\tcode{WeaklyIncrementable}.
\end{note}

\indexlibrary{\idxcode{Incrementable}}%
\begin{codeblock}
template<class I>
  concept Incrementable =
    Regular<I> &&
    WeaklyIncrementable<I> &&
    requires(I i) {
      i++; requires Same<decltype(i++), I>;
    };
\end{codeblock}

\pnum
Let \tcode{a} and \tcode{b} be incrementable objects of type \tcode{I}.
\tcode{I} models \libconcept{Incrementable} only if

\begin{itemize}
\item If \tcode{bool(a == b)} then \tcode{bool(a++ == b)}.
\item If \tcode{bool(a == b)} then \tcode{bool(((void)a++, a) == ++b)}.
\end{itemize}

\pnum
\begin{note}
The requirement that \tcode{a} equals \tcode{b} implies \tcode{++a} equals \tcode{++b}
(which is not true for weakly incrementable types) allows the use of multi-pass one-directional
algorithms with types that satisfy \libconcept{Increment\-able}.
\end{note}

\rSec3[iterator.concept.iterator]{Concept \tcode{Iterator}}

\pnum
The \libconcept{Iterator} concept forms
the basis of the iterator concept taxonomy; every iterator satisfies the
\libconcept{Iterator} requirements. This
concept specifies operations for dereferencing and incrementing
an iterator. Most algorithms will require additional operations
to compare iterators with sentinels\iref{iterator.concept.sentinel}, to
read\iref{iterator.concept.input} or write\iref{iterator.concept.output} values, or
to provide a richer set of iterator movements (\ref{iterator.concept.forward},
\ref{iterator.concept.bidirectional}, \ref{iterator.concept.random.access}).)

\indexlibrary{\idxcode{Iterator}}%
\begin{codeblock}
template<class I>
  concept Iterator =
    requires(I i) {
      @\oldtxt{\{}@ *i @\oldtxt{\} -> auto\&\&; // Requires: i is dereferenceable}@
      @\newtxt{requires \placeholdernc{can-reference}<decltype(*i)>;}@
    } &&
    WeaklyIncrementable<I>;
\end{codeblock}

{\color{oldclr}
\pnum
\begin{note}
\oldtxt{The requirement that the result of dereferencing the iterator is deducible from
\tcode{auto\&\&} means that it cannot be \tcode{void}.}
\end{note}
} %% \color{oldclr}

\rSec3[iterator.concept.sentinel]{Concept \tcode{Sentinel}}

\pnum
The \libconcept{Sentinel} concept specifies the relationship
between an \libconcept{Iterator} type and a \libconcept{Semiregular} type
whose values denote a range.

\indexlibrary{\idxcode{Sentinel}}%
\begin{itemdecl}
template<class S, class I>
  concept Sentinel =
    Semiregular<S> &&
    Iterator<I> &&
    @\placeholder{weakly-equality-comparable-with}@<S, I>; // See \cxxref{concept.equalitycomparable}
\end{itemdecl}

\begin{itemdescr}
\pnum
Let \tcode{s} and \tcode{i} be values of type \tcode{S} and
\tcode{I} such that \range{i}{s} denotes a range. Types
\tcode{S} and \tcode{I} model \tcode{Sentinel<S, I>} only if

\begin{itemize}
\item \tcode{i == s} is well-defined.

\item If \tcode{bool(i != s)} then \tcode{i} is dereferenceable and
      \range{++i}{s} denotes a range.
\end{itemize}
\end{itemdescr}

\pnum
The domain of \tcode{==} \oldtxt{can change over time} \newtxt{is not static}.
Given an iterator \tcode{i} and sentinel \tcode{s} such that \range{i}{s}
denotes a range and \tcode{i != s}, \tcode{i} and \tcode{s} are not required to
continue to denote a range after incrementing any \newtxt{other} iterator equal
to \tcode{i}. Consequently, \tcode{i == s} is no longer required to be
well-defined.

\rSec3[iterator.concept.sizedsentinel]{Concept \tcode{SizedSentinel}}

\pnum
The \libconcept{SizedSentinel} concept specifies
requirements on an \libconcept{Iterator} and a \libconcept{Sentinel}
that allow the use of the \tcode{-} operator to compute the distance
between them in constant time.

\indexlibrary{\idxcode{SizedSentinel}}%

\begin{itemdecl}
template<class S, class I>
  concept SizedSentinel =
    Sentinel<S, I> &&
    !disable_sized_sentinel<remove_cv_t<S>, remove_cv_t<I>> &&
    requires(const I& i, const S& s) {
      s - i; requires Same<decltype(s - i), iter_difference_t<I>>;
      i - s; requires Same<decltype(i - s), iter_difference_t<I>>@\newtxt{;}@
    };
\end{itemdecl}

\begin{itemdescr}
\pnum
Let \tcode{i} be an iterator of type \tcode{I}, and \tcode{s}
a sentinel of type \tcode{S} such that \range{i}{s} denotes a range.
Let $N$ be the smallest number of applications of \tcode{++i}
necessary to make \tcode{bool(i == s)} be \tcode{true}.
\tcode{S} and \tcode{I} model \tcode{SizedSentinel<S, I>} only if

\begin{itemize}
\item If $N$ is representable by \tcode{iter_difference_t<I>},
      then \tcode{s - i} is well-defined and equals $N$.

\item If $-N$ is representable by \tcode{iter_difference_t<I>},
      then \tcode{i - s} is well-defined and equals $-N$.
\end{itemize}
\end{itemdescr}

\pnum
\begin{note}
\tcode{disable_sized_sentinel} \oldtxt{provides a mechanism to}
enable\newtxt{s} use of sentinels and iterators with the library that
\oldtxt{meet the syntactic requirements} \newtxt{satisfy} but do not in fact
\oldtxt{satisfy} \newtxt{model} \libconcept{SizedSentinel}.
\oldtxt{A program that instantiates a library template that requires
\libconcept{SizedSentinel} with an iterator type \tcode{I} and sentinel type
\tcode{S} that meet the syntactic requirements of \tcode{SizedSentinel<S, I>}
but do not model \libconcept{SizedSentinel} is
ill-formed with no diagnostic required\cxxiref{structure.requirements}.}
\end{note}

\pnum
\begin{example}
The \libconcept{SizedSentinel} concept is satisfied by pairs of
\libconcept{RandomAccessIterator}s\iref{iterator.concept.random.access} and by
counted iterators and their sentinels\iref{counted.iterator}.
\end{example}

\rSec3[iterator.concept.input]{Concept \tcode{InputIterator}}

\pnum
The \tcode{InputIterator} concept \oldtxt{is a refinement of
\tcode{Iterator}\iref{iterator.concept.iterator}. It} defines requirements for a type
whose referenced values can be read (from the requirement for
\tcode{Readable}\iref{iterator.concept.readable}) and which can be both pre- and
post-incremented.
\begin{note}
Unlike the \oldtxt{input iterator}
\newtxt{\oldconcept{InputIterator}} requirements\iref{input.iterators},
the \libconcept{InputIterator} concept \oldtxt{does} \newtxt{need} not require
equality comparison \newtxt{since iterators are typically compared to sentinels}.
\end{note}

\indexlibrary{\idxcode{InputIterator}}%
\begin{codeblock}
template<class I>
  concept InputIterator =
    Iterator<I> &&
    Readable<I> &&
    requires { typename @\placeholdernc{ITER_CONCEPT}@(I); } &&
    DerivedFrom<@\placeholdernc{ITER_CONCEPT}@(I), input_iterator_tag>;
\end{codeblock}

\rSec3[iterator.concept.output]{Concept \tcode{OutputIterator}}

\pnum
The \tcode{OutputIterator} concept \oldtxt{is a refinement of
\tcode{Iterator}\iref{iterator.concept.iterator}. It} defines requirements for a type that
can be used to write values (from the requirement for
\tcode{Writable}\iref{iterator.concept.writable}) and which can be both pre- and post-incremented.
\begin{note}
\oldtxt{However,} Output iterators are not required to \oldtxt{satisfy} \newtxt{model} \libconcept{EqualityComparable}.
\end{note}

\indexlibrary{\idxcode{OutputIterator}}%
\begin{codeblock}
template<class I, class T>
  concept OutputIterator =
    Iterator<I> &&
    Writable<I, T> &&
    requires(I i, T&& t) {
      *i++ = std::forward<T>(t); // not required to be equality-preserving
    };
\end{codeblock}

\pnum
Let \tcode{E} be an expression such that \tcode{decltype((E))} is \tcode{T}, and let \tcode{i} be a
dereferenceable object of type \tcode{I}. \tcode{I} and \tcode{T} model \tcode{OutputIterator<I, T>} only if
\tcode{*i++ = E;} has effects equivalent to:
\begin{codeblock}
  *i = E;
  ++i;
\end{codeblock}

\pnum
\begin{note}
Algorithms on output iterators should never attempt to pass through the same iterator twice.
They should be single-pass algorithms.
\end{note}

\rSec3[iterator.concept.forward]{Concept \tcode{ForwardIterator}}

\pnum
The \libconcept{ForwardIterator} concept \oldtxt{refines
\libconcept{InputIterator}\iref{iterator.concept.input},}
adds equality comparison and the multi-pass guarantee, specified below.

\indexlibrary{\idxcode{ForwardIterator}}%
\begin{codeblock}
template<class I>
  concept ForwardIterator =
    InputIterator<I> &&
    DerivedFrom<@\placeholdernc{ITER_CONCEPT}@(I), forward_iterator_tag> &&
    Incrementable<I> &&
    Sentinel<I, I>;
\end{codeblock}

\pnum
The domain of \tcode{==} for forward iterators is that of iterators over the same
underlying sequence. However, value-initialized iterators of the same type
may be compared and shall compare equal to other value-initialized iterators of the same type.
\begin{note}
Value-initialized iterators behave as if they refer past the end of the same
empty sequence.
\end{note}

\pnum
Pointers and references obtained from a forward iterator into a range \range{i}{s}
shall remain valid while \range{i}{s} continues to denote a range.

\pnum
Two dereferenceable iterators \tcode{a} and \tcode{b} of type \tcode{X}
offer the \defn{multi-pass guarantee} if:

\begin{itemize}
\item \tcode{a == b} implies \tcode{++a == ++b} and
\item The expression
\tcode{(\newtxt{(void)}[](X x)\{++x;\}(a), *a)} is equivalent to the expression \tcode{*a}.
\end{itemize}

\pnum
\begin{note}
The requirement that
\tcode{a == b}
implies
\tcode{++a == ++b}
\oldtxt{(which is not true for weaker iterators)}
and the removal of the restrictions on the number of assignments through
a mutable iterator
(which applies to output iterators)
allow the use of multi-pass one-directional algorithms with forward iterators.
\end{note}

\rSec3[iterator.concept.bidirectional]{Concept \libconcept{BidirectionalIterator}}

\pnum
The \libconcept{BidirectionalIterator} concept \oldtxt{refines \libconcept{ForwardIterator}\iref{iterator.concept.forward},
and} adds the ability to move an iterator backward as well as forward.

\indexlibrary{\idxcode{BidirectionalIterator}}%
\begin{codeblock}
template<class I>
  concept BidirectionalIterator =
    ForwardIterator<I> &&
    DerivedFrom<@\placeholdernc{ITER_CONCEPT}@(I), bidirectional_iterator_tag> &&
    requires(I i) {
      { --i } -> Same<I>&;
      i--; requires Same<decltype(i--), I>;
    };
\end{codeblock}

\pnum
A bidirectional iterator \tcode{r} is decrementable if and only if there exists some \tcode{s} such that
\tcode{++s == r}. Decrementable iterators \tcode{r} shall be in the domain of the expressions
\tcode{\dcr{}r} and \tcode{r\dcr{}}.

\pnum
Let \tcode{a} and \tcode{b} be decrementable objects of type \tcode{I}.
\tcode{I} models \libconcept{BidirectionalIterator} only if:

\begin{itemize}
\item \tcode{addressof(\dcr{}a) == addressof(a)}.
\item If \tcode{bool(a == b)}, then \tcode{bool(a\dcr{} == b)}.
\item If \tcode{bool(a == b)}, then after evaluating both \tcode{a\dcr} and \tcode{\dcr{}b},
\tcode{bool(a == b)} still holds.
\item If \tcode{a} is incrementable and \tcode{bool(a == b)}, then
      \tcode{bool(\dcr{}(++a) == b)}.
\item If \tcode{bool(a == b)}, then \tcode{bool(++(\dcr{}a) == b)}.
\end{itemize}

\rSec3[iterator.concept.random.access]{Concept \libconcept{RandomAccessIterator}}

\pnum
The \libconcept{RandomAccessIterator} concept \oldtxt{refines
\libconcept{BidirectionalIterator}\iref{iterator.concept.bidirectional}
and} adds support for constant-time advancement with
\tcode{+=}, \tcode{+}, \tcode{-=}, and \tcode{-}, \oldtxt{and} \newtxt{as well as} the computation
of distance in constant time with \tcode{-}. Random access iterators
also support array notation via subscripting.

\indexlibrary{\idxcode{RandomAccessIterator}}%
\begin{codeblock}
template<class I>
  concept RandomAccessIterator =
    BidirectionalIterator<I> &&
    DerivedFrom<@\placeholdernc{ITER_CONCEPT}@(I), random_access_iterator_tag> &&
    StrictTotallyOrdered<I> &&
    SizedSentinel<I, I> &&
    requires(I i, const I j, const iter_difference_t<I> n) {
      { i += n } -> Same<I>&;
      j + n; requires Same<decltype(j + n), I>;
      n + j; requires Same<decltype(n + j), I>;
      { i -= n } -> Same<I>&;
      j - n; requires Same<decltype(j - n), I>;
      j[n]; requires Same<decltype(j[n]), iter_reference_t<I>>;
    };
\end{codeblock}

\pnum
Let \tcode{a} and \tcode{b} be valid iterators of type \tcode{I} such that
\tcode{b} is reachable from \tcode{a}. Let \tcode{n} be the smallest value
of type \tcode{iter_difference_t<I>} such that after
\tcode{n} applications of \tcode{++a}, then \tcode{bool(a == b)}.
\tcode{I} models \libconcept{RandomAccessIterator} only if

\begin{itemize}
\item \tcode{(a += n)} is equal to \tcode{b}.
\item \tcode{addressof((a += n))} is equal to \tcode{addressof(a)}.
\item \tcode{(a + n)} is equal to \tcode{(a += n)}.
\item For any two positive integers \tcode{x} and \tcode{y}, if \tcode{a + (x + y)} is valid, then
\tcode{a + (x + y)} is equal to \tcode{(a + x) + y}.
\item \tcode{a + 0} is equal to \tcode{a}.
\item If \tcode{(a + (n - 1))} is valid, then \tcode{a + n} is equal to \tcode{++(a + (n - 1))}.
\item \tcode{(b += -n)} is equal to \tcode{a}.
\item \tcode{(b -= n)} is equal to \tcode{a}.
\item \tcode{addressof((b -= n))} is equal to \tcode{addressof(b)}.
\item \tcode{(b - n)} is equal to \tcode{(b -= n)}.
\item If \tcode{b} is dereferenceable, then \tcode{a[n]} is valid and is equal to \tcode{*b}.
\item \tcode{bool(a <= b)} is \tcode{true}.
\end{itemize}

\rSec3[iterator.concept.contiguous]{Concept \libconcept{ContiguousIterator}}

\pnum
The \libconcept{ContiguousIterator} concept \oldtxt{refines \libconcept{RandomAccessIterator} and}
provides a guarantee that the denoted elements are stored contiguously in memory.

\indexlibrary{\idxcode{ContiguousIterator}}%
\begin{codeblock}
template<class I>
  concept @\libconcept{ContiguousIterator}@ =
    RandomAccessIterator<I> &&
    DerivedFrom<@\placeholdernc{ITER_CONCEPT}@(I), contiguous_iterator_tag> &&
    is_lvalue_reference_v<iter_reference_t<I>> &&
    Same<iter_value_t<I>, remove_cvref_t<iter_reference_t<I>>>;
\end{codeblock}

\pnum
Let \tcode{a} and \tcode{b} be dereferenceable iterators of type \tcode{I} such
that \tcode{b} is reachable from \tcode{a}. \tcode{I} models
\libconcept{ContiguousIterator} only if
\tcode{addressof(*(a + (b - a))) \oldtxt{==}} \newtxt{is equal to} \tcode{addressof(*a) + (b - a)} \oldtxt{is \tcode{true}}.

\rSec2[iterator.cpp17]{\Cpp{}17 iterator requirements}

\pnum
In the following sections,
\tcode{a}
and
\tcode{b}
denote values of type
\tcode{X} or \tcode{const X},
\tcode{difference_type} and \tcode{reference} refer to the
types \tcode{iterator_traits<X>::difference_type} and
\tcode{iterator_traits<X>::reference}, respectively,
\tcode{n}
denotes a value of
\tcode{difference_type},
\tcode{u},
\tcode{tmp},
and
\tcode{m}
denote identifiers,
\tcode{r}
denotes a value of
\tcode{X\&},
\tcode{t}
denotes a value of value type
\tcode{T},
\tcode{o}
denotes a value of some type that is writable to the output iterator.
\begin{note} For an iterator type \tcode{X} there must be an instantiation
of \tcode{iterator_traits<X>}\iref{iterator.traits}. \end{note}
\end{addedblock}

\ednote{Relocate [iterator.iterators] here:}
\rSec3[iterator.iterators]{\oldconcept{Iterator}}
[...]

\ednote{Relocate [input.iterators] here:}
\rSec3[input.iterators]{Input iterators}
[...]

\ednote{Relocate [output.iterators] here:}
\rSec3[output.iterators]{Output iterators}
[...]

\ednote{Relocate [forward.iterators] here:}
\rSec3[forward.iterators]{Forward iterators}
[...]

\ednote{Relocate [bidirectional.iterators] here:}
\rSec3[bidirectional.iterators]{Bidirectional iterators}
[...]

\ednote{Relocate [random.access.iterators] here:}
\rSec3[random.access.iterators]{Random access iterators}
[...]

\begin{addedblock}
\rSec2[indirectcallable]{Indirect callable requirements}

\rSec3[indirectcallable.general]{General}

\pnum
There are several concepts that group requirements of algorithms that
take callable objects~(\cxxref{func.require}) as arguments.

\rSec3[indirectcallable.indirectinvocable]{Indirect callables}

\pnum
The indirect callable concepts are used to constrain those algorithms
that accept callable objects~(\cxxref{func.def}) as arguments.

\indexlibrary{\idxcode{IndirectUnaryInvocable}}%
\indexlibrary{\idxcode{IndirectRegularUnaryInvocable}}%
\indexlibrary{\idxcode{IndirectUnaryPredicate}}%
\indexlibrary{\idxcode{IndirectRelation}}%
\indexlibrary{\idxcode{IndirectStrictWeakOrder}}%
\begin{codeblock}
namespace std {
  template<class F, class I>
    concept IndirectUnaryInvocable =
      Readable<I> &&
      CopyConstructible<F> &&
      Invocable<F&, iter_value_t<I>&> &&
      Invocable<F&, iter_reference_t<I>> &&
      Invocable<F&, iter_common_reference_t<I>> &&
      CommonReference<
        invoke_result_t<F&, iter_value_t<I>&>,
        invoke_result_t<F&, iter_reference_t<I>>>;

  template<class F, class I>
    concept IndirectRegularUnaryInvocable =
      Readable<I> &&
      CopyConstructible<F> &&
      RegularInvocable<F&, iter_value_t<I>&> &&
      RegularInvocable<F&, iter_reference_t<I>> &&
      RegularInvocable<F&, iter_common_reference_t<I>> &&
      CommonReference<
        invoke_result_t<F&, iter_value_t<I>&>,
        invoke_result_t<F&, iter_reference_t<I>>>;

  template<class F, class I>
    concept IndirectUnaryPredicate =
      Readable<I> &&
      CopyConstructible<F> &&
      Predicate<F&, iter_value_t<I>&> &&
      Predicate<F&, iter_reference_t<I>> &&
      Predicate<F&, iter_common_reference_t<I>>;

  template<class F, class I1, class I2 = I1>
    concept IndirectRelation =
      Readable<I1> && Readable<I2> &&
      CopyConstructible<F> &&
      Relation<F&, iter_value_t<I1>&, iter_value_t<I2>&> &&
      Relation<F&, iter_value_t<I1>&, iter_reference_t<I2>> &&
      Relation<F&, iter_reference_t<I1>, iter_value_t<I2>&> &&
      Relation<F&, iter_reference_t<I1>, iter_reference_t<I2>> &&
      Relation<F&, iter_common_reference_t<I1>, iter_common_reference_t<I2>>;

  template<class F, class I1, class I2 = I1>
    concept IndirectStrictWeakOrder =
      Readable<I1> && Readable<I2> &&
      CopyConstructible<F> &&
      StrictWeakOrder<F&, iter_value_t<I1>&, iter_value_t<I2>&> &&
      StrictWeakOrder<F&, iter_value_t<I1>&, iter_reference_t<I2>> &&
      StrictWeakOrder<F&, iter_reference_t<I1>, iter_value_t<I2>&> &&
      StrictWeakOrder<F&, iter_reference_t<I1>, iter_reference_t<I2>> &&
      StrictWeakOrder<F&, iter_common_reference_t<I1>, iter_common_reference_t<I2>>;
}
\end{codeblock}

\rSec3[projected]{Class template \tcode{projected}}

\pnum
Class template \tcode{projected} is
\oldtxt{intended for use when specifying the constraints of}
\newtxt{used to constrain} algorithms that accept callable objects
and projections\iref{defns.projection}. It bundles a
\libconcept{Readable} type \tcode{I} and
a function \tcode{Proj} into a new \libconcept{Readable} type
whose \tcode{reference} type is the result of applying
\tcode{Proj} to the \tcode{iter_reference_t} of \tcode{I}.

\indexlibrary{\idxcode{projected}}%
\begin{codeblock}
namespace std {
  template<Readable I, IndirectRegularUnaryInvocable<I> Proj>
  struct projected {
    using value_type = remove_cvref_t<indirect_result_t<Proj&, I>>;
    indirect_result_t<Proj&, I> operator*() const; @\newtxt{\notdef}@
  };

  template<WeaklyIncrementable I, class Proj>
  struct incrementable_traits<projected<I, Proj>> {
    using difference_type = iter_difference_t<I>;
  };
}
\end{codeblock}

{\color{oldclr}
\pnum
\begin{note}
\tcode{projected} is only used to ease constraints specification. Its
member function need not be defined.
\end{note}
} %% \color{oldclr}


\rSec2[commonalgoreq]{Common algorithm requirements}
\rSec3[commonalgoreq.general]{General}

\pnum
There are several additional iterator concepts that are commonly applied
to families of algorithms. These group together iterator requirements
of algorithm families.
There are three relational concepts that specify
how element values are transferred  between \libconcept{Readable} and
\libconcept{Writable} types:
\libconcept{Indirectly\-Movable},
\libconcept{Indir\-ect\-ly\-Copy\-able}, and
\libconcept{Indirectly\-Swappable}.
There are three relational concepts for rearrangements:
\libconcept{Permut\-able},
\libconcept{Mergeable}, and
\libconcept{Sortable}.
There is one relational concept for comparing values from different sequences:
\libconcept{IndirectlyComparable}.

\pnum
\begin{note}
The \tcode{ranges::less<>}\iref{range.comparisons} function object type used
in the concepts below imposes constraints on \oldtxt{their} \newtxt{the concepts'} arguments in addition to
those that appear \oldtxt{explicitly} in the concepts' bodies;
the function call operator of \tcode{ranges::less<>} requires
its arguments to model
\libconcept{StrictTotally\-OrderedWith}~(\cxxref{concept.stricttotallyordered}).
\end{note}

\rSec3[commonalgoreq.indirectlymovable]{Concept \libconcept{IndirectlyMovable}}

\pnum
The \libconcept{IndirectlyMovable} concept specifies the relationship between
a \libconcept{Readable} type and a \libconcept{Writable} type between which
values may be moved.

\indexlibrary{\idxcode{IndirectlyMovable}}%
\begin{codeblock}
template<class In, class Out>
  concept IndirectlyMovable =
    Readable<In> &&
    Writable<Out, iter_rvalue_reference_t<In>>;
\end{codeblock}

\pnum
The \libconcept{IndirectlyMovableStorable} concept augments
\libconcept{IndirectlyMovable} with additional requirements enabling
the transfer to be performed through an intermediate object of the
\libconcept{Readable} type's value type.

\indexlibrary{\idxcode{IndirectlyMovableStorable}}%
\begin{codeblock}
template<class In, class Out>
  concept IndirectlyMovableStorable =
    IndirectlyMovable<In, Out> &&
    Writable<Out, iter_value_t<In>> &&
    Movable<iter_value_t<In>> &&
    Constructible<iter_value_t<In>, iter_rvalue_reference_t<In>> &&
    Assignable<iter_value_t<In>&, iter_rvalue_reference_t<In>>;
\end{codeblock}

\pnum
Let \tcode{i} be a dereferenceable value of type \tcode{In}.
\tcode{In} and \tcode{Out} model \tcode{IndirectlyMovableStorable<In, Out>}
only if after the initialization of the object \tcode{obj} in
\begin{codeblock}
iter_value_t<In> obj(ranges::iter_move(i));
\end{codeblock}
\tcode{obj} is equal to the value previously denoted by \tcode{*i}. If
\tcode{iter_rvalue_reference_t<In>} is an rvalue reference type,
the resulting state of the value denoted by \tcode{*i} is
valid but unspecified\cxxiref{lib.types.movedfrom}.

\rSec3[commonalgoreq.indirectlycopyable]{Concept \libconcept{IndirectlyCopyable}}

\pnum
The \libconcept{IndirectlyCopyable} concept specifies the relationship between
a \libconcept{Readable} type and a \libconcept{Writable} type between which
values may be copied.

\indexlibrary{\idxcode{IndirectlyCopyable}}%
\begin{codeblock}
template<class In, class Out>
  concept IndirectlyCopyable =
    Readable<In> &&
    Writable<Out, iter_reference_t<In>>;
\end{codeblock}

\pnum
The \libconcept{IndirectlyCopyableStorable} concept augments
\libconcept{IndirectlyCopyable} with additional requirements enabling
the transfer to be performed through an intermediate object of the
\libconcept{Readable} type's value type. It also requires the capability
to make copies of values.

\indexlibrary{\idxcode{IndirectlyCopyableStorable}}%
\begin{codeblock}
template<class In, class Out>
  concept IndirectlyCopyableStorable =
    IndirectlyCopyable<In, Out> &&
    Writable<Out, const iter_value_t<In>&> &&
    Copyable<iter_value_t<In>> &&
    Constructible<iter_value_t<In>, iter_reference_t<In>> &&
    Assignable<iter_value_t<In>&, iter_reference_t<In>>;
\end{codeblock}

\pnum
Let \tcode{i} be a dereferenceable value of type \tcode{In}.
\tcode{In} and \tcode{Out} model \tcode{IndirectlyCopyableStorable<In, Out>}
only if after the initialization of the object \tcode{obj} in
\begin{codeblock}
iter_value_t<In> obj(*i);
\end{codeblock}
\tcode{obj} is equal to the value previously denoted by \tcode{*i}. If
\tcode{iter_reference_t<In>} is an rvalue reference type, the resulting state
of the value denoted by \tcode{*i} is
valid but unspecified\cxxiref{lib.types.movedfrom}.

\rSec3[commonalgoreq.indirectlyswappable]{Concept \libconcept{IndirectlySwappable}}

\pnum
The \libconcept{IndirectlySwappable} concept specifies a swappable relationship
between the values referenced by two \libconcept{Readable} types.

\indexlibrary{\idxcode{IndirectlySwappable}}%
\begin{codeblock}
template<class I1, class I2 = I1>
  concept IndirectlySwappable =
    Readable<I1> && Readable<I2> &&
    requires(I1&@\oldtxt{\&}@ i1, I2&@\oldtxt{\&}@ i2) {
      ranges::iter_swap(@\oldtxt{std::forward<I1>(}@i1@\oldtxt{)}@, @\oldtxt{std::forward<I2>(}@i2@\oldtxt{)}@);
      ranges::iter_swap(@\oldtxt{std::forward<I2>(}@i2@\oldtxt{)}@, @\oldtxt{std::forward<I1>(}@i1@\oldtxt{)}@);
      ranges::iter_swap(@\oldtxt{std::forward<I1>(}@i1@\oldtxt{)}@, @\oldtxt{std::forward<I1>(}@i1@\oldtxt{)}@);
      ranges::iter_swap(@\oldtxt{std::forward<I2>(}@i2@\oldtxt{)}@, @\oldtxt{std::forward<I2>(}@i2@\oldtxt{)}@);
    };
\end{codeblock}

\pnum
Given an object \tcode{i1} of type \tcode{I1} and
an object \tcode{i2} of type \tcode{I2},
\tcode{I1} and \tcode{I2} model \tcode{IndirectlySwappable<I1, I2>} only if
after \newtxt{either} \tcode{ranges::iter_swap(i1, i2)} \newtxt{or
}\tcode{\newtxt{ranges::iter_swap(i2, i1)}}, the value of \tcode{*i1}
is equal to the value of \tcode{*i2} before the call, and \textit{vice versa}.

\rSec3[commonalgoreq.indirectlycomparable]{Concept \libconcept{IndirectlyComparable}}

\pnum
The \libconcept{IndirectlyComparable} concept specifies
the common requirements of algorithms that
compare values from two different sequences.

\indexlibrary{\idxcode{IndirectlyComparable}}%
\begin{codeblock}
template<class I1, class I2, class R, class P1 = identity,
         class P2 = identity>
  concept IndirectlyComparable =
    IndirectRelation<R, projected<I1, P1>, projected<I2, P2>>;
\end{codeblock}

\rSec3[commonalgoreq.permutable]{Concept \libconcept{Permutable}}

\pnum
The \libconcept{Permutable} concept specifies the common requirements
of algorithms that reorder elements in place by moving or swapping them.

\indexlibrary{\idxcode{Permutable}}%
\begin{codeblock}
template<class I>
  concept Permutable =
    ForwardIterator<I> &&
    IndirectlyMovableStorable<I, I> &&
    IndirectlySwappable<I, I>;
\end{codeblock}

\rSec3[commonalgoreq.mergeable]{Concept \libconcept{Mergeable}}

\pnum
The \libconcept{Mergeable} concept specifies the requirements of algorithms
that merge sorted sequences into an output sequence by copying elements.

\indexlibrary{\idxcode{Mergeable}}%
\begin{codeblock}
template<class I1, class I2, class Out, class R = ranges::less<>,
         class P1 = identity, class P2 = identity>
  concept Mergeable =
    InputIterator<I1> &&
    InputIterator<I2> &&
    WeaklyIncrementable<Out> &&
    IndirectlyCopyable<I1, Out> &&
    IndirectlyCopyable<I2, Out> &&
    IndirectStrictWeakOrder<R, projected<I1, P1>, projected<I2, P2>>;
\end{codeblock}

\rSec3[commonalgoreq.sortable]{Concept \libconcept{Sortable}}

\pnum
The \libconcept{Sortable} concept specifies the common requirements of
algorithms that permute sequences into ordered sequences (e.g., \tcode{sort}).

\indexlibrary{\idxcode{Sortable}}%
\begin{codeblock}
template<class I, class R = ranges::less<>, class P = identity>
  concept Sortable =
    Permutable<I> &&
    IndirectStrictWeakOrder<R, projected<I, P>>;
\end{codeblock}
\end{addedblock}

\rSec1[iterator.primitives]{Iterator primitives}

\pnum
To simplify the task of defining iterators, the library provides
several classes and functions:

\rSec2[std.iterator.tags]{Standard iterator tags}

\pnum
\indexlibrary{\idxcode{output_iterator_tag}}%
\indexlibrary{\idxcode{input_iterator_tag}}%
\indexlibrary{\idxcode{forward_iterator_tag}}%
\indexlibrary{\idxcode{bidirectional_iterator_tag}}%
\indexlibrary{\idxcode{random_access_iterator_tag}}%
\indexlibrary{\idxcode{contiguous_iterator_tag}}%
It is often desirable for a
function template specialization
to find out what is the most specific category of its iterator
argument, so that the function can select the most efficient algorithm at compile time.
To facilitate this, the
library introduces
\term{category tag}
classes which are used as compile time tags for algorithm selection.
They are:
\added{\tcode{output_iterator_tag},}
\tcode{input_iterator_tag},
\removed{\tcode{output_iterator_tag},}
\tcode{forward_iterator_tag},
\tcode{bidirectional_iterator_tag}\added{,}
\removed{and}
\tcode{random_access_iterator_tag}\added{,}
\added{and}
\tcode{\added{contiguous_iterator_tag}}.
For every iterator of type
\tcode{\changed{Iterator}{I}},
\tcode{iterator_traits<\changed{Iterator}{I}>::it\-er\-a\-tor_ca\-te\-go\-ry}
shall be defined to be the most specific category tag that describes the
iterator's behavior. \added{Additionally \oldtxt{and optionally},
}\tcode{\added{iterator_traits<I>::it\-er\-a\-tor_con\-cept}}\added{
may be used to opt in or out of conformance to the iterator concepts \oldtxt{defined
in}\iref{iterator.concepts}.}

\begin{codeblock}
namespace std {
  @\added{struct output_iterator_tag \{ \};}@
  struct input_iterator_tag { };
  @\removed{struct output_iterator_tag \{ \};}@
  struct forward_iterator_tag: @\removed{public}@ input_iterator_tag { };
  struct bidirectional_iterator_tag: @\removed{public}@ forward_iterator_tag { };
  struct random_access_iterator_tag: @\removed{public}@ bidirectional_iterator_tag { };
  @\added{struct contiguous_iterator_tag:}\added{ }\added{random_access_iterator_tag \{ \};}@
}
\end{codeblock}

[...]

\rSec2[iterator.operations]{Iterator operations}

[...]

\begin{addedblock}
\rSec2[range.iterator.operations]{Range iterator operations}

\pnum
{\color{oldclr}
Since only types that model
\libconcept{RandomAccessIterator} provide the \tcode{+} operator, and
types that model \libconcept{Sized\-Sent\-inel} provide the \tcode{-}
operator, the library provides function templates
\tcode{advance}, \tcode{dist\-ance}, \tcode{next}, and \tcode{prev}.
These function templates use
\tcode{+}
and
\tcode{-}
for random access iterators and ranges that model \libconcept{SizedSentinel}
(and are, therefore, constant time for them); for output, input, forward and
bidirectional iterators they use
\tcode{++}
to provide linear time implementations.
} %% \color{oldclr}

{\color{newclr}
\pnum
The library provides the abstract operations
\tcode{advance}, \tcode{distance}, \tcode{next}, and \tcode{prev}
to manipulate iterators. These operations adapt to the set of operators provided
by each iterator category to provide the most efficient implementation
possible for a concrete iterator type.
\begin{example}
\tcode{ranges::advance} uses the \tcode{+} operator to move a
\libconcept{RandomAccessIterator} forward \tcode{n} steps in constant time.
For an iterator type that does not model \libconcept{RandomAccessIterator},
\tcode{ranges::advance} instead performs \tcode{n} individual increments with
the \tcode{++} operator.
\end{example}
} %% \color{newclr}

\pnum
The function templates defined in this subclause are not found by
argument-dependent name lookup\cxxiref{basic.lookup.argdep}. When found by
unqualified\cxxiref{basic.lookup.unqual} name lookup for the
\grammarterm{postfix-expression} in a function call\cxxiref{expr.call}, they
inhibit argument-dependent name lookup.

\begin{example}
\begin{codeblock}
void foo() {
    using namespace std::ranges;
    std::vector<int> vec{1,2,3};
    distance(begin(vec), end(vec)); // \#1
}
\end{codeblock}
The function call expression at \tcode{\#1} invokes \tcode{std::ranges::distance},
not \tcode{std::distance}, despite that
(a) the iterator type returned from \tcode{begin(vec)} and \tcode{end(vec)}
may be associated with namespace \tcode{std} and
(b) \tcode{std::distance} is more specialized~(\cxxref{temp.func.order}) than
\tcode{std::ranges::distance} since the former requires its first two parameters
to have the same type.
\end{example}

{\color{newclr}
\pnum
The number and order of template parameters for the function templates defined
in this subclause is unspecified, except where explicitly stated otherwise.
} %% \color{newclr}

\rSec3[range.iterator.operations.advance]{\tcode{ranges::advance}}
\indexlibrary{\idxcode{advance}}%
\begin{itemdecl}
template<Iterator I>
  constexpr void advance(I& i, iter_difference_t<I> n);
\end{itemdecl}

\begin{itemdescr}
\pnum
\expects
\tcode{n} shall be negative only \oldtxt{for bidirectional iterators}
\newtxt{if \tcode{I} models \libconcept{BidirectionalIterator}}.

\pnum
\effects
\begin{itemize}
\item \oldtxt{For random access iterators}
  \newtxt{If \tcode{I} models \libconcept{RandomAccessIterator}},
  equivalent to \tcode{i += n}.
\item Otherwise, \newtxt{if \tcode{n} is non-negative,} increments
  \oldtxt{(or decrements for negative \tcode{n}) iterator}
  \tcode{i} by \tcode{n}.
\item \newtxt{Otherwise, decrements \tcode{i} by \tcode{-n}.}
\end{itemize}
\end{itemdescr}

\indexlibrary{\idxcode{advance}}%
\begin{itemdecl}
template<Iterator I, Sentinel<I> S>
  constexpr void advance(I& i, S bound);
\end{itemdecl}

\begin{itemdescr}
\pnum
\expects
\oldtxt{If \tcode{Assignable<I\&, S>} is not satisfied,} \range{i}{bound}
shall denote a range.

\pnum
\effects
\begin{itemize}
\item If \tcode{I} and \tcode{S} model \tcode{Assignable<I\&, S>},
  equivalent to \tcode{i = std::move(bound)}.
\item Otherwise, if \tcode{S} and \tcode{I} model \tcode{SizedSentinel<S, I>},
  equivalent to \tcode{ranges::advance(i, bound - i)}.
\item Otherwise, increments \tcode{i} until
  \tcode{\newtxt{bool(}i == bound\newtxt{)}} \newtxt{is \tcode{true}}.
\end{itemize}
\end{itemdescr}

\indexlibrary{\idxcode{advance}}%
\begin{itemdecl}
template<Iterator I, Sentinel<I> S>
  constexpr iter_difference_t<I> advance(I& i, iter_difference_t<I> n, S bound);
\end{itemdecl}

\begin{itemdescr}
\pnum
\expects
If \tcode{n > 0}, \range{i}{bound} shall denote a range.
If \tcode{n == 0}, \range{i}{bound} or \range{bound}{i} shall denote a range.
If \tcode{n < 0}, \range{bound}{i} shall denote a range,
\tcode{I} shall model \libconcept{BidirectionalIterator}, and
\tcode{I} and \tcode{S} shall model \tcode{Same<I, S>}.

\pnum
\effects
\begin{itemize}
\item If \tcode{S} and \tcode{I} model \tcode{SizedSentinel<S, I>}:
  \begin{itemize}
  \item If \brk{}$|\tcode{n}| >= |\tcode{bound - i}|$,
    equivalent to \tcode{ranges::advance(i, bound)}.
  \item Otherwise, equivalent to \tcode{ranges::advance(i, n)}.
  \end{itemize}
\item Otherwise,
  \begin{itemize}
  \item \newtxt{if \tcode{n} is non-negative,} increments
    \oldtxt{(or decrements for negative \tcode{n}) iterator}
    \tcode{i} either \tcode{n} times or until
    \tcode{\newtxt{bool(}i == bound\newtxt{)}} \newtxt{is \tcode{true}},
    whichever comes first.
  \item \newtxt{Otherwise, decrements \tcode{i} by \tcode{-n} or until
    \tcode{bool(i == bound)} is \tcode{true}.}
  \end{itemize}
\end{itemize}

\pnum
\returns
\tcode{n - $M$}, where $M$ is the
\oldtxt{distance from the starting position of \tcode{i} to the ending position}
\newtxt{difference between the ending and starting positions of \tcode{i}}.
\end{itemdescr}

\rSec3[range.iterator.operations.distance]{\tcode{ranges::distance}}
\indexlibrary{\idxcode{distance}}%
\begin{itemdecl}
template<Iterator I, Sentinel<I> S>
  constexpr iter_difference_t<I> distance(I first, S last);
\end{itemdecl}

\begin{itemdescr}
\pnum
\expects
\range{first}{last} shall denote a range, or
\newtxt{\range{last}{first} shall denote a range and}
\tcode{S} and \tcode{I} shall model
\tcode{Same<S, I> \&\& SizedSentinel<S, I>}
\oldtxt{and \range{last}{first} shall denote a range}.

\pnum
\effects
If \tcode{S} and \tcode{I} model \tcode{SizedSentinel<S, I>},
returns \tcode{(last - first)};
otherwise, returns the number of increments needed to get from
\tcode{first}
to
\tcode{last}.
\end{itemdescr}

\indexlibrary{\idxcode{distance}}%
\begin{itemdecl}
template<Range R>
  constexpr iter_difference_t<iterator_t<R>> distance(R&& r);
\end{itemdecl}

\begin{itemdescr}
\pnum
\effects
If \tcode{R} models \libconcept{SizedRange}, equivalent to:
\begin{codeblock}
return ranges::size(r); // \ref{range.primitives.size}
\end{codeblock}
Otherwise, equivalent to:
\begin{codeblock}
return ranges::distance(ranges::begin(r), ranges::end(r)); // \ref{range.access}
\end{codeblock}
\end{itemdescr}

\rSec3[range.iterator.operations.next]{\tcode{ranges::next}}
\indexlibrary{\idxcode{next}}%
\begin{itemdecl}
template<Iterator I>
  constexpr I next(I x);
\end{itemdecl}

\begin{itemdescr}
\pnum
\effects Equivalent to: \tcode{++x; return x;}
\end{itemdescr}

\indexlibrary{\idxcode{next}}%
\begin{itemdecl}
template<Iterator I>
  constexpr I next(I x, iter_difference_t<I> n);
\end{itemdecl}

\begin{itemdescr}
\pnum
\effects Equivalent to: \tcode{ranges::advance(x, n); return x;}
\end{itemdescr}

\indexlibrary{\idxcode{next}}%
\begin{itemdecl}
template<Iterator I, Sentinel<I> S>
  constexpr I next(I x, S bound);
\end{itemdecl}

\begin{itemdescr}
\pnum
\effects Equivalent to: \tcode{ranges::advance(x, bound); return x;}
\end{itemdescr}

\indexlibrary{\idxcode{next}}%
\begin{itemdecl}
template<Iterator I, Sentinel<I> S>
  constexpr I next(I x, iter_difference_t<I> n, S bound);
\end{itemdecl}

\begin{itemdescr}
\pnum
\effects Equivalent to: \tcode{ranges::advance(x, n, bound); return x;}
\end{itemdescr}

\rSec3[range.iterator.operations.prev]{\tcode{ranges::prev}}
\indexlibrary{\idxcode{prev}}%
\begin{itemdecl}
template<BidirectionalIterator I>
  constexpr I prev(I x);
\end{itemdecl}

\begin{itemdescr}
\pnum
\effects Equivalent to: \tcode{-{-}x; return x;}
\end{itemdescr}

\indexlibrary{\idxcode{prev}}%
\begin{itemdecl}
template<BidirectionalIterator I>
  constexpr I prev(I x, iter_difference_t<I> n);
\end{itemdecl}

\begin{itemdescr}
\pnum
\effects Equivalent to: \tcode{ranges::advance(x, -n); return x;}
\end{itemdescr}

\indexlibrary{\idxcode{prev}}%
\begin{itemdecl}
template<BidirectionalIterator I>
  constexpr I prev(I x, iter_difference_t<I> n, I bound);
\end{itemdecl}

\begin{itemdescr}
\pnum
\effects Equivalent to: \tcode{ranges::advance(x, -n, bound); return x;}
\end{itemdescr}
\end{addedblock}

\rSec1[predef.iterators]{Iterator adaptors}

\rSec2[reverse.iterators]{Reverse iterators}

\pnum
Class template \tcode{reverse_iterator} is an iterator adaptor that iterates
from the end of the sequence defined by its underlying iterator to the beginning
of that sequence.
\removed{The fundamental relation between a reverse iterator
and its corresponding iterator \tcode{i} is established by the identity:
\tcode{\&*(reverse_iterator(i)) == \&*(i - 1)}.}

\rSec3[reverse.iterator]{Class template \tcode{reverse_iterator}}

\indexlibrary{\idxcode{reverse_iterator}}%
\begin{codeblock}
namespace std {
  template<class Iterator>
  class reverse_iterator {
  public:
    using iterator_type     = Iterator;
    @\added{using iterator_concept}@  @\added{= \seebelownc;}@
    @\removed{using iterator_category}@ @\removed{= typename iterator_traits<Iterator>::iterator_category;}@
    @\added{using iterator_category}@ @\added{= \seebelownc;}@
    @\removed{using value_type}@        @\removed{= typename iterator_traits<Iterator>::value_type;}@
    @\added{using value_type}@        @\added{= iter_value_t<Iterator>;}@
    @\removed{using difference_type}@   @\removed{= typename iterator_traits<Iterator>::difference_type;}@
    @\added{using difference_type}@   @\added{= iter_difference_t<Iterator>;}@
    using pointer           = typename iterator_traits<Iterator>::pointer;
    @\removed{using reference}@         @\removed{= typename iterator_traits<Iterator>::reference;}@
    @\added{using reference}@         @\added{= iter_reference_t<Iterator>;}@

    constexpr reverse_iterator();
    constexpr explicit reverse_iterator(Iterator x);
    template<class U> constexpr reverse_iterator(const reverse_iterator<U>& u);
    template<class U> constexpr reverse_iterator& operator=(const reverse_iterator<U>& u);

    constexpr Iterator base() const;      @\removed{// explicit}@
    constexpr reference operator*() const;
    constexpr pointer   operator->() const @\added{requires \seebelownc}@;

    [...]

    constexpr reverse_iterator& operator-=(difference_type n);
    constexpr @\unspec@ operator[](difference_type n) const;

    @\added{friend constexpr iter_rvalue_reference_t<Iterator> iter_move(const reverse_iterator\& i)}@
      @\added{noexcept(\seebelownc);}@
    @\added{template<IndirectlySwappable<Iterator> Iterator2>}@
      @\added{friend constexpr void iter_swap(const reverse_iterator\& x,}@
                                      @\added{const reverse_iterator<Iterator2>\& y)}@
        @\added{noexcept(\seebelownc);}@

  protected:
    Iterator current;
  };

  [...]

  template<class Iterator>
    constexpr reverse_iterator<Iterator> make_reverse_iterator(Iterator i);

  @\added{template<class Iterator1, class Iterator2>}@
    @\added{requires \newtxt{(}!SizedSentinel<Iterator1, Iterator2>\newtxt{)}}@
  @\added{inline constexpr bool disable_sized_sentinel<reverse_iterator<Iterator1>,}@
                                               @\added{reverse_iterator<Iterator2>{>} = true;}@
}
\end{codeblock}

\begin{addedblock}
\pnum
The member \grammarterm{typedef-name} \tcode{iterator_concept} denotes
\tcode{random_access_iterator_tag} if \tcode{Iterator} models
\libconcept{RandomAccessIterator}, and
\tcode{bidirectional_iterator_tag} otherwise.

\pnum
The member \grammarterm{typedef-name} \tcode{iterator_category} denotes
\tcode{random_access_iterator_tag} if
\tcode{iterator_traits<\brk{}Iterator>::iterator_category} is derived from
\tcode{random_access_iterator_tag}, and
\tcode{iterator_traits<\brk{}Iterator>::iterator_category} otherwise.
\end{addedblock}

\rSec3[reverse.iter.requirements]{\tcode{reverse_iterator} requirements}

\pnum
The template parameter
\tcode{Iterator}
shall \changed{satisfy all}{either meet} the requirements of a
\oldconcept{BidirectionalIterator}\iref{bidirectional.iterators}
\added{or model
\libconcept{BidirectionalIterator}\iref{iterator.concept.bidirectional}}.

\pnum
Additionally,
\tcode{Iterator}
shall \changed{satisfy}{either meet} the requirements of a
\oldconcept{RandomAccessIterator}\iref{random.access.iterators}
\added{or model
\libconcept{RandomAccessIterator}\iref{iterator.concept.random.access}}
if \newnewtxt{the definitions of} any of the members
\begin{itemize}
\item
  \tcode{operator+},
  \tcode{operator-},
  \tcode{operator+=},
  \tcode{operator-=}\iref{reverse.iter.nav},
  \newnewtxt{or}
\item
  \tcode{operator[]}\iref{reverse.iter.elem},
\end{itemize}
or the non-member operators\iref{reverse.iter.cmp}
\begin{itemize}
\item
  \tcode{operator<},
  \tcode{operator>},
  \tcode{operator<=},
  \tcode{operator>=},
  \tcode{operator-},
  or
  \tcode{operator+}\iref{reverse.iter.nonmember}
\end{itemize}
are \oldoldtxt{referenced in a way that requires instantiation}
\newnewtxt{instantiated}\cxxiref{temp.inst}.

[...]

\setcounter{subsubsection}{4}
\rSec3[reverse.iter.elem]{\tcode{reverse_iterator} element access}

[...]

\ednote{This change incorporates the PR of
\href{https://wg21.link/lwg1052}{LWG 1052}):}

\setcounter{Paras}{1}

\indexlibrarymember{operator->}{reverse_iterator}%
\begin{itemdecl}
constexpr pointer operator->() const @\added{requires is_pointer_v<Iterator>}@
  @\added{|| requires(const Iterator i) \{ i.operator->(); \}}@;
\end{itemdecl}

\begin{itemdescr}
\pnum
\removed{\returns \tcode{addressof(operator*())}.}
\begin{addedblock}
\effects
\begin{itemize}
\item If \tcode{Iterator} is a pointer type, equivalent to: \tcode{return prev(current);}

\item Otherwise, equivalent to: \tcode{return prev(current).operator->();}
\end{itemize}
\end{addedblock}
\end{itemdescr}

[...]

\rSec3[reverse.iter.nav]{\tcode{reverse_iterator} navigation}

[...]

\rSec3[reverse.iter.cmp]{\tcode{reverse_iterator} comparisons}

{\color{oldclr}
\pnum
\oldtxt{The functions in this subclause only participate in overload resolution
if the expression in their \textit{Returns:} element is well-formed and
implicitly convertible to \tcode{bool}.}
} %% \color{oldclr}

\indexlibrarymember{operator==}{reverse_iterator}%
\begin{itemdecl}
template<class Iterator1, class Iterator2>
  constexpr bool operator==(
    const reverse_iterator<Iterator1>& x,
    const reverse_iterator<Iterator2>& y);
\end{itemdecl}

\begin{itemdescr}
{\color{newclr}
\pnum
\constraints
The expression \tcode{x.current == y.current} shall be valid and
convertible to \tcode{bool}.
} %% \color{newclr}

\pnum
\returns
\tcode{x.current == y.current}.
\end{itemdescr}

\indexlibrarymember{operator"!=}{reverse_iterator}%
\begin{itemdecl}
template<class Iterator1, class Iterator2>
  constexpr bool operator!=(
    const reverse_iterator<Iterator1>& x,
    const reverse_iterator<Iterator2>& y);
\end{itemdecl}

\begin{itemdescr}
{\color{newclr}
\pnum
\constraints
The expression \tcode{x.current != y.current} shall be valid and
convertible to \tcode{bool}.
} %% \color{newclr}

\pnum
\returns
\tcode{x.current != y.current}.
\end{itemdescr}

\indexlibrarymember{operator<}{reverse_iterator}%
\begin{itemdecl}
template<class Iterator1, class Iterator2>
  constexpr bool operator<(
    const reverse_iterator<Iterator1>& x,
    const reverse_iterator<Iterator2>& y);
\end{itemdecl}

\begin{itemdescr}
{\color{newclr}
\pnum
\constraints
The expression \tcode{x.current > y.current} shall be valid and
convertible to \tcode{bool}.
} %% \color{newclr}

\pnum
\returns
\tcode{x.current > y.current}.
\end{itemdescr}

\indexlibrarymember{operator>}{reverse_iterator}%
\begin{itemdecl}
template<class Iterator1, class Iterator2>
  constexpr bool operator>(
    const reverse_iterator<Iterator1>& x,
    const reverse_iterator<Iterator2>& y);
\end{itemdecl}

\begin{itemdescr}
{\color{newclr}
\pnum
\constraints
The expression \tcode{x.current < y.current} shall be valid and
convertible to \tcode{bool}.
} %% \color{newclr}

\pnum
\returns
\tcode{x.current < y.current}.
\end{itemdescr}

\indexlibrarymember{operator<=}{reverse_iterator}%
\begin{itemdecl}
template<class Iterator1, class Iterator2>
  constexpr bool operator<=(
    const reverse_iterator<Iterator1>& x,
    const reverse_iterator<Iterator2>& y);
\end{itemdecl}

\begin{itemdescr}
{\color{newclr}
\pnum
\constraints
The expression \tcode{x.current >= y.current} shall be valid and
convertible to \tcode{bool}.
} %% \color{newclr}

\pnum
\returns
\tcode{x.current >= y.current}.
\end{itemdescr}

\indexlibrarymember{operator>=}{reverse_iterator}%
\begin{itemdecl}
template<class Iterator1, class Iterator2>
  constexpr bool operator>=(
    const reverse_iterator<Iterator1>& x,
    const reverse_iterator<Iterator2>& y);
\end{itemdecl}

\begin{itemdescr}
{\color{newclr}
\pnum
\constraints
The expression \tcode{x.current <= y.current} shall be valid and
convertible to \tcode{bool}.
} %% \color{newclr}

\pnum
\returns
\tcode{x.current <= y.current}.
\end{itemdescr}

\rSec3[reverse.iter.nonmember]{Non-member functions}

[...]

\setcounter{Paras}{1}
\begin{itemdescr}
\pnum
\returns
\tcode{reverse_iterator<Iterator> (x.current - n)}.
\end{itemdescr}

\begin{addedblock}
\indexlibrarymember{iter_move}{reverse_iterator}%
\begin{itemdecl}
friend constexpr iter_rvalue_reference_t<Iterator> iter_move(const reverse_iterator& i)
   noexcept(@\seebelownc@);
\end{itemdecl}

\begin{itemdescr}
\pnum
\effects Equivalent to:
\begin{codeblock}
@\newtxt{auto tmp = i.current;}@
return ranges::iter_move(@\newtxt{--tmp}@ @\oldtxt{prev(i.current)}@);
\end{codeblock}

\pnum
\remarks The expression in \tcode{noexcept} is equivalent to:
\begin{codeblock}
   is_nothrow_copy_constructible_v<Iterator> &&
     @\oldtxt{noexcept(--declval<Iterator\&>()) \&\&}@
     noexcept(ranges::iter_move(@\newtxt{--i.current}@ @\oldtxt{declval<Iterator\&>()}@))
\end{codeblock}
\end{itemdescr}

\indexlibrarymember{iter_swap}{reverse_iterator}%
\begin{itemdecl}
template<IndirectlySwappable<Iterator> Iterator2>
  friend constexpr void iter_swap(const reverse_iterator& x, const reverse_iterator<Iterator2>& y)
    noexcept(@\seebelownc@);
\end{itemdecl}

\begin{itemdescr}
\pnum
\effects Equivalent to:
\begin{codeblock}
@\oldtxt{ranges::iter_swap(ranges::prev(x.current), ranges::prev(y.current));}@
@\newtxt{auto xtmp = x.current;}@
@\newtxt{auto ytmp = y.current;}@
@\newtxt{ranges::iter_swap(std::move(--xtmp), std::move(--ytmp));}@
\end{codeblock}

\pnum
\remarks The expression in \tcode{noexcept} is equivalent to:
\begin{codeblock}
  is_nothrow_copy_constructible_v<Iterator> &&
    @\newtxt{noexcept(ranges::iter_swap(std::move(--x.current), std::move(--y.current)))}@
    @\oldtxt{noexcept(--declval<Iterator\&>()) \&\&}@
    @\oldtxt{noexcept(ranges::iter_swap(declval<Iterator>(), declval<Iterator>()))}@
\end{codeblock}
\end{itemdescr}
\end{addedblock}

\indexlibrary{\idxcode{reverse_iterator}!\idxcode{make_reverse_iterator} non-member function}%
\indexlibrary{\idxcode{make_reverse_iterator}}%
\begin{itemdecl}
template<class Iterator>
  constexpr reverse_iterator<Iterator> make_reverse_iterator(Iterator i);
\end{itemdecl}

[...]

\rSec2[insert.iterators]{Insert iterators}

[...]

\rSec3[back.insert.iterator]{Class template \tcode{back_insert_iterator}}

\indexlibrary{\idxcode{back_insert_iterator}}%
\begin{codeblock}
namespace std {
  template<class Container>
  class back_insert_iterator {
  protected:
    Container* container @\added{= nullptr}@;

  public:
    using iterator_category = output_iterator_tag;
    using value_type        = void;
    using difference_type   = @\changed{void}{ptrdiff_t}@;
    using pointer           = void;
    using reference         = void;
    using container_type    = Container;

    @\added{constexpr back_insert_iterator() noexcept = default;}@
    explicit back_insert_iterator(Container& x);
    back_insert_iterator& operator=(const typename Container::value_type& value);
    back_insert_iterator& operator=(typename Container::value_type&& value);

    back_insert_iterator& operator*();
    back_insert_iterator& operator++();
    back_insert_iterator  operator++(int);
  };

  template<class Container>
    back_insert_iterator<Container> back_inserter(Container& x);
}
\end{codeblock}

[...]

\rSec3[front.insert.iterator]{Class template \tcode{front_insert_iterator}}

\indexlibrary{\idxcode{front_insert_iterator}}%
\begin{codeblock}
namespace std {
  template<class Container>
  class front_insert_iterator {
  protected:
    Container* container @\added{= nullptr}@;

  public:
    using iterator_category = output_iterator_tag;
    using value_type        = void;
    using difference_type   = @\changed{void}{ptrdiff_t}@;
    using pointer           = void;
    using reference         = void;
    using container_type    = Container;

    @\added{constexpr front_insert_iterator() noexcept = default;}@
    explicit front_insert_iterator(Container& x);
    front_insert_iterator& operator=(const typename Container::value_type& value);
    front_insert_iterator& operator=(typename Container::value_type&& value);

    front_insert_iterator& operator*();
    front_insert_iterator& operator++();
    front_insert_iterator  operator++(int);
  };

  template<class Container>
    front_insert_iterator<Container> front_inserter(Container& x);
}
\end{codeblock}

[...]

\rSec3[insert.iterator]{Class template \tcode{insert_iterator}}

\indexlibrary{\idxcode{insert_iterator}}%
\begin{codeblock}
namespace std {
  template<class Container>
  class insert_iterator {
  protected:
    Container* container @\added{= nullptr}@;
    @\changed{typename Container::iterator}{iterator_t<Container>}@ iter @\added{\{\}}@;

  public:
    using iterator_category = output_iterator_tag;
    using value_type        = void;
    using difference_type   = @\changed{void}{ptrdiff_t}@;
    using pointer           = void;
    using reference         = void;
    using container_type    = Container;

    @\added{insert_iterator() = default;}@
    insert_iterator(Container& x, @\changed{typename Container::iterator}{iterator_t<Container>}@ i);
    insert_iterator& operator=(const typename Container::value_type& value);
    insert_iterator& operator=(typename Container::value_type&& value);

    insert_iterator& operator*();
    insert_iterator& operator++();
    insert_iterator& operator++(int);
  };

  template<class Container>
    insert_iterator<Container> inserter(Container& x, @\changed{typename Container::iterator}{iterator_t<Container>}@ i);
}
\end{codeblock}

\rSec4[insert.iter.ops]{\tcode{insert_iterator} operations}

\indexlibrary{\idxcode{insert_iterator}!constructor}%
\begin{itemdecl}
insert_iterator(Container& x, @\changed{typename Container::iterator}{iterator_t<Container>}@ i);
\end{itemdecl}

[...]

\rSec4[inserter]{\tcode{inserter}}

\indexlibrary{\idxcode{inserter}}%
\begin{itemdecl}
template<class Container>
  insert_iterator<Container> inserter(Container& x, @\changed{typename Container::iterator}{iterator_t<Container>}@ i);
\end{itemdecl}

\begin{itemdescr}
\pnum
\returns
\tcode{insert_iterator<Container>(x, i)}.
\end{itemdescr}


\ednote{Note the change to the title of [move.iterators].}
\rSec2[move.iterators]{Move iterators {\color{addclr} and sentinels}}

[...]

\rSec3[move.iterator]{Class template \tcode{move_iterator}}

\indexlibrary{\idxcode{move_iterator}}%
\begin{codeblock}
namespace std {
  template<class Iterator>
  class move_iterator {
  public:
    using iterator_type     = Iterator;
    @\added{using iterator_concept}@  @\added{= input_iterator_tag;}@
    using iterator_category = @\changed{typename iterator_traits<Iterator>::iterator_category}{\seebelownc}@;
    using value_type        = @\changed{typename iterator_traits<Iterator>::value_type}{iter_value_t<Iterator>}@;
    using difference_type   = @\changed{typename iterator_traits<Iterator>::difference_type}{iter_difference_t<Iterator>}@;
    using pointer           = Iterator;
    using reference         = @\changed{\seebelow}{iter_rvalue_reference_t<Iterator>}@;

    constexpr move_iterator();
    constexpr explicit move_iterator(Iterator i);

    [...]

    constexpr move_iterator& operator++();
    constexpr @\changed{move_iterator}{\oldtxt{decltype(}auto\oldtxt{)}}@ operator++(int);
    constexpr move_iterator& operator--();

    [...]

    constexpr move_iterator operator-(difference_type n) const;
    constexpr move_iterator& operator-=(difference_type n);
    constexpr @\changed{\unspec}{reference}@ operator[](difference_type n) const;

    @\added{template<Sentinel<Iterator> S>}@
      @\added{friend constexpr bool operator==(}@
        @\added{const move_iterator\& x, const move_sentinel<S>\& y);}@
    @\added{template<Sentinel<Iterator> S>}@
      @\added{friend constexpr bool operator==(}@
        @\added{const move_sentinel<S>\& x, const move_iterator\& y);}@
    @\added{template<Sentinel<Iterator> S>}@
      @\added{friend constexpr bool operator!=(}@
        @\added{const move_iterator\& x, const move_sentinel<S>\& y);}@
    @\added{template<Sentinel<Iterator> S>}@
      @\added{friend constexpr bool operator!=(}@
        @\added{const move_sentinel<S>\& x, const move_iterator\& y);}@

    @\added{template<SizedSentinel<Iterator> S>}@
      @\added{friend constexpr iter_difference_t<Iterator> operator-(}@
        @\added{const move_sentinel<S>\& x, const move_iterator\& y);}@
    @\added{template<SizedSentinel<Iterator> S>}@
      @\added{friend constexpr iter_difference_t<Iterator> operator-(}@
        @\added{const move_iterator\& x, const move_sentinel<S>\& y);}@

    @\added{friend constexpr iter_rvalue_reference_t<Iterator> iter_move(const move_iterator\& i)}@
      @\added{noexcept(noexcept(ranges::iter_move(i.current)));}@
    @\added{template<IndirectlySwappable<Iterator> Iterator2>}@
      @\added{friend constexpr void iter_swap(const move_iterator\& x, const move_iterator<Iterator2>\& y)}@
        @\added{noexcept(noexcept(ranges::iter_swap(x.current, y.current)));}@

  private:
    Iterator current;   // \expos
  };

  [...]

  template<class Iterator>
    constexpr move_iterator<Iterator> operator+(
      @\changed{typename move_iterator<Iterator>::difference_type}{iter_difference_t<Iterator>}@ n,
      const move_iterator<Iterator>& x);
  template<class Iterator>
    constexpr move_iterator<Iterator> make_move_iterator(Iterator i);
}
\end{codeblock}

\begin{removedblock}
\pnum
Let \tcode{\placeholder{R}} denote \tcode{iterator_traits<Iterator>::reference}.
If \tcode{is_reference_v<\placeholder{R}>} is \tcode{true},
the template specialization \tcode{move_iterator<Iterator>} shall define
the nested type named \tcode{reference} as a synonym for
\tcode{remove_reference_t<\placeholder{R}>\&\&},
otherwise as a synonym for \tcode{\placeholder{R}}.
\end{removedblock}

\begin{addedblock}
\pnum
The member \grammarterm{typedef-name} \tcode{iterator_category} denotes
\tcode{random_access_iterator_tag} if
\tcode{iterator_traits<\brk{}Iterator>::iterator_category} is derived from
\tcode{random_access_iterator_tag}, and
\tcode{iterator_traits<\brk{}Iterator>::iterator_category} otherwise.
\end{addedblock}

\rSec3[move.iter.requirements]{\tcode{move_iterator} requirements}

\pnum
The template parameter \tcode{Iterator} shall \changed{satisfy}{either meet}
the \oldconcept{InputIterator} requirements\iref{input.iterators}
\added{or model \libconcept{InputIterator}\iref{iterator.concept.input}}.
Additionally, if any of the bidirectional \removed{or random access} traversal
functions are instantiated, the template parameter shall \changed{satisfy}{either meet} the
\oldconcept{BidirectionalIterator} requirements\iref{bidirectional.iterators} or
\added{model \libconcept{BidirectionalIterator}\iref{iterator.concept.bidirectional}}
\removed{\oldconcept{RandomAccessIterator} requirements\iref{random.access.iterators}, respectively}.
\added{If any of the random access traversal functions are instantiated, the
template parameter shall either meet the \oldconcept{RandomAccessIterator}
requirements\iref{random.access.iterators} or model}
\libconcept{\added{RandomAccess\-Iterator}}\added{\iref{iterator.concept.random.access}.}

[...]

\setcounter{subsubsection}{4}
\rSec3[move.iter.elem]{\tcode{move_iterator} element access}

\indexlibrarymember{operator*}{move_iterator}%
\begin{itemdecl}
constexpr reference operator*() const;
\end{itemdecl}

\begin{itemdescr}
\pnum
\removed{\returns \tcode{static_cast<reference>(*current)}.}

\added{\effects Equivalent to: }\tcode{\added{return ranges::iter_move(current);}}
\end{itemdescr}

\indexlibrarymember{operator->}{move_iterator}%
\begin{itemdecl}
constexpr pointer operator->() const;
\end{itemdecl}

\begin{itemdescr}
\pnum
\returns \tcode{current}.
\end{itemdescr}

\indexlibrarymember{operator[]}{move_iterator}%
\begin{itemdecl}
constexpr @\changed{\unspec}{reference}@ operator[](difference_type n) const;
\end{itemdecl}

\begin{itemdescr}
\pnum
\removed{\returns }\tcode{\removed{std::move(current[n])}}\removed{.}

\added{\effects Equivalent to: }\tcode{\added{ranges::iter_move(current + n);}}
\end{itemdescr}

\rSec3[move.iter.nav]{\tcode{move_iterator} navigation}

[...]

\setcounter{Paras}{2}
\indexlibrarymember{operator++}{move_iterator}%
\begin{itemdecl}
constexpr @\changed{move_iterator}{\oldtxt{decltype(}auto\oldtxt{)}}@ operator++(int);
\end{itemdecl}

\begin{itemdescr}
\pnum
\effects
\changed{As if by}{If \tcode{Iterator} models \libconcept{ForwardIterator}, equivalent to}:
\begin{codeblock}
move_iterator tmp = *this;
++current;
return tmp;
\end{codeblock}
\added{Otherwise, equivalent to \tcode{++current}.}
\end{itemdescr}

[...]

\rSec3[move.iter.op.comp]{\tcode{move_iterator} comparisons}

{\color{oldclr}
\pnum
The functions in this subsection only participate in overload resolution if the
expression in their \textit{Returns:} element is well-formed.
} %% \color{oldclr}

\indexlibrarymember{operator==}{move_iterator}%
\begin{itemdecl}
template<class Iterator1, class Iterator2>
constexpr bool operator==(const move_iterator<Iterator1>& x, const move_iterator<Iterator2>& y);
@\added{template<Sentinel<Iterator> S>}@
@\added{friend constexpr bool operator==(const move_iterator\& x, const move_sentinel<S>\& y);}@
@\added{template<Sentinel<Iterator> S>}@
@\added{friend constexpr bool operator==(const move_sentinel<S>\& x, const move_iterator\& y);}@
\end{itemdecl}

\begin{itemdescr}
{\color{newclr}
\pnum
\constraints
The expression \tcode{x.base() == y.base()} shall be valid and
convertible to \tcode{bool}.
} %% \color{newclr}

\pnum
\returns \tcode{x.base() == y.base()}.
\end{itemdescr}

\indexlibrarymember{operator"!=}{move_iterator}%
\begin{itemdecl}
template<class Iterator1, class Iterator2>
constexpr bool operator!=(const move_iterator<Iterator1>& x, const move_iterator<Iterator2>& y);
@\added{template<Sentinel<Iterator> S>}@
@\added{friend constexpr bool operator!=(const move_iterator\& x, const move_sentinel<S>\& y);}@
@\added{template<Sentinel<Iterator> S>}@
@\added{friend constexpr bool operator!=(const move_sentinel<S>\& x, const move_iterator\& y);}@
\end{itemdecl}

\begin{itemdescr}
{\color{newclr}
\pnum
\constraints
The expression \tcode{x.base() == y.base()} shall be valid and
convertible to \tcode{bool}.
} %% \color{newclr}

\pnum
\returns \tcode{!(x == y)}.
\end{itemdescr}

\indexlibrarymember{operator<}{move_iterator}%
\begin{itemdecl}
template<class Iterator1, class Iterator2>
constexpr bool operator<(const move_iterator<Iterator1>& x, const move_iterator<Iterator2>& y);
\end{itemdecl}

\begin{itemdescr}
{\color{newclr}
\pnum
\constraints
The expression \tcode{x.base() < y.base()} shall be valid and
convertible to \tcode{bool}.
} %% \color{newclr}

\pnum
\returns \tcode{x.base() < y.base()}.
\end{itemdescr}

\indexlibrarymember{operator>}{move_iterator}%
\begin{itemdecl}
template<class Iterator1, class Iterator2>
constexpr bool operator>(const move_iterator<Iterator1>& x, const move_iterator<Iterator2>& y);
\end{itemdecl}

\begin{itemdescr}
{\color{newclr}
\pnum
\constraints
The expression \tcode{y.base() < x.base()} shall be valid and
convertible to \tcode{bool}.
} %% \color{newclr}

\pnum
\returns \tcode{y < x}.
\end{itemdescr}

\indexlibrarymember{operator<=}{move_iterator}%
\begin{itemdecl}
template<class Iterator1, class Iterator2>
constexpr bool operator<=(const move_iterator<Iterator1>& x, const move_iterator<Iterator2>& y);
\end{itemdecl}

\begin{itemdescr}
{\color{newclr}
\pnum
\constraints
The expression \tcode{y.base() < x.base()} shall be valid and
convertible to \tcode{bool}.
} %% \color{newclr}

\pnum
\returns \tcode{!(y < x)}.
\end{itemdescr}

\indexlibrarymember{operator>=}{move_iterator}%
\begin{itemdecl}
template<class Iterator1, class Iterator2>
constexpr bool operator>=(const move_iterator<Iterator1>& x, const move_iterator<Iterator2>& y);
\end{itemdecl}

\begin{itemdescr}
{\color{newclr}
\pnum
\constraints
The expression \tcode{x.base() < y.base()} shall be valid and
convertible to \tcode{bool}.
} %% \color{newclr}

\pnum
\returns \tcode{!(x < y)}.
\end{itemdescr}

\rSec3[move.iter.nonmember]{\tcode{move_iterator} non-member functions}

{\color{oldclr}
\pnum
The functions in this subsection only participate in overload resolution if the
expression in their \textit{Returns:} element is well-formed.
} %% \color{oldclr}

\indexlibrarymember{operator-}{move_iterator}%
\begin{itemdecl}
template<class Iterator1, class Iterator2>
  constexpr auto operator-(
    const move_iterator<Iterator1>& x,
    const move_iterator<Iterator2>& y) -> decltype(x.base() - y.base());
@\added{template<SizedSentinel<Iterator> S>}@
@\added{friend constexpr iter_difference_t<Iterator> operator-(}@
    @\added{const move_sentinel<S>\& x, const move_iterator\& y);}@
@\added{template<SizedSentinel<Iterator> S>}@
@\added{friend constexpr iter_difference_t<Iterator> operator-(}@
    @\added{const move_iterator\& x, const move_sentinel<S>\& y);}@
\end{itemdecl}

\begin{itemdescr}
{\color{newclr}
\pnum
\constraints
The expression \tcode{x.base() - y.base()} shall be valid and
convertible to \tcode{bool}.
} %% \color{newclr}

\pnum
\returns \tcode{x.base() - y.base()}.
\end{itemdescr}

\indexlibrarymember{operator+}{move_iterator}%
\begin{itemdecl}
template<class Iterator>
  constexpr move_iterator<Iterator> operator+(
    @\changed{typename move_iterator<Iterator>::difference_type}{iter_difference_t<Iterator>}@ n,
    const move_iterator<Iterator>& x);
\end{itemdecl}

\begin{itemdescr}
{\color{newclr}
\pnum
\constraints
The expression \tcode{x + n} shall be a valid and
have type \tcode{Iterator}.
} %% \color{newclr}

\pnum
\returns \tcode{x + n}.
\end{itemdescr}

\begin{addedblock}
\indexlibrarymember{iter_move}{move_iterator}%
\begin{itemdecl}
friend constexpr iter_rvalue_reference_t<Iterator> iter_move(const move_iterator& i)
  noexcept(noexcept(ranges::iter_move(i.current)));
\end{itemdecl}

\begin{itemdescr}
\pnum
\effects Equivalent to: \tcode{return ranges::iter_move(i.current);}
\end{itemdescr}

\indexlibrarymember{iter_swap}{move_iterator}%
\begin{itemdecl}
template<IndirectlySwappable<Iterator> Iterator2>
  friend constexpr void iter_swap(const move_iterator& x, const move_iterator<Iterator2>& y)
    noexcept(noexcept(ranges::iter_swap(x.current, y.current)));
\end{itemdecl}

\begin{itemdescr}
\pnum
\effects Equivalent to: \tcode{ranges::iter_swap(x.current, y.current)}.
\end{itemdescr}
\end{addedblock}

\indexlibrary{\idxcode{make_move_iterator}}%
\begin{itemdecl}
template<class Iterator>
constexpr move_iterator<Iterator> make_move_iterator(Iterator i);
\end{itemdecl}

\begin{itemdescr}
\pnum
\returns \tcode{move_iterator<Iterator>(i)}.
\end{itemdescr}


\begin{addedblock}
\rSec3[move.sentinel]{Class template \tcode{move_sentinel}}

\pnum
Class template \tcode{move_sentinel} is a sentinel adaptor useful for denoting
ranges together with \tcode{move_iterator}. When an input iterator type
\tcode{I} and sentinel type \tcode{S} model \tcode{Sentinel<S, I>},
\tcode{move_sentinel<S>} and \tcode{move_iterator<I>} model
\tcode{Sentinel<move_sentinel<S>, move_iterator<I>{>}} as well.

\pnum
\begin{example}
A \tcode{move_if} algorithm is easily implemented with
\tcode{copy_if} using \tcode{move_iterator} and \tcode{move_sentinel}:

\begin{codeblock}
template<InputIterator I, Sentinel<I> S, WeaklyIncrementable O,
         IndirectUnaryPredicate<I> Pred>
  requires IndirectlyMovable<I, O>
void move_if(I first, S last, O out, Pred pred) {
  std::ranges::copy_if(move_iterator<I>{first}, move_sentinel<S>{last}, out, pred);
}
\end{codeblock}
\end{example}

\indexlibrary{\idxcode{move_sentinel}}%
\begin{codeblock}
namespace std {
  template<Semiregular S>
  class move_sentinel {
  public:
    constexpr move_sentinel();
    explicit constexpr move_sentinel(S s);
    template<@\oldtxt{ConvertibleTo<S>}\newtxt{class}@ S2>
      @\newtxt{requires ConvertibleTo<const S2\&, S>}@
        constexpr move_sentinel(const move_sentinel<S2>& s);
    template<@\oldtxt{ConvertibleTo<S>}\newtxt{class}@ S2>
      @\newtxt{requires Assignable<S\&, const S2\&>}@
        constexpr move_sentinel& operator=(const move_sentinel<S2>& s);

    constexpr S base() const;

  private:
    S last; // \expos
  };
}
\end{codeblock}

\rSec3[move.sent.ops]{\tcode{move_sentinel} operations}

\indexlibrary{\idxcode{move_sentinel}!\idxcode{move_sentinel}}%
\begin{itemdecl}
constexpr move_sentinel();
\end{itemdecl}

\begin{itemdescr}
\pnum
\effects Value-initializes \tcode{last}.
If \tcode{is_trivially_default_constructible_v<S>} is \tcode{true},
then this constructor is a \tcode{constexpr} constructor.
\end{itemdescr}

\indexlibrary{\idxcode{move_sentinel}!constructor}%
\begin{itemdecl}
explicit constexpr move_sentinel(S s);
\end{itemdecl}

\begin{itemdescr}
\pnum
\effects Initializes \tcode{last} with \tcode{std::move(s)}.
\end{itemdescr}

\indexlibrary{\idxcode{move_sentinel}!constructor}%
\begin{itemdecl}
template<@\oldtxt{ConvertibleTo<S>}\newtxt{class}@ S2>
  @\newtxt{requires ConvertibleTo<const S2\&, S>}@
    constexpr move_sentinel(const move_sentinel<S2>& s);
\end{itemdecl}

\begin{itemdescr}
\pnum
\effects Initializes \tcode{last} with \tcode{s.last}.
\end{itemdescr}

\indexlibrary{\idxcode{operator=}!\idxcode{move_sentinel}}%
\indexlibrary{\idxcode{move_sentinel}!\idxcode{operator=}}%
\begin{itemdecl}
template<@\oldtxt{ConvertibleTo<S>}\newtxt{class}@ S2>
  @\newtxt{requires Assignable<S\&, const S2\&>}@
    constexpr move_sentinel& operator=(const move_sentinel<S2>& s);
\end{itemdecl}

\begin{itemdescr}
\pnum
\effects Equivalent to: \tcode{last = s.last; return *this;}
\end{itemdescr}


\rSec2[iterators.common]{Common iterators}

\pnum
Class template \tcode{common_iterator} is an iterator/sentinel adaptor that is
capable of representing a non-common range of elements (where the types of the
iterator and sentinel differ) as a common range (where they are the same). It
does this by holding either an iterator or a sentinel, and implementing the
equality comparison operators appropriately.

\pnum
\begin{note}
The \tcode{common_iterator} type is useful for interfacing with legacy
code that expects the begin and end of a range to have the same type.
\end{note}

\pnum
\begin{example}
\begin{codeblock}
template<class ForwardIterator>
void fun(ForwardIterator begin, ForwardIterator end);

list<int> s;
// populate the list \tcode{s}
using CI =
  common_iterator<counted_iterator<list<int>::iterator>,
                  default_sentinel>;
// call \tcode{fun} on a range of 10 ints
fun(CI(@\oldtxt{make_}@counted_iterator(s.begin(), 10)),
    CI(default_sentinel()));
\end{codeblock}
\end{example}

\rSec3[common.iterator]{Class template \tcode{common_iterator}}

\indexlibrary{\idxcode{common_iterator}}%
\begin{codeblock}
namespace std {
  template<Iterator I, Sentinel<I> S>
    requires @\newtxt{(}@!Same<I, S>@\newtxt{)}@
  class common_iterator {
  public:
    using difference_type = iter_difference_t<I>;

    constexpr common_iterator() = default;
    constexpr common_iterator(I i);
    constexpr common_iterator(S s);
    template<@\oldtxt{ConvertibleTo<I> I2, ConvertibleTo<S> S2}@ @\newtxt{class I2, class S2}@>
      @\newtxt{requires ConvertibleTo<const I2\&, I> \&\& ConvertibleTo<const S2\&, S>}@
        constexpr common_iterator(const common_iterator<I2, S2>& x);

    template<@\oldtxt{ConvertibleTo<I> I2, ConvertibleTo<S> S2}@ @\newtxt{class I2, class S2}@>
      @\newtxt{requires ConvertibleTo<const I2\&, I> \&\& ConvertibleTo<const S2\&, S> \&\&}@
               @\newtxt{Assignable<I\&, const I2\&> \&\& Assignable<S\&, const S2\&>}@
        common_iterator& operator=(const common_iterator<I2, S2>& x);

    decltype(auto) operator*();
    decltype(auto) operator*() const
      requires @\placeholder{dereferenceable}@<const I>;
    decltype(auto) operator->() const
      requires @\seebelownc@;

    common_iterator& operator++();
    decltype(auto) operator++(int);

    template<class I2, Sentinel<I> S2>
      requires Sentinel<S, I2>
    friend bool operator==(
      const common_iterator& x, const common_iterator<I2, S2>& y);
    template<class I2, Sentinel<I> S2>
      requires Sentinel<S, I2> && EqualityComparableWith<I, I2>
    friend bool operator==(
      const common_iterator& x, const common_iterator<I2, S2>& y);
    template<class I2, Sentinel<I> S2>
      requires Sentinel<S, I2>
    friend bool operator!=(
      const common_iterator& x, const common_iterator<I2, S2>& y);

    template<SizedSentinel<I> I2, SizedSentinel<I> S2>
      requires SizedSentinel<S, I2>
    friend iter_difference_t<I2> operator-(
      const common_iterator& x, const common_iterator<I2, S2>& y);

    friend iter_rvalue_reference_t<I> iter_move(const common_iterator& i)
      noexcept(noexcept(ranges::iter_move(declval<const I&>())))
        requires InputIterator<I>;
    template<IndirectlySwappable<I> I2, class S2>
      friend void iter_swap(const common_iterator& x, const common_iterator<I2, S2>& y)
        noexcept(noexcept(ranges::iter_swap(declval<const I&>(), declval<const I2&>())));

  private:
    variant<I, S> v_{}; // \expos
  };

  @\oldtxt{template<Readable I, class S>}@
  @\oldtxt{struct readable_traits<common_iterator<I, S>> \{}@
    @\oldtxt{using value_type = iter_value_t<I>;}@
  @\oldtxt{\};}@

  template<InputIterator I, class S>
  struct iterator_traits<common_iterator<I, S>> {
    using iterator_concept = @\seebelownc@;
    using iterator_category = @\seebelownc@;
    using value_type = iter_value_t<I>;
    using difference_type = iter_difference_t<I>;
    using pointer = @\seebelownc@;
    using reference = iter_reference_t<I>;
  };
}
\end{codeblock}

\rSec3[common.iter.types]{Associated types}

\pnum
The nested \grammarterm{typedef-name}s of the specialization of
\tcode{iterator_traits} for \tcode{common_iterator<I, S>} are defined as follows.
\begin{itemize}
\item \tcode{iterator_concept} denotes \tcode{forward_iterator_tag}
  if \tcode{I} models \libconcept{ForwardIterator};
  otherwise it denotes \tcode{input_iterator_tag}.
\item Let \tcode{C} denote the type
  \tcode{iterator_traits<I>::iterator_category}.
  If \tcode{C} models  \tcode{DerivedFrom<forward_iterator_tag>},
  \tcode{iterator_category} denotes \tcode{forward_iterator_tag}.
  Otherwise,
  \tcode{iterator_category} denotes \tcode{input_iterator_tag}.
\item If the expression \tcode{a.operator->()} is well-formed,
  where \tcode{a} is an lvalue of type \tcode{const common_iterator<I, S>},
  then \tcode{pointer} denotes the type of that expression.
  Otherwise, \tcode{pointer} denotes \tcode{void}.
\end{itemize}

\rSec3[common.iter.const]{Constructors and conversions}

\indexlibrary{\idxcode{common_iterator}!constructor}%
\begin{itemdecl}
constexpr common_iterator(I i);
\end{itemdecl}

\begin{itemdescr}
\pnum
\effects
Initializes \tcode{v_} as if by \tcode{v_\{in_place_type<I>, std::move(i)\}}.
\end{itemdescr}

\indexlibrary{\idxcode{common_iterator}!constructor}%
\begin{itemdecl}
constexpr common_iterator(S s);
\end{itemdecl}

\begin{itemdescr}
\pnum
\effects Initializes \tcode{v_} as if by
\tcode{v_\{in_place_type<S>, std::move(s)\}}.
\end{itemdescr}

\indexlibrary{\idxcode{common_iterator}!constructor}%
\begin{itemdecl}
template<@\oldtxt{ConvertibleTo<I> I2, ConvertibleTo<S> S2}@ @\newtxt{class I2, class S2}@>
  @\newtxt{requires ConvertibleTo<const I2\&, I> \&\& ConvertibleTo<const S2\&, S>}@
    constexpr common_iterator(const common_iterator<I2, S2>& x);
\end{itemdecl}

\begin{itemdescr}
\pnum
\expects \tcode{x.v_.valueless_by_exception()} is \tcode{false}.

\pnum
\effects
Initializes \tcode{v_} as if by
\tcode{v_\{in_place_index<$i$>, get<$i$>(x.v_)\}},
where $i$ is \tcode{x.v_.index()}.
\end{itemdescr}

\indexlibrary{\idxcode{operator=}!\idxcode{common_iterator}}%
\indexlibrary{\idxcode{common_iterator}!\idxcode{operator=}}%
\begin{itemdecl}
template<@\oldtxt{ConvertibleTo<I> I2, ConvertibleTo<S> S2}@ @\newtxt{class I2, class S2}@>
  @\newtxt{requires ConvertibleTo<const I2\&, I> \&\& ConvertibleTo<const S2\&, S> \&\&}@
           @\newtxt{Assignable<I\&, const I2\&> \&\& Assignable<S\&, const S2\&>}@
    common_iterator& operator=(const common_iterator<I2, S2>& x);
\end{itemdecl}

\begin{itemdescr}
\pnum
\expects \tcode{x.v_.valueless_by_exception()} is \tcode{false}.

\pnum
\effects
Equivalent to:
\begin{itemize}
\item If \tcode{v_.index() == x.v_.index()}, then
\tcode{get<$i$>(v_) = get<$i$>(x.v_)}.

\item Otherwise, \tcode{v_.emplace<$i$>(get<$i$>(x.v_))}.
\end{itemize}
where $i$ is \tcode{x.v_.index()}.

\pnum
\returns \tcode{*this}
\end{itemdescr}

\rSec3[common.iter.access]{Accessors}

\indexlibrary{\idxcode{operator*}!\idxcode{common_iterator}}%
\indexlibrary{\idxcode{common_iterator}!\idxcode{operator*}}%
\begin{itemdecl}
decltype(auto) operator*();
decltype(auto) operator*() const
  requires @\placeholder{dereferenceable}@<const I>;
\end{itemdecl}

\begin{itemdescr}
\pnum
\expects \tcode{holds_alternative<I>(v_)}.

\pnum
\effects Equivalent to: \tcode{return *get<I>(v_);}
\end{itemdescr}

\indexlibrary{\idxcode{operator->}!\idxcode{common_iterator}}%
\indexlibrary{\idxcode{common_iterator}!\idxcode{operator->}}%
\begin{itemdecl}
decltype(auto) operator->() const
  requires @\seebelownc@;
\end{itemdecl}

\begin{itemdescr}
\pnum
The expression in the requires clause is equivalent to:
\begin{codeblock}
Readable<const I> &&
  (requires(const I& i) { i.operator->(); } ||
   is_reference_v<iter_reference_t<I>> ||
   Constructible<iter_value_t<I>, iter_reference_t<I>>)
\end{codeblock}

\pnum
\expects \tcode{holds_alternative<I>(v_)}.

\pnum
\effects
\begin{itemize}
\item
If \tcode{I} is a pointer type or if the expression
\tcode{get<I>(v_).operator->()} is
well-formed, equivalent to: \tcode{return get<I>(v_);}

\item
Otherwise, if \tcode{iter_reference_t<I>} is a reference type, equivalent to:
\begin{codeblock}
auto&& tmp = *get<I>(v_);
return addressof(tmp);
\end{codeblock}

\item
Otherwise, equivalent to: \tcode{return \placeholdernc{proxy}(*get<I>(v_));} where
\tcode{\placeholder{proxy}} is the exposition-only class:
\begin{codeblock}
class @\placeholder{proxy}@ {
  iter_value_t<I> keep_;
  @\placeholdernc{proxy}@(iter_reference_t<I>&& x)
    : keep_(std::move(x)) {}
public:
  const iter_value_t<I>* operator->() const {
    return addressof(keep_);
  }
};
\end{codeblock}
\end{itemize}
\end{itemdescr}

\rSec3[common.iter.nav]{Navigation}

\indexlibrary{\idxcode{operator++}!\idxcode{common_iterator}}%
\indexlibrary{\idxcode{common_iterator}!\idxcode{operator++}}%
\begin{itemdecl}
common_iterator& operator++();
\end{itemdecl}

\begin{itemdescr}
\pnum
\expects \tcode{holds_alternative<I>(v_)}.

\pnum
\effects Equivalent to \tcode{++get<I>(v_)}.

\pnum
\returns \tcode{*this}.
\end{itemdescr}

\indexlibrary{\idxcode{operator++}!\idxcode{common_iterator}}%
\indexlibrary{\idxcode{common_iterator}!\idxcode{operator++}}%
\begin{itemdecl}
decltype(auto) operator++(int);
\end{itemdecl}

\begin{itemdescr}
\pnum
\expects \tcode{holds_alternative<I>(v_)}.

\pnum
\effects
If \tcode{I} models \libconcept{ForwardIterator}, equivalent to:
\begin{codeblock}
common_iterator tmp = *this;
++*this;
return tmp;
\end{codeblock}
Otherwise, equivalent to: \tcode{return get<I>(v_)++;}
\end{itemdescr}

\rSec3[common.iter.cmp]{Comparisons}

\indexlibrary{\idxcode{operator==}!\idxcode{common_iterator}}%
\indexlibrary{\idxcode{common_iterator}!\idxcode{operator==}}%
\begin{itemdecl}
template<class I2, Sentinel<I@\oldtxt{1}@> S2>
  requires Sentinel<S@\oldtxt{1}@, I2>
friend bool operator==(
  const common_iterator& x, const common_iterator<I2, S2>& y);
\end{itemdecl}

\begin{itemdescr}
\pnum
\expects
\tcode{x.v_.valueless_by_exception()} and \tcode{y.v_.valueless_by_exception()}
are each \tcode{false}.

\pnum
\returns
\tcode{true} if \tcode{$i$ == $j$},
and otherwise \tcode{get<$i$>(x.v_) == get<$j$>(y.v_)},
where $i$ is \tcode{x.v_.index()} and $j$ is \tcode{y.v_.index()}.
\end{itemdescr}

\indexlibrary{\idxcode{operator==}!\idxcode{common_iterator}}%
\indexlibrary{\idxcode{common_iterator}!\idxcode{operator==}}%
\begin{itemdecl}
template<class I2, Sentinel<I@\oldtxt{1}@> S2>
  requires Sentinel<S@\oldtxt{1}@, I2> && EqualityComparableWith<I@\oldtxt{1}@, I2>
friend bool operator==(
  const common_iterator& x, const common_iterator<I2, S2>& y);
\end{itemdecl}

\begin{itemdescr}
\pnum
\expects
\tcode{x.v_.valueless_by_exception()} and \tcode{y.v_.valueless_by_exception()}
are each \tcode{false}.

\pnum
\returns
\tcode{true} if $i$ and $j$ are each \tcode{1}, and otherwise
\tcode{get<$i$>(x.v_) == get<$j$>(y.v_)}, where
$i$ is \tcode{x.v_.index()} and $j$ is \tcode{y.v_.index()}.
\end{itemdescr}

\indexlibrary{\idxcode{operator"!=}!\idxcode{common_iterator}}%
\indexlibrary{\idxcode{common_iterator}!\idxcode{operator"!=}}%
\begin{itemdecl}
template<class I2, Sentinel<I@\oldtxt{1}@> S2>
  requires Sentinel<S@\oldtxt{1}@, I2>
friend bool operator!=(
  const common_iterator& x, const common_iterator<I2, S2>& y);
\end{itemdecl}

\begin{itemdescr}
\pnum
\effects Equivalent to: \tcode{return !(x == y);}
\end{itemdescr}

\indexlibrary{\idxcode{operator-}!\idxcode{common_iterator}}%
\indexlibrary{\idxcode{common_iterator}!\idxcode{operator-}}%
\begin{itemdecl}
template<SizedSentinel<I> I2, SizedSentinel<I> S2>
  requires SizedSentinel<S, I2>
friend iter_difference_t<I2> operator-(
  const common_iterator& x, const common_iterator<I2, S2>& y);
\end{itemdecl}

\begin{itemdescr}
\pnum
\expects
\tcode{x.v_.valueless_by_exception()} and \tcode{y.v_.valueless_by_exception()}
are each \tcode{false}.

\pnum
\returns
\tcode{0} if $i$ and $j$ are each \tcode{1}, and otherwise
\tcode{get<$i$>(x.v_) - get<$j$>(y.v_)}, where
$i$ is \tcode{x.v_.index()} and $j$ is \tcode{y.v_.index()}.
\end{itemdescr}

\rSec3[common.iter.cust]{Customization}

\indexlibrary{\idxcode{iter_move}!\idxcode{common_iterator}}%
\indexlibrary{\idxcode{common_iterator}!\idxcode{iter_move}}%
\begin{itemdecl}
friend iter_rvalue_reference_t<I> iter_move(const common_iterator& i)
  noexcept(noexcept(ranges::iter_move(declval<const I&>())))
    requires InputIterator<I>;
\end{itemdecl}

\begin{itemdescr}
\pnum
\expects \tcode{holds_alternative<I>(v_)}.

\pnum
\effects Equivalent to: \tcode{return ranges::iter_move(get<I>(i.v_));}
\end{itemdescr}

\indexlibrary{\idxcode{iter_swap}!\idxcode{common_iterator}}%
\indexlibrary{\idxcode{common_iterator}!\idxcode{iter_swap}}%
\begin{itemdecl}
template<IndirectlySwappable<I> I2, class S2>
  friend void iter_swap(const common_iterator& x, const common_iterator<I2, S2>& y)
    noexcept(noexcept(ranges::iter_swap(declval<const I&>(), declval<const I2&>())));
\end{itemdecl}

\begin{itemdescr}
\pnum
\expects
\tcode{holds_alternative<I>(x.v_)} and \tcode{holds_alternative<I\newtxt{2}>(y.v_)}
are each \tcode{true}.

\pnum
\effects Equivalent to \tcode{ranges::iter_swap(get<I>(x.v_), get<I\newtxt{2}>(y.v_))}.
\end{itemdescr}


\rSec2[default.sentinels]{Default sentinels}
\rSec3[default.sent]{Class \tcode{default_sentinel}}

\indexlibrary{\idxcode{default_sentinel}}%
\begin{itemdecl}
namespace std {
  class default_sentinel { };
}
\end{itemdecl}

\pnum
Class \tcode{default_sentinel} is an empty type used to denote the end of a
range. It is intended to be used together with iterator types that know the bound
of their range (e.g., \tcode{counted_iterator}\iref{counted.iterator}).


\rSec2[iterators.counted]{Counted iterators}
\rSec3[counted.iterator]{Class template \tcode{counted_iterator}}

\pnum
Class template \tcode{counted_iterator} is an iterator adaptor
with the same behavior as the underlying iterator except that it
keeps track of its distance from its starting position. It can be
used together with class \tcode{default_sentinel} in calls to generic
algorithms to operate on a range of $N$ elements starting at a given
position without needing to know the end position \textit{a priori}.

\ednote{The following example incorporates the PR for
\href{https://github.com/ericniebler/stl2/issues/554}{stl2\#554}:}

\pnum
\begin{example}
\begin{codeblock}
list<string> s;
// populate the list \tcode{s} with at least 10 strings
vector<string> v;
// copies 10 strings into \tcode{v}:
ranges::copy(@\oldtxt{make_}@counted_iterator(s.begin(), 10), default_sentinel(), back_inserter(v));
\end{codeblock}
\end{example}

\pnum
Two values \tcode{i1} and \tcode{i2} of (possibly differing) types
\tcode{counted_iterator<I1>}
and \
tcode{counted_iterator<I2>}
refer to elements of the same sequence if and only if
\tcode{next(i1.base(), i1.count())}
and
\tcode{next(\brk{}i2.\brk{}base(), i2.count())}
refer to the same (possibly past-the-end) element.

\indexlibrary{\idxcode{counted_iterator}}%
\begin{codeblock}
namespace std {
  template<Iterator I>
  class counted_iterator {
  public:
    using iterator_type = I;
    using difference_type = iter_difference_t<I>;

    constexpr counted_iterator();
    constexpr counted_iterator(I x, iter_difference_t<I> n);
    template<@\oldtxt{ConvertibleTo<I>}@ @\newtxt{class}@ I2>
      @\newtxt{requires ConvertibleTo<const I2\&, I>}@
        constexpr counted_iterator(const counted_iterator<I2>& x);

    template<@\oldtxt{ConvertibleTo<I>}@ @\newtxt{class}@ I2>
      @\newtxt{requires Assignable<I\&, const I2\&>}@
        constexpr counted_iterator& operator=(const counted_iterator<I2>& x);

    constexpr I base() const;
    constexpr iter_difference_t<I> count() const;
    constexpr decltype(auto) operator*();
    constexpr decltype(auto) operator*() const
      requires @\placeholder{dereferenceable}@<const I>;

    constexpr counted_iterator& operator++();
    decltype(auto) operator++(int);
    constexpr counted_iterator operator++(int)
      requires ForwardIterator<I>;
    constexpr counted_iterator& operator--()
      requires BidirectionalIterator<I>;
    constexpr counted_iterator operator--(int)
      requires BidirectionalIterator<I>;

    constexpr counted_iterator operator+(difference_type n) const
      requires RandomAccessIterator<I>;
    friend constexpr counted_iterator operator+(
      iter_difference_t<I> n, const counted_iterator& x)
        requires RandomAccessIterator<I>;
    constexpr counted_iterator& operator+=(difference_type n)
      requires RandomAccessIterator<I>;

    constexpr counted_iterator operator-(difference_type n) const
      requires RandomAccessIterator<I>;
    template<Common<I> I2>
      friend constexpr iter_difference_t<I2> operator-(
        const counted_iterator& x, const counted_iterator<I2>& y);
    friend constexpr iter_difference_t<I> operator-(
      const counted_iterator& x, default_sentinel);
    friend constexpr iter_difference_t<I> operator-(
      default_sentinel, const counted_iterator& y);
    constexpr counted_iterator& operator-=(difference_type n)
      requires RandomAccessIterator<I>;

    constexpr decltype(auto) operator[](difference_type n) const
      requires RandomAccessIterator<I>;

    template<Common<I> I2>
      friend constexpr bool operator==(
        const counted_iterator& x, const counted_iterator<I2>& y);
    friend constexpr bool operator==(
      const counted_iterator& x, default_sentinel);
    friend constexpr bool operator==(
      default_sentinel, const counted_iterator& x);

    template<Common<I> I2>
      friend constexpr bool operator!=(
        const counted_iterator& x, const counted_iterator<I2>& y);
    friend constexpr bool operator!=(
      const counted_iterator& x, default_sentinel y);
    friend constexpr bool operator!=(
      default_sentinel x, const counted_iterator& y);

    template<Common<I> I2>
      friend constexpr bool operator<(
        const counted_iterator& x, const counted_iterator<I2>& y);
    template<Common<I> I2>
      friend constexpr bool operator>(
        const counted_iterator& x, const counted_iterator<I2>& y);
    template<Common<I> I2>
      friend constexpr bool operator<=(
        const counted_iterator& x, const counted_iterator<I2>& y);
    template<Common<I> I2>
      friend constexpr bool operator>=(
        const counted_iterator& x, const counted_iterator<I2>& y);

    friend constexpr iter_rvalue_reference_t<I> iter_move(const counted_iterator& i)
      noexcept(noexcept(ranges::iter_move(i.current)))
        requires InputIterator<I>;
    template<IndirectlySwappable<I> I2>
      friend constexpr void iter_swap(const counted_iterator& x, const counted_iterator<I2>& y)
        noexcept(noexcept(ranges::iter_swap(x.current, y.current)));

  private:
    I current;                // \expos
    iter_difference_t<I> cnt; // \expos
  };

  @\oldtxt{template<Readable I>}@
  @\oldtxt{struct readable_traits<counted_iterator<I>> \{}@
    @\oldtxt{using value_type = iter_value_t<I>;}@
  @\oldtxt{\};}@

  template<InputIterator I>
  struct iterator_traits<counted_iterator<I>> : iterator_traits<I> {
    using pointer = void;
  };
}
\end{codeblock}

\rSec3[counted.iter.ops]{\tcode{counted_iterator} operations}

\rSec4[counted.iter.op.const]{\tcode{counted_iterator} constructors and conversions}

\indexlibrary{\idxcode{counted_iterator}!\idxcode{counted_iterator}}%
\begin{itemdecl}
constexpr counted_iterator();
\end{itemdecl}

\begin{itemdescr}
\pnum
\effects
Value-initializes \tcode{current} and \tcode{cnt}.
Iterator operations applied to the resulting iterator have defined behavior
if and only if the corresponding operations are defined on
a value-initialized iterator of type \tcode{I}.
\end{itemdescr}

\indexlibrary{\idxcode{counted_iterator}!constructor}%
\begin{itemdecl}
constexpr counted_iterator(I i, iter_difference_t<I> n);
\end{itemdecl}

\begin{itemdescr}
\pnum
\expects \tcode{n >= 0}.

\pnum
\effects
Initializes \tcode{current} with \tcode{i} and \tcode{cnt} with \tcode{n}.
\end{itemdescr}

\indexlibrary{\idxcode{counted_iterator}!constructor}%
\begin{itemdecl}
template<@\oldtxt{ConvertibleTo<I>}@ @\newtxt{class}@ I2>
  @\newtxt{requires ConvertibleTo<const I2\&, I>}@
    constexpr counted_iterator(const counted_iterator<I2>& x);
\end{itemdecl}

\begin{itemdescr}
\pnum
\effects
Initializes \tcode{current} with \tcode{x.current} and
\tcode{cnt} with \tcode{x.cnt}.
\end{itemdescr}

\indexlibrary{\idxcode{operator=}!\idxcode{counted_iterator}}%
\indexlibrary{\idxcode{counted_iterator}!\idxcode{operator=}}%
\begin{itemdecl}
template<@\oldtxt{ConvertibleTo<I>}@ @\newtxt{class}@ I2>
  @\newtxt{requires Assignable<I\&, const I2\&>}@
    constexpr counted_iterator& operator=(const counted_iterator<I2>& x);
\end{itemdecl}

\begin{itemdescr}
\pnum
\effects
Assigns \tcode{x.current} to \tcode{current} and \tcode{x.cnt} to \tcode{cnt}.
\end{itemdescr}

\rSec4[counted.iter.op.conv]{\tcode{counted_iterator} conversion}

\indexlibrary{\idxcode{base}!\idxcode{counted_iterator}}%
\indexlibrary{\idxcode{counted_iterator}!\idxcode{base}}%
\begin{itemdecl}
constexpr I base() const;
\end{itemdecl}

\begin{itemdescr}
\pnum
\effects Equivalent to: \tcode{return current;}
\end{itemdescr}

\rSec4[counted.iter.op.cnt]{\tcode{counted_iterator} count}

\indexlibrary{\idxcode{count}!\idxcode{counted_iterator}}%
\indexlibrary{\idxcode{counted_iterator}!\idxcode{count}}%
\begin{itemdecl}
constexpr iter_difference_t<I> count() const;
\end{itemdecl}

\begin{itemdescr}
\pnum
\effects Equivalent to: \tcode{return cnt;}
\end{itemdescr}

\rSec4[counted.iter.op.star]{\tcode{counted_iterator::operator*}}

\indexlibrary{\idxcode{operator*}!\idxcode{counted_iterator}}%
\indexlibrary{\idxcode{counted_iterator}!\idxcode{operator*}}%
\begin{itemdecl}
constexpr decltype(auto) operator*();
constexpr decltype(auto) operator*() const
  requires @\placeholder{dereferenceable}@<const I>;
\end{itemdecl}

\begin{itemdescr}
\pnum
\effects Equivalent to: \tcode{return *current;}
\end{itemdescr}

\rSec4[counted.iter.op.incr]{\tcode{counted_iterator::operator++}}

\indexlibrary{\idxcode{operator++}!\idxcode{counted_iterator}}%
\indexlibrary{\idxcode{counted_iterator}!\idxcode{operator++}}%
\begin{itemdecl}
constexpr counted_iterator& operator++();
\end{itemdecl}

\begin{itemdescr}
\pnum
\expects \tcode{cnt > 0}.

\pnum
\effects Equivalent to:
\begin{codeblock}
++current;
--cnt;
\end{codeblock}

\pnum
\returns \tcode{*this}.
\end{itemdescr}

\indexlibrary{\idxcode{operator++}!\idxcode{counted_iterator}}%
\indexlibrary{\idxcode{counted_iterator}!\idxcode{operator++}}%
\begin{itemdecl}
decltype(auto) operator++(int);
\end{itemdecl}

\begin{itemdescr}
\pnum
\expects \tcode{cnt > 0}.

\pnum
\effects Equivalent to:
\begin{codeblock}
--cnt;
try { return current++; }
catch(...) { ++cnt; throw; }
\end{codeblock}
\end{itemdescr}

\begin{itemdecl}
constexpr counted_iterator operator++(int)
  requires ForwardIterator<I>;
\end{itemdecl}

\begin{itemdescr}
\pnum
\expects \tcode{cnt > 0}.

\pnum
\effects Equivalent to:
\begin{codeblock}
counted_iterator tmp = *this;
++*this;
return tmp;
\end{codeblock}
\end{itemdescr}

\rSec4[counted.iter.op.decr]{\tcode{counted_iterator::operator-{-}}}

\indexlibrary{\idxcode{operator\dcr}!\idxcode{counted_iterator}}%
\indexlibrary{\idxcode{counted_iterator}!\idxcode{operator\dcr}}%
\begin{itemdecl}
  constexpr counted_iterator& operator--();
    requires BidirectionalIterator<I>
\end{itemdecl}

\begin{itemdescr}
\pnum
\effects Equivalent to:
\begin{codeblock}
--current;
++cnt;
\end{codeblock}

\pnum
\returns \tcode{*this}.
\end{itemdescr}

\indexlibrary{\idxcode{operator\dcr}!\idxcode{counted_iterator}}%
\indexlibrary{\idxcode{counted_iterator}!\idxcode{operator\dcr}}%
\begin{itemdecl}
  constexpr counted_iterator operator--(int)
    requires BidirectionalIterator<I>;
\end{itemdecl}

\begin{itemdescr}
\pnum
\effects Equivalent to:
\begin{codeblock}
counted_iterator tmp = *this;
--*this;
return tmp;
\end{codeblock}
\end{itemdescr}

\rSec4[counted.iter.op.+]{\tcode{counted_iterator::operator+}}

\indexlibrary{\idxcode{operator+}!\idxcode{counted_iterator}}%
\indexlibrary{\idxcode{counted_iterator}!\idxcode{operator+}}%
\begin{itemdecl}
  constexpr counted_iterator operator+(difference_type n) const
    requires RandomAccessIterator<I>;
\end{itemdecl}

\begin{itemdescr}
\pnum
\expects \tcode{n <= cnt}.

\pnum
\effects Equivalent to: \tcode{return counted_iterator(current + n, cnt - n);}
\end{itemdescr}

\indexlibrary{\idxcode{operator+}!\idxcode{counted_iterator}}%
\indexlibrary{\idxcode{counted_iterator}!\idxcode{operator+}}%
\begin{itemdecl}
friend constexpr counted_iterator operator+(
  iter_difference_t<I> n, const counted_iterator& x)
    requires RandomAccessIterator<I>;
\end{itemdecl}

\begin{itemdescr}
\pnum
\effects Equivalent to: \tcode{return x + n;}
\end{itemdescr}

\rSec4[counted.iter.op.+=]{\tcode{counted_iterator::operator+=}}

\indexlibrary{\idxcode{operator+=}!\idxcode{counted_iterator}}%
\indexlibrary{\idxcode{counted_iterator}!\idxcode{operator+=}}%
\begin{itemdecl}
  constexpr counted_iterator& operator+=(difference_type n)
    requires RandomAccessIterator<I>;
\end{itemdecl}

\begin{itemdescr}
\pnum
\expects \tcode{n <= cnt}.

\pnum
\effects Equivalent to:
\begin{codeblock}
current += n;
cnt -= n;
\end{codeblock}

\pnum
\returns \tcode{*this}.
\end{itemdescr}

\rSec4[counted.iter.op.-]{\tcode{counted_iterator::operator-}}

\indexlibrary{\idxcode{operator-}!\idxcode{counted_iterator}}%
\indexlibrary{\idxcode{counted_iterator}!\idxcode{operator-}}%
\begin{itemdecl}
  constexpr counted_iterator operator-(difference_type n) const
    requires RandomAccessIterator<I>;
\end{itemdecl}

\begin{itemdescr}
\pnum
\expects \tcode{-n <= cnt}.

\pnum
\effects Equivalent to: \tcode{return counted_iterator(current - n, cnt + n);}
\end{itemdescr}

\indexlibrary{\idxcode{operator-}!\idxcode{counted_iterator}}%
\indexlibrary{\idxcode{counted_iterator}!\idxcode{operator-}}%
\begin{itemdecl}
template<Common<I> I2>
  friend constexpr iter_difference_t<I2> operator-(
    const counted_iterator& x, const counted_iterator<I2>& y);
\end{itemdecl}

\begin{itemdescr}
\pnum
\expects
\tcode{x} and \tcode{y} shall refer to elements of the same
sequence\iref{counted.iterator}.

\pnum
\effects Equivalent to: \tcode{return y.cnt - x.cnt;}
\end{itemdescr}

\begin{itemdecl}
friend constexpr iter_difference_t<I> operator-(
  const counted_iterator& x, default_sentinel);
\end{itemdecl}

\begin{itemdescr}
\pnum
\effects Equivalent to:
\tcode{return -x.cnt;}
\end{itemdescr}

\begin{itemdecl}
friend constexpr iter_difference_t<I> operator-(
  default_sentinel, const counted_iterator& y);
\end{itemdecl}

\begin{itemdescr}
\pnum
\effects Equivalent to: \tcode{return y.cnt;}
\end{itemdescr}

\rSec4[counted.iter.op.-=]{\tcode{counted_iterator::operator-=}}

\indexlibrary{\idxcode{operator-=}!\idxcode{counted_iterator}}%
\indexlibrary{\idxcode{counted_iterator}!\idxcode{operator-=}}%
\begin{itemdecl}
constexpr counted_iterator& operator-=(difference_type n)
  requires RandomAccessIterator<I>;
\end{itemdecl}

\begin{itemdescr}
\pnum
\expects \tcode{-n <= cnt}.

\pnum
\effects Equivalent to:
\begin{codeblock}
current -= n;
cnt += n;
\end{codeblock}

\pnum
\returns \tcode{*this}.
\end{itemdescr}

\rSec4[counted.iter.op.index]{\tcode{counted_iterator::operator[]}}

\indexlibrary{\idxcode{operator[]}!\idxcode{counted_iterator}}%
\indexlibrary{\idxcode{counted_iterator}!\idxcode{operator[]}}%
\begin{itemdecl}
constexpr decltype(auto) operator[](difference_type n) const
  requires RandomAccessIterator<I>;
\end{itemdecl}

\begin{itemdescr}
\pnum
\expects \tcode{n <= cnt}.

\pnum
\effects Equivalent to: \tcode{return current[n];}
\end{itemdescr}

\rSec4[counted.iter.op.comp]{\tcode{counted_iterator} comparisons}

\indexlibrary{\idxcode{operator==}!\idxcode{counted_iterator}}%
\indexlibrary{\idxcode{counted_iterator}!\idxcode{operator==}}%
\begin{itemdecl}
template<Common<I> I2>
  friend constexpr bool operator==(
    const counted_iterator& x, const counted_iterator<I2>& y);
\end{itemdecl}

\begin{itemdescr}
\pnum
\expects
\tcode{x} and \tcode{y} shall refer to
elements of the same sequence\iref{counted.iterator}.

\pnum
\effects Equivalent to: \tcode{return x.cnt == y.cnt;}
\end{itemdescr}

\begin{itemdecl}
friend constexpr bool operator==(
  const counted_iterator& x, default_sentinel);
friend constexpr bool operator==(
  default_sentinel, const counted_iterator& x);
\end{itemdecl}

\begin{itemdescr}
\pnum
\effects Equivalent to: \tcode{return x.cnt == 0;}
\end{itemdescr}

\indexlibrary{\idxcode{operator"!=}!\idxcode{counted_iterator}}%
\indexlibrary{\idxcode{counted_iterator}!\idxcode{operator"!=}}%
\begin{itemdecl}
template<Common<I> I2>
  friend constexpr bool operator!=(
    const counted_iterator& x, const counted_iterator<I2>& y);
friend constexpr bool operator!=(
  const counted_iterator& x, default_sentinel y);
friend constexpr bool operator!=(
  default_sentinel x, const counted_iterator& y);
\end{itemdecl}

\begin{itemdescr}
\pnum
\expects
For the first overload, \tcode{x} and \tcode{y} shall refer to
elements of the same sequence\iref{counted.iterator}.

\pnum
\effects Equivalent to: \tcode{return !(x == y);}
\end{itemdescr}

\indexlibrary{\idxcode{operator<}!\idxcode{counted_iterator}}%
\indexlibrary{\idxcode{counted_iterator}!\idxcode{operator<}}%
\begin{itemdecl}
template<Common<I> I2>
  friend constexpr bool operator<(
    const counted_iterator& x, const counted_iterator<I2>& y);
\end{itemdecl}

\begin{itemdescr}
\pnum
\expects
\tcode{x} and \tcode{y} shall refer to
elements of the same sequence\iref{counted.iterator}.

\pnum
\effects Equivalent to: \tcode{return y.cnt < x.cnt;}

\pnum
\begin{note}
The argument order in the \textit{Effects} element is reversed
because \tcode{cnt} counts down, not up.
\end{note}
\end{itemdescr}

\indexlibrary{\idxcode{operator>}!\idxcode{counted_iterator}}%
\indexlibrary{\idxcode{counted_iterator}!\idxcode{operator>}}%
\begin{itemdecl}
template<Common<I> I2>
  friend constexpr bool operator>(
    const counted_iterator& x, const counted_iterator<I2>& y);
\end{itemdecl}

\begin{itemdescr}
\pnum
\expects
\tcode{x} and \tcode{y} shall refer to
elements of the same sequence\iref{counted.iterator}.

\pnum
\effects Equivalent to: \tcode{return y < x;}
\end{itemdescr}

\indexlibrary{\idxcode{operator<=}!\idxcode{counted_iterator}}%
\indexlibrary{\idxcode{counted_iterator}!\idxcode{operator<=}}%
\begin{itemdecl}
template<Common<I> I2>
  friend constexpr bool operator<=(
    const counted_iterator& x, const counted_iterator<I2>& y);
\end{itemdecl}

\begin{itemdescr}
\pnum
\expects
\tcode{x} and \tcode{y} shall refer to
elements of the same sequence\iref{counted.iterator}.

\pnum
\effects Equivalent to: \tcode{return !(y < x);}
\end{itemdescr}

\indexlibrary{\idxcode{operator>=}!\idxcode{counted_iterator}}%
\indexlibrary{\idxcode{counted_iterator}!\idxcode{operator>=}}%
\begin{itemdecl}
template<Common<I> I2>
  friend constexpr bool operator>=(
    const counted_iterator& x, const counted_iterator<I2>& y);
\end{itemdecl}

\begin{itemdescr}
\pnum
\expects
\tcode{x} and \tcode{y} shall refer to
elements of the same sequence\iref{counted.iterator}.

\pnum
\effects Equivalent to: \tcode{return !(x < y);}
\end{itemdescr}

\rSec4[counted.iter.nonmember]{\tcode{counted_iterator} customizations}

\indexlibrary{\idxcode{iter_move}!\idxcode{counted_iterator}}%
\indexlibrary{\idxcode{counted_iterator}!\idxcode{iter_move}}%
\begin{itemdecl}
friend constexpr iter_rvalue_reference_t<I> iter_move(const counted_iterator& i)
  noexcept(noexcept(ranges::iter_move(i.current)))
    requires InputIterator<I>;
\end{itemdecl}

\begin{itemdescr}
\pnum
\effects Equivalent to: \tcode{return ranges::iter_move(i.current);}
\end{itemdescr}

\indexlibrary{\idxcode{iter_swap}!\idxcode{counted_iterator}}%
\indexlibrary{\idxcode{counted_iterator}!\idxcode{iter_swap}}%
\begin{itemdecl}
template<IndirectlySwappable<I> I2>
  friend constexpr void iter_swap(const counted_iterator& x, const counted_iterator<I2>& y)
    noexcept(noexcept(ranges::iter_swap(x.current, y.current)));
\end{itemdecl}

\begin{itemdescr}
\pnum
\effects Equivalent to \tcode{ranges::iter_swap(x.current, y.current)}.
\end{itemdescr}


\rSec2[unreachable.sentinels]{Unreachable sentinel}
\rSec3[unreachable.sentinel]{Class \tcode{unreachable}}

\ednote{This wording integrates the PR for
\href{https://github.com/ericniebler/stl2/issues/507}{stl2\#507}):}

\pnum
\indexlibrary{\idxcode{unreachable}}%
Class \tcode{unreachable} is a placeholder type that can be
used with any \libconcept{WeaklyIncrementable} type to
denote the ``upper bound'' of an open interval.
Comparing anything for equality with an object of type
\tcode{unreachable} always returns \tcode{false}.

\pnum
\begin{example}
\begin{codeblock}
char* p;
// set p to point to a character buffer containing newlines
char* nl = find(p, unreachable(), '\n');
\end{codeblock}

Provided a newline character really exists in the buffer, the use of
\tcode{unreachable} above potentially makes the call to \tcode{find} more
efficient since the loop test against the sentinel does not require a
conditional branch.
\end{example}

\begin{codeblock}
namespace std {
  class unreachable {
  public:
    template<WeaklyIncrementable I>
      friend constexpr bool operator==(unreachable, const I&) noexcept;
    template<WeaklyIncrementable I>
      friend constexpr bool operator==(const I&, unreachable) noexcept;
    template<WeaklyIncrementable I>
      friend constexpr bool operator!=(unreachable, const I&) noexcept;
    template<WeaklyIncrementable I>
      friend constexpr bool operator!=(const I&, unreachable) noexcept;
  };
}
\end{codeblock}

\rSec3[unreachable.sentinel.ops]{\tcode{unreachable} operations}

\indexlibrary{\idxcode{operator==}!\idxcode{unreachable}}%
\indexlibrary{\idxcode{unreachable}!\idxcode{operator==}}%
\begin{itemdecl}
template<WeaklyIncrementable I>
  friend constexpr bool operator==(unreachable, const I&) noexcept;
template<WeaklyIncrementable I>
  friend constexpr bool operator==(const I&, unreachable) noexcept;
\end{itemdecl}

\begin{itemdescr}
\pnum
\effects Equivalent to: \tcode{return false;}
\end{itemdescr}

\indexlibrary{\idxcode{operator"!=}!\idxcode{unreachable}}%
\indexlibrary{\idxcode{unreachable}!\idxcode{operator"!=}}%
\begin{itemdecl}
template<WeaklyIncrementable I>
  friend constexpr bool operator!=(unreachable, const I&) noexcept;
template<WeaklyIncrementable I>
  friend constexpr bool operator!=(const I&, unreachable) noexcept;
\end{itemdecl}

\begin{itemdescr}
\pnum
\effects Equivalent to: \tcode{return true;}
\end{itemdescr}
\end{addedblock}


\rSec1[stream.iterators]{Stream iterators}

[...]

\rSec2[istream.iterator]{Class template \tcode{istream_iterator}}

[...]

\begin{codeblock}
namespace std {
  template<class T, class charT = char, class traits = char_traits<charT>,
           class Distance = ptrdiff_t>
  class istream_iterator {
  public:
    [...]

    constexpr istream_iterator();
    @\added{constexpr istream_iterator(default_sentinel);}@
    istream_iterator(istream_type& s);

    [...]

    istream_iterator  operator++(int);

    @\added{friend bool operator==(const istream_iterator\& i, default_sentinel);}@
    @\added{friend bool operator==(default_sentinel, const istream_iterator\& i);}@
    @\added{friend bool operator!=(const istream_iterator\& x, default_sentinel y);}@
    @\added{friend bool operator!=(default_sentinel x, const istream_iterator\& y);}@

  private:
    [...]
  };

  [...]
}
\end{codeblock}

\rSec3[istream.iterator.cons]{\tcode{istream_iterator} constructors and destructor}

\indexlibrary{\idxcode{istream_iterator}!constructor}%
\begin{itemdecl}
constexpr istream_iterator();
@\added{constexpr istream_iterator(default_sentinel);}@
\end{itemdecl}

\begin{itemdescr}
\pnum
\effects
Constructs the end-of-stream iterator.
If \tcode{is_trivially_default_constructible_v<T>} is \tcode{true},
then \changed{this constructor is a}{these constructors are} constexpr
constructor\added{s}.

\pnum
\ensures \tcode{in_stream == 0}.
\end{itemdescr}

[...]

\rSec3[istream.iterator.ops]{\tcode{istream_iterator} operations}

[...]

\setcounter{Paras}{7}
\indexlibrarymember{operator==}{istream_iterator}%
\begin{itemdecl}
template<class T, class charT, class traits, class Distance>
  bool operator==(const istream_iterator<T,charT,traits,Distance>& x,
                  const istream_iterator<T,charT,traits,Distance>& y);
\end{itemdecl}

\begin{itemdescr}
\pnum
\returns
\tcode{x.in_stream == y.in_stream}.
\end{itemdescr}

\begin{addedblock}
\indexlibrarymember{operator==}{istream_iterator}%
\begin{itemdecl}
friend bool operator==(default_sentinel, const istream_iterator& i);
friend bool operator==(const istream_iterator& i, default_sentinel);
\end{itemdecl}

\begin{itemdescr}
\pnum
\returns
\tcode{!i.in_stream}.
\end{itemdescr}
\end{addedblock}

\indexlibrarymember{operator"!=}{istream_iterator}%
\begin{itemdecl}
template<class T, class charT, class traits, class Distance>
  bool operator!=(const istream_iterator<T,charT,traits,Distance>& x,
                  const istream_iterator<T,charT,traits,Distance>& y);
@\added{friend bool operator!=(default_sentinel x, const istream_iterator\& y);}@
@\added{friend bool operator!=(const istream_iterator\& x, default_sentinel y);}@
\end{itemdecl}

\begin{itemdescr}
\pnum
\returns
\tcode{!(x == y)}
\end{itemdescr}


\rSec2[ostream.iterator]{Class template \tcode{ostream_iterator}}

[...]

\setcounter{Paras}{1}
\pnum \tcode{ostream_iterator} is defined as:

\begin{codeblock}
namespace std {
  template<class T, class charT = char, class traits = char_traits<charT>>
  class ostream_iterator {
  public:
    using iterator_category = output_iterator_tag;
    using value_type        = void;
    using difference_type   = @\changed{void}{ptrdiff_t}@;
    using pointer           = void;
    using reference         = void;
    using char_type         = charT;
    using traits_type       = traits;
    using ostream_type      = basic_ostream<charT,traits>;

    @\added{constexpr ostream_iterator() noexcept = default;}@
    ostream_iterator(ostream_type& s);

    [...]

  private:
    basic_ostream<charT,traits>* out_stream @\added{= nullptr}@;  // \expos
    const charT* delim @\added{= nullptr}@;                       // \expos
  };
}
\end{codeblock}

[...]

\rSec2[istreambuf.iterator]{Class template \tcode{istreambuf_iterator}}

[...]

\indexlibrary{\idxcode{istreambuf_iterator}}%
\begin{codeblock}
namespace std {
  template<class charT, class traits = char_traits<charT>>
  class istreambuf_iterator {
  public:
    [...]

    constexpr istreambuf_iterator() noexcept;
    @\added{constexpr istreambuf_iterator(default_sentinel) noexcept;}@
    istreambuf_iterator(const istreambuf_iterator&) noexcept = default;

    [...]

    bool equal(const istreambuf_iterator& b) const;

    @\added{friend bool operator==(default_sentinel s, const istreambuf_iterator\& i);}@
    @\added{friend bool operator==(const istreambuf_iterator\& i, default_sentinel s);}@
    @\added{friend bool operator!=(default_sentinel a, const istreambuf_iterator\& b);}@
    @\added{friend bool operator!=(const istreambuf_iterator\& a, default_sentinel b);}@

  private:
    streambuf_type* sbuf_;                // \expos
  };

  [...]
}
\end{codeblock}

[...]

\setcounter{subsubsection}{1}
\rSec3[istreambuf.iterator.cons]{\tcode{istreambuf_iterator} constructors}

[...]

\setcounter{Paras}{1}
\indexlibrary{\idxcode{istreambuf_iterator}!constructor}%
\begin{itemdecl}
constexpr istreambuf_iterator() noexcept;
@\added{constexpr istreambuf_iterator(default_sentinel) noexcept;}@
\end{itemdecl}

\begin{itemdescr}
\pnum
\effects
Initializes \tcode{sbuf_} with \tcode{nullptr}.
\end{itemdescr}

[...]

\rSec3[istreambuf.iterator.ops]{\tcode{istreambuf_iterator} operations}

[...]

\setcounter{Paras}{5}
\indexlibrarymember{operator==}{istreambuf_iterator}%
\begin{itemdecl}
template<class charT, class traits>
  bool operator==(const istreambuf_iterator<charT,traits>& a,
                  const istreambuf_iterator<charT,traits>& b);
\end{itemdecl}

\begin{itemdescr}
\pnum
\returns
\tcode{a.equal(b)}.
\end{itemdescr}

\begin{addedblock}
\indexlibrarymember{operator==}{istreambuf_iterator}%
\begin{itemdecl}
friend bool operator==(default_sentinel s, const istreambuf_iterator& i);
friend bool operator==(const istreambuf_iterator& i, default_sentinel s);
\end{itemdecl}

\begin{itemdescr}
\pnum
\returns \tcode{i.equal(s)}.
\end{itemdescr}
\end{addedblock}

\indexlibrarymember{operator"!=}{istreambuf_iterator}%
\begin{itemdecl}
template<class charT, class traits>
  bool operator!=(const istreambuf_iterator<charT,traits>& a,
                  const istreambuf_iterator<charT,traits>& b);
@\added{friend bool operator!=(default_sentinel a, const istreambuf_iterator\& b);}@
@\added{friend bool operator!=(const istreambuf_iterator\& a, default_sentinel b);}@
\end{itemdecl}

\begin{itemdescr}
\pnum
\returns
\changed{\tcode{!a.equal(b)}}{\tcode{!(a == b)}}.
\end{itemdescr}


\rSec2[ostreambuf.iterator]{Class template \tcode{ostreambuf_iterator}}

\indexlibrary{\idxcode{ostreambuf_iterator}}%
\begin{codeblock}
namespace std {
  template<class charT, class traits = char_traits<charT>>
  class ostreambuf_iterator {
  public:
    using iterator_category = output_iterator_tag;
    using value_type        = void;
    using difference_type   = @\changed{void}{ptrdiff_t}@;
    using pointer           = void;
    using reference         = void;
    using char_type         = charT;
    using traits_type       = traits;
    using streambuf_type    = basic_streambuf<charT,traits>;
    using ostream_type      = basic_ostream<charT,traits>;

    @\added{constexpr ostreambuf_iterator() noexcept = default;}@

    [...]

  private:
    streambuf_type* sbuf_ @\added{= nullptr}@;    // \expos
  };
}
\end{codeblock}
