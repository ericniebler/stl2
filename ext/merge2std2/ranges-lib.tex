%!TEX root = P0896.tex
\setcounter{chapter}{28}
\ednote{Add a new clause between [algorithm] and [numerics] with the following content:}
{\color{addclr}
\rSec0[range]{Ranges library}

\rSec1[range.general]{General}

\pnum
This clause describes components for dealing with ranges of elements.

\pnum
The following subclauses describe
range and view requirements, and
components for
range primitives
as summarized in Table~\ref{tab:ranges.lib.summary}.

\begin{libsumtab}{Ranges library summary}{tab:ranges.lib.summary}
  \ref{range.iterators}    & Iterators         & \tcode{<\oldtxt{experimental/ranges/}range>} \\
  \ref{range.access}       & Range access      & \\
  \ref{range.primitives}   & Range primitives  & \\
  \ref{range.requirements} & Requirements      & \\
  \ref{range.algorithms}   & Algorithms        & \\
\end{libsumtab}

\rSec1[range.decaycopy]{decay_copy}

\ednote{TODO: Replace the definition of [thread.decaycopy] with this definition.}

\pnum
Several places in this clause use the expression \tcode{\textit{DECAY_COPY}(x)},
which is expression-equivalent to:
\begin{codeblock}
  decay_t<decltype((x))>(x)
\end{codeblock}

\rSec1[range.synopsis]{Header \tcode{<range>} synopsis}

\indexlibrary{\idxhdr{range}}%
\begin{codeblock}
@\oldtxt{\#include <experimental/ranges/iterator>}@
#include <initializer_list>

namespace std { @\oldtxt{namespace experimental \{}@
  namespace ranges { @\oldtxt{inline namespace v1 \{}@
    template <class T> concept @\oldtxt{bool}@ @\placeholder{dereferenceable}@ // \expos
      = requires(T& t) { {*t} -> auto&&; };

    // \ref{range.iterator.requirements}, iterator requirements:
    // \ref{range.iterator.custpoints}, customization points:
    @\newtxt{inline}@ namespace @\newtxt{\unspec}@ {
      // \ref{range.iterator.custpoints.iter_move}, iter_move:
      @\newtxt{inline}@ constexpr @\unspec@ iter_move = @\unspec@;

      // \ref{range.iterator.custpoints.iter_swap}, iter_swap:
      @\newtxt{inline}@ constexpr @\unspec@ iter_swap = @\unspec@;
    }

    // \ref{range.iterator.assoc.types}, associated types:
    // \ref{range.iterator.assoc.types.difference_type}, difference_type:
    template <class> struct difference_type;
    template <class T> using difference_type_t
      = typename difference_type<T>::type;

    // \ref{range.iterator.assoc.types.value_type}, value_type:
    template <class> struct value_type;
    template <class T> using value_type_t
      = typename value_type<T>::type;

    // \ref{range.iterator.assoc.types.iterator_category}, iterator_category:
    template <class> struct iterator_category;
    template <class T> using iterator_category_t
      = typename iterator_category<T>::type;

    template <@\placeholder{dereferenceable}@ T> using reference_t
      = decltype(*declval<T&>());

    template <@\placeholder{dereferenceable}@ T>
        requires @\seebelow@ using rvalue_reference_t
      = decltype(ranges::iter_move(declval<T&>()));

    // \ref{range.iterators.readable}, Readable:
    template <class In>
    concept @\oldtxt{bool}@ Readable = @\seebelow@;

    // \ref{range.iterators.writable}, Writable:
    template <class Out, class T>
    concept @\oldtxt{bool}@ Writable = @\seebelow@;

    // \ref{range.iterators.weaklyincrementable}, WeaklyIncrementable:
    template <class I>
    concept @\oldtxt{bool}@ WeaklyIncrementable = @\seebelow@;

    // \ref{range.iterators.incrementable}, Incrementable:
    template <class I>
    concept @\oldtxt{bool}@ Incrementable = @\seebelow@;

    // \ref{range.iterators.iterator}, Iterator:
    template <class I>
    concept @\oldtxt{bool}@ Iterator = @\seebelow@;

    // \ref{range.iterators.sentinel}, Sentinel:
    template <class S, class I>
    concept @\oldtxt{bool}@ Sentinel = @\seebelow@;

    // \ref{range.iterators.sizedsentinel}, SizedSentinel:
    template <class S, class I>
    constexpr bool disable_sized_sentinel = false;

    template <class S, class I>
    concept @\oldtxt{bool}@ SizedSentinel = @\seebelow@;

    // \ref{range.iterators.input}, InputIterator:
    template <class I>
    concept @\oldtxt{bool}@ InputIterator = @\seebelow@;

    // \ref{range.iterators.output}, OutputIterator:
    template <class I>
    concept @\oldtxt{bool}@ OutputIterator = @\seebelow@;

    // \ref{range.iterators.forward}, ForwardIterator:
    template <class I>
    concept @\oldtxt{bool}@ ForwardIterator = @\seebelow@;

    // \ref{range.iterators.bidirectional}, BidirectionalIterator:
    template <class I>
    concept @\oldtxt{bool}@ BidirectionalIterator = @\seebelow@;

    // \ref{range.iterators.random.access}, RandomAccessIterator:
    template <class I>
    concept @\oldtxt{bool}@ RandomAccessIterator = @\seebelow@;

    // \ref{range.indirectcallable}, indirect callable requirements:
    // \ref{range.indirectcallable.indirectinvocable}, indirect callables:
    template <class F, class I>
    concept @\oldtxt{bool}@ IndirectUnaryInvocable = @\seebelow@;

    template <class F, class I>
    concept @\oldtxt{bool}@ IndirectRegularUnaryInvocable = @\seebelow@;

    template <class F, class I>
    concept @\oldtxt{bool}@ IndirectUnaryPredicate = @\seebelow@;

    template <class F, class I1, class I2 = I1>
    concept @\oldtxt{bool}@ IndirectRelation = @\seebelow@;

    template <class F, class I1, class I2 = I1>
    concept @\oldtxt{bool}@ IndirectStrictWeakOrder = @\seebelow@;

    template <class@\newtxt{, class...}@>
    struct indirect_result@\oldtxt{_of}@ @\newtxt{\{ \}}@;

    template <class F, class... Is>
      requires @\newtxt{(Readable<Is> \&\& ...) \&\&}@ Invocable<F, reference_t<Is>...>
    struct indirect_result@\oldtxt{_of}@<F@\oldtxt{(}\newtxt{, }@Is...@\oldtxt{)}@>;

    template <class F@\newtxt{, class... Is}@>
    using indirect_result@\oldtxt{_of}@_t
      = typename indirect_result@\oldtxt{_of}@<F@\newtxt{, Is...}@>::type;

    // \ref{range.projected}, projected:
    template <Readable I, IndirectRegularUnaryInvocable<I> Proj>
    struct projected;

    template <WeaklyIncrementable I, class Proj>
    struct difference_type<projected<I, Proj>>;

    // \ref{range.commonalgoreq}, common algorithm requirements:
    // \ref{range.commonalgoreq.indirectlymovable} IndirectlyMovable:
    template <class In, class Out>
    concept @\oldtxt{bool}@ IndirectlyMovable = @\seebelow@;

    template <class In, class Out>
    concept @\oldtxt{bool}@ IndirectlyMovableStorable = @\seebelow@;

    // \ref{range.commonalgoreq.indirectlycopyable} IndirectlyCopyable:
    template <class In, class Out>
    concept @\oldtxt{bool}@ IndirectlyCopyable = @\seebelow@;

    template <class In, class Out>
    concept @\oldtxt{bool}@ IndirectlyCopyableStorable = @\seebelow@;

    // \ref{range.commonalgoreq.indirectlyswappable} IndirectlySwappable:
    template <class I1, class I2 = I1>
    concept @\oldtxt{bool}@ IndirectlySwappable = @\seebelow@;

    // \ref{range.commonalgoreq.indirectlycomparable} IndirectlyComparable:
    template <class I1, class I2, class R = equal_to<>, class P1 = identity,
        class P2 = identity>
    concept @\oldtxt{bool}@ IndirectlyComparable = @\seebelow@;

    // \ref{range.commonalgoreq.permutable} Permutable:
    template <class I>
    concept @\oldtxt{bool}@ Permutable = @\seebelow@;

    // \ref{range.commonalgoreq.mergeable} Mergeable:
    template <class I1, class I2, class Out,
        class R = less<>, class P1 = identity, class P2 = identity>
    concept @\oldtxt{bool}@ Mergeable = @\seebelow@;

    template <class I, class R = less<>, class P = identity>
    concept @\oldtxt{bool}@ Sortable = @\seebelow@;

    // \ref{range.iterator.primitives}, primitives:
    // \ref{range.iterator.traits}, traits:
    template <class Iterator> using iterator_traits = @\seebelow@;

    template <Readable T> using iter_common_reference_t
      = common_reference_t<reference_t<T>, value_type_t<T>&>;

    // \ref{range.iterator.tags}, iterator tags:
    struct output_iterator_tag { };
    struct input_iterator_tag { };
    struct forward_iterator_tag : input_iterator_tag { };
    struct bidirectional_iterator_tag : forward_iterator_tag { };
    struct random_access_iterator_tag : bidirectional_iterator_tag { };

    // \ref{range.iterator.operations}, iterator operations:
    @\oldtxt{namespace \{}@
      @\oldtxt{constexpr \unspec advance = \unspec;}@
      @\oldtxt{constexpr \unspec distance = \unspec;}@
      @\oldtxt{constexpr \unspec next = \unspec;}@
      @\oldtxt{constexpr \unspec prev = \unspec;}@
    @\oldtxt{\}}@
    @\newtxt{template <Iterator I>}@
      @\newtxt{constexpr void advance(I\& i, difference_type_t<I> n);}@
    @\newtxt{template <Iterator I, Sentinel<I> S>}@
      @\newtxt{constexpr void advance(I\& i, S bound);}@
    @\newtxt{template <Iterator I, Sentinel<I> S>}@
      @\newtxt{constexpr difference_type_t<I> advance(I\& i, difference_type_t<I> n, S bound);}@
    @\newtxt{template <Iterator I, Sentinel<I> S>}@
      @\newtxt{constexpr difference_type_t<I> distance(I first, S last);}@
    @\newtxt{template <Range R>}@
      @\newtxt{constexpr difference_type_t<iterator_t<R>{}> distance(R\&\& r);}@
    @\newtxt{template <Iterator I>}@
      @\newtxt{constexpr I next(I x);}@
    @\newtxt{template <Iterator I>}@
      @\newtxt{constexpr I next(I x, difference_type_t<I> n);}@
    @\newtxt{template <Iterator I, Sentinel<I> S>}@
      @\newtxt{constexpr I next(I x, S bound);}@
    @\newtxt{template <Iterator I, Sentinel<I> S>}@
      @\newtxt{constexpr I next(I x, difference_type_t<I> n, S bound);}@
    @\newtxt{template <BidirectionalIterator I>}@
      @\newtxt{constexpr I prev(I x);}@
    @\newtxt{template <BidirectionalIterator I>}@
      @\newtxt{constexpr I prev(I x, difference_type_t<I> n);}@
    @\newtxt{template <BidirectionalIterator I>}@
      @\newtxt{constexpr I prev(I x, difference_type_t<I> n, I bound);}@


    // \ref{range.iterators.predef}, predefined iterators and sentinels:

    // \ref{range.iterators.reverse}, reverse iterators:
    template <BidirectionalIterator I> class reverse_iterator;

    template <class I1, class I2>
        requires EqualityComparableWith<I1, I2>
      constexpr bool operator==(
        const reverse_iterator<I1>& x,
        const reverse_iterator<I2>& y);
    template <class I1, class I2>
        requires EqualityComparableWith<I1, I2>
      constexpr bool operator!=(
        const reverse_iterator<I1>& x,
        const reverse_iterator<I2>& y);
    template <class I1, class I2>
        requires StrictTotallyOrderedWith<I1, I2>
      constexpr bool operator<(
        const reverse_iterator<I1>& x,
        const reverse_iterator<I2>& y);
    template <class I1, class I2>
        requires StrictTotallyOrderedWith<I1, I2>
      constexpr bool operator>(
        const reverse_iterator<I1>& x,
        const reverse_iterator<I2>& y);
    template <class I1, class I2>
        requires StrictTotallyOrderedWith<I1, I2>
      constexpr bool operator>=(
        const reverse_iterator<I1>& x,
        const reverse_iterator<I2>& y);
    template <class I1, class I2>
        requires StrictTotallyOrderedWith<I1, I2>
      constexpr bool operator<=(
        const reverse_iterator<I1>& x,
        const reverse_iterator<I2>& y);

    template <class I1, class I2>
        requires SizedSentinel<I1, I2>
      constexpr difference_type_t<I2> operator-(
        const reverse_iterator<I1>& x,
        const reverse_iterator<I2>& y);
    template <RandomAccessIterator I>
      constexpr reverse_iterator<I> operator+(
        difference_type_t<I> n,
        const reverse_iterator<I>& x);

    template <BidirectionalIterator I>
      constexpr reverse_iterator<I> make_reverse_iterator(I i);

    // \ref{range.iterators.insert}, insert iterators:
    template <class Container> class back_insert_iterator;
    template <class Container>
      back_insert_iterator<Container> back_inserter(Container& x);

    template <class Container> class front_insert_iterator;
    template <class Container>
      front_insert_iterator<Container> front_inserter(Container& x);

    template <class Container> class insert_iterator;
    template <class Container>
      insert_iterator<Container> inserter(Container& x, iterator_t<Container> i);

    // \ref{range.iterators.move}, move iterators and sentinels:
    template <InputIterator I> class move_iterator;
    template <class I1, class I2>
        requires EqualityComparableWith<I1, I2>
      constexpr bool operator==(
        const move_iterator<I1>& x, const move_iterator<I2>& y);
    template <class I1, class I2>
        requires EqualityComparableWith<I1, I2>
      constexpr bool operator!=(
        const move_iterator<I1>& x, const move_iterator<I2>& y);
    template <class I1, class I2>
        requires StrictTotallyOrderedWith<I1, I2>
      constexpr bool operator<(
        const move_iterator<I1>& x, const move_iterator<I2>& y);
    template <class I1, class I2>
        requires StrictTotallyOrderedWith<I1, I2>
      constexpr bool operator<=(
        const move_iterator<I1>& x, const move_iterator<I2>& y);
    template <class I1, class I2>
        requires StrictTotallyOrderedWith<I1, I2>
      constexpr bool operator>(
        const move_iterator<I1>& x, const move_iterator<I2>& y);
    template <class I1, class I2>
        requires StrictTotallyOrderedWith<I1, I2>
      constexpr bool operator>=(
        const move_iterator<I1>& x, const move_iterator<I2>& y);

    template <class I1, class I2>
        requires SizedSentinel<I1, I2>
      constexpr difference_type_t<I2> operator-(
        const move_iterator<I1>& x,
        const move_iterator<I2>& y);
    template <RandomAccessIterator I>
      constexpr move_iterator<I> operator+(
        difference_type_t<I> n,
        const move_iterator<I>& x);
    template <InputIterator I>
      constexpr move_iterator<I> make_move_iterator(I i);

    template <Semiregular S> class move_sentinel;

    template <class I, Sentinel<I> S>
      constexpr bool operator==(
        const move_iterator<I>& i, const move_sentinel<S>& s);
    template <class I, Sentinel<I> S>
      constexpr bool operator==(
        const move_sentinel<S>& s, const move_iterator<I>& i);
    template <class I, Sentinel<I> S>
      constexpr bool operator!=(
        const move_iterator<I>& i, const move_sentinel<S>& s);
    template <class I, Sentinel<I> S>
      constexpr bool operator!=(
        const move_sentinel<S>& s, const move_iterator<I>& i);

    template <class I, SizedSentinel<I> S>
      constexpr difference_type_t<I> operator-(
        const move_sentinel<S>& s, const move_iterator<I>& i);
    template <class I, SizedSentinel<I> S>
      constexpr difference_type_t<I> operator-(
        const move_iterator<I>& i, const move_sentinel<S>& s);

    template <Semiregular S>
      constexpr move_sentinel<S> make_move_sentinel(S s);

    // \ref{range.iterators.common}, common iterators:
    template <Iterator I, Sentinel<I> S>
      requires !Same<I, S>
    class common_iterator;

    template <Readable I, class S>
    struct value_type<common_iterator<I, S>>;

    template <InputIterator I, class S>
    struct iterator_category<common_iterator<I, S>>;

    template <ForwardIterator I, class S>
    struct iterator_category<common_iterator<I, S>>;

    template <class I1, class I2, Sentinel<I2> S1, Sentinel<I1> S2>
    bool operator==(
      const common_iterator<I1, S1>& x, const common_iterator<I2, S2>& y);
    template <class I1, class I2, Sentinel<I2> S1, Sentinel<I1> S2>
      requires EqualityComparableWith<I1, I2>
    bool operator==(
      const common_iterator<I1, S1>& x, const common_iterator<I2, S2>& y);
    template <class I1, class I2, Sentinel<I2> S1, Sentinel<I1> S2>
    bool operator!=(
      const common_iterator<I1, S1>& x, const common_iterator<I2, S2>& y);

    template <class I2, SizedSentinel<I2> I1, SizedSentinel<I2> S1, SizedSentinel<I1> S2>
    difference_type_t<I2> operator-(
      const common_iterator<I1, S1>& x, const common_iterator<I2, S2>& y);

    // \ref{range.default.sentinels}, default sentinels:
    class default_sentinel;

    // \ref{range.iterators.counted}, counted iterators:
    template <Iterator I> class counted_iterator;

    template <class I1, class I2>
        requires Common<I1, I2>
      constexpr bool operator==(
        const counted_iterator<I1>& x, const counted_iterator<I2>& y);
    @\newtxt{template <class I>}@
      constexpr bool operator==(
        const counted_iterator<@\oldtxt{auto}\newtxt{I}@>& x, default_sentinel);
    @\newtxt{template <class I>}@
      constexpr bool operator==(
        default_sentinel, const counted_iterator<@\oldtxt{auto}\newtxt{I}@>& x);
    template <class I1, class I2>
        requires Common<I1, I2>
      constexpr bool operator!=(
        const counted_iterator<I1>& x, const counted_iterator<I2>& y);
    @\newtxt{template <class I>}@
      constexpr bool operator!=(
        const counted_iterator<@\oldtxt{auto}\newtxt{I}@>& x, default_sentinel y);
    @\newtxt{template <class I>}@
      constexpr bool operator!=(
        default_sentinel x, const counted_iterator<@\oldtxt{auto}\newtxt{I}@>& y);
    template <class I1, class I2>
        requires Common<I1, I2>
      constexpr bool operator<(
        const counted_iterator<I1>& x, const counted_iterator<I2>& y);
    template <class I1, class I2>
        requires Common<I1, I2>
      constexpr bool operator<=(
        const counted_iterator<I1>& x, const counted_iterator<I2>& y);
    template <class I1, class I2>
        requires Common<I1, I2>
      constexpr bool operator>(
        const counted_iterator<I1>& x, const counted_iterator<I2>& y);
    template <class I1, class I2>
        requires Common<I1, I2>
      constexpr bool operator>=(
        const counted_iterator<I1>& x, const counted_iterator<I2>& y);
    template <class I1, class I2>
        requires Common<I1, I2>
      constexpr difference_type_t<I2> operator-(
        const counted_iterator<I1>& x, const counted_iterator<I2>& y);
    template <class I>
      constexpr difference_type_t<I> operator-(
        const counted_iterator<I>& x, default_sentinel y);
    template <class I>
      constexpr difference_type_t<I> operator-(
        default_sentinel x, const counted_iterator<I>& y);
    template <RandomAccessIterator I>
      constexpr counted_iterator<I>
        operator+(difference_type_t<I> n, const counted_iterator<I>& x);
    template <Iterator I>
      constexpr counted_iterator<I> make_counted_iterator(I i, difference_type_t<I> n);

    // \ref{range.unreachable.sentinels}, unreachable sentinels:
    class unreachable;
    template <Iterator I>
      constexpr bool operator==(const I&, unreachable) noexcept;
    template <Iterator I>
      constexpr bool operator==(unreachable, const I&) noexcept;
    template <Iterator I>
      constexpr bool operator!=(const I&, unreachable) noexcept;
    template <Iterator I>
      constexpr bool operator!=(unreachable, const I&) noexcept;

    // \ref{range.dangling.wrap}, dangling wrapper:
    template <class T> class dangling;
    template <Range R> using safe_iterator_t = @\seebelow@;

    // \ref{range.iterators.stream}, stream iterators:
    template <class T, class charT = char, class traits = char_traits<charT>,
        class Distance = ptrdiff_t>
    class istream_iterator;
    template <class T, class charT, class traits, class Distance>
      bool operator==(const istream_iterator<T, charT, traits, Distance>& x,
              const istream_iterator<T, charT, traits, Distance>& y);
    template <class T, class charT, class traits, class Distance>
      bool operator==(default_sentinel x,
              const istream_iterator<T, charT, traits, Distance>& y);
    template <class T, class charT, class traits, class Distance>
      bool operator==(const istream_iterator<T, charT, traits, Distance>& x,
              default_sentinel y);
    template <class T, class charT, class traits, class Distance>
      bool operator!=(const istream_iterator<T, charT, traits, Distance>& x,
              const istream_iterator<T, charT, traits, Distance>& y);
    template <class T, class charT, class traits, class Distance>
    bool operator!=(default_sentinel x,
              const istream_iterator<T, charT, traits, Distance>& y);
    template <class T, class charT, class traits, class Distance>
      bool operator!=(const istream_iterator<T, charT, traits, Distance>& x,
              default_sentinel y);

    template <class T, class charT = char, class traits = char_traits<charT>>
        class ostream_iterator;

    template <class charT, class traits = char_traits<charT> >
      class istreambuf_iterator;
    template <class charT, class traits>
      bool operator==(const istreambuf_iterator<charT, traits>& a,
              const istreambuf_iterator<charT, traits>& b);
    template <class charT, class traits>
      bool operator==(default_sentinel a,
              const istreambuf_iterator<charT, traits>& b);
    template <class charT, class traits>
      bool operator==(const istreambuf_iterator<charT, traits>& a,
              default_sentinel b);
    template <class charT, class traits>
      bool operator!=(const istreambuf_iterator<charT, traits>& a,
              const istreambuf_iterator<charT, traits>& b);
    template <class charT, class traits>
      bool operator!=(default_sentinel a,
              const istreambuf_iterator<charT, traits>& b);
    template <class charT, class traits>
      bool operator!=(const istreambuf_iterator<charT, traits>& a,
              default_sentinel b);

    template <class charT, class traits = char_traits<charT> >
      class ostreambuf_iterator;
  }

  // \ref{range.iterator.stdtraits}, iterator traits:
  template <@\oldtxt{experimental::}@ranges::Iterator Out>
    struct iterator_traits<Out>;
  template <@\oldtxt{experimental::}@ranges::InputIterator In>
    struct iterator_traits<In>;
  template <@\oldtxt{experimental::}@ranges::InputIterator In>
      requires @\oldtxt{experimental::}@ranges::Sentinel<In, In>
    struct iterator_traits;

  namespace ranges {
    @\newtxt{inline}@ namespace @\newtxt{\unspec}@ {
      // \ref{range.access}, range access:
      @\newtxt{inline}@ constexpr @\unspec@ begin = @\unspec@;
      @\newtxt{inline}@ constexpr @\unspec@ end = @\unspec@;
      @\newtxt{inline}@ constexpr @\unspec@ cbegin = @\unspec@;
      @\newtxt{inline}@ constexpr @\unspec@ cend = @\unspec@;
      @\newtxt{inline}@ constexpr @\unspec@ rbegin = @\unspec@;
      @\newtxt{inline}@ constexpr @\unspec@ rend = @\unspec@;
      @\newtxt{inline}@ constexpr @\unspec@ crbegin = @\unspec@;
      @\newtxt{inline}@ constexpr @\unspec@ crend = @\unspec@;

      // \ref{range.primitives}, range primitives:
      @\newtxt{inline}@ constexpr @\unspec@ size = @\unspec@;
      @\newtxt{inline}@ constexpr @\unspec@ empty = @\unspec@;
      @\newtxt{inline}@ constexpr @\unspec@ data = @\unspec@;
      @\newtxt{inline}@ constexpr @\unspec@ cdata = @\unspec@;
    }

    template <class T>
    using iterator_t = decltype(ranges::begin(declval<T&>()));

    template <class T>
    using sentinel_t = decltype(ranges::end(declval<T&>()));

    template <class>
    constexpr bool disable_sized_range = false;

    template <class T>
    struct enable_view { };

    struct view_base { };

    // \ref{range.requirements}, range requirements:

    // \ref{range.range}, Range:
    template <class T>
    concept @\oldtxt{bool}@ Range = @\seebelow@;

    // \ref{range.sized}, SizedRange:
    template <class T>
    concept @\oldtxt{bool}@ SizedRange = @\seebelow@;

    // \ref{range.view}, View:
    template <class T>
    concept @\oldtxt{bool}@ View = @\seebelow@;

    // \ref{range.common}, \oldtxt{BoundedRange}\newtxt{CommonRange}:
    template <class T>
    concept @\oldtxt{bool}@ @\oldtxt{BoundedRange}\newtxt{CommonRange}@ = @\seebelow@;

    // \ref{range.input}, InputRange:
    template <class T>
    concept @\oldtxt{bool}@ InputRange = @\seebelow@;

    // \ref{range.output}, OutputRange:
    template <class R, class T>
    concept @\oldtxt{bool}@ OutputRange = @\seebelow@;

    // \ref{range.forward}, ForwardRange:
    template <class T>
    concept @\oldtxt{bool}@ ForwardRange = @\seebelow@;

    // \ref{range.bidirectional}, BidirectionalRange:
    template <class T>
    concept @\oldtxt{bool}@ BidirectionalRange = @\seebelow@;

    // \ref{range.random.access}, RandomAccessRange:
    template <class T>
    concept @\oldtxt{bool}@ RandomAccessRange = @\seebelow@;

    // \ref{range.alg.nonmodifying}, non-modifying sequence operations:
    template <InputIterator I, Sentinel<I> S, class Proj = identity,
        IndirectUnaryPredicate<projected<I, Proj>> Pred>
      bool all_of(I first, S last, Pred pred, Proj proj = Proj{});

    template <InputRange Rng, class Proj = identity,
        IndirectUnaryPredicate<projected<iterator_t<Rng>, Proj>> Pred>
      bool all_of(Rng&& rng, Pred pred, Proj proj = Proj{});

    template <InputIterator I, Sentinel<I> S, class Proj = identity,
        IndirectUnaryPredicate<projected<I, Proj>> Pred>
      bool any_of(I first, S last, Pred pred, Proj proj = Proj{});

    template <InputRange Rng, class Proj = identity,
        IndirectUnaryPredicate<projected<iterator_t<Rng>, Proj>> Pred>
      bool any_of(Rng&& rng, Pred pred, Proj proj = Proj{});

    template <InputIterator I, Sentinel<I> S, class Proj = identity,
        IndirectUnaryPredicate<projected<I, Proj>> Pred>
      bool none_of(I first, S last, Pred pred, Proj proj = Proj{});

    template <InputRange Rng, class Proj = identity,
        IndirectUnaryPredicate<projected<iterator_t<Rng>, Proj>> Pred>
      bool none_of(Rng&& rng, Pred pred, Proj proj = Proj{});

    template <InputIterator I, Sentinel<I> S, class Proj = identity,
        IndirectUnaryInvocable<projected<I, Proj>> Fun>
      tagged_pair<tag::in(I), tag::fun(Fun)>
        for_each(I first, S last, Fun f, Proj proj = Proj{});

    template <InputRange Rng, class Proj = identity,
        IndirectUnaryInvocable<projected<iterator_t<Rng>, Proj>> Fun>
      tagged_pair<tag::in(safe_iterator_t<Rng>), tag::fun(Fun)>
        for_each(Rng&& rng, Fun f, Proj proj = Proj{});

    template <InputIterator I, Sentinel<I> S, class T, class Proj = identity>
      requires IndirectRelation<equal_to<>, projected<I, Proj>, const T*>
      I find(I first, S last, const T& value, Proj proj = Proj{});

    template <InputRange Rng, class T, class Proj = identity>
      requires IndirectRelation<equal_to<>, projected<iterator_t<Rng>, Proj>, const T*>
      safe_iterator_t<Rng>
        find(Rng&& rng, const T& value, Proj proj = Proj{});

    template <InputIterator I, Sentinel<I> S, class Proj = identity,
        IndirectUnaryPredicate<projected<I, Proj>> Pred>
      I find_if(I first, S last, Pred pred, Proj proj = Proj{});

    template <InputRange Rng, class Proj = identity,
        IndirectUnaryPredicate<projected<iterator_t<Rng>, Proj>> Pred>
      safe_iterator_t<Rng>
        find_if(Rng&& rng, Pred pred, Proj proj = Proj{});

    template <InputIterator I, Sentinel<I> S, class Proj = identity,
        IndirectUnaryPredicate<projected<I, Proj>> Pred>
      I find_if_not(I first, S last, Pred pred, Proj proj = Proj{});

    template <InputRange Rng, class Proj = identity,
        IndirectUnaryPredicate<projected<iterator_t<Rng>, Proj>> Pred>
      safe_iterator_t<Rng>
        find_if_not(Rng&& rng, Pred pred, Proj proj = Proj{});

    template <ForwardIterator I1, Sentinel<I1> S1, ForwardIterator I2,
        Sentinel<I2> S2, class Proj = identity,
        IndirectRelation<I2, projected<I1, Proj>> Pred = equal_to<>>
      I1
        find_end(I1 first1, S1 last1, I2 first2, S2 last2,
                Pred pred = Pred{}, Proj proj = Proj{});

    template <ForwardRange Rng1, ForwardRange Rng2, class Proj = identity,
        IndirectRelation<iterator_t<Rng2>,
          projected<iterator_t<Rng>, Proj>> Pred = equal_to<>>
      safe_iterator_t<Rng1>
        find_end(Rng1&& rng1, Rng2&& rng2, Pred pred = Pred{}, Proj proj = Proj{});

    template <InputIterator I1, Sentinel<I1> S1, ForwardIterator I2, Sentinel<I2> S2,
        class Proj1 = identity, class Proj2 = identity,
        IndirectRelation<projected<I1, Proj1>, projected<I2, Proj2>> Pred = equal_to<>>
      I1
        find_first_of(I1 first1, S1 last1, I2 first2, S2 last2,
                      Pred pred = Pred{},
                      Proj1 proj1 = Proj1{}, Proj2 proj2 = Proj2{});

    template <InputRange Rng1, ForwardRange Rng2, class Proj1 = identity,
        class Proj2 = identity,
        IndirectRelation<projected<iterator_t<Rng1>, Proj1>,
          projected<iterator_t<Rng2>, Proj2>> Pred = equal_to<>>
      safe_iterator_t<Rng1>
        find_first_of(Rng1&& rng1, Rng2&& rng2,
                      Pred pred = Pred{},
                      Proj1 proj1 = Proj1{}, Proj2 proj2 = Proj2{});

    template <ForwardIterator I, Sentinel<I> S, class Proj = identity,
        IndirectRelation<projected<I, Proj>> Pred = equal_to<>>
      I adjacent_find(I first, S last, Pred pred = Pred{},
                      Proj proj = Proj{});

    template <ForwardRange Rng, class Proj = identity,
        IndirectRelation<projected<iterator_t<Rng>, Proj>> Pred = equal_to<>>
      safe_iterator_t<Rng>
        adjacent_find(Rng&& rng, Pred pred = Pred{}, Proj proj = Proj{});

    template <InputIterator I, Sentinel<I> S, class T, class Proj = identity>
      requires IndirectRelation<equal_to<>, projected<I, Proj>, const T*>
      difference_type_t<I>
        count(I first, S last, const T& value, Proj proj = Proj{});

    template <InputRange Rng, class T, class Proj = identity>
      requires IndirectRelation<equal_to<>, projected<iterator_t<Rng>, Proj>, const T*>
      difference_type_t<iterator_t<Rng>>
        count(Rng&& rng, const T& value, Proj proj = Proj{});

    template <InputIterator I, Sentinel<I> S, class Proj = identity,
        IndirectUnaryPredicate<projected<I, Proj>> Pred>
      difference_type_t<I>
        count_if(I first, S last, Pred pred, Proj proj = Proj{});

    template <InputRange Rng, class Proj = identity,
        IndirectUnaryPredicate<projected<iterator_t<Rng>, Proj>> Pred>
      difference_type_t<iterator_t<Rng>>
        count_if(Rng&& rng, Pred pred, Proj proj = Proj{});

    template <InputIterator I1, Sentinel<I1> S1, InputIterator I2, Sentinel<I2> S2,
        class Proj1 = identity, class Proj2 = identity,
        IndirectRelation<projected<I1, Proj1>, projected<I2, Proj2>> Pred = equal_to<>>
      tagged_pair<tag::in1(I1), tag::in2(I2)>
        mismatch(I1 first1, S1 last1, I2 first2, S2 last2, Pred pred = Pred{},
                Proj1 proj1 = Proj1{}, Proj2 proj2 = Proj2{});

    template <InputRange Rng1, InputRange Rng2,
        class Proj1 = identity, class Proj2 = identity,
        IndirectRelation<projected<iterator_t<Rng1>, Proj1>,
          projected<iterator_t<Rng2>, Proj2>> Pred = equal_to<>>
      tagged_pair<tag::in1(safe_iterator_t<Rng1>),
                  tag::in2(safe_iterator_t<Rng2>)>
        mismatch(Rng1&& rng1, Rng2&& rng2, Pred pred = Pred{},
                Proj1 proj1 = Proj1{}, Proj2 proj2 = Proj2{});

    template <InputIterator I1, Sentinel<I1> S1, InputIterator I2, Sentinel<I2> S2,
        class Pred = equal_to<>, class Proj1 = identity, class Proj2 = identity>
      requires IndirectlyComparable<I1, I2, Pred, Proj1, Proj2>
      bool equal(I1 first1, S1 last1, I2 first2, S2 last2,
                Pred pred = Pred{},
                Proj1 proj1 = Proj1{}, Proj2 proj2 = Proj2{});

    template <InputRange Rng1, InputRange Rng2, class Pred = equal_to<>,
        class Proj1 = identity, class Proj2 = identity>
      requires IndirectlyComparable<iterator_t<Rng1>, iterator_t<Rng2>, Pred, Proj1, Proj2>
      bool equal(Rng1&& rng1, Rng2&& rng2, Pred pred = Pred{},
                Proj1 proj1 = Proj1{}, Proj2 proj2 = Proj2{});

    template <ForwardIterator I1, Sentinel<I1> S1, ForwardIterator I2,
        Sentinel<I2> S2, class Pred = equal_to<>, class Proj1 = identity,
        class Proj2 = identity>
      requires IndirectlyComparable<I1, I2, Pred, Proj1, Proj2>
      bool is_permutation(I1 first1, S1 last1, I2 first2, S2 last2,
                          Pred pred = Pred{},
                          Proj1 proj1 = Proj1{}, Proj2 proj2 = Proj2{});

    template <ForwardRange Rng1, ForwardRange Rng2, class Pred = equal_to<>,
        class Proj1 = identity, class Proj2 = identity>
      requires IndirectlyComparable<iterator_t<Rng1>, iterator_t<Rng2>, Pred, Proj1, Proj2>
      bool is_permutation(Rng1&& rng1, Rng2&& rng2, Pred pred = Pred{},
                          Proj1 proj1 = Proj1{}, Proj2 proj2 = Proj2{});

    template <ForwardIterator I1, Sentinel<I1> S1, ForwardIterator I2,
        Sentinel<I2> S2, class Pred = equal_to<>,
        class Proj1 = identity, class Proj2 = identity>
      requires IndirectlyComparable<I1, I2, Pred, Proj1, Proj2>
      I1 search(I1 first1, S1 last1, I2 first2, S2 last2,
                Pred pred = Pred{},
                Proj1 proj1 = Proj1{}, Proj2 proj2 = Proj2{});

    template <ForwardRange Rng1, ForwardRange Rng2, class Pred = equal_to<>,
        class Proj1 = identity, class Proj2 = identity>
      requires IndirectlyComparable<iterator_t<Rng1>, iterator_t<Rng2>, Pred, Proj1, Proj2>
      safe_iterator_t<Rng1>
        search(Rng1&& rng1, Rng2&& rng2, Pred pred = Pred{},
              Proj1 proj1 = Proj1{}, Proj2 proj2 = Proj2{});

    template <ForwardIterator I, Sentinel<I> S, class T,
        class Pred = equal_to<>, class Proj = identity>
      requires IndirectlyComparable<I, const T*, Pred, Proj>
      I search_n(I first, S last, difference_type_t<I> count,
                const T& value, Pred pred = Pred{},
                Proj proj = Proj{});

    template <ForwardRange Rng, class T, class Pred = equal_to<>,
        class Proj = identity>
      requires IndirectlyComparable<iterator_t<Rng>, const T*, Pred, Proj>
      safe_iterator_t<Rng>
        search_n(Rng&& rng, difference_type_t<iterator_t<Rng>> count,
                const T& value, Pred pred = Pred{}, Proj proj = Proj{});

    // \ref{range.alg.modifying.operations}, modifying sequence operations:
    // \ref{range.alg.copy}, copy:
    template <InputIterator I, Sentinel<I> S, WeaklyIncrementable O>
      requires IndirectlyCopyable<I, O>
      tagged_pair<tag::in(I), tag::out(O)>
        copy(I first, S last, O result);

    template <InputRange Rng, WeaklyIncrementable O>
      requires IndirectlyCopyable<iterator_t<Rng>, O>
      tagged_pair<tag::in(safe_iterator_t<Rng>), tag::out(O)>
        copy(Rng&& rng, O result);

    template <InputIterator I, WeaklyIncrementable O>
      requires IndirectlyCopyable<I, O>
      tagged_pair<tag::in(I), tag::out(O)>
        copy_n(I first, difference_type_t<I> n, O result);

    template <InputIterator I, Sentinel<I> S, WeaklyIncrementable O, class Proj = identity,
        IndirectUnaryPredicate<projected<I, Proj>> Pred>
      requires IndirectlyCopyable<I, O>
      tagged_pair<tag::in(I), tag::out(O)>
        copy_if(I first, S last, O result, Pred pred, Proj proj = Proj{});

    template <InputRange Rng, WeaklyIncrementable O, class Proj = identity,
        IndirectUnaryPredicate<projected<iterator_t<Rng>, Proj>> Pred>
      requires IndirectlyCopyable<iterator_t<Rng>, O>
      tagged_pair<tag::in(safe_iterator_t<Rng>), tag::out(O)>
        copy_if(Rng&& rng, O result, Pred pred, Proj proj = Proj{});

    template <BidirectionalIterator I1, Sentinel<I1> S1, BidirectionalIterator I2>
      requires IndirectlyCopyable<I1, I2>
      tagged_pair<tag::in(I1), tag::out(I2)>
        copy_backward(I1 first, S1 last, I2 result);

    template <BidirectionalRange Rng, BidirectionalIterator I>
      requires IndirectlyCopyable<iterator_t<Rng>, I>
      tagged_pair<tag::in(safe_iterator_t<Rng>), tag::out(I)>
        copy_backward(Rng&& rng, I result);

    // \ref{range.alg.move}, move:
    template <InputIterator I, Sentinel<I> S, WeaklyIncrementable O>
      requires IndirectlyMovable<I, O>
      tagged_pair<tag::in(I), tag::out(O)>
        move(I first, S last, O result);

    template <InputRange Rng, WeaklyIncrementable O>
      requires IndirectlyMovable<iterator_t<Rng>, O>
      tagged_pair<tag::in(safe_iterator_t<Rng>), tag::out(O)>
        move(Rng&& rng, O result);

    template <BidirectionalIterator I1, Sentinel<I1> S1, BidirectionalIterator I2>
      requires IndirectlyMovable<I1, I2>
      tagged_pair<tag::in(I1), tag::out(I2)>
        move_backward(I1 first, S1 last, I2 result);

    template <BidirectionalRange Rng, BidirectionalIterator I>
      requires IndirectlyMovable<iterator_t<Rng>, I>
      tagged_pair<tag::in(safe_iterator_t<Rng>), tag::out(I)>
        move_backward(Rng&& rng, I result);

    template <ForwardIterator I1, Sentinel<I1> S1, ForwardIterator I2, Sentinel<I2> S2>
      requires IndirectlySwappable<I1, I2>
      tagged_pair<tag::in1(I1), tag::in2(I2)>
        swap_ranges(I1 first1, S1 last1, I2 first2, S2 last2);

    template <ForwardRange Rng1, ForwardRange Rng2>
      requires IndirectlySwappable<iterator_t<Rng1>, iterator_t<Rng2>>
      tagged_pair<tag::in1(safe_iterator_t<Rng1>), tag::in2(safe_iterator_t<Rng2>)>
        swap_ranges(Rng1&& rng1, Rng2&& rng2);

    template <InputIterator I, Sentinel<I> S, WeaklyIncrementable O,
        CopyConstructible F, class Proj = identity>
      requires Writable<O, indirect_result@\oldtxt{_of}@_t<F&@\oldtxt{(}\newtxt{, }@projected<I, Proj>@\oldtxt{)}@>>
      tagged_pair<tag::in(I), tag::out(O)>
        transform(I first, S last, O result, F op, Proj proj = Proj{});

    template <InputRange Rng, WeaklyIncrementable O, CopyConstructible F,
        class Proj = identity>
      requires Writable<O, indirect_result@\oldtxt{_of}@_t<F&@\oldtxt{(}\newtxt{,}@
        projected<iterator_t<R>, Proj>@\oldtxt{)}@>>
      tagged_pair<tag::in(safe_iterator_t<Rng>), tag::out(O)>
        transform(Rng&& rng, O result, F op, Proj proj = Proj{});

    template <InputIterator I1, Sentinel<I1> S1, InputIterator I2, Sentinel<I2> S2,
        WeaklyIncrementable O, CopyConstructible F, class Proj1 = identity,
        class Proj2 = identity>
      requires Writable<O, indirect_result@\oldtxt{_of}@_t<F&@\oldtxt{(}\newtxt{, }@projected<I1, Proj1>,
        projected<I2, Proj2>@\oldtxt{)}@>>
      tagged_tuple<tag::in1(I1), tag::in2(I2), tag::out(O)>
        transform(I1 first1, S1 last1, I2 first2, S2 last2, O result,
                F binary_op, Proj1 proj1 = Proj1{}, Proj2 proj2 = Proj2{});

    template <InputRange Rng1, InputRange Rng2, WeaklyIncrementable O,
        CopyConstructible F, class Proj1 = identity, class Proj2 = identity>
      requires Writable<O, indirect_result@\oldtxt{_of}@_t<F&@\oldtxt{(}\newtxt{,}@
        projected<iterator_t<Rng1>, Proj1>, projected<iterator_t<Rng2>, Proj2>@\oldtxt{)}@>>
      tagged_tuple<tag::in1(safe_iterator_t<Rng1>),
                   tag::in2(safe_iterator_t<Rng2>),
                   tag::out(O)>
        transform(Rng1&& rng1, Rng2&& rng2, O result,
                  F binary_op, Proj1 proj1 = Proj1{}, Proj2 proj2 = Proj2{});

    template <InputIterator I, Sentinel<I> S, class T1, class T2, class Proj = identity>
      requires Writable<I, const T2&> &&
        IndirectRelation<equal_to<>, projected<I, Proj>, const T1*>
      I replace(I first, S last, const T1& old_value, const T2& new_value, Proj proj = Proj{});

    template <InputRange Rng, class T1, class T2, class Proj = identity>
      requires Writable<iterator_t<Rng>, const T2&> &&
        IndirectRelation<equal_to<>, projected<iterator_t<Rng>, Proj>, const T1*>
      safe_iterator_t<Rng>
        replace(Rng&& rng, const T1& old_value, const T2& new_value, Proj proj = Proj{});

    template <InputIterator I, Sentinel<I> S, class T, class Proj = identity,
        IndirectUnaryPredicate<projected<I, Proj>> Pred>
      requires Writable<I, const T&>
      I replace_if(I first, S last, Pred pred, const T& new_value, Proj proj = Proj{});

    template <InputRange Rng, class T, class Proj = identity,
        IndirectUnaryPredicate<projected<iterator_t<Rng>, Proj>> Pred>
      requires Writable<iterator_t<Rng>, const T&>
      safe_iterator_t<Rng>
        replace_if(Rng&& rng, Pred pred, const T& new_value, Proj proj = Proj{});

    template <InputIterator I, Sentinel<I> S, class T1, class T2, OutputIterator<const T2&> O,
        class Proj = identity>
      requires IndirectlyCopyable<I, O> &&
        IndirectRelation<equal_to<>, projected<I, Proj>, const T1*>
      tagged_pair<tag::in(I), tag::out(O)>
        replace_copy(I first, S last, O result, const T1& old_value, const T2& new_value,
                    Proj proj = Proj{});

    template <InputRange Rng, class T1, class T2, OutputIterator<const T2&> O,
        class Proj = identity>
      requires IndirectlyCopyable<iterator_t<Rng>, O> &&
        IndirectRelation<equal_to<>, projected<iterator_t<Rng>, Proj>, const T1*>
      tagged_pair<tag::in(safe_iterator_t<Rng>), tag::out(O)>
        replace_copy(Rng&& rng, O result, const T1& old_value, const T2& new_value,
                    Proj proj = Proj{});

    template <InputIterator I, Sentinel<I> S, class T, OutputIterator<const T&> O,
        class Proj = identity, IndirectUnaryPredicate<projected<I, Proj>> Pred>
      requires IndirectlyCopyable<I, O>
      tagged_pair<tag::in(I), tag::out(O)>
        replace_copy_if(I first, S last, O result, Pred pred, const T& new_value,
                        Proj proj = Proj{});

    template <InputRange Rng, class T, OutputIterator<const T&> O, class Proj = identity,
        IndirectUnaryPredicate<projected<iterator_t<Rng>, Proj>> Pred>
      requires IndirectlyCopyable<iterator_t<Rng>, O>
      tagged_pair<tag::in(safe_iterator_t<Rng>), tag::out(O)>
        replace_copy_if(Rng&& rng, O result, Pred pred, const T& new_value,
                        Proj proj = Proj{});

    template <class T, OutputIterator<const T&> O, Sentinel<O> S>
      O fill(O first, S last, const T& value);

    template <class T, OutputRange<const T&> Rng>
      safe_iterator_t<Rng>
        fill(Rng&& rng, const T& value);

    template <class T, OutputIterator<const T&> O>
      O fill_n(O first, difference_type_t<O> n, const T& value);

    template <Iterator O, Sentinel<O> S, CopyConstructible F>
        requires Invocable<F&> && Writable<O, @\oldtxt{result_of_t<F\&()>}\newtxt{invoke_result_t<F\&>}@>
      O generate(O first, S last, F gen);

    template <class Rng, CopyConstructible F>
        requires Invocable<F&> && OutputRange<Rng, @\oldtxt{result_of_t<F\&()>}\newtxt{invoke_result_t<F\&>}@>
      safe_iterator_t<Rng>
        generate(Rng&& rng, F gen);

    template <Iterator O, CopyConstructible F>
        requires Invocable<F&> && Writable<O, @\oldtxt{result_of_t<F\&()>}\newtxt{invoke_result_t<F\&>}@>
      O generate_n(O first, difference_type_t<O> n, F gen);

    template <ForwardIterator I, Sentinel<I> S, class T, class Proj = identity>
      requires Permutable<I> &&
        IndirectRelation<equal_to<>, projected<I, Proj>, const T*>
      I remove(I first, S last, const T& value, Proj proj = Proj{});

    template <ForwardRange Rng, class T, class Proj = identity>
      requires Permutable<iterator_t<Rng>> &&
        IndirectRelation<equal_to<>, projected<iterator_t<Rng>, Proj>, const T*>
      safe_iterator_t<Rng>
        remove(Rng&& rng, const T& value, Proj proj = Proj{});

    template <ForwardIterator I, Sentinel<I> S, class Proj = identity,
        IndirectUnaryPredicate<projected<I, Proj>> Pred>
      requires Permutable<I>
      I remove_if(I first, S last, Pred pred, Proj proj = Proj{});

    template <ForwardRange Rng, class Proj = identity,
        IndirectUnaryPredicate<projected<iterator_t<Rng>, Proj>> Pred>
      requires Permutable<iterator_t<Rng>>
      safe_iterator_t<Rng>
        remove_if(Rng&& rng, Pred pred, Proj proj = Proj{});

    template <InputIterator I, Sentinel<I> S, WeaklyIncrementable O, class T,
        class Proj = identity>
      requires IndirectlyCopyable<I, O> &&
        IndirectRelation<equal_to<>, projected<I, Proj>, const T*>
      tagged_pair<tag::in(I), tag::out(O)>
        remove_copy(I first, S last, O result, const T& value, Proj proj = Proj{});

    template <InputRange Rng, WeaklyIncrementable O, class T, class Proj = identity>
      requires IndirectlyCopyable<iterator_t<Rng>, O> &&
        IndirectRelation<equal_to<>, projected<iterator_t<Rng>, Proj>, const T*>
      tagged_pair<tag::in(safe_iterator_t<Rng>), tag::out(O)>
        remove_copy(Rng&& rng, O result, const T& value, Proj proj = Proj{});

    template <InputIterator I, Sentinel<I> S, WeaklyIncrementable O,
        class Proj = identity, IndirectUnaryPredicate<projected<I, Proj>> Pred>
      requires IndirectlyCopyable<I, O>
      tagged_pair<tag::in(I), tag::out(O)>
        remove_copy_if(I first, S last, O result, Pred pred, Proj proj = Proj{});

    template <InputRange Rng, WeaklyIncrementable O, class Proj = identity,
        IndirectUnaryPredicate<projected<iterator_t<Rng>, Proj>> Pred>
      requires IndirectlyCopyable<iterator_t<Rng>, O>
      tagged_pair<tag::in(safe_iterator_t<Rng>), tag::out(O)>
        remove_copy_if(Rng&& rng, O result, Pred pred, Proj proj = Proj{});

    template <ForwardIterator I, Sentinel<I> S, class Proj = identity,
        IndirectRelation<projected<I, Proj>> R = equal_to<>>
      requires Permutable<I>
      I unique(I first, S last, R comp = R{}, Proj proj = Proj{});

    template <ForwardRange Rng, class Proj = identity,
        IndirectRelation<projected<iterator_t<Rng>, Proj>> R = equal_to<>>
      requires Permutable<iterator_t<Rng>>
      safe_iterator_t<Rng>
        unique(Rng&& rng, R comp = R{}, Proj proj = Proj{});

    template <InputIterator I, Sentinel<I> S, WeaklyIncrementable O,
        class Proj = identity, IndirectRelation<projected<I, Proj>> R = equal_to<>>
      requires IndirectlyCopyable<I, O> &&
        (ForwardIterator<I> ||
        (InputIterator<O> && Same<value_type_t<I>, value_type_t<O>>) ||
        IndirectlyCopyableStorable<I, O>)
      tagged_pair<tag::in(I), tag::out(O)>
        unique_copy(I first, S last, O result, R comp = R{}, Proj proj = Proj{});

    template <InputRange Rng, WeaklyIncrementable O, class Proj = identity,
        IndirectRelation<projected<iterator_t<Rng>, Proj>> R = equal_to<>>
      requires IndirectlyCopyable<iterator_t<Rng>, O> &&
        (ForwardIterator<iterator_t<Rng>> ||
        (InputIterator<O> && Same<value_type_t<iterator_t<Rng>>, value_type_t<O>>) ||
        IndirectlyCopyableStorable<iterator_t<Rng>, O>)
      tagged_pair<tag::in(safe_iterator_t<Rng>), tag::out(O)>
        unique_copy(Rng&& rng, O result, R comp = R{}, Proj proj = Proj{});

    template <BidirectionalIterator I, Sentinel<I> S>
      requires Permutable<I>
      I reverse(I first, S last);

    template <BidirectionalRange Rng>
      requires Permutable<iterator_t<Rng>>
      safe_iterator_t<Rng>
        reverse(Rng&& rng);

    template <BidirectionalIterator I, Sentinel<I> S, WeaklyIncrementable O>
      requires IndirectlyCopyable<I, O>
      tagged_pair<tag::in(I), tag::out(O)> reverse_copy(I first, S last, O result);

    template <BidirectionalRange Rng, WeaklyIncrementable O>
      requires IndirectlyCopyable<iterator_t<Rng>, O>
      tagged_pair<tag::in(safe_iterator_t<Rng>), tag::out(O)>
        reverse_copy(Rng&& rng, O result);

    template <ForwardIterator I, Sentinel<I> S>
      requires Permutable<I>
      tagged_pair<tag::begin(I), tag::end(I)>
        rotate(I first, I middle, S last);

    template <ForwardRange Rng>
      requires Permutable<iterator_t<Rng>>
      tagged_pair<tag::begin(safe_iterator_t<Rng>),
                  tag::end(safe_iterator_t<Rng>)>
        rotate(Rng&& rng, iterator_t<Rng> middle);

    template <ForwardIterator I, Sentinel<I> S, WeaklyIncrementable O>
      requires IndirectlyCopyable<I, O>
      tagged_pair<tag::in(I), tag::out(O)>
        rotate_copy(I first, I middle, S last, O result);

    template <ForwardRange Rng, WeaklyIncrementable O>
      requires IndirectlyCopyable<iterator_t<Rng>, O>
      tagged_pair<tag::in(safe_iterator_t<Rng>), tag::out(O)>
        rotate_copy(Rng&& rng, iterator_t<Rng> middle, O result);

    // \ref{range.alg.random.shuffle}, shuffle:
    template <RandomAccessIterator I, Sentinel<I> S, class Gen>
      requires Permutable<I> &&
        UniformRandom@\oldtxt{Number}\newtxt{Bit}@Generator<remove_reference_t<Gen>> &&
        ConvertibleTo<@\oldtxt{result_of_t<Gen\&()>}\newtxt{invoke_result_t<Gen\&>}@, difference_type_t<I>>
      I shuffle(I first, S last, Gen&& g);

    template <RandomAccessRange Rng, class Gen>
      requires Permutable<I> &&
        UniformRandom@\oldtxt{Number}\newtxt{Bit}@Generator<remove_reference_t<Gen>> &&
        ConvertibleTo<@\oldtxt{result_of_t<Gen\&()>}\newtxt{invoke_result_t<Gen\&>}@, difference_type_t<I>>
      safe_iterator_t<Rng>
        shuffle(Rng&& rng, Gen&& g);

    // \ref{range.alg.partitions}, partitions:
    template <InputIterator I, Sentinel<I> S, class Proj = identity,
        IndirectUnaryPredicate<projected<I, Proj>> Pred>
      bool is_partitioned(I first, S last, Pred pred, Proj proj = Proj{});

    template <InputRange Rng, class Proj = identity,
        IndirectUnaryPredicate<projected<iterator_t<Rng>, Proj>> Pred>
      bool is_partitioned(Rng&& rng, Pred pred, Proj proj = Proj{});

    template <ForwardIterator I, Sentinel<I> S, class Proj = identity,
        IndirectUnaryPredicate<projected<I, Proj>> Pred>
      requires Permutable<I>
      I partition(I first, S last, Pred pred, Proj proj = Proj{});

    template <ForwardRange Rng, class Proj = identity,
        IndirectUnaryPredicate<projected<iterator_t<Rng>, Proj>> Pred>
      requires Permutable<iterator_t<Rng>>
      safe_iterator_t<Rng>
        partition(Rng&& rng, Pred pred, Proj proj = Proj{});

    template <BidirectionalIterator I, Sentinel<I> S, class Proj = identity,
        IndirectUnaryPredicate<projected<I, Proj>> Pred>
      requires Permutable<I>
      I stable_partition(I first, S last, Pred pred, Proj proj = Proj{});

    template <BidirectionalRange Rng, class Proj = identity,
        IndirectUnaryPredicate<projected<iterator_t<Rng>, Proj>> Pred>
      requires Permutable<iterator_t<Rng>>
      safe_iterator_t<Rng>
        stable_partition(Rng&& rng, Pred pred, Proj proj = Proj{});

    template <InputIterator I, Sentinel<I> S, WeaklyIncrementable O1, WeaklyIncrementable O2,
        class Proj = identity, IndirectUnaryPredicate<projected<I, Proj>> Pred>
      requires IndirectlyCopyable<I, O1> && IndirectlyCopyable<I, O2>
      tagged_tuple<tag::in(I), tag::out1(O1), tag::out2(O2)>
        partition_copy(I first, S last, O1 out_true, O2 out_false, Pred pred,
                      Proj proj = Proj{});

    template <InputRange Rng, WeaklyIncrementable O1, WeaklyIncrementable O2,
        class Proj = identity,
        IndirectUnaryPredicate<projected<iterator_t<Rng>, Proj>> Pred>
      requires IndirectlyCopyable<iterator_t<Rng>, O1> &&
        IndirectlyCopyable<iterator_t<Rng>, O2>
      tagged_tuple<tag::in(safe_iterator_t<Rng>), tag::out1(O1), tag::out2(O2)>
        partition_copy(Rng&& rng, O1 out_true, O2 out_false, Pred pred, Proj proj = Proj{});

    template <ForwardIterator I, Sentinel<I> S, class Proj = identity,
        IndirectUnaryPredicate<projected<I, Proj>> Pred>
      I partition_point(I first, S last, Pred pred, Proj proj = Proj{});

    template <ForwardRange Rng, class Proj = identity,
        IndirectUnaryPredicate<projected<iterator_t<Rng>, Proj>> Pred>
      safe_iterator_t<Rng>
        partition_point(Rng&& rng, Pred pred, Proj proj = Proj{});

    // \ref{range.alg.sorting}, sorting and related operations:
    // \ref{range.alg.sort}, sorting:
    template <RandomAccessIterator I, Sentinel<I> S, class Comp = less<>,
        class Proj = identity>
      requires Sortable<I, Comp, Proj>
      I sort(I first, S last, Comp comp = Comp{}, Proj proj = Proj{});

    template <RandomAccessRange Rng, class Comp = less<>, class Proj = identity>
      requires Sortable<iterator_t<Rng>, Comp, Proj>
      safe_iterator_t<Rng>
        sort(Rng&& rng, Comp comp = Comp{}, Proj proj = Proj{});

    template <RandomAccessIterator I, Sentinel<I> S, class Comp = less<>,
        class Proj = identity>
      requires Sortable<I, Comp, Proj>
      I stable_sort(I first, S last, Comp comp = Comp{}, Proj proj = Proj{});

    template <RandomAccessRange Rng, class Comp = less<>, class Proj = identity>
      requires Sortable<iterator_t<Rng>, Comp, Proj>
      safe_iterator_t<Rng>
        stable_sort(Rng&& rng, Comp comp = Comp{}, Proj proj = Proj{});

    template <RandomAccessIterator I, Sentinel<I> S, class Comp = less<>,
        class Proj = identity>
      requires Sortable<I, Comp, Proj>
      I partial_sort(I first, I middle, S last, Comp comp = Comp{}, Proj proj = Proj{});

    template <RandomAccessRange Rng, class Comp = less<>, class Proj = identity>
      requires Sortable<iterator_t<Rng>, Comp, Proj>
      safe_iterator_t<Rng>
        partial_sort(Rng&& rng, iterator_t<Rng> middle, Comp comp = Comp{},
                    Proj proj = Proj{});

    template <InputIterator I1, Sentinel<I1> S1, RandomAccessIterator I2, Sentinel<I2> S2,
        class Comp = less<>, class Proj1 = identity, class Proj2 = identity>
      requires IndirectlyCopyable<I1, I2> && Sortable<I2, Comp, Proj2> &&
          IndirectStrictWeakOrder<Comp, projected<I1, Proj1>, projected<I2, Proj2>>
      I2 partial_sort_copy(I1 first, S1 last, I2 result_first, S2 result_last,
                           Comp comp = Comp{}, Proj1 proj1 = Proj1{}, Proj2 proj2 = Proj2{});

    template <InputRange Rng1, RandomAccessRange Rng2, class Comp = less<>,
        class Proj1 = identity, class Proj2 = identity>
      requires IndirectlyCopyable<iterator_t<Rng1>, iterator_t<Rng2>> &&
          Sortable<iterator_t<Rng2>, Comp, Proj2> &&
          IndirectStrictWeakOrder<Comp, projected<iterator_t<Rng1>, Proj1>,
            projected<iterator_t<Rng2>, Proj2>>
      safe_iterator_t<Rng2>
        partial_sort_copy(Rng1&& rng, Rng2&& result_rng, Comp comp = Comp{},
                          Proj1 proj1 = Proj1{}, Proj2 proj2 = Proj2{});

    template <ForwardIterator I, Sentinel<I> S, class Proj = identity,
        IndirectStrictWeakOrder<projected<I, Proj>> Comp = less<>>
      bool is_sorted(I first, S last, Comp comp = Comp{}, Proj proj = Proj{});

    template <ForwardRange Rng, class Proj = identity,
        IndirectStrictWeakOrder<projected<iterator_t<Rng>, Proj>> Comp = less<>>
      bool is_sorted(Rng&& rng, Comp comp = Comp{}, Proj proj = Proj{});

    template <ForwardIterator I, Sentinel<I> S, class Proj = identity,
        IndirectStrictWeakOrder<projected<I, Proj>> Comp = less<>>
      I is_sorted_until(I first, S last, Comp comp = Comp{}, Proj proj = Proj{});

    template <ForwardRange Rng, class Proj = identity,
        IndirectStrictWeakOrder<projected<iterator_t<Rng>, Proj>> Comp = less<>>
      safe_iterator_t<Rng>
        is_sorted_until(Rng&& rng, Comp comp = Comp{}, Proj proj = Proj{});

    template <RandomAccessIterator I, Sentinel<I> S, class Comp = less<>,
        class Proj = identity>
      requires Sortable<I, Comp, Proj>
      I nth_element(I first, I nth, S last, Comp comp = Comp{}, Proj proj = Proj{});

    template <RandomAccessRange Rng, class Comp = less<>, class Proj = identity>
      requires Sortable<iterator_t<Rng>, Comp, Proj>
      safe_iterator_t<Rng>
        nth_element(Rng&& rng, iterator_t<Rng> nth, Comp comp = Comp{}, Proj proj = Proj{});

    // \ref{range.alg.binary.search}, binary search:
    template <ForwardIterator I, Sentinel<I> S, class T, class Proj = identity,
        IndirectStrictWeakOrder<const T*, projected<I, Proj>> Comp = less<>>
      I lower_bound(I first, S last, const T& value, Comp comp = Comp{},
                    Proj proj = Proj{});

    template <ForwardRange Rng, class T, class Proj = identity,
        IndirectStrictWeakOrder<const T*, projected<iterator_t<Rng>, Proj>> Comp = less<>>
      safe_iterator_t<Rng>
        lower_bound(Rng&& rng, const T& value, Comp comp = Comp{}, Proj proj = Proj{});

    template <ForwardIterator I, Sentinel<I> S, class T, class Proj = identity,
        IndirectStrictWeakOrder<const T*, projected<I, Proj>> Comp = less<>>
      I upper_bound(I first, S last, const T& value, Comp comp = Comp{}, Proj proj = Proj{});

    template <ForwardRange Rng, class T, class Proj = identity,
        IndirectStrictWeakOrder<const T*, projected<iterator_t<Rng>, Proj>> Comp = less<>>
      safe_iterator_t<Rng>
        upper_bound(Rng&& rng, const T& value, Comp comp = Comp{}, Proj proj = Proj{});

    template <ForwardIterator I, Sentinel<I> S, class T, class Proj = identity,
        IndirectStrictWeakOrder<const T*, projected<I, Proj>> Comp = less<>>
      tagged_pair<tag::begin(I), tag::end(I)>
        equal_range(I first, S last, const T& value, Comp comp = Comp{}, Proj proj = Proj{});

    template <ForwardRange Rng, class T, class Proj = identity,
        IndirectStrictWeakOrder<const T*, projected<iterator_t<Rng>, Proj>> Comp = less<>>
      tagged_pair<tag::begin(safe_iterator_t<Rng>),
                  tag::end(safe_iterator_t<Rng>)>
        equal_range(Rng&& rng, const T& value, Comp comp = Comp{}, Proj proj = Proj{});

    template <ForwardIterator I, Sentinel<I> S, class T, class Proj = identity,
        IndirectStrictWeakOrder<const T*, projected<I, Proj>> Comp = less<>>
      bool binary_search(I first, S last, const T& value, Comp comp = Comp{},
                         Proj proj = Proj{});

    template <ForwardRange Rng, class T, class Proj = identity,
        IndirectStrictWeakOrder<const T*, projected<iterator_t<Rng>, Proj>> Comp = less<>>
      bool binary_search(Rng&& rng, const T& value, Comp comp = Comp{},
                         Proj proj = Proj{});

    // \ref{range.alg.merge}, merge:
    template <InputIterator I1, Sentinel<I1> S1, InputIterator I2, Sentinel<I2> S2,
        WeaklyIncrementable O, class Comp = less<>, class Proj1 = identity,
        class Proj2 = identity>
      requires Mergeable<I1, I2, O, Comp, Proj1, Proj2>
      tagged_tuple<tag::in1(I1), tag::in2(I2), tag::out(O)>
        merge(I1 first1, S1 last1, I2 first2, S2 last2, O result,
              Comp comp = Comp{}, Proj1 proj1 = Proj1{}, Proj2 proj2 = Proj2{});

    template <InputRange Rng1, InputRange Rng2, WeaklyIncrementable O, class Comp = less<>,
        class Proj1 = identity, class Proj2 = identity>
      requires Mergeable<iterator_t<Rng1>, iterator_t<Rng2>, O, Comp, Proj1, Proj2>
      tagged_tuple<tag::in1(safe_iterator_t<Rng1>),
                   tag::in2(safe_iterator_t<Rng2>),
                   tag::out(O)>
        merge(Rng1&& rng1, Rng2&& rng2, O result,
              Comp comp = Comp{}, Proj1 proj1 = Proj1{}, Proj2 proj2 = Proj2{});

    template <BidirectionalIterator I, Sentinel<I> S, class Comp = less<>,
        class Proj = identity>
      requires Sortable<I, Comp, Proj>
      I inplace_merge(I first, I middle, S last, Comp comp = Comp{}, Proj proj = Proj{});

    template <BidirectionalRange Rng, class Comp = less<>, class Proj = identity>
      requires Sortable<iterator_t<Rng>, Comp, Proj>
      safe_iterator_t<Rng>
        inplace_merge(Rng&& rng, iterator_t<Rng> middle, Comp comp = Comp{},
                      Proj proj = Proj{});

    // \ref{range.alg.set.operations}, set operations:
    template <InputIterator I1, Sentinel<I1> S1, InputIterator I2, Sentinel<I2> S2,
        class Proj1 = identity, class Proj2 = identity,
        IndirectStrictWeakOrder<projected<I1, Proj1>, projected<I2, Proj2>> Comp = less<>>
      bool includes(I1 first1, S1 last1, I2 first2, S2 last2, Comp comp = Comp{},
                    Proj1 proj1 = Proj1{}, Proj2 proj2 = Proj2{});

    template <InputRange Rng1, InputRange Rng2, class Proj1 = identity,
        class Proj2 = identity,
        IndirectStrictWeakOrder<projected<iterator_t<Rng1>, Proj1>,
          projected<iterator_t<Rng2>, Proj2>> Comp = less<>>
      bool includes(Rng1&& rng1, Rng2&& rng2, Comp comp = Comp{},
                    Proj1 proj1 = Proj1{}, Proj2 proj2 = Proj2{});

    template <InputIterator I1, Sentinel<I1> S1, InputIterator I2, Sentinel<I2> S2,
        WeaklyIncrementable O, class Comp = less<>, class Proj1 = identity, class Proj2 = identity>
      requires Mergeable<I1, I2, O, Comp, Proj1, Proj2>
      tagged_tuple<tag::in1(I1), tag::in2(I2), tag::out(O)>
        set_union(I1 first1, S1 last1, I2 first2, S2 last2, O result, Comp comp = Comp{},
                  Proj1 proj1 = Proj1{}, Proj2 proj2 = Proj2{});

    template <InputRange Rng1, InputRange Rng2, WeaklyIncrementable O,
        class Comp = less<>, class Proj1 = identity, class Proj2 = identity>
      requires Mergeable<iterator_t<Rng1>, iterator_t<Rng2>, O, Comp, Proj1, Proj2>
      tagged_tuple<tag::in1(safe_iterator_t<Rng1>),
                   tag::in2(safe_iterator_t<Rng2>),
                   tag::out(O)>
        set_union(Rng1&& rng1, Rng2&& rng2, O result, Comp comp = Comp{},
                  Proj1 proj1 = Proj1{}, Proj2 proj2 = Proj2{});

    template <InputIterator I1, Sentinel<I1> S1, InputIterator I2, Sentinel<I2> S2,
        WeaklyIncrementable O, class Comp = less<>, class Proj1 = identity, class Proj2 = identity>
      requires Mergeable<I1, I2, O, Comp, Proj1, Proj2>
      O set_intersection(I1 first1, S1 last1, I2 first2, S2 last2, O result,
                         Comp comp = Comp{}, Proj1 proj1 = Proj1{}, Proj2 proj2 = Proj2{});

    template <InputRange Rng1, InputRange Rng2, WeaklyIncrementable O,
        class Comp = less<>, class Proj1 = identity, class Proj2 = identity>
      requires Mergeable<iterator_t<Rng1>, iterator_t<Rng2>, O, Comp, Proj1, Proj2>
      O set_intersection(Rng1&& rng1, Rng2&& rng2, O result,
                         Comp comp = Comp{}, Proj1 proj1 = Proj1{}, Proj2 proj2 = Proj2{});

    template <InputIterator I1, Sentinel<I1> S1, InputIterator I2, Sentinel<I2> S2,
        WeaklyIncrementable O, class Comp = less<>, class Proj1 = identity, class Proj2 = identity>
      requires Mergeable<I1, I2, O, Comp, Proj1, Proj2>
      tagged_pair<tag::in1(I1), tag::out(O)>
        set_difference(I1 first1, S1 last1, I2 first2, S2 last2, O result,
                       Comp comp = Comp{}, Proj1 proj1 = Proj1{}, Proj2 proj2 = Proj2{});

    template <InputRange Rng1, InputRange Rng2, WeaklyIncrementable O,
        class Comp = less<>, class Proj1 = identity, class Proj2 = identity>
      requires Mergeable<iterator_t<Rng1>, iterator_t<Rng2>, O, Comp, Proj1, Proj2>
      tagged_pair<tag::in1(safe_iterator_t<Rng1>), tag::out(O)>
        set_difference(Rng1&& rng1, Rng2&& rng2, O result,
                       Comp comp = Comp{}, Proj1 proj1 = Proj1{}, Proj2 proj2 = Proj2{});

    template <InputIterator I1, Sentinel<I1> S1, InputIterator I2, Sentinel<I2> S2,
        WeaklyIncrementable O, class Comp = less<>, class Proj1 = identity, class Proj2 = identity>
      requires Mergeable<I1, I2, O, Comp, Proj1, Proj2>
      tagged_tuple<tag::in1(I1), tag::in2(I2), tag::out(O)>
        set_symmetric_difference(I1 first1, S1 last1, I2 first2, S2 last2, O result,
                                 Comp comp = Comp{}, Proj1 proj1 = Proj1{},
                                 Proj2 proj2 = Proj2{});

    template <InputRange Rng1, InputRange Rng2, WeaklyIncrementable O,
        class Comp = less<>, class Proj1 = identity, class Proj2 = identity>
      requires Mergeable<iterator_t<Rng1>, iterator_t<Rng2>, O, Comp, Proj1, Proj2>
      tagged_tuple<tag::in1(safe_iterator_t<Rng1>),
                   tag::in2(safe_iterator_t<Rng2>),
                   tag::out(O)>
        set_symmetric_difference(Rng1&& rng1, Rng2&& rng2, O result, Comp comp = Comp{},
                                 Proj1 proj1 = Proj1{}, Proj2 proj2 = Proj2{});

    // \ref{range.alg.heap.operations}, heap operations:
    template <RandomAccessIterator I, Sentinel<I> S, class Comp = less<>,
        class Proj = identity>
      requires Sortable<I, Comp, Proj>
      I push_heap(I first, S last, Comp comp = Comp{}, Proj proj = Proj{});

    template <RandomAccessRange Rng, class Comp = less<>, class Proj = identity>
      requires Sortable<iterator_t<Rng>, Comp, Proj>
      safe_iterator_t<Rng>
        push_heap(Rng&& rng, Comp comp = Comp{}, Proj proj = Proj{});

    template <RandomAccessIterator I, Sentinel<I> S, class Comp = less<>,
        class Proj = identity>
      requires Sortable<I, Comp, Proj>
      I pop_heap(I first, S last, Comp comp = Comp{}, Proj proj = Proj{});

    template <RandomAccessRange Rng, class Comp = less<>, class Proj = identity>
      requires Sortable<iterator_t<Rng>, Comp, Proj>
      safe_iterator_t<Rng>
        pop_heap(Rng&& rng, Comp comp = Comp{}, Proj proj = Proj{});

    template <RandomAccessIterator I, Sentinel<I> S, class Comp = less<>,
        class Proj = identity>
      requires Sortable<I, Comp, Proj>
      I make_heap(I first, S last, Comp comp = Comp{}, Proj proj = Proj{});

    template <RandomAccessRange Rng, class Comp = less<>, class Proj = identity>
      requires Sortable<iterator_t<Rng>, Comp, Proj>
      safe_iterator_t<Rng>
        make_heap(Rng&& rng, Comp comp = Comp{}, Proj proj = Proj{});

    template <RandomAccessIterator I, Sentinel<I> S, class Comp = less<>,
        class Proj = identity>
      requires Sortable<I, Comp, Proj>
      I sort_heap(I first, S last, Comp comp = Comp{}, Proj proj = Proj{});

    template <RandomAccessRange Rng, class Comp = less<>, class Proj = identity>
      requires Sortable<iterator_t<Rng>, Comp, Proj>
      safe_iterator_t<Rng>
        sort_heap(Rng&& rng, Comp comp = Comp{}, Proj proj = Proj{});

    template <RandomAccessIterator I, Sentinel<I> S, class Proj = identity,
        IndirectStrictWeakOrder<projected<I, Proj>> Comp = less<>>
      bool is_heap(I first, S last, Comp comp = Comp{}, Proj proj = Proj{});

    template <RandomAccessRange Rng, class Proj = identity,
        IndirectStrictWeakOrder<projected<iterator_t<Rng>, Proj>> Comp = less<>>
      bool is_heap(Rng&& rng, Comp comp = Comp{}, Proj proj = Proj{});

    template <RandomAccessIterator I, Sentinel<I> S, class Proj = identity,
        IndirectStrictWeakOrder<projected<I, Proj>> Comp = less<>>
      I is_heap_until(I first, S last, Comp comp = Comp{}, Proj proj = Proj{});

    template <RandomAccessRange Rng, class Proj = identity,
        IndirectStrictWeakOrder<projected<iterator_t<Rng>, Proj>> Comp = less<>>
      safe_iterator_t<Rng>
        is_heap_until(Rng&& rng, Comp comp = Comp{}, Proj proj = Proj{});

    // \ref{range.alg.min.max}, minimum and maximum:
    template <class T, class Proj = identity,
        IndirectStrictWeakOrder<projected<const T*, Proj>> Comp = less<>>
      constexpr const T& min(const T& a, const T& b, Comp comp = Comp{}, Proj proj = Proj{});

    template <Copyable T, class Proj = identity,
        IndirectStrictWeakOrder<projected<const T*, Proj>> Comp = less<>>
      constexpr T min(initializer_list<T> t, Comp comp = Comp{}, Proj proj = Proj{});

    template <InputRange Rng, class Proj = identity,
        IndirectStrictWeakOrder<projected<iterator_t<Rng>, Proj>> Comp = less<>>
      requires Copyable<value_type_t<iterator_t<Rng>>>
      value_type_t<iterator_t<Rng>>
        min(Rng&& rng, Comp comp = Comp{}, Proj proj = Proj{});

    template <class T, class Proj = identity,
        IndirectStrictWeakOrder<projected<const T*, Proj>> Comp = less<>>
      constexpr const T& max(const T& a, const T& b, Comp comp = Comp{}, Proj proj = Proj{});

    template <Copyable T, class Proj = identity,
        IndirectStrictWeakOrder<projected<const T*, Proj>> Comp = less<>>
      constexpr T max(initializer_list<T> t, Comp comp = Comp{}, Proj proj = Proj{});

    template <InputRange Rng, class Proj = identity,
        IndirectStrictWeakOrder<projected<iterator_t<Rng>, Proj>> Comp = less<>>
      requires Copyable<value_type_t<iterator_t<Rng>>>
      value_type_t<iterator_t<Rng>>
        max(Rng&& rng, Comp comp = Comp{}, Proj proj = Proj{});

    template <class T, class Proj = identity,
        IndirectStrictWeakOrder<projected<const T*, Proj>> Comp = less<>>
      constexpr tagged_pair<tag::min(const T&), tag::max(const T&)>
        minmax(const T& a, const T& b, Comp comp = Comp{}, Proj proj = Proj{});

    template <Copyable T, class Proj = identity,
        IndirectStrictWeakOrder<projected<const T*, Proj>> Comp = less<>>
      constexpr tagged_pair<tag::min(T), tag::max(T)>
        minmax(initializer_list<T> t, Comp comp = Comp{}, Proj proj = Proj{});

    template <InputRange Rng, class Proj = identity,
        IndirectStrictWeakOrder<projected<iterator_t<Rng>, Proj>> Comp = less<>>
      requires Copyable<value_type_t<iterator_t<Rng>>>
      tagged_pair<tag::min(value_type_t<iterator_t<Rng>>),
                  tag::max(value_type_t<iterator_t<Rng>>)>
        minmax(Rng&& rng, Comp comp = Comp{}, Proj proj = Proj{});

    template <ForwardIterator I, Sentinel<I> S, class Proj = identity,
        IndirectStrictWeakOrder<projected<I, Proj>> Comp = less<>>
      I min_element(I first, S last, Comp comp = Comp{}, Proj proj = Proj{});

    template <ForwardRange Rng, class Proj = identity,
        IndirectStrictWeakOrder<projected<iterator_t<Rng>, Proj>> Comp = less<>>
      safe_iterator_t<Rng>
        min_element(Rng&& rng, Comp comp = Comp{}, Proj proj = Proj{});

    template <ForwardIterator I, Sentinel<I> S, class Proj = identity,
        IndirectStrictWeakOrder<projected<I, Proj>> Comp = less<>>
      I max_element(I first, S last, Comp comp = Comp{}, Proj proj = Proj{});

    template <ForwardRange Rng, class Proj = identity,
        IndirectStrictWeakOrder<projected<iterator_t<Rng>, Proj>> Comp = less<>>
      safe_iterator_t<Rng>
        max_element(Rng&& rng, Comp comp = Comp{}, Proj proj = Proj{});

    template <ForwardIterator I, Sentinel<I> S, class Proj = identity,
        IndirectStrictWeakOrder<projected<I, Proj>> Comp = less<>>
      tagged_pair<tag::min(I), tag::max(I)>
        minmax_element(I first, S last, Comp comp = Comp{}, Proj proj = Proj{});

    template <ForwardRange Rng, class Proj = identity,
        IndirectStrictWeakOrder<projected<iterator_t<Rng>, Proj>> Comp = less<>>
      tagged_pair<tag::min(safe_iterator_t<Rng>),
                  tag::max(safe_iterator_t<Rng>)>
        minmax_element(Rng&& rng, Comp comp = Comp{}, Proj proj = Proj{});

    template <InputIterator I1, Sentinel<I1> S1, InputIterator I2, Sentinel<I2> S2,
        class Proj1 = identity, class Proj2 = identity,
        IndirectStrictWeakOrder<projected<I1, Proj1>, projected<I2, Proj2>> Comp = less<>>
      bool
        lexicographical_compare(I1 first1, S1 last1, I2 first2, S2 last2,
                                Comp comp = Comp{}, Proj1 proj1 = Proj1{}, Proj2 proj2 = Proj2{});

    template <InputRange Rng1, InputRange Rng2, class Proj1 = identity,
        class Proj2 = identity,
        IndirectStrictWeakOrder<projected<iterator_t<Rng1>, Proj1>,
          projected<iterator_t<Rng2>, Proj2>> Comp = less<>>
      bool
        lexicographical_compare(Rng1&& rng1, Rng2&& rng2, Comp comp = Comp{},
                                Proj1 proj1 = Proj1{}, Proj2 proj2 = Proj2{});

    // \ref{range.alg.permutation.generators}, permutations:
    template <BidirectionalIterator I, Sentinel<I> S, class Comp = less<>,
        class Proj = identity>
      requires Sortable<I, Comp, Proj>
      bool next_permutation(I first, S last, Comp comp = Comp{}, Proj proj = Proj{});

    template <BidirectionalRange Rng, class Comp = less<>,
        class Proj = identity>
      requires Sortable<iterator_t<Rng>, Comp, Proj>
      bool next_permutation(Rng&& rng, Comp comp = Comp{}, Proj proj = Proj{});

    template <BidirectionalIterator I, Sentinel<I> S, class Comp = less<>,
        class Proj = identity>
      requires Sortable<I, Comp, Proj>
      bool prev_permutation(I first, S last, Comp comp = Comp{}, Proj proj = Proj{});

    template <BidirectionalRange Rng, class Comp = less<>,
        class Proj = identity>
      requires Sortable<iterator_t<Rng>, Comp, Proj>
      bool prev_permutation(Rng&& rng, Comp comp = Comp{}, Proj proj = Proj{});
  }
}@\oldtxt{\}\}}@
\end{codeblock}

%!TEX root = P0896.tex

\setcounter{table}{72}
\rSec0[iterators]{Iterators library}

\rSec1[iterators.general]{General}

\pnum
This Clause describes components that \Cpp{} programs may use to perform
iterations over containers\iref{containers},
streams\cxxiref{iostream.format},
\removed{and} stream buffers\cxxiref{stream.buffers}\added{,
and other ranges\iref{range}}.

\pnum
The following subclauses describe
iterator requirements, and
components for
iterator primitives,
predefined iterators,
and stream iterators,
as summarized in \tref{iterators.lib.summary}.

\begin{libsumtab}{Iterators library summary}{tab:iterators.lib.summary}
\ref{iterator.requirements}     & \changed{R}{Iterator r}equirements      & \added{\tcode{<iterator>}}   \\
\added{\oldtxt{\ref{indirectcallable}}}  & \added{\oldtxt{Indirect callable requirements}}  &                              \\
\added{\oldtxt{\ref{commonalgoreq}}}     & \added{\oldtxt{Common algorithm requirements}}   &                              \\
\ref{iterator.primitives}       & Iterator primitives                     & \removed{\tcode{<iterator>}} \\
\ref{predef.iterators}          & Predefined iterators                    &                              \\
\ref{stream.iterators}          & Stream iterators                        &                              \\
\end{libsumtab}

\ednote{Move [iterator.synopsis] here immediately after [iterators.general]
and modify as follows:}

\rSec1[iterator.synopsis]{Header \tcode{<iterator>} synopsis}

\indexlibrary{\idxhdr{iterator}}%
\begin{codeblock}
@\added{\#include <concepts>}@

namespace std {
\end{codeblock}\begin{addedblock}\begin{codeblock}
  @\newtxt{template<class T> using \placeholder{with-reference} // \expos}@
    @\newtxt{= T\&;}@
  @\newtxt{template<class T> concept \placeholder{can-reference} // \expos}@
    @\newtxt{= requires \{ typename \placeholdernc{with-reference}<T>; \};}@
  template<class T> concept @\placeholder{dereferenceable}@ // \expos
    = requires(T& t) {
      @\oldtxt{\{}@ *t @\oldtxt{\} -> auto\&\&}@; // not required to be equality-preserving
      @\newtxt{requires \placeholdernc{can-reference}<decltype(*t)>;}@
    };

  // \ref{iterator.assoc.types}, associated types
  // \ref{incrementable.traits}, incrementable traits
  template<class> struct incrementable_traits;
  template<class T>
    using iter_difference_t = @\seebelownc@;

  // \ref{readable.traits}, readable traits
  template<class> struct readable_traits;
  template<class T>
    using iter_value_t = @\seebelownc@;
\end{codeblock}\end{addedblock}\begin{codeblock}

  // \changed{\ref{iterator.primitives}, primitives}{\ref{iterator.traits}, Iterator traits}
  template<class @\changed{Iterator}{I}@> struct iterator_traits;
  template<class T> struct iterator_traits<T*>;
\end{codeblock}\begin{addedblock}\begin{codeblock}

  template<@\placeholder{dereferenceable}@ T>
    using iter_reference_t = decltype(*declval<T&>());

  namespace ranges {
    // \ref{iterator.custpoints}, customization points
    inline namespace @\unspec@ {
      // \ref{iterator.custpoints.iter_move}, iter_move
      inline constexpr @\unspec@ iter_move = @\unspecnc@;

      // \ref{iterator.custpoints.iter_swap}, iter_swap
      inline constexpr @\unspec@ iter_swap = @\unspecnc@;
    }
  }

  template<@\placeholder{dereferenceable}@ T>
    requires requires (T& t) {
      @\oldtxt{\{}@ ranges::iter_move(t) @\oldtxt{\} -> auto \&\&}@;
      @\newtxt{requires \placeholder{can-reference}<decltype(ranges::iter_move(t))>;}@
    }
      using iter_rvalue_reference_t
        = decltype(ranges::iter_move(declval<T&>()));

  // \ref{iterator.concepts}, iterator concepts
  // \ref{iterator.concept.readable}, Readable
  template<class In>
    concept Readable = @\seebelownc@;

  template<Readable T>
    using iter_common_reference_t =
      common_reference_t<iter_reference_t<T>, iter_value_t<T>&>;

  // \ref{iterator.concept.writable}, Writable
  template<class Out, class T>
    concept Writable = @\seebelownc@;

  // \ref{iterator.concept.weaklyincrementable}, WeaklyIncrementable
  template<class I>
    concept WeaklyIncrementable = @\seebelownc@;

  // \ref{iterator.concept.incrementable}, Incrementable
  template<class I>
    concept Incrementable = @\seebelownc@;

  // \ref{iterator.concept.iterator}, Iterator
  template<class I>
    concept Iterator = @\seebelownc@;

  // \ref{iterator.concept.incrementable}, Sentinel
  template<class S, class I>
    concept Sentinel = @\seebelownc@;

  // \ref{iterator.concept.sizedsentinel}, SizedSentinel
  template<class S, class I>
    inline constexpr bool disable_sized_sentinel = false;

  template<class S, class I>
    concept SizedSentinel = @\seebelownc@;

  // \ref{iterator.concept.input}, InputIterator
  template<class I>
    concept InputIterator = @\seebelownc@;

  // \ref{iterator.concept.output}, OutputIterator
  template<class I@\newtxt{, class T}@>
    concept OutputIterator = @\seebelownc@;

  // \ref{iterator.concept.forward}, ForwardIterator
  template<class I>
    concept ForwardIterator = @\seebelownc@;

  // \ref{iterator.concept.bidirectional}, BidirectionalIterator
  template<class I>
    concept BidirectionalIterator = @\seebelownc@;

  // \ref{iterator.concept.random.access}, RandomAccessIterator
  template<class I>
    concept RandomAccessIterator = @\seebelownc@;

  // \ref{iterator.concept.contiguous}, ContiguousIterator
  template<class I>
    concept ContiguousIterator = @\seebelownc@;

  // \ref{indirectcallable}, indirect callable requirements
  // \ref{indirectcallable.indirectinvocable}, indirect callables
  template<class F, class I>
    concept IndirectUnaryInvocable = @\seebelownc@;

  template<class F, class I>
    concept IndirectRegularUnaryInvocable = @\seebelownc@;

  template<class F, class I>
    concept IndirectUnaryPredicate = @\seebelownc@;

  template<class F, class I1, class I2 = I1>
    concept IndirectRelation = @\seebelownc@;

  template<class F, class I1, class I2 = I1>
    concept IndirectStrictWeakOrder = @\seebelownc@;

  template<class F, class... Is>
    requires (Readable<Is> && ...) && Invocable<F, iter_reference_t<Is>...>
      using indirect_result_t = invoke_result_t<F, iter_reference_t<Is>...>;

  // \ref{projected}, projected
  template<Readable I, IndirectRegularUnaryInvocable<I> Proj>
    struct projected;

  template<WeaklyIncrementable I, class Proj>
    struct incrementable_traits<projected<I, Proj>>;

  // \ref{commonalgoreq}, common algorithm requirements
  // \ref{commonalgoreq.indirectlymovable} IndirectlyMovable
  template<class In, class Out>
    concept IndirectlyMovable = @\seebelownc@;

  template<class In, class Out>
    concept IndirectlyMovableStorable = @\seebelownc@;

  // \ref{commonalgoreq.indirectlycopyable} IndirectlyCopyable
  template<class In, class Out>
    concept IndirectlyCopyable = @\seebelownc@;

  template<class In, class Out>
    concept IndirectlyCopyableStorable = @\seebelownc@;

  // \ref{commonalgoreq.indirectlyswappable} IndirectlySwappable
  template<class I1, class I2 = I1>
    concept IndirectlySwappable = @\seebelownc@;

  // \ref{commonalgoreq.indirectlycomparable} IndirectlyComparable
  template<class I1, class I2, class R, class P1 = identity, class P2 = identity>
    concept IndirectlyComparable = @\seebelownc@;

  // \ref{commonalgoreq.permutable} Permutable
  template<class I>
    concept Permutable = @\seebelownc@;

  // \ref{commonalgoreq.mergeable} Mergeable
  template<class I1, class I2, class Out,
      class R = ranges::less<>, class P1 = identity, class P2 = identity>
    concept Mergeable = @\seebelownc@;

  template<class I, class R = ranges::less<>, class P = identity>
    concept Sortable = @\seebelownc@;

  // \ref{iterator.primitives}, primitives
  // \ref{std.iterator.tags}, iterator tags
\end{codeblock}\end{addedblock}\begin{codeblock}
  struct input_iterator_tag { };
  struct output_iterator_tag { };
  struct forward_iterator_tag: public input_iterator_tag { };
  struct bidirectional_iterator_tag: public forward_iterator_tag { };
  struct random_access_iterator_tag: public bidirectional_iterator_tag { };
  @\added{struct contiguous_iterator_tag: public random_access_iterator_tag \{ \};}@

  // \ref{iterator.operations}, iterator operations
  template<class InputIterator, class Distance>
    constexpr void
      advance(InputIterator& i, Distance n);
  template<class InputIterator>
    constexpr typename iterator_traits<InputIterator>::difference_type
      distance(InputIterator first, InputIterator last);
  template<class InputIterator>
    constexpr InputIterator
      next(InputIterator x,
           typename iterator_traits<InputIterator>::difference_type n = 1);
  template<class BidirectionalIterator>
    constexpr BidirectionalIterator
      prev(BidirectionalIterator x,
           typename iterator_traits<BidirectionalIterator>::difference_type n = 1);

\end{codeblock}\begin{addedblock}\begin{codeblock}
  // \ref{range.iterator.operations}, range iterator operations
  namespace ranges {
    // \ref{range.iterator.operations.advance}, \tcode{ranges::advance}
    template<Iterator I>
      constexpr void advance(I& i, iter_difference_t<I> n);
    template<Iterator I, Sentinel<I> S>
      constexpr void advance(I& i, S bound);
    template<Iterator I, Sentinel<I> S>
      constexpr iter_difference_t<I> advance(I& i, iter_difference_t<I> n, S bound);

    // \ref{range.iterator.operations.distance}, \tcode{ranges::distance}
    template<Iterator I, Sentinel<I> S>
      constexpr iter_difference_t<I> distance(I first, S last);
    template<Range R>
      constexpr iter_difference_t<iterator_t<R>> distance(R&& r);

    // \ref{range.iterator.operations.next}, \tcode{ranges::next}
    template<Iterator I>
      constexpr I next(I x);
    template<Iterator I>
      constexpr I next(I x, iter_difference_t<I> n);
    template<Iterator I, Sentinel<I> S>
      constexpr I next(I x, S bound);
    template<Iterator I, Sentinel<I> S>
      constexpr I next(I x, iter_difference_t<I> n, S bound);

    // \ref{range.iterator.operations.prev}, \tcode{ranges::prev}
    template<BidirectionalIterator I>
      constexpr I prev(I x);
    template<BidirectionalIterator I>
      constexpr I prev(I x, iter_difference_t<I> n);
    template<BidirectionalIterator I>
      constexpr I prev(I x, iter_difference_t<I> n, I bound);
  }
\end{codeblock}\end{addedblock}\begin{codeblock}

  // \ref{predef.iterators}, predefined iterators \added{and sentinels}
  @\added{// \ref{reverse.iterators}, reverse iterators}@
  template<class Iterator> class reverse_iterator;

  template<class Iterator1, class Iterator2>
    constexpr bool operator==(
      const reverse_iterator<Iterator1>& x,
      const reverse_iterator<Iterator2>& y);
  template<class Iterator1, class Iterator2>
    constexpr bool operator!=(
      const reverse_iterator<Iterator1>& x,
      const reverse_iterator<Iterator2>& y);
  template<class Iterator1, class Iterator2>
    constexpr bool operator<(
      const reverse_iterator<Iterator1>& x,
      const reverse_iterator<Iterator2>& y);
  template<class Iterator1, class Iterator2>
    constexpr bool operator>(
      const reverse_iterator<Iterator1>& x,
      const reverse_iterator<Iterator2>& y);
  template<class Iterator1, class Iterator2>
    constexpr bool operator<=(
      const reverse_iterator<Iterator1>& x,
      const reverse_iterator<Iterator2>& y);
  template<class Iterator1, class Iterator2>
    constexpr bool operator>=(
      const reverse_iterator<Iterator1>& x,
      const reverse_iterator<Iterator2>& y);

  template<class Iterator1, class Iterator2>
    constexpr auto operator-(
      const reverse_iterator<Iterator1>& x,
      const reverse_iterator<Iterator2>& y) -> decltype(y.base() - x.base());
  template<class Iterator>
    constexpr reverse_iterator<Iterator>
      operator+(
    typename reverse_iterator<Iterator>::difference_type n,
    const reverse_iterator<Iterator>& x);

  template<class Iterator>
    constexpr reverse_iterator<Iterator> make_reverse_iterator(Iterator i);

  @\added{// \ref{insert.iterators}, insert iterators}@
  template<class Container> class back_insert_iterator;
  template<class Container>
    back_insert_iterator<Container> back_inserter(Container& x);

  template<class Container> class front_insert_iterator;
  template<class Container>
    front_insert_iterator<Container> front_inserter(Container& x);

  template<class Container> class insert_iterator;
  template<class Container>
    insert_iterator<Container> inserter(Container& x, @\changed{typename Container::iterator}{iterator_t<Container>}@ i);

  @\added{// \ref{move.iterators}, move iterators and sentinels}@
  template<class Iterator> class move_iterator;
  template<class Iterator1, class Iterator2>
    constexpr bool operator==(
      const move_iterator<Iterator1>& x, const move_iterator<Iterator2>& y);
  template<class Iterator1, class Iterator2>
    constexpr bool operator!=(
      const move_iterator<Iterator1>& x, const move_iterator<Iterator2>& y);
  template<class Iterator1, class Iterator2>
    constexpr bool operator<(
      const move_iterator<Iterator1>& x, const move_iterator<Iterator2>& y);
  template<class Iterator1, class Iterator2>
    constexpr bool operator<=(
      const move_iterator<Iterator1>& x, const move_iterator<Iterator2>& y);
  template<class Iterator1, class Iterator2>
    constexpr bool operator>(
      const move_iterator<Iterator1>& x, const move_iterator<Iterator2>& y);
  template<class Iterator1, class Iterator2>
    constexpr bool operator>=(
      const move_iterator<Iterator1>& x, const move_iterator<Iterator2>& y);

  template<class Iterator1, class Iterator2>
    constexpr auto operator-(
    const move_iterator<Iterator1>& x,
    const move_iterator<Iterator2>& y) -> decltype(x.base() - y.base());
  template<class Iterator>
    constexpr move_iterator<Iterator> operator+(
      typename move_iterator<Iterator>::difference_type n, const move_iterator<Iterator>& x);
  template<class Iterator>
    constexpr move_iterator<Iterator> make_move_iterator(Iterator i);

\end{codeblock}\begin{addedblock}\begin{codeblock}
  template<Semiregular S> class move_sentinel;

  // \ref{iterators.common}, common iterators
  template<Iterator I, Sentinel<I> S>
    requires @\newtxt{(}@!Same<I, S>@\newtxt{)}@
      class common_iterator;

  template<Readable I, class S>
    struct readable_traits<common_iterator<I, S>>;

  template<InputIterator I, class S>
    struct iterator_traits<common_iterator<I, S>>;

  // \ref{default.sentinels}, default sentinels
  class default_sentinel;

  // \ref{iterators.counted}, counted iterators
  template<Iterator I> class counted_iterator;

  template<Readable I>
    struct readable_traits<counted_iterator<I>>;

  template<InputIterator I>
    struct iterator_traits<counted_iterator<I>>;

  // \ref{unreachable.sentinels}, unreachable sentinels
  class unreachable;
\end{codeblock}\end{addedblock}\begin{codeblock}

  // \ref{stream.iterators}, stream iterators
  [...]
}
\end{codeblock}

\rSec1[iterator.requirements]{Iterator requirements}

\rSec2[iterator.requirements.general]{In general}

\pnum
\indextext{requirements!iterator}%
Iterators are a generalization of pointers that allow a \Cpp{} program
to work with different data structures
(\added{for example,} containers \added{and ranges}) in a uniform manner.
To be able to construct template algorithms that work correctly and efficiently
on different types of data structures, the library formalizes  not just
the interfaces but also the semantics and complexity assumptions of iterators.
An input iterator
\tcode{i}
supports the expression
\tcode{*i},
resulting in a value of some object type
\tcode{T},
called the
\term{value type}
of the iterator.
% Think about making this "non-empty set of expression type-and-value-categories"
% what's here now is easy to understand and incorrect.
An output iterator \tcode{i} has a non-empty set of types that are
\defn{writable} to the iterator;
for each such type \tcode{T}, the expression \tcode{*i = o}
is valid where \tcode{o} is a value of type \tcode{T}.
\removed{An iterator
\tcode{i}
for which the expression
\tcode{(*i).m}
is well-defined supports the expression
\tcode{i->m}
with the same semantics as
\tcode{(*i).m}.}
For every iterator type
\tcode{X}
\removed{for which
equality is defined}, there is a corresponding signed integer type called the
\term{difference type}
of the iterator.

\pnum
Since iterators are an abstraction of pointers, their semantics is
a generalization of most of the semantics of pointers in \Cpp{}.
This ensures that every
function template
that takes iterators
works as well with regular pointers.
This document defines
\changed{five}{six} categories of iterators, according to the operations
defined on them:
\term{input iterators},
\term{output iterators},
\term{forward iterators},
\term{bidirectional iterators},
\term{random access iterators},
and
\added{\term{contiguous iterators}},
as shown in \tref{iterators.relations}.

\begin{floattable}{Relations among iterator categories}{tab:iterators.relations}
{lllll}
\topline
              \added{\textbf{Contiguous}}    & \added{$\rightarrow$} \textbf{Random Access} &
$\rightarrow$ \textbf{Bidirectional} & $\rightarrow$ \textbf{Forward}       &
$\rightarrow$ \textbf{Input}                                                \\
   &   &   &   &   $\rightarrow$ \textbf{Output}                            \\
\end{floattable}

\begin{addedblock}
\pnum
The six categories of iterators correspond to the iterator concepts
\libconcept{Input\-Iterator}\iref{iterator.concept.input},
\libconcept{Output\-Iterator}\iref{iterator.concept.output},
\libconcept{Forward\-Iterator}\iref{iterator.concept.forward},
\libconcept{Bidirectional\-Iterator}\iref{iterator.concept.bidirectional}
\libconcept{RandomAccess\-Iterator}\iref{iterator.concept.random.access}, and
\libconcept{Contiguous\-Iterator}\iref{iterator.concept.contiguous}, respectively.
The generic term \defn{iterator} refers to any type that models the
\libconcept{Iterator} concept\iref{iterator.concept.iterator}.
\end{addedblock}

\pnum
Forward iterators satisfy all the requirements of input
iterators and can be used whenever
an input iterator is specified;
Bidirectional iterators also satisfy all the requirements of
forward iterators and can be used whenever a forward iterator is specified;
Random access iterators also satisfy all the requirements of bidirectional
iterators and can be used whenever a bidirectional iterator is specified;
\added{Contiguous iterators also satisfy all the requirements of random access
iterators and can be used whenever a random access iterator is specified}.

\pnum
Iterators that further satisfy the requirements of output iterators are
called \defnx{mutable iterators}{mutable iterator}. Nonmutable iterators are referred to
as \defnx{constant iterators}{constant iterator}.

\pnum
In addition to the requirements in this subclause,
the nested \grammarterm{typedef-name}{s} specified in \ref{iterator.traits}
shall be provided for the iterator type.
\begin{note} Either the iterator type must provide the \grammarterm{typedef-name}{s} directly
(in which case \tcode{iterator_traits} pick\added{s} them up automatically), or
an \tcode{iterator_traits} specialization must provide them. \end{note}

\begin{removedblock}
\pnum
Iterators that further satisfy the requirement that,
for integral values \tcode{n} and
dereferenceable iterator values \tcode{a} and \tcode{(a + n)},
\tcode{*(a + n)} is equivalent to \tcode{*(addressof(*a) + n)},
are called \defn{contiguous iterators}.
\begin{note}
For example, the type ``pointer to \tcode{int}'' is a contiguous iterator,
but \tcode{reverse_iterator<int *>} is not.
For a valid iterator range $[$\tcode{a}$, $\tcode{b}$)$ with dereferenceable \tcode{a},
the corresponding range denoted by pointers is
$[$\tcode{addressof(*a)}$, $\tcode{addressof(*a) + (b - a)}$)$;
\tcode{b} might not be dereferenceable.
\end{note}
\end{removedblock}

\pnum
Just as a regular pointer to an array guarantees that there is a pointer value pointing past the last element
of the array, so for any iterator type there is an iterator value that points past the last element of a
corresponding sequence.
These values are called
\term{past-the-end}
values.
Values of an iterator
\tcode{i}
for which the expression
\tcode{*i}
is defined are called
\term{dereferenceable}.
The library never assumes that past-the-end values are dereferenceable.
Iterators can also have singular values that are not associated with any
sequence.
\begin{example}
After the declaration of an uninitialized pointer
\tcode{x}
(as with
\tcode{int* x;}),
\tcode{x}
must always be assumed to have a singular value of a pointer.
\end{example}
Results of most expressions are undefined for singular values;
the only exceptions are destroying an iterator that holds a singular value,
the assignment of a non-singular value to
an iterator that holds a singular value, and, for iterators that satisfy the
\oldconcept{DefaultConstructible} requirements, using a value-initialized iterator
as the source of a copy or move operation. \begin{note} This guarantee is not
offered for default-initialization, although the distinction only matters for types
with trivial default constructors such as pointers or aggregates holding pointers.
\end{note}
In these cases the singular
value is overwritten the same way as any other value.
Dereferenceable
values are always non-singular.

\begin{removedblock}
\pnum
An iterator
\tcode{j}
is called
\term{reachable}
from an iterator
\tcode{i}
if and only if there is a finite sequence of applications of
the expression
\tcode{++i}
that makes
\tcode{i == j}.
If
\tcode{j}
is reachable from
\tcode{i},
they refer to elements of the same sequence.

\pnum
Most of the library's algorithmic templates that operate on data structures have interfaces that use ranges.
A
\term{range}
is a pair of iterators that designate the beginning and end of the computation.
A range \range{i}{i}
is an empty range;
in general, a range \range{i}{j}
refers to the elements in the data structure starting with the element
pointed to by
\tcode{i}
and up to but not including the element pointed to by
\tcode{j}.
Range \range{i}{j}
is valid if and only if
\tcode{j}
is reachable from
\tcode{i}.
The result of the application of functions in the library to invalid ranges is
undefined.
\end{removedblock}

\begin{addedblock}
\pnum
Most of the library's algorithmic templates that operate on data structures have
interfaces that use ranges. A \term{range} is an iterator and a \term{sentinel}
that designate the beginning and end of the computation, or an iterator and a
count that designate the beginning and the number of elements to which the
computation is to be applied.\footnote{The sentinel denoting the end of a range
may have the same type as the iterator denoting the beginning of the range, or a
different type.}

\pnum
An iterator and a sentinel denoting a range are comparable.
A range \range{i}{s}
is empty if \tcode{i == s};
otherwise, \range{i}{s}
refers to the elements in the data structure starting with the element
pointed to by
\tcode{i}
and up to but not including the element\newtxt{, if any,} pointed to by
the first iterator \tcode{j} such that \tcode{j == s}.

\pnum
A sentinel \tcode{s} is called \term{reachable} from an iterator \tcode{i} if
and only if there is a finite sequence of applications of the expression
\tcode{++i} that makes \tcode{i == s}. If \tcode{s} is reachable from \tcode{i},
\range{i}{s} denotes a \newtxt{valid} range.

\pnum
A counted range \range{i}{n} is empty if \tcode{n == 0}; otherwise, \range{i}{n}
refers to the \tcode{n} elements in the data structure starting with the element
pointed to by \tcode{i} and up to but not including the element\newtxt{, if any,} pointed to by
the result of incrementing \tcode{i} \tcode{n} times. \newtxt{A counted range
\range{i}{n} is valid if and only if \tcode{n == 0}; or \tcode{n} is positive,
\tcode{i} is dereferenceable, and \range{++i}{-{-}n} is valid.}

\pnum
\oldtxt{A range \range{i}{s} is valid if and only if \tcode{s} is reachable from
\tcode{i}. A counted range \range{i}{n} is valid if and only if \tcode{n == 0};
or \tcode{n} is positive, \tcode{i} is dereferenceable, and \range{++i}{-{-}n}
is valid.} The result of the application of \newtxt{library} functions \oldtxt{in the library} to invalid
ranges is undefined.
\end{addedblock}

\pnum
All the categories of iterators require only those functions
that are realizable for a given category in constant time (amortized).
Therefore, requirement tables \added{and concept definitions} for the iterators
do not \removed{have a} \added{specify} complexity \removed{column}.

\pnum
Destruction of a\oldoldtxt{n} \newnewtxt{non-forward} iterator \added{\oldtxt{whose category is weaker than forward}}
may invalidate pointers and references previously obtained from that iterator.

\pnum
An
\term{invalid}
iterator is an iterator that may be singular.\footnote{This definition applies
to pointers, since pointers are iterators. The effect of dereferencing
an iterator that has been invalidated is undefined.}

\pnum
\indextext{iterator!constexpr}%
Iterators are called \defn{constexpr iterators}
if all operations provided to satisfy iterator category operations
are constexpr functions, except for
\begin{itemize}
\item \tcode{swap},
\item a pseudo-destructor call\cxxiref{expr.pseudo}, and
\item the construction of an iterator with a singular value.
\end{itemize}
\begin{note}
For example, the types ``pointer to \tcode{int}'' and
\tcode{reverse_iterator<int*>} are constexpr iterators.
\end{note}

\begin{removedblock}
\pnum
In the following sections,
\tcode{a}
and
\tcode{b}
denote values of type
\tcode{X} or \tcode{const X},
\tcode{difference_type} and \tcode{reference} refer to the
types \tcode{iterator_traits<X>::difference_type} and
\tcode{iterator_traits<X>::reference}, respectively,
\tcode{n}
denotes a value of
\tcode{difference_type},
\tcode{u},
\tcode{tmp},
and
\tcode{m}
denote identifiers,
\tcode{r}
denotes a value of
\tcode{X\&},
\tcode{t}
denotes a value of value type
\tcode{T},
\tcode{o}
denotes a value of some type that is writable to the output iterator.
\begin{note} For an iterator type \tcode{X} there must be an instantiation
of \tcode{iterator_traits<X>}\iref{iterator.traits}. \end{note}
\end{removedblock}

\begin{addedblock}
\rSec2[iterator.assoc.types]{Associated types}
\rSec3[incrementable.traits]{Incrementable traits}

\pnum
To implement algorithms only in terms of incrementable types,
it is often necessary to determine the difference type that
corresponds to a particular incrementable type. Accordingly,
it is required that if \tcode{WI} is the name of a type that models  the
\tcode{WeaklyIncrementable} concept\iref{iterator.concept.weaklyincrementable},
the type

\begin{codeblock}
iter_difference_t<WI>
\end{codeblock}

be defined as the incrementable type's difference type.

{\color{oldclr}
\pnum
\tcode{iter_difference_t} is implemented as if:
} %% \color{oldclr}

\indexlibrary{\idxcode{iter_difference_t}}%
\indexlibrary{\idxcode{incrementable_traits}}%
\begin{codeblock}
namespace std {
  template<class> struct incrementable_traits { };

  template<class T>
    requires is_object_v<T>
  struct incrementable_traits<T*> {
    using difference_type = ptrdiff_t;
  };

  template<class I>
  struct incrementable_traits<const I>
    : incrementable_traits<@\oldtxt{decay_t}\newtxt{remove_const_t}@<I>> { };

  template<class T>
    requires requires { typename T::difference_type; }
  struct incrementable_traits<T> {
    using difference_type = typename T::difference_type;
  };

  template<class T>
    requires @\newtxt{(}@!requires { typename T::difference_type; }@\newtxt{)}@ &&
      requires(const T& a, const T& b) { { a - b } -> Integral; }
  struct incrementable_traits<T> {
    using difference_type = make_signed_t<decltype(declval<T>() - declval<T>())>;
  };

  template<class T>
    using iter_difference_t = @\seebelownc@;
}
\end{codeblock}

\pnum
If \tcode{iterator_traits<I>} is a program-defined specialization,
then \tcode{iter_difference_t<I>} denotes
\tcode{iterator_traits<I>::difference_type}; otherwise, it denotes
\tcode{incrementable_traits<I>::difference_type}.

\pnum
Users may specialize \tcode{incrementable_traits} on program-defined types.

\rSec3[readable.traits]{Readable traits}

\pnum
To implement algorithms only in terms of readable types, it is often necessary
to determine the value type that corresponds to a particular readable type.
Accordingly, it is required that if \tcode{R} is the name of a type that
models the \tcode{Readable} concept\iref{iterator.concept.readable},
the type

\begin{codeblock}
iter_value_t<R>
\end{codeblock}

be defined as the readable type's value type.

{\color{oldclr}
\pnum
\tcode{iter_value_t} is implemented as if:
} %% \color{oldclr}

\indexlibrary{\idxcode{iter_value_t}}%
\indexlibrary{\idxcode{readable_traits}}%
\begin{codeblock}
  template<class> struct @\placeholder{cond-value-type}@ { }; // \expos
  template<class T>
    requires is_object_v<T>
  struct @\placeholder{cond-value-type}@ {
    using value_type = remove_cv_t<T>;
  };

  template<class> struct readable_traits { };

  template<class T>
  struct readable_traits<T*>
    : @\placeholder{cond-value-type}@<T> { };

  template<class I>
    requires is_array_v<I>
  struct readable_traits<I>
    : readable_traits<decay_t<I>> { };

  template<class I>
  struct readable_traits<const I>
    : readable_traits<remove_const_t<I>> { };

  template<class T>
    requires requires { typename T::value_type; }
  struct readable_traits<T>
    : @\placeholder{cond-value-type}@<typename T::value_type> { };

  template<class T>
    requires requires { typename T::element_type; }
  struct readable_traits<T>
    : @\placeholder{cond-value-type}@<typename T::element_type> { };

  template<class T> using iter_value_t = @\seebelownc@;
\end{codeblock}

\pnum
If \tcode{iterator_traits<I>} is a program-defined specialization, then
\tcode{iter_value_t<I>} denotes \tcode{iterator_traits<I>::value_type};
otherwise, it denotes \tcode{readable_traits<I>::value_type}.

\pnum
Class template \tcode{readable_traits} may be specialized
on program-defined types.

\pnum
\begin{note}
Some legacy output iterators define a nested type named \tcode{value_type}
that is an alias for \tcode{void}. These types are not \tcode{Readable}
and have no associated value types.
\end{note}

\pnum
\begin{note}
Smart pointers like \tcode{shared_ptr<int>} are \tcode{Readable} and
have an associated value type, but a smart pointer like \tcode{shared_ptr<void>}
is not \tcode{Readable} and has no associated value type.
\end{note}
\end{addedblock}

\ednote{Relocate [iterator.traits] here and modify as follows:}
\rSec3[iterator.traits]{Iterator traits}

\pnum
\indexlibrary{\idxcode{iterator_traits}}%
To implement algorithms only in terms of iterators, it is \changed{often}{sometimes}
necessary to determine the
\changed{value and difference types}{iterator category} that
correspond\added{s} to a particular iterator type. Accordingly,
it is required that if \tcode{\changed{Iterator}{I}} is the type of an iterator,
the type\removed{s}

\indexlibrarymember{iterator_category}{iterator_traits}%
\begin{codeblock}
@\removed{iterator_traits<Iterator>::difference_type}@
@\removed{iterator_traits<Iterator>::value_type}@
iterator_traits<@\changed{Iterator}{I}@>::iterator_category
\end{codeblock}

be defined as the iterator's \removed{difference type, value type
and} iterator category\removed{, respectively}.
In addition, the types

\indexlibrarymember{reference}{iterator_traits}%
\indexlibrarymember{pointer}{iterator_traits}%
\begin{codeblock}
iterator_traits<@\changed{Iterator}{I}@>::reference
iterator_traits<@\changed{Iterator}{I}@>::pointer
\end{codeblock}

shall be defined as the iterator's reference and pointer types\changed{,}{;} that is, for an
iterator object \tcode{a} \newnewtxt{of class type}, the same type as
\oldoldtxt{the type of} \tcode{\newnewtxt{decltype(}*a\newnewtxt{)}} and
\tcode{\newnewtxt{decltype(}a\newnewtxt{.operator}->\newnewtxt{())}},
respectively. \added{The type
}\tcode{\added{iterator_traits<I>::pointer}}\added{ shall be \tcode{void} for
a\newnewtxt{n iterator of class} type \tcode{I} that does not support
\tcode{operator->}. Additionally, i}\removed{I}n the case of an output iterator,
the types

\begin{codeblock}
iterator_traits<@\changed{Iterator}{I}@>::difference_type
iterator_traits<@\changed{Iterator}{I}@>::value_type
iterator_traits<@\changed{Iterator}{I}@>::reference
@\removed{iterator_traits<Iterator>::pointer}@
\end{codeblock}

may be defined as \tcode{void}.

\begin{addedblock}
\pnum
\oldtxt{The member types of non-program-defined specializations of
\tcode{iterator_traits} are computed as defined below. The definition below uses
several exposition-only concepts equivalent to the following:}
\newtxt{The definitions in this subclause make use of the following
exposition-only concepts:}

\begin{codeblock}
template<class I>
concept @\placeholder{\oldtxt{_Cpp17Iterator}} \placeholder{\newtxt{cpp17-iterator}}@ =
  Copyable<I> && requires (I i) {
    @\oldtxt{\{}@ *i @\oldtxt{\} -> auto\&\&;}@
    @\newtxt{requires \placeholdernc{can-reference}<decltype(*i)>;}@
    { ++i } -> Same<I>&;
    @\oldtxt{\{}@ *i++ @\oldtxt{\} -> auto\&\&;}@
    @\newtxt{requires \placeholdernc{can-reference}<decltype(*i++)>;}@
  };

template<class I>
concept @\placeholder{\oldtxt{_Cpp17InputIterator}} \placeholder{\newtxt{cpp17-input-iterator}}@ =
  @\placeholder{\oldtxt{_Cpp17Iterator}} \placeholder{\newtxt{cpp17-iterator}}@<I> && EqualityComparable<I> && requires (I i) {
    typename incrementable_traits<I>::difference_type;
    typename readable_traits<I>::value_type;
    typename common_reference_t<iter_reference_t<I> &&,
                                typename readable_traits<I>::value_type &>;
    typename common_reference_t<decltype(*i++) &&,
                                typename readable_traits<I>::value_type &>;
    requires SignedIntegral<typename incrementable_traits<I>::difference_type>;
  };

template<class I>
concept @\placeholder{\oldtxt{_Cpp17ForwardIterator}} \placeholder{\newtxt{cpp17-forward-iterator}}@ =
  @\placeholder{\oldtxt{_Cpp17InputIterator}} \placeholder{\newtxt{cpp17-input-iterator}}@<I> && Constructible<I> &&
  Same<remove_cvref_t<iter_reference_t<I>>, typename readable_traits<I>::value_type> &&
  requires (I i) {
    { i++ } -> const I&;
    requires Same<iter_reference_t<I>, decltype(*i++)>;
  };

template<class I>
concept @\placeholder{\oldtxt{_Cpp17BidirectionalIterator}} \placeholder{\newtxt{cpp17-bidirectional-iterator}}@ =
  @\placeholder{\oldtxt{_Cpp17ForwardIterator}} \placeholder{\newtxt{cpp17-forward-iterator}}@<I> && requires (I i) {
    { --i } -> Same<I>&;
    { i-- } -> const I&;
    requires Same<iter_reference_t<I>, decltype(*i--)>;
  };

template<class I>
concept @\placeholder{\oldtxt{_Cpp17RandomAccessIterator}} \placeholder{\newtxt{cpp17-random-access-iterator}}@ =
  @\placeholder{\oldtxt{_Cpp17BidirectionalIterator}} \placeholder{\newtxt{cpp17-bidirectional-iterator}}@<I> && StrictTotallyOrdered<I> &&
  requires (I i, typename incrementable_traits<I>::difference_type n) {
    { i += n } -> Same<I>&;
    { i -= n } -> Same<I>&;
    requires Same<I, decltype(i + n)>;
    requires Same<I, decltype(n + i)>;
    requires Same<I, decltype(i - n)>;
    requires Same<decltype(n), decltype(i - i)>;
    { i[n] } -> iter_reference_t<I>;
  };
\end{codeblock}

{\color{newclr}
\pnum
The member types of \tcode{iterator_traits} specializations that are not
program-defined are computed as follows:
} %% \color{newclr}
\end{addedblock}

\begin{itemize}
\item
If \tcode{\changed{Iterator}{I}} has valid\cxxiref{temp.deduct} member
types \tcode{difference_type}, \tcode{value_type}, \removed{\tcode{pointer},}
\tcode{reference}, and \tcode{iterator_category}, \newnewtxt{then}
\tcode{iterator_traits<\changed{Iterator}{I}>}
\oldoldtxt{shall have} \newnewtxt{has} the following \oldoldtxt{as}
publicly accessible members:
\begin{codeblock}
  using difference_type   = typename @\changed{Iterator}{I}@::difference_type;
  using value_type        = typename @\changed{Iterator}{I}@::value_type;
  using pointer           = @\changed{typename Iterator::pointer}{\seebelownc}@;
  using reference         = typename @\changed{Iterator}{I}@::reference;
  using iterator_category = typename @\changed{Iterator}{I}@::iterator_category;
\end{codeblock}
\begin{addedblock}
If the \grammarterm{qualified-id} \tcode{I::pointer} is valid and
denotes a type, then \tcode{iterator_traits<I>::pointer} names that type;
otherwise, it names \tcode{void}.

\item
Otherwise, if \tcode{I} satisfies the exposition-only concept
\tcode{\placeholder{\oldtxt{_Cpp17InputIterator}} \placeholder{\newtxt{cpp17-input-iterator}}},
\newtxt{then} \tcode{iterator_traits<I>} \oldtxt{shall have} \newtxt{has} the following
\oldtxt{as} publicly accessible members:
\begin{codeblock}
  using difference_type   = typename incrementable_traits<I>::difference_type;
  using value_type        = typename readable_traits<I>::value_type;
  using pointer           = @\seebelownc@;
  using reference         = @\seebelownc@;
  using iterator_category = @\seebelownc@;
\end{codeblock}
\begin{itemize}
\item If the \grammarterm{qualified-id} \tcode{I::pointer} is valid and denotes a type,
\tcode{pointer} names that type. Otherwise, if
\tcode{decltype(\brk{}declval<I\&>().operator->())} is well-formed, then
\tcode{pointer} names that type. Otherwise, \tcode{pointer} \oldtxt{is}
\newtxt{names} \tcode{void}.

\item If the \grammarterm{qualified-id} \tcode{I::reference} is valid and denotes a
type, \tcode{reference} names that type. Otherwise, \tcode{reference} \oldtxt{is}
\newtxt{names} \tcode{iter_reference_t<I>}.

\item If the \grammarterm{qualified-id} \tcode{I::iterator_category} is valid and
denotes a type, \tcode{iterator_category} names that type. Otherwise, if
\tcode{I} satisfies \tcode{\placeholder{\oldtxt{_Cpp17RandomAccessIterator}}
\placeholder{\newtxt{cpp17-random-access-iterator}}},
\tcode{iterator_category} \oldtxt{is} \newtxt{names}
\tcode{random_access_iterator_tag}. Otherwise, if
\tcode{I} satisfies \tcode{\placeholder{\oldtxt{_Cpp17BidirectionalIterator}}
\placeholder{\newtxt{cpp17-bidirectional-iterator}}},
\tcode{iterator_category} \oldtxt{is} \newtxt{names}
\tcode{bidirectional_iterator_tag}. Otherwise, if
\tcode{I} satisfies \tcode{\placeholder{\oldtxt{_Cpp17ForwardIterator}}
\placeholder{\newtxt{cpp17-forward-iterator}}},
\tcode{iterator_category} \oldtxt{is} \newtxt{names}
\tcode{forward_iterator_tag}.
Otherwise, \tcode{iterator_category} \oldtxt{is} \newtxt{names}
\tcode{input_iterator_tag}.
\end{itemize}

\item
Otherwise, if \tcode{I} satisfies the exposition-only concept
\tcode{\placeholder{\oldtxt{_Cpp17Iterator}}
\placeholder{\newtxt{cpp17-iterator}}}, \newtxt{then} \tcode{iterator_traits<I>}
\oldtxt{shall have} \newtxt{has} the following \oldtxt{as} publicly accessible
members:
\begin{codeblock}
  using difference_type   = @\seebelownc@;
  using value_type        = void;
  using pointer           = void;
  using reference         = void;
  using iterator_category = output_iterator_tag;
\end{codeblock}
If \tcode{incrementable_traits<I>::difference_type} is well-formed
and names a type, then \tcode{difference_type} names that type; otherwise, it
\oldtxt{is} \newtxt{names} \tcode{void}.
\end{addedblock}

\item
Otherwise, \tcode{iterator_traits<\changed{Iterator}{I}>}
\oldoldtxt{shall have} \newnewtxt{has} no members by any of the above names.
\end{itemize}

\begin{addedblock}
\pnum
Additionally, program-defined specializations of \tcode{iterator_traits} may
have a member type \tcode{iterator_concept} that is used to opt in or out of
conformance to the iterator concepts defined in~\ref{iterator.concepts}.
\oldtxt{If specified, it should be an alias for one of the standard iterator tag
types\iref{std.iterator.tags}, or an empty, copy- and move-constructible,
trivial class type that is publicly and unambiguously derived from one of the
standard iterator tag types.}
\end{addedblock}

\pnum
\removed{It}\tcode{\added{iterator_traits}} is specialized for pointers as

\begin{codeblock}
namespace std {
  template<class T>
    @\added{requires is_object_v<T>}@
  struct iterator_traits<T*> {
    using difference_type   = ptrdiff_t;
    using value_type        = remove_cv_t<T>;
    using pointer           = T*;
    using reference         = T&;
    using iterator_category = random_access_iterator_tag;
    @\added{using iterator_concept}\added{ }\added{ }\added{= contiguous_iterator_tag;}@
  };
}
\end{codeblock}

\pnum
\begin{example}
To implement a generic
\tcode{reverse}
function, a \Cpp{} program can do the following:

\begin{codeblock}
template<class BidirectionalIterator>
void reverse(BidirectionalIterator first, BidirectionalIterator last) {
  typename iterator_traits<BidirectionalIterator>::difference_type n =
    distance(first, last);
  --n;
  while(n > 0) {
    typename iterator_traits<BidirectionalIterator>::value_type
     tmp = *first;
    *first++ = *--last;
    *last = tmp;
    n -= 2;
  }
}
\end{codeblock}
\end{example}

\begin{addedblock}
\rSec2[iterator.custpoints]{Customization points}
\rSec3[iterator.custpoints.iter_move]{\tcode{iter_move}}

\pnum
The name \tcode{iter_move} denotes a \term{customization point
object}\cxxiref{customization.point.object}. The expression
\tcode{ranges::iter_move(E)} for some subexpression \tcode{E} is
expression-equivalent to the following:

\begin{itemize}
\item \tcode{iter_move(E)}, if that expression is well-formed when
evaluated in a context that does not include \tcode{ranges::iter_move}
but does include the lookup set produced by
argument-dependent lookup\cxxiref{basic.lookup.argdep}.

\item Otherwise, if the expression \tcode{*E} is well-formed:
\begin{itemize}
\item if \tcode{*E} is an lvalue, \tcode{std::move(*E)};

\item otherwise, \tcode{*E}.
\end{itemize}

\item Otherwise, \tcode{ranges::iter_move(E)} is ill-formed.
\end{itemize}

\pnum
If \tcode{ranges::iter_move(E)} is not equal to \tcode{*E}, the program is
ill-formed with no diagnostic required.

\rSec3[iterator.custpoints.iter_swap]{\tcode{iter_swap}}

\pnum
The name \tcode{iter_swap} denotes a \term{customization point
object}\cxxiref{customization.point.object}. The expression
\tcode{ranges::iter_swap(E1, E2)} for some subexpressions
\tcode{E1} and \tcode{E2}
is expression-equivalent to the following:

\begin{itemize}
\item \tcode{(void)iter_swap(E1, E2)}, if that expression is well-formed when
evaluated in a context that does not include \tcode{ranges::iter_swap} but does
include the lookup set produced by
argument-dependent lookup\cxxiref{basic.lookup.argdep}
and the following declaration:
\begin{codeblock}
template<class I1, class I2>
  void iter_swap(I1, I2) = delete;
\end{codeblock}

\item Otherwise, if the types of \tcode{E1} and \tcode{E2} both model
\tcode{Readable}, and if the reference type of \tcode{E1} is swappable
with\iref{concept.swappable} the reference type of \tcode{E2},
then \tcode{ranges::swap(*E1, *E2)}

\item Otherwise, if the types \tcode{T1} and \tcode{T2} of \tcode{E1} and
\tcode{E2} model \tcode{IndirectlyMovableStorable<T1, T2>} and
\tcode{IndirectlyMovableStorable<T2, T1>},
\tcode{(void)(*E1 = \placeholdernc{iter-exchange-move}(E2, E1))},
except that \tcode{E1} is evaluated only once.

\item Otherwise, \tcode{ranges::iter_swap(E1, E2)} is ill-formed.
\end{itemize}

\pnum
If \tcode{ranges::iter_swap(E1, E2)} does not swap the values denoted by the
expressions \tcode{E1} and \tcode{E2}, the program is ill-formed with no
diagnostic required.

\pnum
\tcode{\placeholder{iter-exchange-move}} is an exposition-only function specified as:
\begin{itemdecl}
template<class X, class Y>
  constexpr iter_value_t<remove_reference_t<X>> @\placeholdernc{iter-exchange-move}@(X&& x, Y&& y)
    noexcept(@\oldtxt{\seebelow} \newtxt{noexcept(iter_value_t<remove_reference_t<X>>(iter_move(x))) \&\&}@
      @\newtxt{noexcept(*x = iter_move(y))}@);
\end{itemdecl}

\begin{itemdescr}
\pnum
\effects Equivalent to:
\begin{codeblock}
iter_value_t<remove_reference_t<X>> old_value(iter_move(x));
*x = iter_move(y);
return old_value;
\end{codeblock}

\pnum
\oldtxt{\remarks The expression in the \tcode{noexcept} is equivalent to:}
\begin{codeblock}
@\oldtxt{NE(remove_reference_t<X>, remove_reference_t<Y>) \&\&}@
@\oldtxt{NE(remove_reference_t<Y>, remove_reference_t<X>)}@
\end{codeblock}
\oldtxt{Where \tcode{NE(T1, T2)} is the expression:}
\begin{codeblock}
@\oldtxt{is_nothrow_constructible_v<iter_value_t<T1>, iter_rvalue_reference_t<T1>> \&\&}@
@\oldtxt{is_nothrow_assignable_v<iter_value_t<T1>\&, iter_rvalue_reference_t<T1>> \&\&}@
@\oldtxt{is_nothrow_assignable_v<iter_reference_t<T1>, iter_rvalue_reference_t<T2>> \&\&}@
@\oldtxt{is_nothrow_assignable_v<iter_reference_t<T1>, iter_value_t<T2>> \&\&}@
@\oldtxt{is_nothrow_move_constructible_v<iter_value_t<T1>> \&\&}@
@\oldtxt{noexcept(ranges::iter_move(declval<T1\&>()))}@
\end{codeblock}
\end{itemdescr}

\rSec2[iterator.concepts]{Iterator concepts}

\rSec3[iterator.concepts.general]{General}

\pnum
Many of the concepts defined in this subclause\iref{iterator.concepts} use the
exposition-only type function \tcode{\placeholder{ITER_CONCEPT}} in their
specifications.

\pnum
For a type \tcode{I}, let \tcode{\placeholdernc{ITER_TRAITS}(I)} denote the type
\tcode{I} if \tcode{iterator_traits<I>} names an instantiation of the primary
template. Otherwise, \tcode{\placeholdernc{ITER_TRAITS}(I)} denotes
\tcode{iterator_traits<I>}.
\begin{itemize}
\item If the \grammarterm{qualified-id}
  \tcode{\placeholdernc{ITER_TRAITS}(I)::iterator_concept} is valid
  and names a type, then \tcode{\placeholdernc{ITER_CONCEPT}(I)} denotes that
  type.
\item Otherwise, if \tcode{\placeholdernc{ITER_TRAITS}(I)\brk{}::iterator_category}
  is valid and names a type, then \tcode{\placeholdernc{ITER_CONCEPT}(I)}
  denotes that type.
\item Otherwise, if \tcode{iterator_traits<I>} names an instantiation of
  the primary template, then \tcode{\placeholdernc{ITER_CONCEPT}(I)} denotes
  \tcode{random_access_iterator_tag}.
\item Otherwise, \tcode{\placeholdernc{ITER_CONCEPT}(I)} does not denote a type.
\end{itemize}

\rSec3[iterator.concept.readable]{Concept \libconcept{Readable}}

\pnum
The \libconcept{Readable} concept is satisfied by types that are readable by
applying \tcode{operator*} including pointers, smart pointers, and iterators.

\indexlibrary{\idxcode{Readable}}%
\begin{codeblock}
template<class In>
  concept Readable =
    requires {
      typename iter_value_t<In>;
      typename iter_reference_t<In>;
      typename iter_rvalue_reference_t<In>;
    } &&
    CommonReference<iter_reference_t<In>&&, iter_value_t<In>&> &&
    CommonReference<iter_reference_t<In>&&, iter_rvalue_reference_t<In>&&> &&
    CommonReference<iter_rvalue_reference_t<In>&&, const iter_value_t<In>&>;
\end{codeblock}

\pnum
Given a value \tcode{i} of type \tcode{I}, \tcode{I} models \libconcept{Readable}
only if the expression \tcode{*i} (which is indirectly required to be valid via the
exposition-only \placeholder{dereferenceable} concept\iref{iterator.synopsis}) is
equality-preserving.

\rSec3[iterator.concept.writable]{Concept \tcode{Writable}}

\pnum
The \tcode{Writable} concept specifies the requirements for writing a value
into an iterator's referenced object.

\indexlibrary{\idxcode{Writable}}%
\begin{codeblock}
template<class Out, class T>
  concept Writable =
    requires(Out&& o, T&& t) {
      *o = std::forward<T>(t); // not required to be equality-preserving
      *std::forward<Out>(o) = std::forward<T>(t); // not required to be equality-preserving
      const_cast<const iter_reference_t<Out>&&>(*o) =
        std::forward<T>(t); // not required to be equality-preserving
      const_cast<const iter_reference_t<Out>&&>(*std::forward<Out>(o)) =
        std::forward<T>(t); // not required to be equality-preserving
    };
\end{codeblock}

\pnum
Let \tcode{E} be an an expression such that \tcode{decltype((E))} is \tcode{T},
and let \tcode{o} be a dereferenceable object of type \tcode{Out}.
\tcode{Out} and \tcode{T} model \tcode{Writable<Out, T>} only if

\begin{itemize}
\item If \tcode{Readable<Out> \&\& Same<iter_value_t<Out>, decay_t<T>{>}} is satisfied,
then \tcode{*o} after any above assignment is equal
to the value of \tcode{E} before the assignment.
\end{itemize}

\pnum
After evaluating any above assignment expression, \tcode{o} is not required to be dereferenceable.

\pnum
If \tcode{E} is an xvalue\cxxiref{basic.lval}, the resulting
state of the object it denotes is valid but unspecified\cxxiref{lib.types.movedfrom}.

\pnum
\begin{note}
The only valid use of an \tcode{operator*} is on the left side of the assignment statement.
\textit{Assignment through the same value of the writable type happens only once.}
\end{note}

\rSec3[iterator.concept.weaklyincrementable]{Concept \tcode{WeaklyIncrementable}}

\pnum
The \tcode{WeaklyIncrementable} concept specifies the requirements on
types that can be incremented with the pre- and post-increment operators.
The increment operations are not required to be equality-preserving,
nor is the type required to be \libconcept{EqualityComparable}.

\indexlibrary{\idxcode{WeaklyIncrementable}}%
\begin{codeblock}
template<class I>
  concept WeaklyIncrementable =
    Semiregular<I> &&
    requires(I i) {
      typename iter_difference_t<I>;
      requires SignedIntegral<iter_difference_t<I>>;
      { ++i } -> Same<I>&; // not required to be equality-preserving
      i++; // not required to be equality-preserving
    };
\end{codeblock}

\pnum
Let \tcode{i} be an object of type \tcode{I}. When \tcode{i} is in the domain of
both pre- and post-increment, \tcode{i} is said to be \term{incrementable}.
\tcode{WeaklyIncrementable<I>} is satisfied only if

\begin{itemize}
\item The expressions \tcode{++i} and \tcode{i++} have the same domain.
\item If \tcode{i} is incrementable, then both \tcode{++i}
  and \tcode{i++} advance \tcode{i} to the next element.
\item If \tcode{i} is incrementable, then
  \tcode{addressof(++i)} is equal to
  \tcode{addressof(i)}.
\end{itemize}

\pnum
\begin{note}
For \tcode{WeaklyIncrementable} types, \tcode{a} equals \tcode{b} does not imply that \tcode{++a}
equals \tcode{++b}. (Equality does not guarantee the substitution property or referential
transparency.) Algorithms on weakly incrementable types should never attempt to pass
through the same incrementable value twice. They should be single pass algorithms. These algorithms
can be used with istreams as the source of the input data through the \tcode{istream_iterator} class
template.
\end{note}

\rSec3[iterator.concept.incrementable]{Concept \tcode{Incrementable}}

\pnum
The \tcode{Incrementable} concept specifies requirements on types that can be incremented with the pre-
and post-increment operators. The increment operations are required to be equality-preserving,
and the type is required to be \libconcept{EqualityComparable}.
\begin{note}
This requirement
supersedes the annotations on the increment expressions in the definition of
\tcode{WeaklyIncrementable}.
\end{note}

\indexlibrary{\idxcode{Incrementable}}%
\begin{codeblock}
template<class I>
  concept Incrementable =
    Regular<I> &&
    WeaklyIncrementable<I> &&
    requires(I i) {
      i++; requires Same<decltype(i++), I>;
    };
\end{codeblock}

\pnum
Let \tcode{a} and \tcode{b} be incrementable objects of type \tcode{I}.
\tcode{I} models \libconcept{Incrementable} only if

\begin{itemize}
\item If \tcode{bool(a == b)} then \tcode{bool(a++ == b)}.
\item If \tcode{bool(a == b)} then \tcode{bool(((void)a++, a) == ++b)}.
\end{itemize}

\pnum
\begin{note}
The requirement that \tcode{a} equals \tcode{b} implies \tcode{++a} equals \tcode{++b}
(which is not true for weakly incrementable types) allows the use of multi-pass one-directional
algorithms with types that satisfy \libconcept{Increment\-able}.
\end{note}

\rSec3[iterator.concept.iterator]{Concept \tcode{Iterator}}

\pnum
The \libconcept{Iterator} concept forms
the basis of the iterator concept taxonomy; every iterator satisfies the
\libconcept{Iterator} requirements. This
concept specifies operations for dereferencing and incrementing
an iterator. Most algorithms will require additional operations
to compare iterators with sentinels\iref{iterator.concept.sentinel}, to
read\iref{iterator.concept.input} or write\iref{iterator.concept.output} values, or
to provide a richer set of iterator movements (\ref{iterator.concept.forward},
\ref{iterator.concept.bidirectional}, \ref{iterator.concept.random.access}).)

\indexlibrary{\idxcode{Iterator}}%
\begin{codeblock}
template<class I>
  concept Iterator =
    requires(I i) {
      @\oldtxt{\{}@ *i @\oldtxt{\} -> auto\&\&; // Requires: i is dereferenceable}@
      @\newtxt{requires \placeholdernc{can-reference}<decltype(*i)>;}@
    } &&
    WeaklyIncrementable<I>;
\end{codeblock}

{\color{oldclr}
\pnum
\begin{note}
\oldtxt{The requirement that the result of dereferencing the iterator is deducible from
\tcode{auto\&\&} means that it cannot be \tcode{void}.}
\end{note}
} %% \color{oldclr}

\rSec3[iterator.concept.sentinel]{Concept \tcode{Sentinel}}

\pnum
The \libconcept{Sentinel} concept specifies the relationship
between an \libconcept{Iterator} type and a \libconcept{Semiregular} type
whose values denote a range.

\indexlibrary{\idxcode{Sentinel}}%
\begin{itemdecl}
template<class S, class I>
  concept Sentinel =
    Semiregular<S> &&
    Iterator<I> &&
    @\placeholder{weakly-equality-comparable-with}@<S, I>; // See \cxxref{concept.equalitycomparable}
\end{itemdecl}

\begin{itemdescr}
\pnum
Let \tcode{s} and \tcode{i} be values of type \tcode{S} and
\tcode{I} such that \range{i}{s} denotes a range. Types
\tcode{S} and \tcode{I} model \tcode{Sentinel<S, I>} only if

\begin{itemize}
\item \tcode{i == s} is well-defined.

\item If \tcode{bool(i != s)} then \tcode{i} is dereferenceable and
      \range{++i}{s} denotes a range.
\end{itemize}
\end{itemdescr}

\pnum
The domain of \tcode{==} can change over time.
Given an iterator \tcode{i} and sentinel \tcode{s} such that \range{i}{s}
denotes a range and \tcode{i != s}, \range{i}{s} is not required to continue to
denote a range after incrementing any iterator equal to \tcode{i}. Consequently,
\tcode{i == s} is no longer required to be well-defined.

\rSec3[iterator.concept.sizedsentinel]{Concept \tcode{SizedSentinel}}

\pnum
The \libconcept{SizedSentinel} concept specifies
requirements on an \libconcept{Iterator} and a \libconcept{Sentinel}
that allow the use of the \tcode{-} operator to compute the distance
between them in constant time.

\indexlibrary{\idxcode{SizedSentinel}}%

\begin{itemdecl}
template<class S, class I>
  concept SizedSentinel =
    Sentinel<S, I> &&
    !disable_sized_sentinel<remove_cv_t<S>, remove_cv_t<I>> &&
    requires(const I& i, const S& s) {
      s - i; requires Same<decltype(s - i), iter_difference_t<I>>;
      i - s; requires Same<decltype(i - s), iter_difference_t<I>>
    };
\end{itemdecl}

\begin{itemdescr}
\pnum
Let \tcode{i} be an iterator of type \tcode{I}, and \tcode{s}
a sentinel of type \tcode{S} such that \range{i}{s} denotes a range.
Let $N$ be the smallest number of applications of \tcode{++i}
necessary to make \tcode{bool(i == s)} be \tcode{true}.
\tcode{S} and \tcode{I} model \tcode{SizedSentinel<S, I>} only if

\begin{itemize}
\item If $N$ is representable by \tcode{iter_difference_t<I>},
      then \tcode{s - i} is well-defined and equals $N$.

\item If $-N$ is representable by \tcode{iter_difference_t<I>},
      then \tcode{i - s} is well-defined and equals $-N$.
\end{itemize}
\end{itemdescr}

\pnum
\begin{note}
\tcode{disable_sized_sentinel} provides a mechanism to
enable use of sentinels and iterators with the library that meet the
syntactic requirements but do not in fact satisfy \libconcept{SizedSentinel}.
A program that instantiates a library template that requires
\libconcept{SizedSentinel} with an iterator type \tcode{I} and sentinel type
\tcode{S} that meet the syntactic requirements of \tcode{SizedSentinel<S, I>}
but do not model \libconcept{SizedSentinel} is
ill-formed with no diagnostic required\cxxiref{structure.requirements}.
\end{note}

\pnum
\begin{note}
The \libconcept{SizedSentinel} concept is satisfied by pairs of
\libconcept{RandomAccessIterator}s\iref{iterator.concept.random.access} and by
counted iterators and their sentinels\iref{counted.iterator}.
\end{note}

\rSec3[iterator.concept.input]{Concept \tcode{InputIterator}}

\pnum
The \tcode{InputIterator} concept is a refinement of
\tcode{Iterator}\iref{iterator.concept.iterator}. It defines requirements for a type
whose referenced values can be read (from the requirement for
\tcode{Readable}\iref{iterator.concept.readable}) and which can be both pre- and
post-incremented.
\begin{note}
Unlike the input iterator requirements in \ref{input.iterators},
the \libconcept{InputIterator} concept does not require equality comparison.
\end{note}

\indexlibrary{\idxcode{InputIterator}}%
\begin{codeblock}
template<class I>
  concept InputIterator =
    Iterator<I> &&
    Readable<I> &&
    requires { typename @\placeholdernc{ITER_CONCEPT}@(I); } &&
    DerivedFrom<@\placeholdernc{ITER_CONCEPT}@(I), input_iterator_tag>;
\end{codeblock}

\rSec3[iterator.concept.output]{Concept \tcode{OutputIterator}}

\pnum
The \tcode{OutputIterator} concept is a refinement of
\tcode{Iterator}\iref{iterator.concept.iterator}. It defines requirements for a type that
can be used to write values (from the requirement for
\tcode{Writable}\iref{iterator.concept.writable}) and which can be both pre- and post-incremented.
However, output iterators are not required to
satisfy \libconcept{EqualityComparable}.

\indexlibrary{\idxcode{OutputIterator}}%
\begin{codeblock}
template<class I, class T>
  concept OutputIterator =
    Iterator<I> &&
    Writable<I, T> &&
    requires(I i, T&& t) {
      *i++ = std::forward<T>(t); // not required to be equality-preserving
    };
\end{codeblock}

\pnum
Let \tcode{E} be an expression such that \tcode{decltype((E))} is \tcode{T}, and let \tcode{i} be a
dereferenceable object of type \tcode{I}. \tcode{I} and \tcode{T} model \tcode{OutputIterator<I, T>} only if
\tcode{*i++ = E;} has effects equivalent to:
\begin{codeblock}
  *i = E;
  ++i;
\end{codeblock}

\pnum
\begin{note}
Algorithms on output iterators should never attempt to pass through the same iterator twice.
They should be \term{single pass} algorithms.
\end{note}

\rSec3[iterator.concept.forward]{Concept \tcode{ForwardIterator}}

\pnum
The \libconcept{ForwardIterator} concept refines
\libconcept{InputIterator}\iref{iterator.concept.input},
adding equality comparison and the multi-pass guarantee, specified below.

\indexlibrary{\idxcode{ForwardIterator}}%
\begin{codeblock}
template<class I>
  concept ForwardIterator =
    InputIterator<I> &&
    DerivedFrom<@\placeholdernc{ITER_CONCEPT}@(I), forward_iterator_tag> &&
    Incrementable<I> &&
    Sentinel<I, I>;
\end{codeblock}

\pnum
The domain of \tcode{==} for forward iterators is that of iterators over the same
underlying sequence. However, value-initialized iterators of the same type
may be compared and shall compare equal to other value-initialized iterators of the same type.
\begin{note}
Value-initialized iterators behave as if they refer past the end of the same
empty sequence.
\end{note}

\pnum
Pointers and references obtained from a forward iterator into a range \range{i}{s}
shall remain valid while \range{i}{s} continues to denote a range.

\pnum
Two dereferenceable iterators \tcode{a} and \tcode{b} of type \tcode{X}
offer the \defn{multi-pass guarantee} if:

\begin{itemize}
\item \tcode{a == b} implies \tcode{++a == ++b} and
\item The expression
\tcode{([](X x)\{++x;\}(a), *a)} is equivalent to the expression \tcode{*a}.
\end{itemize}

\pnum
\begin{note}
The requirement that
\tcode{a == b}
implies
\tcode{++a == ++b}
\oldtxt{(which is not true for weaker iterators)}
and the removal of the restrictions on the number of assignments through
a mutable iterator
(which applies to output iterators)
allow the use of multi-pass one-directional algorithms with forward iterators.
\end{note}

\rSec3[iterator.concept.bidirectional]{Concept \libconcept{BidirectionalIterator}}

\pnum
The \libconcept{BidirectionalIterator} concept refines \libconcept{ForwardIterator}\iref{iterator.concept.forward},
and adds the ability to move an iterator backward as well as forward.

\indexlibrary{\idxcode{BidirectionalIterator}}%
\begin{codeblock}
template<class I>
  concept BidirectionalIterator =
    ForwardIterator<I> &&
    DerivedFrom<@\placeholdernc{ITER_CONCEPT}@(I), bidirectional_iterator_tag> &&
    requires(I i) {
      { --i } -> Same<I>&;
      i--; requires Same<decltype(i--), I>;
    };
\end{codeblock}

\pnum
A bidirectional iterator \tcode{r} is decrementable if and only if there exists some \tcode{s} such that
\tcode{++s == r}. Decrementable iterators \tcode{r} shall be in the domain of the expressions
\tcode{\dcr{}r} and \tcode{r\dcr{}}.

\pnum
Let \tcode{a} and \tcode{b} be decrementable objects of type \tcode{I}.
\tcode{I} models \libconcept{BidirectionalIterator} only if:

\begin{itemize}
\item \tcode{addressof(\dcr{}a) == addressof(a)}.
\item If \tcode{bool(a == b)}, then \tcode{bool(a\dcr{} == b)}.
\item If \tcode{bool(a == b)}, then after evaluating both \tcode{a\dcr} and \tcode{\dcr{}b},
\tcode{bool(a == b)} still holds.
\item If \tcode{a} is incrementable and \tcode{bool(a == b)}, then
      \tcode{bool(\dcr{}(++a) == b)}.
\item If \tcode{bool(a == b)}, then \tcode{bool(++(\dcr{}a) == b)}.
\end{itemize}

\rSec3[iterator.concept.random.access]{Concept \libconcept{RandomAccessIterator}}

\pnum
The \libconcept{RandomAccessIterator} concept refines
\libconcept{BidirectionalIterator}\iref{iterator.concept.bidirectional}
and adds support for constant-time advancement with
\tcode{+=}, \tcode{+}, \tcode{-=}, and \tcode{-}, and the computation
of distance in constant time with \tcode{-}. Random access iterators
also support array notation via subscripting.

\indexlibrary{\idxcode{RandomAccessIterator}}%
\begin{codeblock}
template<class I>
  concept RandomAccessIterator =
    BidirectionalIterator<I> &&
    DerivedFrom<@\placeholdernc{ITER_CONCEPT}@(I), random_access_iterator_tag> &&
    StrictTotallyOrdered<I> &&
    SizedSentinel<I, I> &&
    requires(I i, const I j, const iter_difference_t<I> n) {
      { i += n } -> Same<I>&;
      j + n; requires Same<decltype(j + n), I>;
      n + j; requires Same<decltype(n + j), I>;
      { i -= n } -> Same<I>&;
      j - n; requires Same<decltype(j - n), I>;
      j[n]; requires Same<decltype(j[n]), iter_reference_t<I>>;
    };
\end{codeblock}

\pnum
Let \tcode{a} and \tcode{b} be valid iterators of type \tcode{I} such that
\tcode{b} is reachable from \tcode{a}. Let \tcode{n} be the smallest value
of type \tcode{iter_difference_t<I>} such that after
\tcode{n} applications of \tcode{++a}, then \tcode{bool(a == b)}.
\tcode{I} models \libconcept{RandomAccessIterator} only if

\begin{itemize}
\item \tcode{(a += n)} is equal to \tcode{b}.
\item \tcode{addressof((a += n))} is equal to \tcode{addressof(a)}.
\item \tcode{(a + n)} is equal to \tcode{(a += n)}.
\item For any two positive integers \tcode{x} and \tcode{y}, if \tcode{a + (x + y)} is valid, then
\tcode{a + (x + y)} is equal to \tcode{(a + x) + y}.
\item \tcode{a + 0} is equal to \tcode{a}.
\item If \tcode{(a + (n - 1))} is valid, then \tcode{a + n} is equal to \tcode{++(a + (n - 1))}.
\item \tcode{(b += -n)} is equal to \tcode{a}.
\item \tcode{(b -= n)} is equal to \tcode{a}.
\item \tcode{addressof((b -= n))} is equal to \tcode{addressof(b)}.
\item \tcode{(b - n)} is equal to \tcode{(b -= n)}.
\item If \tcode{b} is dereferenceable, then \tcode{a[n]} is valid and is equal to \tcode{*b}.
\item \tcode{bool(a <= b)} is \tcode{true}.
\end{itemize}

\rSec3[iterator.concept.contiguous]{Concept \libconcept{ContiguousIterator}}

\pnum
The \libconcept{ContiguousIterator} concept refines \libconcept{RandomAccessIterator} and
provides a guarantee that the denoted elements are stored contiguously in memory.

\indexlibrary{\idxcode{ContiguousIterator}}%
\begin{codeblock}
template<class I>
  concept @\libconcept{ContiguousIterator}@ =
    RandomAccessIterator<I> &&
    DerivedFrom<@\placeholdernc{ITER_CONCEPT}@(I), contiguous_iterator_tag> &&
    is_lvalue_reference_v<iter_reference_t<I>> &&
    Same<iter_value_t<I>, remove_cvref_t<iter_reference_t<I>>>;
\end{codeblock}

\pnum
Let \tcode{a} and \tcode{b} be dereferenceable iterators of type \tcode{I} such
that \tcode{b} is reachable from \tcode{a}. \tcode{I} models
\libconcept{ContiguousIterator} only if
\begin{codeblock}
addressof(*(a + (b - a))) == addressof(*a) + (b - a)
\end{codeblock}
is \tcode{true}.

\rSec2[iterator.cpp17]{\Cpp{}17 iterator requirements}

\pnum
In the following sections,
\tcode{a}
and
\tcode{b}
denote values of type
\tcode{X} or \tcode{const X},
\tcode{difference_type} and \tcode{reference} refer to the
types \tcode{iterator_traits<X>::difference_type} and
\tcode{iterator_traits<X>::reference}, respectively,
\tcode{n}
denotes a value of
\tcode{difference_type},
\tcode{u},
\tcode{tmp},
and
\tcode{m}
denote identifiers,
\tcode{r}
denotes a value of
\tcode{X\&},
\tcode{t}
denotes a value of value type
\tcode{T},
\tcode{o}
denotes a value of some type that is writable to the output iterator.
\begin{note} For an iterator type \tcode{X} there must be an instantiation
of \tcode{iterator_traits<X>}\iref{iterator.traits}. \end{note}
\end{addedblock}

\ednote{Relocate [iterator.iterators] here:}
\rSec3[iterator.iterators]{\oldconcept{Iterator}}
[...]

\ednote{Relocate [input.iterators] here:}
\rSec3[input.iterators]{Input iterators}
[...]

\ednote{Relocate [output.iterators] here:}
\rSec3[output.iterators]{Output iterators}
[...]

\ednote{Relocate [forward.iterators] here:}
\rSec3[forward.iterators]{Forward iterators}
[...]

\ednote{Relocate [bidirectional.iterators] here:}
\rSec3[bidirectional.iterators]{Bidirectional iterators}
[...]

\ednote{Relocate [random.access.iterators] here:}
\rSec3[random.access.iterators]{Random access iterators}
[...]

\begin{addedblock}
\rSec2[indirectcallable]{Indirect callable requirements}

\rSec3[indirectcallable.general]{General}

\pnum
There are several concepts that group requirements of algorithms that
take callable objects~(\cxxref{func.require}) as arguments.

\rSec3[indirectcallable.indirectinvocable]{Indirect callables}

\pnum
The indirect callable concepts are used to constrain those algorithms
that accept callable objects~(\cxxref{func.def}) as arguments.

\indexlibrary{\idxcode{IndirectUnaryInvocable}}%
\indexlibrary{\idxcode{IndirectRegularUnaryInvocable}}%
\indexlibrary{\idxcode{IndirectUnaryPredicate}}%
\indexlibrary{\idxcode{IndirectRelation}}%
\indexlibrary{\idxcode{IndirectStrictWeakOrder}}%
\begin{codeblock}
namespace std {
  template<class F, class I>
    concept IndirectUnaryInvocable =
      Readable<I> &&
      CopyConstructible<F> &&
      Invocable<F&, iter_value_t<I>&> &&
      Invocable<F&, iter_reference_t<I>> &&
      Invocable<F&, iter_common_reference_t<I>> &&
      CommonReference<
        invoke_result_t<F&, iter_value_t<I>&>,
        invoke_result_t<F&, iter_reference_t<I>>>;

  template<class F, class I>
    concept IndirectRegularUnaryInvocable =
      Readable<I> &&
      CopyConstructible<F> &&
      RegularInvocable<F&, iter_value_t<I>&> &&
      RegularInvocable<F&, iter_reference_t<I>> &&
      RegularInvocable<F&, iter_common_reference_t<I>> &&
      CommonReference<
        invoke_result_t<F&, iter_value_t<I>&>,
        invoke_result_t<F&, iter_reference_t<I>>>;

  template<class F, class I>
    concept IndirectUnaryPredicate =
      Readable<I> &&
      CopyConstructible<F> &&
      Predicate<F&, iter_value_t<I>&> &&
      Predicate<F&, iter_reference_t<I>> &&
      Predicate<F&, iter_common_reference_t<I>>;

  template<class F, class I1, class I2 = I1>
    concept IndirectRelation =
      Readable<I1> && Readable<I2> &&
      CopyConstructible<F> &&
      Relation<F&, iter_value_t<I1>&, iter_value_t<I2>&> &&
      Relation<F&, iter_value_t<I1>&, iter_reference_t<I2>> &&
      Relation<F&, iter_reference_t<I1>, iter_value_t<I2>&> &&
      Relation<F&, iter_reference_t<I1>, iter_reference_t<I2>> &&
      Relation<F&, iter_common_reference_t<I1>, iter_common_reference_t<I2>>;

  template<class F, class I1, class I2 = I1>
    concept IndirectStrictWeakOrder =
      Readable<I1> && Readable<I2> &&
      CopyConstructible<F> &&
      StrictWeakOrder<F&, iter_value_t<I1>&, iter_value_t<I2>&> &&
      StrictWeakOrder<F&, iter_value_t<I1>&, iter_reference_t<I2>> &&
      StrictWeakOrder<F&, iter_reference_t<I1>, iter_value_t<I2>&> &&
      StrictWeakOrder<F&, iter_reference_t<I1>, iter_reference_t<I2>> &&
      StrictWeakOrder<F&, iter_common_reference_t<I1>, iter_common_reference_t<I2>>;
}
\end{codeblock}

\rSec3[projected]{Class template \tcode{projected}}

\pnum
Class template \tcode{projected} is intended for use when specifying
the constraints of algorithms that accept callable objects
and projections\iref{defns.projection}. It bundles a
\libconcept{Readable} type \tcode{I} and
a function \tcode{Proj} into a new \libconcept{Readable} type
whose \tcode{reference} type is the result of applying
\tcode{Proj} to the \tcode{iter_reference_t} of \tcode{I}.

\indexlibrary{\idxcode{projected}}%
\begin{codeblock}
namespace std {
  template<Readable I, IndirectRegularUnaryInvocable<I> Proj>
  struct projected {
    using value_type = remove_cvref_t<indirect_result_t<Proj&, I>>;
    indirect_result_t<Proj&, I> operator*() const;
  };

  template<WeaklyIncrementable I, class Proj>
  struct incrementable_traits<projected<I, Proj>> {
    using difference_type = iter_difference_t<I>;
  };
}
\end{codeblock}

\pnum
\begin{note}
\tcode{projected} is only used to ease constraints specification. Its
member function need not be defined.
\end{note}

\rSec2[commonalgoreq]{Common algorithm requirements}
\rSec3[commonalgoreq.general]{General}

\pnum
There are several additional iterator concepts that are commonly applied
to families of algorithms. These group together iterator requirements
of algorithm families.
There are three relational concepts that specify
how element values are transferred  between \libconcept{Readable} and
\libconcept{Writable} types:
\libconcept{Indirectly\-Movable},
\libconcept{Indir\-ect\-ly\-Copy\-able}, and
\libconcept{Indirectly\-Swappable}.
There are three relational concepts for rearrangements:
\libconcept{Permut\-able},
\libconcept{Mergeable}, and
\libconcept{Sortable}.
There is one relational concept for comparing values from different sequences:
\libconcept{IndirectlyComparable}.

\pnum
\begin{note}
The \tcode{ranges::less<>}\iref{range.comparisons} function object type used
in the concepts below imposes constraints on their arguments in addition to
those that appear explicitly in the concepts' bodies.
The function call operator of \tcode{ranges::less<>} requires
its arguments to model
\libconcept{StrictTotally\-OrderedWith}~(\cxxref{concept.stricttotallyordered}).
\end{note}

\rSec3[commonalgoreq.indirectlymovable]{Concept \libconcept{IndirectlyMovable}}

\pnum
The \libconcept{IndirectlyMovable} concept specifies the relationship between
a \libconcept{Readable} type and a \libconcept{Writable} type between which
values may be moved.

\indexlibrary{\idxcode{IndirectlyMovable}}%
\begin{codeblock}
template<class In, class Out>
  concept IndirectlyMovable =
    Readable<In> &&
    Writable<Out, iter_rvalue_reference_t<In>>;
\end{codeblock}

\pnum
The \libconcept{IndirectlyMovableStorable} concept augments
\libconcept{IndirectlyMovable} with additional requirements enabling
the transfer to be performed through an intermediate object of the
\libconcept{Readable} type's value type.

\indexlibrary{\idxcode{IndirectlyMovableStorable}}%
\begin{codeblock}
template<class In, class Out>
  concept IndirectlyMovableStorable =
    IndirectlyMovable<In, Out> &&
    Writable<Out, iter_value_t<In>> &&
    Movable<iter_value_t<In>> &&
    Constructible<iter_value_t<In>, iter_rvalue_reference_t<In>> &&
    Assignable<iter_value_t<In>&, iter_rvalue_reference_t<In>>;
\end{codeblock}

\pnum
Let \tcode{i} be a dereferenceable value of type \tcode{In}.
\tcode{In} and \tcode{Out} model \tcode{IndirectlyMovableStorable<In, Out>}
only if after the initialization of the object \tcode{obj} in
\begin{codeblock}
iter_value_t<In> obj(ranges::iter_move(i));
\end{codeblock}
\tcode{obj} is equal to the value previously denoted by \tcode{*i}. If
\tcode{iter_rvalue_reference_t<In>} is an rvalue reference type,
the resulting state of the value denoted by \tcode{*i} is
valid but unspecified\cxxiref{lib.types.movedfrom}.

\rSec3[commonalgoreq.indirectlycopyable]{Concept \libconcept{IndirectlyCopyable}}

\pnum
The \libconcept{IndirectlyCopyable} concept specifies the relationship between
a \libconcept{Readable} type and a \libconcept{Writable} type between which
values may be copied.

\indexlibrary{\idxcode{IndirectlyCopyable}}%
\begin{codeblock}
template<class In, class Out>
  concept IndirectlyCopyable =
    Readable<In> &&
    Writable<Out, iter_reference_t<In>>;
\end{codeblock}

\pnum
The \libconcept{IndirectlyCopyableStorable} concept augments
\libconcept{IndirectlyCopyable} with additional requirements enabling
the transfer to be performed through an intermediate object of the
\libconcept{Readable} type's value type. It also requires the capability
to make copies of values.

\indexlibrary{\idxcode{IndirectlyCopyableStorable}}%
\begin{codeblock}
template<class In, class Out>
  concept IndirectlyCopyableStorable =
    IndirectlyCopyable<In, Out> &&
    Writable<Out, const iter_value_t<In>&> &&
    Copyable<iter_value_t<In>> &&
    Constructible<iter_value_t<In>, iter_reference_t<In>> &&
    Assignable<iter_value_t<In>&, iter_reference_t<In>>;
\end{codeblock}

\pnum
Let \tcode{i} be a dereferenceable value of type \tcode{In}.
\tcode{In} and \tcode{Out} model \tcode{IndirectlyCopyableStorable<In, Out>}
only if after the initialization of the object \tcode{obj} in
\begin{codeblock}
iter_value_t<In> obj(*i);
\end{codeblock}
\tcode{obj} is equal to the value previously denoted by \tcode{*i}. If
\tcode{iter_reference_t<In>} is an rvalue reference type, the resulting state
of the value denoted by \tcode{*i} is
valid but unspecified\cxxiref{lib.types.movedfrom}.

\rSec3[commonalgoreq.indirectlyswappable]{Concept \libconcept{IndirectlySwappable}}

\pnum
The \libconcept{IndirectlySwappable} concept specifies a swappable relationship
between the values referenced by two \libconcept{Readable} types.

\indexlibrary{\idxcode{IndirectlySwappable}}%
\begin{codeblock}
template<class I1, class I2 = I1>
  concept IndirectlySwappable =
    Readable<I1> && Readable<I2> &&
    requires(I1&& i1, I2&& i2) {
      ranges::iter_swap(std::forward<I1>(i1), std::forward<I2>(i2));
      ranges::iter_swap(std::forward<I2>(i2), std::forward<I1>(i1));
      ranges::iter_swap(std::forward<I1>(i1), std::forward<I1>(i1));
      ranges::iter_swap(std::forward<I2>(i2), std::forward<I2>(i2));
    };
\end{codeblock}

\pnum
Given an object \tcode{i1} of type \tcode{I1} and
an object \tcode{i2} of type \tcode{I2},
\tcode{I1} and \tcode{I2} model \tcode{IndirectlySwappable<I1, I2>} only if
after \tcode{ranges::iter_swap(i1, i2)}, the value of \tcode{*i1}
is equal to the value of \tcode{*i2} before the call, and \textit{vice versa}.

\rSec3[commonalgoreq.indirectlycomparable]{Concept \libconcept{IndirectlyComparable}}

\pnum
The \libconcept{IndirectlyComparable} concept specifies
the common requirements of algorithms that
compare values from two different sequences.

\indexlibrary{\idxcode{IndirectlyComparable}}%
\begin{codeblock}
template<class I1, class I2, class R, class P1 = identity,
         class P2 = identity>
  concept IndirectlyComparable =
    IndirectRelation<R, projected<I1, P1>, projected<I2, P2>>;
\end{codeblock}

\rSec3[commonalgoreq.permutable]{Concept \libconcept{Permutable}}

\pnum
The \libconcept{Permutable} concept specifies the common requirements
of algorithms that reorder elements in place by moving or swapping them.

\indexlibrary{\idxcode{Permutable}}%
\begin{codeblock}
template<class I>
  concept Permutable =
    ForwardIterator<I> &&
    IndirectlyMovableStorable<I, I> &&
    IndirectlySwappable<I, I>;
\end{codeblock}

\rSec3[commonalgoreq.mergeable]{Concept \libconcept{Mergeable}}

\pnum
The \libconcept{Mergeable} concept specifies the requirements of algorithms
that merge sorted sequences into an output sequence by copying elements.

\indexlibrary{\idxcode{Mergeable}}%
\begin{codeblock}
template<class I1, class I2, class Out, class R = ranges::less<>,
         class P1 = identity, class P2 = identity>
  concept Mergeable =
    InputIterator<I1> &&
    InputIterator<I2> &&
    WeaklyIncrementable<Out> &&
    IndirectlyCopyable<I1, Out> &&
    IndirectlyCopyable<I2, Out> &&
    IndirectStrictWeakOrder<R, projected<I1, P1>, projected<I2, P2>>;
\end{codeblock}

\rSec3[commonalgoreq.sortable]{Concept \libconcept{Sortable}}

\pnum
The \libconcept{Sortable} concept specifies the common requirements of
algorithms that permute sequences into ordered sequences (e.g., \tcode{sort}).

\indexlibrary{\idxcode{Sortable}}%
\begin{codeblock}
template<class I, class R = ranges::less<>, class P = identity>
  concept Sortable =
    Permutable<I> &&
    IndirectStrictWeakOrder<R, projected<I, P>>;
\end{codeblock}
\end{addedblock}

\rSec1[iterator.primitives]{Iterator primitives}

\pnum
To simplify the task of defining iterators, the library provides
several classes and functions:

\rSec2[std.iterator.tags]{Standard iterator tags}

\pnum
\indexlibrary{\idxcode{input_iterator_tag}}%
\indexlibrary{\idxcode{output_iterator_tag}}%
\indexlibrary{\idxcode{forward_iterator_tag}}%
\indexlibrary{\idxcode{bidirectional_iterator_tag}}%
\indexlibrary{\idxcode{random_access_iterator_tag}}%
\indexlibrary{\idxcode{contiguous_iterator_tag}}%
It is often desirable for a
function template specialization
to find out what is the most specific category of its iterator
argument, so that the function can select the most efficient algorithm at compile time.
To facilitate this, the
library introduces
\term{category tag}
classes which are used as compile time tags for algorithm selection.
They are:
\tcode{input_iterator_tag},
\tcode{output_iterator_tag},
\tcode{forward_iterator_tag},
\tcode{bidirectional_iterator_tag}\added{,}
\removed{and}
\tcode{random_access_iterator_tag}\added{,}
\added{and}
\tcode{\added{contiguous_iterator_tag}}.
For every iterator of type
\tcode{\changed{Iterator}{I}},
\tcode{iterator_traits<\changed{Iterator}{I}>::it\-er\-a\-tor_ca\-te\-go\-ry}
shall be defined to be the most specific category tag that describes the
iterator's behavior. \added{Additionally and optionally,
}\tcode{\added{iterator_traits<I>::it\-er\-a\-tor_con\-cept}}\added{
may be used to opt in or out of conformance to the iterator concepts defined
in\iref{iterator.concepts}.}

\begin{codeblock}
namespace std {
  struct input_iterator_tag { };
  struct output_iterator_tag { };
  struct forward_iterator_tag: @\removed{public}@ input_iterator_tag { };
  struct bidirectional_iterator_tag: @\removed{public}@ forward_iterator_tag { };
  struct random_access_iterator_tag: @\removed{public}@ bidirectional_iterator_tag { };
  @\added{struct contiguous_iterator_tag: random_access_iterator_tag \{ \};}@
}
\end{codeblock}

[...]

\rSec2[iterator.operations]{Iterator operations}

[...]

\begin{addedblock}
\rSec2[range.iterator.operations]{Range iterator operations}

\pnum
Since only types that model
\libconcept{RandomAccessIterator} provide the \tcode{+} operator, and
types that model \libconcept{Sized\-Sent\-inel} provide the \tcode{-}
operator, the library provides function templates
\tcode{advance}, \tcode{dist\-ance}, \tcode{next}, and \tcode{prev}.
These function templates use
\tcode{+}
and
\tcode{-}
for random access iterators and ranges that model \libconcept{SizedSentinel}
(and are, therefore, constant time for them); for output, input, forward and
bidirectional iterators they use
\tcode{++}
to provide linear time implementations.

\pnum
The function templates defined in this subclause are not found by
argument-dependent name lookup\cxxiref{basic.lookup.argdep}. When found by
unqualified\cxxiref{basic.lookup.unqual} name lookup for the
\grammarterm{postfix-expression} in a function call\cxxiref{expr.call}, they
inhibit argument-dependent name lookup.

\begin{example}
\begin{codeblock}
void foo() {
    using namespace std::ranges;
    std::vector<int> vec{1,2,3};
    distance(begin(vec), end(vec)); // \#1
}
\end{codeblock}
The function call expression at \tcode{\#1} invokes \tcode{std::ranges::distance},
not \tcode{std::distance}, despite that
(a) the iterator type returned from \tcode{begin(vec)} and \tcode{end(vec)}
may be associated with namespace \tcode{std} and
(b) \tcode{std::distance} is more specialized~(\cxxref{temp.func.order}) than
\tcode{std::ranges::distance} since the former requires its first two parameters
to have the same type.
\end{example}

\rSec3[range.iterator.operations.advance]{\tcode{ranges::advance}}
\indexlibrary{\idxcode{advance}}%
\begin{itemdecl}
template<Iterator I>
  constexpr void advance(I& i, iter_difference_t<I> n);
\end{itemdecl}

\begin{itemdescr}
\pnum
\expects
\tcode{n} shall be negative only for bidirectional iterators.

\pnum
\effects
For random access iterators, equivalent to \tcode{i += n}.
Otherwise, increments (or decrements for negative
\tcode{n})
iterator
\tcode{i}
by
\tcode{n}.
\end{itemdescr}

\indexlibrary{\idxcode{advance}}%
\begin{itemdecl}
template<Iterator I, Sentinel<I> S>
  constexpr void advance(I& i, S bound);
\end{itemdecl}

\begin{itemdescr}
\pnum
\expects
If \tcode{Assignable<I\&, S>} is not satisfied, \range{i}{bound}
shall denote a range.

\pnum
\effects
\begin{itemize}
\item If \tcode{I} and \tcode{S} model \tcode{Assignable<I\&, S>},
  equivalent to \tcode{i = std::move(bound)}.
\item Otherwise, if \tcode{S} and \tcode{I} model \tcode{SizedSentinel<S, I>},
  equivalent to \tcode{ranges::advance(i, bound - i)}.
\item Otherwise, increments \tcode{i} until \tcode{i == bound}.
\end{itemize}
\end{itemdescr}

\indexlibrary{\idxcode{advance}}%
\begin{itemdecl}
template<Iterator I, Sentinel<I> S>
  constexpr iter_difference_t<I> advance(I& i, iter_difference_t<I> n, S bound);
\end{itemdecl}

\begin{itemdescr}
\pnum
\expects
If \tcode{n > 0}, \range{i}{bound} shall denote a range.
If \tcode{n == 0}, \range{i}{bound} or \range{bound}{i} shall denote a range.
If \tcode{n < 0}, \range{bound}{i} shall denote a range,
\tcode{I} shall model \libconcept{BidirectionalIterator}, and
\tcode{I} and \tcode{S} shall model \tcode{Same<I, S>}.

\pnum
\effects
\begin{itemize}
\item If \tcode{S} and \tcode{I} model \tcode{SizedSentinel<S, I>}:
  \begin{itemize}
  \item If \brk{}$|\tcode{n}| >= |\tcode{bound - i}|$,
    equivalent to \tcode{ranges::advance(i, bound)}.
  \item Otherwise, equivalent to \tcode{ranges::advance(i, n)}.
  \end{itemize}
\item Otherwise, increments (or decrements for negative \tcode{n})
  iterator \tcode{i} either \tcode{n} times or until \tcode{i == bound},
  whichever comes first.
\end{itemize}

\pnum
\returns
\tcode{n - $M$}, where $M$ is the distance from the starting position of
\tcode{i} to the ending position.
\end{itemdescr}

\rSec3[range.iterator.operations.distance]{\tcode{ranges::distance}}
\indexlibrary{\idxcode{distance}}%
\begin{itemdecl}
template<Iterator I, Sentinel<I> S>
  constexpr iter_difference_t<I> distance(I first, S last);
\end{itemdecl}

\begin{itemdescr}
\pnum
\expects
\range{first}{last} shall denote a range, or
\tcode{S} and \tcode{I} shall model
\tcode{Same<S, I>} and \tcode{SizedSentinel<S, I>}
and \range{last}{first} shall denote a range.

\pnum
\effects
If \tcode{S} and \tcode{I} model \tcode{SizedSentinel<S, I>},
returns \tcode{(last - first)};
otherwise, returns the number of increments needed to get from
\tcode{first}
to
\tcode{last}.
\end{itemdescr}

\indexlibrary{\idxcode{distance}}%
\begin{itemdecl}
template<Range R>
  constexpr iter_difference_t<iterator_t<R>> distance(R&& r);
\end{itemdecl}

\begin{itemdescr}
\pnum
\effects
If \tcode{R} models \libconcept{SizedRange}, equivalent to:
\begin{codeblock}
return ranges::size(r); // \ref{range.primitives.size}
\end{codeblock}
Otherwise, equivalent to:
\begin{codeblock}
return ranges::distance(ranges::begin(r), ranges::end(r)); // \ref{range.access}
\end{codeblock}
\end{itemdescr}

\rSec3[range.iterator.operations.next]{\tcode{ranges::next}}
\indexlibrary{\idxcode{next}}%
\begin{itemdecl}
template<Iterator I>
  constexpr I next(I x);
\end{itemdecl}

\begin{itemdescr}
\pnum
\effects Equivalent to: \tcode{++x; return x;}
\end{itemdescr}

\indexlibrary{\idxcode{next}}%
\begin{itemdecl}
template<Iterator I>
  constexpr I next(I x, iter_difference_t<I> n);
\end{itemdecl}

\begin{itemdescr}
\pnum
\effects Equivalent to: \tcode{ranges::advance(x, n); return x;}
\end{itemdescr}

\indexlibrary{\idxcode{next}}%
\begin{itemdecl}
template<Iterator I, Sentinel<I> S>
  constexpr I next(I x, S bound);
\end{itemdecl}

\begin{itemdescr}
\pnum
\effects Equivalent to: \tcode{ranges::advance(x, bound); return x;}
\end{itemdescr}

\indexlibrary{\idxcode{next}}%
\begin{itemdecl}
template<Iterator I, Sentinel<I> S>
  constexpr I next(I x, iter_difference_t<I> n, S bound);
\end{itemdecl}

\begin{itemdescr}
\pnum
\effects Equivalent to: \tcode{ranges::advance(x, n, bound); return x;}
\end{itemdescr}

\rSec3[range.iterator.operations.prev]{\tcode{ranges::prev}}
\indexlibrary{\idxcode{prev}}%
\begin{itemdecl}
template<BidirectionalIterator I>
  constexpr I prev(I x);
\end{itemdecl}

\begin{itemdescr}
\pnum
\effects Equivalent to: \tcode{-{-}x; return x;}
\end{itemdescr}

\indexlibrary{\idxcode{prev}}%
\begin{itemdecl}
template<BidirectionalIterator I>
  constexpr I prev(I x, iter_difference_t<I> n);
\end{itemdecl}

\begin{itemdescr}
\pnum
\effects Equivalent to: \tcode{ranges::advance(x, -n); return x;}
\end{itemdescr}

\indexlibrary{\idxcode{prev}}%
\begin{itemdecl}
template<BidirectionalIterator I>
  constexpr I prev(I x, iter_difference_t<I> n, I bound);
\end{itemdecl}

\begin{itemdescr}
\pnum
\effects Equivalent to: \tcode{ranges::advance(x, -n, bound); return x;}
\end{itemdescr}
\end{addedblock}

\rSec1[predef.iterators]{Iterator adaptors}

\rSec2[reverse.iterators]{Reverse iterators}

\pnum
Class template \tcode{reverse_iterator} is an iterator adaptor that iterates
from the end of the sequence defined by its underlying iterator to the beginning
of that sequence.
\removed{The fundamental relation between a reverse iterator
and its corresponding iterator \tcode{i} is established by the identity:
\tcode{\&*(reverse_iterator(i)) == \&*(i - 1)}.}

\rSec3[reverse.iterator]{Class template \tcode{reverse_iterator}}

\indexlibrary{\idxcode{reverse_iterator}}%
\begin{codeblock}
namespace std {
  template<class Iterator>
  class reverse_iterator {
  public:
    using iterator_type     = Iterator;
    @\removed{using iterator_category}@ @\removed{= typename iterator_traits<Iterator>::iterator_category;}@
    @\removed{using value_type}@        @\removed{= typename iterator_traits<Iterator>::value_type;}@
    @\removed{using difference_type}@   @\removed{= typename iterator_traits<Iterator>::difference_type;}@
    @\added{using iterator_category}@ @\added{= \seebelownc;}@
    @\added{using iterator_concept}@  @\added{= \seebelownc;}@
    @\added{using value_type}@        @\added{= iter_value_t<Iterator>;}@
    @\added{using difference_type}@   @\added{= iter_difference_t<Iterator>;}@
    using pointer           = typename iterator_traits<Iterator>::pointer;
    @\removed{using reference}@         @\removed{= typename iterator_traits<Iterator>::reference;}@
    @\added{using reference}@         @\added{= iter_reference_t<Iterator>;}@

    constexpr reverse_iterator();
    constexpr explicit reverse_iterator(Iterator x);
    template<class U> constexpr reverse_iterator(const reverse_iterator<U>& u);
    template<class U> constexpr reverse_iterator& operator=(const reverse_iterator<U>& u);

    constexpr Iterator base() const;      @\removed{// explicit}@
    constexpr reference operator*() const;
    constexpr pointer   operator->() const @\added{requires \seebelownc}@;

    [...]

    constexpr reverse_iterator& operator-=(difference_type n);
    constexpr @\unspec@ operator[](difference_type n) const;

    @\added{friend constexpr iter_rvalue_reference_t<Iterator> iter_move(const reverse_iterator\& i)}@
      @\added{noexcept(\seebelownc);}@
    @\added{template<IndirectlySwappable<Iterator> Iterator2>}@
      @\added{friend constexpr void iter_swap(const reverse_iterator\& x,}@
                                      @\added{const reverse_iterator<Iterator2>\& y)}@
        @\added{noexcept(\seebelownc);}@

  protected:
    Iterator current;
  };

  [...]

  template<class Iterator>
    constexpr reverse_iterator<Iterator> make_reverse_iterator(Iterator i);

  @\added{template<class Iterator1, class Iterator2>}@
    @\added{requires \newtxt{(}!SizedSentinel<Iterator1, Iterator2>\newtxt{)}}@
  @\added{inline constexpr bool disable_sized_sentinel<reverse_iterator<Iterator1>,}@
                                               @\added{reverse_iterator<Iterator2>{>} = true;}@
}
\end{codeblock}

\begin{addedblock}
\pnum
The member \grammarterm{typedef-name} \tcode{iterator_category} denotes
\tcode{random_access_iterator_tag} if
\tcode{iterator_traits<\brk{}Iterator>::iterator_category} is derived from
\tcode{random_access_iterator_tag}, and
\tcode{iterator_traits<\brk{}Iterator>::iterator_category} otherwise.

\pnum
The member \grammarterm{typedef-name} \tcode{iterator_concept} denotes
\tcode{random_access_iterator_tag} if \tcode{Iterator} models
\libconcept{RandomAccessIterator}, and
\tcode{bidirectional_iterator_tag} otherwise.
\end{addedblock}

\rSec3[reverse.iter.requirements]{\tcode{reverse_iterator} requirements}

\pnum
The template parameter
\tcode{Iterator}
shall \changed{satisfy all}{either meet} the requirements of a
\oldconcept{BidirectionalIterator}\iref{bidirectional.iterators}
\added{or model
\libconcept{BidirectionalIterator}\iref{iterator.concept.bidirectional}}.

\pnum
Additionally,
\tcode{Iterator}
shall \changed{satisfy}{either meet} the requirements of a
\oldconcept{RandomAccessIterator}\iref{random.access.iterators}
\added{or model
\libconcept{RandomAccessIterator}\iref{iterator.concept.random.access}}
if any of the members
\tcode{operator+},
\tcode{operator-},
\tcode{operator+=},
\tcode{operator-=}\iref{reverse.iter.nav},
\tcode{operator[]}\iref{reverse.iter.elem},
or the non-member operators\iref{reverse.iter.cmp}
\tcode{operator<},
\tcode{operator>},
\tcode{operator<=},
\tcode{operator>=},
\tcode{operator-},
or
\tcode{operator+}\iref{reverse.iter.nonmember}
are referenced in a way that requires instantiation\cxxiref{temp.inst}.


\setcounter{subsubsection}{4}
\rSec3[reverse.iter.elem]{\tcode{reverse_iterator} element access}

[...]

\ednote{This change incorporates the PR of
\href{https://wg21.link/lwg1052}{LWG 1052}):}

\setcounter{Paras}{1}

\indexlibrarymember{operator->}{reverse_iterator}%
\begin{itemdecl}
constexpr pointer operator->() const @\added{requires is_pointer_v<Iterator>}@
  @\added{|| requires(const Iterator i) \{ i.operator->(); \}}@;
\end{itemdecl}

\begin{itemdescr}
\pnum
\removed{\returns \tcode{addressof(operator*())}.}
\begin{addedblock}
\effects
\begin{itemize}
\item If \tcode{Iterator} is a pointer type, equivalent to: \tcode{return prev(current);}

\item Otherwise, equivalent to: \tcode{return prev(current).operator->();}
\end{itemize}
\end{addedblock}
\end{itemdescr}

[...]

\rSec3[reverse.iter.nav]{\tcode{reverse_iterator} navigation}

[...]

\rSec3[reverse.iter.cmp]{\tcode{reverse_iterator} comparisons}
\ednote{Insert a new initial paragraph:}

\pnum
\added{The functions in this subsection only participate in overload resolution
if the expression in their \textit{Returns:} element is well-formed and
implicitly convertible to \tcode{bool}.}

% \ednote{Should we treat the random access traversal member and non-member functions similarly?}

[...]

\rSec3[reverse.iter.nonmember]{Non-member functions}

[...]

\setcounter{Paras}{1}
\begin{itemdescr}
\pnum
\returns
\tcode{reverse_iterator<Iterator> (x.current - n)}.
\end{itemdescr}

\begin{addedblock}
\indexlibrarymember{iter_move}{reverse_iterator}%
\begin{itemdecl}
friend constexpr iter_rvalue_reference_t<Iterator> iter_move(const reverse_iterator& i)
   noexcept(@\seebelownc@);
\end{itemdecl}

\begin{itemdescr}
\pnum
\effects Equivalent to: \tcode{return ranges::iter_move(prev(i.current));}

\pnum
\remarks The expression in \tcode{noexcept} is equivalent to:
\begin{codeblock}
   noexcept(ranges::iter_move(declval<Iterator&>())) && noexcept(--declval<Iterator&>()) &&
     is_nothrow_copy_constructible_v<Iterator>
\end{codeblock}
\end{itemdescr}

\indexlibrarymember{iter_swap}{reverse_iterator}%
\begin{itemdecl}
template<IndirectlySwappable<Iterator> Iterator2>
  friend constexpr void iter_swap(const reverse_iterator& x, const reverse_iterator<Iterator2>& y)
    noexcept(@\seebelownc@);
\end{itemdecl}

\begin{itemdescr}
\pnum
\effects Equivalent to \tcode{ranges::iter_swap(ranges::prev(x.current), ranges::prev(y.current))}.

\pnum
\remarks The expression in \tcode{noexcept} is equivalent to:
\begin{codeblock}
  noexcept(ranges::iter_swap(declval<Iterator>(), declval<Iterator>())) &&
    noexcept(--declval<Iterator&>() && is_nothrow_copy_constructible_v<Iterator>)
\end{codeblock}
\end{itemdescr}
\end{addedblock}

\indexlibrary{\idxcode{reverse_iterator}!\idxcode{make_reverse_iterator} non-member function}%
\indexlibrary{\idxcode{make_reverse_iterator}}%
\begin{itemdecl}
template<class Iterator>
  constexpr reverse_iterator<Iterator> make_reverse_iterator(Iterator i);
\end{itemdecl}

[...]

\rSec2[insert.iterators]{Insert iterators}

[...]

\rSec3[back.insert.iterator]{Class template \tcode{back_insert_iterator}}

\indexlibrary{\idxcode{back_insert_iterator}}%
\begin{codeblock}
namespace std {
  template<class Container>
  class back_insert_iterator {
  protected:
    Container* container @\added{= nullptr}@;

  public:
    using iterator_category = output_iterator_tag;
    using value_type        = void;
    using difference_type   = @\changed{void}{ptrdiff_t}@;
    using pointer           = void;
    using reference         = void;
    using container_type    = Container;

    @\added{constexpr back_insert_iterator() noexcept = default;}@
    explicit back_insert_iterator(Container& x);
    back_insert_iterator& operator=(const typename Container::value_type& value);
    back_insert_iterator& operator=(typename Container::value_type&& value);

    back_insert_iterator& operator*();
    back_insert_iterator& operator++();
    back_insert_iterator  operator++(int);
  };

  template<class Container>
    back_insert_iterator<Container> back_inserter(Container& x);
}
\end{codeblock}

[...]

\rSec3[front.insert.iterator]{Class template \tcode{front_insert_iterator}}

\indexlibrary{\idxcode{front_insert_iterator}}%
\begin{codeblock}
namespace std {
  template<class Container>
  class front_insert_iterator {
  protected:
    Container* container @\added{= nullptr}@;

  public:
    using iterator_category = output_iterator_tag;
    using value_type        = void;
    using difference_type   = @\changed{void}{ptrdiff_t}@;
    using pointer           = void;
    using reference         = void;
    using container_type    = Container;

    @\added{constexpr front_insert_iterator() noexcept = default;}@
    explicit front_insert_iterator(Container& x);
    front_insert_iterator& operator=(const typename Container::value_type& value);
    front_insert_iterator& operator=(typename Container::value_type&& value);

    front_insert_iterator& operator*();
    front_insert_iterator& operator++();
    front_insert_iterator  operator++(int);
  };

  template<class Container>
    front_insert_iterator<Container> front_inserter(Container& x);
}
\end{codeblock}

[...]

\rSec3[insert.iterator]{Class template \tcode{insert_iterator}}

\indexlibrary{\idxcode{insert_iterator}}%
\begin{codeblock}
namespace std {
  template<class Container>
  class insert_iterator {
  protected:
    Container* container @\added{= nullptr}@;
    @\changed{typename Container::iterator}{iterator_t<Container>}@ iter @\added{\{\}}@;

  public:
    using iterator_category = output_iterator_tag;
    using value_type        = void;
    using difference_type   = @\changed{void}{ptrdiff_t}@;
    using pointer           = void;
    using reference         = void;
    using container_type    = Container;

    @\added{insert_iterator() = default;}@
    insert_iterator(Container& x, @\changed{typename Container::iterator}{iterator_t<Container>}@ i);
    insert_iterator& operator=(const typename Container::value_type& value);
    insert_iterator& operator=(typename Container::value_type&& value);

    insert_iterator& operator*();
    insert_iterator& operator++();
    insert_iterator& operator++(int);
  };

  template<class Container>
    insert_iterator<Container> inserter(Container& x, @\changed{typename Container::iterator}{iterator_t<Container>}@ i);
}
\end{codeblock}

\rSec4[insert.iter.ops]{\tcode{insert_iterator} operations}

\indexlibrary{\idxcode{insert_iterator}!constructor}%
\begin{itemdecl}
insert_iterator(Container& x, @\changed{typename Container::iterator}{iterator_t<Container>}@ i);
\end{itemdecl}

[...]

\rSec4[inserter]{\tcode{inserter}}

\indexlibrary{\idxcode{inserter}}%
\begin{itemdecl}
template<class Container>
  insert_iterator<Container> inserter(Container& x, @\changed{typename Container::iterator}{iterator_t<Container>}@ i);
\end{itemdecl}

\begin{itemdescr}
\pnum
\returns
\tcode{insert_iterator<Container>(x, i)}.
\end{itemdescr}


\ednote{Retitle [move.iterators] to ``Move iterators and sentinels'' and modify as follows:}
\rSec2[move.iterators]{Move iterators and sentinels}

[...]

\rSec3[move.iterator]{Class template \tcode{move_iterator}}

\indexlibrary{\idxcode{move_iterator}}%
\begin{codeblock}
namespace std {
  template<class Iterator>
  class move_iterator {
  public:
    using iterator_type     = Iterator;
    using iterator_category = @\changed{typename iterator_traits<Iterator>::iterator_category}{\seebelownc}@;
    using value_type        = @\changed{typename iterator_traits<Iterator>::value_type}{iter_value_t<Iterator>}@;
    using difference_type   = @\changed{typename iterator_traits<Iterator>::difference_type}{iter_difference_t<Iterator>}@;
    using pointer           = Iterator;
    using reference         = @\changed{\seebelow}{iter_rvalue_reference_t<Iterator>}@;
    @\added{using iterator_concept}@  @\added{= input_iterator_tag;}@

    constexpr move_iterator();
    constexpr explicit move_iterator(Iterator i);

    [...]

    constexpr move_iterator& operator++();
    constexpr @\changed{move_iterator}{decltype(auto)}@ operator++(int);
    constexpr move_iterator& operator--();

    [...]

    constexpr move_iterator operator-(difference_type n) const;
    constexpr move_iterator& operator-=(difference_type n);
    constexpr @\changed{\unspec}{reference}@ operator[](difference_type n) const;

    @\added{template<Sentinel<Iterator> S>}@
      @\added{friend constexpr bool operator==(}@
        @\added{const move_iterator\& x, const move_sentinel<S>\& y);}@
    @\added{template<Sentinel<Iterator> S>}@
      @\added{friend constexpr bool operator==(}@
        @\added{const move_sentinel<S>\& x, const move_iterator\& y);}@
    @\added{template<Sentinel<Iterator> S>}@
      @\added{friend constexpr bool operator!=(}@
        @\added{const move_iterator\& x, const move_sentinel<S>\& y);}@
    @\added{template<Sentinel<Iterator> S>}@
      @\added{friend constexpr bool operator!=(}@
        @\added{const move_sentinel<S>\& x, const move_iterator\& y);}@

    @\added{template<SizedSentinel<Iterator> S>}@
      @\added{friend constexpr iter_difference_t<Iterator> operator-(}@
        @\added{const move_sentinel<S>\& x, const move_iterator\& y);}@
    @\added{template<SizedSentinel<Iterator> S>}@
      @\added{friend constexpr iter_difference_t<Iterator> operator-(}@
        @\added{const move_iterator\& x, const move_sentinel<S>\& y);}@

    @\added{friend constexpr iter_rvalue_reference_t<Iterator> iter_move(const move_iterator\& i)}@
      @\added{noexcept(noexcept(ranges::iter_move(i.current)));}@
    @\added{template<IndirectlySwappable<Iterator> Iterator2>}@
      @\added{friend constexpr void iter_swap(const move_iterator\& x, const move_iterator<Iterator2>\& y)}@
        @\added{noexcept(noexcept(ranges::iter_swap(x.current, y.current)));}@

  private:
    Iterator current;   // \expos
  };

  [...]

  template<class Iterator>
    constexpr move_iterator<Iterator> operator+(
      @\changed{typename move_iterator<Iterator>::difference_type}{iter_difference_t<Iterator>}@ n,
      const move_iterator<Iterator>& x);
  template<class Iterator>
    constexpr move_iterator<Iterator> make_move_iterator(Iterator i);
}
\end{codeblock}

\begin{removedblock}
\pnum
Let \tcode{\placeholder{R}} denote \tcode{iterator_traits<Iterator>::reference}.
If \tcode{is_reference_v<\placeholder{R}>} is \tcode{true},
the template specialization \tcode{move_iterator<Iterator>} shall define
the nested type named \tcode{reference} as a synonym for
\tcode{remove_reference_t<\placeholder{R}>\&\&},
otherwise as a synonym for \tcode{\placeholder{R}}.
\end{removedblock}

\begin{addedblock}
\pnum
The member \grammarterm{typedef-name} \tcode{iterator_category} denotes
\tcode{random_access_iterator_tag} if
\tcode{iterator_traits<\brk{}Iterator>::iterator_category} is derived from
\tcode{random_access_iterator_tag}, and
\tcode{iterator_traits<\brk{}Iterator>::iterator_category} otherwise.
\end{addedblock}

\rSec3[move.iter.requirements]{\tcode{move_iterator} requirements}

\pnum
The template parameter \tcode{Iterator} shall \changed{satisfy}{either meet}
the \oldconcept{InputIterator} requirements\iref{input.iterators}
\added{or model \libconcept{InputIterator}\iref{iterator.concept.input}}.
Additionally, if any of the bidirectional \removed{or random access} traversal
functions are instantiated, the template parameter shall \changed{satisfy}{either meet} the
\oldconcept{BidirectionalIterator} requirements\iref{bidirectional.iterators} or
\added{model \libconcept{BidirectionalIterator}\iref{iterator.concept.bidirectional}}
\removed{\oldconcept{RandomAccessIterator} requirements\iref{random.access.iterators}, respectively}.
\added{If any of the random access traversal functions are instantiated, the
template parameter shall either meet the \oldconcept{RandomAccessIterator}
requirements\iref{random.access.iterators} or model
\libconcept{RandomAccess\-Iterator}\iref{iterator.concept.random.access}.}

[...]

\setcounter{subsubsection}{4}
\rSec3[move.iter.elem]{\tcode{move_iterator} element access}

\indexlibrarymember{operator*}{move_iterator}%
\begin{itemdecl}
constexpr reference operator*() const;
\end{itemdecl}

\begin{itemdescr}
\pnum
\removed{\returns \tcode{static_cast<reference>(*current)}.}

\added{\effects Equivalent to: }\tcode{\added{return ranges::iter_move(current);}}
\end{itemdescr}

\ednote{My preference is to remove \tcode{operator->} since for \tcode{move_iterator},
the expressions \tcode{(*i).m} and \tcode{i->m} are not, and cannot be,
equivalent. I am leaving the operator as-is in an excess of caution; perhaps we
should consider deprecation for C++20?}

\indexlibrarymember{operator->}{move_iterator}%
\begin{itemdecl}
constexpr pointer operator->() const;
\end{itemdecl}

\begin{itemdescr}
\pnum
\returns \tcode{current}.
\end{itemdescr}

\indexlibrarymember{operator[]}{move_iterator}%
\begin{itemdecl}
constexpr @\changed{\unspec}{reference}@ operator[](difference_type n) const;
\end{itemdecl}

\begin{itemdescr}
\pnum
\removed{\returns }\tcode{\removed{std::move(current[n])}}\removed{.}

\added{\effects Equivalent to: }\tcode{\added{ranges::iter_move(current + n);}}
\end{itemdescr}

\rSec3[move.iter.nav]{\tcode{move_iterator} navigation}

[...]

\setcounter{Paras}{2}
\indexlibrarymember{operator++}{move_iterator}%
\begin{itemdecl}
constexpr @\changed{move_iterator}{decltype(auto)}@ operator++(int);
\end{itemdecl}

\begin{itemdescr}
\pnum
\effects
\changed{As if by}{If \tcode{Iterator} models \libconcept{ForwardIterator}, equivalent to}:
\begin{codeblock}
move_iterator tmp = *this;
++current;
return tmp;
\end{codeblock}
\added{Otherwise, equivalent to \tcode{++current}.}
\end{itemdescr}

[...]

\rSec3[move.iter.op.comp]{\tcode{move_iterator} comparisons}

\pnum
\added{The functions in this subsection only participate in overload resolution if the
expression in their \textit{Returns:} element is well-formed.}

\indexlibrarymember{operator==}{move_iterator}%
\begin{itemdecl}
template<class Iterator1, class Iterator2>
constexpr bool operator==(const move_iterator<Iterator1>& x, const move_iterator<Iterator2>& y);
@\added{template<Sentinel<Iterator> S>}@
@\added{friend constexpr bool operator==(const move_iterator\& x, const move_sentinel<S>\& y);}@
@\added{template<Sentinel<Iterator> S>}@
@\added{friend constexpr bool operator==(const move_sentinel<S>\& x, const move_iterator\& y);}@
\end{itemdecl}

\begin{itemdescr}
\pnum
\returns \tcode{x.base() == y.base()}.
\end{itemdescr}

\indexlibrarymember{operator"!=}{move_iterator}%
\begin{itemdecl}
template<class Iterator1, class Iterator2>
constexpr bool operator!=(const move_iterator<Iterator1>& x, const move_iterator<Iterator2>& y);
@\added{template<Sentinel<Iterator> S>}@
@\added{friend constexpr bool operator!=(const move_iterator\& x, const move_sentinel<S>\& y);}@
@\added{template<Sentinel<Iterator> S>}@
@\added{friend constexpr bool operator!=(const move_sentinel<S>\& x, const move_iterator\& y);}@
\end{itemdecl}

\begin{itemdescr}
\pnum
\returns \tcode{!(x == y)}.
\end{itemdescr}

[...]

\rSec3[move.iter.nonmember]{\tcode{move_iterator} non-member functions}

\pnum
\added{The functions in this subsection only participate in overload resolution if the
expression in their \textit{Returns:} element is well-formed.}

\indexlibrarymember{operator-}{move_iterator}%
\begin{itemdecl}
template<class Iterator1, class Iterator2>
  constexpr auto operator-(
    const move_iterator<Iterator1>& x,
    const move_iterator<Iterator2>& y) -> decltype(x.base() - y.base());
@\added{template<SizedSentinel<Iterator> S>}@
@\added{friend constexpr iter_difference_t<Iterator> operator-(}@
    @\added{const move_sentinel<S>\& x, const move_iterator\& y);}@
@\added{template<SizedSentinel<Iterator> S>}@
@\added{friend constexpr iter_difference_t<Iterator> operator-(}@
    @\added{const move_iterator\& x, const move_sentinel<S>\& y);}@
\end{itemdecl}

\begin{itemdescr}
\pnum
\returns \tcode{x.base() - y.base()}.
\end{itemdescr}

\indexlibrarymember{operator+}{move_iterator}%
\begin{itemdecl}
template<class Iterator>
  constexpr move_iterator<Iterator> operator+(
    @\changed{typename move_iterator<Iterator>::difference_type}{iter_difference_t<Iterator>}@ n,
    const move_iterator<Iterator>& x);
\end{itemdecl}

\begin{itemdescr}
\pnum
\returns \tcode{x + n}.
\end{itemdescr}

\begin{addedblock}
\indexlibrarymember{iter_move}{move_iterator}%
\begin{itemdecl}
friend constexpr iter_rvalue_reference_t<Iterator> iter_move(const move_iterator& i)
  noexcept(noexcept(ranges::iter_move(i.current)));
\end{itemdecl}

\begin{itemdescr}
\pnum
\effects Equivalent to: \tcode{return ranges::iter_move(i.current);}
\end{itemdescr}

\indexlibrarymember{iter_swap}{move_iterator}%
\begin{itemdecl}
template<IndirectlySwappable<Iterator> Iterator2>
  friend constexpr void iter_swap(const move_iterator& x, const move_iterator<Iterator2>& y)
    noexcept(noexcept(ranges::iter_swap(x.current, y.current)));
\end{itemdecl}

\begin{itemdescr}
\pnum
\effects Equivalent to: \tcode{ranges::iter_swap(x.current, y.current)}.
\end{itemdescr}
\end{addedblock}

\indexlibrary{\idxcode{make_move_iterator}}%
\begin{itemdecl}
template<class Iterator>
constexpr move_iterator<Iterator> make_move_iterator(Iterator i);
\end{itemdecl}

\begin{itemdescr}
\pnum
\returns \tcode{move_iterator<Iterator>(i)}.
\end{itemdescr}


\begin{addedblock}
\rSec3[move.sentinel]{Class template \tcode{move_sentinel}}

\pnum
Class template \tcode{move_sentinel} is a sentinel adaptor useful for denoting
ranges together with \tcode{move_iterator}. When an input iterator type
\tcode{I} and sentinel type \tcode{S} model \tcode{Sentinel<S, I>},
\tcode{move_sentinel<S>} and \tcode{move_iterator<I>} model
\tcode{Sentinel<move_sentinel<S>, move_iterator<I>{>}} as well.

\pnum
\begin{example}
A \tcode{move_if} algorithm is easily implemented with
\tcode{copy_if} using \tcode{move_iterator} and \tcode{move_sentinel}:

\begin{codeblock}
template<InputIterator I, Sentinel<I> S, WeaklyIncrementable O,
         IndirectUnaryPredicate<I> Pred>
  requires IndirectlyMovable<I, O>
void move_if(I first, S last, O out, Pred pred) {
  std::ranges::copy_if(move_iterator<I>{first}, move_sentinel<S>{last}, out, pred);
}
\end{codeblock}
\end{example}

\indexlibrary{\idxcode{move_sentinel}}%
\begin{codeblock}
namespace std {
  template<Semiregular S>
  class move_sentinel {
  public:
    constexpr move_sentinel();
    explicit constexpr move_sentinel(S s);
    template<ConvertibleTo<S> S2>
      constexpr move_sentinel(const move_sentinel<S2>& s);
    template<ConvertibleTo<S> S2>
      constexpr move_sentinel& operator=(const move_sentinel<S2>& s);

    constexpr S base() const;

  private:
    S last; // \expos
  };
}
\end{codeblock}

\rSec3[move.sent.ops]{\tcode{move_sentinel} operations}

\indexlibrary{\idxcode{move_sentinel}!\idxcode{move_sentinel}}%
\begin{itemdecl}
constexpr move_sentinel();
\end{itemdecl}

\begin{itemdescr}
\pnum
\effects Value-initializes \tcode{last}.
If \tcode{is_trivially_default_constructible_v<S>} is \tcode{true},
then this constructor is a \tcode{constexpr} constructor.
\end{itemdescr}

\indexlibrary{\idxcode{move_sentinel}!constructor}%
\begin{itemdecl}
explicit constexpr move_sentinel(S s);
\end{itemdecl}

\begin{itemdescr}
\pnum
\effects Initializes \tcode{last} with \tcode{std::move(s)}.
\end{itemdescr}

\indexlibrary{\idxcode{move_sentinel}!constructor}%
\begin{itemdecl}
template<ConvertibleTo<S> S2>
  constexpr move_sentinel(const move_sentinel<S2>& s);
\end{itemdecl}

\begin{itemdescr}
\pnum
\effects Initializes \tcode{last} with \tcode{s.last}.
\end{itemdescr}

\indexlibrary{\idxcode{operator=}!\idxcode{move_sentinel}}%
\indexlibrary{\idxcode{move_sentinel}!\idxcode{operator=}}%
\begin{itemdecl}
template<ConvertibleTo<S> S2>
  constexpr move_sentinel& operator=(const move_sentinel<S2>& s);
\end{itemdecl}

\begin{itemdescr}
\pnum
\effects  Equivalent to: \tcode{last = s.last; return *this;}
\end{itemdescr}


\rSec2[iterators.common]{Common iterators}

\pnum
Class template \tcode{common_iterator} is an iterator/sentinel adaptor that is
capable of representing a non-common range of elements (where the types of the
iterator and sentinel differ) as a common range (where they are the same). It
does this by holding either an iterator or a sentinel, and implementing the
equality comparison operators appropriately.

\pnum
\begin{note}
The \tcode{common_iterator} type is useful for interfacing with legacy
code that expects the begin and end of a range to have the same type.
\end{note}

\pnum
\begin{example}
\begin{codeblock}
template<class ForwardIterator>
void fun(ForwardIterator begin, ForwardIterator end);

list<int> s;
// populate the list \tcode{s}
using CI =
  common_iterator<counted_iterator<list<int>::iterator>,
                  default_sentinel>;
// call \tcode{fun} on a range of 10 ints
fun(CI(@\oldtxt{make_}@counted_iterator(s.begin(), 10)),
    CI(default_sentinel()));
\end{codeblock}
\end{example}

\rSec3[common.iterator]{Class template \tcode{common_iterator}}

\indexlibrary{\idxcode{common_iterator}}%
\begin{codeblock}
namespace std {
  template<Iterator I, Sentinel<I> S>
    requires @\newtxt{(}@!Same<I, S>@\newtxt{)}@
  class common_iterator {
  public:
    using difference_type = iter_difference_t<I>;

    constexpr common_iterator() = default;
    constexpr common_iterator(I i);
    constexpr common_iterator(S s);
    template<ConvertibleTo<I> I2, ConvertibleTo<S> S2>
      constexpr common_iterator(const common_iterator<I2, S2>& x);

    template<ConvertibleTo<I> I2, ConvertibleTo<S> S2>
      common_iterator& operator=(const common_iterator<I2, S2>& x);

    decltype(auto) operator*();
    decltype(auto) operator*() const
      requires @\placeholder{dereferenceable}@<const I>;
    decltype(auto) operator->() const
      requires @\seebelownc@;

    common_iterator& operator++();
    decltype(auto) operator++(int);

    template<class I2, Sentinel<I> S2>
      requires Sentinel<S, I2>
    friend bool operator==(
      const common_iterator& x, const common_iterator<I2, S2>& y);
    template<class I2, Sentinel<I> S2>
      requires Sentinel<S, I2> && EqualityComparableWith<I, I2>
    friend bool operator==(
      const common_iterator& x, const common_iterator<I2, S2>& y);
    template<class I2, Sentinel<I> S2>
      requires Sentinel<S, I2>
    friend bool operator!=(
      const common_iterator& x, const common_iterator<I2, S2>& y);

    template<SizedSentinel<I> I2, SizedSentinel<I> S2>
      requires SizedSentinel<S, I2>
    friend iter_difference_t<I2> operator-(
      const common_iterator& x, const common_iterator<I2, S2>& y);

    friend iter_rvalue_reference_t<I> iter_move(const common_iterator& i)
      noexcept(noexcept(ranges::iter_move(declval<const I&>())))
        requires InputIterator<I>;
    template<IndirectlySwappable<I> I2, class S2>
      friend void iter_swap(const common_iterator& x, const common_iterator<I2, S2>& y)
        noexcept(noexcept(ranges::iter_swap(declval<const I&>(), declval<const I2&>())));

  private:
    variant<I, S> v_{}; // \expos
  };

  template<Readable I, class S>
  struct readable_traits<common_iterator<I, S>> {
    using value_type = iter_value_t<I>;
  };

  template<InputIterator I, class S>
  struct iterator_traits<common_iterator<I, S>> {
    using difference_type = iter_difference_t<I>;
    using value_type = iter_value_t<I>;
    using reference = iter_reference_t<I>;
    using pointer = @\seebelownc@;
    using iterator_category = @\seebelownc@;
    using iterator_concept = @\seebelownc@;
  };
}
\end{codeblock}

\rSec3[common.iterator.traits]{\tcode{iterator_traits} for \tcode{common_iterator}}

\pnum
The nested \grammarterm{typedef-name}s of the specialization of
\tcode{iterator_traits} for \tcode{common_iterator<I, S>} are defined as follows.

\pnum
If the expression \tcode{a.operator->()} is well-formed, where \tcode{a}
is an lvalue of type \tcode{const common_iterator<I, S>}, then
\tcode{pointer} denotes the type of that expression. Otherwise, \tcode{pointer}
denotes \tcode{void}.

\pnum
Let \tcode{C} denote the type \tcode{iterator_traits<I>::iterator_category}. If
\tcode{C} models \tcode{DerivedFrom<forward_iterator_tag>},
\tcode{iterator_category} denotes \tcode{forward_iterator_tag}. Otherwise,
\tcode{iterator_category} denotes \tcode{input_iterator_tag}.

\pnum
\tcode{iterator_concept} denotes \tcode{forward_iterator_tag} if \tcode{I}
models \libconcept{ForwardIterator}; otherwise it denotes
\tcode{input_iterator_tag}.

\rSec3[common.iterator.ops]{\tcode{common_iterator} operations}

\rSec4[common.iterator.op.const]{\tcode{common_iterator} constructors and conversions}

\indexlibrary{\idxcode{common_iterator}!constructor}%
\begin{itemdecl}
constexpr common_iterator(I i);
\end{itemdecl}

\begin{itemdescr}
\pnum
\effects
Initializes \tcode{v_} as if by \tcode{v_\{in_place_type<I>, std::move(i)\}}.
\end{itemdescr}

\indexlibrary{\idxcode{common_iterator}!constructor}%
\begin{itemdecl}
constexpr common_iterator(S s);
\end{itemdecl}

\begin{itemdescr}
\pnum
\effects Initializes \tcode{v_} as if by
\tcode{v_\{in_place_type<S>, std::move(s)\}}.
\end{itemdescr}

\indexlibrary{\idxcode{common_iterator}!constructor}%
\begin{itemdecl}
template<ConvertibleTo<I> I2, ConvertibleTo<S> S2>
  constexpr common_iterator(const common_iterator<I2, S2>& x);
\end{itemdecl}

\begin{itemdescr}
\pnum
\expects \tcode{x.v_.valueless_by_exception()} is \tcode{false}.

\pnum
\effects
Initializes \tcode{v_} as if by
\tcode{v_\{in_place_index<$i$>, get<$i$>(x.v_)\}},
where $i$ is \tcode{x.v_.index()}.
\end{itemdescr}

\indexlibrary{\idxcode{operator=}!\idxcode{common_iterator}}%
\indexlibrary{\idxcode{common_iterator}!\idxcode{operator=}}%
\begin{itemdecl}
template<ConvertibleTo<I> I2, ConvertibleTo<S> S2>
  common_iterator& operator=(const common_iterator<I2, S2>& x);
\end{itemdecl}

\begin{itemdescr}
\pnum
\expects \tcode{x.v_.valueless_by_exception()} is \tcode{false}.

\pnum
\effects
Equivalent to:
\begin{itemize}
\item If \tcode{v_.index() == x.v_.index()}, then
\tcode{get<$i$>(v_) = get<$i$>(x.v_)}.

\item Otherwise, \tcode{v_.emplace<$i$>(get<$i$>(x.v_))}.
\end{itemize}
where $i$ is \tcode{x.v_.index()}.

\pnum
\returns \tcode{*this}
\end{itemdescr}

\rSec4[common.iterator.op.star]{\tcode{common_iterator::operator*}}

\indexlibrary{\idxcode{operator*}!\idxcode{common_iterator}}%
\indexlibrary{\idxcode{common_iterator}!\idxcode{operator*}}%
\begin{itemdecl}
decltype(auto) operator*();
decltype(auto) operator*() const
  requires @\placeholder{dereferenceable}@<const I>;
\end{itemdecl}

\begin{itemdescr}
\pnum
\expects \tcode{holds_alternative<I>(v_)}.

\pnum
\effects Equivalent to: \tcode{return *get<I>(v_);}
\end{itemdescr}

\rSec4[common.iterator.op.ref]{\tcode{common_iterator::operator->}}

\indexlibrary{\idxcode{operator->}!\idxcode{common_iterator}}%
\indexlibrary{\idxcode{common_iterator}!\idxcode{operator->}}%
\begin{itemdecl}
decltype(auto) operator->() const
  requires @\seebelownc@;
\end{itemdecl}

\begin{itemdescr}
\pnum
\expects \tcode{holds_alternative<I>(v_)}.

\pnum
\effects
\begin{itemize}
\item
If \tcode{I} is a pointer type or if the expression
\tcode{get<I>(v_).operator->()} is
well-formed, equivalent to: \tcode{return get<I>(v_);}

\item
Otherwise, if \tcode{iter_reference_t<I>} is a reference type, equivalent to:
\begin{codeblock}
auto&& tmp = *get<I>(v_);
return addressof(tmp);
\end{codeblock}

\item
Otherwise, equivalent to: \tcode{return \placeholdernc{proxy}(*get<I>(v_));} where
\tcode{\placeholder{proxy}} is the exposition-only class:
\begin{codeblock}
class @\placeholder{proxy}@ {
  iter_value_t<I> keep_;
  @\placeholdernc{proxy}@(iter_reference_t<I>&& x)
    : keep_(std::move(x)) {}
public:
  const iter_value_t<I>* operator->() const {
    return addressof(keep_);
  }
};
\end{codeblock}
\end{itemize}

\pnum
The expression in the requires clause is equivalent to:
\begin{codeblock}
Readable<const I> &&
  (requires(const I& i) { i.operator->(); } ||
   is_reference_v<iter_reference_t<I>> ||
   Constructible<iter_value_t<I>, iter_reference_t<I>>)
\end{codeblock}
\end{itemdescr}

\rSec4[common.iterator.op.incr]{\tcode{common_iterator::operator++}}

\indexlibrary{\idxcode{operator++}!\idxcode{common_iterator}}%
\indexlibrary{\idxcode{common_iterator}!\idxcode{operator++}}%
\begin{itemdecl}
common_iterator& operator++();
\end{itemdecl}

\begin{itemdescr}
\pnum
\expects \tcode{holds_alternative<I>(v_)}.

\pnum
\effects Equivalent to \tcode{++get<I>(v_)}.

\pnum
\returns \tcode{*this}.
\end{itemdescr}

\indexlibrary{\idxcode{operator++}!\idxcode{common_iterator}}%
\indexlibrary{\idxcode{common_iterator}!\idxcode{operator++}}%
\begin{itemdecl}
decltype(auto) operator++(int);
\end{itemdecl}

\begin{itemdescr}
\pnum
\expects \tcode{holds_alternative<I>(v_)}.

\pnum
\effects
If \tcode{I} models \libconcept{ForwardIterator}, equivalent to:
\begin{codeblock}
common_iterator tmp = *this;
++*this;
return tmp;
\end{codeblock}
Otherwise, equivalent to: \tcode{return get<I>(v_)++;}
\end{itemdescr}

\rSec4[common.iterator.op.comp]{\tcode{common_iterator} comparisons}

\indexlibrary{\idxcode{operator==}!\idxcode{common_iterator}}%
\indexlibrary{\idxcode{common_iterator}!\idxcode{operator==}}%
\begin{itemdecl}
template<class I2, Sentinel<I1> S2>
  requires Sentinel<S1, I2>
friend bool operator==(
  const common_iterator& x, const common_iterator<I2, S2>& y);
\end{itemdecl}

\begin{itemdescr}
\pnum
\expects
\tcode{x.v_.valueless_by_exception()} and \tcode{y.v_.valueless_by_exception()}
are each \tcode{false}.

\pnum
\returns
\tcode{true} if \tcode{$i$ == $j$},
and otherwise \tcode{get<$i$>(x.v_) == get<$j$>(y.v_)},
where $i$ is \tcode{x.v_.index()} and $j$ is \tcode{y.v_.index()}.
\end{itemdescr}

\indexlibrary{\idxcode{operator==}!\idxcode{common_iterator}}%
\indexlibrary{\idxcode{common_iterator}!\idxcode{operator==}}%
\begin{itemdecl}
template<class I2, Sentinel<I1> S2>
  requires Sentinel<S1, I2> && EqualityComparableWith<I1, I2>
friend bool operator==(
  const common_iterator& x, const common_iterator<I2, S2>& y);
\end{itemdecl}

\begin{itemdescr}
\pnum
\expects
\tcode{x.v_.valueless_by_exception()} and \tcode{y.v_.valueless_by_exception()}
are each \tcode{false}.

\pnum
\returns
\tcode{true} if $i$ and $j$ are each \tcode{1}, and otherwise
\tcode{get<$i$>(x.v_) == get<$j$>(y.v_)}, where
$i$ is \tcode{x.v_.index()} and $j$ is \tcode{y.v_.index()}.
\end{itemdescr}

\indexlibrary{\idxcode{operator"!=}!\idxcode{common_iterator}}%
\indexlibrary{\idxcode{common_iterator}!\idxcode{operator"!=}}%
\begin{itemdecl}
template<class I2, Sentinel<I1> S2>
  requires Sentinel<S1, I2>
friend bool operator!=(
  const common_iterator& x, const common_iterator<I2, S2>& y);
\end{itemdecl}

\begin{itemdescr}
\pnum
\effects Equivalent to: \tcode{return !(x == y);}
\end{itemdescr}

\indexlibrary{\idxcode{operator-}!\idxcode{common_iterator}}%
\indexlibrary{\idxcode{common_iterator}!\idxcode{operator-}}%
\begin{itemdecl}
template<SizedSentinel<I> I2, SizedSentinel<I> S2>
  requires SizedSentinel<S, I2>
friend iter_difference_t<I2> operator-(
  const common_iterator& x, const common_iterator<I2, S2>& y);
\end{itemdecl}

\begin{itemdescr}
\pnum
\expects
\tcode{x.v_.valueless_by_exception()} and \tcode{y.v_.valueless_by_exception()}
are each \tcode{false}.

\pnum
\returns
\tcode{0} if $i$ and $j$ are each \tcode{1}, and otherwise
\tcode{get<$i$>(x.v_) - get<$j$>(y.v_)}, where
$i$ is \tcode{x.v_.index()} and $j$ is \tcode{y.v_.index()}.
\end{itemdescr}

\rSec4[common.iterator.op.iter_move]{\tcode{iter_move}}

\indexlibrary{\idxcode{iter_move}!\idxcode{common_iterator}}%
\indexlibrary{\idxcode{common_iterator}!\idxcode{iter_move}}%
\begin{itemdecl}
friend iter_rvalue_reference_t<I> iter_move(const common_iterator& i)
  noexcept(noexcept(ranges::iter_move(declval<const I&>())))
    requires InputIterator<I>;
\end{itemdecl}

\begin{itemdescr}
\pnum
\expects \tcode{holds_alternative<I>(v_)}.

\pnum
\effects Equivalent to: \tcode{return ranges::iter_move(get<I>(i.v_));}
\end{itemdescr}

\rSec4[common.iterator.op.iter_swap]{\tcode{iter_swap}}

\indexlibrary{\idxcode{iter_swap}!\idxcode{common_iterator}}%
\indexlibrary{\idxcode{common_iterator}!\idxcode{iter_swap}}%
\begin{itemdecl}
template<IndirectlySwappable<I> I2, class S2>
  friend void iter_swap(const common_iterator& x, const common_iterator<I2, S2>& y)
    noexcept(noexcept(ranges::iter_swap(declval<const I&>(), declval<const I2&>())));
\end{itemdecl}

\begin{itemdescr}
\pnum
\expects
\tcode{holds_alternative<I>(x.v_)} and \tcode{holds_alternative<I>(y.v_)}
are each \tcode{true}.

\pnum
\effects Equivalent to \tcode{ranges::iter_swap(get<I>(x.v_), get<I>(y.v_))}.
\end{itemdescr}


\rSec2[default.sentinels]{Default sentinels}
\rSec3[default.sent]{Class \tcode{default_sentinel}}

\indexlibrary{\idxcode{default_sentinel}}%
\begin{itemdecl}
namespace std {
  class default_sentinel { };
}
\end{itemdecl}

\pnum
Class \tcode{default_sentinel} is an empty type used to denote the end of a
range. It is intended to be used together with iterator types that know the bound
of their range (e.g., \tcode{counted_iterator}\iref{counted.iterator}).


\rSec2[iterators.counted]{Counted iterators}
\rSec3[counted.iterator]{Class template \tcode{counted_iterator}}

\pnum
Class template \tcode{counted_iterator} is an iterator adaptor
with the same behavior as the underlying iterator except that it
keeps track of its distance from its starting position. It can be
used together with class \tcode{default_sentinel} in calls to generic
algorithms to operate on a range of $N$ elements starting at a given
position without needing to know the end position \textit{a priori}.

\ednote{The following example incorporates the PR for
\href{https://github.com/ericniebler/stl2/issues/554}{stl2\#554}:}

\pnum
\begin{example}
\begin{codeblock}
list<string> s;
// populate the list \tcode{s} with at least 10 strings
vector<string> v;
// copies 10 strings into \tcode{v}:
ranges::copy(@\oldtxt{make_}@counted_iterator(s.begin(), 10), default_sentinel(), back_inserter(v));
\end{codeblock}
\end{example}

\pnum
Two values \tcode{i1} and \tcode{i2} of (possibly differing) types
\tcode{counted_iterator<I1>}
and \
tcode{counted_iterator<I2>}
refer to elements of the same sequence if and only if
\tcode{next(i1.base(), i1.count())}
and
\tcode{next(\brk{}i2.\brk{}base(), i2.count())}
refer to the same (possibly past-the-end) element.

\indexlibrary{\idxcode{counted_iterator}}%
\begin{codeblock}
namespace std {
  template<Iterator I>
  class counted_iterator {
  public:
    using iterator_type = I;
    using difference_type = iter_difference_t<I>;

    constexpr counted_iterator();
    constexpr counted_iterator(I x, iter_difference_t<I> n);
    template<ConvertibleTo<I> I2>
      constexpr counted_iterator(const counted_iterator<I2>& x);
    template<ConvertibleTo<I> I2>
      constexpr counted_iterator& operator=(const counted_iterator<I2>& x);

    constexpr I base() const;
    constexpr iter_difference_t<I> count() const;
    constexpr decltype(auto) operator*();
    constexpr decltype(auto) operator*() const
      requires @\placeholder{dereferenceable}@<const I>;

    constexpr counted_iterator& operator++();
    decltype(auto) operator++(int);
    constexpr counted_iterator operator++(int)
      requires ForwardIterator<I>;
    constexpr counted_iterator& operator--()
      requires BidirectionalIterator<I>;
    constexpr counted_iterator operator--(int)
      requires BidirectionalIterator<I>;

    constexpr counted_iterator operator+(difference_type n) const
      requires RandomAccessIterator<I>;
    friend constexpr counted_iterator operator+(
      iter_difference_t<I> n, const counted_iterator& x)
        requires RandomAccessIterator<I>;
    constexpr counted_iterator& operator+=(difference_type n)
      requires RandomAccessIterator<I>;

    constexpr counted_iterator operator-(difference_type n) const
      requires RandomAccessIterator<I>;
    template<Common<I> I2>
      friend constexpr iter_difference_t<I2> operator-(
        const counted_iterator& x, const counted_iterator<I2>& y);
    friend constexpr iter_difference_t<I> operator-(
      const counted_iterator& x, default_sentinel);
    friend constexpr iter_difference_t<I> operator-(
      default_sentinel, const counted_iterator& y);
    constexpr counted_iterator& operator-=(difference_type n)
      requires RandomAccessIterator<I>;

    constexpr decltype(auto) operator[](difference_type n) const
      requires RandomAccessIterator<I>;

    template<Common<I> I2>
      friend constexpr bool operator==(
        const counted_iterator& x, const counted_iterator<I2>& y);
    friend constexpr bool operator==(
      const counted_iterator& x, default_sentinel);
    friend constexpr bool operator==(
      default_sentinel, const counted_iterator& x);

    template<Common<I> I2>
      friend constexpr bool operator!=(
        const counted_iterator& x, const counted_iterator<I2>& y);
    friend constexpr bool operator!=(
      const counted_iterator& x, default_sentinel y);
    friend constexpr bool operator!=(
      default_sentinel x, const counted_iterator& y);

    template<Common<I> I2>
      friend constexpr bool operator<(
        const counted_iterator& x, const counted_iterator<I2>& y);
    template<Common<I> I2>
      friend constexpr bool operator>(
        const counted_iterator& x, const counted_iterator<I2>& y);
    template<Common<I> I2>
      friend constexpr bool operator<=(
        const counted_iterator& x, const counted_iterator<I2>& y);
    template<Common<I> I2>
      friend constexpr bool operator>=(
        const counted_iterator& x, const counted_iterator<I2>& y);

    friend constexpr iter_rvalue_reference_t<I> iter_move(const counted_iterator& i)
      noexcept(noexcept(ranges::iter_move(i.current)))
        requires InputIterator<I>;
    template<IndirectlySwappable<I> I2>
      friend constexpr void iter_swap(const counted_iterator& x, const counted_iterator<I2>& y)
        noexcept(noexcept(ranges::iter_swap(x.current, y.current)));

  private:
    I current;                // \expos
    iter_difference_t<I> cnt; // \expos
  };

  template<Readable I>
  struct readable_traits<counted_iterator<I>> {
    using value_type = iter_value_t<I>;
  };

  template<InputIterator I>
  struct iterator_traits<counted_iterator<I>> : iterator_traits<I> {
    using pointer = void;
  };
}
\end{codeblock}

\rSec3[counted.iter.ops]{\tcode{counted_iterator} operations}

\rSec4[counted.iter.op.const]{\tcode{counted_iterator} constructors and conversions}

\indexlibrary{\idxcode{counted_iterator}!\idxcode{counted_iterator}}%
\begin{itemdecl}
constexpr counted_iterator();
\end{itemdecl}

\begin{itemdescr}
\pnum
\effects
Value-initializes \tcode{current} and \tcode{cnt}.
Iterator operations applied to the resulting iterator have defined behavior
if and only if the corresponding operations are defined on
a value-initialized iterator of type \tcode{I}.
\end{itemdescr}

\indexlibrary{\idxcode{counted_iterator}!constructor}%
\begin{itemdecl}
constexpr counted_iterator(I i, iter_difference_t<I> n);
\end{itemdecl}

\begin{itemdescr}
\pnum
\expects \tcode{n >= 0}.

\pnum
\effects
Initializes \tcode{current} with \tcode{i} and \tcode{cnt} with \tcode{n}.
\end{itemdescr}

\indexlibrary{\idxcode{counted_iterator}!constructor}%
\begin{itemdecl}
template<ConvertibleTo<I> I2>
  constexpr counted_iterator(const counted_iterator<I2>& x);
\end{itemdecl}

\begin{itemdescr}
\pnum
\effects
Initializes \tcode{current} with \tcode{x.current} and
\tcode{cnt} with \tcode{x.cnt}.
\end{itemdescr}

\indexlibrary{\idxcode{operator=}!\idxcode{counted_iterator}}%
\indexlibrary{\idxcode{counted_iterator}!\idxcode{operator=}}%
\begin{itemdecl}
template<ConvertibleTo<I> I2>
  constexpr counted_iterator& operator=(const counted_iterator<I2>& x);
\end{itemdecl}

\begin{itemdescr}
\pnum
\effects
Assigns \tcode{x.current} to \tcode{current} and \tcode{x.cnt} to \tcode{cnt}.
\end{itemdescr}

\rSec4[counted.iter.op.conv]{\tcode{counted_iterator} conversion}

\indexlibrary{\idxcode{base}!\idxcode{counted_iterator}}%
\indexlibrary{\idxcode{counted_iterator}!\idxcode{base}}%
\begin{itemdecl}
constexpr I base() const;
\end{itemdecl}

\begin{itemdescr}
\pnum
\effects Equivalent to: \tcode{return current;}
\end{itemdescr}

\rSec4[counted.iter.op.cnt]{\tcode{counted_iterator} count}

\indexlibrary{\idxcode{count}!\idxcode{counted_iterator}}%
\indexlibrary{\idxcode{counted_iterator}!\idxcode{count}}%
\begin{itemdecl}
constexpr iter_difference_t<I> count() const;
\end{itemdecl}

\begin{itemdescr}
\pnum
\effects Equivalent to: \tcode{return cnt;}
\end{itemdescr}

\rSec4[counted.iter.op.star]{\tcode{counted_iterator::operator*}}

\indexlibrary{\idxcode{operator*}!\idxcode{counted_iterator}}%
\indexlibrary{\idxcode{counted_iterator}!\idxcode{operator*}}%
\begin{itemdecl}
constexpr decltype(auto) operator*();
constexpr decltype(auto) operator*() const
  requires @\placeholder{dereferenceable}@<const I>;
\end{itemdecl}

\begin{itemdescr}
\pnum
\effects Equivalent to: \tcode{return *current;}
\end{itemdescr}

\rSec4[counted.iter.op.incr]{\tcode{counted_iterator::operator++}}

\indexlibrary{\idxcode{operator++}!\idxcode{counted_iterator}}%
\indexlibrary{\idxcode{counted_iterator}!\idxcode{operator++}}%
\begin{itemdecl}
constexpr counted_iterator& operator++();
\end{itemdecl}

\begin{itemdescr}
\pnum
\expects \tcode{cnt > 0}.

\pnum
\effects Equivalent to:
\begin{codeblock}
++current;
--cnt;
\end{codeblock}

\pnum
\returns \tcode{*this}.
\end{itemdescr}

\indexlibrary{\idxcode{operator++}!\idxcode{counted_iterator}}%
\indexlibrary{\idxcode{counted_iterator}!\idxcode{operator++}}%
\begin{itemdecl}
decltype(auto) operator++(int);
\end{itemdecl}

\begin{itemdescr}
\pnum
\expects \tcode{cnt > 0}.

\pnum
\effects Equivalent to:
\begin{codeblock}
--cnt;
try { return current++; }
catch(...) { ++cnt; throw; }
\end{codeblock}
\end{itemdescr}

\begin{itemdecl}
constexpr counted_iterator operator++(int)
  requires ForwardIterator<I>;
\end{itemdecl}

\begin{itemdescr}
\pnum
\expects \tcode{cnt > 0}.

\pnum
\effects Equivalent to:
\begin{codeblock}
counted_iterator tmp = *this;
++*this;
return tmp;
\end{codeblock}
\end{itemdescr}

\rSec4[counted.iter.op.decr]{\tcode{counted_iterator::operator-{-}}}

\indexlibrary{\idxcode{operator\dcr}!\idxcode{counted_iterator}}%
\indexlibrary{\idxcode{counted_iterator}!\idxcode{operator\dcr}}%
\begin{itemdecl}
  constexpr counted_iterator& operator--();
    requires BidirectionalIterator<I>
\end{itemdecl}

\begin{itemdescr}
\pnum
\effects Equivalent to:
\begin{codeblock}
--current;
++cnt;
\end{codeblock}

\pnum
\returns \tcode{*this}.
\end{itemdescr}

\indexlibrary{\idxcode{operator\dcr}!\idxcode{counted_iterator}}%
\indexlibrary{\idxcode{counted_iterator}!\idxcode{operator\dcr}}%
\begin{itemdecl}
  constexpr counted_iterator operator--(int)
    requires BidirectionalIterator<I>;
\end{itemdecl}

\begin{itemdescr}
\pnum
\effects Equivalent to:
\begin{codeblock}
counted_iterator tmp = *this;
--*this;
return tmp;
\end{codeblock}
\end{itemdescr}

\rSec4[counted.iter.op.+]{\tcode{counted_iterator::operator+}}

\indexlibrary{\idxcode{operator+}!\idxcode{counted_iterator}}%
\indexlibrary{\idxcode{counted_iterator}!\idxcode{operator+}}%
\begin{itemdecl}
  constexpr counted_iterator operator+(difference_type n) const
    requires RandomAccessIterator<I>;
\end{itemdecl}

\begin{itemdescr}
\pnum
\expects \tcode{n <= cnt}.

\pnum
\effects Equivalent to: \tcode{return counted_iterator(current + n, cnt - n);}
\end{itemdescr}

\indexlibrary{\idxcode{operator+}!\idxcode{counted_iterator}}%
\indexlibrary{\idxcode{counted_iterator}!\idxcode{operator+}}%
\begin{itemdecl}
friend constexpr counted_iterator operator+(
  iter_difference_t<I> n, const counted_iterator& x)
    requires RandomAccessIterator<I>;
\end{itemdecl}

\begin{itemdescr}
\pnum
\effects Equivalent to: \tcode{return x + n;}
\end{itemdescr}

\rSec4[counted.iter.op.+=]{\tcode{counted_iterator::operator+=}}

\indexlibrary{\idxcode{operator+=}!\idxcode{counted_iterator}}%
\indexlibrary{\idxcode{counted_iterator}!\idxcode{operator+=}}%
\begin{itemdecl}
  constexpr counted_iterator& operator+=(difference_type n)
    requires RandomAccessIterator<I>;
\end{itemdecl}

\begin{itemdescr}
\pnum
\expects \tcode{n <= cnt}.

\pnum
\effects Equivalent to:
\begin{codeblock}
current += n;
cnt -= n;
\end{codeblock}

\pnum
\returns \tcode{*this}.
\end{itemdescr}

\rSec4[counted.iter.op.-]{\tcode{counted_iterator::operator-}}

\indexlibrary{\idxcode{operator-}!\idxcode{counted_iterator}}%
\indexlibrary{\idxcode{counted_iterator}!\idxcode{operator-}}%
\begin{itemdecl}
  constexpr counted_iterator operator-(difference_type n) const
    requires RandomAccessIterator<I>;
\end{itemdecl}

\begin{itemdescr}
\pnum
\expects \tcode{-n <= cnt}.

\pnum
\effects Equivalent to: \tcode{return counted_iterator(current - n, cnt + n);}
\end{itemdescr}

\indexlibrary{\idxcode{operator-}!\idxcode{counted_iterator}}%
\indexlibrary{\idxcode{counted_iterator}!\idxcode{operator-}}%
\begin{itemdecl}
template<Common<I> I2>
  friend constexpr iter_difference_t<I2> operator-(
    const counted_iterator& x, const counted_iterator<I2>& y);
\end{itemdecl}

\begin{itemdescr}
\pnum
\expects
\tcode{x} and \tcode{y} shall refer to elements of the same
sequence\iref{counted.iterator}.

\pnum
\effects Equivalent to: \tcode{return y.cnt - x.cnt;}
\end{itemdescr}

\begin{itemdecl}
friend constexpr iter_difference_t<I> operator-(
  const counted_iterator& x, default_sentinel);
\end{itemdecl}

\begin{itemdescr}
\pnum
\effects Equivalent to:
\tcode{return -x.cnt;}
\end{itemdescr}

\begin{itemdecl}
friend constexpr iter_difference_t<I> operator-(
  default_sentinel, const counted_iterator& y);
\end{itemdecl}

\begin{itemdescr}
\pnum
\effects Equivalent to: \tcode{return y.cnt;}
\end{itemdescr}

\rSec4[counted.iter.op.-=]{\tcode{counted_iterator::operator-=}}

\indexlibrary{\idxcode{operator-=}!\idxcode{counted_iterator}}%
\indexlibrary{\idxcode{counted_iterator}!\idxcode{operator-=}}%
\begin{itemdecl}
constexpr counted_iterator& operator-=(difference_type n)
  requires RandomAccessIterator<I>;
\end{itemdecl}

\begin{itemdescr}
\pnum
\expects \tcode{-n <= cnt}.

\pnum
\effects Equivalent to:
\begin{codeblock}
current -= n;
cnt += n;
\end{codeblock}

\pnum
\returns \tcode{*this}.
\end{itemdescr}

\rSec4[counted.iter.op.index]{\tcode{counted_iterator::operator[]}}

\indexlibrary{\idxcode{operator[]}!\idxcode{counted_iterator}}%
\indexlibrary{\idxcode{counted_iterator}!\idxcode{operator[]}}%
\begin{itemdecl}
constexpr decltype(auto) operator[](difference_type n) const
  requires RandomAccessIterator<I>;
\end{itemdecl}

\begin{itemdescr}
\pnum
\expects \tcode{n <= cnt}.

\pnum
\effects Equivalent to: \tcode{return current[n];}
\end{itemdescr}

\rSec4[counted.iter.op.comp]{\tcode{counted_iterator} comparisons}

\indexlibrary{\idxcode{operator==}!\idxcode{counted_iterator}}%
\indexlibrary{\idxcode{counted_iterator}!\idxcode{operator==}}%
\begin{itemdecl}
template<Common<I> I2>
  friend constexpr bool operator==(
    const counted_iterator& x, const counted_iterator<I2>& y);
\end{itemdecl}

\begin{itemdescr}
\pnum
\expects
\tcode{x} and \tcode{y} shall refer to
elements of the same sequence\iref{counted.iterator}.

\pnum
\effects Equivalent to: \tcode{return x.cnt == y.cnt;}
\end{itemdescr}

\begin{itemdecl}
friend constexpr bool operator==(
  const counted_iterator& x, default_sentinel);
friend constexpr bool operator==(
  default_sentinel, const counted_iterator& x);
\end{itemdecl}

\begin{itemdescr}
\pnum
\effects Equivalent to: \tcode{return x.cnt == 0;}
\end{itemdescr}

\indexlibrary{\idxcode{operator"!=}!\idxcode{counted_iterator}}%
\indexlibrary{\idxcode{counted_iterator}!\idxcode{operator"!=}}%
\begin{itemdecl}
template<Common<I> I2>
  friend constexpr bool operator!=(
    const counted_iterator& x, const counted_iterator<I2>& y);
friend constexpr bool operator!=(
  const counted_iterator& x, default_sentinel y);
friend constexpr bool operator!=(
  default_sentinel x, const counted_iterator& y);
\end{itemdecl}

\begin{itemdescr}
\pnum
\expects
For the first overload, \tcode{x} and \tcode{y} shall refer to
elements of the same sequence\iref{counted.iterator}.

\pnum
\effects Equivalent to: \tcode{return !(x == y);}
\end{itemdescr}

\indexlibrary{\idxcode{operator<}!\idxcode{counted_iterator}}%
\indexlibrary{\idxcode{counted_iterator}!\idxcode{operator<}}%
\begin{itemdecl}
template<Common<I> I2>
  friend constexpr bool operator<(
    const counted_iterator& x, const counted_iterator<I2>& y);
\end{itemdecl}

\begin{itemdescr}
\pnum
\expects
\tcode{x} and \tcode{y} shall refer to
elements of the same sequence\iref{counted.iterator}.

\pnum
\effects Equivalent to: \tcode{return y.cnt < x.cnt;}

\pnum
\begin{note}
The argument order in the \textit{Effects} element is reversed
because \tcode{cnt} counts down, not up.
\end{note}
\end{itemdescr}

\indexlibrary{\idxcode{operator>}!\idxcode{counted_iterator}}%
\indexlibrary{\idxcode{counted_iterator}!\idxcode{operator>}}%
\begin{itemdecl}
template<Common<I> I2>
  friend constexpr bool operator>(
    const counted_iterator& x, const counted_iterator<I2>& y);
\end{itemdecl}

\begin{itemdescr}
\pnum
\expects
\tcode{x} and \tcode{y} shall refer to
elements of the same sequence\iref{counted.iterator}.

\pnum
\effects Equivalent to: \tcode{return y < x;}
\end{itemdescr}

\indexlibrary{\idxcode{operator<=}!\idxcode{counted_iterator}}%
\indexlibrary{\idxcode{counted_iterator}!\idxcode{operator<=}}%
\begin{itemdecl}
template<Common<I> I2>
  friend constexpr bool operator<=(
    const counted_iterator& x, const counted_iterator<I2>& y);
\end{itemdecl}

\begin{itemdescr}
\pnum
\expects
\tcode{x} and \tcode{y} shall refer to
elements of the same sequence\iref{counted.iterator}.

\pnum
\effects Equivalent to: \tcode{return !(y < x);}
\end{itemdescr}

\indexlibrary{\idxcode{operator>=}!\idxcode{counted_iterator}}%
\indexlibrary{\idxcode{counted_iterator}!\idxcode{operator>=}}%
\begin{itemdecl}
template<Common<I> I2>
  friend constexpr bool operator>=(
    const counted_iterator& x, const counted_iterator<I2>& y);
\end{itemdecl}

\begin{itemdescr}
\pnum
\expects
\tcode{x} and \tcode{y} shall refer to
elements of the same sequence\iref{counted.iterator}.

\pnum
\effects Equivalent to: \tcode{return !(x < y);}
\end{itemdescr}

\rSec4[counted.iter.nonmember]{\tcode{counted_iterator} customizations}

\indexlibrary{\idxcode{iter_move}!\idxcode{counted_iterator}}%
\indexlibrary{\idxcode{counted_iterator}!\idxcode{iter_move}}%
\begin{itemdecl}
friend constexpr iter_rvalue_reference_t<I> iter_move(const counted_iterator& i)
  noexcept(noexcept(ranges::iter_move(i.current)))
    requires InputIterator<I>;
\end{itemdecl}

\begin{itemdescr}
\pnum
\effects Equivalent to: \tcode{return ranges::iter_move(i.current);}
\end{itemdescr}

\indexlibrary{\idxcode{iter_swap}!\idxcode{counted_iterator}}%
\indexlibrary{\idxcode{counted_iterator}!\idxcode{iter_swap}}%
\begin{itemdecl}
template<IndirectlySwappable<I> I2>
  friend constexpr void iter_swap(const counted_iterator& x, const counted_iterator<I2>& y)
    noexcept(noexcept(ranges::iter_swap(x.current, y.current)));
\end{itemdecl}

\begin{itemdescr}
\pnum
\effects Equivalent to \tcode{ranges::iter_swap(x.current, y.current)}.
\end{itemdescr}


\rSec2[unreachable.sentinels]{Unreachable sentinel}
\rSec3[unreachable.sentinel]{Class \tcode{unreachable}}

\ednote{This wording integrates the PR for
\href{https://github.com/ericniebler/stl2/issues/507}{stl2\#507}):}

\pnum
\indexlibrary{\idxcode{unreachable}}%
Class \tcode{unreachable} is a placeholder type that can be
used with any \libconcept{WeaklyIncrementable} type to
denote the ``upper bound'' of an open interval.
Comparing anything for equality with an object of type
\tcode{unreachable} always returns \tcode{false}.

\pnum
\begin{example}
\begin{codeblock}
char* p;
// set p to point to a character buffer containing newlines
char* nl = find(p, unreachable(), '\n');
\end{codeblock}

Provided a newline character really exists in the buffer, the use of
\tcode{unreachable} above potentially makes the call to \tcode{find} more
efficient since the loop test against the sentinel does not require a
conditional branch.
\end{example}

\begin{codeblock}
namespace std {
  class unreachable {
  public:
    template<WeaklyIncrementable I>
      friend constexpr bool operator==(unreachable, const I&) noexcept;
    template<WeaklyIncrementable I>
      friend constexpr bool operator==(const I&, unreachable) noexcept;
    template<WeaklyIncrementable I>
      friend constexpr bool operator!=(unreachable, const I&) noexcept;
    template<WeaklyIncrementable I>
      friend constexpr bool operator!=(const I&, unreachable) noexcept;
  };
}
\end{codeblock}

\rSec3[unreachable.sentinel.ops]{\tcode{unreachable} operations}

\indexlibrary{\idxcode{operator==}!\idxcode{unreachable}}%
\indexlibrary{\idxcode{unreachable}!\idxcode{operator==}}%
\begin{itemdecl}
template<WeaklyIncrementable I>
  friend constexpr bool operator==(unreachable, const I&) noexcept;
template<WeaklyIncrementable I>
  friend constexpr bool operator==(const I&, unreachable) noexcept;
\end{itemdecl}

\begin{itemdescr}
\pnum
\effects Equivalent to: \tcode{return false;}
\end{itemdescr}

\indexlibrary{\idxcode{operator"!=}!\idxcode{unreachable}}%
\indexlibrary{\idxcode{unreachable}!\idxcode{operator"!=}}%
\begin{itemdecl}
template<WeaklyIncrementable I>
  friend constexpr bool operator!=(unreachable, const I&) noexcept;
template<WeaklyIncrementable I>
  friend constexpr bool operator!=(const I&, unreachable) noexcept;
\end{itemdecl}

\begin{itemdescr}
\pnum
\effects Equivalent to: \tcode{return true;}
\end{itemdescr}
\end{addedblock}


\rSec1[stream.iterators]{Stream iterators}

[...]

\rSec2[istream.iterator]{Class template \tcode{istream_iterator}}

[...]

\begin{codeblock}
namespace std {
  template<class T, class charT = char, class traits = char_traits<charT>,
           class Distance = ptrdiff_t>
  class istream_iterator {
  public:
    [...]

    constexpr istream_iterator();
    @\added{constexpr istream_iterator(default_sentinel);}@
    istream_iterator(istream_type& s);

    [...]

    istream_iterator  operator++(int);

    @\added{friend bool operator==(const istream_iterator\& i, default_sentinel);}@
    @\added{friend bool operator==(default_sentinel, const istream_iterator\& i);}@
    @\added{friend bool operator!=(const istream_iterator\& x, default_sentinel y);}@
    @\added{friend bool operator!=(default_sentinel x, const istream_iterator\& y);}@

  private:
    [...]
  };

  [...]
}
\end{codeblock}

\rSec3[istream.iterator.cons]{\tcode{istream_iterator} constructors and destructor}

\indexlibrary{\idxcode{istream_iterator}!constructor}%
\begin{itemdecl}
constexpr istream_iterator();
@\added{constexpr istream_iterator(default_sentinel);}@
\end{itemdecl}

\begin{itemdescr}
\pnum
\effects
Constructs the end-of-stream iterator.
If \tcode{is_trivially_default_constructible_v<T>} is \tcode{true},
then \changed{this constructor is a}{these constructors are} constexpr
constructor\added{s}.

\pnum
\ensures \tcode{in_stream == 0}.
\end{itemdescr}

[...]

\rSec3[istream.iterator.ops]{\tcode{istream_iterator} operations}

[...]

\setcounter{Paras}{7}
\indexlibrarymember{operator==}{istream_iterator}%
\begin{itemdecl}
template<class T, class charT, class traits, class Distance>
  bool operator==(const istream_iterator<T,charT,traits,Distance>& x,
                  const istream_iterator<T,charT,traits,Distance>& y);
\end{itemdecl}

\begin{itemdescr}
\pnum
\returns
\tcode{x.in_stream == y.in_stream}.
\end{itemdescr}

\begin{addedblock}
\indexlibrarymember{operator==}{istream_iterator}%
\begin{itemdecl}
friend bool operator==(default_sentinel, const istream_iterator& i);
friend bool operator==(const istream_iterator& i, default_sentinel);
\end{itemdecl}

\begin{itemdescr}
\pnum
\returns
\tcode{!i.in_stream}.
\end{itemdescr}
\end{addedblock}

\indexlibrarymember{operator"!=}{istream_iterator}%
\begin{itemdecl}
template<class T, class charT, class traits, class Distance>
  bool operator!=(const istream_iterator<T,charT,traits,Distance>& x,
                  const istream_iterator<T,charT,traits,Distance>& y);
@\added{friend bool operator!=(default_sentinel x, const istream_iterator\& y);}@
@\added{friend bool operator!=(const istream_iterator\& x, default_sentinel y);}@
\end{itemdecl}

\begin{itemdescr}
\pnum
\returns
\tcode{!(x == y)}
\end{itemdescr}


\rSec2[ostream.iterator]{Class template \tcode{ostream_iterator}}

[...]

\setcounter{Paras}{1}
\pnum \tcode{ostream_iterator} is defined as:

\begin{codeblock}
namespace std {
  template<class T, class charT = char, class traits = char_traits<charT>>
  class ostream_iterator {
  public:
    using iterator_category = output_iterator_tag;
    using value_type        = void;
    using difference_type   = @\changed{void}{ptrdiff_t}@;
    using pointer           = void;
    using reference         = void;
    using char_type         = charT;
    using traits_type       = traits;
    using ostream_type      = basic_ostream<charT,traits>;

    @\added{constexpr ostream_iterator() noexcept = default;}@
    ostream_iterator(ostream_type& s);

    [...]

  private:
    basic_ostream<charT,traits>* out_stream @\added{= nullptr}@;  // \expos
    const charT* delim @\added{= nullptr}@;                       // \expos
  };
}
\end{codeblock}

[...]

\rSec2[istreambuf.iterator]{Class template \tcode{istreambuf_iterator}}

[...]

\indexlibrary{\idxcode{istreambuf_iterator}}%
\begin{codeblock}
namespace std {
  template<class charT, class traits = char_traits<charT>>
  class istreambuf_iterator {
  public:
    [...]

    constexpr istreambuf_iterator() noexcept;
    @\added{constexpr istreambuf_iterator(default_sentinel) noexcept;}@
    istreambuf_iterator(const istreambuf_iterator&) noexcept = default;

    [...]

    bool equal(const istreambuf_iterator& b) const;

    @\added{friend bool operator==(default_sentinel s, const istreambuf_iterator\& i);}@
    @\added{friend bool operator==(const istreambuf_iterator\& i, default_sentinel s);}@
    @\added{friend bool operator!=(default_sentinel a, const istreambuf_iterator\& b);}@
    @\added{friend bool operator!=(const istreambuf_iterator\& a, default_sentinel b);}@

  private:
    streambuf_type* sbuf_;                // \expos
  };

  [...]
}
\end{codeblock}

[...]

\setcounter{subsubsection}{1}
\rSec3[istreambuf.iterator.cons]{\tcode{istreambuf_iterator} constructors}

[...]

\setcounter{Paras}{1}
\indexlibrary{\idxcode{istreambuf_iterator}!constructor}%
\begin{itemdecl}
constexpr istreambuf_iterator() noexcept;
@\added{constexpr istreambuf_iterator(default_sentinel) noexcept;}@
\end{itemdecl}

\begin{itemdescr}
\pnum
\effects
Initializes \tcode{sbuf_} with \tcode{nullptr}.
\end{itemdescr}

[...]

\rSec3[istreambuf.iterator.ops]{\tcode{istreambuf_iterator} operations}

[...]

\setcounter{Paras}{5}
\indexlibrarymember{operator==}{istreambuf_iterator}%
\begin{itemdecl}
template<class charT, class traits>
  bool operator==(const istreambuf_iterator<charT,traits>& a,
                  const istreambuf_iterator<charT,traits>& b);
\end{itemdecl}

\begin{itemdescr}
\pnum
\returns
\tcode{a.equal(b)}.
\end{itemdescr}

\begin{addedblock}
\indexlibrarymember{operator==}{istreambuf_iterator}%
\begin{itemdecl}
friend bool operator==(default_sentinel s, const istreambuf_iterator& i);
friend bool operator==(const istreambuf_iterator& i, default_sentinel s);
\end{itemdecl}

\begin{itemdescr}
\pnum
\returns \tcode{i.equal(s)}.
\end{itemdescr}
\end{addedblock}

\indexlibrarymember{operator"!=}{istreambuf_iterator}%
\begin{itemdecl}
template<class charT, class traits>
  bool operator!=(const istreambuf_iterator<charT,traits>& a,
                  const istreambuf_iterator<charT,traits>& b);
@\added{friend bool operator!=(default_sentinel a, const istreambuf_iterator\& b);}@
@\added{friend bool operator!=(const istreambuf_iterator\& a, default_sentinel b);}@
\end{itemdecl}

\begin{itemdescr}
\pnum
\returns
\changed{\tcode{!a.equal(b)}}{\tcode{!(a == b)}}.
\end{itemdescr}


\rSec2[ostreambuf.iterator]{Class template \tcode{ostreambuf_iterator}}

\indexlibrary{\idxcode{ostreambuf_iterator}}%
\begin{codeblock}
namespace std {
  template<class charT, class traits = char_traits<charT>>
  class ostreambuf_iterator {
  public:
    using iterator_category = output_iterator_tag;
    using value_type        = void;
    using difference_type   = @\changed{void}{ptrdiff_t}@;
    using pointer           = void;
    using reference         = void;
    using char_type         = charT;
    using traits_type       = traits;
    using streambuf_type    = basic_streambuf<charT,traits>;
    using ostream_type      = basic_ostream<charT,traits>;

    @\added{constexpr ostreambuf_iterator() noexcept = default;}@

    [...]

  private:
    streambuf_type* sbuf_ @\added{= nullptr}@;    // \expos
  };
}
\end{codeblock}


\rSec1[range.access]{Range access}

{\color{oldclr}
\pnum
In addition to being available via inclusion of the \tcode{<experimental/ranges/range>}
header, the customization point objects in \ref{range.access} are
available when \tcode{<experimental/ranges/iterator>} is included.
} % \color{oldclr}

\ednote{The customization point objects in this subsection all have deprecated
behavior that permits them to work with rvalues. This is for compatability with
the similarly named facilities in namespace \tcode{std}.
\href{https://wg21.link/p0970}{P0970} proposes a redesign that replaces the
deprecated behavior with proper support for rvalue ranges.}

\rSec2[range.access.begin]{\tcode{begin}}
\pnum
The name \tcode{begin} denotes a customization point
 object~(\cxxref{customization.point.object}). The expression
\tcode{ranges::begin(E)} for some subexpression \tcode{E} is expression-equivalent to:

\begin{itemize}
\item
  \tcode{ranges::begin(static_cast<const T\&>(E))} if \tcode{E} is an rvalue of
  type \tcode{T}. This usage is deprecated.
  \enternote This deprecated usage exists so that
  \tcode{ranges::begin(E)} behaves similarly to \tcode{std::begin(E)}
  \oldtxt{as defined in ISO/IEC 14882} when \tcode{E} is an rvalue. \exitnote

\item
  Otherwise, \tcode{(E) + 0} if \tcode{E} has array
  type~(\cxxref{basic.compound}).

\item
  Otherwise, \tcode{\textit{DECAY_COPY}((E).begin())} if it is a valid expression and its type \tcode{I} meets the
  syntactic requirements of \tcode{Iterator<I>}. If
  \tcode{Iterator} is not satisfied, the program is ill-formed
  with no diagnostic required.

\item
  Otherwise, \tcode{\textit{DECAY_COPY}(begin(E))} if it is a valid expression and its type \tcode{I} meets the
  syntactic requirements of \tcode{Iterator<I>} with overload
  resolution performed in a context that includes the declaration
  \tcode{\newtxt{template <class T>} void begin(\oldtxt{auto}\newtxt{T}\&) = delete;} and does not include
  a declaration of \tcode{ranges::begin}. If \tcode{Iterator}
  is not satisfied, the program is ill-formed with no diagnostic
  required.

\item
  Otherwise, \tcode{ranges::begin(E)} is ill-formed.
\end{itemize}

\pnum
\enternote Whenever \tcode{ranges::begin(E)} is a valid expression, its
type satisfies \tcode{Iterator}. \exitnote

\rSec2[range.access.end]{\tcode{end}}
\pnum
The name \tcode{end} denotes a customization point
object~(\cxxref{customization.point.object}). The expression
\tcode{ranges::end(E)} for some subexpression \tcode{E} is expression-equivalent to:

\begin{itemize}
\item
  \tcode{ranges::end(static_cast<const T\&>(E))} if \tcode{E} is an rvalue of
  type \tcode{T}. This usage is deprecated.
  \enternote This deprecated usage exists so that
  \tcode{ranges::end(E)} behaves similarly to \tcode{std::end(E)}
  \oldtxt{as defined in ISO/IEC 14882} when \tcode{E} is an rvalue. \exitnote

\item
  Otherwise, \tcode{(E) + extent\newtxt{_v}<T>\oldtxt{::value}} if \tcode{E} has array
  type~(\cxxref{basic.compound}) \tcode{T}.

\item
  Otherwise, \tcode{\textit{DECAY_COPY}((E).end())} if it is a valid expression and its type \tcode{S} meets the
  syntactic requirements of
  \tcode{Sentinel<\brk{}S, decltype(\brk{}ranges::\brk{}begin(E))>}. If
  \tcode{Sentinel} is not satisfied, the program is ill-formed with
  no diagnostic required.

\item
  Otherwise, \tcode{\textit{DECAY_COPY}(end(E))} if it is a valid expression and its type \tcode{S} meets the
  syntactic requirements of
  \tcode{Sentinel<\brk{}S, decltype(\brk{}ranges::\brk{}begin(E))>} with overload
  resolution performed in a context that includes the declaration
  \tcode{\newtxt{template <class T>} void end(\oldtxt{auto}\newtxt{T}\&) = delete;} and does not include
  a declaration of \tcode{ranges::end}. If \tcode{Sentinel} is not
  satisfied, the program is ill-formed with no diagnostic required.

\item
  Otherwise, \tcode{ranges::end(E)} is ill-formed.
\end{itemize}

\pnum
\enternote Whenever \tcode{ranges::end(E)} is a valid expression, the
types of \tcode{ranges::end(E)} and \tcode{ranges::\brk{}begin(E)} satisfy
\tcode{Sentinel}. \exitnote

\rSec2[range.access.cbegin]{\tcode{cbegin}}
\pnum
The name \tcode{cbegin} denotes a customization point
object~(\cxxref{customization.point.object}). The expression
\tcode{ranges::\brk{}cbegin(E)} for some subexpression \tcode{E} of type \tcode{T}
is expression-equivalent to \tcode{ranges::\brk{}begin(static_cast<const T\&>(E))}.

\pnum
Use of \tcode{ranges::cbegin(E)} with rvalue \tcode{E} is deprecated.
\enternote This deprecated usage exists so that \tcode{ranges::cbegin(E)}
behaves similarly to \tcode{std::cbegin(E)} \oldtxt{as defined in ISO/IEC 14882} when
\tcode{E} is an rvalue. \exitnote

\pnum
\enternote Whenever \tcode{ranges::cbegin(E)} is a valid expression, its
type satisfies \tcode{Iterator}. \exitnote

\rSec2[range.access.cend]{\tcode{cend}}
\pnum
The name \tcode{cend} denotes a customization point
object~(\cxxref{customization.point.object}). The expression
\tcode{ranges::cend(E)} for some subexpression \tcode{E} of type \tcode{T}
is expression-equivalent to \tcode{ranges::end(static_cast<const T\&>(E))}.

\pnum
Use of \tcode{ranges::cend(E)} with rvalue \tcode{E} is deprecated.
\enternote This deprecated usage exists so that \tcode{ranges::\brk{}cend(E)}
behaves similarly to \tcode{std::cend(E)} \oldtxt{as defined in ISO/IEC 14882} when
\tcode{E} is an rvalue. \exitnote

\pnum
\enternote Whenever \tcode{ranges::cend(E)} is a valid expression, the
types of \tcode{ranges::cend(E)} and \tcode{ranges::\brk{}cbegin(E)} satisfy
\tcode{Sentinel}. \exitnote

\rSec2[range.access.rbegin]{\tcode{rbegin}}
\pnum
The name \tcode{rbegin} denotes a customization point
object~(\cxxref{customization.point.object}). The expression
\tcode{ranges::rbegin(E)} for some subexpression \tcode{E} is expression-equivalent
to:

\begin{itemize}
\item
  \tcode{ranges::rbegin(static_cast<const T\&>(E))} if \tcode{E} is an rvalue of
  type \tcode{T}. This usage is deprecated.
  \enternote This deprecated usage exists so that
  \tcode{ranges::rbegin(E)} behaves similarly to \tcode{std::rbegin(E)}
  \oldtxt{as defined in ISO/IEC 14882} when \tcode{E} is an rvalue. \exitnote

\item
  Otherwise, \tcode{\textit{DECAY_COPY}((E).rbegin())} if it is a valid expression and its type \tcode{I} meets the
  syntactic requirements of \tcode{Iterator<I>}. If \tcode{Iterator}
  is not satisfied, the program is ill-formed with no diagnostic
  required.

\item
  Otherwise, \tcode{make_reverse_iterator(ranges::end(E))} if both
  \tcode{ranges::begin(E)} and \tcode{ranges::\brk{}end(\brk{}E)} are valid expressions of the same
  type \tcode{I} which meets the syntactic requirements of
  \tcode{Bi\-direct\-ional\-Iterator<I>}~(\ref{range.iterators.bidirectional}).

\item
  Otherwise, \tcode{ranges::rbegin(E)} is ill-formed.
\end{itemize}

\pnum
\enternote Whenever \tcode{ranges::rbegin(E)} is a valid expression, its
type satisfies \tcode{Iterator}. \exitnote

\rSec2[range.access.rend]{\tcode{rend}}
\pnum
The name \tcode{rend} denotes a customization point
object~(\cxxref{customization.point.object}). The expression
\tcode{ranges::rend(E)} for some subexpression \tcode{E} is expression-equivalent to:

\begin{itemize}
\item
  \tcode{ranges::rend(static_cast<const T\&>(E))} if \tcode{E} is an rvalue of
  type \tcode{T}. This usage is deprecated.
  \enternote This deprecated usage exists so that
  \tcode{ranges::rend(E)} behaves similarly to \tcode{std::rend(E)}
  \oldtxt{as defined in ISO/IEC 14882} when \tcode{E} is an rvalue. \exitnote

\item
  Otherwise, \tcode{\textit{DECAY_COPY}((E).rend())} if it is a valid expression and its type \tcode{S} meets the
  syntactic requirements of
  \tcode{Sentinel<\brk{}S, decltype(\brk{}ranges::\brk{}rbegin(E))>}. If
  \tcode{Sentinel} is not satisfied, the program is ill-formed with
  no diagnostic required.

\item
  Otherwise, \tcode{make_reverse_iterator(ranges\colcol{}begin(E))} if both
  \tcode{ranges::begin(E)} and \tcode{ranges\colcol{}end(\brk{}E)} are valid expressions of the same
  type \tcode{I} which meets the syntactic requirements of
  \tcode{Bi\-dir\-ect\-ion\-al\-It\-er\-at\-or<I>}~(\ref{range.iterators.bidirectional}).

\item
  Otherwise, \tcode{ranges::rend(E)} is ill-formed.
\end{itemize}

\pnum
\enternote Whenever \tcode{ranges::rend(E)} is a valid expression, the
types of \tcode{ranges::\brk{}rend(E)} and \tcode{ranges::\brk{}rbegin(E)} satisfy
\tcode{Sentinel}. \exitnote

\rSec2[range.access.crbegin]{\tcode{crbegin}}
\pnum
The name \tcode{crbegin} denotes a customization point
object~(\cxxref{customization.point.object}). The expression
\tcode{ranges::\brk{}crbegin(E)} for some subexpression \tcode{E} of type \tcode{T}
is expression-equivalent to \tcode{ranges::\brk{}rbegin(static_cast<const T\&>(E))}.

\pnum
Use of \tcode{ranges::crbegin(E)} with rvalue \tcode{E} is deprecated.
\enternote This deprecated usage exists so that \tcode{ranges::crbegin(E)}
behaves similarly to \tcode{std::crbegin(E)} \oldtxt{as defined in ISO/IEC 14882} when
\tcode{E} is an rvalue. \exitnote

\pnum
\enternote Whenever \tcode{ranges::crbegin(E)} is a valid expression, its
type satisfies \tcode{Iterator}. \exitnote

\rSec2[range.access.crend]{\tcode{crend}}
\pnum
The name \tcode{crend} denotes a customization point
object~(\cxxref{customization.point.object}). The expression
\tcode{ranges::crend(E)} for some subexpression \tcode{E} of type \tcode{T}
is expression-equivalent to \tcode{ranges::rend(static_cast<const T\&>(E))}.

\pnum
Use of \tcode{ranges::crend(E)} with rvalue \tcode{E} is deprecated.
\enternote This deprecated usage exists so that \tcode{ranges::crend(E)}
behaves similarly to \tcode{std::crend(E)} \oldtxt{as defined in ISO/IEC 14882} when
\tcode{E} is an rvalue. \exitnote

\pnum
\enternote Whenever \tcode{ranges::crend(E)} is a valid expression, the
types of \tcode{ranges::crend(E)} and \tcode{ranges::\brk{}crbegin(\brk{}E)} satisfy
\tcode{Sentinel}. \exitnote

\rSec1[range.primitives]{Range primitives}

\pnum
\oldtxt{In addition to being available via inclusion of the \tcode{<experimental/ranges/range>}
header, the customization point objects in \ref{range.primitives} are
available when \tcode{<experimental/ranges/iterator>} is included.}

\rSec2[range.primitives.size]{\tcode{size}}
\pnum
The name \tcode{size} denotes a customization point
object~(\cxxref{customization.point.object}). The expression
\tcode{ranges::size(E)} for some subexpression \tcode{E} with type
\tcode{T} is expression-equivalent to:

\begin{itemize}
\item
  \tcode{\textit{DECAY_COPY}(extent\newtxt{_v}<T>\oldtxt{::value})} if \tcode{T} is an array
  type~(\cxxref{basic.compound}).

\item
  Otherwise, \tcode{\textit{DECAY_COPY}(static_cast<const T\&>(E).size())} if it is a valid expression and its type \tcode{I}
  satisfies \tcode{Integral<I>} and
  \tcode{disable_\-sized_\-range<T>}~(\ref{range.sized}) is
  \tcode{false}.

\item
  Otherwise, \tcode{\textit{DECAY_COPY}(size(static_cast<const T\&>(E)))} if it is a valid expression and its type \tcode{I}
  satisfies \tcode{Integral<I>} with overload resolution
  performed in a context that includes the declaration
  \tcode{\newtxt{template <class T>} void size(const \oldtxt{auto}\newtxt{T}\&) = delete;} and does not include
  a declaration of \tcode{ranges::size}, and
  \tcode{disable_\-sized_\-range<T>} is \tcode{false}.

\item
  Otherwise,
  \tcode{\textit{DECAY_COPY}(ranges::cend(E) - ranges::cbegin(E))}, except that \tcode{E}
  is only evaluated once, if it is a valid expression and the types \tcode{I} and \tcode{S} of
  \tcode{ranges::cbegin(E)} and \tcode{ranges\colcol{}cend(\brk{}E)} meet the
  syntactic requirements of
  \tcode{SizedSentinel<S, I>}~(\ref{range.iterators.sizedsentinel}) and
  \tcode{Forward\-Iter\-at\-or<I>}. If \tcode{SizedSentinel} and
  \tcode{Forward\-Iter\-at\-or} are not satisfied, the program is ill-formed with no
  diagnostic required.

\item
  Otherwise, \tcode{ranges::size(E)} is ill-formed.
\end{itemize}

\pnum
\enternote Whenever \tcode{ranges::size(E)} is a valid expression, its
type satisfies \tcode{Integral}. \exitnote

\rSec2[range.primitives.empty]{\tcode{empty}}
\pnum
The name \tcode{empty} denotes a customization point
object~(\cxxref{customization.point.object}). The expression
\tcode{ranges::empty(E)} for some subexpression \tcode{E} is
expression-equivalent to:

\begin{itemize}
\item
  \tcode{bool((E).empty())} if it is a valid expression.

\item
  Otherwise, \tcode{ranges::size(E) == 0} if it is a valid expression.

\item
  Otherwise, \tcode{bool(ranges::begin(E) == ranges::end(E))},
  except that \tcode{E} is only evaluated once, if it is a valid expression and the type of
  \tcode{ranges::begin(E)} satisfies \tcode{ForwardIterator}.

\item
  Otherwise, \tcode{ranges::empty(E)} is ill-formed.
\end{itemize}

\pnum
\enternote Whenever \tcode{ranges::empty(E)} is a valid expression, it
has type \tcode{bool}. \exitnote

\rSec2[range.primitives.data]{\tcode{data}}
\pnum
The name \tcode{data} denotes a customization point
object~(\cxxref{customization.point.object}). The expression
\tcode{ranges::data(E)} for some subexpression \tcode{E} is
expression-equivalent to:

\begin{itemize}
\item
  \tcode{ranges::data(static_cast<const T\&>(E))} if \tcode{E} is an rvalue of
  type \tcode{T}. This usage is deprecated. \enternote
  This deprecated usage exists so that \tcode{ranges::data(E)} behaves
  similarly to \tcode{std::data(E)} \oldtxt{as defined in the \Cpp Working
  Paper} when \tcode{E} is an rvalue. \exitnote

\item
  Otherwise, \tcode{\textit{DECAY_COPY}((E).data())} if it is a valid expression of pointer to object type.

\item
  Otherwise, \tcode{ranges::begin(E)} if it is a valid expression of pointer to object type.

\item
  Otherwise, \tcode{ranges::data(E)} is ill-formed.
\end{itemize}

\pnum
\enternote Whenever \tcode{ranges::data(E)} is a valid expression, it
has pointer to object type. \exitnote

\rSec2[range.primitives.cdata]{\tcode{cdata}}
\pnum
The name \tcode{cdata} denotes a customization point
object~(\cxxref{customization.point.object}). The expression
\tcode{ranges::cdata(E)} for some subexpression \tcode{E} of type \tcode{T}
is expression-equivalent to \tcode{ranges::data(static_cast<const T\&>(E))}.

\pnum
Use of \tcode{ranges::cdata(E)} with rvalue \tcode{E} is deprecated.
\enternote This deprecated usage exists so that \tcode{ranges::cdata(E)}
has behavior consistent with \tcode{ranges::data(E)} when \tcode{E} is
an rvalue. \exitnote

\pnum
\enternote Whenever \tcode{ranges::cdata(E)} is a valid expression, it
has pointer to object type. \exitnote

\rSec1[range.requirements]{Range requirements}

\rSec2[range.requirements.general]{General}

\pnum
Ranges are an abstraction of containers that allow a \Cpp program to
operate on elements of data structures uniformly. It their simplest form, a
range object is one on which one can call \tcode{begin} and
\tcode{end} to get an iterator~(\ref{range.iterators.iterator}) and a
sentinel~(\ref{range.iterators.sentinel}). To be able to construct
template algorithms and range adaptors that work correctly and efficiently on
different types of sequences, the library formalizes not just the interfaces but
also the semantics and complexity assumptions of ranges.

\pnum
This document defines three fundamental categories of ranges
based on the syntax and semantics supported by each: \techterm{range},
\techterm{sized range} and \techterm{view}, as shown in
Table~\ref{tab:ranges.relations}.

\begin{floattable}{Relations among range categories}{tab:ranges.relations}
  {lll}
  \topline
  \textbf{Sized Range}  &               &                   \\
                        & $\searrow$    &                   \\
                        &               &  \textbf{Range}   \\
                        & $\nearrow$    &                   \\
  \textbf{View}         &               &                   \\
\end{floattable}

\pnum
The \tcode{Range} concept requires only that \tcode{begin} and \tcode{end}
return an iterator and a sentinel. The \tcode{SizedRange} concept refines \tcode{Range}
with the requirement that the number of elements in the range can be determined
in constant time using the \tcode{size} function. The \tcode{View} concept
specifies requirements on a \tcode{Range} type
with constant-time copy and assign operations.

\pnum
In addition to the three fundamental range categories, this document defines
a number of convenience refinements of \tcode{Range} that group together requirements
that appear often in the concepts and algorithms.
\oldtxt{\textit{Bounded ranges}}\newtxt{\techterm{Common ranges}} are ranges for which
\tcode{begin} and \tcode{end} return objects of the
same type. \techterm{Random access ranges} are ranges for which
\tcode{begin} returns a type that satisfies
\tcode{RandomAccessIterator}~(\ref{range.iterators.random.access}). The range
categories \techterm{bidirectional ranges},
\techterm{forward ranges},
\techterm{input ranges}, and
\techterm{output ranges} are defined similarly.

\rSec2[range.range]{Ranges}

\pnum
The \tcode{Range} concept defines the requirements of a type that allows
iteration over its elements by providing a \tcode{begin} iterator and an
\tcode{end} sentinel.
\enternote Most algorithms requiring this concept simply forward to an
\tcode{Iterator}-based algorithm by calling \tcode{begin} and \tcode{end}. \exitnote

\begin{itemdecl}
template <class T>
concept @\oldtxt{bool}@ Range =
  requires(T&& t) {
    @ranges@::begin(t); // not necessarily equality-preserving (see below)
    @ranges@::end(t);
  };
\end{itemdecl}

\begin{itemdescr}

\pnum
Given an lvalue \tcode{t} of type \tcode{remove_reference_t<T>}, \tcode{Range<T>} is satisfied
only if

\begin{itemize}
\item \range{begin(t)}{end(t)} denotes a range.

\item Both \tcode{begin(t)} and \tcode{end(t)} are amortized constant time
and non-modifying. \enternote \tcode{begin(t)} and \tcode{end(t)} do not require
implicit expression variations~(\cxxref{concepts.lib.general.equality}). \exitnote

\item If \tcode{iterator_t<T>} satisfies \tcode{ForwardIterator},
\tcode{begin(t)} is equality preserving.
\end{itemize}
\end{itemdescr}

\pnum \enternote
Equality preservation of both \tcode{begin} and \tcode{end} enables passing a \tcode{Range}
whose iterator type satisfies \tcode{ForwardIterator}
to multiple algorithms and
making multiple passes over the range by repeated calls to \tcode{begin} and \tcode{end}.
Since \tcode{begin} is not required to be equality preserving when the return type does
not satisfy \tcode{ForwardIterator}, repeated calls might not return equal values or
might not be well-defined; \tcode{begin} should be called at most once for such a range.
\exitnote

\rSec2[range.sized]{Sized ranges}

\pnum
The \tcode{SizedRange} concept specifies the requirements
of a \tcode{Range} type that knows its size in constant time with the
\tcode{size} function.

\begin{itemdecl}
template <class T>
concept @\oldtxt{bool}@ SizedRange =
  Range<T> &&
  !disable_sized_range<remove_cv@\oldtxt{_t<remove_}@ref@\oldtxt{erence}@_t<T>@\oldtxt{>}@> &&
  requires(T& t) {
    { @ranges@::size(t) } -> ConvertibleTo<difference_type_t<iterator_t<T>>>;
  };
\end{itemdecl}

\begin{itemdescr}
\pnum
Given an lvalue \tcode{t} of type \tcode{remove_reference_t<T>}, \tcode{SizedRange<T>} is satisfied only if:

\begin{itemize}
\item \tcode{ranges::size(t)} is \bigoh{1}, does not modify \tcode{t}, and is equal
to \tcode{ranges::distance(t)}.

\item If \tcode{iterator_t<T>} satisfies \tcode{ForwardIterator},
\tcode{size(t)} is well-defined regardless of the evaluation of
\tcode{begin(t)}. \enternote \tcode{size(t)} is otherwise not required be
well-defined after evaluating \tcode{begin(t)}. For a \tcode{SizedRange}
whose iterator type does not model \tcode{ForwardIterator}, for
example, \tcode{size(t)} might only be well-defined if evaluated before
the first call to \tcode{begin(t)}. \exitnote
\end{itemize}

\pnum
\enternote The \tcode{disable_sized_range} predicate provides a mechanism to enable use
of range types with the library that meet the syntactic requirements but do
not in fact satisfy \tcode{SizedRange}. A program that instantiates a library template
that requires a \tcode{Range} with such a range type \tcode{R} is ill-formed with no
diagnostic required unless
\tcode{disable_sized_range<remove_cv\oldtxt{_t<remove_}ref\oldtxt{erence}_t<R>\oldtxt{>}>} evaluates
to \tcode{true}~(\cxxref{structure.requirements}). \exitnote
\end{itemdescr}

\rSec2[range.view]{Views}

\pnum
The \tcode{View} concept specifies the requirements of a
\tcode{Range} type that has constant time copy, move and assignment operators; that
is, the cost of these operations is not proportional to the number of elements in
the \tcode{View}.

\pnum
\enterexample
Examples of \tcode{View}s are:

\begin{itemize}
\item A \tcode{Range} type that wraps a pair of iterators.

\item A \tcode{Range} type that holds its elements by \tcode{shared_ptr}
and shares ownership with all its copies.

\item A \tcode{Range} type that generates its elements on demand.
\end{itemize}

A container~(\cxxref{containers}) is not a \tcode{View} since copying the
container copies the elements, which cannot be done in constant time.
\exitexample

\begin{itemdecl}
template <class T>
constexpr bool @\placeholder{view-predicate}@ // \expos
  = @\seebelow@;

template <class T>
concept @\oldtxt{bool}@ View =
  Range<T> &&
  Semiregular<T> &&
  @\placeholder{view-predicate}@<T>;
\end{itemdecl}

\begin{itemdescr}
\pnum
Since the difference between \tcode{Range} and \tcode{View} is largely semantic, the
two are differentiated with the help of the \tcode{enable_view}
trait. Users may specialize \tcode{enable_view}
to derive from \tcode{true_type} or \tcode{false_type}.

\pnum
For a type \tcode{T}, the value of \tcode{\placeholder{view-predicate}<T>} shall be:
\begin{itemize}
\item If \tcode{enable_view<T>} has a member type \tcode{type}, \tcode{enable_view<T>::type::value};
\item Otherwise, if \tcode{T} is derived from \tcode{view_base}, \tcode{true};
\item Otherwise, if \tcode{T} is an instantiation of class template
\tcode{initializer_list}~(\cxxref{support.initlist}),
\tcode{set}~(\cxxref{set}),
\tcode{multiset}~(\cxxref{multiset}),
\tcode{unordered_set}~(\cxxref{unord.set}), or
\tcode{unordered_multiset}~(\cxxref{unord.multiset}), \tcode{false};
\item Otherwise, if both \tcode{T} and \tcode{const T} satisfy \tcode{Range} and
\tcode{reference_t<iterator_t<T>{>}} is not the same type as \tcode{reference_t<iterator_t<const T>{>}},
\tcode{false}; \enternote Deep \tcode{const}-ness implies element ownership, whereas shallow \tcode{const}-ness
implies reference semantics. \exitnote
\item Otherwise, \tcode{true}.
\end{itemize}
\end{itemdescr}

\rSec2[range.common]{Common ranges}

\ednote{We've renamed ``\tcode{BoundedRange}'' to ``\tcode{CommonRange}''. The authors believe
this is a better name than ``\tcode{ClassicRange}'', which LEWG weakly preferred. The reason is
that the iterator and sentinel of a Common range have the same type in \textit{common}.
A non-Common range can be turned into a Common range with the help of \tcode{common_iterator}.
P0789 ``Range Adaptors and Utilities'' will be proposing a \tcode{view::common} adaptor that
does precisely that.}

\pnum
The \oldtxt{\tcode{BoundedRange}}\newtxt{\tcode{CommonRange}} concept specifies requirements
of a \tcode{Range} type for which \tcode{begin} and \tcode{end} return objects of
the same type. \enternote The standard containers~(\cxxref{containers})
satisfy \oldtxt{\tcode{BoundedRange}}\newtxt{\tcode{CommonRange}}.\exitnote

\begin{codeblock}
template <class T>
concept @\oldtxt{bool}@ @\oldtxt{BoundedRange}\newtxt{CommonRange}@ =
  Range<T> && Same<iterator_t<T>, sentinel_t<T>>;
\end{codeblock}

\rSec2[range.input]{Input ranges}

\pnum
The \tcode{InputRange} concept specifies requirements of
a \tcode{Range} type for which \tcode{begin} returns a type
that satisfies \tcode{InputIterator}~(\ref{range.iterators.input}).

\begin{codeblock}
template <class T>
concept @\oldtxt{bool}@ InputRange =
  Range<T> && InputIterator<iterator_t<T>>;
\end{codeblock}

\rSec2[range.output]{Output ranges}

\pnum
The \tcode{OutputRange} concept specifies requirements of
a \tcode{Range} type for which \tcode{begin} returns a type that satisfies
\tcode{OutputIterator}~(\ref{range.iterators.output}).

\begin{codeblock}
template <class R, class T>
concept @\oldtxt{bool}@ OutputRange =
  Range<R> && OutputIterator<iterator_t<R>, T>;
\end{codeblock}

\rSec2[range.forward]{Forward ranges}

\pnum
The \tcode{ForwardRange} concept specifies requirements of an
\tcode{InputRange} type for which \tcode{begin} returns a type that satisfies
\tcode{ForwardIterator}~(\ref{range.iterators.forward}).

\begin{codeblock}
template <class T>
concept @\oldtxt{bool}@ ForwardRange =
  InputRange<T> && ForwardIterator<iterator_t<T>>;
\end{codeblock}

\rSec2[range.bidirectional]{Bidirectional ranges}

\pnum
The \tcode{BidirectionalRange} concept specifies requirements of a
\tcode{ForwardRange} type for which \tcode{begin} returns a type that satisfies
\tcode{BidirectionalIterator}~(\ref{range.iterators.bidirectional}).

\begin{codeblock}
template <class T>
concept @\oldtxt{bool}@ BidirectionalRange =
  ForwardRange<T> && BidirectionalIterator<iterator_t<T>>;
\end{codeblock}

\rSec2[range.random.access]{Random access ranges}

\pnum
The \tcode{RandomAccessRange} concept specifies requirements of a
\tcode{BidirectionalRange} type for which \tcode{begin} returns a type that satisfies
\tcode{RandomAccessIterator}~(\ref{range.iterators.random.access}).

\begin{codeblock}
template <class T>
concept @\oldtxt{bool}@ RandomAccessRange =
  BidirectionalRange<T> && RandomAccessIterator<iterator_t<T>>;
\end{codeblock}

\rSec1[dangling.wrappers]{Dangling wrapper}

\rSec2[range.dangling.wrap]{Class template \tcode{dangling}}

\pnum
\indexlibrary{\idxcode{dangling}}%
Class template \tcode{dangling} is a wrapper for an object that refers to another object whose
lifetime may have ended. It is used by algorithms that accept rvalue ranges and return iterators.

\begin{codeblock}
namespace std@\newtxt{::ranges}@ { @\oldtxt{namespace experimental \{ namespace ranges \{ inline namespace v1 \{}@
  template <CopyConstructible T>
  class dangling {
  public:
    constexpr dangling() requires DefaultConstructible<T>;
    constexpr dangling(T t);
    constexpr T get_unsafe() const;
  private:
    T value; // \expos
  };

  template <Range R>
  using safe_iterator_t =
    conditional_t<is_lvalue_reference@\newtxt{_v}@<R>@\oldtxt{::value}@,
      iterator_t<R>,
      dangling<iterator_t<R>>>;
}@\oldtxt{\}\}\}}@
\end{codeblock}

\rSec3[range.dangling.wrap.ops]{\tcode{dangling} operations}

\rSec4[range.dangling.wrap.op.const]{\tcode{dangling} constructors}

\indexlibrary{\idxcode{dangling}!\idxcode{dangling}}%
\begin{itemdecl}
constexpr dangling() requires DefaultConstructible<T>;
\end{itemdecl}

\begin{itemdescr}
\pnum
\effects Constructs a \tcode{dangling}, value-initializing \tcode{value}.
\end{itemdescr}

\indexlibrary{\idxcode{dangling}!\idxcode{dangling}}%
\begin{itemdecl}
constexpr dangling(T t);
\end{itemdecl}

\begin{itemdescr}
\pnum
\effects Constructs a \tcode{dangling}, initializing \tcode{value} with \tcode{t}.
\end{itemdescr}

\rSec4[range.dangling.wrap.op.get]{\tcode{dangling::get_unsafe}}

\indexlibrary{\idxcode{get_unsafe}!\idxcode{dangling}}%
\indexlibrary{\idxcode{dangling}!\idxcode{get_unsafe}}%
\begin{itemdecl}
constexpr T get_unsafe() const;
\end{itemdecl}

\begin{itemdescr}
\pnum
\returns \tcode{value}.
\end{itemdescr}

%!TEX root = std.tex
\rSec0[std2.algorithms]{Algorithms library}

\rSec1[std2.algorithms.general]{General}

\pnum
This Clause describes components that \Cpp programs may use to perform
algorithmic operations on containers (Clause~\cxxref{containers}) and other sequences.

\pnum
The following subclauses describe components for
non-modifying sequence operations,
modifying sequence operations,
and sorting and related operations,
as summarized in Table~\ref{tab:algorithms.summary}.

\begin{libsumtab}{Algorithms library summary}{tab:algorithms.summary}
\ref{std2.alg.nonmodifying}         & Non-modifying sequence operations & \\
\ref{std2.alg.modifying.operations} & Mutating sequence operations      & \tcode{<\changed{experimental/ranges}{std2}/algorithm>} \\
\ref{std2.alg.sorting}              & Sorting and related operations    & \\ \hline
\end{libsumtab}

\pnum
To ease transition, implementations provide additional algorithm signatures that
are deprecated in this document~(Annex~\ref{depr.algo.range-and-a-half}).

\synopsis{Header \tcode{<\changed{experimental/ranges}{std2}/algorithm>} synopsis}

\indexlibrary{\idxhdr{std2/algorithm}}%

\begin{codeblock}
#include <initializer_list>

namespace @\changed{std \{ namespace experimental \{ namespace ranges}{std2}@ { inline namespace v1 {
  namespace tag {
    // \ref{std2.alg.tagspec}, tag specifiers~(See \ref{std2.taggedtup.tagged}):
    struct in;
    struct in1;
    struct in2;
    struct out;
    struct out1;
    struct out2;
    struct fun;
    struct min;
    struct max;
    struct begin;
    struct end;
  }

  // \ref{std2.alg.nonmodifying}, non-modifying sequence operations:
  template <InputIterator I, Sentinel<I> S, class Proj = identity,
      IndirectUnaryPredicate<projected<I, Proj>> Pred>
    bool all_of(I first, S last, Pred pred, Proj proj = Proj{});

  template <InputRange Rng, class Proj = identity,
      IndirectUnaryPredicate<projected<iterator_t<Rng>, Proj>> Pred>
    bool all_of(Rng&& rng, Pred pred, Proj proj = Proj{});

  template <InputIterator I, Sentinel<I> S, class Proj = identity,
      IndirectUnaryPredicate<projected<I, Proj>> Pred>
    bool any_of(I first, S last, Pred pred, Proj proj = Proj{});

  template <InputRange Rng, class Proj = identity,
      IndirectUnaryPredicate<projected<iterator_t<Rng>, Proj>> Pred>
    bool any_of(Rng&& rng, Pred pred, Proj proj = Proj{});

  template <InputIterator I, Sentinel<I> S, class Proj = identity,
      IndirectUnaryPredicate<projected<I, Proj>> Pred>
    bool none_of(I first, S last, Pred pred, Proj proj = Proj{});

  template <InputRange Rng, class Proj = identity,
      IndirectUnaryPredicate<projected<iterator_t<Rng>, Proj>> Pred>
    bool none_of(Rng&& rng, Pred pred, Proj proj = Proj{});

  template <InputIterator I, Sentinel<I> S, class Proj = identity,
      IndirectUnaryInvocable<projected<I, Proj>> Fun>
    tagged_pair<tag::in(I), tag::fun(Fun)>
      for_each(I first, S last, Fun f, Proj proj = Proj{});

  template <InputRange Rng, class Proj = identity,
      IndirectUnaryInvocable<projected<iterator_t<Rng>, Proj>> Fun>
    tagged_pair<tag::in(safe_iterator_t<Rng>), tag::fun(Fun)>
      for_each(Rng&& rng, Fun f, Proj proj = Proj{});

  template <InputIterator I, Sentinel<I> S, class T, class Proj = identity>
    requires IndirectRelation<equal_to<>, projected<I, Proj>, const T*>
    I find(I first, S last, const T& value, Proj proj = Proj{});

  template <InputRange Rng, class T, class Proj = identity>
    requires IndirectRelation<equal_to<>, projected<iterator_t<Rng>, Proj>, const T*>
    safe_iterator_t<Rng>
      find(Rng&& rng, const T& value, Proj proj = Proj{});

  template <InputIterator I, Sentinel<I> S, class Proj = identity,
      IndirectUnaryPredicate<projected<I, Proj>> Pred>
    I find_if(I first, S last, Pred pred, Proj proj = Proj{});

  template <InputRange Rng, class Proj = identity,
      IndirectUnaryPredicate<projected<iterator_t<Rng>, Proj>> Pred>
    safe_iterator_t<Rng>
      find_if(Rng&& rng, Pred pred, Proj proj = Proj{});

  template <InputIterator I, Sentinel<I> S, class Proj = identity,
      IndirectUnaryPredicate<projected<I, Proj>> Pred>
    I find_if_not(I first, S last, Pred pred, Proj proj = Proj{});

  template <InputRange Rng, class Proj = identity,
      IndirectUnaryPredicate<projected<iterator_t<Rng>, Proj>> Pred>
    safe_iterator_t<Rng>
      find_if_not(Rng&& rng, Pred pred, Proj proj = Proj{});

  template <ForwardIterator I1, Sentinel<I1> S1, ForwardIterator I2,
      Sentinel<I2> S2, class Proj = identity,
      IndirectRelation<I2, projected<I1, Proj>> Pred = equal_to<>>
    I1
      find_end(I1 first1, S1 last1, I2 first2, S2 last2,
               Pred pred = Pred{}, Proj proj = Proj{});

  template <ForwardRange Rng1, ForwardRange Rng2, class Proj = identity,
      IndirectRelation<iterator_t<Rng2>,
        projected<iterator_t<Rng>, Proj>> Pred = equal_to<>>
    safe_iterator_t<Rng1>
      find_end(Rng1&& rng1, Rng2&& rng2, Pred pred = Pred{}, Proj proj = Proj{});

  template <InputIterator I1, Sentinel<I1> S1, ForwardIterator I2, Sentinel<I2> S2,
      class Proj1 = identity, class Proj2 = identity,
      IndirectRelation<projected<I1, Proj1>, projected<I2, Proj2>> Pred = equal_to<>>
    I1
      find_first_of(I1 first1, S1 last1, I2 first2, S2 last2,
                    Pred pred = Pred{},
                    Proj1 proj1 = Proj1{}, Proj2 proj2 = Proj2{});

  template <InputRange Rng1, ForwardRange Rng2, class Proj1 = identity,
      class Proj2 = identity,
      IndirectRelation<projected<iterator_t<Rng1>, Proj1>,
        projected<iterator_t<Rng2>, Proj2>> Pred = equal_to<>>
    safe_iterator_t<Rng1>
      find_first_of(Rng1&& rng1, Rng2&& rng2,
                    Pred pred = Pred{},
                    Proj1 proj1 = Proj1{}, Proj2 proj2 = Proj2{});

  template <ForwardIterator I, Sentinel<I> S, class Proj = identity,
      IndirectRelation<projected<I, Proj>> Pred = equal_to<>>
    I
      adjacent_find(I first, S last, Pred pred = Pred{},
                    Proj proj = Proj{});

  template <ForwardRange Rng, class Proj = identity,
      IndirectRelation<projected<iterator_t<Rng>, Proj>> Pred = equal_to<>>
    safe_iterator_t<Rng>
      adjacent_find(Rng&& rng, Pred pred = Pred{}, Proj proj = Proj{});

  template <InputIterator I, Sentinel<I> S, class T, class Proj = identity>
    requires IndirectRelation<equal_to<>, projected<I, Proj>, const T*>
    difference_type_t<I>
      count(I first, S last, const T& value, Proj proj = Proj{});

  template <InputRange Rng, class T, class Proj = identity>
    requires IndirectRelation<equal_to<>, projected<iterator_t<Rng>, Proj>, const T*>
    difference_type_t<iterator_t<Rng>>
      count(Rng&& rng, const T& value, Proj proj = Proj{});

  template <InputIterator I, Sentinel<I> S, class Proj = identity,
      IndirectUnaryPredicate<projected<I, Proj>> Pred>
    difference_type_t<I>
      count_if(I first, S last, Pred pred, Proj proj = Proj{});

  template <InputRange Rng, class Proj = identity,
      IndirectUnaryPredicate<projected<iterator_t<Rng>, Proj>> Pred>
    difference_type_t<iterator_t<Rng>>
      count_if(Rng&& rng, Pred pred, Proj proj = Proj{});

  template <InputIterator I1, Sentinel<I1> S1, InputIterator I2, Sentinel<I2> S2,
      class Proj1 = identity, class Proj2 = identity,
      IndirectRelation<projected<I1, Proj1>, projected<I2, Proj2>> Pred = equal_to<>>
    tagged_pair<tag::in1(I1), tag::in2(I2)>
      mismatch(I1 first1, S1 last1, I2 first2, S2 last2, Pred pred = Pred{},
               Proj1 proj1 = Proj1{}, Proj2 proj2 = Proj2{});

  template <InputRange Rng1, InputRange Rng2,
      class Proj1 = identity, class Proj2 = identity,
      IndirectRelation<projected<iterator_t<Rng1>, Proj1>,
        projected<iterator_t<Rng2>, Proj2>> Pred = equal_to<>>
    tagged_pair<tag::in1(safe_iterator_t<Rng1>),
                tag::in2(safe_iterator_t<Rng2>)>
      mismatch(Rng1&& rng1, Rng2&& rng2, Pred pred = Pred{},
               Proj1 proj1 = Proj1{}, Proj2 proj2 = Proj2{});

  template <InputIterator I1, Sentinel<I1> S1, InputIterator I2, Sentinel<I2> S2,
      class Pred = equal_to<>, class Proj1 = identity, class Proj2 = identity>
    requires IndirectlyComparable<I1, I2, Pred, Proj1, Proj2>
    bool equal(I1 first1, S1 last1, I2 first2, S2 last2,
               Pred pred = Pred{},
               Proj1 proj1 = Proj1{}, Proj2 proj2 = Proj2{});

  template <InputRange Rng1, InputRange Rng2, class Pred = equal_to<>,
      class Proj1 = identity, class Proj2 = identity>
    requires IndirectlyComparable<iterator_t<Rng1>, iterator_t<Rng2>, Pred, Proj1, Proj2>
    bool equal(Rng1&& rng1, Rng2&& rng2, Pred pred = Pred{},
               Proj1 proj1 = Proj1{}, Proj2 proj2 = Proj2{});


  template <ForwardIterator I1, Sentinel<I1> S1, ForwardIterator I2,
      Sentinel<I2> S2, class Pred = equal_to<>, class Proj1 = identity,
      class Proj2 = identity>
    requires IndirectlyComparable<I1, I2, Pred, Proj1, Proj2>
    bool is_permutation(I1 first1, S1 last1, I2 first2, S2 last2,
                        Pred pred = Pred{},
                        Proj1 proj1 = Proj1{}, Proj2 proj2 = Proj2{});

  template <ForwardRange Rng1, ForwardRange Rng2, class Pred = equal_to<>,
      class Proj1 = identity, class Proj2 = identity>
    requires IndirectlyComparable<iterator_t<Rng1>, iterator_t<Rng2>, Pred, Proj1, Proj2>
    bool is_permutation(Rng1&& rng1, Rng2&& rng2, Pred pred = Pred{},
                        Proj1 proj1 = Proj1{}, Proj2 proj2 = Proj2{});

  template <ForwardIterator I1, Sentinel<I1> S1, ForwardIterator I2,
      Sentinel<I2> S2, class Pred = equal_to<>,
      class Proj1 = identity, class Proj2 = identity>
    requires IndirectlyComparable<I1, I2, Pred, Proj1, Proj2>
    I1
      search(I1 first1, S1 last1, I2 first2, S2 last2,
             Pred pred = Pred{},
             Proj1 proj1 = Proj1{}, Proj2 proj2 = Proj2{});

  template <ForwardRange Rng1, ForwardRange Rng2, class Pred = equal_to<>,
      class Proj1 = identity, class Proj2 = identity>
    requires IndirectlyComparable<iterator_t<Rng1>, iterator_t<Rng2>, Pred, Proj1, Proj2>
    safe_iterator_t<Rng1>
      search(Rng1&& rng1, Rng2&& rng2, Pred pred = Pred{},
             Proj1 proj1 = Proj1{}, Proj2 proj2 = Proj2{});

  template <ForwardIterator I, Sentinel<I> S, class T,
      class Pred = equal_to<>, class Proj = identity>
    requires IndirectlyComparable<I, const T*, Pred, Proj>
    I
      search_n(I first, S last, difference_type_t<I> count,
               const T& value, Pred pred = Pred{},
               Proj proj = Proj{});

  template <ForwardRange Rng, class T, class Pred = equal_to<>,
      class Proj = identity>
    requires IndirectlyComparable<iterator_t<Rng>, const T*, Pred, Proj>
    safe_iterator_t<Rng>
      search_n(Rng&& rng, difference_type_t<iterator_t<Rng>> count,
               const T& value, Pred pred = Pred{}, Proj proj = Proj{});

  // \ref{std2.alg.modifying.operations}, modifying sequence operations:
  // \ref{std2.alg.copy}, copy:
  template <InputIterator I, Sentinel<I> S, WeaklyIncrementable O>
    requires IndirectlyCopyable<I, O>
    tagged_pair<tag::in(I), tag::out(O)>
      copy(I first, S last, O result);

  template <InputRange Rng, WeaklyIncrementable O>
    requires IndirectlyCopyable<iterator_t<Rng>, O>
    tagged_pair<tag::in(safe_iterator_t<Rng>), tag::out(O)>
      copy(Rng&& rng, O result);

  template <InputIterator I, WeaklyIncrementable O>
    requires IndirectlyCopyable<I, O>
    tagged_pair<tag::in(I), tag::out(O)>
      copy_n(I first, difference_type_t<I> n, O result);

  template <InputIterator I, Sentinel<I> S, WeaklyIncrementable O, class Proj = identity,
      IndirectUnaryPredicate<projected<I, Proj>> Pred>
    requires IndirectlyCopyable<I, O>
    tagged_pair<tag::in(I), tag::out(O)>
      copy_if(I first, S last, O result, Pred pred, Proj proj = Proj{});

  template <InputRange Rng, WeaklyIncrementable O, class Proj = identity,
      IndirectUnaryPredicate<projected<iterator_t<Rng>, Proj>> Pred>
    requires IndirectlyCopyable<iterator_t<Rng>, O>
    tagged_pair<tag::in(safe_iterator_t<Rng>), tag::out(O)>
      copy_if(Rng&& rng, O result, Pred pred, Proj proj = Proj{});

  template <BidirectionalIterator I1, Sentinel<I1> S1, BidirectionalIterator I2>
    requires IndirectlyCopyable<I1, I2>
    tagged_pair<tag::in(I1), tag::out(I2)>
      copy_backward(I1 first, S1 last, I2 result);

  template <BidirectionalRange Rng, BidirectionalIterator I>
    requires IndirectlyCopyable<iterator_t<Rng>, I>
    tagged_pair<tag::in(safe_iterator_t<Rng>), tag::out(I)>
      copy_backward(Rng&& rng, I result);

  // \ref{std2.alg.move}, move:
  template <InputIterator I, Sentinel<I> S, WeaklyIncrementable O>
    requires IndirectlyMovable<I, O>
    tagged_pair<tag::in(I), tag::out(O)>
      move(I first, S last, O result);

  template <InputRange Rng, WeaklyIncrementable O>
    requires IndirectlyMovable<iterator_t<Rng>, O>
    tagged_pair<tag::in(safe_iterator_t<Rng>), tag::out(O)>
      move(Rng&& rng, O result);

  template <BidirectionalIterator I1, Sentinel<I1> S1, BidirectionalIterator I2>
    requires IndirectlyMovable<I1, I2>
    tagged_pair<tag::in(I1), tag::out(I2)>
      move_backward(I1 first, S1 last, I2 result);

  template <BidirectionalRange Rng, BidirectionalIterator I>
    requires IndirectlyMovable<iterator_t<Rng>, I>
    tagged_pair<tag::in(safe_iterator_t<Rng>), tag::out(I)>
      move_backward(Rng&& rng, I result);

  template <ForwardIterator I1, Sentinel<I1> S1, ForwardIterator I2, Sentinel<I2> S2>
    requires IndirectlySwappable<I1, I2>
    tagged_pair<tag::in1(I1), tag::in2(I2)>
      swap_ranges(I1 first1, S1 last1, I2 first2, S2 last2);

  template <ForwardRange Rng1, ForwardRange Rng2>
    requires IndirectlySwappable<iterator_t<Rng1>, iterator_t<Rng2>>
    tagged_pair<tag::in1(safe_iterator_t<Rng1>), tag::in2(safe_iterator_t<Rng2>)>
      swap_ranges(Rng1&& rng1, Rng2&& rng2);

  template <InputIterator I, Sentinel<I> S, WeaklyIncrementable O,
      CopyConstructible F, class Proj = identity>
    requires Writable<O, indirect_result_of_t<F&(projected<I, Proj>)>>
    tagged_pair<tag::in(I), tag::out(O)>
      transform(I first, S last, O result, F op, Proj proj = Proj{});

  template <InputRange Rng, WeaklyIncrementable O, CopyConstructible F,
      class Proj = identity>
    requires Writable<O, indirect_result_of_t<F&(
      projected<iterator_t<R>, Proj>)>>
    tagged_pair<tag::in(safe_iterator_t<Rng>), tag::out(O)>
      transform(Rng&& rng, O result, F op, Proj proj = Proj{});

  template <InputIterator I1, Sentinel<I1> S1, InputIterator I2, Sentinel<I2> S2,
      WeaklyIncrementable O, CopyConstructible F, class Proj1 = identity,
      class Proj2 = identity>
    requires Writable<O, indirect_result_of_t<F&(projected<I1, Proj1>,
      projected<I2, Proj2>)>>
    tagged_tuple<tag::in1(I1), tag::in2(I2), tag::out(O)>
      transform(I1 first1, S1 last1, I2 first2, S2 last2, O result,
              F binary_op, Proj1 proj1 = Proj1{}, Proj2 proj2 = Proj2{});

  template <InputRange Rng1, InputRange Rng2, WeaklyIncrementable O,
      CopyConstructible F, class Proj1 = identity, class Proj2 = identity>
    requires Writable<O, indirect_result_of_t<F&(
      projected<iterator_t<Rng1>, Proj1>, projected<iterator_t<Rng2>, Proj2>)>>
    tagged_tuple<tag::in1(safe_iterator_t<Rng1>),
                 tag::in2(safe_iterator_t<Rng2>),
                 tag::out(O)>
      transform(Rng1&& rng1, Rng2&& rng2, O result,
                F binary_op, Proj1 proj1 = Proj1{}, Proj2 proj2 = Proj2{});

  template <InputIterator I, Sentinel<I> S, class T1, class T2, class Proj = identity>
    requires Writable<I, const T2&> &&
      IndirectRelation<equal_to<>, projected<I, Proj>, const T1*>
    I
      replace(I first, S last, const T1& old_value, const T2& new_value, Proj proj = Proj{});

  template <InputRange Rng, class T1, class T2, class Proj = identity>
    requires Writable<iterator_t<Rng>, const T2&> &&
      IndirectRelation<equal_to<>, projected<iterator_t<Rng>, Proj>, const T1*>
    safe_iterator_t<Rng>
      replace(Rng&& rng, const T1& old_value, const T2& new_value, Proj proj = Proj{});

  template <InputIterator I, Sentinel<I> S, class T, class Proj = identity,
      IndirectUnaryPredicate<projected<I, Proj>> Pred>
    requires Writable<I, const T&>
    I
      replace_if(I first, S last, Pred pred, const T& new_value, Proj proj = Proj{});

  template <InputRange Rng, class T, class Proj = identity,
      IndirectUnaryPredicate<projected<iterator_t<Rng>, Proj>> Pred>
    requires Writable<iterator_t<Rng>, const T&>
    safe_iterator_t<Rng>
      replace_if(Rng&& rng, Pred pred, const T& new_value, Proj proj = Proj{});

  template <InputIterator I, Sentinel<I> S, class T1, class T2, OutputIterator<const T2&> O,
      class Proj = identity>
    requires IndirectlyCopyable<I, O> &&
      IndirectRelation<equal_to<>, projected<I, Proj>, const T1*>
    tagged_pair<tag::in(I), tag::out(O)>
      replace_copy(I first, S last, O result, const T1& old_value, const T2& new_value,
                   Proj proj = Proj{});

  template <InputRange Rng, class T1, class T2, OutputIterator<const T2&> O,
      class Proj = identity>
    requires IndirectlyCopyable<iterator_t<Rng>, O> &&
      IndirectRelation<equal_to<>, projected<iterator_t<Rng>, Proj>, const T1*>
    tagged_pair<tag::in(safe_iterator_t<Rng>), tag::out(O)>
      replace_copy(Rng&& rng, O result, const T1& old_value, const T2& new_value,
                   Proj proj = Proj{});

  template <InputIterator I, Sentinel<I> S, class T, OutputIterator<const T&> O,
      class Proj = identity, IndirectUnaryPredicate<projected<I, Proj>> Pred>
    requires IndirectlyCopyable<I, O>
    tagged_pair<tag::in(I), tag::out(O)>
      replace_copy_if(I first, S last, O result, Pred pred, const T& new_value,
                      Proj proj = Proj{});

  template <InputRange Rng, class T, OutputIterator<const T&> O, class Proj = identity,
      IndirectUnaryPredicate<projected<iterator_t<Rng>, Proj>> Pred>
    requires IndirectlyCopyable<iterator_t<Rng>, O>
    tagged_pair<tag::in(safe_iterator_t<Rng>), tag::out(O)>
      replace_copy_if(Rng&& rng, O result, Pred pred, const T& new_value,
                      Proj proj = Proj{});

  template <class T, OutputIterator<const T&> O, Sentinel<O> S>
    O fill(O first, S last, const T& value);

  template <class T, OutputRange<const T&> Rng>
    safe_iterator_t<Rng>
      fill(Rng&& rng, const T& value);

  template <class T, OutputIterator<const T&> O>
    O fill_n(O first, difference_type_t<O> n, const T& value);

  template <Iterator O, Sentinel<O> S, CopyConstructible F>
      requires Invocable<F&> && Writable<O, result_of_t<F&()>>
    O generate(O first, S last, F gen);

  template <class Rng, CopyConstructible F>
      requires Invocable<F&> && OutputRange<Rng, result_of_t<F&()>>
    safe_iterator_t<Rng>
      generate(Rng&& rng, F gen);

  template <Iterator O, CopyConstructible F>
      requires Invocable<F&> && Writable<O, result_of_t<F&()>>
    O generate_n(O first, difference_type_t<O> n, F gen);

  template <ForwardIterator I, Sentinel<I> S, class T, class Proj = identity>
    requires Permutable<I> &&
      IndirectRelation<equal_to<>, projected<I, Proj>, const T*>
    I remove(I first, S last, const T& value, Proj proj = Proj{});

  template <ForwardRange Rng, class T, class Proj = identity>
    requires Permutable<iterator_t<Rng>> &&
      IndirectRelation<equal_to<>, projected<iterator_t<Rng>, Proj>, const T*>
    safe_iterator_t<Rng>
      remove(Rng&& rng, const T& value, Proj proj = Proj{});

  template <ForwardIterator I, Sentinel<I> S, class Proj = identity,
      IndirectUnaryPredicate<projected<I, Proj>> Pred>
    requires Permutable<I>
    I remove_if(I first, S last, Pred pred, Proj proj = Proj{});

  template <ForwardRange Rng, class Proj = identity,
      IndirectUnaryPredicate<projected<iterator_t<Rng>, Proj>> Pred>
    requires Permutable<iterator_t<Rng>>
    safe_iterator_t<Rng>
      remove_if(Rng&& rng, Pred pred, Proj proj = Proj{});

  template <InputIterator I, Sentinel<I> S, WeaklyIncrementable O, class T,
      class Proj = identity>
    requires IndirectlyCopyable<I, O> &&
      IndirectRelation<equal_to<>, projected<I, Proj>, const T*>
    tagged_pair<tag::in(I), tag::out(O)>
      remove_copy(I first, S last, O result, const T& value, Proj proj = Proj{});

  template <InputRange Rng, WeaklyIncrementable O, class T, class Proj = identity>
    requires IndirectlyCopyable<iterator_t<Rng>, O> &&
      IndirectRelation<equal_to<>, projected<iterator_t<Rng>, Proj>, const T*>
    tagged_pair<tag::in(safe_iterator_t<Rng>), tag::out(O)>
      remove_copy(Rng&& rng, O result, const T& value, Proj proj = Proj{});

  template <InputIterator I, Sentinel<I> S, WeaklyIncrementable O,
      class Proj = identity, IndirectUnaryPredicate<projected<I, Proj>> Pred>
    requires IndirectlyCopyable<I, O>
    tagged_pair<tag::in(I), tag::out(O)>
      remove_copy_if(I first, S last, O result, Pred pred, Proj proj = Proj{});

  template <InputRange Rng, WeaklyIncrementable O, class Proj = identity,
      IndirectUnaryPredicate<projected<iterator_t<Rng>, Proj>> Pred>
    requires IndirectlyCopyable<iterator_t<Rng>, O>
    tagged_pair<tag::in(safe_iterator_t<Rng>), tag::out(O)>
      remove_copy_if(Rng&& rng, O result, Pred pred, Proj proj = Proj{});

  template <ForwardIterator I, Sentinel<I> S, class Proj = identity,
      IndirectRelation<projected<I, Proj>> R = equal_to<>>
    requires Permutable<I>
    I unique(I first, S last, R comp = R{}, Proj proj = Proj{});

  template <ForwardRange Rng, class Proj = identity,
      IndirectRelation<projected<iterator_t<Rng>, Proj>> R = equal_to<>>
    requires Permutable<iterator_t<Rng>>
    safe_iterator_t<Rng>
      unique(Rng&& rng, R comp = R{}, Proj proj = Proj{});

  template <InputIterator I, Sentinel<I> S, WeaklyIncrementable O,
      class Proj = identity, IndirectRelation<projected<I, Proj>> R = equal_to<>>
    requires IndirectlyCopyable<I, O> &&
      (ForwardIterator<I> ||
       (InputIterator<O> && Same<value_type_t<I>, value_type_t<O>>) ||
       IndirectlyCopyableStorable<I, O>)
    tagged_pair<tag::in(I), tag::out(O)>
      unique_copy(I first, S last, O result, R comp = R{}, Proj proj = Proj{});

  template <InputRange Rng, WeaklyIncrementable O, class Proj = identity,
      IndirectRelation<projected<iterator_t<Rng>, Proj>> R = equal_to<>>
    requires IndirectlyCopyable<iterator_t<Rng>, O> &&
      (ForwardIterator<iterator_t<Rng>> ||
       (InputIterator<O> && Same<value_type_t<iterator_t<Rng>>, value_type_t<O>>) ||
       IndirectlyCopyableStorable<iterator_t<Rng>, O>)
    tagged_pair<tag::in(safe_iterator_t<Rng>), tag::out(O)>
      unique_copy(Rng&& rng, O result, R comp = R{}, Proj proj = Proj{});

  template <BidirectionalIterator I, Sentinel<I> S>
    requires Permutable<I>
    I reverse(I first, S last);

  template <BidirectionalRange Rng>
    requires Permutable<iterator_t<Rng>>
    safe_iterator_t<Rng>
      reverse(Rng&& rng);

  template <BidirectionalIterator I, Sentinel<I> S, WeaklyIncrementable O>
    requires IndirectlyCopyable<I, O>
    tagged_pair<tag::in(I), tag::out(O)> reverse_copy(I first, S last, O result);

  template <BidirectionalRange Rng, WeaklyIncrementable O>
    requires IndirectlyCopyable<iterator_t<Rng>, O>
    tagged_pair<tag::in(safe_iterator_t<Rng>), tag::out(O)>
      reverse_copy(Rng&& rng, O result);

  template <ForwardIterator I, Sentinel<I> S>
    requires Permutable<I>
    tagged_pair<tag::begin(I), tag::end(I)>
      rotate(I first, I middle, S last);

  template <ForwardRange Rng>
    requires Permutable<iterator_t<Rng>>
    tagged_pair<tag::begin(safe_iterator_t<Rng>),
                tag::end(safe_iterator_t<Rng>)>
      rotate(Rng&& rng, iterator_t<Rng> middle);

  template <ForwardIterator I, Sentinel<I> S, WeaklyIncrementable O>
    requires IndirectlyCopyable<I, O>
    tagged_pair<tag::in(I), tag::out(O)>
      rotate_copy(I first, I middle, S last, O result);

  template <ForwardRange Rng, WeaklyIncrementable O>
    requires IndirectlyCopyable<iterator_t<Rng>, O>
    tagged_pair<tag::in(safe_iterator_t<Rng>), tag::out(O)>
      rotate_copy(Rng&& rng, iterator_t<Rng> middle, O result);

  // \ref{std2.alg.random.shuffle}, shuffle:
  template <RandomAccessIterator I, Sentinel<I> S, class Gen>
    requires Permutable<I> &&
      UniformRandomNumberGenerator<remove_reference_t<Gen>> &&
      ConvertibleTo<result_of_t<Gen&()>, difference_type_t<I>>
    I shuffle(I first, S last, Gen&& g);

  template <RandomAccessRange Rng, class Gen>
    requires Permutable<I> &&
      UniformRandomNumberGenerator<remove_reference_t<Gen>> &&
      ConvertibleTo<result_of_t<Gen&()>, difference_type_t<I>>
    safe_iterator_t<Rng>
      shuffle(Rng&& rng, Gen&& g);

  // \ref{std2.alg.partitions}, partitions:
  template <InputIterator I, Sentinel<I> S, class Proj = identity,
      IndirectUnaryPredicate<projected<I, Proj>> Pred>
    bool is_partitioned(I first, S last, Pred pred, Proj proj = Proj{});

  template <InputRange Rng, class Proj = identity,
      IndirectUnaryPredicate<projected<iterator_t<Rng>, Proj>> Pred>
    bool
      is_partitioned(Rng&& rng, Pred pred, Proj proj = Proj{});

  template <ForwardIterator I, Sentinel<I> S, class Proj = identity,
      IndirectUnaryPredicate<projected<I, Proj>> Pred>
    requires Permutable<I>
    I partition(I first, S last, Pred pred, Proj proj = Proj{});

  template <ForwardRange Rng, class Proj = identity,
      IndirectUnaryPredicate<projected<iterator_t<Rng>, Proj>> Pred>
    requires Permutable<iterator_t<Rng>>
    safe_iterator_t<Rng>
      partition(Rng&& rng, Pred pred, Proj proj = Proj{});

  template <BidirectionalIterator I, Sentinel<I> S, class Proj = identity,
      IndirectUnaryPredicate<projected<I, Proj>> Pred>
    requires Permutable<I>
    I stable_partition(I first, S last, Pred pred, Proj proj = Proj{});

  template <BidirectionalRange Rng, class Proj = identity,
      IndirectUnaryPredicate<projected<iterator_t<Rng>, Proj>> Pred>
    requires Permutable<iterator_t<Rng>>
    safe_iterator_t<Rng>
      stable_partition(Rng&& rng, Pred pred, Proj proj = Proj{});

  template <InputIterator I, Sentinel<I> S, WeaklyIncrementable O1, WeaklyIncrementable O2,
      class Proj = identity, IndirectUnaryPredicate<projected<I, Proj>> Pred>
    requires IndirectlyCopyable<I, O1> && IndirectlyCopyable<I, O2>
    tagged_tuple<tag::in(I), tag::out1(O1), tag::out2(O2)>
      partition_copy(I first, S last, O1 out_true, O2 out_false, Pred pred,
                     Proj proj = Proj{});

  template <InputRange Rng, WeaklyIncrementable O1, WeaklyIncrementable O2,
      class Proj = identity,
      IndirectUnaryPredicate<projected<iterator_t<Rng>, Proj>> Pred>
    requires IndirectlyCopyable<iterator_t<Rng>, O1> &&
      IndirectlyCopyable<iterator_t<Rng>, O2>
    tagged_tuple<tag::in(safe_iterator_t<Rng>), tag::out1(O1), tag::out2(O2)>
      partition_copy(Rng&& rng, O1 out_true, O2 out_false, Pred pred, Proj proj = Proj{});

  template <ForwardIterator I, Sentinel<I> S, class Proj = identity,
      IndirectUnaryPredicate<projected<I, Proj>> Pred>
    I partition_point(I first, S last, Pred pred, Proj proj = Proj{});

  template <ForwardRange Rng, class Proj = identity,
      IndirectUnaryPredicate<projected<iterator_t<Rng>, Proj>> Pred>
    safe_iterator_t<Rng>
      partition_point(Rng&& rng, Pred pred, Proj proj = Proj{});

  // \ref{std2.alg.sorting}, sorting and related operations:
  // \ref{std2.alg.sort}, sorting:
  template <RandomAccessIterator I, Sentinel<I> S, class Comp = less<>,
      class Proj = identity>
    requires Sortable<I, Comp, Proj>
    I sort(I first, S last, Comp comp = Comp{}, Proj proj = Proj{});

  template <RandomAccessRange Rng, class Comp = less<>, class Proj = identity>
    requires Sortable<iterator_t<Rng>, Comp, Proj>
    safe_iterator_t<Rng>
      sort(Rng&& rng, Comp comp = Comp{}, Proj proj = Proj{});

  template <RandomAccessIterator I, Sentinel<I> S, class Comp = less<>,
      class Proj = identity>
    requires Sortable<I, Comp, Proj>
    I stable_sort(I first, S last, Comp comp = Comp{}, Proj proj = Proj{});

  template <RandomAccessRange Rng, class Comp = less<>, class Proj = identity>
    requires Sortable<iterator_t<Rng>, Comp, Proj>
    safe_iterator_t<Rng>
      stable_sort(Rng&& rng, Comp comp = Comp{}, Proj proj = Proj{});

  template <RandomAccessIterator I, Sentinel<I> S, class Comp = less<>,
      class Proj = identity>
    requires Sortable<I, Comp, Proj>
    I partial_sort(I first, I middle, S last, Comp comp = Comp{}, Proj proj = Proj{});

  template <RandomAccessRange Rng, class Comp = less<>, class Proj = identity>
    requires Sortable<iterator_t<Rng>, Comp, Proj>
    safe_iterator_t<Rng>
      partial_sort(Rng&& rng, iterator_t<Rng> middle, Comp comp = Comp{},
                   Proj proj = Proj{});

  template <InputIterator I1, Sentinel<I1> S1, RandomAccessIterator I2, Sentinel<I2> S2,
      class Comp = less<>, class Proj1 = identity, class Proj2 = identity>
    requires IndirectlyCopyable<I1, I2> && Sortable<I2, Comp, Proj2> &&
        IndirectStrictWeakOrder<Comp, projected<I1, Proj1>, projected<I2, Proj2>>
    I2
      partial_sort_copy(I1 first, S1 last, I2 result_first, S2 result_last,
                        Comp comp = Comp{}, Proj1 proj1 = Proj1{}, Proj2 proj2 = Proj2{});

  template <InputRange Rng1, RandomAccessRange Rng2, class Comp = less<>,
      class Proj1 = identity, class Proj2 = identity>
    requires IndirectlyCopyable<iterator_t<Rng1>, iterator_t<Rng2>> &&
        Sortable<iterator_t<Rng2>, Comp, Proj2> &&
        IndirectStrictWeakOrder<Comp, projected<iterator_t<Rng1>, Proj1>,
          projected<iterator_t<Rng2>, Proj2>>
    safe_iterator_t<Rng2>
      partial_sort_copy(Rng1&& rng, Rng2&& result_rng, Comp comp = Comp{},
                        Proj1 proj1 = Proj1{}, Proj2 proj2 = Proj2{});

  template <ForwardIterator I, Sentinel<I> S, class Proj = identity,
      IndirectStrictWeakOrder<projected<I, Proj>> Comp = less<>>
    bool is_sorted(I first, S last, Comp comp = Comp{}, Proj proj = Proj{});

  template <ForwardRange Rng, class Proj = identity,
      IndirectStrictWeakOrder<projected<iterator_t<Rng>, Proj>> Comp = less<>>
    bool
      is_sorted(Rng&& rng, Comp comp = Comp{}, Proj proj = Proj{});

  template <ForwardIterator I, Sentinel<I> S, class Proj = identity,
      IndirectStrictWeakOrder<projected<I, Proj>> Comp = less<>>
    I is_sorted_until(I first, S last, Comp comp = Comp{}, Proj proj = Proj{});

  template <ForwardRange Rng, class Proj = identity,
      IndirectStrictWeakOrder<projected<iterator_t<Rng>, Proj>> Comp = less<>>
    safe_iterator_t<Rng>
      is_sorted_until(Rng&& rng, Comp comp = Comp{}, Proj proj = Proj{});

  template <RandomAccessIterator I, Sentinel<I> S, class Comp = less<>,
      class Proj = identity>
    requires Sortable<I, Comp, Proj>
    I nth_element(I first, I nth, S last, Comp comp = Comp{}, Proj proj = Proj{});

  template <RandomAccessRange Rng, class Comp = less<>, class Proj = identity>
    requires Sortable<iterator_t<Rng>, Comp, Proj>
    safe_iterator_t<Rng>
      nth_element(Rng&& rng, iterator_t<Rng> nth, Comp comp = Comp{}, Proj proj = Proj{});

  // \ref{std2.alg.binary.search}, binary search:
  template <ForwardIterator I, Sentinel<I> S, class T, class Proj = identity,
      IndirectStrictWeakOrder<const T*, projected<I, Proj>> Comp = less<>>
    I
      lower_bound(I first, S last, const T& value, Comp comp = Comp{},
                  Proj proj = Proj{});

  template <ForwardRange Rng, class T, class Proj = identity,
      IndirectStrictWeakOrder<const T*, projected<iterator_t<Rng>, Proj>> Comp = less<>>
    safe_iterator_t<Rng>
      lower_bound(Rng&& rng, const T& value, Comp comp = Comp{}, Proj proj = Proj{});

  template <ForwardIterator I, Sentinel<I> S, class T, class Proj = identity,
      IndirectStrictWeakOrder<const T*, projected<I, Proj>> Comp = less<>>
    I
      upper_bound(I first, S last, const T& value, Comp comp = Comp{}, Proj proj = Proj{});

  template <ForwardRange Rng, class T, class Proj = identity,
      IndirectStrictWeakOrder<const T*, projected<iterator_t<Rng>, Proj>> Comp = less<>>
    safe_iterator_t<Rng>
      upper_bound(Rng&& rng, const T& value, Comp comp = Comp{}, Proj proj = Proj{});

  template <ForwardIterator I, Sentinel<I> S, class T, class Proj = identity,
      IndirectStrictWeakOrder<const T*, projected<I, Proj>> Comp = less<>>
    tagged_pair<tag::begin(I), tag::end(I)>
      equal_range(I first, S last, const T& value, Comp comp = Comp{}, Proj proj = Proj{});

  template <ForwardRange Rng, class T, class Proj = identity,
      IndirectStrictWeakOrder<const T*, projected<iterator_t<Rng>, Proj>> Comp = less<>>
    tagged_pair<tag::begin(safe_iterator_t<Rng>),
                tag::end(safe_iterator_t<Rng>)>
      equal_range(Rng&& rng, const T& value, Comp comp = Comp{}, Proj proj = Proj{});

  template <ForwardIterator I, Sentinel<I> S, class T, class Proj = identity,
      IndirectStrictWeakOrder<const T*, projected<I, Proj>> Comp = less<>>
    bool
      binary_search(I first, S last, const T& value, Comp comp = Comp{},
                    Proj proj = Proj{});

  template <ForwardRange Rng, class T, class Proj = identity,
      IndirectStrictWeakOrder<const T*, projected<iterator_t<Rng>, Proj>> Comp = less<>>
    bool
      binary_search(Rng&& rng, const T& value, Comp comp = Comp{},
                    Proj proj = Proj{});

  // \ref{std2.alg.merge}, merge:
  template <InputIterator I1, Sentinel<I1> S1, InputIterator I2, Sentinel<I2> S2,
      WeaklyIncrementable O, class Comp = less<>, class Proj1 = identity,
      class Proj2 = identity>
    requires Mergeable<I1, I2, O, Comp, Proj1, Proj2>
    tagged_tuple<tag::in1(I1), tag::in2(I2), tag::out(O)>
      merge(I1 first1, S1 last1, I2 first2, S2 last2, O result,
            Comp comp = Comp{}, Proj1 proj1 = Proj1{}, Proj2 proj2 = Proj2{});

  template <InputRange Rng1, InputRange Rng2, WeaklyIncrementable O, class Comp = less<>,
      class Proj1 = identity, class Proj2 = identity>
    requires Mergeable<iterator_t<Rng1>, iterator_t<Rng2>, O, Comp, Proj1, Proj2>
    tagged_tuple<tag::in1(safe_iterator_t<Rng1>),
                 tag::in2(safe_iterator_t<Rng2>),
                 tag::out(O)>
      merge(Rng1&& rng1, Rng2&& rng2, O result,
            Comp comp = Comp{}, Proj1 proj1 = Proj1{}, Proj2 proj2 = Proj2{});

  template <BidirectionalIterator I, Sentinel<I> S, class Comp = less<>,
      class Proj = identity>
    requires Sortable<I, Comp, Proj>
    I
      inplace_merge(I first, I middle, S last, Comp comp = Comp{}, Proj proj = Proj{});

  template <BidirectionalRange Rng, class Comp = less<>, class Proj = identity>
    requires Sortable<iterator_t<Rng>, Comp, Proj>
    safe_iterator_t<Rng>
      inplace_merge(Rng&& rng, iterator_t<Rng> middle, Comp comp = Comp{},
                    Proj proj = Proj{});

  // \ref{std2.alg.set.operations}, set operations:
  template <InputIterator I1, Sentinel<I1> S1, InputIterator I2, Sentinel<I2> S2,
      class Proj1 = identity, class Proj2 = identity,
      IndirectStrictWeakOrder<projected<I1, Proj1>, projected<I2, Proj2>> Comp = less<>>
    bool
      includes(I1 first1, S1 last1, I2 first2, S2 last2, Comp comp = Comp{},
               Proj1 proj1 = Proj1{}, Proj2 proj2 = Proj2{});

  template <InputRange Rng1, InputRange Rng2, class Proj1 = identity,
      class Proj2 = identity,
      IndirectStrictWeakOrder<projected<iterator_t<Rng1>, Proj1>,
        projected<iterator_t<Rng2>, Proj2>> Comp = less<>>
    bool
      includes(Rng1&& rng1, Rng2&& rng2, Comp comp = Comp{},
               Proj1 proj1 = Proj1{}, Proj2 proj2 = Proj2{});

  template <InputIterator I1, Sentinel<I1> S1, InputIterator I2, Sentinel<I2> S2,
      WeaklyIncrementable O, class Comp = less<>, class Proj1 = identity, class Proj2 = identity>
    requires Mergeable<I1, I2, O, Comp, Proj1, Proj2>
    tagged_tuple<tag::in1(I1), tag::in2(I2), tag::out(O)>
      set_union(I1 first1, S1 last1, I2 first2, S2 last2, O result, Comp comp = Comp{},
                Proj1 proj1 = Proj1{}, Proj2 proj2 = Proj2{});

  template <InputRange Rng1, InputRange Rng2, WeaklyIncrementable O,
      class Comp = less<>, class Proj1 = identity, class Proj2 = identity>
    requires Mergeable<iterator_t<Rng1>, iterator_t<Rng2>, O, Comp, Proj1, Proj2>
    tagged_tuple<tag::in1(safe_iterator_t<Rng1>),
                 tag::in2(safe_iterator_t<Rng2>),
                 tag::out(O)>
      set_union(Rng1&& rng1, Rng2&& rng2, O result, Comp comp = Comp{},
                Proj1 proj1 = Proj1{}, Proj2 proj2 = Proj2{});

  template <InputIterator I1, Sentinel<I1> S1, InputIterator I2, Sentinel<I2> S2,
      WeaklyIncrementable O, class Comp = less<>, class Proj1 = identity, class Proj2 = identity>
    requires Mergeable<I1, I2, O, Comp, Proj1, Proj2>
    O
      set_intersection(I1 first1, S1 last1, I2 first2, S2 last2, O result,
                       Comp comp = Comp{}, Proj1 proj1 = Proj1{}, Proj2 proj2 = Proj2{});

  template <InputRange Rng1, InputRange Rng2, WeaklyIncrementable O,
      class Comp = less<>, class Proj1 = identity, class Proj2 = identity>
    requires Mergeable<iterator_t<Rng1>, iterator_t<Rng2>, O, Comp, Proj1, Proj2>
    O
      set_intersection(Rng1&& rng1, Rng2&& rng2, O result,
                       Comp comp = Comp{}, Proj1 proj1 = Proj1{}, Proj2 proj2 = Proj2{});

  template <InputIterator I1, Sentinel<I1> S1, InputIterator I2, Sentinel<I2> S2,
      WeaklyIncrementable O, class Comp = less<>, class Proj1 = identity, class Proj2 = identity>
    requires Mergeable<I1, I2, O, Comp, Proj1, Proj2>
    tagged_pair<tag::in1(I1), tag::out(O)>
      set_difference(I1 first1, S1 last1, I2 first2, S2 last2, O result,
                     Comp comp = Comp{}, Proj1 proj1 = Proj1{}, Proj2 proj2 = Proj2{});

  template <InputRange Rng1, InputRange Rng2, WeaklyIncrementable O,
      class Comp = less<>, class Proj1 = identity, class Proj2 = identity>
    requires Mergeable<iterator_t<Rng1>, iterator_t<Rng2>, O, Comp, Proj1, Proj2>
    tagged_pair<tag::in1(safe_iterator_t<Rng1>), tag::out(O)>
      set_difference(Rng1&& rng1, Rng2&& rng2, O result,
                     Comp comp = Comp{}, Proj1 proj1 = Proj1{}, Proj2 proj2 = Proj2{});

  template <InputIterator I1, Sentinel<I1> S1, InputIterator I2, Sentinel<I2> S2,
      WeaklyIncrementable O, class Comp = less<>, class Proj1 = identity, class Proj2 = identity>
    requires Mergeable<I1, I2, O, Comp, Proj1, Proj2>
    tagged_tuple<tag::in1(I1), tag::in2(I2), tag::out(O)>
      set_symmetric_difference(I1 first1, S1 last1, I2 first2, S2 last2, O result,
                               Comp comp = Comp{}, Proj1 proj1 = Proj1{},
                               Proj2 proj2 = Proj2{});

  template <InputRange Rng1, InputRange Rng2, WeaklyIncrementable O,
      class Comp = less<>, class Proj1 = identity, class Proj2 = identity>
    requires Mergeable<iterator_t<Rng1>, iterator_t<Rng2>, O, Comp, Proj1, Proj2>
    tagged_tuple<tag::in1(safe_iterator_t<Rng1>),
                 tag::in2(safe_iterator_t<Rng2>),
                 tag::out(O)>
      set_symmetric_difference(Rng1&& rng1, Rng2&& rng2, O result, Comp comp = Comp{},
                               Proj1 proj1 = Proj1{}, Proj2 proj2 = Proj2{});

  // \ref{std2.alg.heap.operations}, heap operations:
  template <RandomAccessIterator I, Sentinel<I> S, class Comp = less<>,
      class Proj = identity>
    requires Sortable<I, Comp, Proj>
    I push_heap(I first, S last, Comp comp = Comp{}, Proj proj = Proj{});

  template <RandomAccessRange Rng, class Comp = less<>, class Proj = identity>
    requires Sortable<iterator_t<Rng>, Comp, Proj>
    safe_iterator_t<Rng>
      push_heap(Rng&& rng, Comp comp = Comp{}, Proj proj = Proj{});

  template <RandomAccessIterator I, Sentinel<I> S, class Comp = less<>,
      class Proj = identity>
    requires Sortable<I, Comp, Proj>
    I pop_heap(I first, S last, Comp comp = Comp{}, Proj proj = Proj{});

  template <RandomAccessRange Rng, class Comp = less<>, class Proj = identity>
    requires Sortable<iterator_t<Rng>, Comp, Proj>
    safe_iterator_t<Rng>
      pop_heap(Rng&& rng, Comp comp = Comp{}, Proj proj = Proj{});

  template <RandomAccessIterator I, Sentinel<I> S, class Comp = less<>,
      class Proj = identity>
    requires Sortable<I, Comp, Proj>
    I make_heap(I first, S last, Comp comp = Comp{}, Proj proj = Proj{});

  template <RandomAccessRange Rng, class Comp = less<>, class Proj = identity>
    requires Sortable<iterator_t<Rng>, Comp, Proj>
    safe_iterator_t<Rng>
      make_heap(Rng&& rng, Comp comp = Comp{}, Proj proj = Proj{});

  template <RandomAccessIterator I, Sentinel<I> S, class Comp = less<>,
      class Proj = identity>
    requires Sortable<I, Comp, Proj>
    I sort_heap(I first, S last, Comp comp = Comp{}, Proj proj = Proj{});

  template <RandomAccessRange Rng, class Comp = less<>, class Proj = identity>
    requires Sortable<iterator_t<Rng>, Comp, Proj>
    safe_iterator_t<Rng>
      sort_heap(Rng&& rng, Comp comp = Comp{}, Proj proj = Proj{});

  template <RandomAccessIterator I, Sentinel<I> S, class Proj = identity,
      IndirectStrictWeakOrder<projected<I, Proj>> Comp = less<>>
    bool is_heap(I first, S last, Comp comp = Comp{}, Proj proj = Proj{});

  template <RandomAccessRange Rng, class Proj = identity,
      IndirectStrictWeakOrder<projected<iterator_t<Rng>, Proj>> Comp = less<>>
    bool
      is_heap(Rng&& rng, Comp comp = Comp{}, Proj proj = Proj{});

  template <RandomAccessIterator I, Sentinel<I> S, class Proj = identity,
      IndirectStrictWeakOrder<projected<I, Proj>> Comp = less<>>
    I is_heap_until(I first, S last, Comp comp = Comp{}, Proj proj = Proj{});

  template <RandomAccessRange Rng, class Proj = identity,
      IndirectStrictWeakOrder<projected<iterator_t<Rng>, Proj>> Comp = less<>>
    safe_iterator_t<Rng>
      is_heap_until(Rng&& rng, Comp comp = Comp{}, Proj proj = Proj{});

  // \ref{std2.alg.min.max}, minimum and maximum:
  template <class T, class Proj = identity,
      IndirectStrictWeakOrder<projected<const T*, Proj>> Comp = less<>>
    constexpr const T& min(const T& a, const T& b, Comp comp = Comp{}, Proj proj = Proj{});

  template <Copyable T, class Proj = identity,
      IndirectStrictWeakOrder<projected<const T*, Proj>> Comp = less<>>
    constexpr T min(initializer_list<T> t, Comp comp = Comp{}, Proj proj = Proj{});

  template <InputRange Rng, class Proj = identity,
      IndirectStrictWeakOrder<projected<iterator_t<Rng>, Proj>> Comp = less<>>
    requires Copyable<value_type_t<iterator_t<Rng>>>
    value_type_t<iterator_t<Rng>>
      min(Rng&& rng, Comp comp = Comp{}, Proj proj = Proj{});

  template <class T, class Proj = identity,
      IndirectStrictWeakOrder<projected<const T*, Proj>> Comp = less<>>
    constexpr const T& max(const T& a, const T& b, Comp comp = Comp{}, Proj proj = Proj{});

  template <Copyable T, class Proj = identity,
      IndirectStrictWeakOrder<projected<const T*, Proj>> Comp = less<>>
    constexpr T max(initializer_list<T> t, Comp comp = Comp{}, Proj proj = Proj{});

  template <InputRange Rng, class Proj = identity,
      IndirectStrictWeakOrder<projected<iterator_t<Rng>, Proj>> Comp = less<>>
    requires Copyable<value_type_t<iterator_t<Rng>>>
    value_type_t<iterator_t<Rng>>
      max(Rng&& rng, Comp comp = Comp{}, Proj proj = Proj{});

  template <class T, class Proj = identity,
      IndirectStrictWeakOrder<projected<const T*, Proj>> Comp = less<>>
    constexpr tagged_pair<tag::min(const T&), tag::max(const T&)>
      minmax(const T& a, const T& b, Comp comp = Comp{}, Proj proj = Proj{});

  template <Copyable T, class Proj = identity,
      IndirectStrictWeakOrder<projected<const T*, Proj>> Comp = less<>>
    constexpr tagged_pair<tag::min(T), tag::max(T)>
      minmax(initializer_list<T> t, Comp comp = Comp{}, Proj proj = Proj{});

  template <InputRange Rng, class Proj = identity,
      IndirectStrictWeakOrder<projected<iterator_t<Rng>, Proj>> Comp = less<>>
    requires Copyable<value_type_t<iterator_t<Rng>>>
    tagged_pair<tag::min(value_type_t<iterator_t<Rng>>),
                tag::max(value_type_t<iterator_t<Rng>>)>
      minmax(Rng&& rng, Comp comp = Comp{}, Proj proj = Proj{});

  template <ForwardIterator I, Sentinel<I> S, class Proj = identity,
      IndirectStrictWeakOrder<projected<I, Proj>> Comp = less<>>
    I min_element(I first, S last, Comp comp = Comp{}, Proj proj = Proj{});

  template <ForwardRange Rng, class Proj = identity,
      IndirectStrictWeakOrder<projected<iterator_t<Rng>, Proj>> Comp = less<>>
    safe_iterator_t<Rng>
      min_element(Rng&& rng, Comp comp = Comp{}, Proj proj = Proj{});

  template <ForwardIterator I, Sentinel<I> S, class Proj = identity,
      IndirectStrictWeakOrder<projected<I, Proj>> Comp = less<>>
    I max_element(I first, S last, Comp comp = Comp{}, Proj proj = Proj{});

  template <ForwardRange Rng, class Proj = identity,
      IndirectStrictWeakOrder<projected<iterator_t<Rng>, Proj>> Comp = less<>>
    safe_iterator_t<Rng>
      max_element(Rng&& rng, Comp comp = Comp{}, Proj proj = Proj{});

  template <ForwardIterator I, Sentinel<I> S, class Proj = identity,
      IndirectStrictWeakOrder<projected<I, Proj>> Comp = less<>>
    tagged_pair<tag::min(I), tag::max(I)>
      minmax_element(I first, S last, Comp comp = Comp{}, Proj proj = Proj{});

  template <ForwardRange Rng, class Proj = identity,
      IndirectStrictWeakOrder<projected<iterator_t<Rng>, Proj>> Comp = less<>>
    tagged_pair<tag::min(safe_iterator_t<Rng>),
                tag::max(safe_iterator_t<Rng>)>
      minmax_element(Rng&& rng, Comp comp = Comp{}, Proj proj = Proj{});

  template <InputIterator I1, Sentinel<I1> S1, InputIterator I2, Sentinel<I2> S2,
      class Proj1 = identity, class Proj2 = identity,
      IndirectStrictWeakOrder<projected<I1, Proj1>, projected<I2, Proj2>> Comp = less<>>
    bool
      lexicographical_compare(I1 first1, S1 last1, I2 first2, S2 last2,
                              Comp comp = Comp{}, Proj1 proj1 = Proj1{}, Proj2 proj2 = Proj2{});

  template <InputRange Rng1, InputRange Rng2, class Proj1 = identity,
      class Proj2 = identity,
      IndirectStrictWeakOrder<projected<iterator_t<Rng1>, Proj1>,
        projected<iterator_t<Rng2>, Proj2>> Comp = less<>>
    bool
      lexicographical_compare(Rng1&& rng1, Rng2&& rng2, Comp comp = Comp{},
                              Proj1 proj1 = Proj1{}, Proj2 proj2 = Proj2{});

  // \ref{std2.alg.permutation.generators}, permutations:
  template <BidirectionalIterator I, Sentinel<I> S, class Comp = less<>,
      class Proj = identity>
    requires Sortable<I, Comp, Proj>
    bool next_permutation(I first, S last, Comp comp = Comp{}, Proj proj = Proj{});

  template <BidirectionalRange Rng, class Comp = less<>,
      class Proj = identity>
    requires Sortable<iterator_t<Rng>, Comp, Proj>
    bool
      next_permutation(Rng&& rng, Comp comp = Comp{}, Proj proj = Proj{});

  template <BidirectionalIterator I, Sentinel<I> S, class Comp = less<>,
      class Proj = identity>
    requires Sortable<I, Comp, Proj>
    bool prev_permutation(I first, S last, Comp comp = Comp{}, Proj proj = Proj{});

  template <BidirectionalRange Rng, class Comp = less<>,
      class Proj = identity>
    requires Sortable<iterator_t<Rng>, Comp, Proj>
    bool
      prev_permutation(Rng&& rng, Comp comp = Comp{}, Proj proj = Proj{});
}}@\removed{\}\}}@
\end{codeblock}

\pnum
All of the algorithms are separated from the particular implementations of data structures and are
parameterized by iterator types.
Because of this, they can work with program-defined data structures, as long
as these data structures have iterator types satisfying the assumptions on the algorithms.

\pnum
For purposes of determining the existence of data races, algorithms shall
not modify objects referenced through an iterator argument unless the
specification requires such modification.

\pnum
Both in-place and copying versions are provided for certain
algorithms.\footnote{The decision whether to include a copying version was
usually based on complexity considerations. When the cost of doing the operation
dominates the cost of copy, the copying version is not included. For example,
\tcode{sort_copy} is not included because the cost of sorting is much more
significant, and users might as well do \tcode{copy} followed by \tcode{sort}.}
When such a version is provided for \textit{algorithm} it is called
\textit{algorithm\tcode{_copy}}. Algorithms that take predicates end with the
suffix \tcode{_if} (which follows the suffix \tcode{_copy}).

\pnum
\enternote
Unless otherwise specified, algorithms that take function objects as arguments
are permitted to copy those function objects freely. Programmers for whom object
identity is important should consider using a wrapper class that points to a
noncopied implementation object such as \tcode{reference_wrapper<T>}~(\cxxref{refwrap}), or some equivalent solution.
\exitnote

\pnum
In the description of the algorithms operators
\tcode{+}
and
\tcode{-}
are used for some of the iterator categories for which
they do not have to be defined.
In these cases the semantics of
\tcode{a+n}
is the same as that of

\begin{codeblock}
X tmp = a;
advance(tmp, n);
return tmp;
\end{codeblock}

and that of
\tcode{b-a}
is the same as of

\begin{codeblock}
return distance(a, b);
\end{codeblock}

\pnum
In the description of algorithm return values, sentinel values are sometimes
returned where an iterator is expected. In these cases, the semantics are as
if the sentinel is converted into an iterator as follows:

\begin{codeblock}
I tmp = first;
while(tmp != last)
  ++tmp;
return tmp;
\end{codeblock}

\pnum
Overloads of algorithms that take \tcode{Range} arguments~(\ref{std2.ranges.range})
behave as if they are implemented by calling \tcode{begin} and \tcode{end} on
the \tcode{Range} and dispatching to the overload that takes separate
iterator and sentinel arguments.

\pnum
The number and order of template parameters for algorithm declarations
is unspecified, except where explicitly stated otherwise.

\rSec1[std2.alg.tagspec]{Tag specifiers}

\begin{itemdecl}
namespace tag {
  struct in { /* @\impdef@ */ };
  struct in1 { /* @\impdef@ */ };
  struct in2 { /* @\impdef@ */ };
  struct out { /* @\impdef@ */ };
  struct out1 { /* @\impdef@ */ };
  struct out2 { /* @\impdef@ */ };
  struct fun { /* @\impdef@ */ };
  struct min { /* @\impdef@ */ };
  struct max { /* @\impdef@ */ };
  struct begin { /* @\impdef@ */ };
  struct end { /* @\impdef@ */ };
}
\end{itemdecl}

\begin{itemdescr}
\pnum In the following description, let \tcode{$X$} be the name of a type in the \tcode{tag}
namespace above.

\pnum \tcode{tag::$X$} is a tag specifier~(\ref{std2.taggedtup.tagged}) such that
\tcode{\textit{TAGGET}($D$, tag::$X$, $N$)} names a tagged getter~(\ref{std2.taggedtup.tagged})
with DerivedCharacteristic \tcode{$D$}, ElementIndex \tcode{$N$}, and ElementName \tcode{$X$}.

\pnum \enterexample \tcode{tag::in} is a type such that \tcode{\textit{TAGGET}($D$, tag::in, $N$)}
names a type with the following interface:

\begin{codeblock}
struct @\xname{input_getter}@ {
  constexpr decltype(auto) in() &       { return get<@$N$@>(static_cast<@$D$@&>(*this)); }
  constexpr decltype(auto) in() &&      { return get<@$N$@>(static_cast<@$D$@&&>(*this)); }
  constexpr decltype(auto) in() const & { return get<@$N$@>(static_cast<const @$D$@&>(*this)); }
};
\end{codeblock}
\exitexample
\end{itemdescr}

\rSec1[std2.alg.nonmodifying]{Non-modifying sequence operations}

\rSec2[std2.alg.all_of]{All of}

\indexlibrary{\idxcode{all_of}}%
\begin{itemdecl}
template <InputIterator I, Sentinel<I> S, class Proj = identity,
    IndirectUnaryPredicate<projected<I, Proj>> Pred>
  bool all_of(I first, S last, Pred pred, Proj proj = Proj{});

template <InputRange Rng, class Proj = identity,
    IndirectUnaryPredicate<projected<iterator_t<Rng>, Proj>> Pred>
  bool all_of(Rng&& rng, Pred pred, Proj proj = Proj{});
\end{itemdecl}

\begin{itemdescr}
\pnum
\returns \tcode{true} if
\range{first}{last} is empty or if
\tcode{invoke(pred, invoke(proj, *i))}
is \tcode{true} for every iterator \tcode{i} in the range \range{first}{last},
and \tcode{false} otherwise.

\pnum
\complexity At most \tcode{last - first} applications of the predicate
and \tcode{last - first} applications of the projection.
\end{itemdescr}

\rSec2[std2.alg.any_of]{Any of}

\indexlibrary{\idxcode{any_of}}%
\begin{itemdecl}
template <InputIterator I, Sentinel<I> S, class Proj = identity,
    IndirectUnaryPredicate<projected<I, Proj>> Pred>
  bool any_of(I first, S last, Pred pred, Proj proj = Proj{});

template <InputRange Rng, class Proj = identity,
    IndirectUnaryPredicate<projected<iterator_t<Rng>, Proj>> Pred>
  bool any_of(Rng&& rng, Pred pred, Proj proj = Proj{});
\end{itemdecl}

\begin{itemdescr}
\pnum
\returns \tcode{false} if \range{first}{last} is empty or
if there is no iterator \tcode{i} in the range
\range{first}{last} such that
\tcode{invoke(pred, invoke(proj, *i))}
is \tcode{true}, and \tcode{true} otherwise.

\pnum
\complexity At most \tcode{last - first} applications of the predicate
and \tcode{last - first} applications of the projection.
\end{itemdescr}

\rSec2[std2.alg.none_of]{None of}

\indexlibrary{\idxcode{none_of}}%
\begin{itemdecl}
template <InputIterator I, Sentinel<I> S, class Proj = identity,
    IndirectUnaryPredicate<projected<I, Proj>> Pred>
  bool none_of(I first, S last, Pred pred, Proj proj = Proj{});

template <InputRange Rng, class Proj = identity,
    IndirectUnaryPredicate<projected<iterator_t<Rng>, Proj>> Pred>
  bool none_of(Rng&& rng, Pred pred, Proj proj = Proj{});
\end{itemdecl}

\begin{itemdescr}
\pnum
\returns \tcode{true} if
\range{first}{last} is empty or if
\tcode{invoke(pred, invoke(proj, *i))}
is \tcode{false} for every iterator \tcode{i} in the range \range{first}{last},
and \tcode{false} otherwise.

\pnum
\complexity At most \tcode{last - first} applications of the predicate
and \tcode{last - first} applications of the projection.
\end{itemdescr}

\rSec2[std2.alg.foreach]{For each}

\indexlibrary{\idxcode{for_each}}%
\begin{itemdecl}
template <InputIterator I, Sentinel<I> S, class Proj = identity,
    IndirectUnaryInvocable<projected<I, Proj>> Fun>
  tagged_pair<tag::in(I), tag::fun(Fun)>
    for_each(I first, S last, Fun f, Proj proj = Proj{});

template <InputRange Rng, class Proj = identity,
    IndirectUnaryInvocable<projected<iterator_t<Rng>, Proj>> Fun>
  tagged_pair<tag::in(safe_iterator_t<Rng>), tag::fun(Fun)>
    for_each(Rng&& rng, Fun f, Proj proj = Proj{});
\end{itemdecl}

\begin{itemdescr}
\pnum
\effects
Calls
\tcode{invoke(f, invoke(proj, *i))} for every iterator
\tcode{i} in the range
\range{first}{last},
starting from
\tcode{first}
and proceeding to
\tcode{last - 1}.
\enternote If the result of
\tcode{invoke(proj, *i)} is a mutable reference, \tcode{f} may apply
nonconstant functions.\exitnote

\pnum
\returns
\tcode{\{last, std::move(f)\}}.

\pnum
\complexity
Applies \tcode{f} and \tcode{proj}
exactly
\tcode{last - first}
times.

\pnum
\notes
If \tcode{f} returns a result, the result is ignored.

\pnum
\enternote The requirements of this algorithm are more strict than those
specified in \cxxref{alg.foreach}. This algorithm requires \tcode{Fun} to
satisfy \tcode{CopyConstructible}, whereas the algorithm in the \Cpp Standard
requires only \tcode{MoveConstructible}. \exitnote
\end{itemdescr}

\rSec2[std2.alg.find]{Find}

\indexlibrary{\idxcode{find}}%
\indexlibrary{\idxcode{find_if}}%
\indexlibrary{\idxcode{find_if_not}}%
\begin{itemdecl}
template <InputIterator I, Sentinel<I> S, class T, class Proj = identity>
  requires IndirectRelation<equal_to<>, projected<I, Proj>, const T*>
  I find(I first, S last, const T& value, Proj proj = Proj{});

template <InputRange Rng, class T, class Proj = identity>
  requires IndirectRelation<equal_to<>, projected<iterator_t<Rng>, Proj>, const T*>
  safe_iterator_t<Rng>
    find(Rng&& rng, const T& value, Proj proj = Proj{});

template <InputIterator I, Sentinel<I> S, class Proj = identity,
    IndirectUnaryPredicate<projected<I, Proj>> Pred>
  I find_if(I first, S last, Pred pred, Proj proj = Proj{});

template <InputRange Rng, class Proj = identity,
    IndirectUnaryPredicate<projected<iterator_t<Rng>, Proj>> Pred>
  safe_iterator_t<Rng>
    find_if(Rng&& rng, Pred pred, Proj proj = Proj{});

template <InputIterator I, Sentinel<I> S, class Proj = identity,
    IndirectUnaryPredicate<projected<I, Proj>> Pred>
  I find_if_not(I first, S last, Pred pred, Proj proj = Proj{});

template <InputRange Rng, class Proj = identity,
    IndirectUnaryPredicate<projected<iterator_t<Rng>, Proj>> Pred>
  safe_iterator_t<Rng>
    find_if_not(Rng&& rng, Pred pred, Proj proj = Proj{});
\end{itemdecl}

\begin{itemdescr}
\pnum
\returns
The first iterator
\tcode{i}
in the range
\range{first}{last}
for which the following corresponding
conditions hold:
\tcode{invoke(proj, *i) == value},
\tcode{invoke(pred, invoke(proj, *i)) != false},
\tcode{invoke(\brk{}pred, invoke(proj, *i)) == false}.
Returns \tcode{last} if no such iterator is found.

\pnum
\complexity
At most
\tcode{last - first}
applications of the corresponding predicate and projection.
\end{itemdescr}

\rSec2[std2.alg.find.end]{Find end}

\indexlibrary{\idxcode{find_end}}%
\begin{itemdecl}
template <ForwardIterator I1, Sentinel<I1> S1, ForwardIterator I2,
    Sentinel<I2> S2, class Proj = identity,
    IndirectRelation<I2, projected<I1, Proj>> Pred = equal_to<>>
  I1
    find_end(I1 first1, S1 last1, I2 first2, S2 last2,
             Pred pred = Pred{}, Proj proj = Proj{});

template <ForwardRange Rng1, ForwardRange Rng2,
    class Proj = identity,
    IndirectRelation<iterator_t<Rng2>,
      projected<iterator_t<Rng>, Proj>> Pred = equal_to<>>
  safe_iterator_t<Rng1>
    find_end(Rng1&& rng1, Rng2&& rng2, Pred pred = Pred{}, Proj proj = Proj{});
\end{itemdecl}

\begin{itemdescr}
\pnum
\effects
Finds a subsequence of equal values in a sequence.

\pnum
\returns
The last iterator
\tcode{i}
in the range \range{first1}{last1 - (last2 - first2)}
such that for every non-negative integer
\tcode{n < (last2 - first2)},
the following condition holds:
\tcode{invoke(pred, invoke(\brk{}proj, *(i + n)), *(first2 + n)) != false}.
Returns \tcode{last1}
if
\range{first2}{last2} is empty or if
no such iterator is found.

\pnum
\complexity
At most
\tcode{(last2 - first2) * (last1 - first1 - (last2 - first2) + 1)}
applications of the corresponding predicate and projection.
\end{itemdescr}

\rSec2[std2.alg.find.first.of]{Find first of}

\indexlibrary{\idxcode{find_first_of}}%
\begin{itemdecl}
template <InputIterator I1, Sentinel<I1> S1, ForwardIterator I2, Sentinel<I2> S2,
    class Proj1 = identity, class Proj2 = identity,
    IndirectRelation<projected<I1, Proj1>, projected<I2, Proj2>> Pred = equal_to<>>
  I1
    find_first_of(I1 first1, S1 last1, I2 first2, S2 last2, Pred pred = Pred{},
                  Proj1 proj1 = Proj1{}, Proj2 proj2 = Proj2{});

template <InputRange Rng1, ForwardRange Rng2, class Proj1 = identity,
    class Proj2 = identity,
    IndirectRelation<projected<iterator_t<Rng1>, Proj1>,
      projected<iterator_t<Rng2>, Proj2>> Pred = equal_to<>>
  safe_iterator_t<Rng1>
    find_first_of(Rng1&& rng1, Rng2&& rng2, Pred pred = Pred{},
                  Proj1 proj1 = Proj1{}, Proj2 proj2 = Proj2{});
\end{itemdecl}

\begin{itemdescr}
\pnum
\effects
Finds an element that matches one of a set of values.

\pnum
\returns
The first iterator
\tcode{i}
in the range \range{first1}{last1}
such that for some
iterator
\tcode{j}
in the range \range{first2}{last2}
the following condition holds:
\tcode{invoke(pred, invoke(proj1, *i), invoke(proj2, *j)) != false}.
Returns \tcode{last1}
if \range{first2}{last2} is empty or if
no such iterator is found.

\pnum
\complexity
At most
\tcode{(last1-first1) * (last2-first2)}
applications of the corresponding predicate and the two projections.
\end{itemdescr}

\rSec2[std2.alg.adjacent.find]{Adjacent find}

\indexlibrary{\idxcode{adjacent_find}}%
\begin{itemdecl}
template <ForwardIterator I, Sentinel<I> S, class Proj = identity,
    IndirectRelation<projected<I, Proj>> Pred = equal_to<>>
  I
    adjacent_find(I first, S last, Pred pred = Pred{},
                  Proj proj = Proj{});

template <ForwardRange Rng, class Proj = identity,
    IndirectRelation<projected<iterator_t<Rng>, Proj>> Pred = equal_to<>>
  safe_iterator_t<Rng>
    adjacent_find(Rng&& rng, Pred pred = Pred{}, Proj proj = Proj{});
\end{itemdecl}

\begin{itemdescr}
\pnum
\returns
The first iterator
\tcode{i}
such that both
\tcode{i}
and
\tcode{i + 1}
are in
the range
\range{first}{last}
for which
the following corresponding condition holds:
\tcode{invoke(pred, invoke(proj, *i), invoke(proj, *(i + 1))) != false}.
Returns \tcode{last}
if no such iterator is found.

\pnum
\complexity
For a nonempty range, exactly
\tcode{min((i - first) + 1, (last - first) - 1)}
applications of the corresponding predicate, where \tcode{i} is
\tcode{adjacent_find}'s
return value, and no more than twice as many applications of the projection.
\end{itemdescr}

\rSec2[std2.alg.count]{Count}

\indexlibrary{\idxcode{count}}%
\indexlibrary{\idxcode{count_if}}%
\begin{itemdecl}
template <InputIterator I, Sentinel<I> S, class T, class Proj = identity>
  requires IndirectRelation<equal_to<>, projected<I, Proj>, const T*>
  difference_type_t<I>
    count(I first, S last, const T& value, Proj proj = Proj{});

template <InputRange Rng, class T, class Proj = identity>
  requires IndirectRelation<equal_to<>, projected<iterator_t<Rng>, Proj>, const T*>
  difference_type_t<iterator_t<Rng>>
    count(Rng&& rng, const T& value, Proj proj = Proj{});

template <InputIterator I, Sentinel<I> S, class Proj = identity,
    IndirectUnaryPredicate<projected<I, Proj>> Pred>
  difference_type_t<I>
    count_if(I first, S last, Pred pred, Proj proj = Proj{});

template <InputRange Rng, class Proj = identity,
    IndirectUnaryPredicate<projected<iterator_t<Rng>, Proj>> Pred>
  difference_type_t<iterator_t<Rng>>
    count_if(Rng&& rng, Pred pred, Proj proj = Proj{});
\end{itemdecl}

\begin{itemdescr}
\pnum
\effects
Returns the number of iterators
\tcode{i}
in the range \range{first}{last}
for which the following corresponding
conditions hold:
\tcode{invoke(proj, *i) == value, invoke(pred, invoke(proj, *i)) != false}.

\pnum
\complexity
Exactly
\tcode{last - first}
applications of the corresponding predicate and projection.
\end{itemdescr}

\rSec2[std2.mismatch]{Mismatch}

\indexlibrary{\idxcode{mismatch}}%
\begin{itemdecl}
template <InputIterator I1, Sentinel<I1> S1, InputIterator I2, Sentinel<I2> S2,
    class Proj1 = identity, class Proj2 = identity,
    IndirectRelation<projected<I1, Proj1>, projected<I2, Proj2>> Pred = equal_to<>>
  tagged_pair<tag::in1(I1), tag::in2(I2)>
    mismatch(I1 first1, S1 last1, I2 first2, S2 last2, Pred pred = Pred{},
             Proj1 proj1 = Proj1{}, Proj2 proj2 = Proj2{});

template <InputRange Rng1, InputRange Rng2,
    class Proj1 = identity, class Proj2 = identity,
    IndirectRelation<projected<iterator_t<Rng1>, Proj1>,
      projected<iterator_t<Rng2>, Proj2>> Pred = equal_to<>>
  tagged_pair<tag::in1(safe_iterator_t<Rng1>), tag::in2(safe_iterator_t<Rng2>)>
    mismatch(Rng1&& rng1, Rng2&& rng2, Pred pred = Pred{},
             Proj1 proj1 = Proj1{}, Proj2 proj2 = Proj2{});
\end{itemdecl}

\begin{itemdescr}
\pnum
\returns
A pair of iterators
\tcode{i}
and
\tcode{j}
such that
\tcode{j == first2 + (i - first1)}
and
\tcode{i}
is the first iterator
in the range \range{first1}{last1}
for which the following corresponding conditions hold:

\begin{itemize}
\item \tcode{j} is in the range \tcode{[first2, last2)}.
\item \tcode{*i != *(first2 + (i - first1))}
\item \tcode{!invoke(pred, invoke(proj1, *i), invoke(proj2, *(first2 + (i - first1))))}
\end{itemize}

Returns the pair
\tcode{first1 + min(last1 - first1, last2 - first2)}
and
\tcode{first2 + min(last1 - first1, last2 - first2)}
if such an iterator
\tcode{i}
is not found.

\pnum
\complexity
At most
\tcode{last1 - first1}
applications of the corresponding predicate and both projections.
\end{itemdescr}

\rSec2[std2.alg.equal]{Equal}

\indexlibrary{\idxcode{equal}}%
\begin{itemdecl}
template <InputIterator I1, Sentinel<I1> S1, InputIterator I2, Sentinel<I2> S2,
    class Pred = equal_to<>, class Proj1 = identity, class Proj2 = identity>
  requires IndirectlyComparable<I1, I2, Pred, Proj1, Proj2>
  bool equal(I1 first1, S1 last1, I2 first2, S2 last2,
             Pred pred = Pred{},
             Proj1 proj1 = Proj1{}, Proj2 proj2 = Proj2{});

template <InputRange Rng1, InputRange Rng2, class Pred = equal_to<>,
    class Proj1 = identity, class Proj2 = identity>
  requires IndirectlyComparable<iterator_t<Rng1>, iterator_t<Rng2>, Pred, Proj1, Proj2>
  bool equal(Rng1&& rng1, Rng2&& rng2, Pred pred = Pred{},
             Proj1 proj1 = Proj1{}, Proj2 proj2 = Proj2{});
\end{itemdecl}

\begin{itemdescr}
\pnum
\returns
If
\tcode{last1 - first1 != last2 - first2},
return
\tcode{false}.
Otherwise return
\tcode{true}
if for every iterator
\tcode{i}
in the range \range{first1}{last1}
the following condition holds:
\tcode{invoke(pred, invoke(\brk{}proj1, *i), invoke(proj2, *(first2 + (i - first1))))}.
Otherwise, returns
\tcode{fal\-se}.

\pnum
\complexity
No applications of the corresponding predicate and projections if:
\begin{itemize}
\item \tcode{SizedSentinel<S1, I1>} is satisfied, and
\item \tcode{SizedSentinel<S2, I2>} is satisfied, and
\item \tcode{last1 - first1 != last2 - first2}.
\end{itemize}
Otherwise, at most
\tcode{min(last1 - first1, last2 - \brk{}first2)}
applications of the corresponding predicate and projections.
\end{itemdescr}

\rSec2[std2.alg.is_permutation]{Is permutation}

\indexlibrary{\idxcode{is_permutation}}%
\begin{itemdecl}
template <ForwardIterator I1, Sentinel<I1> S1, ForwardIterator I2,
    Sentinel<I2> S2, class Pred = equal_to<>, class Proj1 = identity,
    class Proj2 = identity>
  requires IndirectlyComparable<I1, I2, Pred, Proj1, Proj2>
  bool is_permutation(I1 first1, S1 last1, I2 first2, S2 last2,
                      Pred pred = Pred{},
                      Proj1 proj1 = Proj1{}, Proj2 proj2 = Proj2{});

template <ForwardRange Rng1, ForwardRange Rng2, class Pred = equal_to<>,
    class Proj1 = identity, class Proj2 = identity>
  requires IndirectlyComparable<iterator_t<Rng1>, iterator_t<Rng2>, Pred, Proj1, Proj2>
  bool is_permutation(Rng1&& rng1, Rng2&& rng2, Pred pred = Pred{},
                      Proj1 proj1 = Proj1{}, Proj2 proj2 = Proj2{});
\end{itemdecl}

\begin{itemdescr}
\pnum
\returns If \tcode{last1 - first1 != last2 - first2}, return \tcode{false}.
Otherwise return \tcode{true} if there exists a permutation of the elements in the
range \range{first2}{first2 + (last1 - first1)}, beginning with
\tcode{I2 begin}, such that
\tcode{equal(first1, last1, begin, pred, proj1, proj2)} returns \tcode{true}
; otherwise, returns \tcode{false}.

\pnum
\complexity
No applications of the corresponding predicate and projections if:
\begin{itemize}
\item \tcode{SizedSentinel<S1, I1>} is satisfied, and
\item \tcode{SizedSentinel<S2, I2>} is satisfied, and
\item \tcode{last1 - first1 != last2 - first2}.
\end{itemize}
Otherwise, exactly \tcode{last1 - first1} applications of the
corresponding predicate and projections if
\tcode{equal(\brk{}first1, last1, first2, last2, pred, proj1, proj2)}
would return \tcode{true}; otherwise, at
worst \bigoh{N^2}, where $N$ has the value \tcode{last1 - first1}.
\end{itemdescr}

\rSec2[std2.alg.search]{Search}

\indexlibrary{\idxcode{search}}%
\begin{itemdecl}
template <ForwardIterator I1, Sentinel<I1> S1, ForwardIterator I2,
    Sentinel<I2> S2, class Pred = equal_to<>,
    class Proj1 = identity, class Proj2 = identity>
  requires IndirectlyComparable<I1, I2, Pred, Proj1, Proj2>
  I1
    search(I1 first1, S1 last1, I2 first2, S2 last2,
           Pred pred = Pred{},
           Proj1 proj1 = Proj1{}, Proj2 proj2 = Proj2{});

template <ForwardRange Rng1, ForwardRange Rng2, class Pred = equal_to<>,
    class Proj1 = identity, class Proj2 = identity>
  requires IndirectlyComparable<iterator_t<Rng1>, iterator_t<Rng2>, Pred, Proj1, Proj2>
  safe_iterator_t<Rng1>
    search(Rng1&& rng1, Rng2&& rng2, Pred pred = Pred{},
           Proj1 proj1 = Proj1{}, Proj2 proj2 = Proj2{});
\end{itemdecl}

\begin{itemdescr}
\pnum
\effects
Finds a subsequence of equal values in a sequence.

\pnum
\returns
The first iterator
\tcode{i}
in the range \range{first1}{last1 - (last2-first2)}
such that for every non-negative integer
\tcode{n}
less than
\tcode{last2 - first2}
the following condition holds:
\begin{codeblock}
invoke(pred, invoke(proj1, *(i + n)), invoke(proj2, *(first2 + n))) != false.
\end{codeblock}
Returns \tcode{first1}
if \brk{}\range{first2}{last2} is empty,
otherwise returns \tcode{last1}
if no such iterator is found.

\pnum
\complexity
At most
\tcode{(last1 - first1) * (last2 - first2)}
applications of the corresponding predicate and projections.
\end{itemdescr}

\indexlibrary{\idxcode{search_n}}%
\begin{itemdecl}
template <ForwardIterator I, Sentinel<I> S, class T,
    class Pred = equal_to<>, class Proj = identity>
  requires IndirectlyComparable<I, const T*, Pred, Proj>
  I
    search_n(I first, S last, difference_type_t<I> count,
             const T& value, Pred pred = Pred{},
             Proj proj = Proj{});

template <ForwardRange Rng, class T, class Pred = equal_to<>,
    class Proj = identity>
  requires IndirectlyComparable<iterator_t<Rng>, const T*, Pred, Proj>
  safe_iterator_t<Rng>
    search_n(Rng&& rng, difference_type_t<iterator_t<Rng>> count,
             const T& value, Pred pred = Pred{}, Proj proj = Proj{});
\end{itemdecl}

\begin{itemdescr}
\pnum
\effects
Finds a subsequence of equal values in a sequence.

\pnum
\returns
The first iterator
\tcode{i}
in the range \range{first}{last-count}
such that for every non-negative integer
\tcode{n}
less than
\tcode{count}
the following condition holds:
\tcode{invoke(pred, invoke(proj, *(i + n)), value) != false}.
Returns \tcode{last}
if no such iterator is found.

\pnum
\complexity
At most
\tcode{last - first}
applications of the corresponding predicate and projection.
\end{itemdescr}

\rSec1[std2.alg.modifying.operations]{Mutating sequence operations}

\rSec2[std2.alg.copy]{Copy}

\indexlibrary{\idxcode{copy}}%
\begin{itemdecl}
template <InputIterator I, Sentinel<I> S, WeaklyIncrementable O>
  requires IndirectlyCopyable<I, O>
  tagged_pair<tag::in(I), tag::out(O)>
    copy(I first, S last, O result);

template <InputRange Rng, WeaklyIncrementable O>
  requires IndirectlyCopyable<iterator_t<Rng>, O>
  tagged_pair<tag::in(safe_iterator_t<Rng>), tag::out(O)>
    copy(Rng&& rng, O result);
\end{itemdecl}

\begin{itemdescr}
\pnum
\effects Copies elements in the range \range{first}{last} into the range
\range{result}{result + (last - first)} starting from \tcode{first} and
proceeding to \tcode{last}. For each non-negative integer
\tcode{n < (last - first)}, performs \tcode{*(result + n) = *(first + n)}.

\pnum
\returns \tcode{\{last, result + (last - first)\}}.

\pnum
\requires \tcode{result} shall not be in the range \range{first}{last}.

\pnum
\complexity Exactly \tcode{last - first} assignments.
\end{itemdescr}

\indexlibrary{\idxcode{copy_n}}%
\begin{itemdecl}
template <InputIterator I, WeaklyIncrementable O>
  requires IndirectlyCopyable<I, O>
  tagged_pair<tag::in(I), tag::out(O)>
    copy_n(I first, difference_type_t<I> n, O result);
\end{itemdecl}

\begin{itemdescr}
\pnum
\effects For each non-negative integer
$i < n$, performs \tcode{*(result + i) = *(first + i)}.

\pnum
\returns \tcode{\{first + n, result + n\}}.

\pnum
\complexity Exactly \tcode{n} assignments.
\end{itemdescr}

\indexlibrary{\idxcode{copy_n}}%
\begin{itemdecl}
template <InputIterator I, Sentinel<I> S, WeaklyIncrementable O, class Proj = identity,
    IndirectUnaryPredicate<projected<I, Proj>> Pred>
  requires IndirectlyCopyable<I, O>
  tagged_pair<tag::in(I), tag::out(O)>
    copy_if(I first, S last, O result, Pred pred, Proj proj = Proj{});

template <InputRange Rng, WeaklyIncrementable O, class Proj = identity,
    IndirectUnaryPredicate<projected<iterator_t<Rng>, Proj>> Pred>
  requires IndirectlyCopyable<iterator_t<Rng>, O>
  tagged_pair<tag::in(safe_iterator_t<Rng>), tag::out(O)>
    copy_if(Rng&& rng, O result, Pred pred, Proj proj = Proj{});
\end{itemdecl}

\begin{itemdescr}
\pnum
Let $N$ be the number of iterators \tcode{i} in the range \range{first}{last}
for which the condition \tcode{invoke(pred, invoke(proj, *i))} holds.

\pnum
\requires The ranges \range{first}{last} and \range{result}{result + $N$} shall not overlap.

\pnum
\effects Copies all of the elements referred to by the iterator \tcode{i} in the range \range{first}{last}
for which \tcode{invoke(pred, invoke(proj, *i))} is \tcode{true}.

\pnum
\returns \tcode{\{last, result + $N$\}}.

\pnum
\complexity Exactly \tcode{last - first} applications of the corresponding predicate and projection.

\pnum
\remarks Stable~(\cxxref{algorithm.stable}).
\end{itemdescr}


\indexlibrary{\idxcode{copy_backward}}%
\begin{itemdecl}
template <BidirectionalIterator I1, Sentinel<I1> S1, BidirectionalIterator I2>
  requires IndirectlyCopyable<I1, I2>
  tagged_pair<tag::in(I1), tag::out(I2)>
    copy_backward(I1 first, S1 last, I2 result);

template <BidirectionalRange Rng, BidirectionalIterator I>
  requires IndirectlyCopyable<iterator_t<Rng>, I>
  tagged_pair<tag::in(safe_iterator_t<Rng>), tag::out(I)>
    copy_backward(Rng&& rng, I result);
\end{itemdecl}

\begin{itemdescr}
\pnum
\effects
Copies elements in the range \range{first}{last}
into the
range \range{result - (last-first)}{result}
starting from
\tcode{last - 1}
and proceeding to \tcode{first}.\footnote{\tcode{copy_backward}
should be used instead of copy when \tcode{last}
is in
the range
\range{result - (last - first)}{result}.}
For each positive integer
\tcode{n <= (last - first)},
performs
\tcode{*(result - n) = *(last - n)}.

\pnum
\requires
\tcode{result}
shall not be in the range
\brange{first}{last}.

\pnum
\returns
\tcode{\{last, result - (last - first)\}}.

\pnum
\complexity
Exactly
\tcode{last - first}
assignments.
\end{itemdescr}

\rSec2[std2.alg.move]{Move}

\indexlibrary{move\tcode{move}}%
\begin{itemdecl}
template <InputIterator I, Sentinel<I> S, WeaklyIncrementable O>
  requires IndirectlyMovable<I, O>
  tagged_pair<tag::in(I), tag::out(O)>
    move(I first, S last, O result);

template <InputRange Rng, WeaklyIncrementable O>
  requires IndirectlyMovable<iterator_t<Rng>, O>
  tagged_pair<tag::in(safe_iterator_t<Rng>), tag::out(O)>
    move(Rng&& rng, O result);
\end{itemdecl}

\begin{itemdescr}
\pnum
\effects
Moves elements in the range \range{first}{last}
into the range \range{result}{result + (last - first)}
starting from first and proceeding to last.
For each non-negative integer
\tcode{n < (last-first)},
performs
\tcode{*(result + n) = \changed{ranges}{::std2}::iter_move(first + n)}.

\pnum
\returns
\tcode{\{last, result + (last - first)\}}.

\pnum
\requires
\tcode{result}
shall not be in the range
\range{first}{last}.

\pnum
\complexity
Exactly
\tcode{last - first}
move assignments.
\end{itemdescr}

\indexlibrary{\idxcode{move_backward}}%
\begin{itemdecl}
template <BidirectionalIterator I1, Sentinel<I1> S1, BidirectionalIterator I2>
  requires IndirectlyMovable<I1, I2>
  tagged_pair<tag::in(I1), tag::out(I2)>
    move_backward(I1 first, S1 last, I2 result);

template <BidirectionalRange Rng, BidirectionalIterator I>
  requires IndirectlyMovable<iterator_t<Rng>, I>
  tagged_pair<tag::in(safe_iterator_t<Rng>), tag::out(I)>
    move_backward(Rng&& rng, I result);
\end{itemdecl}

\begin{itemdescr}
\pnum
\effects
Moves elements in the range \range{first}{last}
into the
range \range{result - (last-first)}{result}
starting from
\tcode{last - 1}
and proceeding to first.\footnote{\tcode{move_backward}
should be used instead of move when last
is in
the range
\range{result - (last - first)}{result}.}
For each positive integer
\tcode{n <= (last - first)},
performs
\tcode{*(result - n) = \changed{ranges}{::std2}::iter_move(last - n)}.

\pnum
\requires
\tcode{result}
shall not be in the range
\brange{first}{last}.

\pnum
\returns
\tcode{\{last, result - (last - first)\}}.

\pnum
\complexity
Exactly
\tcode{last - first}
assignments.
\end{itemdescr}

\rSec2[std2.alg.swap]{swap}

\indexlibrary{\idxcode{swap_ranges}}%
\begin{itemdecl}
template <ForwardIterator I1, Sentinel<I1> S1, ForwardIterator I2, Sentinel<I2> S2>
  requires IndirectlySwappable<I1, I2>
  tagged_pair<tag::in1(I1), tag::in2(I2)>
    swap_ranges(I1 first1, S1 last1, I2 first2, S2 last2);

template <ForwardRange Rng1, ForwardRange Rng2>
  requires IndirectlySwappable<iterator_t<Rng1>, iterator_t<Rng2>>
  tagged_pair<tag::in1(safe_iterator_t<Rng1>), tag::in2(safe_iterator_t<Rng2>)>
    swap_ranges(Rng1&& rng1, Rng2&& rng2);
\end{itemdecl}

\begin{itemdescr}
\pnum
\effects
For each non-negative integer \tcode{n < min(last1 - first1, last2 - first2)}
performs: \\
\tcode{\changed{ranges}{::std2}::iter_swap(first1 + n, first2 + n)}.

\pnum
\requires
The two ranges \range{first1}{last1}
and
\range{first2}{last2}
shall not overlap.
\tcode{*(first1 + n)} shall be swappable with~(\ref{concepts.lib.corelang.swappable})
\tcode{*(first2 + n)}.

\pnum
\returns
\tcode{\{first1 + n, first2 + n\}}, where
\tcode{n} is \tcode{min(last1 - first1, last2 - first2)}.

\pnum
\complexity
Exactly
\tcode{min(last1 - first1, last2 - first2)}
swaps.
\end{itemdescr}

\rSec2[std2.alg.transform]{Transform}

\indexlibrary{\idxcode{transform}}%
\begin{itemdecl}
template <InputIterator I, Sentinel<I> S, WeaklyIncrementable O,
    CopyConstructible F, class Proj = identity>
  requires Writable<O, indirect_result_of_t<F&(projected<I, Proj>)>>
  tagged_pair<tag::in(I), tag::out(O)>
    transform(I first, S last, O result, F op, Proj proj = Proj{});

template <InputRange Rng, WeaklyIncrementable O, CopyConstructible F,
    class Proj = identity>
  requires Writable<O, indirect_result_of_t<F&(
    projected<iterator_t<R>, Proj>)>>
  tagged_pair<tag::in(safe_iterator_t<Rng>), tag::out(O)>
    transform(Rng&& rng, O result, F op, Proj proj = Proj{});

template <InputIterator I1, Sentinel<I1> S1, InputIterator I2, Sentinel<I2> S2,
    WeaklyIncrementable O, CopyConstructible F, class Proj1 = identity,
    class Proj2 = identity>
  requires Writable<O, indirect_result_of_t<F&(projected<I1, Proj1>,
    projected<I2, Proj2>)>>
  tagged_tuple<tag::in1(I1), tag::in2(I2), tag::out(O)>
    transform(I1 first1, S1 last1, I2 first2, S2 last2, O result,
            F binary_op, Proj1 proj1 = Proj1{}, Proj2 proj2 = Proj2{});

template <InputRange Rng1, InputRange Rng2, WeaklyIncrementable O,
    CopyConstructible F, class Proj1 = identity, class Proj2 = identity>
  requires Writable<O, indirect_result_of_t<F&(
    projected<iterator_t<Rng1>, Proj1>, projected<iterator_t<Rng2>, Proj2>)>>
  tagged_tuple<tag::in1(safe_iterator_t<Rng1>),
               tag::in2(safe_iterator_t<Rng2>),
               tag::out(O)>
    transform(Rng1&& rng1, Rng2&& rng2, O result,
              F binary_op, Proj1 proj1 = Proj1{}, Proj2 proj2 = Proj2{});
\end{itemdecl}

\begin{itemdescr}
\pnum
Let $N$ be \tcode{(last1 - first1)}
for unary transforms, or \tcode{min(last1 - first1, last2 - first2) for binary
transforms.}

\pnum
\effects
Assigns through every iterator
\tcode{i}
in the range
\range{result}{result + $N$}
a new
corresponding value equal to
\tcode{invoke(op, invoke(proj, *(first1 + (i - result))))}
or
\tcode{invoke(\brk{}binary_op, invoke(proj1, *(first1 + (i - result))), invoke(proj2, *(first2 + (i - result))))}.

\pnum
\requires
\tcode{op} and \tcode{binary_op}
shall not invalidate iterators or subranges, or modify elements in the ranges
\crange{first1}{first1 + $N$},
\crange{first2}{first2 + $N$},
and
\crange{result}{result + $N$}.\footnote{The use of fully
closed ranges is intentional.}

\pnum
\returns
\tcode{\{first1 + $N$, result + $N$\}}
 or \tcode{make_tagged_tuple<tag::in1, tag::in2, tag\colcol{}out\brk{}>(\brk{}first1 + $N$, first2 + $N$, result + $N$)}.

\pnum
\complexity
Exactly
\tcode{$N$}
applications of
\tcode{op} or \tcode{binary_op} and the corresponding projection(s).

\pnum
\notes
\tcode{result} may be equal to \tcode{first1}
in case of unary transform,
or to \tcode{first1} or \tcode{first2}
in case of binary transform.
\end{itemdescr}

\rSec2[std2.alg.replace]{Replace}

\indexlibrary{\idxcode{replace}}%
\indexlibrary{\idxcode{replace_if}}%
\begin{itemdecl}
template <InputIterator I, Sentinel<I> S, class T1, class T2, class Proj = identity>
  requires Writable<I, const T2&> &&
    IndirectRelation<equal_to<>, projected<I, Proj>, const T1*>
  I
    replace(I first, S last, const T1& old_value, const T2& new_value, Proj proj = Proj{});

template <InputRange Rng, class T1, class T2, class Proj = identity>
  requires Writable<iterator_t<Rng>, const T2&> &&
    IndirectRelation<equal_to<>, projected<iterator_t<Rng>, Proj>, const T1*>
  safe_iterator_t<Rng>
    replace(Rng&& rng, const T1& old_value, const T2& new_value, Proj proj = Proj{});

template <InputIterator I, Sentinel<I> S, class T, class Proj = identity,
    IndirectUnaryPredicate<projected<I, Proj>> Pred>
  requires Writable<I, const T&>
  I
    replace_if(I first, S last, Pred pred, const T& new_value, Proj proj = Proj{});

template <InputRange Rng, class T, class Proj = identity,
    IndirectUnaryPredicate<projected<iterator_t<Rng>, Proj>> Pred>
  requires Writable<iterator_t<Rng>, const T&>
  safe_iterator_t<Rng>
    replace_if(Rng&& rng, Pred pred, const T& new_value, Proj proj = Proj{});
\end{itemdecl}

\begin{itemdescr}
\pnum
\effects
Assigns \tcode{new_value} through each iterator
\tcode{i}
in the range \range{first}{last}
when the following corresponding conditions hold:
\tcode{invoke(proj, *i) == old_value},
\tcode{invoke(pred, invoke(proj, *i)) != false}.

\pnum
\returns
\tcode{last}.

\pnum
\complexity
Exactly
\tcode{last - first}
applications of the corresponding predicate and projection.
\end{itemdescr}

\indexlibrary{\idxcode{replace_copy}}%
\indexlibrary{\idxcode{replace_copy_if}}%
\begin{itemdecl}
template <InputIterator I, Sentinel<I> S, class T1, class T2, OutputIterator<const T2&> O,
    class Proj = identity>
  requires IndirectlyCopyable<I, O> &&
    IndirectRelation<equal_to<>, projected<I, Proj>, const T1*>
  tagged_pair<tag::in(I), tag::out(O)>
    replace_copy(I first, S last, O result, const T1& old_value, const T2& new_value,
                 Proj proj = Proj{});

template <InputRange Rng, class T1, class T2, OutputIterator<const T2&> O,
    class Proj = identity>
  requires IndirectlyCopyable<iterator_t<Rng>, O> &&
    IndirectRelation<equal_to<>, projected<iterator_t<Rng>, Proj>, const T1*>
  tagged_pair<tag::in(safe_iterator_t<Rng>), tag::out(O)>
    replace_copy(Rng&& rng, O result, const T1& old_value, const T2& new_value,
                 Proj proj = Proj{});

template <InputIterator I, Sentinel<I> S, class T, OutputIterator<const T&> O,
    class Proj = identity, IndirectUnaryPredicate<projected<I, Proj>> Pred>
  requires IndirectlyCopyable<I, O>
  tagged_pair<tag::in(I), tag::out(O)>
    replace_copy_if(I first, S last, O result, Pred pred, const T& new_value,
                    Proj proj = Proj{});

template <InputRange Rng, class T, OutputIterator<const T&> O, class Proj = identity,
    IndirectUnaryPredicate<projected<iterator_t<Rng>, Proj>> Pred>
  requires IndirectlyCopyable<iterator_t<Rng>, O>
  tagged_pair<tag::in(safe_iterator_t<Rng>), tag::out(O)>
    replace_copy_if(Rng&& rng, O result, Pred pred, const T& new_value,
                    Proj proj = Proj{});
\end{itemdecl}

\begin{itemdescr}
\pnum
\requires
The ranges
\range{first}{last}
and
\range{result}{result + (last - first)}
shall not overlap.

\pnum
\effects
Assigns to every iterator
\tcode{i}
in the
range
\range{result}{result + (last - first)}
either
\tcode{new_value}
or
\tcode{*\brk(first + (i - result))}
depending on whether the following corresponding conditions hold:

\begin{codeblock}
invoke(proj, *(first + (i - result))) == old_value
invoke(pred, invoke(proj, *(first + (i - result)))) != false
\end{codeblock}

\pnum
\returns
\tcode{\{last, result + (last - first)\}}.

\pnum
\complexity
Exactly
\tcode{last - first}
applications of the corresponding predicate and projection.
\end{itemdescr}

\rSec2[std2.alg.fill]{Fill}

\indexlibrary{\idxcode{fill}}%
\indexlibrary{\idxcode{fill_n}}%
\begin{itemdecl}
template <class T, OutputIterator<const T&> O, Sentinel<O> S>
  O fill(O first, S last, const T& value);

template <class T, OutputRange<const T&> Rng>
  safe_iterator_t<Rng>
    fill(Rng&& rng, const T& value);

template <class T, OutputIterator<const T&> O>
  O fill_n(O first, difference_type_t<O> n, const T& value);
\end{itemdecl}

\begin{itemdescr}
\pnum
\effects
\tcode{fill} assigns \tcode{value} through all the
iterators in the range \range{first}{last}. \tcode{fill_n}
assigns \tcode{value} through all the iterators in the counted range \range{first}{n}
if \tcode{n} is positive, otherwise it does nothing.

\pnum
\returns
\tcode{last}, where \tcode{last} is \tcode{first + max(n, 0)} for \tcode{fill_n}.

\pnum
\complexity
Exactly \tcode{last - first} assignments.
\end{itemdescr}

\rSec2[std2.alg.generate]{Generate}

\indexlibrary{\idxcode{generate}}%
\indexlibrary{\idxcode{generate_n}}%
\begin{itemdecl}
template <Iterator O, Sentinel<O> S, CopyConstructible F>
    requires Invocable<F&> && Writable<O, result_of_t<F&()>>
  O generate(O first, S last, F gen);

template <class Rng, CopyConstructible F>
    requires Invocable<F&> && OutputRange<Rng, result_of_t<F&()>>
  safe_iterator_t<Rng>
    generate(Rng&& rng, F gen);

template <Iterator O, CopyConstructible F>
    requires Invocable<F&> && Writable<O, result_of_t<F&()>>
  O generate_n(O first, difference_type_t<O> n, F gen);
\end{itemdecl}

\begin{itemdescr}
\pnum
\effects
The generate algorithms invoke the function object \tcode{gen} and assign the
return value of \tcode{gen} through all the iterators in the range
\range{first}{last}. The \tcode{generate_n} algorithm invokes the function object
\tcode{gen} and assigns the return value of \tcode{gen} through all the iterators
in the counted range \range{first}{n} if \tcode{n} is positive, otherwise it does
nothing.

\pnum
\returns
\tcode{last}, where \tcode{last} is \tcode{first + max(n, 0)} for \tcode{generate_n}.

\pnum
\complexity
Exactly \tcode{last - first} evaluations of \tcode{invoke(gen)} and assignments.
\end{itemdescr}

\rSec2[std2.alg.remove]{Remove}

\indexlibrary{\idxcode{remove}}%
\indexlibrary{\idxcode{remove_if}}%
\begin{itemdecl}
template <ForwardIterator I, Sentinel<I> S, class T, class Proj = identity>
  requires Permutable<I> &&
    IndirectRelation<equal_to<>, projected<I, Proj>, const T*>
  I remove(I first, S last, const T& value, Proj proj = Proj{});

template <ForwardRange Rng, class T, class Proj = identity>
  requires Permutable<iterator_t<Rng>> &&
    IndirectRelation<equal_to<>, projected<iterator_t<Rng>, Proj>, const T*>
  safe_iterator_t<Rng>
    remove(Rng&& rng, const T& value, Proj proj = Proj{});

template <ForwardIterator I, Sentinel<I> S, class Proj = identity,
    IndirectUnaryPredicate<projected<I, Proj>> Pred>
  requires Permutable<I>
  I remove_if(I first, S last, Pred pred, Proj proj = Proj{});

template <ForwardRange Rng, class Proj = identity,
    IndirectUnaryPredicate<projected<iterator_t<Rng>, Proj>> Pred>
  requires Permutable<iterator_t<Rng>>
  safe_iterator_t<Rng>
    remove_if(Rng&& rng, Pred pred, Proj proj = Proj{});
\end{itemdecl}

\begin{itemdescr}
\pnum
\effects
Eliminates all the elements referred to by iterator
\tcode{i}
in the range \range{first}{last}
for which the following corresponding conditions hold:
\tcode{invoke(proj, *i) == value},
\tcode{invoke(pred, invoke(proj, *i)) != false}.

\pnum
\returns
The end of the resulting range.

\pnum
\remarks Stable~(\cxxref{algorithm.stable}).

\pnum
\complexity
Exactly
\tcode{last - first}
applications of the corresponding predicate and projection.

\pnum
\realnote each element in the range \range{ret}{last}, where \tcode{ret} is
the returned value, has a valid but unspecified state, because the algorithms
can eliminate elements by moving from elements that were originally
in that range.
\end{itemdescr}

\indexlibrary{\idxcode{remove_copy}}%
\indexlibrary{\idxcode{remove_copy_if}}%
\begin{itemdecl}
template <InputIterator I, Sentinel<I> S, WeaklyIncrementable O, class T,
    class Proj = identity>
  requires IndirectlyCopyable<I, O> &&
    IndirectRelation<equal_to<>, projected<I, Proj>, const T*>
  tagged_pair<tag::in(I), tag::out(O)>
    remove_copy(I first, S last, O result, const T& value, Proj proj = Proj{});

template <InputRange Rng, WeaklyIncrementable O, class T, class Proj = identity>
  requires IndirectlyCopyable<iterator_t<Rng>, O> &&
    IndirectRelation<equal_to<>, projected<iterator_t<Rng>, Proj>, const T*>
  tagged_pair<tag::in(safe_iterator_t<Rng>), tag::out(O)>
    remove_copy(Rng&& rng, O result, const T& value, Proj proj = Proj{});

template <InputIterator I, Sentinel<I> S, WeaklyIncrementable O,
    class Proj = identity, IndirectUnaryPredicate<projected<I, Proj>> Pred>
  requires IndirectlyCopyable<I, O>
  tagged_pair<tag::in(I), tag::out(O)>
    remove_copy_if(I first, S last, O result, Pred pred, Proj proj = Proj{});

template <InputRange Rng, WeaklyIncrementable O, class Proj = identity,
    IndirectUnaryPredicate<projected<iterator_t<Rng>, Proj>> Pred>
  requires IndirectlyCopyable<iterator_t<Rng>, O>
  tagged_pair<tag::in(safe_iterator_t<Rng>), tag::out(O)>
    remove_copy_if(Rng&& rng, O result, Pred pred, Proj proj = Proj{});
\end{itemdecl}

\begin{itemdescr}
\pnum
\requires
The ranges
\range{first}{last}
and
\range{result}{result + (last - first)}
shall not overlap.

\pnum
\effects
Copies all the elements referred to by the iterator
\tcode{i}
in the range
\range{first}{last}
for which the following corresponding conditions do not hold:
\tcode{invoke(proj, *i) == value},
\tcode{invoke(pred, invoke(proj, *i)) != false}.

\pnum
\returns
A pair consisting of \tcode{last} and the end of the resulting range.

\pnum
\complexity
Exactly
\tcode{last - first}
applications of the corresponding predicate and projection.

\pnum
\remarks Stable~(\cxxref{algorithm.stable}).
\end{itemdescr}

\rSec2[std2.alg.unique]{Unique}

\indexlibrary{\idxcode{unique}}%
\begin{itemdecl}
template <ForwardIterator I, Sentinel<I> S, class Proj = identity,
    IndirectRelation<projected<I, Proj>> R = equal_to<>>
  requires Permutable<I>
  I unique(I first, S last, R comp = R{}, Proj proj = Proj{});

template <ForwardRange Rng, class Proj = identity,
    IndirectRelation<projected<iterator_t<Rng>, Proj>> R = equal_to<>>
  requires Permutable<iterator_t<Rng>>
  safe_iterator_t<Rng>
    unique(Rng&& rng, R comp = R{}, Proj proj = Proj{});
\end{itemdecl}

\begin{itemdescr}
\pnum
\effects
For a nonempty range, eliminates all but the first element from every
consecutive group of equivalent elements referred to by the iterator
\tcode{i}
in the range
\range{first + 1}{last}
for which the following conditions hold:
\tcode{invoke(proj, *(i - 1)) == invoke(proj, *i)}
or
\tcode{invoke(pred, invoke(proj, *(i - 1)), invoke(proj, *i)) != false}.

\pnum
\returns
The end of the resulting range.

\pnum
\complexity
For nonempty ranges, exactly
\tcode{(last - first) - 1}
applications of the corresponding predicate and no more than twice as many
applications of the projection.
\end{itemdescr}

\indexlibrary{\idxcode{unique_copy}}%
\begin{itemdecl}
template <InputIterator I, Sentinel<I> S, WeaklyIncrementable O,
    class Proj = identity, IndirectRelation<projected<I, Proj>> R = equal_to<>>
  requires IndirectlyCopyable<I, O> &&
    (ForwardIterator<I> ||
     (InputIterator<O> && Same<value_type_t<I>, value_type_t<O>>) ||
     IndirectlyCopyableStorable<I, O>)
  tagged_pair<tag::in(I), tag::out(O)>
    unique_copy(I first, S last, O result, R comp = R{}, Proj proj = Proj{});

template <InputRange Rng, WeaklyIncrementable O, class Proj = identity,
    IndirectRelation<projected<iterator_t<Rng>, Proj>> R = equal_to<>>
  requires IndirectlyCopyable<iterator_t<Rng>, O> &&
    (ForwardIterator<iterator_t<Rng>> ||
     (InputIterator<O> && Same<value_type_t<iterator_t<Rng>>, value_type_t<O>>) ||
     IndirectlyCopyableStorable<iterator_t<Rng>, O>)
  tagged_pair<tag::in(safe_iterator_t<Rng>), tag::out(O)>
    unique_copy(Rng&& rng, O result, R comp = R{}, Proj proj = Proj{});
\end{itemdecl}

\begin{itemdescr}
\pnum
\requires
The ranges
\range{first}{last}
and
\range{result}{result+(last-first)}
shall not overlap.

\pnum
\effects
Copies only the first element from every consecutive group of equal elements referred to by
the iterator
\tcode{i}
in the range
\range{first}{last}
for which the following corresponding conditions hold:
\begin{codeblock}
invoke(proj, *i) == invoke(proj, *(i - 1))
\end{codeblock}
or
\begin{codeblock}
invoke(pred, invoke(proj, *i), invoke(proj, *(i - 1))) != false.
\end{codeblock}

\pnum
\returns
A pair consisting of \tcode{last} and the end of the resulting range.

\pnum
\complexity
For nonempty ranges, exactly
\tcode{last - first - 1}
applications of the corresponding predicate and no more than twice as many
applications of the projection.
\end{itemdescr}

\rSec2[std2.alg.reverse]{Reverse}

\indexlibrary{\idxcode{reverse}}%
\begin{itemdecl}
template <BidirectionalIterator I, Sentinel<I> S>
  requires Permutable<I>
  I reverse(I first, S last);

template <BidirectionalRange Rng>
  requires Permutable<iterator_t<Rng>>
  safe_iterator_t<Rng>
    reverse(Rng&& rng);
\end{itemdecl}

\begin{itemdescr}
\pnum
\effects
For each non-negative integer
\tcode{i < (last - first)/2},
applies
\tcode{iter_swap}
to all pairs of iterators
\tcode{first + i, (last - i) - 1}.

\pnum
\returns \tcode{last}.

\pnum
\complexity
Exactly
\tcode{(last - first)/2}
swaps.
\end{itemdescr}

\indexlibrary{\idxcode{reverse_copy}}%
\begin{itemdecl}
template <BidirectionalIterator I, Sentinel<I> S, WeaklyIncrementable O>
  requires IndirectlyCopyable<I, O>
  tagged_pair<tag::in(I), tag::out(O)> reverse_copy(I first, S last, O result);

template <BidirectionalRange Rng, WeaklyIncrementable O>
  requires IndirectlyCopyable<iterator_t<Rng>, O>
  tagged_pair<tag::in(safe_iterator_t<Rng>), tag::out(O)>
    reverse_copy(Rng&& rng, O result);
\end{itemdecl}

\begin{itemdescr}
\pnum
\effects
Copies the range
\range{first}{last}
to the range
\range{result}{result+(last-first)}
such that
for every non-negative integer
\tcode{i < (last - first)}
the following assignment takes place:
\tcode{*(result + (last - first) - 1 - i) = *(first + i)}.

\pnum
\requires
The ranges
\range{first}{last}
and
\range{result}{result+(last-first)}
shall not overlap.

\pnum
\returns
\tcode{\{last, result + (last - first)\}}.

\pnum
\complexity
Exactly
\tcode{last - first}
assignments.
\end{itemdescr}

\rSec2[std2.alg.rotate]{Rotate}

\indexlibrary{\idxcode{rotate}}%
\begin{itemdecl}
template <ForwardIterator I, Sentinel<I> S>
  requires Permutable<I>
  tagged_pair<tag::begin(I), tag::end(I)> rotate(I first, I middle, S last);

template <ForwardRange Rng>
  requires Permutable<iterator_t<Rng>>
  tagged_pair<tag::begin(safe_iterator_t<Rng>), tag::end(safe_iterator_t<Rng>)>
    rotate(Rng&& rng, iterator_t<Rng> middle);
\end{itemdecl}

\begin{itemdescr}
\pnum
\effects
For each non-negative integer
\tcode{i < (last - first)},
places the element from the position
\tcode{first + i}
into position
\tcode{first + (i + (last - middle)) \% (last - first)}.

\pnum
\returns \tcode{\{first + (last - middle), last\}}.

\pnum
\notes
This is a left rotate.

\pnum
\requires
\range{first}{middle}
and
\range{middle}{last}
shall be valid ranges.

\pnum
\complexity
At most
\tcode{last - first}
swaps.
\end{itemdescr}

\indexlibrary{\idxcode{rotate_copy}}%
\begin{itemdecl}
template <ForwardIterator I, Sentinel<I> S, WeaklyIncrementable O>
  requires IndirectlyCopyable<I, O>
  tagged_pair<tag::in(I), tag::out(O)>
    rotate_copy(I first, I middle, S last, O result);

template <ForwardRange Rng, WeaklyIncrementable O>
  requires IndirectlyCopyable<iterator_t<Rng>, O>
  tagged_pair<tag::in(safe_iterator_t<Rng>), tag::out(O)>
    rotate_copy(Rng&& rng, iterator_t<Rng> middle, O result);
\end{itemdecl}

\begin{itemdescr}
\pnum
\effects
Copies the range
\range{first}{last}
to the range
\range{result}{result + (last - first)}
such that for each non-negative integer
\tcode{i < (last - first)}
the following assignment takes place:
\tcode{*(result + i) =  *(first +
(i + (middle - first)) \% (last - first))}.

\pnum
\returns
\tcode{\{last, result + (last - first)\}}.

\pnum
\requires
The ranges
\range{first}{last}
and
\range{result}{result + (last - first)}
shall not overlap.

\pnum
\complexity
Exactly
\tcode{last - first}
assignments.
\end{itemdescr}

\rSec2[std2.alg.random.shuffle]{Shuffle}

\indexlibrary{\idxcode{shuffle}}%
\begin{itemdecl}
template <RandomAccessIterator I, Sentinel<I> S, class Gen>
  requires Permutable<I> &&
    UniformRandomNumberGenerator<remove_reference_t<Gen>> &&
    ConvertibleTo<result_of_t<Gen&()>, difference_type_t<I>>
  I shuffle(I first, S last, Gen&& g);

template <RandomAccessRange Rng, class Gen>
  requires Permutable<I> &&
    UniformRandomNumberGenerator<remove_reference_t<Gen>> &&
    ConvertibleTo<result_of_t<Gen&()>, difference_type_t<I>>
  safe_iterator_t<Rng>
    shuffle(Rng&& rng, Gen&& g);
\end{itemdecl}

\begin{itemdescr}
\pnum
\effects
Permutes the elements in the range
\range{first}{last}
such that each possible permutation of those elements has equal probability of appearance.

\pnum
\complexity
Exactly
\tcode{(last - first) - 1}
swaps.

\pnum
\returns \tcode{last}

\pnum
\notes
To the extent that the implementation of this function makes use of random
numbers, the object \tcode{g} shall serve as the implementation's source of
randomness.

\end{itemdescr}

\rSec2[std2.alg.partitions]{Partitions}

\indexlibrary{\idxcode{is_partitioned}}%
\begin{itemdecl}
template <InputIterator I, Sentinel<I> S, class Proj = identity,
    IndirectUnaryPredicate<projected<I, Proj>> Pred>
  bool is_partitioned(I first, S last, Pred pred, Proj proj = Proj{});

template <InputRange Rng, class Proj = identity,
    IndirectUnaryPredicate<projected<iterator_t<Rng>, Proj>> Pred>
  bool
    is_partitioned(Rng&& rng, Pred pred, Proj proj = Proj{});
\end{itemdecl}

\begin{itemdescr}
\pnum
\returns \tcode{true} if
\range{first}{last} is empty or if
\range{first}{last} is partitioned by \tcode{pred} and \tcode{proj}, i.e. if all
iterators \tcode{i} for which
\tcode{invoke(pred, invoke(proj, *i)) != false} come before those that do not,
for every \tcode{i} in \range{first}{last}.

\pnum
\complexity Linear. At most \tcode{last - first} applications of \tcode{pred} and \tcode{proj}.
\end{itemdescr}

\indexlibrary{\idxcode{partition}}%
\begin{itemdecl}
template <ForwardIterator I, Sentinel<I> S, class Proj = identity,
    IndirectUnaryPredicate<projected<I, Proj>> Pred>
  requires Permutable<I>
  I partition(I first, S last, Pred pred, Proj proj = Proj{});

template <ForwardRange Rng, class Proj = identity,
    IndirectUnaryPredicate<projected<iterator_t<Rng>, Proj>> Pred>
  requires Permutable<iterator_t<Rng>>
  safe_iterator_t<Rng>
    partition(Rng&& rng, Pred pred, Proj proj = Proj{});
\end{itemdecl}

\begin{itemdescr}
\pnum
\effects Permutes the elements in the range \range{first}{last} such that there exists an iterator \tcode{i}
such that for every iterator \tcode{j} in the range \range{first}{i}
\tcode{invoke(pred, invoke(\brk{}proj, *j)) != false}, and for every iterator \tcode{k} in the
range \range{i}{last}, \tcode{invoke(pred, invoke(proj, *k)) == false}.

\pnum
\returns An iterator \tcode{i} such that for every iterator \tcode{j} in the range \range{first}{i}
\tcode{invoke(pred, invoke(\brk{}proj, *j)) != false},
and for every iterator \tcode{k} in the range \range{i}{last},
\tcode{invoke(pred, invoke(proj, *k)) == false}.

\pnum
\complexity If I meets the requirements for a BidirectionalIterator, at most
\tcode{(last - first) / 2} swaps; otherwise at most \tcode{last - first} swaps.
Exactly \tcode{last - first} applications of the predicate and projection.
\end{itemdescr}

\indexlibrary{\idxcode{stable_partition}}%
\begin{itemdecl}
template <BidirectionalIterator I, Sentinel<I> S, class Proj = identity,
    IndirectUnaryPredicate<projected<I, Proj>> Pred>
  requires Permutable<I>
  I stable_partition(I first, S last, Pred pred, Proj proj = Proj{});

template <BidirectionalRange Rng, class Proj = identity,
    IndirectUnaryPredicate<projected<iterator_t<Rng>, Proj>> Pred>
  requires Permutable<iterator_t<Rng>>
  safe_iterator_t<Rng>
    stable_partition(Rng&& rng, Pred pred, Proj proj = Proj{});
\end{itemdecl}

\begin{itemdescr}
\pnum
\effects Permutes the elements in the range \range{first}{last} such that there exists an iterator \tcode{i}
such that for every iterator \tcode{j} in the range \range{first}{i}
\tcode{invoke(pred, invoke(proj, *j)) != false}, and for every iterator \tcode{k} in the
range \range{i}{last}, \tcode{invoke(pred, invoke(proj, *k)) == false}.

\pnum
\returns
An iterator
\tcode{i}
such that for every iterator
\tcode{j}
in the range
\range{first}{i},
\tcode{invoke(pred, invoke(\brk{}proj, *j)) != false},
and for every iterator
\tcode{k}
in the range
\range{i}{last},
\tcode{invoke(pred, invoke(\brk{}proj, *k)) == false}.
The relative order of the elements in both groups is preserved.

\pnum
\complexity
At most
\tcode{(last - first) * log(last - first)}
swaps, but only linear number of swaps if there is enough extra memory.
Exactly
\tcode{last - first}
applications of the predicate and projection.
\end{itemdescr}

\indexlibrary{\idxcode{partition_copy}}%
\begin{itemdecl}
template <InputIterator I, Sentinel<I> S, WeaklyIncrementable O1, WeaklyIncrementable O2,
    class Proj = identity, IndirectUnaryPredicate<projected<I, Proj>> Pred>
  requires IndirectlyCopyable<I, O1> && IndirectlyCopyable<I, O2>
  tagged_tuple<tag::in(I), tag::out1(O1), tag::out2(O2)>
    partition_copy(I first, S last, O1 out_true, O2 out_false, Pred pred,
                   Proj proj = Proj{});

template <InputRange Rng, WeaklyIncrementable O1, WeaklyIncrementable O2,
    class Proj = identity,
    IndirectUnaryPredicate<projected<iterator_t<Rng>, Proj>> Pred>
  requires IndirectlyCopyable<iterator_t<Rng>, O1> &&
    IndirectlyCopyable<iterator_t<Rng>, O2>
  tagged_tuple<tag::in(safe_iterator_t<Rng>), tag::out1(O1), tag::out2(O2)>
    partition_copy(Rng&& rng, O1 out_true, O2 out_false, Pred pred, Proj proj = Proj{});
\end{itemdecl}

\begin{itemdescr}
\pnum
\requires The input range shall not overlap with
either of the output ranges.

\pnum
\effects For each iterator \tcode{i} in \range{first}{last}, copies \tcode{*i} to the output range
beginning with \tcode{out_true} if
\tcode{invoke(pred, invoke(proj, *i))} is \tcode{true}, or to
the output range beginning with \tcode{out_false} otherwise.

\pnum
\returns A tuple \tcode{p} such that \tcode{get<0>(p)} is \tcode{last},
\tcode{get<1>(p)} is the end of the output range beginning at \tcode{out_true},
and \tcode{get<2>(p)} is the end of the output range beginning at \tcode{out_false}.

\pnum
\complexity Exactly \tcode{last - first} applications of \tcode{pred} and \tcode{proj}.
\end{itemdescr}

\indexlibrary{\idxcode{partition_point}}%
\begin{itemdecl}
template <ForwardIterator I, Sentinel<I> S, class Proj = identity,
    IndirectUnaryPredicate<projected<I, Proj>> Pred>
  I partition_point(I first, S last, Pred pred, Proj proj = Proj{});

template <ForwardRange Rng, class Proj = identity,
    IndirectUnaryPredicate<projected<iterator_t<Rng>, Proj>> Pred>
  safe_iterator_t<Rng>
    partition_point(Rng&& rng, Pred pred, Proj proj = Proj{});
\end{itemdecl}

\begin{itemdescr}
\pnum
\requires \range{first}{last} shall be partitioned by \tcode{pred} and \tcode{proj}, i.e.
there shall be an iterator \tcode{mid} such that
\tcode{all_of(first, mid, pred, proj)} and \tcode{none_of(mid, last, pred, proj)}
are both true.

\pnum
\returns An iterator \tcode{mid} such that \tcode{all_of(first, mid, pred, proj)} and
\tcode{none_of(mid, last, pred, proj)} are both true.

\pnum
\complexity \bigoh{\log(\tcode{last - first})} applications of \tcode{pred} and \tcode{proj}.
\end{itemdescr}


\rSec1[std2.alg.sorting]{Sorting and related operations}

\pnum
All the operations in~\ref{std2.alg.sorting} take an optional binary callable predicate of type \tcode{Comp} that defaults to \tcode{less<>}.

\pnum
\tcode{Comp}
is a callable object~(\cxxref{func.require}). The return value of the \tcode{invoke} operation applied to
an object of type \tcode{Comp}, when contextually converted to
\tcode{bool} (Clause~\cxxref{conv}),
yields \tcode{true} if the first argument of the call
is less than the second, and
\tcode{false}
otherwise.
\tcode{Comp comp}
is used throughout for algorithms assuming an ordering relation.
It is assumed that
\tcode{comp}
will not apply any non-constant function through the dereferenced iterator.

\pnum
A sequence is
\techterm{sorted with respect to a comparator and projection}
\tcode{comp} and \tcode{proj} if for every iterator
\tcode{i}
pointing to the sequence and every non-negative integer
\tcode{n}
such that
\tcode{i + n}
is a valid iterator pointing to an element of the sequence,
\tcode{invoke(comp, invoke(proj, *(i + n)), invoke(proj, *i)) == false}.

\pnum
A sequence
\range{start}{finish}
is
\techterm{partitioned with respect to an expression}
\tcode{f(e)}
if there exists an integer
\tcode{n}
such that for all
\tcode{0 <= i < distance(start, finish)},
\tcode{f(*(start + i))}
is true if and only if
\tcode{i < n}.

\pnum
In the descriptions of the functions that deal with ordering relationships we frequently use a notion of
equivalence to describe concepts such as stability.
The equivalence to which we refer is not necessarily an
\tcode{operator==},
but an equivalence relation induced by the strict weak ordering.
That is, two elements
\tcode{a}
and
\tcode{b}
are considered equivalent if and only if
\tcode{!(a < b) \&\& !(b < a)}.

\rSec2[std2.alg.sort]{Sorting}

\rSec3[std2.sort]{\tcode{sort}}

\indexlibrary{\idxcode{sort}}%
\begin{itemdecl}
template <RandomAccessIterator I, Sentinel<I> S, class Comp = less<>,
    class Proj = identity>
  requires Sortable<I, Comp, Proj>
  I sort(I first, S last, Comp comp = Comp{}, Proj proj = Proj{});

template <RandomAccessRange Rng, class Comp = less<>, class Proj = identity>
  requires Sortable<iterator_t<Rng>, Comp, Proj>
  safe_iterator_t<Rng>
    sort(Rng&& rng, Comp comp = Comp{}, Proj proj = Proj{});
\end{itemdecl}

\begin{itemdescr}
\pnum
\effects
Sorts the elements in the range
\range{first}{last}.

\pnum
\returns \tcode{last}.

\pnum
\complexity
\bigoh{N\log(N)}
(where
\tcode{$N$ == last - first})
comparisons, and twice as many applications of the projection.
\end{itemdescr}

\rSec3[std2.stable.sort]{\tcode{stable_sort}}

\indexlibrary{\idxcode{stable_sort}}%
\begin{itemdecl}
template <RandomAccessIterator I, Sentinel<I> S, class Comp = less<>,
    class Proj = identity>
  requires Sortable<I, Comp, Proj>
  I stable_sort(I first, S last, Comp comp = Comp{}, Proj proj = Proj{});

template <RandomAccessRange Rng, class Comp = less<>, class Proj = identity>
  requires Sortable<iterator_t<Rng>, Comp, Proj>
  safe_iterator_t<Rng>
    stable_sort(Rng&& rng, Comp comp = Comp{}, Proj proj = Proj{});
\end{itemdecl}

\begin{itemdescr}
\pnum
\effects
Sorts the elements in the range \range{first}{last}.

\pnum
\returns \tcode{last}.

\pnum
\complexity
Let \tcode{$N$ == last - first}.
If enough extra memory is available, $N \log(N)$ comparisons.
Otherwise, at most $N \log^2(N)$ comparisons.
In either case, twice as many applications of the projection as the number of
comparisons.

\pnum
\remarks Stable~(\cxxref{algorithm.stable}).
\end{itemdescr}

\rSec3[std2.partial.sort]{\tcode{partial_sort}}

\indexlibrary{\idxcode{partial_sort}}%
\begin{itemdecl}
template <RandomAccessIterator I, Sentinel<I> S, class Comp = less<>,
    class Proj = identity>
  requires Sortable<I, Comp, Proj>
  I partial_sort(I first, I middle, S last, Comp comp = Comp{}, Proj proj = Proj{});

template <RandomAccessRange Rng, class Comp = less<>, class Proj = identity>
  requires Sortable<iterator_t<Rng>, Comp, Proj>
  safe_iterator_t<Rng>
    partial_sort(Rng&& rng, iterator_t<Rng> middle, Comp comp = Comp{},
                 Proj proj = Proj{});
\end{itemdecl}

\begin{itemdescr}
\pnum
\effects
Places the first
\tcode{middle - first}
sorted elements from the range
\range{first}{last}
into the range
\range{first}{middle}.
The rest of the elements in the range
\range{middle}{last}
are placed in an unspecified order.
\indextext{unspecified}%

\pnum
\returns \tcode{last}.

\pnum
\complexity
It takes approximately
\tcode{(last - first) * log(middle - first)}
comparisons, and exactly twice as many applications of the projection.
\end{itemdescr}

\rSec3[std2.partial.sort.copy]{\tcode{partial_sort_copy}}

\indexlibrary{\idxcode{partial_sort_copy}}%
\begin{itemdecl}
template <InputIterator I1, Sentinel<I1> S1, RandomAccessIterator I2, Sentinel<I2> S2,
    class Comp = less<>, class Proj1 = identity, class Proj2 = identity>
  requires IndirectlyCopyable<I1, I2> && Sortable<I2, Comp, Proj2> &&
      IndirectStrictWeakOrder<Comp, projected<I1, Proj1>, projected<I2, Proj2>>
  I2
    partial_sort_copy(I1 first, S1 last, I2 result_first, S2 result_last,
                      Comp comp = Comp{}, Proj1 proj1 = Proj1{}, Proj2 proj2 = Proj2{});

template <InputRange Rng1, RandomAccessRange Rng2, class Comp = less<>,
    class Proj1 = identity, class Proj2 = identity>
  requires IndirectlyCopyable<iterator_t<Rng1>, iterator_t<Rng2>> &&
      Sortable<iterator_t<Rng2>, Comp, Proj2> &&
      IndirectStrictWeakOrder<Comp, projected<iterator_t<Rng1>, Proj1>,
        projected<iterator_t<Rng2>, Proj2>>
  safe_iterator_t<Rng2>
    partial_sort_copy(Rng1&& rng, Rng2&& result_rng, Comp comp = Comp{},
                      Proj1 proj1 = Proj1{}, Proj2 proj2 = Proj2{});
\end{itemdecl}

\begin{itemdescr}
\pnum
\effects
Places the first
\tcode{min(last - first, result_last - result_first)}
sorted elements into the range
\range{result_first}{result_first + min(last - first, result_last - result_first)}.

\pnum
\returns
The smaller of:
\tcode{result_last} or
\tcode{result_first + (last - first)}.

\pnum
\complexity
Approximately
\begin{codeblock}
(last - first) * log(min(last - first, result_last - result_first))
\end{codeblock}
comparisons, and exactly twice as many applications of the projection.
\end{itemdescr}

\rSec3[std2.is.sorted]{\tcode{is_sorted}}

\indexlibrary{\idxcode{is_sorted}}%
\begin{itemdecl}
template <ForwardIterator I, Sentinel<I> S, class Proj = identity,
    IndirectStrictWeakOrder<projected<I, Proj>> Comp = less<>>
  bool is_sorted(I first, S last, Comp comp = Comp{}, Proj proj = Proj{});

template <ForwardRange Rng, class Proj = identity,
    IndirectStrictWeakOrder<projected<iterator_t<Rng>, Proj>> Comp = less<>>
  bool
    is_sorted(Rng&& rng, Comp comp = Comp{}, Proj proj = Proj{});
\end{itemdecl}

\begin{itemdescr}
\pnum
\returns \tcode{is_sorted_until(first, last, comp, proj) == last}
\end{itemdescr}

\indexlibrary{\idxcode{is_sorted_until}}%
\begin{itemdecl}
template <ForwardIterator I, Sentinel<I> S, class Proj = identity,
    IndirectStrictWeakOrder<projected<I, Proj>> Comp = less<>>
  I is_sorted_until(I first, S last, Comp comp = Comp{}, Proj proj = Proj{});

template <ForwardRange Rng, class Proj = identity,
    IndirectStrictWeakOrder<projected<iterator_t<Rng>, Proj>> Comp = less<>>
  safe_iterator_t<Rng>
    is_sorted_until(Rng&& rng, Comp comp = Comp{}, Proj proj = Proj{});
\end{itemdecl}

\begin{itemdescr}
\pnum
\returns If \tcode{distance(first, last) < 2}, returns
\tcode{last}. Otherwise, returns
the last iterator \tcode{i} in \crange{first}{last} for which the
range \range{first}{i} is sorted.

\pnum
\complexity Linear.
\end{itemdescr}

\rSec2[std2.alg.nth.element]{Nth element}

\indexlibrary{\idxcode{nth_element}}%
\begin{itemdecl}
template <RandomAccessIterator I, Sentinel<I> S, class Comp = less<>,
    class Proj = identity>
  requires Sortable<I, Comp, Proj>
  I nth_element(I first, I nth, S last, Comp comp = Comp{}, Proj proj = Proj{});

template <RandomAccessRange Rng, class Comp = less<>, class Proj = identity>
  requires Sortable<iterator_t<Rng>, Comp, Proj>
  safe_iterator_t<Rng>
    nth_element(Rng&& rng, iterator_t<Rng> nth, Comp comp = Comp{}, Proj proj = Proj{});
\end{itemdecl}

\begin{itemdescr}
\pnum
After
\tcode{nth_element}
the element in the position pointed to by \tcode{nth}
is the element that would be
in that position if the whole range were sorted, unless \tcode{nth == last}.
Also for every iterator
\tcode{i}
in the range
\range{first}{nth}
and every iterator
\tcode{j}
in the range
\range{nth}{last}
it holds that:
\tcode{invoke(comp, invoke(proj, *j), invoke(proj, *i)) == false}.

\pnum
\returns \tcode{last}.

\pnum
\complexity
Linear on average.
\end{itemdescr}

\rSec2[std2.alg.binary.search]{Binary search}

\pnum
All of the algorithms in this section are versions of binary search
and assume that the sequence being searched is partitioned with respect to
an expression formed by binding the search key to an argument of the
comparison function and projection.
They work on non-random access iterators minimizing the number of comparisons,
which will be logarithmic for all types of iterators.
They are especially appropriate for random access iterators,
because these algorithms do a logarithmic number of steps
through the data structure.
For non-random access iterators they execute a linear number of steps.

\rSec3[std2.lower.bound]{\tcode{lower_bound}}

\indexlibrary{\idxcode{lower_bound}}%
\begin{itemdecl}
template <ForwardIterator I, Sentinel<I> S, class T, class Proj = identity,
    IndirectStrictWeakOrder<const T*, projected<I, Proj>> Comp = less<>>
  I
    lower_bound(I first, S last, const T& value, Comp comp = Comp{},
                Proj proj = Proj{});

template <ForwardRange Rng, class T, class Proj = identity,
    IndirectStrictWeakOrder<const T*, projected<iterator_t<Rng>, Proj>> Comp = less<>>
  safe_iterator_t<Rng>
    lower_bound(Rng&& rng, const T& value, Comp comp = Comp{}, Proj proj = Proj{});
\end{itemdecl}

\begin{itemdescr}
\pnum
\requires
The elements
\tcode{e}
of
\range{first}{last}
shall be partitioned with respect to the expression
\tcode{invoke(\brk{}comp, invoke(proj, e), value)}.

\pnum
\returns
The furthermost iterator
\tcode{i}
in the range
\crange{first}{last}
such that for every iterator
\tcode{j}
in the range
\range{first}{i}
the following corresponding condition holds:
\tcode{invoke(comp, invoke(proj, *j), value) != false}.

\pnum
\complexity
At most
$\log_2(\tcode{last - first}) + \bigoh{1}$
applications of the comparison function and projection.
\end{itemdescr}

\rSec3[std2.upper.bound]{\tcode{upper_bound}}

\indexlibrary{\idxcode{upper_bound}}%
\begin{itemdecl}
template <ForwardIterator I, Sentinel<I> S, class T, class Proj = identity,
    IndirectStrictWeakOrder<const T*, projected<I, Proj>> Comp = less<>>
  I
    upper_bound(I first, S last, const T& value, Comp comp = Comp{}, Proj proj = Proj{});

template <ForwardRange Rng, class T, class Proj = identity,
    IndirectStrictWeakOrder<const T*, projected<iterator_t<Rng>, Proj>> Comp = less<>>
  safe_iterator_t<Rng>
    upper_bound(Rng&& rng, const T& value, Comp comp = Comp{}, Proj proj = Proj{});
\end{itemdecl}

\begin{itemdescr}
\pnum
\requires
The elements
\tcode{e}
of
\range{first}{last}
shall be partitioned with respect to the expression
\tcode{!invoke(\brk{}comp, value, invoke(proj, e))}.

\pnum
\returns
The furthermost iterator
\tcode{i}
in the range
\crange{first}{last}
such that for every iterator
\tcode{j}
in the range
\range{first}{i}
the following corresponding condition holds:
\tcode{invoke(comp, value, invoke(proj, *j)) == false}.

\pnum
\complexity
At most
$\log_2(\tcode{last - first}) + \bigoh{1}$
applications of the comparison function and projection.
\end{itemdescr}

\rSec3[std2.equal.range]{\tcode{equal_range}}

\indexlibrary{\idxcode{equal_range}}%
\begin{itemdecl}
template <ForwardIterator I, Sentinel<I> S, class T, class Proj = identity,
    IndirectStrictWeakOrder<const T*, projected<I, Proj>> Comp = less<>>
  tagged_pair<tag::begin(I), tag::end(I)>
    equal_range(I first, S last, const T& value, Comp comp = Comp{}, Proj proj = Proj{});

template <ForwardRange Rng, class T, class Proj = identity,
    IndirectStrictWeakOrder<const T*, projected<iterator_t<Rng>, Proj>> Comp = less<>>
  tagged_pair<tag::begin(safe_iterator_t<Rng>),
              tag::end(safe_iterator_t<Rng>)>
    equal_range(Rng&& rng, const T& value, Comp comp = Comp{}, Proj proj = Proj{});
\end{itemdecl}

\begin{itemdescr}
\pnum
\requires
The elements
\tcode{e}
of
\range{first}{last}
shall be partitioned with respect to the expressions
\tcode{invoke(comp, invoke(proj, e), value)}
and
\tcode{!invoke(\brk{}comp, value, invoke(proj, e))}.
Also, for all elements
\tcode{e}
of
\tcode{[first, last)},
\tcode{invoke(comp, invoke(proj, e), value)}
shall imply \\
\tcode{!invoke(\brk{}comp, value, invoke(proj, e))}.

\pnum
\returns
\begin{codeblock}
{lower_bound(first, last, value, comp, proj),
 upper_bound(first, last, value, comp, proj)}
\end{codeblock}

\pnum
\complexity
At most
$2 * \log_2(\tcode{last - first}) + \bigoh{1}$
applications of the comparison function and projection.
\end{itemdescr}

\rSec3[std2.binary.search]{\tcode{binary_search}}

\indexlibrary{\idxcode{binary_search}}%
\begin{itemdecl}
template <ForwardIterator I, Sentinel<I> S, class T, class Proj = identity,
    IndirectStrictWeakOrder<const T*, projected<I, Proj>> Comp = less<>>
  bool
    binary_search(I first, S last, const T& value, Comp comp = Comp{},
                  Proj proj = Proj{});

template <ForwardRange Rng, class T, class Proj = identity,
    IndirectStrictWeakOrder<const T*, projected<iterator_t<Rng>, Proj>> Comp = less<>>
  bool
    binary_search(Rng&& rng, const T& value, Comp comp = Comp{},
                  Proj proj = Proj{});
\end{itemdecl}

\begin{itemdescr}
\pnum
\requires
The elements
\tcode{e}
of
\range{first}{last}
are partitioned with respect to the expressions
\tcode{invoke(\brk{}comp, invoke(proj, e), value)}
and
\tcode{!invoke(comp, value, invoke(proj, e))}.
Also, for all elements
\tcode{e}
of
\tcode{[first, last)},
\tcode{invoke(comp, invoke(proj, e), value)}
shall imply
\tcode{!invoke(\brk{}comp, value, invoke(proj, e))}.

\pnum
\returns
\tcode{true}
if there is an iterator
\tcode{i}
in the range
\range{first}{last}
that satisfies the corresponding conditions:
\tcode{
invoke(comp, invoke(proj, *i), value) == false \&\&
invoke(comp, value, invoke(proj, *i)) == false}.

\pnum
\complexity
At most
$\log_2(\tcode{last - first}) + \bigoh{1}$
applications of the comparison function and projection.
\end{itemdescr}

\rSec2[std2.alg.merge]{Merge}

\indexlibrary{\idxcode{merge}}%
\begin{itemdecl}
template <InputIterator I1, Sentinel<I1> S1, InputIterator I2, Sentinel<I2> S2,
    WeaklyIncrementable O, class Comp = less<>, class Proj1 = identity,
    class Proj2 = identity>
  requires Mergeable<I1, I2, O, Comp, Proj1, Proj2>
  tagged_tuple<tag::in1(I1), tag::in2(I2), tag::out(O)>
    merge(I1 first1, S1 last1, I2 first2, S2 last2, O result,
          Comp comp = Comp{}, Proj1 proj1 = Proj1{}, Proj2 proj2 = Proj2{});

template <InputRange Rng1, InputRange Rng2, WeaklyIncrementable O, class Comp = less<>,
    class Proj1 = identity, class Proj2 = identity>
  requires Mergeable<iterator_t<Rng1>, iterator_t<Rng2>, O, Comp, Proj1, Proj2>
  tagged_tuple<tag::in1(safe_iterator_t<Rng1>),
               tag::in2(safe_iterator_t<Rng2>),
               tag::out(O)>
    merge(Rng1&& rng1, Rng2&& rng2, O result,
          Comp comp = Comp{}, Proj1 proj1 = Proj1{}, Proj2 proj2 = Proj2{});
\end{itemdecl}

\begin{itemdescr}
\pnum
\effects\ Copies all the elements of the two ranges \range{first1}{last1} and
\range{first2}{last2} into the range \range{result}{result_last}, where \tcode{result_last}
is \tcode{result + (last1 - first1) + (last2 - first2)}.
If an element \tcode{a} precedes \tcode{b} in an input range,
\tcode{a} is copied into the output range before \tcode{b}. If \tcode{e1} is
an element of \range{first1}{last1} and \tcode{e2} of \range{first2}{last2},
\tcode{e2} is copied into the output range before \tcode{e1} if and only if
\tcode{bool(invoke(comp, invoke(proj2, e2), invoke(proj1, e1)))} is
\tcode{true}.

\pnum
\requires The ranges \range{first1}{last1} and \range{first2}{last2} shall be
sorted with respect to \tcode{comp}, \tcode{proj1}, and \tcode{proj2}.
The resulting range shall not overlap with either of the original ranges.

\pnum
\returns
\tcode{make_tagged_tuple<tag::in1, tag::in2, tag::out>(last1, last2, result_last)}.

\pnum
\complexity
At most
\tcode{(last1 - first1) + (last2 - first2) - 1}
applications of the comparison function and each projection.

\pnum
\remarks Stable~(\cxxref{algorithm.stable}).
\end{itemdescr}

\indexlibrary{\idxcode{inplace_merge}}%
\begin{itemdecl}
template <BidirectionalIterator I, Sentinel<I> S, class Comp = less<>,
    class Proj = identity>
  requires Sortable<I, Comp, Proj>
  I
    inplace_merge(I first, I middle, S last, Comp comp = Comp{}, Proj proj = Proj{});

template <BidirectionalRange Rng, class Comp = less<>, class Proj = identity>
  requires Sortable<iterator_t<Rng>, Comp, Proj>
  safe_iterator_t<Rng>
    inplace_merge(Rng&& rng, iterator_t<Rng> middle, Comp comp = Comp{},
                  Proj proj = Proj{});
\end{itemdecl}

\begin{itemdescr}
\pnum
\effects
Merges two sorted consecutive ranges
\range{first}{middle}
and
\range{middle}{last},
putting the result of the merge into the range
\range{first}{last}.
The resulting range will be in non-decreasing order;
that is, for every iterator
\tcode{i}
in
\range{first}{last}
other than
\tcode{first},
the condition
\tcode{invoke(comp, invoke(proj, *i), invoke(proj, *(i - 1)))}
will be false.

\pnum
\requires
The ranges \range{first}{middle} and \range{middle}{last} shall be
sorted with respect to \tcode{comp} and \tcode{proj}.

\pnum
\returns \tcode{last}

\pnum
\complexity
When enough additional memory is available,
\tcode{(last - first) - 1}
applications of the comparison function and projection.
If no additional memory is available, an algorithm with complexity
$N \log(N)$
(where
\tcode{N}
is equal to
\tcode{last - first})
may be used.

\pnum
\remarks Stable~(\cxxref{algorithm.stable}).
\end{itemdescr}

\rSec2[std2.alg.set.operations]{Set operations on sorted structures}

\pnum
This section defines all the basic set operations on sorted structures.
They also work with
\tcode{multiset}s~(\cxxref{multiset})
containing multiple copies of equivalent elements.
The semantics of the set operations are generalized to
\tcode{multiset}s
in a standard way by defining
\tcode{set_union()}
to contain the maximum number of occurrences of every element,
\tcode{set_intersection()}
to contain the minimum, and so on.

\rSec3[std2.includes]{\tcode{includes}}

\indexlibrary{\idxcode{includes}}%
\begin{itemdecl}
template <InputIterator I1, Sentinel<I1> S1, InputIterator I2, Sentinel<I2> S2,
    class Proj1 = identity, class Proj2 = identity,
    IndirectStrictWeakOrder<projected<I1, Proj1>, projected<I2, Proj2>> Comp = less<>>
  bool
    includes(I1 first1, S1 last1, I2 first2, S2 last2, Comp comp = Comp{},
             Proj1 proj1 = Proj1{}, Proj2 proj2 = Proj2{});

template <InputRange Rng1, InputRange Rng2, class Proj1 = identity,
    class Proj2 = identity,
    IndirectStrictWeakOrder<projected<iterator_t<Rng1>, Proj1>,
      projected<iterator_t<Rng2>, Proj2>> Comp = less<>>
  bool
    includes(Rng1&& rng1, Rng2&& rng2, Comp comp = Comp{},
             Proj1 proj1 = Proj1{}, Proj2 proj2 = Proj2{});
\end{itemdecl}

\begin{itemdescr}
\pnum
\returns
\tcode{true}
if \range{first2}{last2} is empty or
if every element in the range
\range{first2}{last2}
is contained in the range
\range{first1}{last1}.
Returns
\tcode{false}
otherwise.

\pnum
\complexity
At most
\tcode{2 * ((last1 - first1) + (last2 - first2)) - 1}
applications of the comparison function and projections.
\end{itemdescr}

\rSec3[std2.set.union]{\tcode{set_union}}

\indexlibrary{\idxcode{set_union}}%
\begin{itemdecl}
template <InputIterator I1, Sentinel<I1> S1, InputIterator I2, Sentinel<I2> S2,
    WeaklyIncrementable O, class Comp = less<>, class Proj1 = identity, class Proj2 = identity>
  requires Mergeable<I1, I2, O, Comp, Proj1, Proj2>
  tagged_tuple<tag::in1(I1), tag::in2(I2), tag::out(O)>
    set_union(I1 first1, S1 last1, I2 first2, S2 last2, O result, Comp comp = Comp{},
              Proj1 proj1 = Proj1{}, Proj2 proj2 = Proj2{});

template <InputRange Rng1, InputRange Rng2, WeaklyIncrementable O,
    class Comp = less<>, class Proj1 = identity, class Proj2 = identity>
  requires Mergeable<iterator_t<Rng1>, iterator_t<Rng2>, O, Comp, Proj1, Proj2>
  tagged_tuple<tag::in1(safe_iterator_t<Rng1>),
               tag::in2(safe_iterator_t<Rng2>),
               tag::out(O)>
    set_union(Rng1&& rng1, Rng2&& rng2, O result, Comp comp = Comp{},
              Proj1 proj1 = Proj1{}, Proj2 proj2 = Proj2{});
\end{itemdecl}

\begin{itemdescr}
\pnum
\effects
Constructs a sorted union of the elements from the two ranges;
that is, the set of elements that are present in one or both of the ranges.

\pnum
\requires
The resulting range shall not overlap with either of the original ranges.

\pnum
\returns
\tcode{make_tagged_tuple<tag::in1, tag::in2, tag::out>(last1, last2, result + $n$)}, \\ where \tcode{$n$} is
the number of elements in the constructed range.

\pnum
\complexity
At most
\tcode{2 * ((last1 - first1) + (last2 - first2)) - 1}
applications of the comparison function and projections.

\pnum
\notes If \range{first1}{last1} contains $m$ elements that are equivalent to
each other and \range{first2}{last2} contains $n$ elements that are equivalent
to them, then all $m$ elements from the first range shall be copied to the output
range, in order, and then $\max(n - m, 0)$ elements from the second range shall
be copied to the output range, in order.
\end{itemdescr}

\rSec3[std2.set.intersection]{\tcode{set_intersection}}

\indexlibrary{\idxcode{set_intersection}}%
\begin{itemdecl}
template <InputIterator I1, Sentinel<I1> S1, InputIterator I2, Sentinel<I2> S2,
    WeaklyIncrementable O, class Comp = less<>, class Proj1 = identity, class Proj2 = identity>
  requires Mergeable<I1, I2, O, Comp, Proj1, Proj2>
  O
    set_intersection(I1 first1, S1 last1, I2 first2, S2 last2, O result,
                     Comp comp = Comp{}, Proj1 proj1 = Proj1{}, Proj2 proj2 = Proj2{});

template <InputRange Rng1, InputRange Rng2, WeaklyIncrementable O,
    class Comp = less<>, class Proj1 = identity, class Proj2 = identity>
  requires Mergeable<iterator_t<Rng1>, iterator_t<Rng2>, O, Comp, Proj1, Proj2>
  O
    set_intersection(Rng1&& rng1, Rng2&& rng2, O result,
                     Comp comp = Comp{}, Proj1 proj1 = Proj1{}, Proj2 proj2 = Proj2{});
\end{itemdecl}

\begin{itemdescr}
\pnum
\effects
Constructs a sorted intersection of the elements from the two ranges;
that is, the set of elements that are present in both of the ranges.

\pnum
\requires
The resulting range shall not overlap with either of the original ranges.

\pnum
\returns
The end of the constructed range.

\pnum
\complexity
At most
\tcode{2 * ((last1 - first1) + (last2 - first2)) - 1}
applications of the comparison function and projections.

\pnum
\notes If \range{first1}{last1} contains $m$ elements that are equivalent to
each other and \range{first2}{last2} contains $n$ elements that are equivalent
to them, the first $\min(m, n)$ elements shall be copied from the first range
to the output range, in order.
\end{itemdescr}

\rSec3[std2.set.difference]{\tcode{set_difference}}

\indexlibrary{\idxcode{set_difference}}%
\begin{itemdecl}
template <InputIterator I1, Sentinel<I1> S1, InputIterator I2, Sentinel<I2> S2,
    WeaklyIncrementable O, class Comp = less<>, class Proj1 = identity, class Proj2 = identity>
  requires Mergeable<I1, I2, O, Comp, Proj1, Proj2>
  tagged_pair<tag::in1(I1), tag::out(O)>
    set_difference(I1 first1, S1 last1, I2 first2, S2 last2, O result,
                   Comp comp = Comp{}, Proj1 proj1 = Proj1{}, Proj2 proj2 = Proj2{});

template <InputRange Rng1, InputRange Rng2, WeaklyIncrementable O,
    class Comp = less<>, class Proj1 = identity, class Proj2 = identity>
  requires Mergeable<iterator_t<Rng1>, iterator_t<Rng2>, O, Comp, Proj1, Proj2>
  tagged_pair<tag::in1(safe_iterator_t<Rng1>), tag::out(O)>
    set_difference(Rng1&& rng1, Rng2&& rng2, O result,
                   Comp comp = Comp{}, Proj1 proj1 = Proj1{}, Proj2 proj2 = Proj2{});
\end{itemdecl}

\begin{itemdescr}
\pnum
\effects
Copies the elements of the range
\range{first1}{last1}
which are not present in the range
\range{first2}{last2}
to the range beginning at
\tcode{result}.
The elements in the constructed range are sorted.

\pnum
\requires
The resulting range shall not overlap with either of the original ranges.

\pnum
\returns
\tcode{\{last1, result + $n$\}}, where $n$
is the number of elements in the constructed range.

\pnum
\complexity
At most
\tcode{2 * ((last1 - first1) + (last2 - first2)) - 1}
applications of the comparison function and projections.

\pnum
\notes
If
\range{first1}{last1}
contains $m$
elements that are equivalent to each other and
\range{first2}{last2}
contains $n$
elements that are equivalent to them, the last
$\max(m - n, 0)$
elements from
\range{first1}{last1}
shall be copied to the output range.
\end{itemdescr}

\rSec3[std2.set.symmetric.difference]{\tcode{set_symmetric_difference}}

\indexlibrary{\idxcode{set_symmetric_difference}}%
\begin{itemdecl}
template <InputIterator I1, Sentinel<I1> S1, InputIterator I2, Sentinel<I2> S2,
    WeaklyIncrementable O, class Comp = less<>, class Proj1 = identity, class Proj2 = identity>
  requires Mergeable<I1, I2, O, Comp, Proj1, Proj2>
  tagged_tuple<tag::in1(I1), tag::in2(I2), tag::out(O)>
    set_symmetric_difference(I1 first1, S1 last1, I2 first2, S2 last2, O result,
                             Comp comp = Comp{}, Proj1 proj1 = Proj1{},
                             Proj2 proj2 = Proj2{});

template <InputRange Rng1, InputRange Rng2, WeaklyIncrementable O,
    class Comp = less<>, class Proj1 = identity, class Proj2 = identity>
  requires Mergeable<iterator_t<Rng1>, iterator_t<Rng2>, O, Comp, Proj1, Proj2>
  tagged_tuple<tag::in1(safe_iterator_t<Rng1>),
               tag::in2(safe_iterator_t<Rng2>),
               tag::out(O)>
    set_symmetric_difference(Rng1&& rng1, Rng2&& rng2, O result, Comp comp = Comp{},
                             Proj1 proj1 = Proj1{}, Proj2 proj2 = Proj2{});
\end{itemdecl}

\begin{itemdescr}
\pnum
\effects
Copies the elements of the range
\range{first1}{last1}
that are not present in the range
\range{first2}{last2},
and the elements of the range
\range{first2}{last2}
that are not present in the range
\range{first1}{last1}
to the range beginning at
\tcode{result}.
The elements in the constructed range are sorted.

\pnum
\requires
The resulting range shall not overlap with either of the original ranges.

\pnum
\returns
\tcode{make_tagged_tuple<tag::in1, tag::in2, tag::out>(last1, last2, result + $n$)}, \\ where \tcode{$n$} is
the number of elements in the constructed range.

\pnum
\complexity
At most
\tcode{2 * ((last1 - first1) + (last2 - first2)) - 1}
applications of the comparison function and projections.

\pnum
\notes
If \range{first1}{last1} contains $m$ elements that are equivalent to each other and
\range{first2}{last2} contains $n$ elements that are equivalent to them, then
$|m - n|$ of those elements shall be copied to the output range: the last
$m - n$ of these elements from \range{first1}{last1} if $m > n$, and the last
$n - m$ of these elements from \range{first2}{last2} if $m < n$.
\end{itemdescr}

\rSec2[std2.alg.heap.operations]{Heap operations}

\pnum
A
\techterm{heap}
is a particular organization of elements in a range between two random access iterators
\range{a}{b}.
Its two key properties are:

\begin{description}
\item{(1)} There is no element greater than
\tcode{*a}
in the range and
\item{(2)} \tcode{*a}
may be removed by
\tcode{pop_heap()},
or a new element added by
\tcode{push_heap()},
in
\bigoh{\log(N)}
time.
\end{description}

\pnum
These properties make heaps useful as priority queues.

\pnum
\tcode{make_heap()}
converts a range into a heap and
\tcode{sort_heap()}
turns a heap into a sorted sequence.

\rSec3[std2.push.heap]{\tcode{push_heap}}

\indexlibrary{\idxcode{push_heap}}%
\begin{itemdecl}
template <RandomAccessIterator I, Sentinel<I> S, class Comp = less<>,
    class Proj = identity>
  requires Sortable<I, Comp, Proj>
  I push_heap(I first, S last, Comp comp = Comp{}, Proj proj = Proj{});

template <RandomAccessRange Rng, class Comp = less<>, class Proj = identity>
  requires Sortable<iterator_t<Rng>, Comp, Proj>
  safe_iterator_t<Rng>
    push_heap(Rng&& rng, Comp comp = Comp{}, Proj proj = Proj{});
\end{itemdecl}

\begin{itemdescr}
\pnum
\effects
Places the value in the location
\tcode{last - 1}
into the resulting heap
\range{first}{last}.

\pnum
\requires
The range
\range{first}{last - 1}
shall be a valid heap.

\pnum
\returns \tcode{last}

\pnum
\complexity
At most
\tcode{log(last - first)}
applications of the comparison function and projection.
\end{itemdescr}

\rSec3[std2.pop.heap]{\tcode{pop_heap}}

\indexlibrary{\idxcode{pop_heap}}%
\begin{itemdecl}
template <RandomAccessIterator I, Sentinel<I> S, class Comp = less<>,
    class Proj = identity>
  requires Sortable<I, Comp, Proj>
  I pop_heap(I first, S last, Comp comp = Comp{}, Proj proj = Proj{});

template <RandomAccessRange Rng, class Comp = less<>, class Proj = identity>
  requires Sortable<iterator_t<Rng>, Comp, Proj>
  safe_iterator_t<Rng>
    pop_heap(Rng&& rng, Comp comp = Comp{}, Proj proj = Proj{});
\end{itemdecl}

\begin{itemdescr}
\pnum
\requires
The range
\range{first}{last}
shall be a valid non-empty heap.

\pnum
\effects
Swaps the value in the location \tcode{first}
with the value in the location
\tcode{last - 1}
and makes
\range{first}{last - 1}
into a heap.

\pnum
\returns \tcode{last}

\pnum
\complexity
At most
\tcode{2 * log(last - first)}
applications of the comparison function and projection.
\end{itemdescr}

\rSec3[std2.make.heap]{\tcode{make_heap}}

\indexlibrary{\idxcode{make_heap}}%
\begin{itemdecl}
template <RandomAccessIterator I, Sentinel<I> S, class Comp = less<>,
    class Proj = identity>
  requires Sortable<I, Comp, Proj>
  I make_heap(I first, S last, Comp comp = Comp{}, Proj proj = Proj{});

template <RandomAccessRange Rng, class Comp = less<>, class Proj = identity>
  requires Sortable<iterator_t<Rng>, Comp, Proj>
  safe_iterator_t<Rng>
    make_heap(Rng&& rng, Comp comp = Comp{}, Proj proj = Proj{});
\end{itemdecl}

\begin{itemdescr}
\pnum
\effects
Constructs a heap out of the range
\range{first}{last}.

\pnum
\returns \tcode{last}

\pnum
\complexity
At most
\tcode{3 * (last - first)}
applications of the comparison function and projection.
\end{itemdescr}

\rSec3[std2.sort.heap]{\tcode{sort_heap}}

\indexlibrary{\idxcode{sort_heap}}%
\begin{itemdecl}
template <RandomAccessIterator I, Sentinel<I> S, class Comp = less<>,
    class Proj = identity>
  requires Sortable<I, Comp, Proj>
  I sort_heap(I first, S last, Comp comp = Comp{}, Proj proj = Proj{});

template <RandomAccessRange Rng, class Comp = less<>, class Proj = identity>
  requires Sortable<iterator_t<Rng>, Comp, Proj>
  safe_iterator_t<Rng>
    sort_heap(Rng&& rng, Comp comp = Comp{}, Proj proj = Proj{});
\end{itemdecl}

\begin{itemdescr}
\pnum
\effects
Sorts elements in the heap
\range{first}{last}.

\pnum
\requires The range \range{first}{last} shall be a valid heap.

\pnum
\returns \tcode{last}

\pnum
\complexity
At most $N \log(N)$
comparisons (where
\tcode{N == last - first}), and exactly twice as many applications of the projection.
\end{itemdescr}

\rSec3[std2.is.heap]{\tcode{is_heap}}

\indexlibrary{\idxcode{is_heap}}%
\begin{itemdecl}
template <RandomAccessIterator I, Sentinel<I> S, class Proj = identity,
    IndirectStrictWeakOrder<projected<I, Proj>> Comp = less<>>
  bool is_heap(I first, S last, Comp comp = Comp{}, Proj proj = Proj{});

template <RandomAccessRange Rng, class Proj = identity,
    IndirectStrictWeakOrder<projected<iterator_t<Rng>, Proj>> Comp = less<>>
  bool
    is_heap(Rng&& rng, Comp comp = Comp{}, Proj proj = Proj{});
\end{itemdecl}

\begin{itemdescr}
\pnum
\returns \tcode{is_heap_until(first, last, comp, proj) == last}
\end{itemdescr}

\indexlibrary{\idxcode{is_heap_until}}%
\begin{itemdecl}
template <RandomAccessIterator I, Sentinel<I> S, class Proj = identity,
    IndirectStrictWeakOrder<projected<I, Proj>> Comp = less<>>
  I is_heap_until(I first, S last, Comp comp = Comp{}, Proj proj = Proj{});

template <RandomAccessRange Rng, class Proj = identity,
    IndirectStrictWeakOrder<projected<iterator_t<Rng>, Proj>> Comp = less<>>
  safe_iterator_t<Rng>
    is_heap_until(Rng&& rng, Comp comp = Comp{}, Proj proj = Proj{});
\end{itemdecl}

\begin{itemdescr}
\pnum
\returns If \tcode{distance(first, last) < 2}, returns
\tcode{last}. Otherwise, returns
the last iterator \tcode{i} in \crange{first}{last} for which the
range \range{first}{i} is a heap.

\pnum
\complexity Linear.
\end{itemdescr}

\rSec2[std2.alg.min.max]{Minimum and maximum}

\indexlibrary{\idxcode{min}}%
\begin{itemdecl}
template <class T, class Proj = identity,
    IndirectStrictWeakOrder<projected<const T*, Proj>> Comp = less<>>
  constexpr const T& min(const T& a, const T& b, Comp comp = Comp{}, Proj proj = Proj{});
\end{itemdecl}

\begin{itemdescr}
\pnum
\returns
The smaller value.

\pnum
\notes
Returns the first argument when the arguments are equivalent.
\end{itemdescr}

\indexlibrary{\idxcode{min}}%
\begin{itemdecl}
template <Copyable T, class Proj = identity,
    IndirectStrictWeakOrder<projected<const T*, Proj>> Comp = less<>>
  constexpr T min(initializer_list<T> rng, Comp comp = Comp{}, Proj proj = Proj{});

template <InputRange Rng, class Proj = identity,
    IndirectStrictWeakOrder<projected<iterator_t<Rng>, Proj>> Comp = less<>>
  requires Copyable<value_type_t<iterator_t<Rng>>>
  value_type_t<iterator_t<Rng>>
    min(Rng&& rng, Comp comp = Comp{}, Proj proj = Proj{});
\end{itemdecl}

\begin{itemdescr}
\pnum
\requires \tcode{distance(rng) > 0}.

\pnum
\returns The smallest value in the \tcode{initializer_list} or range.

\pnum
\remarks Returns a copy of the leftmost argument when several arguments are equivalent to the smallest.
\end{itemdescr}

\indexlibrary{\idxcode{max}}%
\begin{itemdecl}
template <class T, class Proj = identity,
    IndirectStrictWeakOrder<projected<const T*, Proj>> Comp = less<>>
  constexpr const T& max(const T& a, const T& b, Comp comp = Comp{}, Proj proj = Proj{});
\end{itemdecl}

\begin{itemdescr}
\pnum
\returns
The larger value.

\pnum
\notes
Returns the first argument when the arguments are equivalent.
\end{itemdescr}

\indexlibrary{\idxcode{max}}%
\begin{itemdecl}
template <Copyable T, class Proj = identity,
    IndirectStrictWeakOrder<projected<const T*, Proj>> Comp = less<>>
  constexpr T max(initializer_list<T> rng, Comp comp = Comp{}, Proj proj = Proj{});

template <InputRange Rng, class Proj = identity,
    IndirectStrictWeakOrder<projected<iterator_t<Rng>, Proj>> Comp = less<>>
  requires Copyable<value_type_t<iterator_t<Rng>>>
  value_type_t<iterator_t<Rng>>
    max(Rng&& rng, Comp comp = Comp{}, Proj proj = Proj{});
\end{itemdecl}

\begin{itemdescr}
\pnum
\requires \tcode{distance(rng) > 0}.

\pnum
\returns The largest value in the \tcode{initializer_list} or range.

\pnum
\remarks Returns a copy of the leftmost argument when several arguments are equivalent to the largest.
\end{itemdescr}

\indexlibrary{\idxcode{minmax}}%
\begin{itemdecl}
template <class T, class Proj = identity,
    IndirectStrictWeakOrder<projected<const T*, Proj>> Comp = less<>>
  constexpr tagged_pair<tag::min(const T&), tag::max(const T&)>
    minmax(const T& a, const T& b, Comp comp = Comp{}, Proj proj = Proj{});
\end{itemdecl}

\begin{itemdescr}
\pnum
\returns
\tcode{\{b, a\}} if \tcode{b} is smaller
than \tcode{a}, and
\tcode{\{a, b\}} otherwise.

\pnum
\notes
Returns \tcode{\{a, b\}} when the arguments are equivalent.

\pnum
\complexity
Exactly one comparison and exactly two applications of the projection.
\end{itemdescr}

\indexlibrary{\idxcode{minmax}}%
\begin{itemdecl}
template <Copyable T, class Proj = identity,
    IndirectStrictWeakOrder<projected<const T*, Proj>> Comp = less<>>
  constexpr tagged_pair<tag::min(T), tag::max(T)>
    minmax(initializer_list<T> rng, Comp comp = Comp{}, Proj proj = Proj{});

template <InputRange Rng, class Proj = identity,
    IndirectStrictWeakOrder<projected<iterator_t<Rng>, Proj> Comp = less<>>
  requires Copyable<value_type_t<iterator_t<Rng>>>
  tagged_pair<tag::min(value_type_t<iterator_t<Rng>>),
              tag::max(value_type_t<iterator_t<Rng>>)>
    minmax(Rng&& rng, Comp comp = Comp{}, Proj proj = Proj{});
\end{itemdecl}

\begin{itemdescr}
\pnum
\requires \tcode{distance(rng) > 0}.

\pnum
\returns \tcode{\{x, y\}}, where \tcode{x} has the smallest and \tcode{y} has the
largest value in the \tcode{initializer_list} or range.

\pnum
\remarks \tcode{x} is a copy of the leftmost argument when several arguments are equivalent to
the smallest. \tcode{y} is a copy of the rightmost argument when several arguments are
equivalent to the largest.

\pnum
\complexity At most \tcode{(3/2) * distance(rng)}
applications of the corresponding predicate, and at most twice as many applications of the projection.
\end{itemdescr}

\indexlibrary{\idxcode{min_element}}%
\begin{itemdecl}
template <ForwardIterator I, Sentinel<I> S, class Proj = identity,
    IndirectStrictWeakOrder<projected<I, Proj>> Comp = less<>>
  I min_element(I first, S last, Comp comp = Comp{}, Proj proj = Proj{});

template <ForwardRange Rng, class Proj = identity,
    IndirectStrictWeakOrder<projected<iterator_t<Rng>, Proj>> Comp = less<>>
  safe_iterator_t<Rng>
    min_element(Rng&& rng, Comp comp = Comp{}, Proj proj = Proj{});
\end{itemdecl}

\begin{itemdescr}
\pnum
\returns
The first iterator
\tcode{i}
in the range
\range{first}{last}
such that for every iterator
\tcode{j}
in the range
\range{first}{last}
the following corresponding condition holds: \\
\tcode{invoke(comp, invoke(proj, *j), invoke(\brk{}proj, *i)) == false}.
Returns
\tcode{last}
if
\tcode{first == last}.

\pnum
\complexity
Exactly
\tcode{max((last - first) - 1, 0)}
applications of the comparison function and
exactly twice as many applications of the projection.
\end{itemdescr}

\indexlibrary{\idxcode{max_element}}%
\begin{itemdecl}
template <ForwardIterator I, Sentinel<I> S, class Proj = identity,
    IndirectStrictWeakOrder<projected<I, Proj>> Comp = less<>>
  I max_element(I first, S last, Comp comp = Comp{}, Proj proj = Proj{});

template <ForwardRange Rng, class Proj = identity,
    IndirectStrictWeakOrder<projected<iterator_t<Rng>, Proj>> Comp = less<>>
  safe_iterator_t<Rng>
    max_element(Rng&& rng, Comp comp = Comp{}, Proj proj = Proj{});
\end{itemdecl}

\begin{itemdescr}
\pnum
\returns
The first iterator
\tcode{i}
in the range
\range{first}{last}
such that for every iterator
\tcode{j}
in the range
\range{first}{last}
the following corresponding condition holds: \\
\tcode{invoke(comp, invoke(proj, *i), invoke(proj, *j)) == false}.
Returns
\tcode{last}
if
\tcode{first == last}.

\pnum
\complexity
Exactly
\tcode{max((last - first) - 1, 0)}
applications of the comparison function and
exactly twice as many applications of the projection.
\end{itemdescr}

\indexlibrary{\idxcode{minmax_element}}%
\begin{itemdecl}
template <ForwardIterator I, Sentinel<I> S, class Proj = identity,
    IndirectStrictWeakOrder<projected<I, Proj>> Comp = less<>>
  tagged_pair<tag::min(I), tag::max(I)>
    minmax_element(I first, S last, Comp comp = Comp{}, Proj proj = Proj{});

template <ForwardRange Rng, class Proj = identity,
    IndirectStrictWeakOrder<projected<iterator_t<Rng>, Proj>> Comp = less<>>
  tagged_pair<tag::min(safe_iterator_t<Rng>),
              tag::max(safe_iterator_t<Rng>)>
    minmax_element(Rng&& rng, Comp comp = Comp{}, Proj proj = Proj{});
\end{itemdecl}

\begin{itemdescr}
\pnum
\returns
\tcode{\{first, first\}} if \range{first}{last} is empty, otherwise
\tcode{\{m, M\}}, where \tcode{m} is
the first iterator in \range{first}{last} such that no iterator in the range refers to a smaller
element, and where \tcode{M} is the last iterator in \range{first}{last} such that no iterator
in the range refers to a larger element.

\pnum
\complexity
At most
$max(\lfloor{\frac{3}{2}} (N-1)\rfloor, 0)$
applications of the comparison function and
at most twice as many applications of the projection,
where $N$ is \tcode{distance(first, last)}.
\end{itemdescr}

\rSec2[std2.alg.lex.comparison]{Lexicographical comparison}

\indexlibrary{\idxcode{lexicographical_compare}}%
\begin{itemdecl}
template <InputIterator I1, Sentinel<I1> S1, InputIterator I2, Sentinel<I2> S2,
    class Proj1 = identity, class Proj2 = identity,
    IndirectStrictWeakOrder<projected<I1, Proj1>, projected<I2, Proj2>> Comp = less<>>
  bool
    lexicographical_compare(I1 first1, S1 last1, I2 first2, S2 last2,
                            Comp comp = Comp{}, Proj1 proj1 = Proj1{}, Proj2 proj2 = Proj2{});

template <InputRange Rng1, InputRange Rng2, class Proj1 = identity,
    class Proj2 = identity,
    IndirectStrictWeakOrder<projected<iterator_t<Rng1>, Proj1>,
      projected<iterator_t<Rng2>, Proj2>> Comp = less<>>
  bool
    lexicographical_compare(Rng1&& rng1, Rng2&& rng2, Comp comp = Comp{},
                            Proj1 proj1 = Proj1{}, Proj2 proj2 = Proj2{});
\end{itemdecl}

\begin{itemdescr}
\pnum
\returns
\tcode{true}
if the sequence of elements defined by the range
\range{first1}{last1}
is lexicographically less than the sequence of elements defined by the range
\range{first2}{last2} and
\tcode{false}
otherwise.

\pnum
\complexity
At most
\tcode{2*min((last1 - first1), (last2 - first2))}
applications of the corresponding comparison and projections.

\pnum
\notes
If two sequences have the same number of elements and their corresponding
elements are equivalent, then neither sequence is lexicographically
less than the other.
If one sequence is a prefix of the other, then the shorter sequence is
lexicographically less than the longer sequence.
Otherwise, the lexicographical comparison of the sequences yields the same
result as the comparison of the first corresponding pair of
elements that are not equivalent.

\begin{codeblock}
for ( ; first1 != last1 && first2 != last2 ; ++first1, (void) ++first2) {
  if (invoke(comp, invoke(proj1, *first1), invoke(proj2, *first2))) return true;
  if (invoke(comp, invoke(proj2, *first2), invoke(proj1, *first1))) return false;
}
return first1 == last1 && first2 != last2;
\end{codeblock}

\pnum
\remarks An empty sequence is lexicographically less than any non-empty sequence, but
not less than any empty sequence.

\end{itemdescr}

\rSec2[std2.alg.permutation.generators]{Permutation generators}

\indexlibrary{\idxcode{next_permutation}}%
\begin{itemdecl}
template <BidirectionalIterator I, Sentinel<I> S, class Comp = less<>,
    class Proj = identity>
  requires Sortable<I, Comp, Proj>
  bool next_permutation(I first, S last, Comp comp = Comp{}, Proj proj = Proj{});

template <BidirectionalRange Rng, class Comp = less<>,
    class Proj = identity>
  requires Sortable<iterator_t<Rng>, Comp, Proj>
  bool
    next_permutation(Rng&& rng, Comp comp = Comp{}, Proj proj = Proj{});
\end{itemdecl}

\begin{itemdescr}
\pnum
\effects
Takes a sequence defined by the range
\range{first}{last}
and transforms it into the next permutation.
The next permutation is found by assuming that the set of all permutations is
lexicographically sorted with respect to
\tcode{comp} and \tcode{proj}.
If such a permutation exists, it returns
\tcode{true}.
Otherwise, it transforms the sequence into the smallest permutation,
that is, the ascendingly sorted one, and returns
\tcode{false}.

\pnum
\complexity
At most
\tcode{(last - first)/2}
swaps.
\end{itemdescr}

\indexlibrary{\idxcode{prev_permutation}}%
\begin{itemdecl}
template <BidirectionalIterator I, Sentinel<I> S, class Comp = less<>,
    class Proj = identity>
  requires Sortable<I, Comp, Proj>
  bool prev_permutation(I first, S last, Comp comp = Comp{}, Proj proj = Proj{});

template <BidirectionalRange Rng, class Comp = less<>,
    class Proj = identity>
  requires Sortable<iterator_t<Rng>, Comp, Proj>
  bool
    prev_permutation(Rng&& rng, Comp comp = Comp{}, Proj proj = Proj{});
\end{itemdecl}

\begin{itemdescr}
\pnum
\effects
Takes a sequence defined by the range
\range{first}{last}
and transforms it into the previous permutation.
The previous permutation is found by assuming that the set of all permutations is
lexicographically sorted with respect to
\tcode{comp} and \tcode{proj}.

\pnum
\returns
\tcode{true}
if such a permutation exists.
Otherwise, it transforms the sequence into the largest permutation,
that is, the descendingly sorted one, and returns
\tcode{false}.

\pnum
\complexity
At most
\tcode{(last - first)/2}
swaps.
\end{itemdescr}

} % \color{addclr}
