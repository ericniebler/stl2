%!TEX root = P0896.tex
\setcounter{chapter}{28}
\ednote{Add a new clause between [algorithm] and [numerics] with the following content:}
{\color{addclr}
\rSec0[range]{Ranges library}

\rSec1[range.general]{General}

\pnum
This clause describes components for dealing with ranges of elements.

\pnum
The following subclauses describe
range and view requirements, and
components for
range primitives
as summarized in Table~\ref{tab:ranges.lib.summary}.

\begin{libsumtab}{Ranges library summary}{tab:ranges.lib.summary}
  \ref{range.iterators}    & Iterators         & \tcode{<\oldtxt{experimental/ranges/}range>} \\
  \ref{range.access}       & Range access      & \\
  \ref{range.primitives}   & Range primitives  & \\
  \ref{range.requirements} & Requirements      & \\
  \ref{range.algorithms}   & Algorithms        & \\
\end{libsumtab}

\rSec1[range.decaycopy]{decay_copy}

\ednote{TODO: Replace the definition of [thread.decaycopy] with this definition.}

\pnum
Several places in this clause use the expression \tcode{\textit{DECAY_COPY}(x)},
which is expression-equivalent to:
\begin{codeblock}
  decay_t<decltype((x))>(x)
\end{codeblock}

\rSec1[range.synopsis]{Header \tcode{<range>} synopsis}

\indexlibrary{\idxhdr{range}}%
\begin{codeblock}
@\oldtxt{\#include <experimental/ranges/iterator>}@
#include <initializer_list>

namespace std { @\oldtxt{namespace experimental \{}@
  namespace ranges { @\oldtxt{inline namespace v1 \{}@
    template <class T> concept @\oldtxt{bool}@ @\placeholder{dereferenceable}@ // \expos
      = requires(T& t) { {*t} -> auto&&; };

    // \ref{range.iterator.requirements}, iterator requirements:
    // \ref{range.iterator.custpoints}, customization points:
    @\newtxt{inline}@ namespace @\newtxt{\unspec}@ {
      // \ref{range.iterator.custpoints.iter_move}, iter_move:
      @\newtxt{inline}@ constexpr @\unspec@ iter_move = @\unspec@;

      // \ref{range.iterator.custpoints.iter_swap}, iter_swap:
      @\newtxt{inline}@ constexpr @\unspec@ iter_swap = @\unspec@;
    }

    // \ref{range.iterator.assoc.types}, associated types:
    // \ref{range.iterator.assoc.types.difference_type}, difference_type:
    template <class> struct difference_type;
    template <class T> using difference_type_t
      = typename difference_type<T>::type;

    // \ref{range.iterator.assoc.types.value_type}, value_type:
    template <class> struct value_type;
    template <class T> using value_type_t
      = typename value_type<T>::type;

    // \ref{range.iterator.assoc.types.iterator_category}, iterator_category:
    template <class> struct iterator_category;
    template <class T> using iterator_category_t
      = typename iterator_category<T>::type;

    template <@\placeholder{dereferenceable}@ T> using reference_t
      = decltype(*declval<T&>());

    template <@\placeholder{dereferenceable}@ T>
        requires @\seebelow@ using rvalue_reference_t
      = decltype(ranges::iter_move(declval<T&>()));

    // \ref{range.iterators.readable}, Readable:
    template <class In>
    concept @\oldtxt{bool}@ Readable = @\seebelow@;

    // \ref{range.iterators.writable}, Writable:
    template <class Out, class T>
    concept @\oldtxt{bool}@ Writable = @\seebelow@;

    // \ref{range.iterators.weaklyincrementable}, WeaklyIncrementable:
    template <class I>
    concept @\oldtxt{bool}@ WeaklyIncrementable = @\seebelow@;

    // \ref{range.iterators.incrementable}, Incrementable:
    template <class I>
    concept @\oldtxt{bool}@ Incrementable = @\seebelow@;

    // \ref{range.iterators.iterator}, Iterator:
    template <class I>
    concept @\oldtxt{bool}@ Iterator = @\seebelow@;

    // \ref{range.iterators.sentinel}, Sentinel:
    template <class S, class I>
    concept @\oldtxt{bool}@ Sentinel = @\seebelow@;

    // \ref{range.iterators.sizedsentinel}, SizedSentinel:
    template <class S, class I>
    constexpr bool disable_sized_sentinel = false;

    template <class S, class I>
    concept @\oldtxt{bool}@ SizedSentinel = @\seebelow@;

    // \ref{range.iterators.input}, InputIterator:
    template <class I>
    concept @\oldtxt{bool}@ InputIterator = @\seebelow@;

    // \ref{range.iterators.output}, OutputIterator:
    template <class I>
    concept @\oldtxt{bool}@ OutputIterator = @\seebelow@;

    // \ref{range.iterators.forward}, ForwardIterator:
    template <class I>
    concept @\oldtxt{bool}@ ForwardIterator = @\seebelow@;

    // \ref{range.iterators.bidirectional}, BidirectionalIterator:
    template <class I>
    concept @\oldtxt{bool}@ BidirectionalIterator = @\seebelow@;

    // \ref{range.iterators.random.access}, RandomAccessIterator:
    template <class I>
    concept @\oldtxt{bool}@ RandomAccessIterator = @\seebelow@;

    // \ref{range.indirectcallable}, indirect callable requirements:
    // \ref{range.indirectcallable.indirectinvocable}, indirect callables:
    template <class F, class I>
    concept @\oldtxt{bool}@ IndirectUnaryInvocable = @\seebelow@;

    template <class F, class I>
    concept @\oldtxt{bool}@ IndirectRegularUnaryInvocable = @\seebelow@;

    template <class F, class I>
    concept @\oldtxt{bool}@ IndirectUnaryPredicate = @\seebelow@;

    template <class F, class I1, class I2 = I1>
    concept @\oldtxt{bool}@ IndirectRelation = @\seebelow@;

    template <class F, class I1, class I2 = I1>
    concept @\oldtxt{bool}@ IndirectStrictWeakOrder = @\seebelow@;

    template <class@\newtxt{, class...}@>
    struct indirect_result@\oldtxt{_of}@ @\newtxt{\{ \}}@;

    template <class F, class... Is>
      requires @\newtxt{(Readable<Is> \&\& ...) \&\&}@ Invocable<F, reference_t<Is>...>
    struct indirect_result@\oldtxt{_of}@<F@\oldtxt{(}\newtxt{, }@Is...@\oldtxt{)}@>;

    template <class F@\newtxt{, class... Is}@>
    using indirect_result@\oldtxt{_of}@_t
      = typename indirect_result@\oldtxt{_of}@<F@\newtxt{, Is...}@>::type;

    // \ref{range.projected}, projected:
    template <Readable I, IndirectRegularUnaryInvocable<I> Proj>
    struct projected;

    template <WeaklyIncrementable I, class Proj>
    struct difference_type<projected<I, Proj>>;

    // \ref{range.commonalgoreq}, common algorithm requirements:
    // \ref{range.commonalgoreq.indirectlymovable} IndirectlyMovable:
    template <class In, class Out>
    concept @\oldtxt{bool}@ IndirectlyMovable = @\seebelow@;

    template <class In, class Out>
    concept @\oldtxt{bool}@ IndirectlyMovableStorable = @\seebelow@;

    // \ref{range.commonalgoreq.indirectlycopyable} IndirectlyCopyable:
    template <class In, class Out>
    concept @\oldtxt{bool}@ IndirectlyCopyable = @\seebelow@;

    template <class In, class Out>
    concept @\oldtxt{bool}@ IndirectlyCopyableStorable = @\seebelow@;

    // \ref{range.commonalgoreq.indirectlyswappable} IndirectlySwappable:
    template <class I1, class I2 = I1>
    concept @\oldtxt{bool}@ IndirectlySwappable = @\seebelow@;

    // \ref{range.commonalgoreq.indirectlycomparable} IndirectlyComparable:
    template <class I1, class I2, class R = equal_to<>, class P1 = identity,
        class P2 = identity>
    concept @\oldtxt{bool}@ IndirectlyComparable = @\seebelow@;

    // \ref{range.commonalgoreq.permutable} Permutable:
    template <class I>
    concept @\oldtxt{bool}@ Permutable = @\seebelow@;

    // \ref{range.commonalgoreq.mergeable} Mergeable:
    template <class I1, class I2, class Out,
        class R = less<>, class P1 = identity, class P2 = identity>
    concept @\oldtxt{bool}@ Mergeable = @\seebelow@;

    template <class I, class R = less<>, class P = identity>
    concept @\oldtxt{bool}@ Sortable = @\seebelow@;

    // \ref{range.iterator.primitives}, primitives:
    // \ref{range.iterator.traits}, traits:
    template <class Iterator> using iterator_traits = @\seebelow@;

    template <Readable T> using iter_common_reference_t
      = common_reference_t<reference_t<T>, value_type_t<T>&>;

    // \ref{range.iterator.tags}, iterator tags:
    struct output_iterator_tag { };
    struct input_iterator_tag { };
    struct forward_iterator_tag : input_iterator_tag { };
    struct bidirectional_iterator_tag : forward_iterator_tag { };
    struct random_access_iterator_tag : bidirectional_iterator_tag { };

    // \ref{range.iterator.operations}, iterator operations:
    @\newtxt{inline}@ namespace @\newtxt{\unspec}@ {
      @\newtxt{inline}@ constexpr @\unspec@ advance = @\unspec@;
      @\newtxt{inline}@ constexpr @\unspec@ distance = @\unspec@;
      @\newtxt{inline}@ constexpr @\unspec@ next = @\unspec@;
      @\newtxt{inline}@ constexpr @\unspec@ prev = @\unspec@;
    }

    // \ref{range.iterators.predef}, predefined iterators and sentinels:

    // \ref{range.iterators.reverse}, reverse iterators:
    template <BidirectionalIterator I> class reverse_iterator;

    template <class I1, class I2>
        requires EqualityComparableWith<I1, I2>
      constexpr bool operator==(
        const reverse_iterator<I1>& x,
        const reverse_iterator<I2>& y);
    template <class I1, class I2>
        requires EqualityComparableWith<I1, I2>
      constexpr bool operator!=(
        const reverse_iterator<I1>& x,
        const reverse_iterator<I2>& y);
    template <class I1, class I2>
        requires StrictTotallyOrderedWith<I1, I2>
      constexpr bool operator<(
        const reverse_iterator<I1>& x,
        const reverse_iterator<I2>& y);
    template <class I1, class I2>
        requires StrictTotallyOrderedWith<I1, I2>
      constexpr bool operator>(
        const reverse_iterator<I1>& x,
        const reverse_iterator<I2>& y);
    template <class I1, class I2>
        requires StrictTotallyOrderedWith<I1, I2>
      constexpr bool operator>=(
        const reverse_iterator<I1>& x,
        const reverse_iterator<I2>& y);
    template <class I1, class I2>
        requires StrictTotallyOrderedWith<I1, I2>
      constexpr bool operator<=(
        const reverse_iterator<I1>& x,
        const reverse_iterator<I2>& y);

    template <class I1, class I2>
        requires SizedSentinel<I1, I2>
      constexpr difference_type_t<I2> operator-(
        const reverse_iterator<I1>& x,
        const reverse_iterator<I2>& y);
    template <RandomAccessIterator I>
      constexpr reverse_iterator<I> operator+(
        difference_type_t<I> n,
        const reverse_iterator<I>& x);

    template <BidirectionalIterator I>
      constexpr reverse_iterator<I> make_reverse_iterator(I i);

    // \ref{range.iterators.insert}, insert iterators:
    template <class Container> class back_insert_iterator;
    template <class Container>
      back_insert_iterator<Container> back_inserter(Container& x);

    template <class Container> class front_insert_iterator;
    template <class Container>
      front_insert_iterator<Container> front_inserter(Container& x);

    template <class Container> class insert_iterator;
    template <class Container>
      insert_iterator<Container> inserter(Container& x, iterator_t<Container> i);

    // \ref{range.iterators.move}, move iterators and sentinels:
    template <InputIterator I> class move_iterator;
    template <class I1, class I2>
        requires EqualityComparableWith<I1, I2>
      constexpr bool operator==(
        const move_iterator<I1>& x, const move_iterator<I2>& y);
    template <class I1, class I2>
        requires EqualityComparableWith<I1, I2>
      constexpr bool operator!=(
        const move_iterator<I1>& x, const move_iterator<I2>& y);
    template <class I1, class I2>
        requires StrictTotallyOrderedWith<I1, I2>
      constexpr bool operator<(
        const move_iterator<I1>& x, const move_iterator<I2>& y);
    template <class I1, class I2>
        requires StrictTotallyOrderedWith<I1, I2>
      constexpr bool operator<=(
        const move_iterator<I1>& x, const move_iterator<I2>& y);
    template <class I1, class I2>
        requires StrictTotallyOrderedWith<I1, I2>
      constexpr bool operator>(
        const move_iterator<I1>& x, const move_iterator<I2>& y);
    template <class I1, class I2>
        requires StrictTotallyOrderedWith<I1, I2>
      constexpr bool operator>=(
        const move_iterator<I1>& x, const move_iterator<I2>& y);

    template <class I1, class I2>
        requires SizedSentinel<I1, I2>
      constexpr difference_type_t<I2> operator-(
        const move_iterator<I1>& x,
        const move_iterator<I2>& y);
    template <RandomAccessIterator I>
      constexpr move_iterator<I> operator+(
        difference_type_t<I> n,
        const move_iterator<I>& x);
    template <InputIterator I>
      constexpr move_iterator<I> make_move_iterator(I i);

    template <Semiregular S> class move_sentinel;

    template <class I, Sentinel<I> S>
      constexpr bool operator==(
        const move_iterator<I>& i, const move_sentinel<S>& s);
    template <class I, Sentinel<I> S>
      constexpr bool operator==(
        const move_sentinel<S>& s, const move_iterator<I>& i);
    template <class I, Sentinel<I> S>
      constexpr bool operator!=(
        const move_iterator<I>& i, const move_sentinel<S>& s);
    template <class I, Sentinel<I> S>
      constexpr bool operator!=(
        const move_sentinel<S>& s, const move_iterator<I>& i);

    template <class I, SizedSentinel<I> S>
      constexpr difference_type_t<I> operator-(
        const move_sentinel<S>& s, const move_iterator<I>& i);
    template <class I, SizedSentinel<I> S>
      constexpr difference_type_t<I> operator-(
        const move_iterator<I>& i, const move_sentinel<S>& s);

    template <Semiregular S>
      constexpr move_sentinel<S> make_move_sentinel(S s);

    // \ref{range.iterators.common}, common iterators:
    template <Iterator I, Sentinel<I> S>
      requires !Same<I, S>
    class common_iterator;

    template <Readable I, class S>
    struct value_type<common_iterator<I, S>>;

    template <InputIterator I, class S>
    struct iterator_category<common_iterator<I, S>>;

    template <ForwardIterator I, class S>
    struct iterator_category<common_iterator<I, S>>;

    template <class I1, class I2, Sentinel<I2> S1, Sentinel<I1> S2>
    bool operator==(
      const common_iterator<I1, S1>& x, const common_iterator<I2, S2>& y);
    template <class I1, class I2, Sentinel<I2> S1, Sentinel<I1> S2>
      requires EqualityComparableWith<I1, I2>
    bool operator==(
      const common_iterator<I1, S1>& x, const common_iterator<I2, S2>& y);
    template <class I1, class I2, Sentinel<I2> S1, Sentinel<I1> S2>
    bool operator!=(
      const common_iterator<I1, S1>& x, const common_iterator<I2, S2>& y);

    template <class I2, SizedSentinel<I2> I1, SizedSentinel<I2> S1, SizedSentinel<I1> S2>
    difference_type_t<I2> operator-(
      const common_iterator<I1, S1>& x, const common_iterator<I2, S2>& y);

    // \ref{range.default.sentinels}, default sentinels:
    class default_sentinel;

    // \ref{range.iterators.counted}, counted iterators:
    template <Iterator I> class counted_iterator;

    template <class I1, class I2>
        requires Common<I1, I2>
      constexpr bool operator==(
        const counted_iterator<I1>& x, const counted_iterator<I2>& y);
    @\newtxt{template <class I>}@
      constexpr bool operator==(
        const counted_iterator<@\oldtxt{auto}\newtxt{I}@>& x, default_sentinel);
    @\newtxt{template <class I>}@
      constexpr bool operator==(
        default_sentinel, const counted_iterator<@\oldtxt{auto}\newtxt{I}@>& x);
    template <class I1, class I2>
        requires Common<I1, I2>
      constexpr bool operator!=(
        const counted_iterator<I1>& x, const counted_iterator<I2>& y);
    @\newtxt{template <class I>}@
      constexpr bool operator!=(
        const counted_iterator<@\oldtxt{auto}\newtxt{I}@>& x, default_sentinel y);
    @\newtxt{template <class I>}@
      constexpr bool operator!=(
        default_sentinel x, const counted_iterator<@\oldtxt{auto}\newtxt{I}@>& y);
    template <class I1, class I2>
        requires Common<I1, I2>
      constexpr bool operator<(
        const counted_iterator<I1>& x, const counted_iterator<I2>& y);
    template <class I1, class I2>
        requires Common<I1, I2>
      constexpr bool operator<=(
        const counted_iterator<I1>& x, const counted_iterator<I2>& y);
    template <class I1, class I2>
        requires Common<I1, I2>
      constexpr bool operator>(
        const counted_iterator<I1>& x, const counted_iterator<I2>& y);
    template <class I1, class I2>
        requires Common<I1, I2>
      constexpr bool operator>=(
        const counted_iterator<I1>& x, const counted_iterator<I2>& y);
    template <class I1, class I2>
        requires Common<I1, I2>
      constexpr difference_type_t<I2> operator-(
        const counted_iterator<I1>& x, const counted_iterator<I2>& y);
    template <class I>
      constexpr difference_type_t<I> operator-(
        const counted_iterator<I>& x, default_sentinel y);
    template <class I>
      constexpr difference_type_t<I> operator-(
        default_sentinel x, const counted_iterator<I>& y);
    template <RandomAccessIterator I>
      constexpr counted_iterator<I>
        operator+(difference_type_t<I> n, const counted_iterator<I>& x);
    template <Iterator I>
      constexpr counted_iterator<I> make_counted_iterator(I i, difference_type_t<I> n);

    // \ref{range.unreachable.sentinels}, unreachable sentinels:
    class unreachable;
    template <Iterator I>
      constexpr bool operator==(const I&, unreachable) noexcept;
    template <Iterator I>
      constexpr bool operator==(unreachable, const I&) noexcept;
    template <Iterator I>
      constexpr bool operator!=(const I&, unreachable) noexcept;
    template <Iterator I>
      constexpr bool operator!=(unreachable, const I&) noexcept;

    // \ref{range.dangling.wrap}, dangling wrapper:
    template <class T> class dangling;
    template <Range R> using safe_iterator_t = @\seebelow@;

    // \ref{range.iterators.stream}, stream iterators:
    template <class T, class charT = char, class traits = char_traits<charT>,
        class Distance = ptrdiff_t>
    class istream_iterator;
    template <class T, class charT, class traits, class Distance>
      bool operator==(const istream_iterator<T, charT, traits, Distance>& x,
              const istream_iterator<T, charT, traits, Distance>& y);
    template <class T, class charT, class traits, class Distance>
      bool operator==(default_sentinel x,
              const istream_iterator<T, charT, traits, Distance>& y);
    template <class T, class charT, class traits, class Distance>
      bool operator==(const istream_iterator<T, charT, traits, Distance>& x,
              default_sentinel y);
    template <class T, class charT, class traits, class Distance>
      bool operator!=(const istream_iterator<T, charT, traits, Distance>& x,
              const istream_iterator<T, charT, traits, Distance>& y);
    template <class T, class charT, class traits, class Distance>
    bool operator!=(default_sentinel x,
              const istream_iterator<T, charT, traits, Distance>& y);
    template <class T, class charT, class traits, class Distance>
      bool operator!=(const istream_iterator<T, charT, traits, Distance>& x,
              default_sentinel y);

    template <class T, class charT = char, class traits = char_traits<charT>>
        class ostream_iterator;

    template <class charT, class traits = char_traits<charT> >
      class istreambuf_iterator;
    template <class charT, class traits>
      bool operator==(const istreambuf_iterator<charT, traits>& a,
              const istreambuf_iterator<charT, traits>& b);
    template <class charT, class traits>
      bool operator==(default_sentinel a,
              const istreambuf_iterator<charT, traits>& b);
    template <class charT, class traits>
      bool operator==(const istreambuf_iterator<charT, traits>& a,
              default_sentinel b);
    template <class charT, class traits>
      bool operator!=(const istreambuf_iterator<charT, traits>& a,
              const istreambuf_iterator<charT, traits>& b);
    template <class charT, class traits>
      bool operator!=(default_sentinel a,
              const istreambuf_iterator<charT, traits>& b);
    template <class charT, class traits>
      bool operator!=(const istreambuf_iterator<charT, traits>& a,
              default_sentinel b);

    template <class charT, class traits = char_traits<charT> >
      class ostreambuf_iterator;
  }

  // \ref{range.iterator.stdtraits}, iterator traits:
  template <@\oldtxt{experimental::}@ranges::Iterator Out>
    struct iterator_traits<Out>;
  template <@\oldtxt{experimental::}@ranges::InputIterator In>
    struct iterator_traits<In>;
  template <@\oldtxt{experimental::}@ranges::InputIterator In>
      requires @\oldtxt{experimental::}@ranges::Sentinel<In, In>
    struct iterator_traits;

  namespace ranges {
    @\newtxt{inline}@ namespace @\newtxt{\unspec}@ {
      // \ref{range.access}, range access:
      @\newtxt{inline}@ constexpr @\unspec@ begin = @\unspec@;
      @\newtxt{inline}@ constexpr @\unspec@ end = @\unspec@;
      @\newtxt{inline}@ constexpr @\unspec@ cbegin = @\unspec@;
      @\newtxt{inline}@ constexpr @\unspec@ cend = @\unspec@;
      @\newtxt{inline}@ constexpr @\unspec@ rbegin = @\unspec@;
      @\newtxt{inline}@ constexpr @\unspec@ rend = @\unspec@;
      @\newtxt{inline}@ constexpr @\unspec@ crbegin = @\unspec@;
      @\newtxt{inline}@ constexpr @\unspec@ crend = @\unspec@;

      // \ref{range.primitives}, range primitives:
      @\newtxt{inline}@ constexpr @\unspec@ size = @\unspec@;
      @\newtxt{inline}@ constexpr @\unspec@ empty = @\unspec@;
      @\newtxt{inline}@ constexpr @\unspec@ data = @\unspec@;
      @\newtxt{inline}@ constexpr @\unspec@ cdata = @\unspec@;
    }

    template <class T>
    using iterator_t = decltype(ranges::begin(declval<T&>()));

    template <class T>
    using sentinel_t = decltype(ranges::end(declval<T&>()));

    template <class>
    constexpr bool disable_sized_range = false;

    template <class T>
    struct enable_view { };

    struct view_base { };

    // \ref{range.requirements}, range requirements:

    // \ref{range.range}, Range:
    template <class T>
    concept @\oldtxt{bool}@ Range = @\seebelow@;

    // \ref{range.sized}, SizedRange:
    template <class T>
    concept @\oldtxt{bool}@ SizedRange = @\seebelow@;

    // \ref{range.view}, View:
    template <class T>
    concept @\oldtxt{bool}@ View = @\seebelow@;

    // \ref{range.common}, \oldtxt{BoundedRange}\newtxt{CommonRange}:
    template <class T>
    concept @\oldtxt{bool}@ @\oldtxt{BoundedRange}\newtxt{CommonRange}@ = @\seebelow@;

    // \ref{range.input}, InputRange:
    template <class T>
    concept @\oldtxt{bool}@ InputRange = @\seebelow@;

    // \ref{range.output}, OutputRange:
    template <class R, class T>
    concept @\oldtxt{bool}@ OutputRange = @\seebelow@;

    // \ref{range.forward}, ForwardRange:
    template <class T>
    concept @\oldtxt{bool}@ ForwardRange = @\seebelow@;

    // \ref{range.bidirectional}, BidirectionalRange:
    template <class T>
    concept @\oldtxt{bool}@ BidirectionalRange = @\seebelow@;

    // \ref{range.random.access}, RandomAccessRange:
    template <class T>
    concept @\oldtxt{bool}@ RandomAccessRange = @\seebelow@;

    // \ref{range.alg.nonmodifying}, non-modifying sequence operations:
    template <InputIterator I, Sentinel<I> S, class Proj = identity,
        IndirectUnaryPredicate<projected<I, Proj>> Pred>
      bool all_of(I first, S last, Pred pred, Proj proj = Proj{});

    template <InputRange Rng, class Proj = identity,
        IndirectUnaryPredicate<projected<iterator_t<Rng>, Proj>> Pred>
      bool all_of(Rng&& rng, Pred pred, Proj proj = Proj{});

    template <InputIterator I, Sentinel<I> S, class Proj = identity,
        IndirectUnaryPredicate<projected<I, Proj>> Pred>
      bool any_of(I first, S last, Pred pred, Proj proj = Proj{});

    template <InputRange Rng, class Proj = identity,
        IndirectUnaryPredicate<projected<iterator_t<Rng>, Proj>> Pred>
      bool any_of(Rng&& rng, Pred pred, Proj proj = Proj{});

    template <InputIterator I, Sentinel<I> S, class Proj = identity,
        IndirectUnaryPredicate<projected<I, Proj>> Pred>
      bool none_of(I first, S last, Pred pred, Proj proj = Proj{});

    template <InputRange Rng, class Proj = identity,
        IndirectUnaryPredicate<projected<iterator_t<Rng>, Proj>> Pred>
      bool none_of(Rng&& rng, Pred pred, Proj proj = Proj{});

    template <InputIterator I, Sentinel<I> S, class Proj = identity,
        IndirectUnaryInvocable<projected<I, Proj>> Fun>
      tagged_pair<tag::in(I), tag::fun(Fun)>
        for_each(I first, S last, Fun f, Proj proj = Proj{});

    template <InputRange Rng, class Proj = identity,
        IndirectUnaryInvocable<projected<iterator_t<Rng>, Proj>> Fun>
      tagged_pair<tag::in(safe_iterator_t<Rng>), tag::fun(Fun)>
        for_each(Rng&& rng, Fun f, Proj proj = Proj{});

    template <InputIterator I, Sentinel<I> S, class T, class Proj = identity>
      requires IndirectRelation<equal_to<>, projected<I, Proj>, const T*>
      I find(I first, S last, const T& value, Proj proj = Proj{});

    template <InputRange Rng, class T, class Proj = identity>
      requires IndirectRelation<equal_to<>, projected<iterator_t<Rng>, Proj>, const T*>
      safe_iterator_t<Rng>
        find(Rng&& rng, const T& value, Proj proj = Proj{});

    template <InputIterator I, Sentinel<I> S, class Proj = identity,
        IndirectUnaryPredicate<projected<I, Proj>> Pred>
      I find_if(I first, S last, Pred pred, Proj proj = Proj{});

    template <InputRange Rng, class Proj = identity,
        IndirectUnaryPredicate<projected<iterator_t<Rng>, Proj>> Pred>
      safe_iterator_t<Rng>
        find_if(Rng&& rng, Pred pred, Proj proj = Proj{});

    template <InputIterator I, Sentinel<I> S, class Proj = identity,
        IndirectUnaryPredicate<projected<I, Proj>> Pred>
      I find_if_not(I first, S last, Pred pred, Proj proj = Proj{});

    template <InputRange Rng, class Proj = identity,
        IndirectUnaryPredicate<projected<iterator_t<Rng>, Proj>> Pred>
      safe_iterator_t<Rng>
        find_if_not(Rng&& rng, Pred pred, Proj proj = Proj{});

    template <ForwardIterator I1, Sentinel<I1> S1, ForwardIterator I2,
        Sentinel<I2> S2, class Proj = identity,
        IndirectRelation<I2, projected<I1, Proj>> Pred = equal_to<>>
      I1
        find_end(I1 first1, S1 last1, I2 first2, S2 last2,
                Pred pred = Pred{}, Proj proj = Proj{});

    template <ForwardRange Rng1, ForwardRange Rng2, class Proj = identity,
        IndirectRelation<iterator_t<Rng2>,
          projected<iterator_t<Rng>, Proj>> Pred = equal_to<>>
      safe_iterator_t<Rng1>
        find_end(Rng1&& rng1, Rng2&& rng2, Pred pred = Pred{}, Proj proj = Proj{});

    template <InputIterator I1, Sentinel<I1> S1, ForwardIterator I2, Sentinel<I2> S2,
        class Proj1 = identity, class Proj2 = identity,
        IndirectRelation<projected<I1, Proj1>, projected<I2, Proj2>> Pred = equal_to<>>
      I1
        find_first_of(I1 first1, S1 last1, I2 first2, S2 last2,
                      Pred pred = Pred{},
                      Proj1 proj1 = Proj1{}, Proj2 proj2 = Proj2{});

    template <InputRange Rng1, ForwardRange Rng2, class Proj1 = identity,
        class Proj2 = identity,
        IndirectRelation<projected<iterator_t<Rng1>, Proj1>,
          projected<iterator_t<Rng2>, Proj2>> Pred = equal_to<>>
      safe_iterator_t<Rng1>
        find_first_of(Rng1&& rng1, Rng2&& rng2,
                      Pred pred = Pred{},
                      Proj1 proj1 = Proj1{}, Proj2 proj2 = Proj2{});

    template <ForwardIterator I, Sentinel<I> S, class Proj = identity,
        IndirectRelation<projected<I, Proj>> Pred = equal_to<>>
      I adjacent_find(I first, S last, Pred pred = Pred{},
                      Proj proj = Proj{});

    template <ForwardRange Rng, class Proj = identity,
        IndirectRelation<projected<iterator_t<Rng>, Proj>> Pred = equal_to<>>
      safe_iterator_t<Rng>
        adjacent_find(Rng&& rng, Pred pred = Pred{}, Proj proj = Proj{});

    template <InputIterator I, Sentinel<I> S, class T, class Proj = identity>
      requires IndirectRelation<equal_to<>, projected<I, Proj>, const T*>
      difference_type_t<I>
        count(I first, S last, const T& value, Proj proj = Proj{});

    template <InputRange Rng, class T, class Proj = identity>
      requires IndirectRelation<equal_to<>, projected<iterator_t<Rng>, Proj>, const T*>
      difference_type_t<iterator_t<Rng>>
        count(Rng&& rng, const T& value, Proj proj = Proj{});

    template <InputIterator I, Sentinel<I> S, class Proj = identity,
        IndirectUnaryPredicate<projected<I, Proj>> Pred>
      difference_type_t<I>
        count_if(I first, S last, Pred pred, Proj proj = Proj{});

    template <InputRange Rng, class Proj = identity,
        IndirectUnaryPredicate<projected<iterator_t<Rng>, Proj>> Pred>
      difference_type_t<iterator_t<Rng>>
        count_if(Rng&& rng, Pred pred, Proj proj = Proj{});

    template <InputIterator I1, Sentinel<I1> S1, InputIterator I2, Sentinel<I2> S2,
        class Proj1 = identity, class Proj2 = identity,
        IndirectRelation<projected<I1, Proj1>, projected<I2, Proj2>> Pred = equal_to<>>
      tagged_pair<tag::in1(I1), tag::in2(I2)>
        mismatch(I1 first1, S1 last1, I2 first2, S2 last2, Pred pred = Pred{},
                Proj1 proj1 = Proj1{}, Proj2 proj2 = Proj2{});

    template <InputRange Rng1, InputRange Rng2,
        class Proj1 = identity, class Proj2 = identity,
        IndirectRelation<projected<iterator_t<Rng1>, Proj1>,
          projected<iterator_t<Rng2>, Proj2>> Pred = equal_to<>>
      tagged_pair<tag::in1(safe_iterator_t<Rng1>),
                  tag::in2(safe_iterator_t<Rng2>)>
        mismatch(Rng1&& rng1, Rng2&& rng2, Pred pred = Pred{},
                Proj1 proj1 = Proj1{}, Proj2 proj2 = Proj2{});

    template <InputIterator I1, Sentinel<I1> S1, InputIterator I2, Sentinel<I2> S2,
        class Pred = equal_to<>, class Proj1 = identity, class Proj2 = identity>
      requires IndirectlyComparable<I1, I2, Pred, Proj1, Proj2>
      bool equal(I1 first1, S1 last1, I2 first2, S2 last2,
                Pred pred = Pred{},
                Proj1 proj1 = Proj1{}, Proj2 proj2 = Proj2{});

    template <InputRange Rng1, InputRange Rng2, class Pred = equal_to<>,
        class Proj1 = identity, class Proj2 = identity>
      requires IndirectlyComparable<iterator_t<Rng1>, iterator_t<Rng2>, Pred, Proj1, Proj2>
      bool equal(Rng1&& rng1, Rng2&& rng2, Pred pred = Pred{},
                Proj1 proj1 = Proj1{}, Proj2 proj2 = Proj2{});

    template <ForwardIterator I1, Sentinel<I1> S1, ForwardIterator I2,
        Sentinel<I2> S2, class Pred = equal_to<>, class Proj1 = identity,
        class Proj2 = identity>
      requires IndirectlyComparable<I1, I2, Pred, Proj1, Proj2>
      bool is_permutation(I1 first1, S1 last1, I2 first2, S2 last2,
                          Pred pred = Pred{},
                          Proj1 proj1 = Proj1{}, Proj2 proj2 = Proj2{});

    template <ForwardRange Rng1, ForwardRange Rng2, class Pred = equal_to<>,
        class Proj1 = identity, class Proj2 = identity>
      requires IndirectlyComparable<iterator_t<Rng1>, iterator_t<Rng2>, Pred, Proj1, Proj2>
      bool is_permutation(Rng1&& rng1, Rng2&& rng2, Pred pred = Pred{},
                          Proj1 proj1 = Proj1{}, Proj2 proj2 = Proj2{});

    template <ForwardIterator I1, Sentinel<I1> S1, ForwardIterator I2,
        Sentinel<I2> S2, class Pred = equal_to<>,
        class Proj1 = identity, class Proj2 = identity>
      requires IndirectlyComparable<I1, I2, Pred, Proj1, Proj2>
      I1 search(I1 first1, S1 last1, I2 first2, S2 last2,
                Pred pred = Pred{},
                Proj1 proj1 = Proj1{}, Proj2 proj2 = Proj2{});

    template <ForwardRange Rng1, ForwardRange Rng2, class Pred = equal_to<>,
        class Proj1 = identity, class Proj2 = identity>
      requires IndirectlyComparable<iterator_t<Rng1>, iterator_t<Rng2>, Pred, Proj1, Proj2>
      safe_iterator_t<Rng1>
        search(Rng1&& rng1, Rng2&& rng2, Pred pred = Pred{},
              Proj1 proj1 = Proj1{}, Proj2 proj2 = Proj2{});

    template <ForwardIterator I, Sentinel<I> S, class T,
        class Pred = equal_to<>, class Proj = identity>
      requires IndirectlyComparable<I, const T*, Pred, Proj>
      I search_n(I first, S last, difference_type_t<I> count,
                const T& value, Pred pred = Pred{},
                Proj proj = Proj{});

    template <ForwardRange Rng, class T, class Pred = equal_to<>,
        class Proj = identity>
      requires IndirectlyComparable<iterator_t<Rng>, const T*, Pred, Proj>
      safe_iterator_t<Rng>
        search_n(Rng&& rng, difference_type_t<iterator_t<Rng>> count,
                const T& value, Pred pred = Pred{}, Proj proj = Proj{});

    // \ref{range.alg.modifying.operations}, modifying sequence operations:
    // \ref{range.alg.copy}, copy:
    template <InputIterator I, Sentinel<I> S, WeaklyIncrementable O>
      requires IndirectlyCopyable<I, O>
      tagged_pair<tag::in(I), tag::out(O)>
        copy(I first, S last, O result);

    template <InputRange Rng, WeaklyIncrementable O>
      requires IndirectlyCopyable<iterator_t<Rng>, O>
      tagged_pair<tag::in(safe_iterator_t<Rng>), tag::out(O)>
        copy(Rng&& rng, O result);

    template <InputIterator I, WeaklyIncrementable O>
      requires IndirectlyCopyable<I, O>
      tagged_pair<tag::in(I), tag::out(O)>
        copy_n(I first, difference_type_t<I> n, O result);

    template <InputIterator I, Sentinel<I> S, WeaklyIncrementable O, class Proj = identity,
        IndirectUnaryPredicate<projected<I, Proj>> Pred>
      requires IndirectlyCopyable<I, O>
      tagged_pair<tag::in(I), tag::out(O)>
        copy_if(I first, S last, O result, Pred pred, Proj proj = Proj{});

    template <InputRange Rng, WeaklyIncrementable O, class Proj = identity,
        IndirectUnaryPredicate<projected<iterator_t<Rng>, Proj>> Pred>
      requires IndirectlyCopyable<iterator_t<Rng>, O>
      tagged_pair<tag::in(safe_iterator_t<Rng>), tag::out(O)>
        copy_if(Rng&& rng, O result, Pred pred, Proj proj = Proj{});

    template <BidirectionalIterator I1, Sentinel<I1> S1, BidirectionalIterator I2>
      requires IndirectlyCopyable<I1, I2>
      tagged_pair<tag::in(I1), tag::out(I2)>
        copy_backward(I1 first, S1 last, I2 result);

    template <BidirectionalRange Rng, BidirectionalIterator I>
      requires IndirectlyCopyable<iterator_t<Rng>, I>
      tagged_pair<tag::in(safe_iterator_t<Rng>), tag::out(I)>
        copy_backward(Rng&& rng, I result);

    // \ref{range.alg.move}, move:
    template <InputIterator I, Sentinel<I> S, WeaklyIncrementable O>
      requires IndirectlyMovable<I, O>
      tagged_pair<tag::in(I), tag::out(O)>
        move(I first, S last, O result);

    template <InputRange Rng, WeaklyIncrementable O>
      requires IndirectlyMovable<iterator_t<Rng>, O>
      tagged_pair<tag::in(safe_iterator_t<Rng>), tag::out(O)>
        move(Rng&& rng, O result);

    template <BidirectionalIterator I1, Sentinel<I1> S1, BidirectionalIterator I2>
      requires IndirectlyMovable<I1, I2>
      tagged_pair<tag::in(I1), tag::out(I2)>
        move_backward(I1 first, S1 last, I2 result);

    template <BidirectionalRange Rng, BidirectionalIterator I>
      requires IndirectlyMovable<iterator_t<Rng>, I>
      tagged_pair<tag::in(safe_iterator_t<Rng>), tag::out(I)>
        move_backward(Rng&& rng, I result);

    template <ForwardIterator I1, Sentinel<I1> S1, ForwardIterator I2, Sentinel<I2> S2>
      requires IndirectlySwappable<I1, I2>
      tagged_pair<tag::in1(I1), tag::in2(I2)>
        swap_ranges(I1 first1, S1 last1, I2 first2, S2 last2);

    template <ForwardRange Rng1, ForwardRange Rng2>
      requires IndirectlySwappable<iterator_t<Rng1>, iterator_t<Rng2>>
      tagged_pair<tag::in1(safe_iterator_t<Rng1>), tag::in2(safe_iterator_t<Rng2>)>
        swap_ranges(Rng1&& rng1, Rng2&& rng2);

    template <InputIterator I, Sentinel<I> S, WeaklyIncrementable O,
        CopyConstructible F, class Proj = identity>
      requires Writable<O, indirect_result@\oldtxt{_of}@_t<F&@\oldtxt{(}\newtxt{, }@projected<I, Proj>@\oldtxt{)}@>>
      tagged_pair<tag::in(I), tag::out(O)>
        transform(I first, S last, O result, F op, Proj proj = Proj{});

    template <InputRange Rng, WeaklyIncrementable O, CopyConstructible F,
        class Proj = identity>
      requires Writable<O, indirect_result@\oldtxt{_of}@_t<F&@\oldtxt{(}\newtxt{,}@
        projected<iterator_t<R>, Proj>@\oldtxt{)}@>>
      tagged_pair<tag::in(safe_iterator_t<Rng>), tag::out(O)>
        transform(Rng&& rng, O result, F op, Proj proj = Proj{});

    template <InputIterator I1, Sentinel<I1> S1, InputIterator I2, Sentinel<I2> S2,
        WeaklyIncrementable O, CopyConstructible F, class Proj1 = identity,
        class Proj2 = identity>
      requires Writable<O, indirect_result@\oldtxt{_of}@_t<F&@\oldtxt{(}\newtxt{, }@projected<I1, Proj1>,
        projected<I2, Proj2>@\oldtxt{)}@>>
      tagged_tuple<tag::in1(I1), tag::in2(I2), tag::out(O)>
        transform(I1 first1, S1 last1, I2 first2, S2 last2, O result,
                F binary_op, Proj1 proj1 = Proj1{}, Proj2 proj2 = Proj2{});

    template <InputRange Rng1, InputRange Rng2, WeaklyIncrementable O,
        CopyConstructible F, class Proj1 = identity, class Proj2 = identity>
      requires Writable<O, indirect_result@\oldtxt{_of}@_t<F&@\oldtxt{(}\newtxt{,}@
        projected<iterator_t<Rng1>, Proj1>, projected<iterator_t<Rng2>, Proj2>@\oldtxt{)}@>>
      tagged_tuple<tag::in1(safe_iterator_t<Rng1>),
                   tag::in2(safe_iterator_t<Rng2>),
                   tag::out(O)>
        transform(Rng1&& rng1, Rng2&& rng2, O result,
                  F binary_op, Proj1 proj1 = Proj1{}, Proj2 proj2 = Proj2{});

    template <InputIterator I, Sentinel<I> S, class T1, class T2, class Proj = identity>
      requires Writable<I, const T2&> &&
        IndirectRelation<equal_to<>, projected<I, Proj>, const T1*>
      I replace(I first, S last, const T1& old_value, const T2& new_value, Proj proj = Proj{});

    template <InputRange Rng, class T1, class T2, class Proj = identity>
      requires Writable<iterator_t<Rng>, const T2&> &&
        IndirectRelation<equal_to<>, projected<iterator_t<Rng>, Proj>, const T1*>
      safe_iterator_t<Rng>
        replace(Rng&& rng, const T1& old_value, const T2& new_value, Proj proj = Proj{});

    template <InputIterator I, Sentinel<I> S, class T, class Proj = identity,
        IndirectUnaryPredicate<projected<I, Proj>> Pred>
      requires Writable<I, const T&>
      I replace_if(I first, S last, Pred pred, const T& new_value, Proj proj = Proj{});

    template <InputRange Rng, class T, class Proj = identity,
        IndirectUnaryPredicate<projected<iterator_t<Rng>, Proj>> Pred>
      requires Writable<iterator_t<Rng>, const T&>
      safe_iterator_t<Rng>
        replace_if(Rng&& rng, Pred pred, const T& new_value, Proj proj = Proj{});

    template <InputIterator I, Sentinel<I> S, class T1, class T2, OutputIterator<const T2&> O,
        class Proj = identity>
      requires IndirectlyCopyable<I, O> &&
        IndirectRelation<equal_to<>, projected<I, Proj>, const T1*>
      tagged_pair<tag::in(I), tag::out(O)>
        replace_copy(I first, S last, O result, const T1& old_value, const T2& new_value,
                    Proj proj = Proj{});

    template <InputRange Rng, class T1, class T2, OutputIterator<const T2&> O,
        class Proj = identity>
      requires IndirectlyCopyable<iterator_t<Rng>, O> &&
        IndirectRelation<equal_to<>, projected<iterator_t<Rng>, Proj>, const T1*>
      tagged_pair<tag::in(safe_iterator_t<Rng>), tag::out(O)>
        replace_copy(Rng&& rng, O result, const T1& old_value, const T2& new_value,
                    Proj proj = Proj{});

    template <InputIterator I, Sentinel<I> S, class T, OutputIterator<const T&> O,
        class Proj = identity, IndirectUnaryPredicate<projected<I, Proj>> Pred>
      requires IndirectlyCopyable<I, O>
      tagged_pair<tag::in(I), tag::out(O)>
        replace_copy_if(I first, S last, O result, Pred pred, const T& new_value,
                        Proj proj = Proj{});

    template <InputRange Rng, class T, OutputIterator<const T&> O, class Proj = identity,
        IndirectUnaryPredicate<projected<iterator_t<Rng>, Proj>> Pred>
      requires IndirectlyCopyable<iterator_t<Rng>, O>
      tagged_pair<tag::in(safe_iterator_t<Rng>), tag::out(O)>
        replace_copy_if(Rng&& rng, O result, Pred pred, const T& new_value,
                        Proj proj = Proj{});

    template <class T, OutputIterator<const T&> O, Sentinel<O> S>
      O fill(O first, S last, const T& value);

    template <class T, OutputRange<const T&> Rng>
      safe_iterator_t<Rng>
        fill(Rng&& rng, const T& value);

    template <class T, OutputIterator<const T&> O>
      O fill_n(O first, difference_type_t<O> n, const T& value);

    template <Iterator O, Sentinel<O> S, CopyConstructible F>
        requires Invocable<F&> && Writable<O, @\oldtxt{result_of_t<F\&()>}\newtxt{invoke_result_t<F\&>}@>
      O generate(O first, S last, F gen);

    template <class Rng, CopyConstructible F>
        requires Invocable<F&> && OutputRange<Rng, @\oldtxt{result_of_t<F\&()>}\newtxt{invoke_result_t<F\&>}@>
      safe_iterator_t<Rng>
        generate(Rng&& rng, F gen);

    template <Iterator O, CopyConstructible F>
        requires Invocable<F&> && Writable<O, @\oldtxt{result_of_t<F\&()>}\newtxt{invoke_result_t<F\&>}@>
      O generate_n(O first, difference_type_t<O> n, F gen);

    template <ForwardIterator I, Sentinel<I> S, class T, class Proj = identity>
      requires Permutable<I> &&
        IndirectRelation<equal_to<>, projected<I, Proj>, const T*>
      I remove(I first, S last, const T& value, Proj proj = Proj{});

    template <ForwardRange Rng, class T, class Proj = identity>
      requires Permutable<iterator_t<Rng>> &&
        IndirectRelation<equal_to<>, projected<iterator_t<Rng>, Proj>, const T*>
      safe_iterator_t<Rng>
        remove(Rng&& rng, const T& value, Proj proj = Proj{});

    template <ForwardIterator I, Sentinel<I> S, class Proj = identity,
        IndirectUnaryPredicate<projected<I, Proj>> Pred>
      requires Permutable<I>
      I remove_if(I first, S last, Pred pred, Proj proj = Proj{});

    template <ForwardRange Rng, class Proj = identity,
        IndirectUnaryPredicate<projected<iterator_t<Rng>, Proj>> Pred>
      requires Permutable<iterator_t<Rng>>
      safe_iterator_t<Rng>
        remove_if(Rng&& rng, Pred pred, Proj proj = Proj{});

    template <InputIterator I, Sentinel<I> S, WeaklyIncrementable O, class T,
        class Proj = identity>
      requires IndirectlyCopyable<I, O> &&
        IndirectRelation<equal_to<>, projected<I, Proj>, const T*>
      tagged_pair<tag::in(I), tag::out(O)>
        remove_copy(I first, S last, O result, const T& value, Proj proj = Proj{});

    template <InputRange Rng, WeaklyIncrementable O, class T, class Proj = identity>
      requires IndirectlyCopyable<iterator_t<Rng>, O> &&
        IndirectRelation<equal_to<>, projected<iterator_t<Rng>, Proj>, const T*>
      tagged_pair<tag::in(safe_iterator_t<Rng>), tag::out(O)>
        remove_copy(Rng&& rng, O result, const T& value, Proj proj = Proj{});

    template <InputIterator I, Sentinel<I> S, WeaklyIncrementable O,
        class Proj = identity, IndirectUnaryPredicate<projected<I, Proj>> Pred>
      requires IndirectlyCopyable<I, O>
      tagged_pair<tag::in(I), tag::out(O)>
        remove_copy_if(I first, S last, O result, Pred pred, Proj proj = Proj{});

    template <InputRange Rng, WeaklyIncrementable O, class Proj = identity,
        IndirectUnaryPredicate<projected<iterator_t<Rng>, Proj>> Pred>
      requires IndirectlyCopyable<iterator_t<Rng>, O>
      tagged_pair<tag::in(safe_iterator_t<Rng>), tag::out(O)>
        remove_copy_if(Rng&& rng, O result, Pred pred, Proj proj = Proj{});

    template <ForwardIterator I, Sentinel<I> S, class Proj = identity,
        IndirectRelation<projected<I, Proj>> R = equal_to<>>
      requires Permutable<I>
      I unique(I first, S last, R comp = R{}, Proj proj = Proj{});

    template <ForwardRange Rng, class Proj = identity,
        IndirectRelation<projected<iterator_t<Rng>, Proj>> R = equal_to<>>
      requires Permutable<iterator_t<Rng>>
      safe_iterator_t<Rng>
        unique(Rng&& rng, R comp = R{}, Proj proj = Proj{});

    template <InputIterator I, Sentinel<I> S, WeaklyIncrementable O,
        class Proj = identity, IndirectRelation<projected<I, Proj>> R = equal_to<>>
      requires IndirectlyCopyable<I, O> &&
        (ForwardIterator<I> ||
        (InputIterator<O> && Same<value_type_t<I>, value_type_t<O>>) ||
        IndirectlyCopyableStorable<I, O>)
      tagged_pair<tag::in(I), tag::out(O)>
        unique_copy(I first, S last, O result, R comp = R{}, Proj proj = Proj{});

    template <InputRange Rng, WeaklyIncrementable O, class Proj = identity,
        IndirectRelation<projected<iterator_t<Rng>, Proj>> R = equal_to<>>
      requires IndirectlyCopyable<iterator_t<Rng>, O> &&
        (ForwardIterator<iterator_t<Rng>> ||
        (InputIterator<O> && Same<value_type_t<iterator_t<Rng>>, value_type_t<O>>) ||
        IndirectlyCopyableStorable<iterator_t<Rng>, O>)
      tagged_pair<tag::in(safe_iterator_t<Rng>), tag::out(O)>
        unique_copy(Rng&& rng, O result, R comp = R{}, Proj proj = Proj{});

    template <BidirectionalIterator I, Sentinel<I> S>
      requires Permutable<I>
      I reverse(I first, S last);

    template <BidirectionalRange Rng>
      requires Permutable<iterator_t<Rng>>
      safe_iterator_t<Rng>
        reverse(Rng&& rng);

    template <BidirectionalIterator I, Sentinel<I> S, WeaklyIncrementable O>
      requires IndirectlyCopyable<I, O>
      tagged_pair<tag::in(I), tag::out(O)> reverse_copy(I first, S last, O result);

    template <BidirectionalRange Rng, WeaklyIncrementable O>
      requires IndirectlyCopyable<iterator_t<Rng>, O>
      tagged_pair<tag::in(safe_iterator_t<Rng>), tag::out(O)>
        reverse_copy(Rng&& rng, O result);

    template <ForwardIterator I, Sentinel<I> S>
      requires Permutable<I>
      tagged_pair<tag::begin(I), tag::end(I)>
        rotate(I first, I middle, S last);

    template <ForwardRange Rng>
      requires Permutable<iterator_t<Rng>>
      tagged_pair<tag::begin(safe_iterator_t<Rng>),
                  tag::end(safe_iterator_t<Rng>)>
        rotate(Rng&& rng, iterator_t<Rng> middle);

    template <ForwardIterator I, Sentinel<I> S, WeaklyIncrementable O>
      requires IndirectlyCopyable<I, O>
      tagged_pair<tag::in(I), tag::out(O)>
        rotate_copy(I first, I middle, S last, O result);

    template <ForwardRange Rng, WeaklyIncrementable O>
      requires IndirectlyCopyable<iterator_t<Rng>, O>
      tagged_pair<tag::in(safe_iterator_t<Rng>), tag::out(O)>
        rotate_copy(Rng&& rng, iterator_t<Rng> middle, O result);

    // \ref{range.alg.random.shuffle}, shuffle:
    template <RandomAccessIterator I, Sentinel<I> S, class Gen>
      requires Permutable<I> &&
        UniformRandom@\oldtxt{Number}\newtxt{Bit}@Generator<remove_reference_t<Gen>> &&
        ConvertibleTo<@\oldtxt{result_of_t<Gen\&()>}\newtxt{invoke_result_t<Gen\&>}@, difference_type_t<I>>
      I shuffle(I first, S last, Gen&& g);

    template <RandomAccessRange Rng, class Gen>
      requires Permutable<I> &&
        UniformRandom@\oldtxt{Number}\newtxt{Bit}@Generator<remove_reference_t<Gen>> &&
        ConvertibleTo<@\oldtxt{result_of_t<Gen\&()>}\newtxt{invoke_result_t<Gen\&>}@, difference_type_t<I>>
      safe_iterator_t<Rng>
        shuffle(Rng&& rng, Gen&& g);

    // \ref{range.alg.partitions}, partitions:
    template <InputIterator I, Sentinel<I> S, class Proj = identity,
        IndirectUnaryPredicate<projected<I, Proj>> Pred>
      bool is_partitioned(I first, S last, Pred pred, Proj proj = Proj{});

    template <InputRange Rng, class Proj = identity,
        IndirectUnaryPredicate<projected<iterator_t<Rng>, Proj>> Pred>
      bool is_partitioned(Rng&& rng, Pred pred, Proj proj = Proj{});

    template <ForwardIterator I, Sentinel<I> S, class Proj = identity,
        IndirectUnaryPredicate<projected<I, Proj>> Pred>
      requires Permutable<I>
      I partition(I first, S last, Pred pred, Proj proj = Proj{});

    template <ForwardRange Rng, class Proj = identity,
        IndirectUnaryPredicate<projected<iterator_t<Rng>, Proj>> Pred>
      requires Permutable<iterator_t<Rng>>
      safe_iterator_t<Rng>
        partition(Rng&& rng, Pred pred, Proj proj = Proj{});

    template <BidirectionalIterator I, Sentinel<I> S, class Proj = identity,
        IndirectUnaryPredicate<projected<I, Proj>> Pred>
      requires Permutable<I>
      I stable_partition(I first, S last, Pred pred, Proj proj = Proj{});

    template <BidirectionalRange Rng, class Proj = identity,
        IndirectUnaryPredicate<projected<iterator_t<Rng>, Proj>> Pred>
      requires Permutable<iterator_t<Rng>>
      safe_iterator_t<Rng>
        stable_partition(Rng&& rng, Pred pred, Proj proj = Proj{});

    template <InputIterator I, Sentinel<I> S, WeaklyIncrementable O1, WeaklyIncrementable O2,
        class Proj = identity, IndirectUnaryPredicate<projected<I, Proj>> Pred>
      requires IndirectlyCopyable<I, O1> && IndirectlyCopyable<I, O2>
      tagged_tuple<tag::in(I), tag::out1(O1), tag::out2(O2)>
        partition_copy(I first, S last, O1 out_true, O2 out_false, Pred pred,
                      Proj proj = Proj{});

    template <InputRange Rng, WeaklyIncrementable O1, WeaklyIncrementable O2,
        class Proj = identity,
        IndirectUnaryPredicate<projected<iterator_t<Rng>, Proj>> Pred>
      requires IndirectlyCopyable<iterator_t<Rng>, O1> &&
        IndirectlyCopyable<iterator_t<Rng>, O2>
      tagged_tuple<tag::in(safe_iterator_t<Rng>), tag::out1(O1), tag::out2(O2)>
        partition_copy(Rng&& rng, O1 out_true, O2 out_false, Pred pred, Proj proj = Proj{});

    template <ForwardIterator I, Sentinel<I> S, class Proj = identity,
        IndirectUnaryPredicate<projected<I, Proj>> Pred>
      I partition_point(I first, S last, Pred pred, Proj proj = Proj{});

    template <ForwardRange Rng, class Proj = identity,
        IndirectUnaryPredicate<projected<iterator_t<Rng>, Proj>> Pred>
      safe_iterator_t<Rng>
        partition_point(Rng&& rng, Pred pred, Proj proj = Proj{});

    // \ref{range.alg.sorting}, sorting and related operations:
    // \ref{range.alg.sort}, sorting:
    template <RandomAccessIterator I, Sentinel<I> S, class Comp = less<>,
        class Proj = identity>
      requires Sortable<I, Comp, Proj>
      I sort(I first, S last, Comp comp = Comp{}, Proj proj = Proj{});

    template <RandomAccessRange Rng, class Comp = less<>, class Proj = identity>
      requires Sortable<iterator_t<Rng>, Comp, Proj>
      safe_iterator_t<Rng>
        sort(Rng&& rng, Comp comp = Comp{}, Proj proj = Proj{});

    template <RandomAccessIterator I, Sentinel<I> S, class Comp = less<>,
        class Proj = identity>
      requires Sortable<I, Comp, Proj>
      I stable_sort(I first, S last, Comp comp = Comp{}, Proj proj = Proj{});

    template <RandomAccessRange Rng, class Comp = less<>, class Proj = identity>
      requires Sortable<iterator_t<Rng>, Comp, Proj>
      safe_iterator_t<Rng>
        stable_sort(Rng&& rng, Comp comp = Comp{}, Proj proj = Proj{});

    template <RandomAccessIterator I, Sentinel<I> S, class Comp = less<>,
        class Proj = identity>
      requires Sortable<I, Comp, Proj>
      I partial_sort(I first, I middle, S last, Comp comp = Comp{}, Proj proj = Proj{});

    template <RandomAccessRange Rng, class Comp = less<>, class Proj = identity>
      requires Sortable<iterator_t<Rng>, Comp, Proj>
      safe_iterator_t<Rng>
        partial_sort(Rng&& rng, iterator_t<Rng> middle, Comp comp = Comp{},
                    Proj proj = Proj{});

    template <InputIterator I1, Sentinel<I1> S1, RandomAccessIterator I2, Sentinel<I2> S2,
        class Comp = less<>, class Proj1 = identity, class Proj2 = identity>
      requires IndirectlyCopyable<I1, I2> && Sortable<I2, Comp, Proj2> &&
          IndirectStrictWeakOrder<Comp, projected<I1, Proj1>, projected<I2, Proj2>>
      I2 partial_sort_copy(I1 first, S1 last, I2 result_first, S2 result_last,
                           Comp comp = Comp{}, Proj1 proj1 = Proj1{}, Proj2 proj2 = Proj2{});

    template <InputRange Rng1, RandomAccessRange Rng2, class Comp = less<>,
        class Proj1 = identity, class Proj2 = identity>
      requires IndirectlyCopyable<iterator_t<Rng1>, iterator_t<Rng2>> &&
          Sortable<iterator_t<Rng2>, Comp, Proj2> &&
          IndirectStrictWeakOrder<Comp, projected<iterator_t<Rng1>, Proj1>,
            projected<iterator_t<Rng2>, Proj2>>
      safe_iterator_t<Rng2>
        partial_sort_copy(Rng1&& rng, Rng2&& result_rng, Comp comp = Comp{},
                          Proj1 proj1 = Proj1{}, Proj2 proj2 = Proj2{});

    template <ForwardIterator I, Sentinel<I> S, class Proj = identity,
        IndirectStrictWeakOrder<projected<I, Proj>> Comp = less<>>
      bool is_sorted(I first, S last, Comp comp = Comp{}, Proj proj = Proj{});

    template <ForwardRange Rng, class Proj = identity,
        IndirectStrictWeakOrder<projected<iterator_t<Rng>, Proj>> Comp = less<>>
      bool is_sorted(Rng&& rng, Comp comp = Comp{}, Proj proj = Proj{});

    template <ForwardIterator I, Sentinel<I> S, class Proj = identity,
        IndirectStrictWeakOrder<projected<I, Proj>> Comp = less<>>
      I is_sorted_until(I first, S last, Comp comp = Comp{}, Proj proj = Proj{});

    template <ForwardRange Rng, class Proj = identity,
        IndirectStrictWeakOrder<projected<iterator_t<Rng>, Proj>> Comp = less<>>
      safe_iterator_t<Rng>
        is_sorted_until(Rng&& rng, Comp comp = Comp{}, Proj proj = Proj{});

    template <RandomAccessIterator I, Sentinel<I> S, class Comp = less<>,
        class Proj = identity>
      requires Sortable<I, Comp, Proj>
      I nth_element(I first, I nth, S last, Comp comp = Comp{}, Proj proj = Proj{});

    template <RandomAccessRange Rng, class Comp = less<>, class Proj = identity>
      requires Sortable<iterator_t<Rng>, Comp, Proj>
      safe_iterator_t<Rng>
        nth_element(Rng&& rng, iterator_t<Rng> nth, Comp comp = Comp{}, Proj proj = Proj{});

    // \ref{range.alg.binary.search}, binary search:
    template <ForwardIterator I, Sentinel<I> S, class T, class Proj = identity,
        IndirectStrictWeakOrder<const T*, projected<I, Proj>> Comp = less<>>
      I lower_bound(I first, S last, const T& value, Comp comp = Comp{},
                    Proj proj = Proj{});

    template <ForwardRange Rng, class T, class Proj = identity,
        IndirectStrictWeakOrder<const T*, projected<iterator_t<Rng>, Proj>> Comp = less<>>
      safe_iterator_t<Rng>
        lower_bound(Rng&& rng, const T& value, Comp comp = Comp{}, Proj proj = Proj{});

    template <ForwardIterator I, Sentinel<I> S, class T, class Proj = identity,
        IndirectStrictWeakOrder<const T*, projected<I, Proj>> Comp = less<>>
      I upper_bound(I first, S last, const T& value, Comp comp = Comp{}, Proj proj = Proj{});

    template <ForwardRange Rng, class T, class Proj = identity,
        IndirectStrictWeakOrder<const T*, projected<iterator_t<Rng>, Proj>> Comp = less<>>
      safe_iterator_t<Rng>
        upper_bound(Rng&& rng, const T& value, Comp comp = Comp{}, Proj proj = Proj{});

    template <ForwardIterator I, Sentinel<I> S, class T, class Proj = identity,
        IndirectStrictWeakOrder<const T*, projected<I, Proj>> Comp = less<>>
      tagged_pair<tag::begin(I), tag::end(I)>
        equal_range(I first, S last, const T& value, Comp comp = Comp{}, Proj proj = Proj{});

    template <ForwardRange Rng, class T, class Proj = identity,
        IndirectStrictWeakOrder<const T*, projected<iterator_t<Rng>, Proj>> Comp = less<>>
      tagged_pair<tag::begin(safe_iterator_t<Rng>),
                  tag::end(safe_iterator_t<Rng>)>
        equal_range(Rng&& rng, const T& value, Comp comp = Comp{}, Proj proj = Proj{});

    template <ForwardIterator I, Sentinel<I> S, class T, class Proj = identity,
        IndirectStrictWeakOrder<const T*, projected<I, Proj>> Comp = less<>>
      bool binary_search(I first, S last, const T& value, Comp comp = Comp{},
                         Proj proj = Proj{});

    template <ForwardRange Rng, class T, class Proj = identity,
        IndirectStrictWeakOrder<const T*, projected<iterator_t<Rng>, Proj>> Comp = less<>>
      bool binary_search(Rng&& rng, const T& value, Comp comp = Comp{},
                         Proj proj = Proj{});

    // \ref{range.alg.merge}, merge:
    template <InputIterator I1, Sentinel<I1> S1, InputIterator I2, Sentinel<I2> S2,
        WeaklyIncrementable O, class Comp = less<>, class Proj1 = identity,
        class Proj2 = identity>
      requires Mergeable<I1, I2, O, Comp, Proj1, Proj2>
      tagged_tuple<tag::in1(I1), tag::in2(I2), tag::out(O)>
        merge(I1 first1, S1 last1, I2 first2, S2 last2, O result,
              Comp comp = Comp{}, Proj1 proj1 = Proj1{}, Proj2 proj2 = Proj2{});

    template <InputRange Rng1, InputRange Rng2, WeaklyIncrementable O, class Comp = less<>,
        class Proj1 = identity, class Proj2 = identity>
      requires Mergeable<iterator_t<Rng1>, iterator_t<Rng2>, O, Comp, Proj1, Proj2>
      tagged_tuple<tag::in1(safe_iterator_t<Rng1>),
                   tag::in2(safe_iterator_t<Rng2>),
                   tag::out(O)>
        merge(Rng1&& rng1, Rng2&& rng2, O result,
              Comp comp = Comp{}, Proj1 proj1 = Proj1{}, Proj2 proj2 = Proj2{});

    template <BidirectionalIterator I, Sentinel<I> S, class Comp = less<>,
        class Proj = identity>
      requires Sortable<I, Comp, Proj>
      I inplace_merge(I first, I middle, S last, Comp comp = Comp{}, Proj proj = Proj{});

    template <BidirectionalRange Rng, class Comp = less<>, class Proj = identity>
      requires Sortable<iterator_t<Rng>, Comp, Proj>
      safe_iterator_t<Rng>
        inplace_merge(Rng&& rng, iterator_t<Rng> middle, Comp comp = Comp{},
                      Proj proj = Proj{});

    // \ref{range.alg.set.operations}, set operations:
    template <InputIterator I1, Sentinel<I1> S1, InputIterator I2, Sentinel<I2> S2,
        class Proj1 = identity, class Proj2 = identity,
        IndirectStrictWeakOrder<projected<I1, Proj1>, projected<I2, Proj2>> Comp = less<>>
      bool includes(I1 first1, S1 last1, I2 first2, S2 last2, Comp comp = Comp{},
                    Proj1 proj1 = Proj1{}, Proj2 proj2 = Proj2{});

    template <InputRange Rng1, InputRange Rng2, class Proj1 = identity,
        class Proj2 = identity,
        IndirectStrictWeakOrder<projected<iterator_t<Rng1>, Proj1>,
          projected<iterator_t<Rng2>, Proj2>> Comp = less<>>
      bool includes(Rng1&& rng1, Rng2&& rng2, Comp comp = Comp{},
                    Proj1 proj1 = Proj1{}, Proj2 proj2 = Proj2{});

    template <InputIterator I1, Sentinel<I1> S1, InputIterator I2, Sentinel<I2> S2,
        WeaklyIncrementable O, class Comp = less<>, class Proj1 = identity, class Proj2 = identity>
      requires Mergeable<I1, I2, O, Comp, Proj1, Proj2>
      tagged_tuple<tag::in1(I1), tag::in2(I2), tag::out(O)>
        set_union(I1 first1, S1 last1, I2 first2, S2 last2, O result, Comp comp = Comp{},
                  Proj1 proj1 = Proj1{}, Proj2 proj2 = Proj2{});

    template <InputRange Rng1, InputRange Rng2, WeaklyIncrementable O,
        class Comp = less<>, class Proj1 = identity, class Proj2 = identity>
      requires Mergeable<iterator_t<Rng1>, iterator_t<Rng2>, O, Comp, Proj1, Proj2>
      tagged_tuple<tag::in1(safe_iterator_t<Rng1>),
                   tag::in2(safe_iterator_t<Rng2>),
                   tag::out(O)>
        set_union(Rng1&& rng1, Rng2&& rng2, O result, Comp comp = Comp{},
                  Proj1 proj1 = Proj1{}, Proj2 proj2 = Proj2{});

    template <InputIterator I1, Sentinel<I1> S1, InputIterator I2, Sentinel<I2> S2,
        WeaklyIncrementable O, class Comp = less<>, class Proj1 = identity, class Proj2 = identity>
      requires Mergeable<I1, I2, O, Comp, Proj1, Proj2>
      O set_intersection(I1 first1, S1 last1, I2 first2, S2 last2, O result,
                         Comp comp = Comp{}, Proj1 proj1 = Proj1{}, Proj2 proj2 = Proj2{});

    template <InputRange Rng1, InputRange Rng2, WeaklyIncrementable O,
        class Comp = less<>, class Proj1 = identity, class Proj2 = identity>
      requires Mergeable<iterator_t<Rng1>, iterator_t<Rng2>, O, Comp, Proj1, Proj2>
      O set_intersection(Rng1&& rng1, Rng2&& rng2, O result,
                         Comp comp = Comp{}, Proj1 proj1 = Proj1{}, Proj2 proj2 = Proj2{});

    template <InputIterator I1, Sentinel<I1> S1, InputIterator I2, Sentinel<I2> S2,
        WeaklyIncrementable O, class Comp = less<>, class Proj1 = identity, class Proj2 = identity>
      requires Mergeable<I1, I2, O, Comp, Proj1, Proj2>
      tagged_pair<tag::in1(I1), tag::out(O)>
        set_difference(I1 first1, S1 last1, I2 first2, S2 last2, O result,
                       Comp comp = Comp{}, Proj1 proj1 = Proj1{}, Proj2 proj2 = Proj2{});

    template <InputRange Rng1, InputRange Rng2, WeaklyIncrementable O,
        class Comp = less<>, class Proj1 = identity, class Proj2 = identity>
      requires Mergeable<iterator_t<Rng1>, iterator_t<Rng2>, O, Comp, Proj1, Proj2>
      tagged_pair<tag::in1(safe_iterator_t<Rng1>), tag::out(O)>
        set_difference(Rng1&& rng1, Rng2&& rng2, O result,
                       Comp comp = Comp{}, Proj1 proj1 = Proj1{}, Proj2 proj2 = Proj2{});

    template <InputIterator I1, Sentinel<I1> S1, InputIterator I2, Sentinel<I2> S2,
        WeaklyIncrementable O, class Comp = less<>, class Proj1 = identity, class Proj2 = identity>
      requires Mergeable<I1, I2, O, Comp, Proj1, Proj2>
      tagged_tuple<tag::in1(I1), tag::in2(I2), tag::out(O)>
        set_symmetric_difference(I1 first1, S1 last1, I2 first2, S2 last2, O result,
                                 Comp comp = Comp{}, Proj1 proj1 = Proj1{},
                                 Proj2 proj2 = Proj2{});

    template <InputRange Rng1, InputRange Rng2, WeaklyIncrementable O,
        class Comp = less<>, class Proj1 = identity, class Proj2 = identity>
      requires Mergeable<iterator_t<Rng1>, iterator_t<Rng2>, O, Comp, Proj1, Proj2>
      tagged_tuple<tag::in1(safe_iterator_t<Rng1>),
                   tag::in2(safe_iterator_t<Rng2>),
                   tag::out(O)>
        set_symmetric_difference(Rng1&& rng1, Rng2&& rng2, O result, Comp comp = Comp{},
                                 Proj1 proj1 = Proj1{}, Proj2 proj2 = Proj2{});

    // \ref{range.alg.heap.operations}, heap operations:
    template <RandomAccessIterator I, Sentinel<I> S, class Comp = less<>,
        class Proj = identity>
      requires Sortable<I, Comp, Proj>
      I push_heap(I first, S last, Comp comp = Comp{}, Proj proj = Proj{});

    template <RandomAccessRange Rng, class Comp = less<>, class Proj = identity>
      requires Sortable<iterator_t<Rng>, Comp, Proj>
      safe_iterator_t<Rng>
        push_heap(Rng&& rng, Comp comp = Comp{}, Proj proj = Proj{});

    template <RandomAccessIterator I, Sentinel<I> S, class Comp = less<>,
        class Proj = identity>
      requires Sortable<I, Comp, Proj>
      I pop_heap(I first, S last, Comp comp = Comp{}, Proj proj = Proj{});

    template <RandomAccessRange Rng, class Comp = less<>, class Proj = identity>
      requires Sortable<iterator_t<Rng>, Comp, Proj>
      safe_iterator_t<Rng>
        pop_heap(Rng&& rng, Comp comp = Comp{}, Proj proj = Proj{});

    template <RandomAccessIterator I, Sentinel<I> S, class Comp = less<>,
        class Proj = identity>
      requires Sortable<I, Comp, Proj>
      I make_heap(I first, S last, Comp comp = Comp{}, Proj proj = Proj{});

    template <RandomAccessRange Rng, class Comp = less<>, class Proj = identity>
      requires Sortable<iterator_t<Rng>, Comp, Proj>
      safe_iterator_t<Rng>
        make_heap(Rng&& rng, Comp comp = Comp{}, Proj proj = Proj{});

    template <RandomAccessIterator I, Sentinel<I> S, class Comp = less<>,
        class Proj = identity>
      requires Sortable<I, Comp, Proj>
      I sort_heap(I first, S last, Comp comp = Comp{}, Proj proj = Proj{});

    template <RandomAccessRange Rng, class Comp = less<>, class Proj = identity>
      requires Sortable<iterator_t<Rng>, Comp, Proj>
      safe_iterator_t<Rng>
        sort_heap(Rng&& rng, Comp comp = Comp{}, Proj proj = Proj{});

    template <RandomAccessIterator I, Sentinel<I> S, class Proj = identity,
        IndirectStrictWeakOrder<projected<I, Proj>> Comp = less<>>
      bool is_heap(I first, S last, Comp comp = Comp{}, Proj proj = Proj{});

    template <RandomAccessRange Rng, class Proj = identity,
        IndirectStrictWeakOrder<projected<iterator_t<Rng>, Proj>> Comp = less<>>
      bool is_heap(Rng&& rng, Comp comp = Comp{}, Proj proj = Proj{});

    template <RandomAccessIterator I, Sentinel<I> S, class Proj = identity,
        IndirectStrictWeakOrder<projected<I, Proj>> Comp = less<>>
      I is_heap_until(I first, S last, Comp comp = Comp{}, Proj proj = Proj{});

    template <RandomAccessRange Rng, class Proj = identity,
        IndirectStrictWeakOrder<projected<iterator_t<Rng>, Proj>> Comp = less<>>
      safe_iterator_t<Rng>
        is_heap_until(Rng&& rng, Comp comp = Comp{}, Proj proj = Proj{});

    // \ref{range.alg.min.max}, minimum and maximum:
    template <class T, class Proj = identity,
        IndirectStrictWeakOrder<projected<const T*, Proj>> Comp = less<>>
      constexpr const T& min(const T& a, const T& b, Comp comp = Comp{}, Proj proj = Proj{});

    template <Copyable T, class Proj = identity,
        IndirectStrictWeakOrder<projected<const T*, Proj>> Comp = less<>>
      constexpr T min(initializer_list<T> t, Comp comp = Comp{}, Proj proj = Proj{});

    template <InputRange Rng, class Proj = identity,
        IndirectStrictWeakOrder<projected<iterator_t<Rng>, Proj>> Comp = less<>>
      requires Copyable<value_type_t<iterator_t<Rng>>>
      value_type_t<iterator_t<Rng>>
        min(Rng&& rng, Comp comp = Comp{}, Proj proj = Proj{});

    template <class T, class Proj = identity,
        IndirectStrictWeakOrder<projected<const T*, Proj>> Comp = less<>>
      constexpr const T& max(const T& a, const T& b, Comp comp = Comp{}, Proj proj = Proj{});

    template <Copyable T, class Proj = identity,
        IndirectStrictWeakOrder<projected<const T*, Proj>> Comp = less<>>
      constexpr T max(initializer_list<T> t, Comp comp = Comp{}, Proj proj = Proj{});

    template <InputRange Rng, class Proj = identity,
        IndirectStrictWeakOrder<projected<iterator_t<Rng>, Proj>> Comp = less<>>
      requires Copyable<value_type_t<iterator_t<Rng>>>
      value_type_t<iterator_t<Rng>>
        max(Rng&& rng, Comp comp = Comp{}, Proj proj = Proj{});

    template <class T, class Proj = identity,
        IndirectStrictWeakOrder<projected<const T*, Proj>> Comp = less<>>
      constexpr tagged_pair<tag::min(const T&), tag::max(const T&)>
        minmax(const T& a, const T& b, Comp comp = Comp{}, Proj proj = Proj{});

    template <Copyable T, class Proj = identity,
        IndirectStrictWeakOrder<projected<const T*, Proj>> Comp = less<>>
      constexpr tagged_pair<tag::min(T), tag::max(T)>
        minmax(initializer_list<T> t, Comp comp = Comp{}, Proj proj = Proj{});

    template <InputRange Rng, class Proj = identity,
        IndirectStrictWeakOrder<projected<iterator_t<Rng>, Proj>> Comp = less<>>
      requires Copyable<value_type_t<iterator_t<Rng>>>
      tagged_pair<tag::min(value_type_t<iterator_t<Rng>>),
                  tag::max(value_type_t<iterator_t<Rng>>)>
        minmax(Rng&& rng, Comp comp = Comp{}, Proj proj = Proj{});

    template <ForwardIterator I, Sentinel<I> S, class Proj = identity,
        IndirectStrictWeakOrder<projected<I, Proj>> Comp = less<>>
      I min_element(I first, S last, Comp comp = Comp{}, Proj proj = Proj{});

    template <ForwardRange Rng, class Proj = identity,
        IndirectStrictWeakOrder<projected<iterator_t<Rng>, Proj>> Comp = less<>>
      safe_iterator_t<Rng>
        min_element(Rng&& rng, Comp comp = Comp{}, Proj proj = Proj{});

    template <ForwardIterator I, Sentinel<I> S, class Proj = identity,
        IndirectStrictWeakOrder<projected<I, Proj>> Comp = less<>>
      I max_element(I first, S last, Comp comp = Comp{}, Proj proj = Proj{});

    template <ForwardRange Rng, class Proj = identity,
        IndirectStrictWeakOrder<projected<iterator_t<Rng>, Proj>> Comp = less<>>
      safe_iterator_t<Rng>
        max_element(Rng&& rng, Comp comp = Comp{}, Proj proj = Proj{});

    template <ForwardIterator I, Sentinel<I> S, class Proj = identity,
        IndirectStrictWeakOrder<projected<I, Proj>> Comp = less<>>
      tagged_pair<tag::min(I), tag::max(I)>
        minmax_element(I first, S last, Comp comp = Comp{}, Proj proj = Proj{});

    template <ForwardRange Rng, class Proj = identity,
        IndirectStrictWeakOrder<projected<iterator_t<Rng>, Proj>> Comp = less<>>
      tagged_pair<tag::min(safe_iterator_t<Rng>),
                  tag::max(safe_iterator_t<Rng>)>
        minmax_element(Rng&& rng, Comp comp = Comp{}, Proj proj = Proj{});

    template <InputIterator I1, Sentinel<I1> S1, InputIterator I2, Sentinel<I2> S2,
        class Proj1 = identity, class Proj2 = identity,
        IndirectStrictWeakOrder<projected<I1, Proj1>, projected<I2, Proj2>> Comp = less<>>
      bool
        lexicographical_compare(I1 first1, S1 last1, I2 first2, S2 last2,
                                Comp comp = Comp{}, Proj1 proj1 = Proj1{}, Proj2 proj2 = Proj2{});

    template <InputRange Rng1, InputRange Rng2, class Proj1 = identity,
        class Proj2 = identity,
        IndirectStrictWeakOrder<projected<iterator_t<Rng1>, Proj1>,
          projected<iterator_t<Rng2>, Proj2>> Comp = less<>>
      bool
        lexicographical_compare(Rng1&& rng1, Rng2&& rng2, Comp comp = Comp{},
                                Proj1 proj1 = Proj1{}, Proj2 proj2 = Proj2{});

    // \ref{range.alg.permutation.generators}, permutations:
    template <BidirectionalIterator I, Sentinel<I> S, class Comp = less<>,
        class Proj = identity>
      requires Sortable<I, Comp, Proj>
      bool next_permutation(I first, S last, Comp comp = Comp{}, Proj proj = Proj{});

    template <BidirectionalRange Rng, class Comp = less<>,
        class Proj = identity>
      requires Sortable<iterator_t<Rng>, Comp, Proj>
      bool next_permutation(Rng&& rng, Comp comp = Comp{}, Proj proj = Proj{});

    template <BidirectionalIterator I, Sentinel<I> S, class Comp = less<>,
        class Proj = identity>
      requires Sortable<I, Comp, Proj>
      bool prev_permutation(I first, S last, Comp comp = Comp{}, Proj proj = Proj{});

    template <BidirectionalRange Rng, class Comp = less<>,
        class Proj = identity>
      requires Sortable<iterator_t<Rng>, Comp, Proj>
      bool prev_permutation(Rng&& rng, Comp comp = Comp{}, Proj proj = Proj{});
  }
}@\oldtxt{\}\}}@
\end{codeblock}

%!TEX root = std.tex

\setcounter{chapter}{23}
\rSec0[iterators]{Iterators library}

\rSec1[iterators.general]{General}

\pnum
This Clause describes components that \Cpp programs may use to perform
iterations over containers (Clause \cxxref{containers}),
streams~(\cxxref{iostream.format}),
\removed{and} stream buffers~(\cxxref{stream.buffers})
\added{, and ranges~(\ref{range.iterables})}.

\pnum
The following subclauses describe
iterator requirements, and
components for
iterator primitives,
predefined iterators,
and stream iterators,
as summarized in Table~\ref{tab:iterators.lib.summary}.

\begin{libsumtab}{Iterators library summary}{tab:iterators.lib.summary}
\ref{iterator.requirements} & Requirements        &                           \\ \rowsep
\ref{iterator.primitives} & Iterator primitives   &   \tcode{<iterator>}      \\
\ref{predef.iterators} & Predefined iterators     &                           \\
\ref{stream.iterators} & Stream iterators         &                           \\
\added{\ref{iterables}} & \added{Iterables}       &                           \\
\end{libsumtab}


\rSec1[iterator.requirements]{Iterator requirements}

\rSec2[iterator.requirements.general]{In general}

\pnum
\indextext{requirements!iterator}%
Iterators are a generalization of pointers that allow a \Cpp program to work with different data structures
(containers\added{ and ranges}) in a uniform manner.
To be able to construct template algorithms that work correctly and
efficiently on different types of data structures, the library formalizes not just the interfaces but also the
semantics and complexity assumptions of iterators.
All input iterators
\tcode{i}
support the expression
\tcode{*i},
resulting in a value of some object type
\tcode{T},
called the
\term{value type}
of the iterator.
All output iterators support the expression
\tcode{*i = o}
where
\tcode{o}
is a value of some type that is in the set of types that are
\term{writable}
to the particular iterator type of
\tcode{i}.
\removed{All iterators
\tcode{i}
for which the expression
\tcode{(*i).m}
is well-defined, support the expression
\tcode{i->m}
with the same semantics as
\tcode{(*i).m}.}
For every iterator type
\tcode{X}
for which
equality is defined, there is a corresponding signed integer type called the
\term{difference type}
of the iterator.

\pnum
Since iterators are an abstraction of pointers, their semantics is
a generalization of most of the semantics of pointers in \Cpp.
This ensures that every
function template
that takes iterators
works as well with regular pointers.
This International Standard defines
\changed{five}{seven} categories of iterators, according to the operations
defined on them:
\added{\techterm{weak input iterators}, }
\techterm{input iterators},
\added{\techterm{weak output iterators}, }
\techterm{output iterators},
\techterm{forward iterators},
\techterm{bidirectional iterators}
and
\techterm{random access iterators},
as shown in Table~\ref{tab:iterators.relations}.

\begin{floattable}{Relations among iterator categories}{tab:iterators.relations}
{lllll}
\topline
\textbf{Random Access}          &   $\rightarrow$ \textbf{Bidirectional}    &
$\rightarrow$ \textbf{Forward}  &   $\rightarrow$ \textbf{Input}            & \added{   $\rightarrow$ \textbf{WeakInput}}           \\
                        &   &   &   $\rightarrow$ \textbf{Output}           & \added{   $\rightarrow$ \textbf{WeakOutput}}          \\
\end{floattable}

\pnum
\changed{Forward}{Input} iterators satisfy all the requirements of \added{weak }input
iterators and can be used whenever \changed{an}{a weak} input iterator is specified;
\added{Forward iterators also satisfy all the requirements of
input iterators and can be used whenever an input iterator is specified;}
Bidirectional iterators also satisfy all the requirements of
forward iterators and can be used whenever a forward iterator is specified;
Random access iterators also satisfy all the requirements of bidirectional
iterators and can be used whenever a bidirectional iterator is specified.

\pnum
Iterators that further satisfy the requirements of \added{weak }output iterators are
called \defn{mutable iterator}{s}. Nonmutable iterators are referred to
as \defn{constant iterator}{s}.

\pnum
Just as a regular pointer to an array guarantees that there is a pointer value pointing past the last element
of the array, so for any iterator type there is an iterator value that points past the last element of a
corresponding sequence.
These values are called
\term{past-the-end}
values.
Values of an iterator
\tcode{i}
for which the expression
\tcode{*i}
is defined are called
\term{dereferenceable}.
The library never assumes that past-the-end values are dereferenceable.
Iterators can also have singular values that are not associated with any
sequence.
\enterexample
After the declaration of an uninitialized pointer
\tcode{x}
(as with
\tcode{int* x;}),
\tcode{x}
must always be assumed to have a singular value of a pointer.
\exitexample
Results of most expressions are undefined for singular values;
the only exceptions are destroying an iterator that holds a singular value,
the assignment of a non-singular value to
an iterator that holds a singular value, and\removed{, for iterators that satisfy the
\tcode{DefaultConstructible} requirements,} using a value-initialized iterator
as the source of a copy or move operation. \enternote This guarantee is not
offered for default initialization, although the distinction only matters for types
with trivial default constructors such as pointers or aggregates holding pointers.
\exitnote
In these cases the singular
value is overwritten the same way as any other value.
Dereferenceable
values are always non-singular.

\pnum
An iterator\added{ or sentinel}
\tcode{j}
is called
\term{reachable}
from an iterator
\tcode{i}
if and only if there is a finite sequence of applications of
the expression
\tcode{++i}
that makes
\tcode{i == j}.
If
\tcode{j}
is reachable from
\tcode{i},
they refer to elements of the same sequence.

\pnum
Most of the library's algorithmic templates that operate on data structures have interfaces that use ranges.
A
\term{range}
is a pair of iterators\added{ or an iterator and a sentinel} that designate the beginning and end of the computation.
A range \range{i}{i}
is an empty range;
in general, a range \range{i}{j}
refers to the elements in the data structure starting with the element
pointed to by
\tcode{i}
and up to but not including the element \changed{pointed to}{denoted} by
\tcode{j}.
Range \range{i}{j}
is valid if and only if
\tcode{j}
is reachable from
\tcode{i}.
The result of the application of functions in the library to invalid ranges is
undefined.

\added{
\pnum
A
\term{sentinel}
is an abstraction of a past-the-end iterator. Sentinels are Regular types that can be used to denote
the end of a range. A sentinel and an iterator denoting a range shall be EqualityComparable. A
sentinel denotes an element when an iterator
\tcode{i}
compares equal to the sentinel, and
\tcode{i}
points to that element.}

\pnum
All the categories of iterators require only those functions that are realizable for a given category in
constant time (amortized).
\removed{Therefore, requirement tables for the iterators do not have a complexity column.}

\pnum
Destruction of an iterator may invalidate pointers and references
previously obtained from that iterator.

\pnum
An
\techterm{invalid}
iterator is an iterator that may be singular.\footnote{This definition applies to pointers, since pointers are iterators.
The effect of dereferencing an iterator that has been invalidated
is undefined.
}

\pnum
In the following sections,
\tcode{a}
and
\tcode{b}
denote values of type
\tcode{X} or \tcode{const X},
\tcode{difference_type} and \tcode{reference} refer to the
types \tcode{\removed{iterator_traits<X>::difference_type}}\tcode{\added{DifferenceType<X>}} and
\tcode{\removed{iterator_traits<X>::ref\-erence}}~\tcode{\added{ReferenceType<X>}}, respectively,
\tcode{n}
denotes a value of
\tcode{difference_type},
\tcode{u},
\tcode{tmp},
and
\tcode{m}
denote identifiers,
\tcode{r}
denotes a value of
\tcode{X\&},
\tcode{t}
denotes a value of value type
\tcode{T},
\tcode{o}
denotes a value of some type that is writable to the \added{weak }output iterator.
\enternote For an iterator type \tcode{X} \changed{there must be an instantiation
of \tcode{iterator_traits<X>}~(\cxxref{iterator.traits})}{the type aliases
\tcode{DifferenceType<X>} and \tcode{ReferenceType<X>} must be well-formed}. \exitnote

\begin{addedblock}
\rSec2[readable.iterators]{Readable types}

\pnum
The \tcode{Readable} concept is modeled by types that are readable by applying \tcode{operator*}
including pointers, smart pointers, and iterators.

\indexlibrary{\idxcode{Readable}}%
\begin{codeblock}
  template <class I>
  concept bool Readable =
    Semiregular<I> &&
    requires (I i) {
      typename ValueType<I>;
      { *i } -> const ValueType<I>&; // pre: i is dereferencable
    };
\end{codeblock}

\pnum
A \tcode{Readable} type has an associated value type that can be accessed with the \tcode{ValueType}
alias template.

\indexlibrary{\idxcode{ValueType}}%
\begin{codeblock}
  template <class, class = void> struct value_type { };
  template <class T>
  struct value_type<T*>
    : enable_if<!is_void<T>::value, remove_cv_t<T>> { };
  template <class T>
  struct value_type<T[]> : remove_cv<T> { };
  template <class T, size_t N>
  struct value_type<T[N]> : remove_cv<T> { };
  template <class T>
  struct value_type<T, void_t<typename T::value_type>>
    : enable_if<!is_void<typename T::value_type>::value, typename T::value_type> { };
  template <class T>
  struct value_type<T, void_t<typename T::element_type>>
    : enable_if<!is_void<typename T::element_type>::value, typename T::element_type> { };

  template <class T>
  using ValueType = typename value_type<T>::type;
\end{codeblock}

\pnum
If a type \tcode{I} has an associated value type, then \tcode{value_type<I>::type} shall name the
value type. Otherwise, there shall be no nested type \tcode{type}.

\pnum
The \tcode{value_type} class template may be specialized on user-defined types.

\pnum
When instantiated with a type \tcode{I} that has a nested type \tcode{value_type},
\tcode{value_type<I>::type} names that type, unless it is \tcode{void} in which case
\tcode{value_type<I>} shall have no nested type \tcode{type}. \enternote Some legacy output
iterators define a nested type named \tcode{value_type} that is an alias for \tcode{void}. These
types are not \tcode{Readable} and have no associated value types.\exitnote

\pnum
When instantiated with a type \tcode{I} that has a nested type \tcode{element_type},
\tcode{value_type<I>::type} names that type, unless it is \tcode{void} in which case
\tcode{value_type<I>} shall have no nested type \tcode{type}. \enternote Smart pointers like
\tcode{shared_ptr<int>} are \tcode{Readable} and have an associated value type. But a smart pointer
like \tcode{shared_ptr<void>} is not \tcode{Readable} and has no associated value type.\exitnote

\rSec2[movewritable.iterators]{MoveWritable types}

\pnum
The \tcode{MoveWritable} concept describes a requirements for moving a value into an iterator's
referenced object.

\indexlibrary{\idxcode{MoveWritable}}%
\begin{codeblock}
  template <class Out, class T>
  concept bool MoveWritable =
    Semiregular<Out> &&
    requires (Out o, T value) {
      { *o = move(value) };
    };
\end{codeblock}

\pnum
Let \tcode{value} be an rvalue or a non-const lvalue of type \tcode{T}, and let \tcode{o} be an
object of type \tcode{Out}. Then types \tcode{T} and \tcode{Out} model \tcode{MoveWritable} if and
only if

\begin{itemize}
\item If \tcode{o} is dereferencable, then after the assignment \tcode{*o = move(value)}, the value
referred to by \tcode{*o} is equal to the value of \tcode{value} before the assignment.
\end{itemize}

\pnum
After the expression \tcode{*o = move(value)}, object \tcode{o} is not required to be dereferencable.

\pnum
\tcode{value}'s state is unspecified. \enternote \tcode{value} must still meet the
requirements of the library component that is using it. The operations listed
in those requirements must work as specified whether \tcode{value} has been moved
from or not.\exitnote

\pnum
\enternote
The only valid use of an \tcode{operator*} is on the left side of the assignment statement.
\textit{Assignment through the same value of the writable type happens only once.}
\exitnote

\rSec2[writable.iterators]{Writable types}

\pnum
The \tcode{Writable} concept describes a requirements for copying a value into an iterator's
referenced object.

\indexlibrary{\idxcode{Writable}}%
\begin{codeblock}
  template <class Out, class T>
  concept bool Writable =
    Semiregular<Out> &&
    MoveWritable<Out, T> &&
    requires (Out o, T value) {
      { *o = value };
    };
\end{codeblock}

\pnum
Let \tcode{value} be an lvalue of type (possibly \tcode{const}) \tcode{T} or an rvalue
of type \tcode{const T}, and let \tcode{o} be an object of type \tcode{Out}. Then types
\tcode{T} and \tcode{Out} model \tcode{Writable} if and only if

\begin{itemize}
\item After the assignment \tcode{*o = value}, the value referred to by \tcode{*o} is equal to
the value of \tcode{value} and \tcode{value} is unchanged.
\end{itemize}

\rSec2[indirectlymovable.iterators]{IndirectlyMovable types}

\pnum
The \tcode{IndirectlyMovable} concept describes the move relationship between a \tcode{Readable}
type and a \tcode{MoveWritable} type.

\indexlibrary{\idxcode{IndirectlyMovable}}%
\begin{codeblock}
  template <class I, class Out>
  concept bool IndirectlyMovable =
    Readable<I> &&
    Semiregular<Out> &&
    MoveWritable<Out, ValueType<I>>;
\end{codeblock}

\pnum
Let \tcode{i} be an object of type \tcode{I}, and let \tcode{o} be an object of type \tcode{Out}.
Then types \tcode{I} and \tcode{Out} model \tcode{IndirectlyMovable} if and only if

\begin{itemize}
\item After the assignment \tcode{*o = move(*i)}, the value referred to by \tcode{*o} is equal to
the value of \tcode{*i} before the assignment.
\end{itemize}

\rSec2[indirectlycopyable.iterators]{IndirectlyCopyable types}

\pnum
The \tcode{IndirectlyCopyable} concept describes the copy relationship between a \tcode{Readable}
type and a \tcode{Writable} type.

\indexlibrary{\idxcode{IndirectlyCopyable}}%
\begin{codeblock}
  template <class I, class Out>
  concept bool IndirectlyCopyable =
    Readable<I> &&
    Semiregular<I> &&
    IndirectlyMovable<I, Out> &&
    Writable<Out, ValueType<I>>;
\end{codeblock}

\pnum
Let \tcode{i} be an object of type \tcode{I}, and let \tcode{o} be an object of type \tcode{Out}.
Then types \tcode{I} and \tcode{Out} model \tcode{IndirectlyCopyable} if and only if

\begin{itemize}
\item After the assignment \tcode{*o = *i}, the value referred to by \tcode{*o} is equal to
the value of \tcode{*i} and the value of \tcode{*i} is unchanged.
\end{itemize}

\rSec2[indirectlyswappable.iterators]{IndirectlySwappable types}

\pnum
The \tcode{IndirectlySwappable} concept describes a swappable relationship between the
value types of two \tcode{Readable} types.

\indexlibrary{\idxcode{IndirectlySwappable}}%
\begin{codeblock}
  template <class I1, class I2 = I1>
  concept bool IndirectlySwappable =
    Readable<I1> && 
    Readable<I2> &&
    Swappable<ValueType<I1>, ValueType<I2>> &&
    Swappable<ValueType<I1>> &&
    Swappable<ValueType<I2>>;
\end{codeblock}

\rSec2[weaklyincrementable.iterators]{WeaklyIncrementable types}

The \tcode{WeaklyIncrementable} concept describes types that can be incremented with the pre-
and post-increment operators. The increment operations are not required to be equality-preserving,
nor is the type required to be \tcode{EqualityComparable}.

\indexlibrary{\idxcode{WeaklyIncrementable}}%
\begin{codeblock}
  template <class I>
  concept bool WeaklyIncrementable =
    Semiregular<I> &&
    requires (I i) {
      typename DifferenceType<I>;
      requires SignedIntegral<DifferenceType<I>>;
      { ++i };
      requires Same<I&, decltype(++i)>;
      { i++ };
    };
\end{codeblock}

\pnum
Not all arguments will be incrementable for a given type. For example $NaN$ is not a well-formed
floating point value and hence is not incrementable. Likewise, past-the-end iterators are also not
incrementable. This does not mean that the type does not model \tcode{WeaklyIncrementable}.

\pnum
Let \tcode{i} be an object of type \tcode{I}.
Then the type \tcode{I} models \tcode{WeaklyIncrementable} if and only if

\begin{itemize}
\item If \tcode{i} is incrementable, then \tcode{++i} moves \tcode{i} to the next element.
\item If \tcode{i} is incrementable, then \tcode{(\&++i == \&i) != false}.
\item If \tcode{i} is incrementable, then \tcode{i++} moves \tcode{i} to the next element.
\item \tcode{++i} is valid if and only if \tcode{i++} is valid.
\end{itemize}

\ednote{Copied almost verbatim from the InputIterator description. Remove this wording there.}

\pnum
\enternote For \tcode{WeaklyIncrementable} types, \tcode{a} equals \tcode{b} does imply that \tcode{++a}
equals \tcode{++b}. (Equality does not guarantee the substitution property or referential
transparency.) Algorithms on weakly incrementable types should never attempt to pass
through the same incrementable value twice. They should be single pass algorithms. These algorithms
can be used with istreams as the source of the input data through the \tcode{istream_iterator} class
template.\exitnote

\rSec2[incrementable.iterators]{Incrementable types}

The \tcode{Incrementable} concept describes types that can be incremented with the pre-
and post-increment operators. The increment operations are required to be equality-preserving,
and the type is required to be \tcode{EqualityComparable}.

\indexlibrary{\idxcode{Incrementable}}%
\begin{codeblock}
  template <class I>
  concept bool Incrementable =
    Regular<I> &&
    WeaklyIncrementable<I> &&
    Same<I, decltype(i++)>;
\end{codeblock}

\pnum
Let \tcode{a} and \tcode{b} be incrementable objects of type \tcode{I}.
Then the type \tcode{I} models \tcode{Incrementable} if and only if

\begin{itemize}
\item If \tcode{(a == b) != false} then \tcode{(++a == ++b) != false}.
\item If \tcode{(a == b) != false} then \tcode{(a++, a) == (b++, b) != false}.
\item If \tcode{(a == b) != false} then \tcode{(a++ == b) != false}.
\item If \tcode{(a == b) != false} then \tcode{((a++, a) == ++b) != false}.
\end{itemize}

\ednote{Copied in part from the ForwardIterator description. Remove this wording there.}

\pnum
\enternote The requirement that \tcode{a} equals \tcode{b} implies \tcode{++a} equals \tcode{++b}
(which is not true for weakly incrementable types) allows the use of multi-pass one-directional
algorithms with types that model \tcode{Incrementable}.\exitnote

\end{addedblock}

\rSec2[weakiterator.iterators]{Weak iterators}

\pnum
The \added{\tcode{Weak}}\tcode{Iterator} \changed{requirements}{concept} form\added{s}
the basis of the iterator concept taxonomy; every iterator satisfies the
\added{\tcode{Weak}}\tcode{Iterator} requirements. This
\changed{set of requirements}{concept} specifies operations for dereferencing and incrementing
an iterator. Most algorithms will require additional operations
\added{to compare iterators~(\ref{iterator.iterators}),} to
read~(\ref{input.iterators}) or write~(\ref{output.iterators}) values, or
to provide a richer set of iterator movements~(\ref{forward.iterators},
\ref{bidirectional.iterators}, \ref{random.access.iterators}).)

\ednote{Remove para 2 and Table 106.}

\begin{addedblock}
\indexlibrary{\idxcode{WeakIterator}}%
\begin{codeblock}
  template <class I>
  concept bool WeakIterator =
    WeaklyIncrementable<I>
    requires(I i) {
      { *i } -> auto&&; // pre: i is dereferenceable
    };
\end{codeblock}

\enternote The requirement that the result of dereferencing the iterator is deducable from
\tcode{auto\&\&} effectively means that it cannot be \tcode{void}.\exitnote
\end{addedblock}

\begin{addedblock}
\rSec2[iterator.iterators]{Iterators}

\pnum
The \tcode{Iterator} concept refines \tcode{WeakIterator}~(\ref{weakiterator.iterators}) and adds
the requirement that the iterator is equality comparable.

\pnum
In the \tcode{Iterator} concept, the set of values over which
\tcode{==} is (required to be) defined can change over time.
Each algorithm places additional requirements on the domain of
\tcode{==} for the iterator values it uses.
These requirements can be inferred from the uses that algorithm
makes of \tcode{==} and \tcode{!=}.
\enterexample
the call \tcode{find(a,b,x)}
is defined only if the value of \tcode{a}
has the property \textit{p}
defined as follows:
\tcode{b} has property \textit{p}
and a value \tcode{i}
has property \textit{p}
if
\tcode{(*i==x)}
or if
\tcode{(*i!=x}
and
\tcode{++i}
has property
\tcode{p}).
\exitexample

\indexlibrary{\idxcode{Iterator}}%
\begin{codeblock}
  template <class I>
  concept bool Iterator =
    WeakIterator<I> &&
    EqualityComparable<I>();
\end{codeblock}

\rSec2[sentinel.iterators]{Sentinels}

The \tcode{Sentinel} concept defines requirements for a type that
is an abstraction of the past-the-end iterator. Its values can be
compared to an iterator for equality.

\indexlibrary{\idxcode{Sentinel}}%
\begin{codeblock}
  template <class T, class I>
  concept bool Sentinel =
    Regular<T> &&
    Iterator<I> &&
    EqualityComparable<T, I>();
\end{codeblock}

\rSec2[weakinput.iterators]{Weak input iterators}

\pnum
The \tcode{WeakInputIterator} concept is a refinement of
\tcode{WeakIterator}~(\ref{weakiterator.iterators}). It
defines requirements for a type whose referred to values can be read (from the requirement for
\tcode{Readable}~(\ref{readable.iterators})) and which can be both pre- and post-incremented. However,
weak input iterators are not required to be comparable for equality.

\indexlibrary{\idxcode{WeakInputIterator}}%
\begin{codeblock}
  template <class I>
  concept bool WeakInputIterator =
    WeakIterator<I> &&
    Readable<I> &&
    requires(I i) {
      typename IteratorCategory<I>;
      { i++ } -> Readable;
      requires Derived<IteratorCategory<I>, weak_input_iterator_tag>;
    };
\end{codeblock}
\end{addedblock}

\rSec2[input.iterators]{Input iterators}

\ednote{Remove para 1, 2 and Table 107}

\begin{addedblock}
\pnum
The \tcode{InputIterator} concept is a refinement of \tcode{Iterator}~(\ref{iterator.iterators}) and
\tcode{WeakInputIterator}~(\ref{weakinput.iterators}).

\indexlibrary{\idxcode{InputIterator}}%
\begin{codeblock}
  template <class I>
  concept bool InputIterator =
    WeakInputIterator<I> &&
    Iterator<I> &&
    Derived<IteratorCategory<I>, input_iterator_tag>;
\end{codeblock}

\end{addedblock}

\pnum
\enternote
For input iterators,
\tcode{a == b}
does not imply
\tcode{++a == ++b}.
(Equality does not guarantee the substitution property or referential transparency.)
Algorithms on input iterators should never attempt to pass through the same iterator twice.
They should be
\term{single pass}
algorithms.
Value type T is not required to be a \tcode{CopyAssignable} type (\removed{Table}~\ref{concepts.lib.copyassignable}).
These algorithms can be used with istreams as the source of the input data through the
\tcode{istream_iterator}
class template.
\exitnote

\ednote{Section Output iterators renamed to Weak output iterators below:}

\rSec2[weakoutput.iterators]{Weak output iterators}

\ednote{Remove para 1 and Table 108}

\begin{addedblock}
\pnum
The \tcode{WeakOutputIterator} concept is a refinement of
\tcode{WeakIterator}~(\ref{weakiterator.iterators}). It defines requirements for a type that
can be used to write values (from the requirement for
\tcode{Writable}~(\ref{writable.iterators})) and which can be both pre- and post-incremented.
However, weak output iterators are not required to be comparable for equality.

\indexlibrary{\idxcode{WeakOutputIterator}}%
\begin{codeblock}
  template <class I, class T>
  concept bool WeakOutputIterator =
    WeakIterator<I> && Writable<I, T>;
\end{codeblock}
\end{addedblock}

\pnum
\enternote
\removed{The only valid use of an
\tcode{operator*}
is on the left side of the assignment statement.}
\textit{\removed{Assignment through the same value of the iterator happens only once.}}
Algorithms on output iterators should never attempt to pass through the same iterator twice.
They should be
\term{single pass}
algorithms.
\removed{Equality and inequality might not be defined.}
Algorithms that take \added{weak }output iterators can be used with ostreams as the destination
for placing data through the
\tcode{ostream_iterator}
class as well as with insert iterators and insert pointers.
\exitnote

\begin{addedblock}
\rSec2[output.iterators]{Output iterators}

\pnum
The \tcode{OutputIterator} concept is a refinement of \tcode{Iterator}~(\ref{iterator.iterators}) and
\tcode{WeakOutputIterator}~(\ref{weakoutput.iterators}).

\indexlibrary{\idxcode{OutputIterator}}%
\begin{codeblock}
  template <class I, class T>
  concept bool OutputIterator =
    WeakOutputIterator<I, T> && Iterator<I>;
\end{codeblock}

\pnum
\enternote Output iterators are used by single-pass
algoritms that write into a bounded range, like \tcode{generate}.
\exitnote

\end{addedblock}

\rSec2[forward.iterators]{Forward iterators}

\begin{removedblock}
\pnum
A class or pointer type
\tcode{X}
satisfies the requirements of a forward iterator if

\begin{itemize}
\item \tcode{X} satisfies the requirements of an input iterator~(\ref{input.iterators}),

\item X satisfies the \tcode{DefaultConstructible}
requirements~(\cxxref{utility.arg.requirements}),

\item if \tcode{X} is a mutable iterator, \tcode{reference} is a reference to \tcode{T};
if \tcode{X} is a const iterator, \tcode{reference} is a reference to \tcode{const T},

\item the expressions in Table~\cxxref{tab:iterator.forward.requirements}
are valid and have the indicated semantics, and

\item objects of type \tcode{X} offer the multi-pass guarantee, described below.
\end{itemize}
\end{removedblock}

\begin{addedblock}
\pnum
The \tcode{ForwardIterator} concept refines \tcode{InputIterator}~(\ref{input.iterators})
and adds the multi-pass guarantee, described below.

\indexlibrary{\idxcode{ForwardIterator}}%
\begin{codeblock}
  template <class I>
  concept bool ForwardIterator =
    InputIterator<I> &&
    Incrementable<I> &&
    Derived<IteratorCategory<I>, forward_iterator_tag>;
\end{codeblock}
\end{addedblock}

\pnum
The domain of \tcode{==} for forward iterators is that of iterators over the same
underlying sequence. However, value-initialized iterators may be compared and
shall compare equal to other value-initialized iterators of the same type.
\enternote value initialized iterators behave as if they refer past the end of
the same empty sequence \exitnote

\pnum
Two dereferenceable iterators \tcode{a} and \tcode{b} of type \tcode{X} offer the
\defn{multi-pass guarantee} if:

\begin{itemize}
\item \tcode{a == b} implies \tcode{++a == ++b} and
\item \tcode{X} is a pointer type or the expression
\tcode{(void)++X(a), *a} is equivalent to the expression \tcode{*a}.
\end{itemize}

\pnum
\enternote
The requirement that
\tcode{a == b}
implies
\tcode{++a == ++b}
(which is not true for input and output iterators)
and the removal of the restrictions on the number of the assignments through
a mutable iterator
(which applies to output iterators)
allows the use of multi-pass one-directional algorithms with forward iterators.
\exitnote

\ednote{Remove Table 109}

\pnum
If \tcode{a} and \tcode{b} are equal, then either \tcode{a} and \tcode{b}
are both dereferenceable
or else neither is dereferenceable.

\pnum
If \tcode{a} and \tcode{b} are both dereferenceable, then \tcode{a == b}
if and only if
\tcode{*a} and \tcode{*b} are bound to the same object.

\rSec2[bidirectional.iterators]{Bidirectional iterators}

\begin{removedblock}
\pnum
A class or pointer type
\tcode{X}
satisfies the requirements of a bidirectional iterator if,
in addition to satisfying the requirements for forward iterators,
the following expressions are valid as shown in Table~\cxxref{tab:iterator.bidirectional.requirements}.
\end{removedblock}

\begin{addedblock}
\pnum
The \tcode{BidirectionalIterator} concept refines \tcode{ForwardIterator}~(\ref{forward.iterators}),
and adds the ability to move an iterator backward as well as forward.

\indexlibrary{\idxcode{BidirectionalIterator}}%
\begin{codeblock}
  template <class I>
  concept bool BidirectionalIterator =
    ForwardIterator<I> &&
    Derived<IteratorCategory<I>, bidirectional_iterator_tag> &&
    requires (I i, I j) {
      { --i };
      requires Same<I&, decltype(--i)>;
      { i-- };
      requires Same<I, decltype(i--)>;
    };
\end{codeblock}
\end{addedblock}

\ednote{Remove table 110}

\begin{addedblock}
\pnum
A bidirectional iterator \tcode{r} is decrementable only if there exists some \tcode{s} such that
\tcode{++s == r}. The expressions \tcode{\dcr{}r} and \tcode{r\dcr{}} are only valid if \tcode{r} is
decrementable.

\pnum
Let \tcode{a} and \tcode{b} be decrementable objects of type \tcode{I}. Then \tcode{I} models
\tcode{BidirectionalIterator} if and only if:

\begin{itemize}
\item \tcode{\&\dcr{}a == \&a}.
\item If \tcode{(a == b) != false}, then \tcode{(a\dcr{} == j) != false}.
\item If \tcode{(a == b) != false}, then \tcode{((a\dcr{}, a) == \dcr{}j) != false}.
\item If \tcode{a} is incrementable and \tcode{(a == b) != false}, then
      \tcode{(\dcr{}(++a) == j) != false}.
\item If \tcode{(a == b) != false}, then \tcode{(++(\dcr{}a) == j) != false}.
\end{itemize}
\end{addedblock}

\begin{removedblock}
\pnum
\enternote
Bidirectional iterators allow algorithms to move iterators backward as well as forward.
\exitnote
\end{removedblock}

\rSec2[random.access.iterators]{Random access iterators}

\begin{removedblock}
\pnum
A class or pointer type
\tcode{X}
satisfies the requirements of a random access iterator if,
in addition to satisfying the requirements for bidirectional iterators,
the following expressions are valid as shown in Table~\cxxref{tab:iterator.random.access.requirements}.
\end{removedblock}

\begin{addedblock}
The \tcode{RandomAccessIterator} concept refines \tcode{BidirectionalIterator}~(\ref{bidirectional.iterators})
and adds support for constant-time advancement with \tcode{+=}, \tcode{+}, and \tcode{-=}, and the
computation of distance in constant time with \tcode{-}. Random access iterators also support array
notation via subscripting.

\indexlibrary{\idxcode{RandomAccessIterator}}%
\begin{codeblock}
  template <class I>
  concept bool RandomAccessIterator =
    BidirectionalIterator<I> &&
    TotallyOrdered<I>() &&
    Derived<IteratorCategory<I>, random_access_iterator_tag> &&
    SizedIteratorRange<I, I> && // see below
    requires (I i, I j, DifferenceType<I> n) {
      { i += n };
      { i + n };
      { n + i };
      { i -= n };
      { i - n };
      requires Same<decltype(i += n), I&>;
      requires Same<decltype(i + n), I>;
      requires Same<decltype(n + i), I>;
      requires Same<decltype(i -= n), I&>;
      requires Same<decltype(i - n), I>;
      { i[n] } -> const ValueType<I>&;
    };
\end{codeblock}
\end{addedblock}

\ednote{Remove Table 111}

\begin{addedblock}
\pnum
Let \tcode{a} and \tcode{b} be valid iterators of type \tcode{I} such that \tcode{b} is reachable
from \tcode{a}. Let \tcode{n} be an object of type \tcode{DifferenceType<I>} such that after
\tcode{n} applications of \tcode{++a}, \tcode{(a == b) != false}. Then \tcode{I} models
\tcode{RandomAccessIterator} if and only if:

\begin{itemize}
\item \tcode{(a += n) == b}.
\item \tcode{\&(a += n) == \&a}.
\item \tcode{(a + n) == (a += n)}.
\item For any two positive integers \tcode{x} and \tcode{y}, if \tcode{a + (x + y)} is valid, then
\tcode{a + (x + y) == (a + x) + y}.
\item \tcode{a + 0 == a}.
\item If \tcode{(a + (n - 1))} is valid, then \tcode{a + n == ++(a + (n - 1))}.
\item \tcode{(b += -n) == a}.
\item \tcode{(b -= n) == a}.
\item \tcode{\&(b -= n) == \&b}.
\item \tcode{(b - n) == (b -= n)}.
\item If \tcode{b} is dereferenceable, then \tcode{a[n]} is valid and is equal to \tcode{*b}.
\end{itemize}

\rSec1[indirectfunctions]{Indirect function requirements}

\rSec2[indirectfunctions.general]{In general}

\pnum
There are several concepts that group requirements of the higher-order algorithms; that is,
algorithms that take functions as arguments.

\ednote{Specifying the algorithms in terms of these indirect callable concepts would ease
the transition should we ever decide to support proxy iterators in the future. See the
Future Work appendix~(\ref{future}).}

\rSec2[functiontype.indirectfunctions]{Function type}

\pnum
The \tcode{FunctionType} is an alias used to turn a callable type~(\cxxref{func.def}) into a function
object type~(\ref{function.objects}).

\indexlibrary{\idxcode{FunctionType}}%
\begin{codeblock}
  // Exposition only
  template <class T>
  auto @\xname{as_fun_obj}@(T&& t, std::true_type) {
    return mem_fn(t);
  }
  template <class T>
  T @\xname{as_fun_obj}@(T&& t, std::false_type) {
    return forward<T>(t);
  }

  template <class T>
  using FunctionType =
    decltype(@\xname{as_fun_obj}@(declval<T>(), is_member_pointer<decay_t<T>>{}));
\end{codeblock}

\rSec2[projected.indirectfunctions]{Projected iterator}

\pnum
The \tcode{Projected} class template is intended for use when specifying the constraints of
higher-order algorithms that accept optional projections. It bundles a \tcode{Readable} type
\tcode{I} and a projection function \tcode{Proj} into a new \tcode{Readable} type whose
\tcode{reference} type is the result of applying \tcode{Proj} to the \tcode{reference} type
of \tcode{I}.

\indexlibrary{\idxcode{Projected}}%
\begin{codeblock}
  template <Readable I, class Proj>
    requires RegularFunction<FunctionType<Proj>, ValueType<I>>
  struct Projected {
    using difference_type = DifferenceType<I>;
    using value_type = decay_t<ResultType<FunctionType<Proj>, ValueType<I>>>;
    ResultType<FunctionType<Proj>, ReferenceType<I>> operator*() const;
  };
\end{codeblock}

\pnum
\enternote \tcode{Projected} is only used to ease constraints specification. Its
member function need not be defined.\exitnote

\rSec2[indirectfunc.indirectcallables]{Indirect callables}

\pnum
The indirect callable concepts are used to constrain those algorithms that accept
callable objects~(\cxxref{func.def}) as arguments.

\indexlibrary{\idxcode{IndirectCallable}}%
\indexlibrary{\idxcode{IndirectRegularCallable}}%
\indexlibrary{\idxcode{IndirectCallablePredicate}}%
\indexlibrary{\idxcode{IndirectCallableRelation}}%
\begin{codeblock}
  template <class F, class...Is>
  concept bool IndirectCallable =
    (Readable<Is> && ...) &&
    Function<FunctionType<F>, ValueType<Is>...>;

  template <class F, class...Is>
  concept bool IndirectRegularCallable =
    (Readable<Is> && ...) &&
    RegularFunction<FunctionType<F>, ValueType<Is>...>;

  template <class F, class...Is>
  concept bool IndirectCallablePredicate =
    (Readable<Is> && ...) &&
    Predicate<FunctionType<F>, ValueType<Is>...>;

  template <class F, class I1, class I2 = I1>
  concept bool IndirectCallableRelation =
    Readable<I1> &&
    Readable<I2> &&
    Relation<FunctionType<F>, ValueType<I1>, ValueType<I2>>();
\end{codeblock}

\rSec1[commmonalgoreq]{Common algorithm requirements}

\rSec2[commmonalgoreq.general]{In general}

\pnum
There are several additional iterator concepts that are commonly applied to families of algorithms.
These group together iterator requirements of algorithm families. There are three relational concepts
for rearrangements: \tcode{Permutable}, \tcode{Mergeable}, and \tcode{Sortable}. There is one
relational concept for comparing values from different sequences: \tcode{IndirectlyComparable}.

\rSec2[indirectlycomparable.commmonalgoreq]{Indirectly comparable iterators}

\pnum
The \tcode{IndirectlyComparable} concept specifies the common requirements of algorithms that
compare values from two different sequences.

\indexlibrary{\idxcode{IndirectlyComparable}}%
\begin{codeblock}
  template <class I1, class I2, class R = equal_to<>, class P1 = identity,
    class P2 = identity>
  concept bool IndirectlyComparable =
    IndirectCallableRelation<R, Projected<I1, P1>, Projected<I2, P2>>;
\end{codeblock}

\rSec2[permutable.commmonalgoreq]{Permutable iterators}

\pnum
The \tcode{Permutable} concept specifies the common requirements of algorithms that reorder
elements in place by moving or swapping them.

\indexlibrary{\idxcode{Permutable}}%
\begin{codeblock}
  template <class I>
  concept bool Permutable =
    ForwardIterator<I> &&
    Semiregular<ValueType<I>> &&
    IndirectlyMovable<I>;
\end{codeblock}

\ednote{\tcode{Semiregular} here overconstrains by adding a default-constructibility requirement.
See Appendix D of the ``The Palo Alto'' report for an alternate design.}

\rSec2[mergeable.commmonalgoreq]{Mergeable iterators}

\pnum
The \tcode{Mergeable} concept describes the requirements of algorithms that merge sorted sequences
into an output sequence.

\indexlibrary{\idxcode{Permutable}}%
\begin{codeblock}
  template <class I1, class I2, class Out,
      class R = less<>, class P1 = identity, class P2 = identity>
  concept bool Mergeable =
    InputIterator<I1> &&
    InputIterator<I2> &&
    WeaklyIncrementable<I2> &&
    IndirectlyCopyable<I1, Out> &&
    IndirectlyCopyable<I2, Out> &&
    IndirectlyComparable<I1, I2, R, P1, P2>;
\end{codeblock}

\pnum
\enternote When \tcode{less<>} is used as the
relation, the value type must model \tcode{TotallyOrdered}.\exitnote

\rSec2[sortable.commmonalgoreq]{Sortable iterators}

\pnum
The \tcode{Sortable} concept describes the common requirements of algorithms that permute sequences
of iterators into an ordered sequence (e.g., \tcode{sort}).

\indexlibrary{\idxcode{Permutable}}%
\begin{codeblock}
  template <class I, class R = less<>, class P = identity>
  concept bool Sortable =
    ForwardIterator<I> &&
    Permutable<I> &&
    IndirectCallableRelation<R, Projected<I, P>>;
\end{codeblock}

\pnum
\enternote When \tcode{less<>} is used as the
relation, the value type must model \tcode{TotallyOrdered}.\exitnote

\rSec1[iteratorranges]{Iterator range requirements}

\rSec2[iteratorrange.iteratorranges]{Iterator range}

The \tcode{IteratorRange} concept defines a pair of types (an
\tcode{Iterator}~(\ref{iterator.iterators}) and a \tcode{Sentinel}), that can be compared for
equality. This concept is the key that allows iterator ranges to be defined by pairs of types
that are not the same.

\indexlibrary{\idxcode{IteratorRange}}%
\begin{codeblock}
  template <class I, class S>
  concept bool IteratorRange =
    Iterator<I> &&
    Regular<S> &&
    EqualityComparable<I, S>();
\end{codeblock}

\pnum
Let {a} be a valid iterator of type \tcode{I} and let \tcode{b} be a valid sentinel of type
\tcode{S}. Then \tcode{I} and \tcode{S} model concept \tcode{IteratorRange} if and only if
\tcode{b} is reachable from \tcode{a}.

\rSec2[sizediteratorrange.iteratorranges]{Sized Iterator range}

The \tcode{SizedIteratorRange} concept refines \tcode{IteratorRange}~(\ref{iteratorrange.iteratorranges})
and allows the use of the \tcode{-} operator to compute the distance
between an \tcode{Iterator}~(\ref{iterator.iterators}) and a \tcode{Sentinel} in constant time.

\indexlibrary{\idxcode{SizedIteratorRange}}%
\begin{codeblock}
  template <class I, class S>
  concept bool SizedIteratorRange =
    Iterator<I> &&
    Regular<S> &&
    IteratorRange<I, S> &&
    requires (I i, S j) {
      { i - i } -> DifferenceType<I>;
      { j - j } -> DifferenceType<I>;
      { i - j } -> DifferenceType<I>;
      { j - i } -> DifferenceType<I>;
    };
\end{codeblock}

\pnum
Let \tcode{a} be a valid iterator of type \tcode{I} and \tcode{b} be a valid sentinel of type
\tcode{S}. Let \tcode{n} be an object of type \tcode{DifferenceType<I>} such that after
\tcode{n} applications of \tcode{++a}, \tcode{(a == b) != false}. Then types \tcode{I} and
\tcode{S} model \tcode{SizedIteratorRange} if and only if:

\begin{itemize}
\item \tcode{(b - a) == n}.
\item \tcode{(a - b) == -n}.
\end{itemize}

\enternote The \tcode{SizedIteratorRange} concept is modeled by pairs of
\tcode{RandomAccessIterator}s(~\ref{random.access.iterators}) and by counted iterators and their
sentinels~(\ref{counted.iterator}).\exitnote

\end{addedblock}

\rSec1[iterator.synopsis]{Header \tcode{<iterator>}\ synopsis}

\indexlibrary{\idxhdr{iterator}}%
\begin{codeblock}
namespace std {
  // \ref{iterator.primitives}, primitives:
  template<class Iterator> @\changed{struct}{using}@ iterator_traits@\added{ = \seebelow}@;
  @\removed{template<class T> struct iterator_traits<T*>;}@

  template<class Category, class T, class Distance = ptrdiff_t,
       class Pointer = T*, class Reference = T&> struct iterator;
\end{codeblock}

\begin{addedblock}
\begin{codeblock}
  template <class, class = void> struct difference_type;
  template <class, class = void> struct value_type;
  template <class, class = void> struct iterator_category;
  template <class WeaklyIncrementable> using DifferenceType
    = typename difference_type<WeaklyIncrementable>::type;
  template <class Readable> using ValueType
    = typename value_type<Readable>::type;
  template <class WeakInputIterator> using IteratorCategory
    = typename iterator_category<WeakInputIterator>::type;

\end{codeblock}
\end{addedblock}
\begin{codeblock}
  @\added{struct weak_input_iterator_tag \{ \};}@
  struct input_iterator_tag @\added{: public weak_input_iterator_tag }@{ };
  struct output_iterator_tag { };
  struct forward_iterator_tag: public input_iterator_tag { };
  struct bidirectional_iterator_tag: public forward_iterator_tag { };
  struct random_access_iterator_tag: public bidirectional_iterator_tag { };

  // \ref{iterator.operations}, iterator operations:
\end{codeblock}
\begin{removedblock}
\begin{codeblock}
  template <class InputIterator, class Distance>
    void advance(InputIterator& i, Distance n);
  template <class InputIterator>
    typename iterator_traits<InputIterator>::difference_type
    distance(InputIterator first, InputIterator last);
  template <class ForwardIterator>
    ForwardIterator next(ForwardIterator x,
      typename std::iterator_traits<ForwardIterator>::difference_type n = 1);
  template <class BidirectionalIterator>
    BidirectionalIterator prev(BidirectionalIterator x,
      typename std::iterator_traits<BidirectionalIterator>::difference_type n = 1);
\end{codeblock}
\end{removedblock}
\begin{addedblock}
\begin{codeblock}
  template <WeakIterator I>
    void advance(I& i, DifferenceType<I> n);
  template <Iterator I, Sentinel<I> S>
    void advance(I& i, S bound);
  template <Iterator I, Sentinel<I> S>
    DifferenceType<I> advance(I& i, DifferenceType<I> n, S bound);
  template <Iterator I, Sentinel<I> S>
    DifferenceType<I> distance(I first, S last);
  template <WeakIterator I>
    I next(I x, DifferenceType<I> n = 1);
  template <Iterator I, Sentinel<I> S>
    I next(I x, S bound);
  template <Iterator I, Sentinel<I> S>
    I next(I x, DifferenceType<I> n, S bound);
  template <BidirectionalIterator I>
    I prev(I x, DifferenceType<I> n = 1);
  template <BidirectionalIterator I>
    I prev(I x, DifferenceType<I> n, I bound);
\end{codeblock}
\end{addedblock}

\begin{codeblock}
  // \ref{predef.iterators}, predefined iterators\added{ and sentinels}:

  @\added{\itshape{\rmfamily{// \ref{reverse.iterators} Reverse iterators}}}@
  template <@\changed{class Iterator}{BidirectionalIterator I}@> class reverse_iterator;

  template <@\changed{class Iterator1}{BidirectionalIterator I1}@, @\changed{class Iterator2}{BidirectionalIterator I2}@>
      @\added{requires EqualityComparable<I1, I2>()}@
    bool operator==(
      const reverse_iterator<@\changed{Iterator1}{I1}@>& x,
      const reverse_iterator<@\changed{Iterator2}{I2}@>& y);
  template <@\changed{class Iterator1}{RandomAccessIterator I1}@, @\changed{class Iterator2}{RandomAccessIterator I2}@>
      @\added{requires TotallyOrdered<I1, I2>()}@
    bool operator<(
      const reverse_iterator<@\changed{Iterator1}{I1}@>& x,
      const reverse_iterator<@\changed{Iterator2}{I2}@>& y);
  template <@\changed{class Iterator1}{BidirectionalIterator I1}@, @\changed{class Iterator2}{BidirectionalIterator I2}@>
      @\added{requires EqualityComparable<I1, I2>()}@
    bool operator!=(
      const reverse_iterator<@\changed{Iterator1}{I1}@>& x,
      const reverse_iterator<@\changed{Iterator2}{I2}@>& y);
  template <@\changed{class Iterator1}{RandomAccessIterator I1}@, @\changed{class Iterator2}{RandomAccessIterator I2}@>
      @\added{requires TotallyOrdered<I1, I2>()}@
    bool operator>(
      const reverse_iterator<@\changed{Iterator1}{I1}@>& x,
      const reverse_iterator<@\changed{Iterator2}{I2}@>& y);
  template <@\changed{class Iterator1}{RandomAccessIterator I1}@, @\changed{class Iterator2}{RandomAccessIterator I2}@>
      @\added{requires TotallyOrdered<I1, I2>()}@
    bool operator>=(
      const reverse_iterator<@\changed{Iterator1}{I1}@>& x,
      const reverse_iterator<@\changed{Iterator2}{I2}@>& y);
  template <@\changed{class Iterator1}{RandomAccessIterator I1}@, @\changed{class Iterator2}{RandomAccessIterator I2}@>
      @\added{requires TotallyOrdered<I1, I2>()}@
    bool operator<=(
      const reverse_iterator<@\changed{Iterator1}{I1}@>& x,
      const reverse_iterator<@\changed{Iterator2}{I2}@>& y);

  template <@\changed{class Iterator1}{BidirectionalIterator I1}@, @\changed{class Iterator2}{BidirectionalIterator I2}@>
      @\added{requires SizedIteratorRange<I2, I1>}@
    @\changed{auto}{DifferenceType<I2>}@ operator-(
      const reverse_iterator<@\changed{Iterator1}{I1}@>& x,
      const reverse_iterator<@\changed{Iterator2}{I2}@>& y)@\removed{ ->decltype(y.base() - x.base())}@;
  template <@\changed{class Iterator}{RandomAccessIterator I}@>
    reverse_iterator<@\changed{Iterator}{I}@>
      operator+(
    @\changed{typename reverse_iterator<Iterator>::difference_type}{DifferenceType<I>}@ n,
    const reverse_iterator<@\changed{Iterator}{I}@>& x);

  template <@\changed{class Iterator}{BidirectionalIterator I}@>
    reverse_iterator<@\changed{Iterator}{I}@> make_reverse_iterator(@\changed{Iterator}{I}@ i);

  @\added{\itshape{\rmfamily{// \ref{insert.iterators} Insert iterators}}}@
  template <class Container> class back_insert_iterator;
  template <class Container>
    back_insert_iterator<Container> back_inserter(Container& x);

  template <class Container> class front_insert_iterator;
  template <class Container>
    front_insert_iterator<Container> front_inserter(Container& x);

  template <class Container> class insert_iterator;
  template <class Container>
    insert_iterator<Container> inserter(Container& x, @\removed{typename Container::iterator}@
      @\added{IteratorType<Container>}@ i);

  @\added{\itshape{\rmfamily{// \ref{move.iterators} Move iterators}}}@
  template <@\changed{class Iterator}{WeakInputIterator I}@> class move_iterator;
  template <@\changed{class Iterator1}{InputIterator I1}@, @\changed{class Iterator2}{InputIterator I2}@>
      @\added{requires EqualityComparable<I1, I2>()}@
    bool operator==(
      const move_iterator<@\changed{Iterator1}{I1}@>& x, const move_iterator<@\changed{Iterator2}{I2}@>& y);
  template <@\changed{class Iterator1}{InputIterator I1}@, @\changed{class Iterator2}{InputIterator I2}@>
      @\added{requires EqualityComparable<I1, I2>()}@
    bool operator!=(
      const move_iterator<@\changed{Iterator1}{I1}@>& x, const move_iterator<@\changed{Iterator2}{I2}@>& y);
  template <@\changed{class Iterator1}{RandomAccessIterator I1}@, @\changed{class Iterator2}{RandomAccessIterator I2}@>
      @\added{requires TotallyOrdered<I1, I2>()}@
    bool operator<(
      const move_iterator<@\changed{Iterator1}{I1}@>& x, const move_iterator<@\changed{Iterator2}{I2}@>& y);
  template <@\changed{class Iterator1}{RandomAccessIterator I1}@, @\changed{class Iterator2}{RandomAccessIterator I2}@>
      @\added{requires TotallyOrdered<I1, I2>()}@
    bool operator<=(
      const move_iterator<@\changed{Iterator1}{I1}@>& x, const move_iterator<@\changed{Iterator2}{I2}@>& y);
  template <@\changed{class Iterator1}{RandomAccessIterator I1}@, @\changed{class Iterator2}{RandomAccessIterator I2}@>
      @\added{requires TotallyOrdered<I1, I2>()}@
    bool operator>(
      const move_iterator<@\changed{Iterator1}{I1}@>& x, const move_iterator<@\changed{Iterator2}{I2}@>& y);
  template <@\changed{class Iterator1}{RandomAccessIterator I1}@, @\changed{class Iterator2}{RandomAccessIterator I2}@>
      @\added{requires TotallyOrdered<I1, I2>()}@
    bool operator>=(
      const move_iterator<@\changed{Iterator1}{I1}@>& x, const move_iterator<@\changed{Iterator2}{I2}@>& y);

  template <@\changed{class Iterator1}{WeakInputIterator I1}@, @\changed{class Iterator2}{WeakInputIterator I2}@>
      @\added{requires SizedIteratorRange<I2, I1>}@
    @\changed{auto}{DifferenceType<I2>}@ operator-(
      const move_iterator<@\changed{Iterator1}{I1}@>& x,
      const move_iterator<@\changed{Iterator2}{I2}@>& y)@\removed{ ->decltype(y.base() - x.base())}@;
  template <@\changed{class Iterator}{RandomAccessIterator I}@>
    move_iterator<@\changed{Iterator}{I}@>
      operator+(
    @\changed{typename move_iterator<Iterator>::difference_type}{DifferenceType<I>}@ n,
    const move_iterator<@\changed{Iterator}{I}@>& x);
  template <@\changed{class Iterator}{WeakInputIterator I}@>
    move_iterator<@\changed{Iterator}{I}@> make_move_iterator(@\changed{Iterator}{I}@ i);
\end{codeblock}

\begin{addedblock}
\begin{codeblock}
  // \ref{common.iterators} Common iterators
  template<typename A, typename B>
  concept bool WeaklyEqualityComparable;  // \expos
  template<Iterator I, Regular S>
  concept bool WeakSentinel;              // \expos

  template <InputIterator I, WeakSentinel<I> S> class common_iterator;
  template <InputIterator I1, WeakSentinel<I1> S1,
            InputIterator I2, WeakSentinel<I2> S2>
    requires EqualityComparable<I1, I2>() && WeaklyEqualityComparable<I1, S2> &&
      WeaklyEqualityComparable<I2, S1>
  bool operator==(
    const common_iterator<I1, S1>& x, const common_iterator<I2, S2>& y);
  template <InputIterator I1, WeakSentinel<I1> S1,
            InputIterator I2, WeakSentinel<I2> S2>
    requires EqualityComparable<I1, I2>() && WeaklyEqualityComparable<I1, S2> &&
      WeaklyEqualityComparable<I2, S1>
  bool operator!=(
    const common_iterator<I1, S1>& x, const common_iterator<I2, S2>& y);

  template <InputIterator I1, WeakSentinel<I1> S1,
            InputIterator I2, WeakSentinel<I2> S2>
    requires SizedIteratorRange<I1, I1> && SizedIteratorRange<I2, I2> &&
      requires (I1 a, I2 b) { {a-b}->DifferenceType<I2>; {b-a}->DifferenceType<I2>; }
      requires (I1 i, S2 s) { {i-s}->DifferenceType<I2>; {s-i}->DifferenceType<I2>; }
      requires (I2 i, S1 s) { {i-s}->DifferenceType<I2>; {s-i}->DifferenceType<I2>; }
  DifferenceType<I2> operator-(
    const common_iterator<I1, S1>& x, const common_iterator<I2, S2>& y);

  // \ref{counted.iterators} Counted iterators and sentinels
  @\ednote{Why not WeakIterator?}@
  template <WeakInputIterator I> class counted_iterator;
  class counted_sentinel;

  template <WeakInputIterator I1, WeakInputIterator I2>
    bool operator==(
      const counted_iterator<I1>& x, const counted_iterator<I2>& y);
  template <WeakInputIterator I>
    bool operator==(
      const counted_iterator<I>& x, counted_sentinel y);
  template <WeakInputIterator I>
    bool operator==(
      counted_sentinel x, const counted_iterator<I>& y);
  bool operator==(counted_sentinel x, counted_sentinel y);
  template <WeakInputIterator I1, WeakInputIterator I2>
    bool operator!=(
      const counted_iterator<I1>& x, const counted_iterator<I2>& y);
  template <WeakInputIterator I>
    bool operator!=(
      const counted_iterator<I>& x, counted_sentinel y);
  template <WeakInputIterator I>
    bool operator!=(
      counted_sentinel x, const counted_iterator<I>& y);
  bool operator!=(counted_sentinel x, counted_sentinel y);

  template <RandomAccessIterator I1, RandomAccessIterator I2>
      requires TotallyOrdered<I1, I2>()
    bool operator<(
      const counted_iterator<I1>& x, const counted_iterator<I2>& y);
  template <RandomAccessIterator I>
    bool operator<(
      const counted_iterator<I>& x, counted_sentinel y);
  template <RandomAccessIterator I>
    bool operator<(
      counted_sentinel x, const counted_iterator<I>& y);
  bool operator<(counted_sentinel x, counted_sentinel y);
  template <RandomAccessIterator I1, RandomAccessIterator I2>
      requires TotallyOrdered<I1, I2>()
    bool operator<=(
      const counted_iterator<I1>& x, const counted_iterator<I2>& y);
  template <RandomAccessIterator I>
    bool operator<=(
      const counted_iterator<I>& x, counted_sentinel y);
  template <RandomAccessIterator I>
    bool operator<=(
      counted_sentinel x, const counted_iterator<I>& y);
  bool operator<=(counted_sentinel x, counted_sentinel y);
  template <RandomAccessIterator I1, RandomAccessIterator I2>
      requires TotallyOrdered<I1, I2>()
    bool operator>(
      const counted_iterator<I1>& x, const counted_iterator<I2>& y);
  template <RandomAccessIterator I>
    bool operator>(
      const counted_iterator<I>& x, counted_sentinel y);
  template <RandomAccessIterator I>
    bool operator>(
      counted_sentinel x, const counted_iterator<I>& y);
  bool operator>(counted_sentinel x, counted_sentinel y);
  template <RandomAccessIterator I1, RandomAccessIterator I2>
      requires TotallyOrdered<I1, I2>()
    bool operator>=(
      const counted_iterator<I1>& x, const counted_iterator<I2>& y);
  template <RandomAccessIterator I>
    bool operator>=(
      const counted_iterator<I>& x, counted_sentinel y);
  template <RandomAccessIterator I>
    bool operator>=(
      counted_sentinel x, const counted_iterator<I>& y);
  bool operator>=(counted_sentinel x, counted_sentinel y);

  template <WeakInputIterator I1, WeakInputIterator I2>
    DifferenceType<I2> operator-(
      const counted_iterator<I1>& x, const counted_iterator<I2>& y);
  template <WeakInputIterator I>
    DifferenceType<I> operator-(
      const counted_iterator<I>& x, counted_sentinel y);
  template <WeakInputIterator I>
    DifferenceType<I> operator-(
      counted_sentinel x, const counted_iterator<I>& y);
  ptrdiff_t operator-(counted_sentinel x, counted_sentinel y);
  template <RandomAccessIterator I>
    counted_iterator<I>
      operator+(DifferenceType<I> n, const counted_iterator<I>& x);
  template <WeakInputIterator I>
    counted_iterator<I> make_counted_iterator(I i, DifferenceType<I> n);

  template <WeakInputIterator I>
    void advance(counted_iterator<I>& i, DifferenceType<I> n);

  // \ref{counted.traits.specializations} \tcode{common_type} specializations
  template<WeakInputIterator I>
    struct common_type<counted_iterator<I>, counted_sentinel>;
  template<WeakInputIterator I>
    struct common_type<counted_sentinel, counted_iterator<I>>;

  // \ref{unreachable.sentinels} Unreachable sentinels
  struct unreachable { };
  template <Iterator I>
    constexpr bool operator==(I const &, unreachable) noexcept;
  template <Iterator I>
    constexpr bool operator==(unreachable, I const &) noexcept;
  constexpr bool operator==(unreachable, unreachable) noexcept;
  template <Iterator I>
    constexpr bool operator!=(I const &, unreachable) noexcept;
  template <Iterator I>
    constexpr bool operator!=(unreachable, I const &) noexcept;
  constexpr bool operator!=(unreachable, unreachable) noexcept;

  // \ref{unreachable.traits.specializations} \tcode{common_type} specializations
  template<Iterator I>
    struct common_type<I, unreachable>;
  template<Iterator I>
    struct common_type<unreachable, I>;
\end{codeblock}
\end{addedblock}
\begin{codeblock}

  // \ref{stream.iterators}, stream iterators:
  template <class T, class charT = char, class traits = char_traits<charT>,
      class Distance = ptrdiff_t>
  class istream_iterator;
  template <class T, class charT, class traits, class Distance>
    bool operator==(const istream_iterator<T,charT,traits,Distance>& x,
            const istream_iterator<T,charT,traits,Distance>& y);
  template <class T, class charT, class traits, class Distance>
    bool operator!=(const istream_iterator<T,charT,traits,Distance>& x,
            const istream_iterator<T,charT,traits,Distance>& y);

  template <class T, class charT = char, class traits = char_traits<charT> >
      class ostream_iterator;

  template<class charT, class traits = char_traits<charT> >
    class istreambuf_iterator;
  template <class charT, class traits>
    bool operator==(const istreambuf_iterator<charT,traits>& a,
            const istreambuf_iterator<charT,traits>& b);
  template <class charT, class traits>
    bool operator!=(const istreambuf_iterator<charT,traits>& a,
            const istreambuf_iterator<charT,traits>& b);

  template <class charT, class traits = char_traits<charT> >
    class ostreambuf_iterator;

  // \ref{iterator.range}, range access:
  template <class C> auto begin(C& c) -> decltype(c.begin());
  template <class C> auto begin(const C& c) -> decltype(c.begin());
  template <class C> auto end(C& c) -> decltype(c.end());
  template <class C> auto end(const C& c) -> decltype(c.end());
  template <class T, size_t N> constexpr T* begin(T (&array)[N]) noexcept;
  template <class T, size_t N> constexpr T* end(T (&array)[N]) noexcept;
  template <class C> constexpr auto cbegin(const C& c) noexcept(noexcept(std::begin(c)))
    -> decltype(std::begin(c));
  template <class C> constexpr auto cend(const C& c) noexcept(noexcept(std::end(c)))
    -> decltype(std::end(c));
  template <class C> auto rbegin(C& c) -> decltype(c.rbegin());
  template <class C> auto rbegin(const C& c) -> decltype(c.rbegin());
  template <class C> auto rend(C& c) -> decltype(c.rend());
  template <class C> auto rend(const C& c) -> decltype(c.rend());
  template <class T, size_t N> reverse_iterator<T*> rbegin(T (&array)[N]);
  template <class T, size_t N> reverse_iterator<T*> rend(T (&array)[N]);
  template <class E> reverse_iterator<const E*> rbegin(initializer_list<E> il);
  template <class E> reverse_iterator<const E*> rend(initializer_list<E> il);
  template <class C> auto crbegin(const C& c) -> decltype(std::rbegin(c));
  template <class C> auto crend(const C& c) -> decltype(std::rend(c));
\end{codeblock}
\begin{addedblock}
\begin{codeblock}
  template <class C> auto size(const C& c) -> decltype(c.size());
  template <class T, size_t N> constexpr size_t begin(T (&array)[N]) noexcept;
  template <class E> size_t size(initializer_list<E> il) noexcept;
\end{codeblock}
\end{addedblock}
\begin{codeblock}
}
\end{codeblock}

\rSec1[iterator.primitives]{Iterator primitives}

\pnum
To simplify the task of defining iterators, the library provides
several classes and functions:

\rSec2[iterator.assoc]{Iterator \textcolor{remclr}{traits}\textcolor{addclr}{associated types}}

\pnum
To implement algorithms only in terms of iterators, it is often necessary to
determine the value and
difference types that correspond to a particular iterator type.
Accordingly, it is required that if
\removed{\tcode{Iterator}
is the type of an iterator}\added{\tcode{WeaklyIncrementable} is the name of a type that models the
WeaklyIncrementable concept(~\ref{weaklyincrementable.iterators}), \tcode{Readable} is the name of a type that
models the Readable concept(~\ref{readable.iterators}), and \tcode{WeakInputIterator} is the name of a
type that models the WeakInputIterator(~\ref{weakinput.iterators}) concept}, the types

\begin{removedblock}
\begin{codeblock}
iterator_traits<Iterator>::difference_type
iterator_traits<Iterator>::value_type
iterator_traits<Iterator>::iterator_category
\end{codeblock}
\end{removedblock}
\begin{addedblock}
\begin{codeblock}
DifferenceType<WeaklyIncrementable>
ValueType<Readable>
IteratorCategory<WeakInputIterator>
\end{codeblock}
\end{addedblock}

be defined as the iterator's difference type, value type and iterator category, respectively.
In addition, the type\removed{s}

\begin{addedblock}
\begin{codeblock}
ReferenceType<Readable>
\end{codeblock}

shall be an alias for \tcode{decltype(*declval<Readable>())}.
\end{addedblock}

\begin{removedblock}
\begin{codeblock}
iterator_traits<Iterator>::reference
iterator_traits<Iterator>::pointer
\end{codeblock}

shall be defined as the iterator's reference and pointer types, that is, for an
iterator object \tcode{a}, the same type as the type of \tcode{*a} and \tcode{a->},
respectively. In the case of an output iterator, the types

\begin{codeblock}
iterator_traits<Iterator>::difference_type
iterator_traits<Iterator>::value_type
iterator_traits<Iterator>::reference
iterator_traits<Iterator>::pointer
\end{codeblock}

may be defined as \tcode{void}.

\pnum
The template
\tcode{iterator_traits<Iterator>}
is defined as

\begin{codeblock}
namespace std {
  template<class Iterator> struct iterator_traits {
    typedef typename Iterator::difference_type difference_type;
    typedef typename Iterator::value_type value_type;
    typedef typename Iterator::pointer pointer;
    typedef typename Iterator::reference reference;
    typedef typename Iterator::iterator_category iterator_category;
  };
}
\end{codeblock}

\pnum
It is specialized for pointers as

\begin{codeblock}
namespace std {
  template<class T> struct iterator_traits<T*> {
    typedef ptrdiff_t difference_type;
    typedef T value_type;
    typedef T* pointer;
    typedef T& reference;
    typedef random_access_iterator_tag iterator_category;
  };
}
\end{codeblock}

and for pointers to const as

\begin{codeblock}
namespace std {
  template<class T> struct iterator_traits<const T*> {
    typedef ptrdiff_t difference_type;
    typedef T value_type;
    typedef const T* pointer;
    typedef const T& reference;
    typedef random_access_iterator_tag iterator_category;
  };
}
\end{codeblock}
\end{removedblock}

\begin{addedblock}
\pnum
\indexlibrary{\idxcode{DifferenceType}}%
\tcode{DifferenceType<T>} is implemented as if:

\indexlibrary{\idxcode{difference_type}}%
\begin{codeblock}
  template <class, class = void> struct difference_type { };
  template <class T> struct difference_type<T*> {
    using type = ptrdiff_t;
  };
  template <> struct difference_type<nullptr_t> {
    using type = ptrdiff_t;
  };
  template <class T> struct difference_type<T[]> {
    using type = ptrdiff_t;
  };
  template <class T, size_t N> struct difference_type<T[N]> {
    using type = ptrdiff_t;
  };
  template <class T>
  struct difference_type<T, void_t<typename T::difference_type>> {
    using type = typename T::difference_type;
  }
  template <class T>
  struct difference_type<T, enable_if_t<is_integral<T>::value>> {
    using type = decltype(declval<T>() - declval<T>());
  };
  template <class T>
    using DifferenceType = typename difference_type<T>::type;
\end{codeblock}

\pnum
Users may specialize \tcode{difference_type} on user-defined types.

\pnum
\indexlibrary{\idxcode{IteratorCategory}}%
\tcode{IteratorCategory<T>} is implemented as if:

\indexlibrary{\idxcode{iterator_category}}%
\begin{codeblock}
  template <class, class = void> struct iterator_category { };
  template <class T> struct iterator_category<T*> {
    using type = random_access_iterator_tag;
  };
  template <class T>
  struct iterator_category<T, void_t<typename T::iterator_category>> {
    using type = typename T::iterator_category;
  };
  template <class T>
    using IteratorCategory = typename iterator_category<T>::type;
\end{codeblock}

\pnum
Users may specialize \tcode{iterator_category} on user-defined types.
\end{addedblock}

\pnum
\enternote
If there is an additional pointer type
\tcode{\,\xname{far}}
such that the difference of two
\tcode{\,\xname{far}}
is of type
\tcode{long},
an implementation may define

\begin{removedblock}
\begin{codeblock}
  template<class T> struct iterator_traits<T @\xname{far}@*> {
    typedef long difference_type;
    typedef T value_type;
    typedef T @\xname{far}@* pointer;
    typedef T @\xname{far}@& reference;
    typedef random_access_iterator_tag iterator_category;
  };
\end{codeblock}
\end{removedblock}
\begin{addedblock}
\begin{codeblock}
  template<class T> struct difference_type<T @\xname{far}@*> {
    using type = long;
  };
  template<class T> struct value_type<T @\xname{far}@*> : remove_cv<T> { };
  template<class T> struct iterator_category<T @\xname{far}@*> {
    using type = random_access_iterator_tag;
  };
\end{codeblock}
\end{addedblock}
\exitnote

\begin{addedblock}
\pnum
For the sake of backwards compatibility, this standard specifies the existence of an \tcode{iterator_traits}
alias that collects an iterator's associated types. It is defined as if:

\indexlibrary{\idxcode{iterator_traits}}%
\begin{codeblock}
  template <WeakInputIterator I, class = void> struct @\xname{pointer_type}@ {
    using type = add_pointer_t<ReferenceType<I>>;
  };
  template <WeakInputIterator I>
  struct @\xname{pointer_type}@<I, void_t<decltype(declval<I>().operator->())>> {
    using type = decltype(declval<I>().operator->());
  };
  template <class> struct @\xname{iterator_traits}@ { };
  template <WeakIterator I> struct @\xname{iterator_traits}@<I> {
    using difference_type = DifferenceType<I>;
    using value_type = void;
    using reference = void;
    using pointer = void;
    using iterator_category = output_iterator_tag;
  };
  template <WeakInputIterator I> struct @\xname{iterator_traits}@<I> {
    using difference_type = DifferenceType<I>;
    using value_type = ValueType<I>;
    using reference = ReferenceType<I>;
    using pointer = typename @\xname{pointer_type}@<I>::type;
    using iterator_category = IteratorCategory<I>;
  };
  template <class I>
    using iterator_traits = @\xname{iterator_traits}@<I>;
\end{codeblock}

\pnum
\enternote
\tcode{iterator_traits} is a template alias to intentionally break code that tries to specialize
it.
\exitnote

\end{addedblock}

\pnum
\enterexample
To implement a generic
\tcode{reverse}
function, a \Cpp program can do the following:

\begin{codeblock}
template <@\removed{class }@BidirectionalIterator@\added{ I}@>
void reverse(@\changed{BidirectionalIterator}{I}@ first, @\changed{BidirectionalIterator}{I}@ last) {
  @\changed{typename iterator_traits<BidirectionalIterator>::difference_type}{DifferenceType<I>}@ n =
    distance(first, last);
  --n;
  while(n > 0) {
    @\changed{typename iterator_traits<BidirectionalIterator>::value_type}{ValueType<I>}@
      tmp = *first;
    *first++ = *--last;
    *last = tmp;
    n -= 2;
  }
}
\end{codeblock}
\exitexample

\rSec2[iterator.basic]{Basic iterator}

\pnum
The
\tcode{iterator}
template may be used as a base class to ease the definition of required types
for new iterators.

\indexlibrary{\idxcode{iterator}}%
\begin{codeblock}
namespace std {
  template<class Category, class T, class Distance = ptrdiff_t,
    class Pointer = T*, class Reference = T&>
  struct iterator {
    typedef T         value_type;
    typedef Distance  difference_type;
    typedef Pointer   pointer;
    typedef Reference reference;
    typedef Category  iterator_category;
  };
}
\end{codeblock}

\begin{addedblock}
\pnum
\enternote The \tcode{Pointer} and \tcode{Reference} template parameters, and the nested \tcode{pointer}
and \tcode{reference} type aliases are for backward compatibility only; they are never used by any
other part of this standard.\exitnote
\end{addedblock}

\rSec2[std.iterator.tags]{Standard iterator tags}

\pnum
\indexlibrary{\idxcode{weak_input_iterator_tag}}%
\indexlibrary{\idxcode{input_iterator_tag}}%
\indexlibrary{\idxcode{output_iterator_tag}}%
\indexlibrary{\idxcode{forward_iterator_tag}}%
\indexlibrary{\idxcode{bidirectional_iterator_tag}}%
\indexlibrary{\idxcode{random_access_iterator_tag}}%
It is often desirable for a
function template specialization
to find out what is the most specific category of its iterator
argument, so that the function can select the most efficient algorithm at compile time.
To facilitate this, the
library introduces
\techterm{category tag}
classes which \changed{are}{can be} used as compile time tags for algorithm selection.
\added{\enternote The preferred way to dispatch to more specialized algorithm implementations is
with concept-based overloading.\exitnote}
\changed{They}{The category tags} are:
\tcode{\added{weak_input_iterator_tag}},
\tcode{input_iterator_tag},
\tcode{output_iterator_tag},
\tcode{forward_iterator_tag},
\tcode{bidirectional_iterator_tag}
and
\tcode{random_access_iterator_tag}.
For every \added{weak input }iterator of type
\tcode{Iterator},
\tcode{\changed{iterator_traits<Iterator>::it\-er\-a\-tor_ca\-te\-go\-ry}{It\-er\-a\-tor\-Ca\-te\-go\-ry<Iterator>}}
shall be defined to be the most specific category tag that describes the
iterator's behavior.

\begin{codeblock}
namespace std {
  @\added{struct weak_input_iterator_tag \{ \};}@
  struct input_iterator_tag@\added{: public weak_input_iterator_tag}@ { };
  struct output_iterator_tag { };
  struct forward_iterator_tag: public input_iterator_tag { };
  struct bidirectional_iterator_tag: public forward_iterator_tag { };
  struct random_access_iterator_tag: public bidirectional_iterator_tag { };
}
\end{codeblock}

\begin{addedblock}
\pnum
\enternote
The \tcode{output_iterator_tag} is provided for the sake of backward compatibility.
\exitnote
\end{addedblock}

\pnum
\indexlibrary{\idxcode{empty}}%
\indexlibrary{\idxcode{weak_input_iterator_tag}}%
\indexlibrary{\idxcode{input_iterator_tag}}%
\indexlibrary{\idxcode{output_iterator_tag}}%
\indexlibrary{\idxcode{forward_iterator_tag}}%
\indexlibrary{\idxcode{bidirectional_iterator_tag}}%
\indexlibrary{\idxcode{random_access_iterator_tag}}%
\enterexample
For a program-defined iterator
\tcode{BinaryTreeIterator},
it could be included
into the bidirectional iterator category by specializing the
\tcode{\removed{iterator_traits}}\tcode{\added{difference_type}}\added{, }\tcode{\added{value_type}}\added{, and }
\tcode{\added{iterator_category}} template\added{s}:

\begin{removedblock}
\begin{codeblock}
template<class T> struct iterator_traits<BinaryTreeIterator<T> > {
  typedef std::ptrdiff_t difference_type;
  typedef T value_type;
  typedef T* pointer;
  typedef T& reference;
  typedef bidirectional_iterator_tag iterator_category;
};
\end{codeblock}
\end{removedblock}
\begin{addedblock}
\begin{codeblock}
template<class T> struct difference_type<BinaryTreeIterator<T> > {
  using type = std::ptrdiff_t;
};
template<class T> struct value_type<BinaryTreeIterator<T> > {
  using type = T;
};
template<class T> struct iterator_category<BinaryTreeIterator<T> > {
  using type = bidirectional_iterator_tag;
};
\end{codeblock}
\end{addedblock}

Typically, however, it would be easier to derive
\tcode{BinaryTreeIterator<T>}
from
\tcode{iterator<bidirectional_iterator_tag,T,ptrdiff_t\removed{,T*,T\&}>}.
\exitexample

\pnum
\enterexample
If
\tcode{evolve()}
is well defined for bidirectional iterators, but can be implemented more
efficiently for random access iterators, then \changed{the}{one possible} implementation is as
follows:

\begin{codeblock}
template <class BidirectionalIterator>
inline void
evolve(BidirectionalIterator first, BidirectionalIterator last) {
  evolve(first, last,
    @\removed{typename iterator_traits<BidirectionalIterator>::iterator_category()}@
    @\added{IteratorCategory<BidirectionalIterator>\{\}}@);
}

template <class BidirectionalIterator>
void evolve(BidirectionalIterator first, BidirectionalIterator last,
  bidirectional_iterator_tag) {
  // more generic, but less efficient algorithm
}

template <class RandomAccessIterator>
void evolve(RandomAccessIterator first, RandomAccessIterator last,
  random_access_iterator_tag) {
  // more efficient, but less generic algorithm
}
\end{codeblock}
\exitexample

\pnum
\enterexample
If a \Cpp program wants to define a bidirectional iterator for some data structure containing
\tcode{double}
and such that it
works on a large memory model of the implementation, it can do so with:

\begin{codeblock}
class MyIterator :
  public iterator<bidirectional_iterator_tag, double, long@\removed{, T*, T\&}@> {
  // code implementing \tcode{++}, etc.
};
\end{codeblock}

\pnum
Then there is no need to specialize the
\tcode{\removed{iterator_traits}}\tcode{\added{difference_type}}\added{, }
\tcode{\added{value_type}}\added{, or } \tcode{\added{iterator_category}} template\added{s}.
\exitexample

\rSec2[iterator.operations]{Iterator operations}

\pnum
Since only \changed{random access iterators}{models of \tcode{RandomAccessIterator}} provide
\added{the }\tcode{+} \changed{and}{operator, and models of \tcode{SizedIteratorRange} provide the}
\tcode{-}
operator\removed{s}, the library provides two
function templates
\tcode{advance}
and
\tcode{distance}.
These
function templates
use
\tcode{+}
and
\tcode{-}
for random access iterators\added{ and sized iterator ranges, respectively} (and are, therefore, constant
time for them); for input, forward and bidirectional iterators they use
\tcode{++}
to provide linear time
implementations.

\indexlibrary{\idxcode{advance}}%
\begin{removedblock}
\begin{itemdecl}
template <class InputIterator, class Distance>
  void advance(InputIterator& i, Distance n);
\end{itemdecl}
\end{removedblock}
\begin{addedblock}
\begin{itemdecl}
template <WeakIterator I>
  void advance(I& i, DifferenceType<I> n);
\end{itemdecl}
\end{addedblock}

\begin{itemdescr}
\pnum
\requires
\tcode{n}
shall be negative only for bidirectional and random access iterators.

\pnum
\effects
Increments (or decrements for negative
\tcode{n})
iterator reference
\tcode{i}
by
\tcode{n}.
\end{itemdescr}

\begin{addedblock}
\begin{itemdecl}
template <Iterator I, Sentinel<I> S>
  void advance(I& i, S bound);
\end{itemdecl}

\begin{itemdescr}
\pnum
\requires
\tcode{bound} shall be reachable from \tcode{i}.

\pnum
\effects
Increments iterator reference \tcode{i} until
\tcode{i == bound}.

\pnum
If \tcode{I} and \tcode{S} are the same type, this function
is constant time.

\pnum
If \tcode{I} and \tcode{S} model the concept \tcode{SizedIteratorRange}, this
function shall dispatch to \tcode{advance(i, bound - i)}.
\end{itemdescr}

\begin{itemdecl}
template <Iterator I, Sentinel<I> S>
  DifferenceType<I> advance(I& i, DifferenceType<I> n, S bound);
\end{itemdecl}

\begin{itemdescr}
\pnum
\requires
\tcode{n}
shall be negative only for bidirectional and random access iterators. If \tcode{n} is
negative, \tcode{i} shall be reachable from \tcode{bound}; otherwise, \tcode{bound}
shall be reachable from \tcode{i}.

\pnum
\effects
Increments (or decrements for negative \tcode{n}) iterator reference \tcode{i} either
\tcode{n} times or until \tcode{i == bound}, whichever comes first.

\pnum
If \tcode{I} and \tcode{S} model \tcode{SizedIteratorRange}:

\begin{itemize}
\item If \tcode{(0 <= n ? n >= $D$ : n <= $D$)} is true, where $D$ is \tcode{bound - i},
this function dispatches to \tcode{advance(i, bound)},
\item Otherwise, this function dispatches to \tcode{advance(i, n)}.
\end{itemize}

\pnum
\returns
\tcode{n - $M$}, where $M$ is the distance from the starting position of \tcode{i} to
the ending position.
\end{itemdescr}
\end{addedblock}

\indexlibrary{\idxcode{distance}}%
\begin{removedblock}
\begin{itemdecl}
  template<class InputIterator>
      typename iterator_traits<InputIterator>::difference_type
         distance(InputIterator first, InputIterator last);
\end{itemdecl}
\end{removedblock}
\begin{addedblock}
\begin{itemdecl}
template <Iterator I, Sentinel<I> S>
  DifferenceType<I> distance(I first, S last);
\end{itemdecl}
\end{addedblock}

\begin{itemdescr}
\pnum
\effects
If \tcode{\changed{InputIterator}{I}}\added{ and \tcode{S}} \changed{meets the
requirements of random access iterator}{model \tcode{SizedIteratorRange}},
returns \tcode{(last - first)}; otherwise, returns
the number of increments needed to get from
\tcode{first}
to
\tcode{last}.

\pnum
\requires
If \tcode{\changed{InputIterator}{I}}\added{ and \tcode{S}} \changed{meets the
requirements of random access iterator}{model \tcode{SizedIteratorRange}},
\tcode{last} shall be reachable from \tcode{first} or \tcode{first} shall be
reachable from \tcode{last}; otherwise,
\tcode{last}
shall be reachable from
\tcode{first}.
\end{itemdescr}

\indexlibrary{\idxcode{next}}%
\begin{removedblock}
\begin{itemdecl}
template <class ForwardIterator>
  ForwardIterator next(ForwardIterator x,
    typename std::iterator_traits<ForwardIterator>::difference_type n = 1);
\end{itemdecl}
\end{removedblock}
\begin{addedblock}
\begin{itemdecl}
template <WeakIterator I>
  I next(I x, DifferenceType<I> n = 1);
\end{itemdecl}
\end{addedblock}

\begin{itemdescr}
\pnum
\effects Equivalent to \tcode{advance(x, n); return x;}
\end{itemdescr}

\begin{addedblock}
\begin{itemdecl}
template <Iterator I, Sentinel<I> S>
  I next(I x, S bound);
\end{itemdecl}

\begin{itemdescr}
\pnum
\effects Equivalent to \tcode{advance(x, bound); return x;}
\end{itemdescr}

\begin{itemdecl}
template <Iterator I, Sentinel<I> S>
  I next(I x, DifferenceType<I> n, S bound);
\end{itemdecl}

\begin{itemdescr}
\pnum
\effects Equivalent to \tcode{advance(x, n, bound); return x;}
\end{itemdescr}
\end{addedblock}

\indexlibrary{\idxcode{prev}}%
\begin{removedblock}
\begin{itemdecl}
template <class BidirectionalIterator>
  BidirectionalIterator prev(BidirectionalIterator x,
    typename std::iterator_traits<BidirectionalIterator>::difference_type n = 1);
\end{itemdecl}
\end{removedblock}
\begin{addedblock}
\begin{itemdecl}
template <BidirectionalIterator I>
  I prev(I x, DifferenceType<I> n = 1);
\end{itemdecl}
\end{addedblock}

\begin{itemdescr}
\pnum
\effects Equivalent to \tcode{advance(x, -n); return x;}
\end{itemdescr}

\begin{addedblock}
\begin{itemdecl}
template <BidirectionalIterator I>
  I prev(I x, DifferenceType<I> n, I bound);
\end{itemdecl}

\begin{itemdescr}
\pnum
\effects Equivalent to \tcode{advance(x, -n, bound); return x;}
\end{itemdescr}
\end{addedblock}

\rSec1[predef.iterators]{Iterator adaptors}

\rSec2[reverse.iterators]{Reverse iterators}

\pnum
Class template \tcode{reverse_iterator} is an iterator adaptor that iterates from the end of the sequence defined by its underlying iterator to the beginning of that sequence.
The fundamental relation between a reverse iterator and its corresponding iterator
\tcode{i}
is established by the identity:
\tcode{\&*(reverse_iterator(i)) == \&*(i - 1)}.

\rSec3[reverse.iterator]{Class template \tcode{reverse_iterator}}

\indexlibrary{\idxcode{reverse_iterator}}%
\begin{codeblock}
namespace std {
  template <@\changed{class Iterator}{BidirectionalIterator I}@>
  class reverse_iterator @\changed{: public}{\{}@
\end{codeblock}\begin{removedblock}\begin{codeblock}
        iterator<typename iterator_traits<Iterator>::iterator_category,
        typename iterator_traits<Iterator>::value_type,
        typename iterator_traits<Iterator>::difference_type,
        typename iterator_traits<Iterator>::pointer,
        typename iterator_traits<Iterator>::reference> {
\end{codeblock}\end{removedblock}\begin{codeblock}
  public:
\end{codeblock}\begin{removedblock}\begin{codeblock}
    typedef Iterator                                            iterator_type;
    typedef typename iterator_traits<Iterator>::difference_type difference_type;
    typedef typename iterator_traits<Iterator>::reference       reference;
    typedef typename iterator_traits<Iterator>::pointer         pointer;
\end{codeblock}\end{removedblock}\begin{addedblock}\begin{codeblock}
    using iterator_type = I;
    using difference_type = DifferenceType<I>;
    using value_type = ValueType<I>;
    using iterator_category = IteratorCategory<I>;
    using reference = ReferenceType<I>;
\end{codeblock}\end{addedblock}\begin{codeblock}
    reverse_iterator()@\added{ = default}@;
    explicit reverse_iterator(@\changed{Iterator}{I}@ x);
    template <@\changed{class}{BidirectionalIterator}@ U>
      @\added{requires Convertible<U, I>}@
    reverse_iterator(const reverse_iterator<U>& u);
    template <@\changed{class}{BidirectionalIterator}@ U>
      @\added{requires Convertible<U, I>}@
    reverse_iterator& operator=(const reverse_iterator<U>& u);

    @\changed{Iterator}{I}@ base() const;      // explicit
    reference operator*() const;
    @\removed{pointer operator->() const;}@

    reverse_iterator& operator++();
    reverse_iterator  operator++(int);
    reverse_iterator& operator--();
    reverse_iterator  operator--(int);

    reverse_iterator  operator+ (difference_type n) const@\removed{;}@
      @\added{requires RandomAccessIterator<I>;}@
    reverse_iterator& operator+=(difference_type n)@\removed{;}@
      @\added{requires RandomAccessIterator<I>;}@
    reverse_iterator  operator- (difference_type n) const@\removed{;}@
      @\added{requires RandomAccessIterator<I>;}@
    reverse_iterator& operator-=(difference_type n)@\removed{;}@
      @\added{requires RandomAccessIterator<I>;}@
    @\unspec@ operator[](difference_type n) const@\removed{;}@
      @\added{requires RandomAccessIterator<I>}@
  protected:
    @\changed{Iterator}{I}@ current;
  };

  template <@\changed{class Iterator1}{BidirectionalIterator I1}@, @\changed{class Iterator2}{BidirectionalIterator I2}@>
      @\added{requires EqualityComparable<I1, I2>()}@
    bool operator==(
      const reverse_iterator<@\changed{Iterator1}{I1}@>& x,
      const reverse_iterator<@\changed{Iterator2}{I2}@>& y);
  template <@\changed{class Iterator1}{RandomAccessIterator I1}@, @\changed{class Iterator2}{RandomAccessIterator I2}@>
      @\added{requires TotallyOrdered<I1, I2>()}@
    bool operator<(
      const reverse_iterator<@\changed{Iterator1}{I1}@>& x,
      const reverse_iterator<@\changed{Iterator2}{I2}@>& y);
  template <@\changed{class Iterator1}{BidirectionalIterator I1}@, @\changed{class Iterator2}{BidirectionalIterator I2}@>
      @\added{requires EqualityComparable<I1, I2>()}@
    bool operator!=(
      const reverse_iterator<@\changed{Iterator1}{I1}@>& x,
      const reverse_iterator<@\changed{Iterator2}{I2}@>& y);
  template <@\changed{class Iterator1}{RandomAccessIterator I1}@, @\changed{class Iterator2}{RandomAccessIterator I2}@>
      @\added{requires TotallyOrdered<I1, I2>()}@
    bool operator>(
      const reverse_iterator<@\changed{Iterator1}{I1}@>& x,
      const reverse_iterator<@\changed{Iterator2}{I2}@>& y);
  template <@\changed{class Iterator1}{RandomAccessIterator I1}@, @\changed{class Iterator2}{RandomAccessIterator I2}@>
      @\added{requires TotallyOrdered<I1, I2>()}@
    bool operator>=(
      const reverse_iterator<@\changed{Iterator1}{I1}@>& x,
      const reverse_iterator<@\changed{Iterator2}{I2}@>& y);
  template <@\changed{class Iterator1}{RandomAccessIterator I1}@, @\changed{class Iterator2}{RandomAccessIterator I2}@>
      @\added{requires TotallyOrdered<I1, I2>()}@
    bool operator<=(
      const reverse_iterator<@\changed{Iterator1}{I1}@>& x,
      const reverse_iterator<@\changed{Iterator2}{I2}@>& y);
  template <@\changed{class Iterator1}{BidirectionalIterator I1}@, @\changed{class Iterator2}{BidirectionalIterator I2}@>
      @\added{requires SizedIteratorRange<I2, I1>}@
    @\changed{auto}{DifferenceType<I2>}@ operator-(
      const reverse_iterator<@\changed{Iterator1}{I1}@>& x,
      const reverse_iterator<@\changed{Iterator2}{I2}@>& y)@\removed{ ->decltype(y.base() - x.base())}@;
  template <@\changed{class Iterator}{RandomAccessIterator I}@>
    reverse_iterator<@\changed{Iterator}{I}@>
      operator+(
    @\changed{typename reverse_iterator<Iterator>::difference_type}{DifferenceType<I>}@ n,
    const reverse_iterator<@\changed{Iterator}{I}@>& x);

  template <@\changed{class Iterator}{BidirectionalIterator I}@>
    reverse_iterator<@\changed{Iterator}{I}@> make_reverse_iterator(@\changed{Iterator}{I}@ i);
}
\end{codeblock}

\begin{removedblock}
\rSec3[reverse.iter.requirements]{\tcode{reverse_iterator} requirements}

\pnum
The template parameter
\tcode{Iterator}
shall meet all the requirements of a Bidirectional Iterator~(\ref{bidirectional.iterators}).

\pnum
Additionally,
\tcode{Iterator}
shall meet the requirements of a Random Access Iterator~(\ref{random.access.iterators})
if any of the members
\tcode{operator+}~(\ref{reverse.iter.op+}),
\tcode{operator-}~(\ref{reverse.iter.op-}),
\tcode{operator+=}~(\ref{reverse.iter.op+=}),
\tcode{operator-=}~(\ref{reverse.iter.op-=}),
\tcode{operator\,[]}~(\ref{reverse.iter.opindex}),
or the global operators
\tcode{operator<}~(\ref{reverse.iter.op<}),
\tcode{operator>}~(\ref{reverse.iter.op>}),\\
\tcode{operator\,<=}~(\ref{reverse.iter.op<=}),
\tcode{operator>=}~(\ref{reverse.iter.op>=}),
\tcode{operator-}~(\ref{reverse.iter.opdiff})
or
\tcode{operator+}~(\ref{reverse.iter.opsum})
are referenced in a way that requires instantiation~(\cxxref{temp.inst}).
\end{removedblock}

\rSec3[reverse.iter.ops]{\tcode{reverse_iterator} operations}

\rSec4[reverse.iter.cons]{\tcode{reverse_iterator} constructor}

\indexlibrary{\idxcode{reverse_iterator}!\tcode{reverse_iterator}}%
\begin{itemdecl}
reverse_iterator()@\added{ = default}@;
\end{itemdecl}

\begin{itemdescr}
\pnum
\effects
\changed{Value}{Default} initializes
\tcode{current}.
Iterator operations applied to the resulting iterator have defined behavior
if and only if the corresponding operations are defined on a
\changed{value}{default}-initialized iterator of type
\tcode{\changed{Iterator}{I}}.\added{If \tcode{I} is a literal type, then this
constructor shall be a trivial constructor.}
\end{itemdescr}

\indexlibrary{\idxcode{reverse_iterator}!constructor}%

\begin{itemdecl}
explicit reverse_iterator(@\changed{Iterator}{I}@ x);
\end{itemdecl}

\begin{itemdescr}
\pnum
\effects
Initializes
\tcode{current}
with \tcode{x}.
\end{itemdescr}

\indexlibrary{\idxcode{reverse_iterator}!constructor}%

\begin{itemdecl}
template <@\changed{class}{BidirectionalIterator}@ U>
  @\added{requires Convertible<U, I>}@
reverse_iterator(const reverse_iterator<U>& u);
\end{itemdecl}

\begin{itemdescr}
\pnum
\effects
Initializes
\tcode{current}
with
\tcode{u.current}.
\end{itemdescr}

\rSec4[reverse.iter.op=]{\tcode{reverse_iterator::operator=}}

\indexlibrary{\idxcode{operator=}!\tcode{reverse_iterator}}%
\begin{itemdecl}
template <@\changed{class}{BidirectionalIterator}@ U>
  @\added{requires Convertible<U, I>}@
reverse_iterator&
  operator=(const reverse_iterator<U>& u);
\end{itemdecl}

\begin{itemdescr}
\pnum
\effects
Assigns \tcode{u.base()} to \tcode{current}.

\pnum
\returns
\tcode{*this}.
\end{itemdescr}

\rSec4[reverse.iter.conv]{Conversion}

\indexlibrary{\idxcode{base}!\idxcode{reverse_iterator}}%
\indexlibrary{\idxcode{reverse_iterator}!\idxcode{base}}%
\begin{itemdecl}
@\changed{Iterator}{I} base() const;          // explicit
\end{itemdecl}

\begin{itemdescr}
\pnum
\returns
\tcode{current}.
\end{itemdescr}

\rSec4[reverse.iter.op.star]{\tcode{operator*}}

\indexlibrary{\idxcode{operator*}!\idxcode{reverse_iterator}}%
\begin{itemdecl}
reference operator*() const;
\end{itemdecl}

\begin{itemdescr}
\pnum
\effects
\begin{codeblock}
Iterator tmp = current;
return *--tmp;
\end{codeblock}

\end{itemdescr}

\begin{removedblock}
\rSec4[reverse.iter.opref]{\tcode{operator->}}

\indexlibrary{\idxcode{operator->}!\idxcode{reverse_iterator}}%
\begin{itemdecl}
pointer operator->() const;
\end{itemdecl}

\begin{itemdescr}
\pnum
\returns \tcode{std::addressof(operator*())}.
\end{itemdescr}
\end{removedblock}

\rSec4[reverse.iter.op++]{\tcode{operator++}}

\indexlibrary{\idxcode{operator++}!\idxcode{reverse_iterator}}%
\begin{itemdecl}
reverse_iterator& operator++();
\end{itemdecl}

\begin{itemdescr}
\pnum
\effects
\tcode{\dcr current;}

\pnum
\returns
\tcode{*this}.
\end{itemdescr}

\indexlibrary{\idxcode{operator++}!\idxcode{reverse_iterator}}%
\indexlibrary{\idxcode{reverse_iterator}!\idxcode{operator++}}%
\begin{itemdecl}
reverse_iterator operator++(int);
\end{itemdecl}

\begin{itemdescr}
\pnum
\effects
\begin{codeblock}
reverse_iterator tmp = *this;
--current;
return tmp;
\end{codeblock}
\end{itemdescr}

\rSec4[reverse.iter.op\dcr]{\tcode{operator\dcr}}

\indexlibrary{\idxcode{operator\dcr}!\idxcode{reverse_iterator}}%
\begin{itemdecl}
reverse_iterator& operator--();
\end{itemdecl}

\begin{itemdescr}
\pnum
\effects
\tcode{++current}

\pnum
\returns
\tcode{*this}.
\end{itemdescr}

\indexlibrary{\idxcode{operator\dcr}!\idxcode{reverse_iterator}}%
\indexlibrary{\idxcode{reverse_iterator}!\idxcode{operator\dcr}}%
\begin{itemdecl}
reverse_iterator operator--(int);
\end{itemdecl}

\begin{itemdescr}
\pnum
\effects
\begin{codeblock}
reverse_iterator tmp = *this;
++current;
return tmp;
\end{codeblock}
\end{itemdescr}

\rSec4[reverse.iter.op+]{\tcode{operator+}}

\indexlibrary{\idxcode{operator+}!\idxcode{reverse_iterator}}%
\begin{itemdecl}
reverse_iterator
operator+(typename reverse_iterator<@\changed{Iterator}{I}@>::difference_type n) const@\removed{;}@
  @\added{requires RandomAccessIterator<I>;}@
\end{itemdecl}

\begin{itemdescr}
\pnum
\returns
\tcode{reverse_iterator(current-n)}.
\end{itemdescr}

\rSec4[reverse.iter.op+=]{\tcode{operator+=}}

\indexlibrary{\idxcode{operator+=}!\idxcode{reverse_iterator}}%
\begin{itemdecl}
reverse_iterator&
operator+=(typename reverse_iterator<@\changed{Iterator}{I}@>::difference_type n)@\removed{;}@
  @\added{requires RandomAccessIterator<I>;}@
\end{itemdecl}

\begin{itemdescr}
\pnum
\effects
\tcode{current -= n;}

\pnum
\returns
\tcode{*this}.
\end{itemdescr}

\rSec4[reverse.iter.op-]{\tcode{operator-}}

\indexlibrary{\idxcode{operator-}!\idxcode{reverse_iterator}}%
\begin{itemdecl}
reverse_iterator
operator-(typename reverse_iterator<@\changed{Iterator}{I}@>::difference_type n) const@\removed{;}@
  @\added{requires RandomAccessIterator<I>;}@
\end{itemdecl}

\begin{itemdescr}
\pnum
\returns
\tcode{reverse_iterator(current+n)}.
\end{itemdescr}

\rSec4[reverse.iter.op-=]{\tcode{operator-=}}

\indexlibrary{\idxcode{operator-=}!\idxcode{reverse_iterator}}%
\begin{itemdecl}
reverse_iterator&
operator-=(typename reverse_iterator<@\changed{Iterator}{I}@>::difference_type n)@\removed{;}@
  @\added{requires RandomAccessIterator<I>;}@
\end{itemdecl}

\begin{itemdescr}
\pnum
\effects
\tcode{current += n;}

\pnum
\returns
\tcode{*this}.
\end{itemdescr}

\rSec4[reverse.iter.opindex]{\tcode{operator[]}}

\indexlibrary{\idxcode{operator[]}!\idxcode{reverse_iterator}}%
\begin{itemdecl}
@\unspec@ operator[](
  typename reverse_iterator<@\changed{Iterator}{I}@>::difference_type n) const@\removed{;}@
    @\added{requires RandomAccessIterator<I>;}@
\end{itemdecl}

\begin{itemdescr}
\pnum
\returns
\tcode{current[-n-1]}.
\end{itemdescr}

\rSec4[reverse.iter.op==]{\tcode{operator==}}

\indexlibrary{\idxcode{operator==}!\idxcode{reverse_iterator}}%
\begin{itemdecl}
template <@\changed{class Iterator1}{BidirectionalIterator I1}@, @\changed{class Iterator2}{BidirectionalIterator I2}@>
    @\added{requires EqualityComparable<I1, I2>()}@
  bool operator==(
    const reverse_iterator<@\changed{Iterator1}{I1}@>& x,
    const reverse_iterator<@\changed{Iterator2}{I2}@>& y);
\end{itemdecl}

\begin{itemdescr}
\pnum
\returns
\tcode{x.current == y.current}.
\end{itemdescr}

\rSec4[reverse.iter.op<]{\tcode{operator<}}

\indexlibrary{\idxcode{operator<}!\idxcode{reverse_iterator}}%
\begin{itemdecl}
template <@\changed{class Iterator1}{RandomAccessIterator I1}@, @\changed{class Iterator2}{RandomAccessIterator I2}@>
    @\added{requires TotallyOrdered<I1, I2>()}@
  bool operator<(
    const reverse_iterator<@\changed{Iterator1}{I1}@>& x,
    const reverse_iterator<@\changed{Iterator2}{I2}@>& y);
\end{itemdecl}

\begin{itemdescr}
\pnum
\returns
\tcode{x.current > y.current}.
\end{itemdescr}

\rSec4[reverse.iter.op!=]{\tcode{operator!=}}

\indexlibrary{\idxcode{operator"!=}!\idxcode{reverse_iterator}}%
\begin{itemdecl}
template <@\changed{class Iterator1}{BidirectionalIterator I1}@, @\changed{class Iterator2}{BidirectionalIterator I2}@>
    @\added{requires EqualityComparable<I1, I2>()}@
  bool operator!=(
    const reverse_iterator<@\changed{Iterator1}{I1}@>& x,
    const reverse_iterator<@\changed{Iterator2}{I2}@>& y);
\end{itemdecl}

\begin{itemdescr}
\pnum
\returns
\tcode{x.current != y.current}.
\end{itemdescr}

\rSec4[reverse.iter.op>]{\tcode{operator>}}

\indexlibrary{\idxcode{operator>}!\idxcode{reverse_iterator}}%
\begin{itemdecl}
template <@\changed{class Iterator1}{RandomAccessIterator I1}@, @\changed{class Iterator2}{RandomAccessIterator I2}@>
    @\added{requires TotallyOrdered<I1, I2>()}@
  bool operator>(
    const reverse_iterator<@\changed{Iterator1}{I1}@>& x,
    const reverse_iterator<@\changed{Iterator2}{I2}@>& y);
\end{itemdecl}

\begin{itemdescr}
\pnum
\returns
\tcode{x.current < y.current}.
\end{itemdescr}

\rSec4[reverse.iter.op>=]{\tcode{operator>=}}

\indexlibrary{\idxcode{operator>=}!\idxcode{reverse_iterator}}%
\begin{itemdecl}
template <@\changed{class Iterator1}{RandomAccessIterator I1}@, @\changed{class Iterator2}{RandomAccessIterator I2}@>
    @\added{requires TotallyOrdered<I1, I2>()}@
  bool operator>=(
    const reverse_iterator<@\changed{Iterator1}{I1}@>& x,
    const reverse_iterator<@\changed{Iterator2}{I2}@>& y);
\end{itemdecl}

\begin{itemdescr}
\pnum
\returns
\tcode{x.current <= y.current}.
\end{itemdescr}

\rSec4[reverse.iter.op<=]{\tcode{operator<=}}

\indexlibrary{\idxcode{operator<=}!\idxcode{reverse_iterator}}%
\begin{itemdecl}
template <@\changed{class Iterator1}{RandomAccessIterator I1}@, @\changed{class Iterator2}{RandomAccessIterator I2}@>
    @\added{requires TotallyOrdered<I1, I2>()}@
  bool operator<=(
    const reverse_iterator<@\changed{Iterator1}{I1}@>& x,
    const reverse_iterator<@\changed{Iterator2}{I2}@>& y);
\end{itemdecl}

\begin{itemdescr}
\pnum
\returns
\tcode{x.current >= y.current}.
\end{itemdescr}

\rSec4[reverse.iter.opdiff]{\tcode{operator-}}

\indexlibrary{\idxcode{operator-}!\idxcode{reverse_iterator}}%
\begin{itemdecl}
template <@\changed{class Iterator1}{BidirectionalIterator I1}@, @\changed{class Iterator2}{BidirectionalIterator I2}@>
    @\added{requires SizedIteratorRange<I2, I1>}@
  @\changed{auto}{DifferenceType<I2>}@ operator-(
    const reverse_iterator<@\changed{Iterator1}{I1}@>& x,
    const reverse_iterator<@\changed{Iterator2}{I2}@>& y)@\removed{ ->decltype(y.base() - x.base())}@;
\end{itemdecl}

\begin{itemdescr}
\pnum
\returns
\tcode{y.current - x.current}.
\end{itemdescr}

\rSec4[reverse.iter.opsum]{\tcode{operator+}}

\indexlibrary{\idxcode{operator+}!\idxcode{reverse_iterator}}%
\begin{itemdecl}
template <@\changed{class Iterator}{RandomAccessIterator I}@>
  reverse_iterator<@\changed{Iterator}{I}@>
    operator+(
  @\changed{typename reverse_iterator<Iterator>::difference_type}{DifferenceType<I>}@ n,
  const reverse_iterator<@\changed{Iterator}{I}@>& x);
\end{itemdecl}

\begin{itemdescr}
\pnum
\returns
\tcode{reverse_iterator<\changed{Iterator}{I}> (x.current - n)}.
\end{itemdescr}

\rSec4[reverse.iter.make]{Non-member function \tcode{make_reverse_iterator()}}

\indexlibrary{\idxcode{reverse_iterator}!\idxcode{make_reverse_iterator}~non-member~function}
\indexlibrary{\idxcode{make_reverse_iterator}}%
\begin{itemdecl}
template <@\changed{class Iterator}{BidirectionalIterator I}@>
  reverse_iterator<@\changed{Iterator}{I}@> make_reverse_iterator(@\changed{Iterator}{I}@ i);
\end{itemdecl}

\begin{itemdescr}
\pnum
\returns
\tcode{reverse_iterator<\changed{Iterator}{I}>(i)}.
\end{itemdescr}

\rSec2[insert.iterators]{Insert iterators}

\pnum
To make it possible to deal with insertion in the same way as writing into an array, a special kind of iterator
adaptors, called
\techterm{insert iterators},
are provided in the library.
With regular iterator classes,

\begin{codeblock}
while (first != last) *result++ = *first++;
\end{codeblock}

causes a range \range{first}{last}
to be copied into a range starting with result.
The same code with
\tcode{result}
being an insert iterator will insert corresponding elements into the container.
This device allows all of the
copying algorithms in the library to work in the
\techterm{insert mode}
instead of the \techterm{regular overwrite} mode.

\pnum
An insert iterator is constructed from a container and possibly one of its iterators pointing to where
insertion takes place if it is neither at the beginning nor at the end of the container.
Insert iterators satisfy the requirements of output iterators.
\tcode{operator*}
returns the insert iterator itself.
The assignment
\tcode{operator=(const T\& x)}
is defined on insert iterators to allow writing into them, it inserts
\tcode{x}
right before where the insert iterator is pointing.
In other words, an insert iterator is like a cursor pointing into the
container where the insertion takes place.
\tcode{back_insert_iterator}
inserts elements at the end of a container,
\tcode{front_insert_iterator}
inserts elements at the beginning of a container, and
\tcode{insert_iterator}
inserts elements where the iterator points to in a container.
\tcode{back_inserter},
\tcode{front_inserter},
and
\tcode{inserter}
are three
functions making the insert iterators out of a container.

\rSec3[back.insert.iterator]{Class template \tcode{back_insert_iterator}}

\indexlibrary{\idxcode{back_insert_iterator}}%
\ednote{Specify this in terms of a Container concept? Or Iterable? Or leave it?}
\begin{codeblock}
namespace std {
  template <class Container>
  class back_insert_iterator @\changed{:}{\{}@
    @\removed{public iterator<output_iterator_tag,void,void,void,void> \}}@
  protected:
    Container* container;

  public:
    @\changed{typedef Container}{using}@ container_type@\added{ = Container}@;
    @\added{using difference_type = ptrdiff_t;}@
    @\added{using iterator_category = output_iterator_tag;}@
    @\added{back_insert_iterator() = default;}@
    explicit back_insert_iterator(Container& x);
    back_insert_iterator<Container>&
      operator=(const typename Container::value_type& value);
    back_insert_iterator<Container>&
      operator=(typename Container::value_type&& value);

    back_insert_iterator<Container>& operator*();
    back_insert_iterator<Container>& operator++();
    back_insert_iterator<Container>  operator++(int);
  };

  template <class Container>
    back_insert_iterator<Container> back_inserter(Container& x);
}
\end{codeblock}

\rSec3[back.insert.iter.ops]{\tcode{back_insert_iterator} operations}

\rSec4[back.insert.iter.cons]{\tcode{back_insert_iterator} constructor}

\indexlibrary{\idxcode{back_insert_iterator}!\idxcode{back_insert_iterator}}%
\begin{addedblock}
\begin{itemdecl}
back_insert_iterator() = default;
\end{itemdecl}

\begin{itemdescr}
\pnum
\effects
Default-initializes
\tcode{container}. This constructor shall be a trivial constructor.
\end{itemdescr}
\end{addedblock}

\indexlibrary{\idxcode{back_insert_iterator}!constructor}%

\begin{itemdecl}
explicit back_insert_iterator(Container& x);
\end{itemdecl}

\begin{itemdescr}
\pnum
\effects
Initializes
\tcode{container}
with \tcode{std::addressof(x)}.
\end{itemdescr}

\rSec4[back.insert.iter.op=]{\tcode{back_insert_iterator::operator=}}

\indexlibrary{\idxcode{operator=}!\idxcode{back_insert_iterator}}%
\begin{itemdecl}
back_insert_iterator<Container>&
  operator=(const typename Container::value_type& value);
\end{itemdecl}

\begin{itemdescr}
\pnum
\effects
\tcode{container->push_back(value);}

\pnum
\returns
\tcode{*this}.
\end{itemdescr}

\indexlibrary{\idxcode{operator=}!\idxcode{back_insert_iterator}}%
\begin{itemdecl}
back_insert_iterator<Container>&
  operator=(typename Container::value_type&& value);
\end{itemdecl}

\begin{itemdescr}
\pnum
\effects
\tcode{container->push_back(std::move(value));}

\pnum
\returns
\tcode{*this}.
\end{itemdescr}

\rSec4[back.insert.iter.op*]{\tcode{back_insert_iterator::operator*}}

\indexlibrary{\idxcode{operator*}!\idxcode{back_insert_iterator}}%
\begin{itemdecl}
back_insert_iterator<Container>& operator*();
\end{itemdecl}

\begin{itemdescr}
\pnum
\returns
\tcode{*this}.
\end{itemdescr}

\rSec4[back.insert.iter.op++]{\tcode{back_insert_iterator::operator++}}

\indexlibrary{\idxcode{operator++}!\idxcode{back_insert_iterator}}%
\begin{itemdecl}
back_insert_iterator<Container>& operator++();
back_insert_iterator<Container>  operator++(int);
\end{itemdecl}

\begin{itemdescr}
\pnum
\returns
\tcode{*this}.
\end{itemdescr}

\rSec4[back.inserter]{ \tcode{back_inserter}}

\indexlibrary{\idxcode{back_inserter}}%
\begin{itemdecl}
template <class Container>
  back_insert_iterator<Container> back_inserter(Container& x);
\end{itemdecl}

\begin{itemdescr}
\pnum
\returns
\tcode{back_insert_iterator<Container>(x)}.
\end{itemdescr}

\rSec3[front.insert.iterator]{Class template \tcode{front_insert_iterator}}

\indexlibrary{\idxcode{front_insert_iterator}}%
\begin{codeblock}
namespace std {
  template <class Container>
  class front_insert_iterator @\changed{:}{\{}@
    @\removed{public iterator<output_iterator_tag,void,void,void,void> \}}@
  protected:
    Container* container;

  public:
    @\changed{typedef Container}{using}@ container_type@\added{ = Container}@;
    @\added{using difference_type = ptrdiff_t;}@
    @\added{using iterator_category = output_iterator_tag;}@
    @\added{front_insert_iterator() = default;}@
    explicit front_insert_iterator(Container& x);
    front_insert_iterator<Container>&
      operator=(const typename Container::value_type& value);
    front_insert_iterator<Container>&
      operator=(typename Container::value_type&& value);

    front_insert_iterator<Container>& operator*();
    front_insert_iterator<Container>& operator++();
    front_insert_iterator<Container>  operator++(int);
  };

  template <class Container>
    front_insert_iterator<Container> front_inserter(Container& x);
}
\end{codeblock}

\rSec3[front.insert.iter.ops]{\tcode{front_insert_iterator} operations}

\rSec4[front.insert.iter.cons]{\tcode{front_insert_iterator} constructor}

\indexlibrary{\idxcode{front_insert_iterator}!\idxcode{front_insert_iterator}}%
\begin{addedblock}
\begin{itemdecl}
front_insert_iterator() = default;
\end{itemdecl}

\begin{itemdescr}
\pnum
\effects
Default-initializes
\tcode{container}. This constructor shall be a trivial constructor.
\end{itemdescr}
\end{addedblock}

\indexlibrary{\idxcode{front_insert_iterator}!constructor}%

\begin{itemdecl}
explicit front_insert_iterator(Container& x);
\end{itemdecl}

\begin{itemdescr}
\pnum
\effects
Initializes
\tcode{container}
with \tcode{std::addressof(x)}.
\end{itemdescr}

\rSec4[front.insert.iter.op=]{\tcode{front_insert_iterator::operator=}}

\indexlibrary{\idxcode{operator=}!\idxcode{front_insert_iterator}}%
\begin{itemdecl}
front_insert_iterator<Container>&
  operator=(const typename Container::value_type& value);
\end{itemdecl}

\begin{itemdescr}
\pnum
\effects
\tcode{container->push_front(value);}

\pnum
\returns
\tcode{*this}.
\end{itemdescr}

\indexlibrary{\idxcode{operator=}!\idxcode{front_insert_iterator}}%
\begin{itemdecl}
front_insert_iterator<Container>&
  operator=(typename Container::value_type&& value);
\end{itemdecl}

\begin{itemdescr}
\pnum
\effects
\tcode{container->push_front(std::move(value));}

\pnum
\returns
\tcode{*this}.
\end{itemdescr}

\rSec4[front.insert.iter.op*]{\tcode{front_insert_iterator::operator*}}

\indexlibrary{\idxcode{operator*}!\idxcode{front_insert_iterator}}%
\begin{itemdecl}
front_insert_iterator<Container>& operator*();
\end{itemdecl}

\begin{itemdescr}
\pnum
\returns
\tcode{*this}.
\end{itemdescr}

\rSec4[front.insert.iter.op++]{\tcode{front_insert_iterator::operator++}}

\indexlibrary{\idxcode{operator++}!\idxcode{front_insert_iterator}}%
\begin{itemdecl}
front_insert_iterator<Container>& operator++();
front_insert_iterator<Container>  operator++(int);
\end{itemdecl}

\begin{itemdescr}
\pnum
\returns
\tcode{*this}.
\end{itemdescr}

\rSec4[front.inserter]{\tcode{front_inserter}}

\indexlibrary{\idxcode{front_inserter}}%
\begin{itemdecl}
template <class Container>
  front_insert_iterator<Container> front_inserter(Container& x);
\end{itemdecl}

\begin{itemdescr}
\pnum
\returns
\tcode{front_insert_iterator<Container>(x)}.
\end{itemdescr}

\rSec3[insert.iterator]{Class template \tcode{insert_iterator}}

\indexlibrary{\idxcode{insert_iterator}}%
\begin{codeblock}
namespace std {
  template <class Container>
  class insert_iterator @\changed{:}{\{}@
    @\removed{public iterator<output_iterator_tag,void,void,void,void> \}}@
  protected:
    Container* container;
    typename Container::iterator iter;

  public:
    @\changed{typedef Container}{using}@ container_type@\added{ = Container}@;
    @\added{using difference_type = ptrdiff_t;}@
    @\added{using iterator_category = output_iterator_tag;}@
    @\added{insert_iterator() = default;}@
    insert_iterator(Container& x, typename Container::iterator i);
    insert_iterator<Container>&
      operator=(const typename Container::value_type& value);
    insert_iterator<Container>&
      operator=(typename Container::value_type&& value);

    insert_iterator<Container>& operator*();
    insert_iterator<Container>& operator++();
    insert_iterator<Container>& operator++(int);
  };

  template <class Container>
    insert_iterator<Container> inserter(Container& x, typename Container::iterator i);
}
\end{codeblock}

\rSec3[insert.iter.ops]{\tcode{insert_iterator} operations}

\rSec4[insert.iter.cons]{\tcode{insert_iterator} constructor}

\indexlibrary{\idxcode{insert_iterator}!\idxcode{insert_iterator}}%
\begin{addedblock}
\begin{itemdecl}
insert_iterator() = default;
\end{itemdecl}

\begin{itemdescr}
\pnum
\effects
Default-initializes
\tcode{container} and \tcode{iter}. If \tcode{Container::iterator} is a literal
type, then this constructor shall be a trivial constructor.
\end{itemdescr}
\end{addedblock}

\indexlibrary{\idxcode{insert_iterator}!constructor}%

\begin{itemdecl}
insert_iterator(Container& x, typename Container::iterator i);
\end{itemdecl}

\begin{itemdescr}
\pnum
\effects
Initializes
\tcode{container}
with \tcode{std::addressof(x)} and
\tcode{iter}
with \tcode{i}.
\end{itemdescr}

\rSec4[insert.iter.op=]{\tcode{insert_iterator::operator=}}

\indexlibrary{\idxcode{operator=}!\idxcode{insert_iterator}}%
\begin{itemdecl}
insert_iterator<Container>&
  operator=(const typename Container::value_type& value);
\end{itemdecl}

\begin{itemdescr}
\pnum
\effects
\begin{codeblock}
iter = container->insert(iter, value);
++iter;
\end{codeblock}

\pnum
\returns
\tcode{*this}.
\end{itemdescr}

\indexlibrary{\idxcode{operator=}!\idxcode{insert_iterator}}%
\begin{itemdecl}
insert_iterator<Container>&
  operator=(typename Container::value_type&& value);
\end{itemdecl}

\begin{itemdescr}
\pnum
\effects
\begin{codeblock}
iter = container->insert(iter, std::move(value));
++iter;
\end{codeblock}

\pnum
\returns
\tcode{*this}.
\end{itemdescr}

\rSec4[insert.iter.op*]{\tcode{insert_iterator::operator*}}

\indexlibrary{\idxcode{operator*}!\idxcode{insert_iterator}}%
\begin{itemdecl}
insert_iterator<Container>& operator*();
\end{itemdecl}

\begin{itemdescr}
\pnum
\returns
\tcode{*this}.
\end{itemdescr}

\rSec4[insert.iter.op++]{\tcode{insert_iterator::operator++}}

\indexlibrary{\idxcode{operator++}!\idxcode{insert_iterator}}%
\begin{itemdecl}
insert_iterator<Container>& operator++();
insert_iterator<Container>& operator++(int);
\end{itemdecl}

\begin{itemdescr}
\pnum
\returns
\tcode{*this}.
\end{itemdescr}

\rSec4[inserter]{\tcode{inserter}}

\indexlibrary{\idxcode{inserter}}%
\begin{itemdecl}
template <class Container>
  insert_iterator<Container> inserter(Container& x, typename Container::iterator i);
\end{itemdecl}

\begin{itemdescr}
\pnum
\returns
\tcode{insert_iterator<Container>(x, i)}.
\end{itemdescr}

\rSec2[move.iterators]{Move iterators}

\pnum
Class template \tcode{move_iterator} is an iterator adaptor
with the same behavior as the underlying iterator except that its
indirection operator implicitly converts the value returned by the
underlying iterator's indirection operator to an rvalue reference.
Some generic algorithms can be called with move iterators to replace
copying with moving. \ednote{Pretty sure this is untrue now given how
the algorithms that do copying are constrained with IndirectlyCopyable.}

\pnum
\enterexample

\begin{codeblock}
list<string> s;
// populate the list \tcode{s}
vector<string> v1(s.begin(), s.end());          // copies strings into \tcode{v1}
vector<string> v2(make_move_iterator(s.begin()),
                  make_move_iterator(s.end())); // moves strings into \tcode{v2}
\end{codeblock}

\exitexample

\rSec3[move.iterator]{Class template \tcode{move_iterator}}

\indexlibrary{\idxcode{move_iterator}}%
\begin{codeblock}
namespace std {
  template <@\changed{class Iterator}{WeakInputIterator I}@>
    @\added{requires Same<ReferenceType<I>, ValueType<I>\&>}@
  class move_iterator {
  public:
\end{codeblock}\begin{removedblock}\begin{codeblock}
    typedef Iterator                                              iterator_type;
    typedef typename iterator_traits<Iterator>::difference_type   difference_type;
    typedef Iterator                                              pointer;
    typedef typename iterator_traits<Iterator>::value_type        value_type;
    typedef typename iterator_traits<Iterator>::iterator_category iterator_category;
    typedef value_type&&                                          reference;
\end{codeblock}\end{removedblock}\begin{addedblock}\begin{codeblock}
    using iterator_type = I;
    using difference_type = DifferenceType<I>;
    using value_type = ValueType<I>;
    using iterator_category = IteratorCategory<I>;
    using reference = ValueType<I>&&;
\end{codeblock}\end{addedblock}\begin{codeblock}

    move_iterator()@\added{ = default}@;
    explicit move_iterator(@\changed{Iterator}{I}@ i);
    template <@\changed{class}{WeakInputIterator}@ U>
      @\added{requires Convertible<U, I>}@
    move_iterator(const move_iterator<U>& u);
    template <@\changed{class}{WeakInputIterator}@ U>
      @\added{requires Convertible<U, I>}@
    move_iterator& operator=(const move_iterator<U>& u);

    iterator_type base() const;
    reference operator*() const;
    @\removed{pointer operator->() const;}@

    move_iterator& operator++();
    move_iterator operator++(int);
    move_iterator& operator--()@\removed{;}@
      @\added{requires BidirectionalIterator<I>;}@
    @\added{requires BidirectionalIterator<I>}@
      move_iterator operator--(int)@\removed{;}@

    move_iterator operator+(difference_type n) const@\removed{;}@
      @\added{requires RandomAccessIterator<I>;}@
    move_iterator& operator+=(difference_type n)@\removed{;}@
      @\added{requires RandomAccessIterator<I>;}@
    move_iterator operator-(difference_type n) const@\removed{;}@
      @\added{requires RandomAccessIterator<I>;}@
    move_iterator& operator-=(difference_type n)@\removed{;}@
      @\added{requires RandomAccessIterator<I>;}@
    @\unspec@ operator[](difference_type n) const@\removed{;}@
      @\added{requires RandomAccessIterator<I>;}@

  private:
    @\changed{Iterator}{I}@ current;   // \expos
  };

  template <@\changed{class Iterator1}{InputIterator I1}@, @\changed{class Iterator2}{InputIterator I2}@>
      @\added{requires EqualityComparable<I1, I2>()}@
    bool operator==(
      const move_iterator<@\changed{Iterator1}{I1}@>& x, const move_iterator<@\changed{Iterator2}{I2}@>& y);
  template <@\changed{class Iterator1}{InputIterator I1}@, @\changed{class Iterator2}{InputIterator I2}@>
      @\added{requires EqualityComparable<I1, I2>()}@
    bool operator!=(
      const move_iterator<@\changed{Iterator1}{I1}@>& x, const move_iterator<@\changed{Iterator2}{I2}@>& y);
  template <@\changed{class Iterator1}{RandomAccessIterator I1}@, @\changed{class Iterator2}{RandomAccessIterator I2}@>
      @\added{requires TotallyOrdered<I1, I2>()}@
    bool operator<(
      const move_iterator<@\changed{Iterator1}{I1}@>& x, const move_iterator<@\changed{Iterator2}{I2}@>& y);
  template <@\changed{class Iterator1}{RandomAccessIterator I1}@, @\changed{class Iterator2}{RandomAccessIterator I2}@>
      @\added{requires TotallyOrdered<I1, I2>()}@
    bool operator<=(
      const move_iterator<@\changed{Iterator1}{I1}@>& x, const move_iterator<@\changed{Iterator2}{I2}@>& y);
  template <@\changed{class Iterator1}{RandomAccessIterator I1}@, @\changed{class Iterator2}{RandomAccessIterator I2}@>
      @\added{requires TotallyOrdered<I1, I2>()}@
    bool operator>(
      const move_iterator<@\changed{Iterator1}{I1}@>& x, const move_iterator<@\changed{Iterator2}{I2}@>& y);
  template <@\changed{class Iterator1}{RandomAccessIterator I1}@, @\changed{class Iterator2}{RandomAccessIterator I2}@>
      @\added{requires TotallyOrdered<I1, I2>()}@
    bool operator>=(
      const move_iterator<@\changed{Iterator1}{I1}@>& x, const move_iterator<@\changed{Iterator2}{I2}@>& y);

  template <@\changed{class Iterator1}{WeakInputIterator I1}@, @\changed{class Iterator2}{WeakInputIterator I2}@>
      @\added{requires SizedIteratorRange<I2, I1>}@
    @\changed{auto}{DifferenceType<I2>}@ operator-(
      const move_iterator<@\changed{Iterator1}{I1}@>& x,
      const move_iterator<@\changed{Iterator2}{I2}@>& y)@\removed{ ->decltype(y.base() - x.base())}@;
  template <@\changed{class Iterator}{RandomAccessIterator I}@>
    move_iterator<@\changed{Iterator}{I}@>
      operator+(
        @\changed{typename move_iterator<Iterator>::difference_type}{DifferenceType<I>}@ n,
        const move_iterator<@\changed{Iterator}{I}@>& x);
  template <@\changed{class Iterator}{WeakInputIterator I}@>
    move_iterator<@\changed{Iterator}{I}@> make_move_iterator(@\changed{Iterator}{I}@ i);
}
\end{codeblock}

\begin{removedblock}
\rSec3[move.iter.requirements]{\tcode{move_iterator} requirements}

\pnum
The template parameter \tcode{Iterator} shall meet
the requirements for an Input Iterator~(\ref{input.iterators}).
Additionally, if any of the bidirectional or random access traversal
functions are instantiated, the template parameter shall meet the
requirements for a Bidirectional Iterator~(\ref{bidirectional.iterators})
or a Random Access Iterator~(\ref{random.access.iterators}), respectively.
\end{removedblock}

\rSec3[move.iter.ops]{\tcode{move_iterator} operations}

\rSec4[move.iter.op.const]{\tcode{move_iterator} constructors}

\indexlibrary{\idxcode{move_iterator}!\idxcode{move_iterator}}%
\begin{itemdecl}
move_iterator()@\added{ = default}@;
\end{itemdecl}

\begin{itemdescr}
\pnum
\effects Constructs a \tcode{move_iterator}, \changed{value}{default}
initializing \tcode{current}. Iterator operations applied to the resulting
iterator have defined behavior if and only if the corresponding operations are defined
on a \changed{value}{default}-initialized iterator of type \tcode{\changed{Iterator}{I}}.
\end{itemdescr}


\indexlibrary{\idxcode{move_iterator}!constructor}%
\begin{itemdecl}
explicit move_iterator(@\changed{Iterator}{I}@ i);
\end{itemdecl}

\begin{itemdescr}
\pnum
\effects Constructs a \tcode{move_iterator}, initializing
\tcode{current} with \tcode{i}.
\end{itemdescr}


\indexlibrary{\idxcode{move_iterator}!constructor}%
\begin{itemdecl}
template <@\changed{class}{WeakInputIterator}@ U>
  @\added{requires Convertible<U, I>}@
move_iterator(const move_iterator<U>& u);
\end{itemdecl}

\begin{itemdescr}
\pnum
\effects Constructs a \tcode{move_iterator}, initializing
\tcode{current} with \tcode{u.base()}.

\begin{removedblock}
\pnum
\requires \tcode{U} shall be convertible to
\tcode{Iterator}.
\end{removedblock}
\end{itemdescr}

\rSec4[move.iter.op=]{\tcode{move_iterator::operator=}}

\indexlibrary{\idxcode{operator=}!\idxcode{move_iterator}}%
\indexlibrary{\idxcode{move_iterator}!\idxcode{operator=}}%
\begin{itemdecl}
template <@\changed{class}{WeakInputIterator}@ U>
  @\added{requires Convertible<U, I>}@
move_iterator& operator=(const move_iterator<U>& u);
\end{itemdecl}

\begin{itemdescr}
\pnum
\effects Assigns \tcode{u.base()} to
\tcode{current}.

\begin{removedblock}
\pnum
\requires \tcode{U} shall be convertible to
\tcode{Iterator}.
\end{removedblock}
\end{itemdescr}

\rSec4[move.iter.op.conv]{\tcode{move_iterator} conversion}

\indexlibrary{\idxcode{base}!\idxcode{move_iterator}}%
\indexlibrary{\idxcode{move_iterator}!\idxcode{base}}%
\begin{itemdecl}
@\changed{Iterator}{I} base() const;
\end{itemdecl}

\begin{itemdescr}
\pnum
\returns \tcode{current}.
\end{itemdescr}

\rSec4[move.iter.op.star]{\tcode{move_iterator::operator*}}

\indexlibrary{\idxcode{operator*}!\idxcode{move_iterator}}%
\indexlibrary{\idxcode{move_iterator}!\idxcode{operator*}}%
\begin{itemdecl}
reference operator*() const;
\end{itemdecl}

\begin{itemdescr}
\pnum
\returns \tcode{std::move(*current)}.
\end{itemdescr}

\begin{removedblock}
\rSec4[move.iter.op.ref]{\tcode{move_iterator::operator->}}

\indexlibrary{\idxcode{operator->}!\idxcode{move_iterator}}%
\indexlibrary{\idxcode{move_iterator}!\idxcode{operator->}}%
\begin{itemdecl}
pointer operator->() const;
\end{itemdecl}

\begin{itemdescr}
\pnum
\returns \tcode{current}.
\end{itemdescr}
\end{removedblock}

\rSec4[move.iter.op.incr]{\tcode{move_iterator::operator++}}

\indexlibrary{\idxcode{operator++}!\idxcode{move_iterator}}%
\indexlibrary{\idxcode{move_iterator}!\idxcode{operator++}}%
\begin{itemdecl}
move_iterator& operator++();
\end{itemdecl}

\begin{itemdescr}
\pnum
\effects \tcode{++current}.

\pnum
\returns \tcode{*this}.
\end{itemdescr}

\indexlibrary{\idxcode{operator++}!\idxcode{move_iterator}}%
\indexlibrary{\idxcode{move_iterator}!\idxcode{operator++}}%
\begin{itemdecl}
move_iterator operator++(int);
\end{itemdecl}

\begin{itemdescr}
\pnum
\effects
\begin{codeblock}
move_iterator tmp = *this;
++current;
return tmp;
\end{codeblock}
\end{itemdescr}

\rSec4[move.iter.op.decr]{\tcode{move_iterator::operator-{-}}}

\indexlibrary{\idxcode{operator\dcr}!\idxcode{move_iterator}}%
\indexlibrary{\idxcode{move_iterator}!\idxcode{operator\dcr}}%
\begin{itemdecl}
move_iterator& operator--()@\removed{;}@
  @\added{requires BidirectionalIterator<I>;}@
\end{itemdecl}

\begin{itemdescr}
\pnum
\effects \tcode{\dcr{}current}.

\pnum
\returns \tcode{*this}.
\end{itemdescr}

\indexlibrary{\idxcode{operator\dcr}!\idxcode{move_iterator}}%
\indexlibrary{\idxcode{move_iterator}!\idxcode{operator\dcr}}%
\begin{itemdecl}
move_iterator operator--(int)@\removed{;}@
  @\added{requires BidirectionalIterator<I>;}@
\end{itemdecl}

\begin{itemdescr}
\pnum
\effects
\begin{codeblock}
move_iterator tmp = *this;
--current;
return tmp;
\end{codeblock}
\end{itemdescr}

\rSec4[move.iter.op.+]{\tcode{move_iterator::operator+}}

\indexlibrary{\idxcode{operator+}!\idxcode{move_iterator}}%
\indexlibrary{\idxcode{move_iterator}!\idxcode{operator+}}%
\begin{itemdecl}
move_iterator operator+(difference_type n) const@\removed{;}@
  @\added{requires RandomAccessIterator<I>;}@
\end{itemdecl}

\begin{itemdescr}
\pnum
\returns \tcode{move_iterator(current + n)}.
\end{itemdescr}

\rSec4[move.iter.op.+=]{\tcode{move_iterator::operator+=}}

\indexlibrary{\idxcode{operator+=}!\idxcode{move_iterator}}%
\indexlibrary{\idxcode{move_iterator}!\idxcode{operator+=}}%
\begin{itemdecl}
move_iterator& operator+=(difference_type n)@\removed{;}@
  @\added{requires RandomAccessIterator<I>;}@
\end{itemdecl}

\begin{itemdescr}
\pnum
\effects \tcode{current += n}.

\pnum
\returns \tcode{*this}.
\end{itemdescr}

\rSec4[move.iter.op.-]{\tcode{move_iterator::operator-}}

\indexlibrary{\idxcode{operator-}!\idxcode{move_iterator}}%
\indexlibrary{\idxcode{move_iterator}!\idxcode{operator-}}%
\begin{itemdecl}
move_iterator operator-(difference_type n) const@\removed{;}@
  @\added{requires RandomAccessIterator<I>;}@
\end{itemdecl}

\begin{itemdescr}
\pnum
\returns \tcode{move_iterator(current - n)}.
\end{itemdescr}

\rSec4[move.iter.op.-=]{\tcode{move_iterator::operator-=}}

\indexlibrary{\idxcode{operator-=}!\idxcode{move_iterator}}%
\indexlibrary{\idxcode{move_iterator}!\idxcode{operator-=}}%
\begin{itemdecl}
move_iterator& operator-=(difference_type n)@\removed{;}@
  @\added{requires RandomAccessIterator<I>;}@
\end{itemdecl}

\begin{itemdescr}
\pnum
\effects \tcode{current -= n}.

\pnum
\returns \tcode{*this}.
\end{itemdescr}

\rSec4[move.iter.op.index]{\tcode{move_iterator::operator[]}}

\indexlibrary{\idxcode{operator[]}!\idxcode{move_iterator}}%
\indexlibrary{\idxcode{move_iterator}!\idxcode{operator[]}}%
\begin{itemdecl}
@\unspec@ operator[](difference_type n) const@\removed{;}@
  @\added{requires RandomAccessIterator<I>;}@
\end{itemdecl}

\begin{itemdescr}
\pnum
\returns \tcode{std::move(current[n])}.
\end{itemdescr}

\rSec4[move.iter.op.comp]{\tcode{move_iterator} comparisons}

\indexlibrary{\idxcode{operator==}!\idxcode{move_iterator}}%
\indexlibrary{\idxcode{move_iterator}!\idxcode{operator==}}%
\begin{itemdecl}
template <@\changed{class Iterator1}{InputIterator I1}@, @\changed{class Iterator2}{InputIterator I2}@>
    @\added{requires EqualityComparable<I1, I2>()}@
  bool operator==(
    const move_iterator<@\changed{Iterator1}{I1}@>& x, const move_iterator<@\changed{Iterator2}{I2}@>& y);
\end{itemdecl}

\begin{itemdescr}
\pnum
\returns \tcode{x.base() == y.base()}.
\end{itemdescr}

\indexlibrary{\idxcode{operator"!=}!\idxcode{move_iterator}}%
\indexlibrary{\idxcode{move_iterator}!\idxcode{operator"!=}}%
\begin{itemdecl}
template <@\changed{class Iterator1}{InputIterator I1}@, @\changed{class Iterator2}{InputIterator I2}@>
    @\added{requires EqualityComparable<I1, I2>()}@
  bool operator!=(
    const move_iterator<@\changed{Iterator1}{I1}@>& x, const move_iterator<@\changed{Iterator2}{I2}@>& y);
\end{itemdecl}

\begin{itemdescr}
\pnum
\returns \tcode{!(x == y)}.
\end{itemdescr}

\indexlibrary{\idxcode{operator<}!\idxcode{move_iterator}}%
\indexlibrary{\idxcode{move_iterator}!\idxcode{operator<}}%
\begin{itemdecl}
template <@\changed{class Iterator1}{RandomAccessIterator I1}@, @\changed{class Iterator2}{RandomAccessIterator I2}@>
    @\added{requires TotallyOrdered<I1, I2>()}@
  bool operator<(
    const move_iterator<@\changed{Iterator1}{I1}@>& x, const move_iterator<@\changed{Iterator2}{I2}@>& y);
\end{itemdecl}

\begin{itemdescr}
\pnum
\returns \tcode{x.base() < y.base()}.
\end{itemdescr}

\indexlibrary{\idxcode{operator<=}!\idxcode{move_iterator}}%
\indexlibrary{\idxcode{move_iterator}!\idxcode{operator<=}}%
\begin{itemdecl}
template <@\changed{class Iterator1}{RandomAccessIterator I1}@, @\changed{class Iterator2}{RandomAccessIterator I2}@>
    @\added{requires TotallyOrdered<I1, I2>()}@
  bool operator<=(
    const move_iterator<@\changed{Iterator1}{I1}@>& x, const move_iterator<@\changed{Iterator2}{I2}@>& y);
\end{itemdecl}

\begin{itemdescr}
\pnum
\returns \tcode{!(y < x)}.
\end{itemdescr}

\indexlibrary{\idxcode{operator>}!\idxcode{move_iterator}}%
\indexlibrary{\idxcode{move_iterator}!\idxcode{operator>}}%
\begin{itemdecl}
template <@\changed{class Iterator1}{RandomAccessIterator I1}@, @\changed{class Iterator2}{RandomAccessIterator I2}@>
    @\added{requires TotallyOrdered<I1, I2>()}@
  bool operator>(
    const move_iterator<@\changed{Iterator1}{I1}@>& x, const move_iterator<@\changed{Iterator2}{I2}@>& y);
\end{itemdecl}

\begin{itemdescr}
\pnum
\returns \tcode{y < x}.
\end{itemdescr}

\indexlibrary{\idxcode{operator>=}!\idxcode{move_iterator}}%
\indexlibrary{\idxcode{move_iterator}!\idxcode{operator>=}}%
\begin{itemdecl}
template <@\changed{class Iterator1}{RandomAccessIterator I1}@, @\changed{class Iterator2}{RandomAccessIterator I2}@>
    @\added{requires TotallyOrdered<I1, I2>()}@
  bool operator>=(
    const move_iterator<@\changed{Iterator1}{I1}@>& x, const move_iterator<@\changed{Iterator2}{I2}@>& y);
\end{itemdecl}

\begin{itemdescr}
\pnum
\returns \tcode{!(x < y)}.
\end{itemdescr}

\rSec4[move.iter.nonmember]{\tcode{move_iterator} non-member functions}

\indexlibrary{\idxcode{operator-}!\idxcode{move_iterator}}%
\indexlibrary{\idxcode{move_iterator}!\idxcode{operator-}}%
\begin{itemdecl}
template <@\changed{class Iterator1}{WeakInputIterator I1}@, @\changed{class Iterator2}{WeakInputIterator I2}@>
    @\added{requires SizedIteratorRange<I2, I1>}@
  @\changed{auto}{DifferenceType<I2>}@ operator-(
    const move_iterator<@\changed{Iterator1}{I1}@>& x,
    const move_iterator<@\changed{Iterator2}{I2}@>& y)@\removed{ ->decltype(y.base() - x.base())}@;
\end{itemdecl}

\begin{itemdescr}
\pnum
\returns \tcode{x.base() - y.base()}.
\end{itemdescr}

\indexlibrary{\idxcode{operator+}!\idxcode{move_iterator}}%
\indexlibrary{\idxcode{move_iterator}!\idxcode{operator+}}%
\begin{itemdecl}
template <@\changed{class Iterator}{RandomAccessIterator I}@>
  move_iterator<@\changed{Iterator}{I}@>
    operator+(
      @\changed{typename move_iterator<Iterator>::difference_type}{DifferenceType<I>}@ n,
      const move_iterator<@\changed{Iterator}{I}@>& x);
\end{itemdecl}

\begin{itemdescr}
\pnum
\returns \tcode{x + n}.
\end{itemdescr}

\indexlibrary{\idxcode{make_move_iterator}}%
\begin{itemdecl}
template <@\changed{class Iterator}{WeakInputIterator I}@>
  move_iterator<@\changed{Iterator}{I}@> make_move_iterator(@\changed{Iterator}{I}@ i);
\end{itemdecl}

\begin{itemdescr}
\pnum
\returns \tcode{move_iterator<\changed{Iterator}{I}>(i)}.
\end{itemdescr}

\begin{addedblock}

\rSec2[common.iterators]{Common iterators}

\pnum
Class template \tcode{common_iterator} is an iterator/sentinel adaptor that is
capable of representing a non-bounded range of elements (where the types of the
iterator and sentinel differ) as a bounded range (where they are the same). It
does this by holding either an iterator or a sentinel, and implementing the
equality comparison operators appropriately.

\pnum
\enternote The \tcode{common_iterator} type is useful for interfacing with legacy
code that expects the begin and end of a range to have the same type, and for
use in \tcode{common_type} specializations that are required to make
iterator/sentinel pairs model the \tcode{EqualityComparable} concept.\exitnote

\pnum
\enterexample
\begin{codeblock}
template<class ForwardIterator>
void fun(ForwardIterator begin, ForwardIterator end);

list<int> s;
// populate the list \tcode{s}
using CI =
  common_iterator<counted_iterator<list<int>::iterator>,
                  counted_sentinel>;
// call \tcode{fun} on a range of 10 ints
fun(CI(make_counted_iterator(s.begin(), 10)),
    CI(counted_sentinel()));
\end{codeblock}
\exitexample

\rSec3[common.iterator]{Class template \tcode{common_iterator}}

\indexlibrary{\idxcode{counted_iterator}}%
\begin{codeblock}
namespace std {
  // \expos
  template<typename A, typename B>
  concept bool WeaklyEqualityComparable =
    EqualityComparable<A>() && EqualityComparable<B>() &&
    requires(A a, B b) {
      {a==b} -> bool;
      {a!=b} -> bool;
      {b==a} -> bool;
      {b!=a} -> bool;
    };
  // \expos
  template<Iterator I, Regular S>
  concept bool WeakSentinel =
    WeaklyEqualityComparable<I, S>;

  template <InputIterator I, WeakSentinel<I> S>
    requires !Same<I, S>
  class common_iterator {
  public:
    using difference_type = DifferenceType<I>;
    using value_type = ValueType<I>;
    using iterator_category =
      conditional_t<ForwardIterator<I>,
                    std::forward_iterator_tag,
                    std::input_iterator_tag>;
    using reference = ReferenceType<I>;

    common_iterator();
    common_iterator(I i);
    common_iterator(S s);
    template <InputIterator U, WeakSentinel<U> V>
      requires Convertible<U, I> && Convertible<V, S>
    common_iterator(const common_iterator<U, V>& u);
    template <InputIterator U, WeakSentinel<U> V>
      requires Convertible<U, I> && Convertible<V, S>
    common_iterator& operator=(const common_iterator<U, V>& u);

    ~common_iterator();

    reference operator*() const;

    common_iterator& operator++();
    common_iterator operator++(int);

  private:
    bool is_sentinel; // \expos
    I iter;           // \expos
    S sent;           // \expos
  };

  template <InputIterator I1, WeakSentinel<I1> S1,
            InputIterator I2, WeakSentinel<I2> S2>
    requires EqualityComparable<I1, I2>() && WeaklyEqualityComparable<I1, S2> &&
      WeaklyEqualityComparable<I2, S1>
  bool operator==(
    const common_iterator<I1, S1>& x, const common_iterator<I2, S2>& y);
  template <InputIterator I1, WeakSentinel<I1> S1,
            InputIterator I2, WeakSentinel<I2> S2>
    requires EqualityComparable<I1, I2>() && WeaklyEqualityComparable<I1, S2> &&
      WeaklyEqualityComparable<I2, S1>
  bool operator!=(
    const common_iterator<I1, S1>& x, const common_iterator<I2, S2>& y);

  template <InputIterator I1, WeakSentinel<I1> S1,
            InputIterator I2, WeakSentinel<I2> S2>
    requires SizedIteratorRange<I1, I1> && SizedIteratorRange<I2, I2> &&
      requires (I1 a, I2 b) { {a-b}->DifferenceType<I2>; {b-a}->DifferenceType<I2>; }
      requires (I1 i, S2 s) { {i-s}->DifferenceType<I2>; {s-i}->DifferenceType<I2>; }
      requires (I2 i, S1 s) { {i-s}->DifferenceType<I2>; {s-i}->DifferenceType<I2>; }
  DifferenceType<I2> operator-(
    const common_iterator<I1, S1>& x, const common_iterator<I2, S2>& y);
}
\end{codeblock}

\pnum
\enternote The use of the expository \tcode{WeaklyEqualityComparable} and
\tcode{WeakSentinel} concepts is avoid the self-referential requirements that
would happen if parameters \tcode{I} and \tcode{S} use \tcode{common_iterator<I, S>}
as their common type.\exitnote

\pnum
\enternote The ad hoc constraints on \tcode{common_iterator}'s \tcode{operator-}
exist for the same reason.\exitnote

\pnum
\enternote It is unspecified whether \tcode{common_iterator}'s members
\tcode{iter} and \tcode{sent} have distinct addresses or not.\exitnote

\rSec3[common.iter.ops]{\tcode{common_iterator} operations}

\rSec4[common.iter.op.const]{\tcode{common_iterator} constructors}

\indexlibrary{\idxcode{common_iterator}!\idxcode{common_iterator}}%
\begin{itemdecl}
common_iterator();
\end{itemdecl}

\begin{itemdescr}
\pnum
\effects Constructs a \tcode{common_iterator}, value-initializing \tcode{is_sentinel}
and \tcode{iter}. It is unspecified whether any initialization is performed for
\tcode{sent}. Iterator operations applied to the resulting iterator have defined
behavior if and only if the corresponding operations are defined on a
value-initialized iterator of type \tcode{I}.
\end{itemdescr}

\indexlibrary{\idxcode{common_iterator}!constructor}%
\begin{itemdecl}
common_iterator(I i);
\end{itemdecl}

\begin{itemdescr}
\pnum
\effects Constructs a \tcode{common_iterator}, initializing
\tcode{is_sentinel} with \tcode{false} and \tcode{iter} with \tcode{i}. It is
unspecified whether any initialization is performed for \tcode{sent}.
\end{itemdescr}

\indexlibrary{\idxcode{common_iterator}!constructor}%
\begin{itemdecl}
common_iterator(S s);
\end{itemdecl}

\begin{itemdescr}
\pnum
\effects Constructs a \tcode{common_iterator}, initializing
\tcode{is_sentinel} with \tcode{true} and \tcode{sent} with \tcode{s}. It is
unspecified whether any initialization is performed for \tcode{iter}.
\end{itemdescr}

\indexlibrary{\idxcode{common_iterator}!constructor}%
\begin{itemdecl}
template <InputIterator U, WeakSentinel<U> V>
  requires Convertible<U, I> && Convertible<V, S>
common_iterator(const common_iterator<U, V>& u);
\end{itemdecl}

\begin{itemdescr}
\pnum
\effects Constructs a \tcode{common_iterator}, initializing
\tcode{is_sentinel} with \tcode{u.is_sentinel}.
\begin{itemize}
\item If \tcode{u.is_sentinel} is true, \tcode{sent} is initialized with \tcode{u.sent}.
It is unspecified whether any initialization is performed for \tcode{iter}.
\item If \tcode{u.is_sentinel} is false, \tcode{iter} is initialized with \tcode{u.iter}.
It is unspecified whether any initialization is performed for \tcode{sent}.
\end{itemize}
\end{itemdescr}

\rSec4[common.iter.op=]{\tcode{common_iterator::operator=}}

\indexlibrary{\idxcode{operator=}!\idxcode{common_iterator}}%
\indexlibrary{\idxcode{common_iterator}!\idxcode{operator=}}%
\begin{itemdecl}
template <InputIterator U, WeakSentinel<U> V>
  requires Convertible<U, I> && Convertible<V, S>
common_iterator& operator=(const common_iterator<U, V>& u);
\end{itemdecl}

\begin{itemdescr}
\pnum
\effects Assigns \tcode{u.is_sentinel} to \tcode{is_sentinel}.
\begin{itemize}
\item If \tcode{u.is_sentinel} is true, assigns \tcode{u.sent} to \tcode{sent}.
It is unspecified whether any operation is performed on \tcode{iter}.
\item If \tcode{u.is_sentinel} is false, assigns \tcode{u.iter} to \tcode{iter}.
It is unspecified whether any operation is performed on \tcode{sent}.
\end{itemize}

\pnum
\returns \tcode{*this}
\end{itemdescr}

\indexlibrary{\idxcode{common_iterator}!destructor}%
\begin{itemdecl}
~common_iterator();
\end{itemdecl}

\begin{itemdescr}
\pnum
\effects
Runs the destructor(s) for any members that are currently initialized.
\end{itemdescr}

\rSec4[common.iter.op.star]{\tcode{common_iterator::operator*}}

\indexlibrary{\idxcode{operator*}!\idxcode{common_iterator}}%
\indexlibrary{\idxcode{common_iterator}!\idxcode{operator*}}%
\begin{itemdecl}
reference operator*() const;
\end{itemdecl}

\begin{itemdescr}
\pnum
\requires \tcode{!is_sentinel}

\pnum
\returns \tcode{*iter}.
\end{itemdescr}

\rSec4[common.iter.op.incr]{\tcode{common_iterator::operator++}}

\indexlibrary{\idxcode{operator++}!\idxcode{common_iterator}}%
\indexlibrary{\idxcode{common_iterator}!\idxcode{operator++}}%
\begin{itemdecl}
common_iterator& operator++();
\end{itemdecl}

\begin{itemdescr}
\pnum
\requires \tcode{!is_sentinel}

\pnum
\effects \tcode{++iter}.

\pnum
\returns \tcode{*this}.
\end{itemdescr}

\indexlibrary{\idxcode{operator++}!\idxcode{common_iterator}}%
\indexlibrary{\idxcode{common_iterator}!\idxcode{operator++}}%
\begin{itemdecl}
common_iterator operator++(int);
\end{itemdecl}

\begin{itemdescr}
\pnum
\requires \tcode{!is_sentinel}

\pnum
\effects
\begin{codeblock}
common_iterator tmp = *this;
++iter;
return tmp;
\end{codeblock}
\end{itemdescr}

\rSec4[common.iter.op.comp]{\tcode{common_iterator} comparisons}

\indexlibrary{\idxcode{operator==}!\idxcode{common_iterator}}%
\indexlibrary{\idxcode{common_iterator}!\idxcode{operator==}}%
\begin{itemdecl}
template <InputIterator I1, WeakSentinel<I1> S1,
          InputIterator I2, WeakSentinel<I2> S2>
  requires EqualityComparable<I1, I2>() && WeaklyEqualityComparable<I1, S2> &&
    WeaklyEqualityComparable<I2, S1>
bool operator==(
  const common_iterator<I1, S1>& x, const common_iterator<I2, S2>& y);
\end{itemdecl}

\begin{itemdescr}
\pnum
\returns
\begin{codeblock}
x.is_sentinel ?
    (y.is_sentinel || y.iter == x.sent) :
    (y.is_sentinel ?
        x.iter == y.sent :
        x.iter == y.iter;
\end{codeblock}
\end{itemdescr}

\indexlibrary{\idxcode{operator"!=}!\idxcode{common_iterator}}%
\indexlibrary{\idxcode{common_iterator}!\idxcode{operator"!=}}%
\begin{itemdecl}
template <InputIterator I1, WeakSentinel<I1> S1,
          InputIterator I2, WeakSentinel<I2> S2>
  requires EqualityComparable<I1, I2>() && WeaklyEqualityComparable<I1, S2> &&
    WeaklyEqualityComparable<I2, S1>
bool operator!=(
  const common_iterator<I1, S1>& x, const common_iterator<I2, S2>& y);
\end{itemdecl}

\begin{itemdescr}
\pnum
\returns \tcode{!(x == y)}.
\end{itemdescr}

\indexlibrary{\idxcode{operator-}!\idxcode{common_iterator}}%
\indexlibrary{\idxcode{common_iterator}!\idxcode{operator-}}%
\begin{itemdecl}
template <InputIterator I1, WeakSentinel<I1> S1,
          InputIterator I2, WeakSentinel<I2> S2>
  requires SizedIteratorRange<I1, I1> && SizedIteratorRange<I2, I2> &&
    requires (I1 a, I2 b) { {a-b}->DifferenceType<I2>; {b-a}->DifferenceType<I2>; }
    requires (I1 i, S2 s) { {i-s}->DifferenceType<I2>; {s-i}->DifferenceType<I2>; }
    requires (I2 i, S1 s) { {i-s}->DifferenceType<I2>; {s-i}->DifferenceType<I2>; }
DifferenceType<I2> operator-(
  const common_iterator<I1, S1>& x, const common_iterator<I2, S2>& y);
\end{itemdecl}

\begin{itemdescr}
\pnum
\returns
\begin{codeblock}
x.is_sentinel ?
    (y.is_sentinel ? 0 : x.sent - y.iter) :
    (y.is_sentinel ?
         x.iter - y.sent :
         x.iter - y.iter;
\end{codeblock}
\end{itemdescr}

\rSec2[counted.iterators]{Counted iterators and sentinels}

\pnum
Class template \tcode{counted_iterator} is an iterator adaptor
with the same behavior as the underlying iterator except that it
keeps track of its distance from its starting position. It can be
used together with class \tcode{counted_sentinel} in calls to generic
algorithms to operate on a range of $N$ elements starting at a given
position without needing to know the end position \textit{a priori}.

\pnum
\enterexample

\begin{codeblock}
list<string> s;
// populate the list \tcode{s}
vector<string> v(make_counted_iterator(s.begin(), 10),
                 counted_sentinel()); // copies 10 strings into \tcode{v}
\end{codeblock}

\exitexample

\rSec3[counted.iterator]{Class template \tcode{counted_iterator}}

\indexlibrary{\idxcode{counted_iterator}}%
\begin{codeblock}
namespace std {
  template <WeakInputIterator I>
  class counted_iterator {
  public:
    using iterator_type = I;
    using difference_type = DifferenceType<I>;
    using value_type = ValueType<I>;
    using iterator_category =
      conditional_t<ForwardIterator<I>,
                    IteratorCategory<I>,
                    std::input_iterator_tag>;
    using reference = ReferenceType<I>;

    counted_iterator() = default;
    counted_iterator(I x, DifferenceType<I> n);
    template <WeakInputIterator U>
      requires Convertible<U, I>
    counted_iterator(const counted_iterator<U>& u);
    template <WeakInputIterator U>
      requires Convertible<U, I>
    counted_iterator& operator=(const counted_iterator<U>& u);

    I base() const;
    DifferenceType<I> count() const;
    reference operator*() const;

    counted_iterator& operator++();
    counted_iterator operator++(int);
    counted_iterator& operator--()
      requires BidirectionalIterator<I>;
    counted_iterator operator--(int)
      requires BidirectionalIterator<I>;


    counted_iterator  operator+ (difference_type n) const
      requires RandomAccessIterator<I>;
    counted_iterator& operator+=(difference_type n)
      requires RandomAccessIterator<I>;
    counted_iterator  operator- (difference_type n) const
      requires RandomAccessIterator<I>;
    counted_iterator& operator-=(difference_type n)
      requires RandomAccessIterator<I>;
    @\unspec@ operator[](difference_type n) const
      requires RandomAccessIterator<I>;
  protected:
    I current;
    DifferenceType<I> cnt;
  };

  template <WeakInputIterator I1, WeakInputIterator I2>
    bool operator==(
      const counted_iterator<I1>& x, const counted_iterator<I2>& y);
  template <WeakInputIterator I>
    bool operator==(
      const counted_iterator<I>& x, counted_sentinel y);
  template <WeakInputIterator I>
    bool operator==(
      counted_sentinel x, const counted_iterator<I>& y);
  bool operator==(counted_sentinel x, counted_sentinel y);
  template <WeakInputIterator I1, WeakInputIterator I2>
    bool operator!=(
      const counted_iterator<I1>& x, const counted_iterator<I2>& y);
  template <WeakInputIterator I>
    bool operator!=(
      const counted_iterator<I>& x, counted_sentinel y);
  template <WeakInputIterator I>
    bool operator!=(
      counted_sentinel x, const counted_iterator<I>& y);
  bool operator!=(counted_sentinel x, counted_sentinel y);

  template <RandomAccessIterator I1, RandomAccessIterator I2>
      requires TotallyOrdered<I1, I2>()
    bool operator<(
      const counted_iterator<I1>& x, const counted_iterator<I2>& y);
  template <RandomAccessIterator I>
    bool operator<(
      const counted_iterator<I>& x, counted_sentinel y);
  template <RandomAccessIterator I>
    bool operator<(
      counted_sentinel x, const counted_iterator<I>& y);
  bool operator<(counted_sentinel x, counted_sentinel y);
  template <RandomAccessIterator I1, RandomAccessIterator I2>
      requires TotallyOrdered<I1, I2>()
    bool operator<=(
      const counted_iterator<I1>& x, const counted_iterator<I2>& y);
  template <RandomAccessIterator I>
    bool operator<=(
      const counted_iterator<I>& x, counted_sentinel y);
  template <RandomAccessIterator I>
    bool operator<=(
      counted_sentinel x, const counted_iterator<I>& y);
  bool operator<=(counted_sentinel x, counted_sentinel y);
  template <RandomAccessIterator I1, RandomAccessIterator I2>
      requires TotallyOrdered<I1, I2>()
    bool operator>(
      const counted_iterator<I1>& x, const counted_iterator<I2>& y);
  template <RandomAccessIterator I>
    bool operator>(
      const counted_iterator<I>& x, counted_sentinel y);
  template <RandomAccessIterator I>
    bool operator>(
      counted_sentinel x, const counted_iterator<I>& y);
  bool operator>(counted_sentinel x, counted_sentinel y);
  template <RandomAccessIterator I1, RandomAccessIterator I2>
      requires TotallyOrdered<I1, I2>()
    bool operator>=(
      const counted_iterator<I1>& x, const counted_iterator<I2>& y);
  template <RandomAccessIterator I>
    bool operator>=(
      const counted_iterator<I>& x, counted_sentinel y);
  template <RandomAccessIterator I>
    bool operator>=(
      counted_sentinel x, const counted_iterator<I>& y);
  bool operator>=(counted_sentinel x, counted_sentinel y);

  template <WeakInputIterator I1, WeakInputIterator I2>
    DifferenceType<I2> operator-(
      const counted_iterator<I1>& x, const counted_iterator<I2>& y);
  template <WeakInputIterator I>
    DifferenceType<I> operator-(
      const counted_iterator<I>& x, counted_sentinel y);
  template <WeakInputIterator I>
    DifferenceType<I> operator-(
      counted_sentinel x, const counted_iterator<I>& y);
  ptrdiff_t operator-(counted_sentinel x, counted_sentinel y);
  template <RandomAccessIterator I>
    counted_iterator<I>
      operator+(DifferenceType<I> n, const counted_iterator<I>& x);
  template <WeakInputIterator I>
    counted_iterator<I> make_counted_iterator(I i, DifferenceType<I> n);

  template <WeakInputIterator I>
    void advance(counted_iterator<I>& i, DifferenceType<I> n);
}
\end{codeblock}

\rSec3[counted.iter.ops]{\tcode{counted_iterator} operations}

\rSec4[counted.iter.op.const]{\tcode{counted_iterator} constructors}

\indexlibrary{\idxcode{counted_iterator}!\idxcode{counted_iterator}}%
\begin{itemdecl}
counted_iterator() = default;
\end{itemdecl}

\begin{itemdescr}
\pnum
\effects Constructs a \tcode{counted_iterator}, default
initializing \tcode{current} and \tcode{cnt}. Iterator operations applied to the
resulting iterator have defined behavior if and only if the corresponding operations
are defined on a default-initialized iterator of type \tcode{I}.
\end{itemdescr}

\indexlibrary{\idxcode{counted_iterator}!constructor}%
\begin{itemdecl}
counted_iterator(I i, DifferenceType<I> n);
\end{itemdecl}

\begin{itemdescr}
\pnum
\requires \tcode{n >= 0}

\pnum
\effects Constructs a \tcode{counted_iterator}, initializing
\tcode{current} with \tcode{i} and \tcode{cnt} with \tcode{n}.
\end{itemdescr}

\indexlibrary{\idxcode{counted_iterator}!constructor}%
\begin{itemdecl}
template <WeakInputIterator U>
  requires Convertible<U, I>
count_iterator(const counted_iterator<U>& u);
\end{itemdecl}

\begin{itemdescr}
\pnum
\effects Constructs a \tcode{counted_iterator}, initializing
\tcode{current} with \tcode{u.base()} and \tcode{cnt} with \tcode{u.count()}.
\end{itemdescr}

\rSec4[counted.iter.op=]{\tcode{counted_iterator::operator=}}

\indexlibrary{\idxcode{operator=}!\idxcode{counted_iterator}}%
\indexlibrary{\idxcode{counted_iterator}!\idxcode{operator=}}%
\begin{itemdecl}
template <WeakInputIterator U>
  requires Convertible<U, I>
counted_iterator& operator=(const counted_iterator<U>& u);
\end{itemdecl}

\begin{itemdescr}
\pnum
\effects Assigns \tcode{u.base()} to
\tcode{current} and \tcode{u.count()} to \tcode{cnt}.

\end{itemdescr}

\rSec4[counted.iter.op.conv]{\tcode{counted_iterator} conversion}

\indexlibrary{\idxcode{base}!\idxcode{counted_iterator}}%
\indexlibrary{\idxcode{counted_iterator}!\idxcode{base}}%
\begin{itemdecl}
I base() const;
\end{itemdecl}

\begin{itemdescr}
\pnum
\returns \tcode{current}.
\end{itemdescr}

\rSec4[counted.iter.op.cnt]{\tcode{counted_iterator} count}

\indexlibrary{\idxcode{count}!\idxcode{counted_iterator}}%
\indexlibrary{\idxcode{counted_iterator}!\idxcode{count}}%
\begin{itemdecl}
DifferenceType<I> count() const;
\end{itemdecl}

\begin{itemdescr}
\pnum
\returns \tcode{cnt}.
\end{itemdescr}

\rSec4[counted.iter.op.star]{\tcode{count_iterator::operator*}}

\indexlibrary{\idxcode{operator*}!\idxcode{counted_iterator}}%
\indexlibrary{\idxcode{counted_iterator}!\idxcode{operator*}}%
\begin{itemdecl}
reference operator*() const;
\end{itemdecl}

\begin{itemdescr}
\pnum
\returns \tcode{*current}.
\end{itemdescr}

\rSec4[counted.iter.op.incr]{\tcode{counted_iterator::operator++}}

\indexlibrary{\idxcode{operator++}!\idxcode{counted_iterator}}%
\indexlibrary{\idxcode{counted_iterator}!\idxcode{operator++}}%
\begin{itemdecl}
counted_iterator& operator++();
\end{itemdecl}

\begin{itemdescr}
\pnum
\requires \tcode{cnt > 0}

\pnum
\effects
\begin{codeblock}
++current;
@\dcr@cnt;
\end{codeblock}

\pnum
\returns \tcode{*this}.
\end{itemdescr}

\indexlibrary{\idxcode{operator++}!\idxcode{counted_iterator}}%
\indexlibrary{\idxcode{counted_iterator}!\idxcode{operator++}}%
\begin{itemdecl}
counted_iterator operator++(int);
\end{itemdecl}

\begin{itemdescr}
\pnum
\requires \tcode{cnt > 0}

\pnum
\effects
\begin{codeblock}
counted_iterator tmp = *this;
++current;
@\dcr@cnt;
return tmp;
\end{codeblock}
\end{itemdescr}

\rSec4[counted.iter.op.decr]{\tcode{counted_iterator::operator-{-}}}

\indexlibrary{\idxcode{operator\dcr}!\idxcode{counted_iterator}}%
\indexlibrary{\idxcode{counted_iterator}!\idxcode{operator\dcr}}%
\begin{itemdecl}
  counted_iterator& operator--();
    requires BidirectionalIterator<I>
\end{itemdecl}

\begin{itemdescr}
\pnum
\effects
\begin{codeblock}
--current;
++cnt;
\end{codeblock}

\pnum
\returns \tcode{*this}.
\end{itemdescr}

\indexlibrary{\idxcode{operator\dcr}!\idxcode{counted_iterator}}%
\indexlibrary{\idxcode{counted_iterator}!\idxcode{operator\dcr}}%
\begin{itemdecl}
  counted_iterator operator--(int)
    requires BidirectionalIterator<I>;
\end{itemdecl}

\begin{itemdescr}
\pnum
\effects
\begin{codeblock}
counted_iterator tmp = *this;
--current;
++cnt;
return tmp;
\end{codeblock}
\end{itemdescr}

\rSec4[counted.iter.op.+]{\tcode{counted_iterator::operator+}}

\indexlibrary{\idxcode{operator+}!\idxcode{counted_iterator}}%
\indexlibrary{\idxcode{counted_iterator}!\idxcode{operator+}}%
\begin{itemdecl}
  counted_iterator operator+(difference_type n) const
    requires RandomAccessIterator<I>;
\end{itemdecl}

\begin{itemdescr}
\pnum
\requires \tcode{n <= cnt}

\pnum
\returns \tcode{counted_iterator(current + n, cnt - n)}.
\end{itemdescr}

\rSec4[counted.iter.op.+=]{\tcode{counted_iterator::operator+=}}

\indexlibrary{\idxcode{operator+=}!\idxcode{counted_iterator}}%
\indexlibrary{\idxcode{counted_iterator}!\idxcode{operator+=}}%
\begin{itemdecl}
  counted_iterator& operator+=(difference_type n)
    requires RandomAccessIterator<I>;
\end{itemdecl}

\begin{itemdescr}
\pnum
\requires \tcode{n <= cnt}

\pnum
\effects
\begin{codeblock}
current += n;
cnt -= n;
\end{codeblock}

\pnum
\returns \tcode{*this}.
\end{itemdescr}

\rSec4[counted.iter.op.-]{\tcode{counted_iterator::operator-}}

\indexlibrary{\idxcode{operator-}!\idxcode{counted_iterator}}%
\indexlibrary{\idxcode{counted_iterator}!\idxcode{operator-}}%
\begin{itemdecl}
  counted_iterator operator-(difference_type n) const
    requires RandomAccessIterator<I>;
\end{itemdecl}

\begin{itemdescr}
\pnum
\requires \tcode{-n <= cnt}

\pnum
\returns \tcode{counted_iterator(current - n, cnt + n)}.
\end{itemdescr}

\rSec4[counted.iter.op.-=]{\tcode{counted_iterator::operator-=}}

\indexlibrary{\idxcode{operator-=}!\idxcode{counted_iterator}}%
\indexlibrary{\idxcode{counted_iterator}!\idxcode{operator-=}}%
\begin{itemdecl}
  counted_iterator& operator-=(difference_type n)
    requires RandomAccessIterator<I>;
\end{itemdecl}

\begin{itemdescr}
\pnum
\requires \tcode{-n <= cnt}

\pnum
\effects
\begin{codeblock}
current -= n;
cnt += n;
\end{codeblock}

\pnum
\returns \tcode{*this}.
\end{itemdescr}

\rSec4[counted.iter.op.index]{\tcode{counted_iterator::operator[]}}

\indexlibrary{\idxcode{operator[]}!\idxcode{counted_iterator}}%
\indexlibrary{\idxcode{counted_iterator}!\idxcode{operator[]}}%
\begin{itemdecl}
  @\unspec@ operator[](difference_type n) const
    requires RandomAccessIterator<I>;
\end{itemdecl}

\begin{itemdescr}
\pnum
\requires \tcode{n <= cnt}

\pnum
\returns \tcode{current[n]}.
\end{itemdescr}

\rSec4[counted.iter.op.comp]{\tcode{counted_iterator} comparisons}

\indexlibrary{\idxcode{operator==}!\idxcode{counted_iterator}}%
\indexlibrary{\idxcode{counted_iterator}!\idxcode{operator==}}%
\begin{itemdecl}
template <WeakInputIterator I1, WeakInputIterator I2>
  bool operator==(
    const counted_iterator<I1>& x, const counted_iterator<I2>& y);
\end{itemdecl}

\begin{itemdescr}
\pnum
\returns \tcode{x.base() == y.base()} if \tcode{EqualityComparable<I1, I2>()};
  otherwise, \tcode{x.count() == y.count()}.
\end{itemdescr}

\begin{itemdecl}
template <WeakInputIterator I>
  bool operator==(
    const counted_iterator<I>& x, counted_sentinel y);
\end{itemdecl}

\begin{itemdescr}
\pnum
\returns \tcode{x.count() == 0}.
\end{itemdescr}

\begin{itemdecl}
template <WeakInputIterator I>
  bool operator==(
    counted_sentinel x, const counted_iterator<I>& y);
\end{itemdecl}

\begin{itemdescr}
\pnum
\returns \tcode{y.count() == 0}.
\end{itemdescr}

\begin{itemdecl}
bool operator==(counted_sentinel x, counted_sentinel y);
\end{itemdecl}

\begin{itemdescr}
\pnum
\returns \tcode{true}.
\end{itemdescr}

\indexlibrary{\idxcode{operator"!=}!\idxcode{counted_iterator}}%
\indexlibrary{\idxcode{counted_iterator}!\idxcode{operator"!=}}%
\begin{itemdecl}
template <WeakInputIterator I1, WeakInputIterator I2>
  bool operator!=(
    const counted_iterator<I1>& x, const counted_iterator<I2>& y);
template <WeakInputIterator I>
  bool operator!=(
    const counted_iterator<I>& x, counted_sentinel y);
template <WeakInputIterator I>
  bool operator!=(
    counted_sentinel x, const counted_iterator<I>& y);
bool operator!=(counted_sentinel x, counted_sentinel y);
\end{itemdecl}

\begin{itemdescr}
\pnum
\returns \tcode{!(x == y)}.
\end{itemdescr}

\indexlibrary{\idxcode{operator<}!\idxcode{counted_iterator}}%
\indexlibrary{\idxcode{counted_iterator}!\idxcode{operator<}}%
\begin{itemdecl}
template <RandomAccessIterator I1, RandomAccessIterator I2>
    requires TotallyOrdered<I1, I2>()
  bool operator<(
    const counted_iterator<I1>& x, const counted_iterator<I2>& y);
\end{itemdecl}

\begin{itemdescr}
\pnum
\returns \tcode{x.base() < y.base()}.
\end{itemdescr}

\begin{itemdecl}
template <RandomAccessIterator I>
  bool operator<(
    const counted_iterator<I>& x, counted_sentinel y);
\end{itemdecl}

\begin{itemdescr}
\pnum
\returns \tcode{x.count() != 0}.
\end{itemdescr}

\begin{itemdecl}
template <RandomAccessIterator I>
  bool operator<(
    counted_sentinel x, const counted_iterator<I>& y);
\end{itemdecl}

\begin{itemdescr}
\pnum
\returns \tcode{false}.
\end{itemdescr}

\begin{itemdecl}
bool operator<(counted_sentinel x, counted_sentinel y);
\end{itemdecl}

\begin{itemdescr}
\pnum
\returns \tcode{false}.
\end{itemdescr}

\indexlibrary{\idxcode{operator<=}!\idxcode{counted_iterator}}%
\indexlibrary{\idxcode{counted_iterator}!\idxcode{operator<=}}%
\begin{itemdecl}
template <RandomAccessIterator I1, RandomAccessIterator I2>
    requires TotallyOrdered<I1, I2>()
  bool operator<=(
    const counted_iterator<I1>& x, const counted_iterator<I2>& y);
template <RandomAccessIterator I>
  bool operator<=(
    const counted_iterator<I>& x, counted_sentinel y);
template <RandomAccessIterator I>
  bool operator<=(
    counted_sentinel x, const counted_iterator<I>& y);
bool operator<=(counted_sentinel x, counted_sentinel y);
\end{itemdecl}

\begin{itemdescr}
\pnum
\returns \tcode{!(y < x)}.
\end{itemdescr}

\indexlibrary{\idxcode{operator>}!\idxcode{counted_iterator}}%
\indexlibrary{\idxcode{counted_iterator}!\idxcode{operator>}}%
\begin{itemdecl}
template <RandomAccessIterator I1, RandomAccessIterator I2>
    requires TotallyOrdered<I1, I2>()
  bool operator>(
    const counted_iterator<I1>& x, const counted_iterator<I2>& y);
template <RandomAccessIterator I>
  bool operator>(
    const counted_iterator<I>& x, counted_sentinel y);
template <RandomAccessIterator I>
  bool operator>(
    counted_sentinel x, const counted_iterator<I>& y);
bool operator>(counted_sentinel x, counted_sentinel y);
\end{itemdecl}

\begin{itemdescr}
\pnum
\returns \tcode{y < x}.
\end{itemdescr}

\indexlibrary{\idxcode{operator>=}!\idxcode{counted_iterator}}%
\indexlibrary{\idxcode{counted_iterator}!\idxcode{operator>=}}%
\begin{itemdecl}
template <RandomAccessIterator I1, RandomAccessIterator I2>
    requires TotallyOrdered<I1, I2>()
  bool operator>=(
    const counted_iterator<I1>& x, const counted_iterator<I2>& y);
template <RandomAccessIterator I>
  bool operator>=(
    const counted_iterator<I>& x, counted_sentinel y);
template <RandomAccessIterator I>
  bool operator>=(
    counted_sentinel x, const counted_iterator<I>& y);
bool operator>=(counted_sentinel x, counted_sentinel y);
\end{itemdecl}

\begin{itemdescr}
\pnum
\returns \tcode{!(x < y)}.
\end{itemdescr}

\rSec4[counted.iter.nonmember]{\tcode{counted_iterator} non-member functions}

\indexlibrary{\idxcode{operator-}!\idxcode{counted_iterator}}%
\indexlibrary{\idxcode{counted_iterator}!\idxcode{operator-}}%
\begin{itemdecl}
template <WeakInputIterator I1, WeakInputIterator I2>
  DifferenceType<I2> operator-(
    const counted_iterator<I1>& x, const counted_iterator<I2>& y);
\end{itemdecl}

\begin{itemdescr}
\pnum
\returns \tcode{x.base() - y.base()} if \tcode{SizedIteratorRange<I2, I1>};
otherwise, \tcode{y.count() - x.count()}.
\end{itemdescr}

\begin{itemdecl}
template <WeakInputIterator I>
  DifferenceType<I> operator-(
    const counted_iterator<I>& x, counted_sentinel y);
\end{itemdecl}

\begin{itemdescr}
\pnum
\returns \tcode{-x.count()}.
\end{itemdescr}

\begin{itemdecl}
template <WeakInputIterator I>
  DifferenceType<I> operator-(
    counted_sentinel x, const counted_iterator<I>& y);
\end{itemdecl}

\begin{itemdescr}
\pnum
\returns \tcode{y.count()}.
\end{itemdescr}

\begin{itemdecl}
ptrdiff_t operator-(counted_sentinel x, counted_sentinel y);
\end{itemdecl}

\begin{itemdescr}
\pnum
\returns \tcode{0}.
\end{itemdescr}

\indexlibrary{\idxcode{operator+}!\idxcode{counted_iterator}}%
\indexlibrary{\idxcode{counted_iterator}!\idxcode{operator+}}%
\begin{itemdecl}
template <RandomAccessIterator I>
  counted_iterator<I>
    operator+(DifferenceType<I> n, const counted_iterator<I>& x);
\end{itemdecl}

\begin{itemdescr}
\pnum
\requires \tcode{n <= x.count()}.

\pnum
\returns \tcode{x + n}.
\end{itemdescr}

\indexlibrary{\idxcode{make_counted_iterator}}%
\begin{itemdecl}
template <WeakInputIterator I>
  counted_iterator<I> make_counted_iterator(I i, DifferenceType<I> n);
\end{itemdecl}

\begin{itemdescr}
\pnum
\requires \tcode{n >= 0}.

\pnum
\returns \tcode{counted_iterator<I>(i, n)}.
\end{itemdescr}

\indexlibrary{\idxcode{advance}}%
\begin{itemdecl}
template <WeakInputIterator I>
  void advance(counted_iterator<I>& i, DifferenceType<I> n);
\end{itemdecl}

\begin{itemdescr}
\pnum
\requires \tcode{n <= i.count()}.

\pnum
\effects
\begin{codeblock}
i = make_counted_iterator(next(i.base(), n), i.count() - n);
\end{codeblock}
\end{itemdescr}

\rSec3[counted.sentinel]{Counted sentinel}

\pnum
Class \tcode{counted_sentinel} is an empty type used to represent the end of a counted
range. It is used together with class template
\tcode{counted_iterator}(~\ref{counted.iterator}) to denote a range of elements that
starts at a known position and includes the subsequent $N$ elements.

\indexlibrary{\idxcode{counted_sentinel}}%
\begin{itemdecl}
namespace std {
  class counted_sentinel { };
}
\end{itemdecl}

\rSec3[counted.traits.specializations]{Specializations of \tcode{common_type}}

\indexlibrary{\idxcode{common_type}}%
\begin{itemdecl}
namespace std {
  template<WeakInputIterator I>
  struct common_type<counted_iterator<I>, counted_sentinel> {
    using type = common_iterator<counted_iterator<I>, counted_sentinel>;
  };
  template<WeakInputIterator I>
  struct common_type<counted_sentinel, counted_iterator<I>> {
    using type = common_iterator<counted_iterator<I>, counted_sentinel>;
  };
}
\end{itemdecl}

\begin{itemdescr}
\pnum
\enternote By specializing \tcode{common_type} this way, \tcode{counted_iterator}
and \tcode{counted_sentinel} can satisfy the \tcode{Common} requirement of the
\tcode{EqualityComparable} concept.\exitnote
\end{itemdescr}

\rSec2[unreachable.sentinels]{Unreachable sentinel}

\rSec3[unreachable.sentinel]{Class \tcode{unreachable} sentinel}

\pnum
\indexlibrary{\idxcode{unreachable}}%
Class \tcode{unreachable} is a sentinel type that can be used with any
\tcode{Iterator} to denote an infinite range. Comparing an iterator for equality with
an object of type \tcode{unreachable} always returns \tcode{false}.

\enterexample
\begin{codeblock}
char* p;
// set \tcode{p} to point to a character buffer containing newlines
char* nl = find(p, unreachable(), '@\textbackslash@n');
\end{codeblock}

Provided a newline character really exists in the buffer, the use of \tcode{unreachable}
above potentially make the call to \tcode{find} more efficient since the loop test against
the sentinel does not require a conditional branch.
\exitexample

\begin{codeblock}
namespace std {
  class unreachable { };
  template <Iterator I>
    constexpr bool operator==(I const &, unreachable) noexcept;
  template <Iterator I>
    constexpr bool operator==(unreachable, I const &) noexcept;
  constexpr bool operator==(unreachable, unreachable) noexcept;
  template <Iterator I>
    constexpr bool operator!=(I const &, unreachable) noexcept;
  template <Iterator I>
    constexpr bool operator!=(unreachable, I const &) noexcept;
  constexpr bool operator!=(unreachable, unreachable) noexcept;
}
\end{codeblock}

\rSec3[unreachable.sentinel.ops]{\tcode{unreachable} operations}

\rSec4[unreachable.sentinel.op==]{\tcode{operator==}}

\indexlibrary{\idxcode{operator==}!\idxcode{unreachable}}%
\indexlibrary{\idxcode{unreachable}!\idxcode{operator==}}%
\begin{itemdecl}
template <Iterator I>
  constexpr bool operator==(I const &, unreachable) noexcept;
template <Iterator I>
  constexpr bool operator==(unreachable, I const &) noexcept;
\end{itemdecl}

\begin{itemdescr}
\pnum
\returns \tcode{false}.
\end{itemdescr}

\begin{itemdecl}
constexpr bool operator==(unreachable, unreachable) noexcept;
\end{itemdecl}

\begin{itemdescr}
\pnum
\returns \tcode{true}.
\end{itemdescr}

\rSec4[unreachable.sentinel.op!=]{\tcode{operator!=}}

\indexlibrary{\idxcode{operator"!=}!\idxcode{unreachable}}%
\indexlibrary{\idxcode{unreachable}!\idxcode{operator"!=}}%
\begin{itemdecl}
template <Iterator I>
  constexpr bool operator!=(I const & x, unreachable y) noexcept;
template <Iterator I>
  constexpr bool operator!=(unreachable x, I const & y) noexcept;
constexpr bool operator!=(unreachable x, unreachable y) noexcept;
\end{itemdecl}

\begin{itemdescr}
\pnum
\returns
\tcode{!(x == y)}
\end{itemdescr}

\rSec3[unreachable.traits.specializations]{Specializations of \tcode{common_type}}

\indexlibrary{\idxcode{common_type}}%
\begin{itemdecl}
namespace std {
  template<Iterator I>
  struct common_type<I, unreachable> {
    using type = common_iterator<I, unreachable>;
  };
  template<Iterator I>
  struct common_type<unreachable, I> {
    using type = common_iterator<I, unreachable>;
  };
}
\end{itemdecl}

\begin{itemdescr}
\pnum
\enternote By specializing \tcode{common_type} this way, any iterator and
\tcode{unreachable} can satisfy the \tcode{Common} requirement of the
\tcode{EqualityComparable} concept.\exitnote
\end{itemdescr}

\end{addedblock}

\rSec1[stream.iterators]{Stream iterators}

\pnum
To make it possible for algorithmic templates to work directly with input/output streams, appropriate
iterator-like
class templates
are provided.

\enterexample
\begin{codeblock}
partial_sum(istream_iterator<double, char>(cin),
  istream_iterator<double, char>(),
  ostream_iterator<double, char>(cout, "@\textbackslash@n"));
\end{codeblock}

reads a file containing floating point numbers from
\tcode{cin},
and prints the partial sums onto
\tcode{cout}.
\exitexample

\rSec2[istream.iterator]{Class template \tcode{istream_iterator}}

\pnum
\indexlibrary{\idxcode{istream_iterator}}%
The class template
\tcode{istream_iterator}
is an input iterator~(\ref{input.iterators}) that
reads (using
\tcode{operator\shr})
successive elements from the input stream for which it was constructed.
After it is constructed, and every time
\tcode{++}
is used, the iterator reads and stores a value of
\tcode{T}.
If the iterator fails to read and store a value of \tcode{T}
(\tcode{fail()}
on the stream returns
\tcode{true}),
the iterator becomes equal to the
\term{end-of-stream}
iterator value.
The constructor with no arguments
\tcode{istream_iterator()}
always constructs
an end-of-stream input iterator object, which is the only legitimate iterator to be used
for the end condition.
The result of
\tcode{operator*}
on an end-of-stream iterator is not defined.
For any other iterator value a
\tcode{const T\&}
is returned.
\removed{The result of
\tcode{operator->}
on an end-of-stream iterator is not defined.
For any other iterator value a
\tcode{const T*}
is returned.}
The behavior of a program that applies \tcode{operator++()} to an end-of-stream
iterator is undefined.
It is impossible to store things into istream iterators.

\pnum
Two end-of-stream iterators are always equal.
An end-of-stream iterator is not
equal to a non-end-of-stream iterator.
Two non-end-of-stream iterators are equal when they are constructed from the same stream.

\begin{codeblock}
namespace std {
  template <class T, class charT = char, class traits = char_traits<charT>,
      class Distance = ptrdiff_t>
  class istream_iterator@\removed{:}@
    @\removed{public iterator<input_iterator_tag, T, Distance, const T*, const T\&>}@ {
  public:
    @\added{typedef input_iterator_tag iterator_category;}@
    @\added{typedef Distance difference_type;}@
    @\added{typedef T value_type;}@
    @\added{typedef const T\& reference;}@
    typedef charT char_type;
    typedef traits traits_type;
    typedef basic_istream<charT,traits> istream_type;
    @\seebelow@ istream_iterator();
    istream_iterator(istream_type& s);
    istream_iterator(const istream_iterator& x) = default;
   ~istream_iterator() = default;

    const T& operator*() const;
    @\removed{const T* operator->() const;}@
    istream_iterator<T,charT,traits,Distance>& operator++();
    istream_iterator<T,charT,traits,Distance>  operator++(int);
  private:
    basic_istream<charT,traits>* in_stream; // \expos
    T value;                                // \expos
  };

  template <class T, class charT, class traits, class Distance>
    bool operator==(const istream_iterator<T,charT,traits,Distance>& x,
            const istream_iterator<T,charT,traits,Distance>& y);
  template <class T, class charT, class traits, class Distance>
    bool operator!=(const istream_iterator<T,charT,traits,Distance>& x,
            const istream_iterator<T,charT,traits,Distance>& y);
}
\end{codeblock}

\rSec3[istream.iterator.cons]{\tcode{istream_iterator} constructors and destructor}


\indexlibrary{\idxcode{istream_iterator}!constructor}%
\begin{itemdecl}
@\seebelow@ istream_iterator();
\end{itemdecl}

\begin{itemdescr}
\pnum
\effects
Constructs the end-of-stream iterator. If \tcode{T} is a literal type, then this
constructor shall be a \tcode{constexpr} constructor.

\pnum
\postcondition \tcode{in_stream == 0}.
\end{itemdescr}


\indexlibrary{\idxcode{istream_iterator}!constructor}%
\begin{itemdecl}
istream_iterator(istream_type& s);
\end{itemdecl}

\begin{itemdescr}
\pnum
\effects
Initializes \textit{in_stream} with \tcode{\&s}. \textit{value} may be initialized during
construction or the first time it is referenced.

\pnum
\postcondition \tcode{in_stream == \&s}.
\end{itemdescr}

\indexlibrary{\idxcode{istream_iterator}!constructor}%
\begin{itemdecl}
istream_iterator(const istream_iterator& x) = default;
\end{itemdecl}

\begin{itemdescr}
\pnum
\effects
Constructs a copy of \tcode{x}. If \tcode{T} is a literal type, then this constructor shall be a trivial copy constructor.

\pnum
\postcondition \tcode{in_stream == x.in_stream}.
\end{itemdescr}

\indexlibrary{\idxcode{istream_iterator}!destructor}%
\begin{itemdecl}
~istream_iterator() = default;
\end{itemdecl}

\begin{itemdescr}
\pnum
\effects
The iterator is destroyed. If \tcode{T} is a literal type, then this destructor shall be a trivial destructor.
\end{itemdescr}

\rSec3[istream.iterator.ops]{\tcode{istream_iterator} operations}

\indexlibrary{\idxcode{operator*}!\idxcode{istream_iterator}}%
\indexlibrary{\idxcode{istream_iterator}!\idxcode{operator*}}%
\begin{itemdecl}
const T& operator*() const;
\end{itemdecl}

\begin{itemdescr}
\pnum
\returns
\textit{value}.
\end{itemdescr}

\begin{removedblock}
\indexlibrary{\idxcode{operator->}!\idxcode{istream_iterator}}%
\indexlibrary{\idxcode{istream_iterator}!\idxcode{operator->}}%
\begin{itemdecl}
const T* operator->() const;
\end{itemdecl}

\begin{itemdescr}
\pnum
\returns
\tcode{\&(operator*())}.
\end{itemdescr}
\end{removedblock}

\indexlibrary{\idxcode{operator++}!\idxcode{istream_iterator}}%
\indexlibrary{\idxcode{istream_iterator}!\idxcode{operator++}}%
\begin{itemdecl}
istream_iterator<T,charT,traits,Distance>& operator++();
\end{itemdecl}

\begin{itemdescr}
\pnum
\requires \tcode{in_stream != 0}.

\pnum
\effects
\tcode{*in_stream \shr value}.

\pnum
\returns
\tcode{*this}.
\end{itemdescr}

\indexlibrary{\idxcode{operator++}!\idxcode{istream_iterator}}%
\indexlibrary{\idxcode{istream_iterator}!\idxcode{operator++}}%
\begin{itemdecl}
istream_iterator<T,charT,traits,Distance> operator++(int);
\end{itemdecl}

\begin{itemdescr}
\pnum
\requires \tcode{in_stream != 0}.

\pnum
\effects
\begin{codeblock}
istream_iterator<T,charT,traits,Distance> tmp = *this;
*in_stream >> value;
return (tmp);
\end{codeblock}
\end{itemdescr}

\indexlibrary{\idxcode{operator==}!\idxcode{istream_iterator}}%
\indexlibrary{\idxcode{istream_iterator}!\idxcode{operator==}}%
\begin{itemdecl}
template <class T, class charT, class traits, class Distance>
  bool operator==(const istream_iterator<T,charT,traits,Distance> &x,
                  const istream_iterator<T,charT,traits,Distance> &y);
\end{itemdecl}

\begin{itemdescr}
\pnum
\returns
\tcode{x.in_stream == y.in_stream}.%
\indexlibrary{\idxcode{istream_iterator}!\idxcode{operator==}}
\end{itemdescr}

\indexlibrary{\idxcode{operator"!=}!\idxcode{istream_iterator}}%
\indexlibrary{\idxcode{istream_iterator}!\idxcode{operator"!=}}%
\begin{itemdecl}
template <class T, class charT, class traits, class Distance>
  bool operator!=(const istream_iterator<T,charT,traits,Distance> &x,
                  const istream_iterator<T,charT,traits,Distance> &y);
\end{itemdecl}

\indexlibrary{\idxcode{istream_iterator}!\idxcode{operator"!=}}%
\begin{itemdescr}
\pnum
\returns
\tcode{!(x == y)}
\end{itemdescr}

\rSec2[ostream.iterator]{Class template \tcode{ostream_iterator}}

\pnum
\indexlibrary{\idxcode{ostream_iterator}}%
\tcode{ostream_iterator}
writes (using
\tcode{operator\shl})
successive elements onto the output stream from which it was constructed.
If it was constructed with
\tcode{charT*}
as a constructor argument, this string, called a
\term{delimiter string},
is written to the stream after every
\tcode{T}
is written.
It is not possible to get a value out of the output iterator.
Its only use is as an output iterator in situations like

\begin{codeblock}
while (first != last)
  *result++ = *first++;
\end{codeblock}

\pnum
\tcode{ostream_iterator}
is defined as:

\begin{codeblock}
namespace std {
  template <class T, class charT = char, class traits = char_traits<charT> >
  class ostream_iterator@\removed{:}@
    @\removed{public iterator<output_iterator_tag, void, void, void, void>}@ {
  public:
    @\added{typedef output_iterator_tag iterator_category;}@
    @\added{typedef ptrdiff_t difference_type;}@
    typedef charT char_type;
    typedef traits traits_type;
    typedef basic_ostream<charT,traits> ostream_type;
    @\added{constexpr ostream_iterator() noexcept;}@
    ostream_iterator(ostream_type& s);
    ostream_iterator(ostream_type& s, const charT* delimiter);
    ostream_iterator(const ostream_iterator<T,charT,traits>& x);
   ~ostream_iterator();
    ostream_iterator<T,charT,traits>& operator=(const T& value);

    ostream_iterator<T,charT,traits>& operator*();
    ostream_iterator<T,charT,traits>& operator++();
    ostream_iterator<T,charT,traits>& operator++(int);
  private:
    basic_ostream<charT,traits>* out_stream;  // \expos
    const charT* delim;                       // \expos
  };
}
\end{codeblock}

\rSec3[ostream.iterator.cons.des]{\tcode{ostream_iterator} constructors and destructor}

\begin{addedblock}
\indexlibrary{\idxcode{ostream_iterator}!constructor}%
\begin{itemdecl}
constexpr ostream_iterator() noexcept;
\end{itemdecl}

\begin{itemdescr}
\pnum
\effects
Initializes \textit{out_stream} and \tcode{delim} with null.
\end{itemdescr}
\end{addedblock}

\indexlibrary{\idxcode{ostream_iterator}!constructor}%
\begin{itemdecl}
ostream_iterator(ostream_type& s);
\end{itemdecl}

\begin{itemdescr}
\pnum
\effects
Initializes \textit{out_stream} with \tcode{\&s} and \textit{delim} with null.
\end{itemdescr}


\indexlibrary{\idxcode{ostream_iterator}!constructor}%
\begin{itemdecl}
ostream_iterator(ostream_type& s, const charT* delimiter);
\end{itemdecl}

\begin{itemdescr}
\pnum
\effects
Initializes \textit{out_stream} with \tcode{\&s} and \textit{delim} with \tcode{delimiter}.
\end{itemdescr}


\indexlibrary{\idxcode{ostream_iterator}!constructor}%
\begin{itemdecl}
ostream_iterator(const ostream_iterator& x);
\end{itemdecl}

\begin{itemdescr}
\pnum
\effects
Constructs a copy of \tcode{x}.
\end{itemdescr}

\indexlibrary{\idxcode{ostream_iterator}!destructor}%
\begin{itemdecl}
~ostream_iterator();
\end{itemdecl}

\begin{itemdescr}
\pnum
\effects
The iterator is destroyed.
\end{itemdescr}

\rSec3[ostream.iterator.ops]{\tcode{ostream_iterator} operations}

\indexlibrary{\idxcode{operator=}!\idxcode{ostream_iterator}}%
\indexlibrary{\idxcode{ostream_iterator}!\idxcode{operator=}}%
\begin{itemdecl}
ostream_iterator& operator=(const T& value);
\end{itemdecl}

\begin{itemdescr}
\pnum
\effects
\begin{codeblock}
*@\textit{out_stream}@ << value;
if(delim != 0)
  *@\textit{out_stream}@ << @\textit{delim}@;
return (*this);
\end{codeblock}

\begin{addedblock}
\pnum
\requires \tcode{out_stream!= 0}.
\end{addedblock}
\end{itemdescr}

\indexlibrary{\idxcode{operator*}!\idxcode{ostream_iterator}}%
\indexlibrary{\idxcode{ostream_iterator}!\idxcode{operator*}}%
\begin{itemdecl}
ostream_iterator& operator*();
\end{itemdecl}

\begin{itemdescr}
\pnum
\returns
\tcode{*this}.
\end{itemdescr}

\indexlibrary{\idxcode{operator++}!\idxcode{ostream_iterator}}%
\indexlibrary{\idxcode{ostream_iterator}!\idxcode{operator++}}%
\begin{itemdecl}
ostream_iterator& operator++();
ostream_iterator& operator++(int);
\end{itemdecl}

\begin{itemdescr}
\pnum
\returns
\tcode{*this}.
\end{itemdescr}

\rSec2[istreambuf.iterator]{Class template \tcode{istreambuf_iterator}}

\pnum
The
class template
\tcode{istreambuf_iterator}
defines an input iterator~(\ref{input.iterators}) that
reads successive
\textit{characters}
from the streambuf for which it was constructed.
\tcode{operator*}
provides access to the current input character, if any.
\removed{\enternote \tcode{operator->} may return a proxy. \exitnote}
Each time
\tcode{operator++}
is evaluated, the iterator advances to the next input character.
If the end of stream is reached (\tcode{streambuf_type::sgetc()} returns
\tcode{traits::eof()}),
the iterator becomes equal to the
\term{end-of-stream}
iterator value.
The default constructor
\tcode{istreambuf_iterator()}
and the constructor
\tcode{istreambuf_iterator(0)}
both construct an end-of-stream iterator object suitable for use
as an end-of-range.
All specializations of \tcode{istreambuf_iterator} shall have a trivial copy
constructor, a \tcode{constexpr} default constructor, and a trivial destructor.

\pnum
The result of
\tcode{operator*()}
on an end-of-stream iterator is undefined.
\indextext{undefined behavior}%
For any other iterator value a
\tcode{char_type}
value is returned.
It is impossible to assign a character via an input iterator.

\indexlibrary{\idxcode{istreambuf_iterator}}%
\begin{codeblock}
namespace std {
  template<class charT, class traits = char_traits<charT> >
  class istreambuf_iterator
     @\removed{: public iterator<input_iterator_tag, charT,}@
                       @\removed{typename traits::off_type, \unspec, charT>}@ {
  public:
    @\added{typedef input_iterator_tag            iterator_category;}@
    @\added{typedef charT                         value_type;}@
    @\added{typedef typename traits::off_type     difference_type;}@
    @\added{typedef charT                         reference;}@
    typedef charT                         char_type;
    typedef traits                        traits_type;
    typedef typename traits::int_type     int_type;
    typedef basic_streambuf<charT,traits> streambuf_type;
    typedef basic_istream<charT,traits>   istream_type;

    class proxy;                          // \expos

    constexpr istreambuf_iterator() noexcept;
    istreambuf_iterator(const istreambuf_iterator&) noexcept = default;
    ~istreambuf_iterator() = default;
    istreambuf_iterator(istream_type& s) noexcept;
    istreambuf_iterator(streambuf_type* s) noexcept;
    istreambuf_iterator(const proxy& p) noexcept;
    charT operator*() const;
    @\removed{pointer operator->() const;}@
    istreambuf_iterator<charT,traits>& operator++();
    proxy operator++(int);
    bool equal(const istreambuf_iterator& b) const;
  private:
    streambuf_type* sbuf_;                // \expos
  };

  template <class charT, class traits>
    bool operator==(const istreambuf_iterator<charT,traits>& a,
            const istreambuf_iterator<charT,traits>& b);
  template <class charT, class traits>
    bool operator!=(const istreambuf_iterator<charT,traits>& a,
            const istreambuf_iterator<charT,traits>& b);
}
\end{codeblock}

\rSec3[istreambuf.iterator::proxy]{Class template \tcode{istreambuf_iterator::proxy}}

\indexlibrary{\idxcode{proxy}!\idxcode{istreambuf_iterator}}%
\begin{codeblock}
namespace std {
  template <class charT, class traits = char_traits<charT> >
  class istreambuf_iterator<charT, traits>::proxy { // \expos
    charT keep_;
    basic_streambuf<charT,traits>* sbuf_;
    proxy(charT c, basic_streambuf<charT,traits>* sbuf)
      : keep_(c), sbuf_(sbuf) { }
  public:
    charT operator*() { return keep_; }
  };
}
\end{codeblock}

\pnum
Class
\tcode{istreambuf_iterator<charT,traits>::proxy}
is for exposition only.
An implementation is permitted to provide equivalent functionality without
providing a class with this name.
Class
\tcode{istreambuf_iterator<charT, traits>\colcol{}proxy}
provides a temporary
placeholder as the return value of the post-increment operator
(\tcode{operator++}).
It keeps the character pointed to by the previous value
of the iterator for some possible future access to get the character.

\rSec3[istreambuf.iterator.cons]{\tcode{istreambuf_iterator} constructors}


\indexlibrary{\idxcode{istreambuf_iterator}!constructor}%
\begin{itemdecl}
constexpr istreambuf_iterator() noexcept;
\end{itemdecl}

\begin{itemdescr}
\pnum
\effects
Constructs the end-of-stream iterator.
\end{itemdescr}


\indexlibrary{\idxcode{istreambuf_iterator}!constructor}%
\begin{itemdecl}
istreambuf_iterator(basic_istream<charT,traits>& s) noexcept;
istreambuf_iterator(basic_streambuf<charT,traits>* s) noexcept;
\end{itemdecl}

\begin{itemdescr}
\pnum
\effects
Constructs an
\tcode{istreambuf_iterator<>}
that uses the
\tcode{basic_streambuf<>}
object
\tcode{*(s.rdbuf())},
or
\tcode{*s},
respectively.
Constructs an end-of-stream iterator if
\tcode{s.rdbuf()}
is null.
\end{itemdescr}


\indexlibrary{\idxcode{istreambuf_iterator}!constructor}%
\begin{itemdecl}
istreambuf_iterator(const proxy& p) noexcept;
\end{itemdecl}

\begin{itemdescr}
\pnum
\effects
Constructs a
\tcode{istreambuf_iterator<>}
that uses the
\tcode{basic_streambuf<>}
object pointed to by the
\tcode{proxy}
object's constructor argument \tcode{p}.
\end{itemdescr}

\rSec3[istreambuf.iterator::op*]{\tcode{istreambuf_iterator::operator*}}

\indexlibrary{\idxcode{operator*}!\idxcode{istreambuf_iterator}}%
\begin{itemdecl}
charT operator*() const
\end{itemdecl}

\begin{itemdescr}
\pnum
\returns
The character obtained via the
\tcode{streambuf}
member
\tcode{sbuf_->sgetc()}.
\end{itemdescr}

\rSec3[istreambuf.iterator::op++]{\tcode{istreambuf_iterator::operator++}}

\indexlibrary{\idxcode{operator++}!\idxcode{istreambuf_iterator}}%
\begin{itemdecl}
istreambuf_iterator<charT,traits>&
    istreambuf_iterator<charT,traits>::operator++();
\end{itemdecl}

\begin{itemdescr}
\pnum
\effects
\tcode{sbuf_->sbumpc()}.

\pnum
\returns
\tcode{*this}.
\end{itemdescr}

\indexlibrary{\idxcode{operator++}!\idxcode{istreambuf_iterator}}%
\indexlibrary{\idxcode{istreambuf_iterator}!\idxcode{operator++}}%
\begin{itemdecl}
proxy istreambuf_iterator<charT,traits>::operator++(int);
\end{itemdecl}

\begin{itemdescr}
\pnum
\returns
\tcode{proxy(sbuf_->sbumpc(), sbuf_)}.
\end{itemdescr}

\rSec3[istreambuf.iterator::equal]{\tcode{istreambuf_iterator::equal}}

\indexlibrary{\idxcode{equal}!\idxcode{istreambuf_iterator}}%
\begin{itemdecl}
bool equal(const istreambuf_iterator<charT,traits>& b) const;
\end{itemdecl}

\begin{itemdescr}
\pnum
\returns
\tcode{true}
if and only if both iterators are at end-of-stream,
or neither is at end-of-stream, regardless of what
\tcode{streambuf}
object they use.
\end{itemdescr}

\rSec3[istreambuf.iterator::op==]{\tcode{operator==}}

\indexlibrary{\idxcode{operator==}!\idxcode{istreambuf_iterator}}%
\begin{itemdecl}
template <class charT, class traits>
  bool operator==(const istreambuf_iterator<charT,traits>& a,
                  const istreambuf_iterator<charT,traits>& b);
\end{itemdecl}

\begin{itemdescr}
\pnum
\returns
\tcode{a.equal(b)}.
\end{itemdescr}

\rSec3[istreambuf.iterator::op!=]{\tcode{operator!=}}

\indexlibrary{\idxcode{operator"!=}!\idxcode{istreambuf_iterator}}%
\begin{itemdecl}
template <class charT, class traits>
  bool operator!=(const istreambuf_iterator<charT,traits>& a,
                  const istreambuf_iterator<charT,traits>& b);
\end{itemdecl}

\begin{itemdescr}
\pnum
\returns
\tcode{!a.equal(b)}.
\end{itemdescr}

\rSec2[ostreambuf.iterator]{Class template \tcode{ostreambuf_iterator}}

\indexlibrary{\idxcode{ostreambuf_iterator}}%
\begin{codeblock}
namespace std {
  template <class charT, class traits = char_traits<charT> >
  class ostreambuf_iterator @\removed{:}@
    @\removed{public iterator<output_iterator_tag, void, void, void, void>}@ {
  public:
    @\added{typedef output_iterator_tag           iterator_category;}@
    @\added{typedef ptrdiff_t                     difference_type;}@
    typedef charT                         char_type;
    typedef traits                        traits_type;
    typedef basic_streambuf<charT,traits> streambuf_type;
    typedef basic_ostream<charT,traits>   ostream_type;

  public:
    @\added{constexpr ostreambuf_iterator() noexcept;}@
    ostreambuf_iterator(ostream_type& s) noexcept;
    ostreambuf_iterator(streambuf_type* s) noexcept;
    ostreambuf_iterator& operator=(charT c);

    ostreambuf_iterator& operator*();
    ostreambuf_iterator& operator++();
    ostreambuf_iterator& operator++(int);
    bool failed() const noexcept;

  private:
    streambuf_type* sbuf_;                // \expos
  };
}
\end{codeblock}

\pnum
The
class template
\tcode{ostreambuf_iterator}
writes successive
\textit{characters}
onto the output stream from which it was constructed.
It is not possible to get a character value out of the output iterator.

\rSec3[ostreambuf.iter.cons]{\tcode{ostreambuf_iterator} constructors}

\begin{addedblock}
\indexlibrary{\idxcode{ostreambuf_iterator}!constructor}%
\begin{itemdecl}
constexpr ostreambuf_iterator() noexcept;
\end{itemdecl}

\begin{itemdescr}
\pnum
\effects
Initializes \tcode{sbuf_} with null.
\end{itemdescr}
\end{addedblock}

\indexlibrary{\idxcode{ostreambuf_iterator}!constructor}%
\begin{itemdecl}
ostreambuf_iterator(ostream_type& s) noexcept;
\end{itemdecl}

\begin{itemdescr}
\pnum
\requires
\tcode{s.rdbuf()}
shall not null pointer.
\end{itemdescr}

\begin{itemdescr}
\pnum
\effects
Initializes \tcode{sbuf_} with \tcode{s.rdbuf()}.
\end{itemdescr}


\indexlibrary{\idxcode{ostreambuf_iterator}!constructor}%
\begin{itemdecl}
ostreambuf_iterator(streambuf_type* s) noexcept;
\end{itemdecl}

\begin{itemdescr}
\pnum
\requires
\tcode{s}
shall not be a null pointer.

\pnum
\effects
Initializes \tcode{sbuf_} with \tcode{s}.
\end{itemdescr}

\rSec3[ostreambuf.iter.ops]{\tcode{ostreambuf_iterator} operations}

\indexlibrary{\idxcode{operator=}!\idxcode{ostreambuf_iterator}}%
\begin{itemdecl}
ostreambuf_iterator<charT,traits>&
  operator=(charT c);
\end{itemdecl}

\begin{itemdescr}
\pnum
\effects
If
\tcode{failed()}
yields
\tcode{false},
calls
\tcode{sbuf_->sputc(c)};
otherwise has no effect.

\begin{addedblock}
\pnum
\requires \tcode{sbuf_ != 0}.
\end{addedblock}

\pnum
\returns
\tcode{*this}.
\end{itemdescr}

\indexlibrary{\idxcode{operator*}!\idxcode{ostreambuf_iterator}}%
\begin{itemdecl}
ostreambuf_iterator<charT,traits>& operator*();
\end{itemdecl}

\begin{itemdescr}
\pnum
\returns
\tcode{*this}.
\end{itemdescr}

\indexlibrary{\idxcode{operator++}!\idxcode{ostreambuf_iterator}}%
\begin{itemdecl}
ostreambuf_iterator<charT,traits>& operator++();
ostreambuf_iterator<charT,traits>& operator++(int);
\end{itemdecl}

\begin{itemdescr}
\pnum
\returns
\tcode{*this}.
\end{itemdescr}

\indexlibrary{\idxcode{failed}!\idxcode{ostreambuf_iterator}}%
\begin{itemdecl}
bool failed() const noexcept;
\end{itemdecl}

\begin{itemdescr}
\pnum
\returns
\tcode{true}
if in any prior use of member
\tcode{operator=},
the call to
\tcode{sbuf_->sputc()}
returned
\tcode{traits::eof()};
or
\tcode{false}
otherwise.

\begin{addedblock}
\pnum
\requires \tcode{sbuf_ != 0}.
\end{addedblock}
\end{itemdescr}

\begin{addedblock}

\rSec1[iterables]{Iterable concepts}

\rSec2[iterables.general]{General}

\pnum
This subclause describes components for dealing with ranges of elements.

\pnum
The following subclauses describe
iterable and range requirements, and
components for
iterable primitives,
predefined ranges,
and stream ranges,
as summarized in Table~\ref{tab:iterables.lib.summary}.

\begin{libsumtab}{Iterables library summary}{tab:iterables.lib.summary}
  \ref{iterables.requirements} & Requirements       &                           \\ \rowsep
  \ref{iterable.primitives} & Iterable primitives   &   \tcode{<iterator>}      \\
  \ref{predef.range} & Predefined ranges            &                           \\
  \ref{stream.ranges} & Stream ranges               &                           \\
\end{libsumtab}

\rSec2[iterables.requirements]{Iterable requirements}

\rSec3[iterables.requirements.general]{In general}

\pnum
Iterables are an abstraction of containers that allow a C++ program to
operate on elements of data structures uniformly. It their simplest form, an
iterable object is one on which one can call \tcode{begin} and
\tcode{end} to get an iterator~(\ref{iterator.iterators}) and a
sentinel~(\ref{sentinel.iterators}) or an iterator. To be able to construct
template algorithms and range adaptors that work correctly and efficiently on
different types of sequences, the library formalizes not just the interfaces but
also the semantics and complexity assumptions of iterables.

\pnum
This International Standard defines three fundamental categories of iterables
based on the syntax and semantics supported by each: \techterm{iterable},
\techterm{sized iterable} and \techterm{range}, as shown in
Table~\ref{tab:iterables.relations}.

\begin{floattable}{Relations among iterable categories}{tab:iterables.relations}
  {lll}
  \topline
  \textbf{Sized Iterable} &            &                   \\
                          & $\searrow$ &                   \\
                          &            & \textbf{Iterable} \\
                          & $\nearrow$ &                   \\
  \textbf{Range}          &            &                   \\
\end{floattable}

\pnum
The \tcode{Iterable} concept requires only that \tcode{begin} and \tcode{end}
return an iterator and a sentinel. \enternote An iterator is a valid sentinel.
\exitnote The \tcode{SizedIterable} concept refines \tcode{Iterable} but adds
the requirement that the number of elements in the iterable can be determined
in constant time with the \tcode{size} function. The \tcode{Range} concept describes
requirements on an iterable type with constant-time copy, WeakInputIterator and assignment
operators.

\pnum
In addition to the three fundamental iterable categories, this standard defines
a number of convenience refinements of \tcode{Iterable} that group together requirements
that appear often in the concepts, algorithms, and range views. \techterm{Bounded iterables}
are iterables for which \tcode{begin} and \tcode{end} return objects of the
same type. \techterm{Random access iterables} are iterables for which
\tcode{begin} returns a model of
\tcode{RandomAccessIterator}~(\ref{random.access.iterators}). The iterable
categories \techterm{bidirectional iterable}, \techterm{forward iterable},
\techterm{input iterable} and \techterm{output iterable} are defined similarly.
\enternote There is no \techterm{weak input iterable} or
\techterm{weak output iterable} because of the \tcode{EqualityComparable}
requirement on iterators and sentinels. \exitnote \ednote{Rethink that because
a weak input iterable would not require (strongly) incrementable iterators.}

\rSec3[iterable.iterables]{Iterables}

\pnum
The \tcode{Iterable} concept defines the requirements of a type that allows
iteration over its elements by providing a \tcode{begin} iterator and an
\tcode{end} iterator or sentinel.

\begin{codeblock}
template <class T>
concept bool Iterable =
  requires(T t) {
    typename IteratorType<T>;
    typename SentinelType<T>;
    { begin(t) } -> IteratorType<T>;
    { end(t) } -> SentinelType<T>;
    requires IteratorRange<IteratorType, SentinelType>;
  };
\end{codeblock}

\tcode{begin} and \tcode{end} are required to be amortized constant time.
\enternote Most algorithms requiring this concept simply forward to an
Iterator-based algorithm by calling \tcode{begin} and \tcode{end}. \exitnote

\rSec3[sized.iterables]{Sized iterables}

\pnum
The \tcode{SizedIterable} concept describes the requirements of an Iterable
type that knows its size in constant time with the \tcode{size} function.

\begin{codeblock}
// For exposition only:
template <Iterator T>
concept bool SizedIterableLike_ =
  requires(T t) {
    typename SizeType<T>;
    { size(t) } -> SizeType<T>;
    requires Integral<SizeType<T>>;
  };

template <SizedIterableLike_ T>
concept bool SizedIterable =
  is_sized_iterable<T>::value;
\end{codeblock}

\pnum
Any \tcode{Iterable} object \tcode{o} for which \tcode{size(o)} compiles and
returns an \tcode{Integral} type is a \tcode{SizedIterable} by default. The
\tcode{is_sized_iterable} trait allows users to override the default in the
case of accidental conformance.

\enternote
A possible implementation for \tcode{is_sized_iterable} is given below:

\begin{codeblock}
// For exposition only:
template<typename R>
struct is_sized_iterable_impl_
  : std::integral_constant< bool, SizedIterableLike_<R> >
{};

template<typename R>
struct is_sized_iterable
  : conditional<
        is_same<R, remove_const_t<remove_reference_t<R>>>::value,
        is_sized_iterable_impl_<R>,
        is_sized_iterable<remove_const_t<remove_reference_t<R>>>
    >::type
{};
\end{codeblock}
\ednote{The handling of top-level reference here is inconsistent with the other
type traits.}
\exitnote

\rSec3[range.iterables]{Ranges}

\pnum
The \tcode{Range} concept describes the requirements of an Iterable type that
has constant time copy, move and assignment operators; that is, the cost of
these operations is not proportional to the number of elements in the Range.

\pnum
\enterexample
Examples of Ranges are:

\begin{itemize}
\item An Iterable type that wraps a pair of iterators.

\item An Iterable type that hold its elements by \tcode{shared_ptr}
and shares ownership with all its copies.

\item An Iterable type that generates its elements on demand.
\end{itemize}

A container~(\cxxref{containers}) is not a Range since copying the
container copies the elements, which cannot be done in constant time.
\exitexample

\begin{codeblock}
template <Iterable T>
concept bool Range =
  Semiregular<T> && is_range<T>::value;
\end{codeblock}

\pnum
Since the difference between Iterable and Range is largely semantic, the
two are differentiated with the help of the \tcode{is_range} trait. Users may
specialize the \tcode{is_range} trait. By default, \tcode{is_range} uses the
following heuristics to determine whether an Iterable type \tcode{T} is a Range:

\begin{itemize}
\item If \tcode{T} derives from \tcode{range_base}, \tcode{is_range<T>::value}
is true.
\item If a top-level const changes \tcode{T}'s \tcode{IteratorType}'s
\tcode{ReferenceType} type, \tcode{is_range<T>::value}
is false. \enternote Deep constness implies element ownership, whereas shallow
constness implies reference semantics. \exitnote
\end{itemize}

\pnum
\enternote
Below is a possible implementation of the \tcode{is_range} trait.

\begin{codeblock}
struct range_base
{};

// For exposition only:
template <Iterable T>
concept bool ContainerLike_ =
  !Same<decltype(*begin(declval<T &>())),
        decltype(*begin(declval<T const &>()))>;

// For exposition only:
template<typename T>
struct is_range_impl_
  : std::integral_constant<
      bool,
      Iterable<T> && (!ContainerLike_<T> || Derived<T, range_base>)
    >
{};

template<typename T, typename Enable = void>
struct is_range
  : conditional<
      is_same<T, remove_const_t<remove_reference_t<T>>>::value,
      is_range_impl_<T>,
      is_range<remove_const_t<remove_reference_t<T>>>
    >::type
{};
\end{codeblock}
\ednote{The handling of top-level reference here is inconsistent with the other
type traits.}
\exitnote

\rSec3[bounded.iterables]{Bounded iterables}

\pnum
The \tcode{BoundedIterable} concept describes requirements of an Iterable type
for which \tcode{begin} and \tcode{end} return objects of the same type.
\enternote The standard containers~(\cxxref{containers}) are models of
\tcode{BoundedIterable}. \exitnote

\begin{codeblock}
template <Iterable T>
concept bool BoundedIterable =
  Same<IteratorType<T>, SentinelType<T>>;
\end{codeblock}

\rSec3[input.iterables]{Input iterables}

\pnum
The \tcode{InputIterable} concept describes requirements of an Iterable type
for which \tcode{begin} returns a model of
\tcode{InputIterator}~(\ref{input.iterators}).

\begin{codeblock}
template <Iterable T>
concept bool InputIterable =
  InputIterator<IteratorType<T>>;
\end{codeblock}

\rSec3[forward.iterables]{Forward iterables}

\pnum
The \tcode{ForwardIterable} concept describes requirements of an
InputIterable type for which \tcode{begin} returns a model of
\tcode{ForwardIterator}~(\ref{forward.iterators}).

\begin{codeblock}
template <InputIterable T>
concept bool ForwardIterable =
  ForwardIterator<IteratorType<T>>;
\end{codeblock}

\rSec3[bidirectional.iterables]{Bidirectional iterables}

\pnum
The \tcode{BidirectionalIterable} concept describes requirements of a
ForwardIterable type for which \tcode{begin} returns a model of
\tcode{BidirectionalIterator}~(\ref{bidirectional.iterators}).

\begin{codeblock}
template <ForwardIterable T>
concept bool BidirectionalIterable =
  BidirectionalIterator<IteratorType<T>>;
\end{codeblock}

\rSec3[random.access.iterables]{Random access iterables}

\pnum
The \tcode{RandomAccessIterable} concept describes requirements of a
BidirectionalIterable type for which \tcode{begin} returns a model of
\tcode{RandomAccessIterator}~(\ref{random.access.iterators}).

\begin{codeblock}
template <BidirectionalIterable T>
concept bool RandomAccessIterable =
  RandomAccessIterator<IteratorType<T>>;
\end{codeblock}

\end{addedblock}

\rSec1[iterator.range]{range access}

\pnum
In addition to being available via inclusion of the \tcode{<iterator>} header,
the function templates in \ref{iterator.range} are available when any of the following
headers are included: \tcode{<array>}, \tcode{<deque>}, \tcode{<forward_list>},
\tcode{<list>}, \tcode{<map>}, \tcode{<regex>}, \tcode{<set>}, \tcode{<string>},
\tcode{<unordered_map>}, \tcode{<unordered_set>}, and \tcode{<vector>}.

\indexlibrary{\idxcode{begin(C\&)}}%
\begin{itemdecl}
template <class C> auto begin(C& c) -> decltype(c.begin());
template <class C> auto begin(const C& c) -> decltype(c.begin());
\end{itemdecl}

\begin{itemdescr}
\pnum
\returns \tcode{c.begin()}.
\end{itemdescr}

\indexlibrary{\idxcode{end(C\&)}}%
\begin{itemdecl}
template <class C> auto end(C& c) -> decltype(c.end());
template <class C> auto end(const C& c) -> decltype(c.end());
\end{itemdecl}

\begin{itemdescr}
\pnum
\returns \tcode{c.end()}.
\end{itemdescr}

\indexlibrary{\idxcode{begin(T (\&)[N])}}%
\begin{itemdecl}
template <class T, size_t N> constexpr T* begin(T (&array)[N]) noexcept;
\end{itemdecl}

\begin{itemdescr}
\pnum
\returns \tcode{array}.
\end{itemdescr}

\indexlibrary{\idxcode{end(T (\&)[N])}}%
\begin{itemdecl}
template <class T, size_t N> constexpr T* end(T (&array)[N]) noexcept;
\end{itemdecl}

\begin{itemdescr}
\pnum
\returns \tcode{array + N}.
\end{itemdescr}

\indexlibrary{\idxcode{cbegin(const C\&)}}%
\begin{itemdecl}
template <class C> constexpr auto cbegin(const C& c) noexcept(noexcept(std::begin(c)))
  -> decltype(std::begin(c));
\end{itemdecl}
\begin{itemdescr}
\pnum \returns \tcode{std::begin(c)}.
\end{itemdescr}

\indexlibrary{\idxcode{cend(const C\&)}}%
\begin{itemdecl}
template <class C> constexpr auto cend(const C& c) noexcept(noexcept(std::end(c)))
  -> decltype(std::end(c));
\end{itemdecl}
\begin{itemdescr}
\pnum \returns \tcode{std::end(c)}.
\end{itemdescr}

\indexlibrary{\idxcode{rbegin(C\&)}}%
\begin{itemdecl}
template <class C> auto rbegin(C& c) -> decltype(c.rbegin());
template <class C> auto rbegin(const C& c) -> decltype(c.rbegin());
\end{itemdecl}
\begin{itemdescr}
\pnum \returns \tcode{c.rbegin()}.
\end{itemdescr}

\indexlibrary{\idxcode{rend(const C\&)}}%
\begin{itemdecl}
template <class C> auto rend(C& c) -> decltype(c.rend());
template <class C> auto rend(const C& c) -> decltype(c.rend());
\end{itemdecl}
\begin{itemdescr}
\pnum \returns \tcode{c.rend()}.
\end{itemdescr}

\indexlibrary{\idxcode{rbegin(T (\&array)[N])}}%
\begin{itemdecl}
template <class T, size_t N> reverse_iterator<T*> rbegin(T (&array)[N]);
\end{itemdecl}
\begin{itemdescr}
\pnum \returns \tcode{reverse_iterator<T*>(array + N)}.
\end{itemdescr}

\indexlibrary{\idxcode{rend(T (\&array)[N])}}%
\begin{itemdecl}
template <class T, size_t N> reverse_iterator<T*> rend(T (&array)[N]);
\end{itemdecl}
\begin{itemdescr}
\pnum \returns \tcode{reverse_iterator<T*>(array)}.
\end{itemdescr}

\indexlibrary{\idxcode{rbegin(initializer_list<E>)}}%
\begin{itemdecl}
template <class E> reverse_iterator<const E*> rbegin(initializer_list<E> il);
\end{itemdecl}
\begin{itemdescr}
\pnum \returns \tcode{reverse_iterator<const E*>(il.end())}.
\end{itemdescr}

\indexlibrary{\idxcode{rend(initializer_list<E>)}}%
\begin{itemdecl}
template <class E> reverse_iterator<const E*> rend(initializer_list<E> il);
\end{itemdecl}
\begin{itemdescr}
\pnum \returns \tcode{reverse_iterator<const E*>(il.begin())}.
\end{itemdescr}

\indexlibrary{\idxcode{crbegin(const C\& c)}}%
\begin{itemdecl}
template <class C> auto crbegin(const C& c) -> decltype(std::rbegin(c));
\end{itemdecl}
\begin{itemdescr}
\pnum \returns \tcode{std::rbegin(c)}.
\end{itemdescr}

\indexlibrary{\idxcode{crend(const C\& c)}}%
\begin{itemdecl}
template <class C> auto crend(const C& c) -> decltype(std::rend(c));
\end{itemdecl}
\begin{itemdescr}
\pnum \returns \tcode{std::rend(c)}.
\end{itemdescr}


\rSec1[range.access]{Range access}

{\color{oldclr}
\pnum
In addition to being available via inclusion of the \tcode{<experimental/ranges/range>}
header, the customization point objects in \ref{range.access} are
available when \tcode{<experimental/ranges/iterator>} is included.
} % \color{oldclr}

\ednote{The customization point objects in this subsection all have deprecated
behavior that permits them to work with rvalues. This is for compatability with
the similarly named facilities in namespace \tcode{std}. The authors intend to
replace the deprecated behavior with proper support for rvalue ranges, pending
some ongoing work on an improved design. We hope to bring forward such a design
in the summer meeting in Geneva later this year. See discussion in issue
\href{https://github.com/ericniebler/stl2/issues/547}{stl2\#547}.}

\rSec2[range.access.begin]{\tcode{begin}}
\pnum
The name \tcode{begin} denotes a customization point
 object~(\cxxref{customization.point.object}). The expression
\tcode{ranges::begin(E)} for some subexpression \tcode{E} is expression-equivalent to:

\begin{itemize}
\item
  \tcode{ranges::begin(static_cast<const T\&>(E))} if \tcode{E} is an rvalue of
  type \tcode{T}. This usage is deprecated.
  \enternote This deprecated usage exists so that
  \tcode{ranges::begin(E)} behaves similarly to \tcode{std::begin(E)}
  \oldtxt{as defined in ISO/IEC 14882} when \tcode{E} is an rvalue. \exitnote

\item
  Otherwise, \tcode{(E) + 0} if \tcode{E} has array
  type~(\cxxref{basic.compound}).

\item
  Otherwise, \tcode{\textit{DECAY_COPY}((E).begin())} if it is a valid expression and its type \tcode{I} meets the
  syntactic requirements of \tcode{Iterator<I>}. If
  \tcode{Iterator} is not satisfied, the program is ill-formed
  with no diagnostic required.

\item
  Otherwise, \tcode{\textit{DECAY_COPY}(begin(E))} if it is a valid expression and its type \tcode{I} meets the
  syntactic requirements of \tcode{Iterator<I>} with overload
  resolution performed in a context that includes the declaration
  \tcode{\newtxt{template <class T>} void begin(\oldtxt{auto}\newtxt{T}\&) = delete;} and does not include
  a declaration of \tcode{ranges::begin}. If \tcode{Iterator}
  is not satisfied, the program is ill-formed with no diagnostic
  required.

\item
  Otherwise, \tcode{ranges::begin(E)} is ill-formed.
\end{itemize}

\pnum
\enternote Whenever \tcode{ranges::begin(E)} is a valid expression, its
type satisfies \tcode{Iterator}. \exitnote

\rSec2[range.access.end]{\tcode{end}}
\pnum
The name \tcode{end} denotes a customization point
object~(\cxxref{customization.point.object}). The expression
\tcode{ranges::end(E)} for some subexpression \tcode{E} is expression-equivalent to:

\begin{itemize}
\item
  \tcode{ranges::end(static_cast<const T\&>(E))} if \tcode{E} is an rvalue of
  type \tcode{T}. This usage is deprecated.
  \enternote This deprecated usage exists so that
  \tcode{ranges::end(E)} behaves similarly to \tcode{std::end(E)}
  \oldtxt{as defined in ISO/IEC 14882} when \tcode{E} is an rvalue. \exitnote

\item
  Otherwise, \tcode{(E) + extent\newtxt{_v}<T>\oldtxt{::value}} if \tcode{E} has array
  type~(\cxxref{basic.compound}) \tcode{T}.

\item
  Otherwise, \tcode{\textit{DECAY_COPY}((E).end())} if it is a valid expression and its type \tcode{S} meets the
  syntactic requirements of
  \tcode{Sentinel<\brk{}S, decltype(\brk{}ranges::\brk{}begin(E))>}. If
  \tcode{Sentinel} is not satisfied, the program is ill-formed with
  no diagnostic required.

\item
  Otherwise, \tcode{\textit{DECAY_COPY}(end(E))} if it is a valid expression and its type \tcode{S} meets the
  syntactic requirements of
  \tcode{Sentinel<\brk{}S, decltype(\brk{}ranges::\brk{}begin(E))>} with overload
  resolution performed in a context that includes the declaration
  \tcode{\newtxt{template <class T>} void end(\oldtxt{auto}\newtxt{T}\&) = delete;} and does not include
  a declaration of \tcode{ranges::end}. If \tcode{Sentinel} is not
  satisfied, the program is ill-formed with no diagnostic required.

\item
  Otherwise, \tcode{ranges::end(E)} is ill-formed.
\end{itemize}

\pnum
\enternote Whenever \tcode{ranges::end(E)} is a valid expression, the
types of \tcode{ranges::end(E)} and \tcode{ranges::\brk{}begin(E)} satisfy
\tcode{Sentinel}. \exitnote

\rSec2[range.access.cbegin]{\tcode{cbegin}}
\pnum
The name \tcode{cbegin} denotes a customization point
object~(\cxxref{customization.point.object}). The expression
\tcode{ranges::\brk{}cbegin(E)} for some subexpression \tcode{E} of type \tcode{T}
is expression-equivalent to \tcode{ranges::\brk{}begin(static_cast<const T\&>(E))}.

\pnum
Use of \tcode{ranges::cbegin(E)} with rvalue \tcode{E} is deprecated.
\enternote This deprecated usage exists so that \tcode{ranges::cbegin(E)}
behaves similarly to \tcode{std::cbegin(E)} \oldtxt{as defined in ISO/IEC 14882} when
\tcode{E} is an rvalue. \exitnote

\pnum
\enternote Whenever \tcode{ranges::cbegin(E)} is a valid expression, its
type satisfies \tcode{Iterator}. \exitnote

\rSec2[range.access.cend]{\tcode{cend}}
\pnum
The name \tcode{cend} denotes a customization point
object~(\cxxref{customization.point.object}). The expression
\tcode{ranges::cend(E)} for some subexpression \tcode{E} of type \tcode{T}
is expression-equivalent to \tcode{ranges::end(static_cast<const T\&>(E))}.

\pnum
Use of \tcode{ranges::cend(E)} with rvalue \tcode{E} is deprecated.
\enternote This deprecated usage exists so that \tcode{ranges::\brk{}cend(E)}
behaves similarly to \tcode{std::cend(E)} \oldtxt{as defined in ISO/IEC 14882} when
\tcode{E} is an rvalue. \exitnote

\pnum
\enternote Whenever \tcode{ranges::cend(E)} is a valid expression, the
types of \tcode{ranges::cend(E)} and \tcode{ranges::\brk{}cbegin(E)} satisfy
\tcode{Sentinel}. \exitnote

\rSec2[range.access.rbegin]{\tcode{rbegin}}
\pnum
The name \tcode{rbegin} denotes a customization point
object~(\cxxref{customization.point.object}). The expression
\tcode{ranges::rbegin(E)} for some subexpression \tcode{E} is expression-equivalent
to:

\begin{itemize}
\item
  \tcode{ranges::rbegin(static_cast<const T\&>(E))} if \tcode{E} is an rvalue of
  type \tcode{T}. This usage is deprecated.
  \enternote This deprecated usage exists so that
  \tcode{ranges::rbegin(E)} behaves similarly to \tcode{std::rbegin(E)}
  \oldtxt{as defined in ISO/IEC 14882} when \tcode{E} is an rvalue. \exitnote

\item
  Otherwise, \tcode{\textit{DECAY_COPY}((E).rbegin())} if it is a valid expression and its type \tcode{I} meets the
  syntactic requirements of \tcode{Iterator<I>}. If \tcode{Iterator}
  is not satisfied, the program is ill-formed with no diagnostic
  required.

\item
  Otherwise, \tcode{make_reverse_iterator(ranges::end(E))} if both
  \tcode{ranges::begin(E)} and \tcode{ranges::\brk{}end(\brk{}E)} are valid expressions of the same
  type \tcode{I} which meets the syntactic requirements of
  \tcode{Bi\-direct\-ional\-Iterator<I>}~(\ref{range.iterators.bidirectional}).

\item
  Otherwise, \tcode{ranges::rbegin(E)} is ill-formed.
\end{itemize}

\pnum
\enternote Whenever \tcode{ranges::rbegin(E)} is a valid expression, its
type satisfies \tcode{Iterator}. \exitnote

\rSec2[range.access.rend]{\tcode{rend}}
\pnum
The name \tcode{rend} denotes a customization point
object~(\cxxref{customization.point.object}). The expression
\tcode{ranges::rend(E)} for some subexpression \tcode{E} is expression-equivalent to:

\begin{itemize}
\item
  \tcode{ranges::rend(static_cast<const T\&>(E))} if \tcode{E} is an rvalue of
  type \tcode{T}. This usage is deprecated.
  \enternote This deprecated usage exists so that
  \tcode{ranges::rend(E)} behaves similarly to \tcode{std::rend(E)}
  \oldtxt{as defined in ISO/IEC 14882} when \tcode{E} is an rvalue. \exitnote

\item
  Otherwise, \tcode{\textit{DECAY_COPY}((E).rend())} if it is a valid expression and its type \tcode{S} meets the
  syntactic requirements of
  \tcode{Sentinel<\brk{}S, decltype(\brk{}ranges::\brk{}rbegin(E))>}. If
  \tcode{Sentinel} is not satisfied, the program is ill-formed with
  no diagnostic required.

\item
  Otherwise, \tcode{make_reverse_iterator(ranges\colcol{}begin(E))} if both
  \tcode{ranges::begin(E)} and \tcode{ranges\colcol{}end(\brk{}E)} are valid expressions of the same
  type \tcode{I} which meets the syntactic requirements of
  \tcode{Bi\-dir\-ect\-ion\-al\-It\-er\-at\-or<I>}~(\ref{range.iterators.bidirectional}).

\item
  Otherwise, \tcode{ranges::rend(E)} is ill-formed.
\end{itemize}

\pnum
\enternote Whenever \tcode{ranges::rend(E)} is a valid expression, the
types of \tcode{ranges::\brk{}rend(E)} and \tcode{ranges::\brk{}rbegin(E)} satisfy
\tcode{Sentinel}. \exitnote

\rSec2[range.access.crbegin]{\tcode{crbegin}}
\pnum
The name \tcode{crbegin} denotes a customization point
object~(\cxxref{customization.point.object}). The expression
\tcode{ranges::\brk{}crbegin(E)} for some subexpression \tcode{E} of type \tcode{T}
is expression-equivalent to \tcode{ranges::\brk{}rbegin(static_cast<const T\&>(E))}.

\pnum
Use of \tcode{ranges::crbegin(E)} with rvalue \tcode{E} is deprecated.
\enternote This deprecated usage exists so that \tcode{ranges::crbegin(E)}
behaves similarly to \tcode{std::crbegin(E)} \oldtxt{as defined in ISO/IEC 14882} when
\tcode{E} is an rvalue. \exitnote

\pnum
\enternote Whenever \tcode{ranges::crbegin(E)} is a valid expression, its
type satisfies \tcode{Iterator}. \exitnote

\rSec2[range.access.crend]{\tcode{crend}}
\pnum
The name \tcode{crend} denotes a customization point
object~(\cxxref{customization.point.object}). The expression
\tcode{ranges::crend(E)} for some subexpression \tcode{E} of type \tcode{T}
is expression-equivalent to \tcode{ranges::rend(static_cast<const T\&>(E))}.

\pnum
Use of \tcode{ranges::crend(E)} with rvalue \tcode{E} is deprecated.
\enternote This deprecated usage exists so that \tcode{ranges::crend(E)}
behaves similarly to \tcode{std::crend(E)} \oldtxt{as defined in ISO/IEC 14882} when
\tcode{E} is an rvalue. \exitnote

\pnum
\enternote Whenever \tcode{ranges::crend(E)} is a valid expression, the
types of \tcode{ranges::crend(E)} and \tcode{ranges::\brk{}crbegin(\brk{}E)} satisfy
\tcode{Sentinel}. \exitnote

\rSec1[range.primitives]{Range primitives}

\pnum
\oldtxt{In addition to being available via inclusion of the \tcode{<experimental/ranges/range>}
header, the customization point objects in \ref{range.primitives} are
available when \tcode{<experimental/ranges/iterator>} is included.}

\rSec2[range.primitives.size]{\tcode{size}}
\pnum
The name \tcode{size} denotes a customization point
object~(\cxxref{customization.point.object}). The expression
\tcode{ranges::size(E)} for some subexpression \tcode{E} with type
\tcode{T} is expression-equivalent to:

\begin{itemize}
\item
  \tcode{\textit{DECAY_COPY}(extent\newtxt{_v}<T>\oldtxt{::value})} if \tcode{T} is an array
  type~(\cxxref{basic.compound}).

\item
  Otherwise, \tcode{\textit{DECAY_COPY}(static_cast<const T\&>(E).size())} if it is a valid expression and its type \tcode{I}
  satisfies \tcode{Integral<I>} and
  \tcode{disable_\-sized_\-range<T>}~(\ref{range.sized}) is
  \tcode{false}.

\item
  Otherwise, \tcode{\textit{DECAY_COPY}(size(static_cast<const T\&>(E)))} if it is a valid expression and its type \tcode{I}
  satisfies \tcode{Integral<I>} with overload resolution
  performed in a context that includes the declaration
  \tcode{\newtxt{template <class T>} void size(const \oldtxt{auto}\newtxt{T}\&) = delete;} and does not include
  a declaration of \tcode{ranges::size}, and
  \tcode{disable_\-sized_\-range<T>} is \tcode{false}.

\item
  Otherwise,
  \tcode{\textit{DECAY_COPY}(ranges::cend(E) - ranges::cbegin(E))}, except that \tcode{E}
  is only evaluated once, if it is a valid expression and the types \tcode{I} and \tcode{S} of
  \tcode{ranges::cbegin(E)} and \tcode{ranges\colcol{}cend(\brk{}E)} meet the
  syntactic requirements of
  \tcode{SizedSentinel<S, I>}~(\ref{range.iterators.sizedsentinel}) and
  \tcode{Forward\-Iter\-at\-or<I>}. If \tcode{SizedSentinel} and
  \tcode{Forward\-Iter\-at\-or} are not satisfied, the program is ill-formed with no
  diagnostic required.

\item
  Otherwise, \tcode{ranges::size(E)} is ill-formed.
\end{itemize}

\pnum
\enternote Whenever \tcode{ranges::size(E)} is a valid expression, its
type satisfies \tcode{Integral}. \exitnote

\rSec2[range.primitives.empty]{\tcode{empty}}
\pnum
The name \tcode{empty} denotes a customization point
object~(\cxxref{customization.point.object}). The expression
\tcode{ranges::empty(E)} for some subexpression \tcode{E} is
expression-equivalent to:

\begin{itemize}
\item
  \tcode{bool((E).empty())} if it is a valid expression.

\item
  Otherwise, \tcode{ranges::size(E) == 0} if it is a valid expression.

\item
  Otherwise, \tcode{bool(ranges::begin(E) == ranges::end(E))},
  except that \tcode{E} is only evaluated once, if it is a valid expression and the type of
  \tcode{ranges::begin(E)} satisfies \tcode{ForwardIterator}.

\item
  Otherwise, \tcode{ranges::empty(E)} is ill-formed.
\end{itemize}

\pnum
\enternote Whenever \tcode{ranges::empty(E)} is a valid expression, it
has type \tcode{bool}. \exitnote

\rSec2[range.primitives.data]{\tcode{data}}
\pnum
The name \tcode{data} denotes a customization point
object~(\cxxref{customization.point.object}). The expression
\tcode{ranges::data(E)} for some subexpression \tcode{E} is
expression-equivalent to:

\begin{itemize}
\item
  \tcode{ranges::data(static_cast<const T\&>(E))} if \tcode{E} is an rvalue of
  type \tcode{T}. This usage is deprecated. \enternote
  This deprecated usage exists so that \tcode{ranges::data(E)} behaves
  similarly to \tcode{std::data(E)} \oldtxt{as defined in the \Cpp Working
  Paper} when \tcode{E} is an rvalue. \exitnote

\item
  Otherwise, \tcode{\textit{DECAY_COPY}((E).data())} if it is a valid expression of pointer to object type.

\item
  Otherwise, \tcode{ranges::begin(E)} if it is a valid expression of pointer to object type.

\item
  Otherwise, \tcode{ranges::data(E)} is ill-formed.
\end{itemize}

\pnum
\enternote Whenever \tcode{ranges::data(E)} is a valid expression, it
has pointer to object type. \exitnote

\rSec2[range.primitives.cdata]{\tcode{cdata}}
\pnum
The name \tcode{cdata} denotes a customization point
object~(\cxxref{customization.point.object}). The expression
\tcode{ranges::cdata(E)} for some subexpression \tcode{E} of type \tcode{T}
is expression-equivalent to \tcode{ranges::data(static_cast<const T\&>(E))}.

\pnum
Use of \tcode{ranges::cdata(E)} with rvalue \tcode{E} is deprecated.
\enternote This deprecated usage exists so that \tcode{ranges::cdata(E)}
has behavior consistent with \tcode{ranges::data(E)} when \tcode{E} is
an rvalue. \exitnote

\pnum
\enternote Whenever \tcode{ranges::cdata(E)} is a valid expression, it
has pointer to object type. \exitnote

\rSec1[range.requirements]{Range requirements}

\rSec2[range.requirements.general]{General}

\pnum
Ranges are an abstraction of containers that allow a \Cpp program to
operate on elements of data structures uniformly. It their simplest form, a
range object is one on which one can call \tcode{begin} and
\tcode{end} to get an iterator~(\ref{range.iterators.iterator}) and a
sentinel~(\ref{range.iterators.sentinel}). To be able to construct
template algorithms and range adaptors that work correctly and efficiently on
different types of sequences, the library formalizes not just the interfaces but
also the semantics and complexity assumptions of ranges.

\pnum
This document defines three fundamental categories of ranges
based on the syntax and semantics supported by each: \techterm{range},
\techterm{sized range} and \techterm{view}, as shown in
Table~\ref{tab:ranges.relations}.

\begin{floattable}{Relations among range categories}{tab:ranges.relations}
  {lll}
  \topline
  \textbf{Sized Range}  &               &                   \\
                        & $\searrow$    &                   \\
                        &               &  \textbf{Range}   \\
                        & $\nearrow$    &                   \\
  \textbf{View}         &               &                   \\
\end{floattable}

\pnum
The \tcode{Range} concept requires only that \tcode{begin} and \tcode{end}
return an iterator and a sentinel. The \tcode{SizedRange} concept refines \tcode{Range}
with the requirement that the number of elements in the range can be determined
in constant time using the \tcode{size} function. The \tcode{View} concept
specifies requirements on a \tcode{Range} type
with constant-time copy and assign operations.

\pnum
In addition to the three fundamental range categories, this document defines
a number of convenience refinements of \tcode{Range} that group together requirements
that appear often in the concepts and algorithms.
\oldtxt{\textit{Bounded ranges}}\newtxt{\techterm{Common ranges}} are ranges for which
\tcode{begin} and \tcode{end} return objects of the
same type. \techterm{Random access ranges} are ranges for which
\tcode{begin} returns a type that satisfies
\tcode{RandomAccessIterator}~(\ref{range.iterators.random.access}). The range
categories \techterm{bidirectional ranges},
\techterm{forward ranges},
\techterm{input ranges}, and
\techterm{output ranges} are defined similarly.

\rSec2[range.range]{Ranges}

\pnum
The \tcode{Range} concept defines the requirements of a type that allows
iteration over its elements by providing a \tcode{begin} iterator and an
\tcode{end} sentinel.
\enternote Most algorithms requiring this concept simply forward to an
\tcode{Iterator}-based algorithm by calling \tcode{begin} and \tcode{end}. \exitnote

\begin{itemdecl}
template <class T>
concept @\oldtxt{bool}@ Range =
  requires(T&& t) {
    @ranges@::begin(t); // not necessarily equality-preserving (see below)
    @ranges@::end(t);
  };
\end{itemdecl}

\begin{itemdescr}

\pnum
Given an lvalue \tcode{t} of type \tcode{remove_reference_t<T>}, \tcode{Range<T>} is satisfied
only if

\begin{itemize}
\item \range{begin(t)}{end(t)} denotes a range.

\item Both \tcode{begin(t)} and \tcode{end(t)} are amortized constant time
and non-modifying. \enternote \tcode{begin(t)} and \tcode{end(t)} do not require
implicit expression variations~(\cxxref{concepts.lib.general.equality}). \exitnote

\item If \tcode{iterator_t<T>} satisfies \tcode{ForwardIterator},
\tcode{begin(t)} is equality preserving.
\end{itemize}
\end{itemdescr}

\pnum \enternote
Equality preservation of both \tcode{begin} and \tcode{end} enables passing a \tcode{Range}
whose iterator type satisfies \tcode{ForwardIterator}
to multiple algorithms and
making multiple passes over the range by repeated calls to \tcode{begin} and \tcode{end}.
Since \tcode{begin} is not required to be equality preserving when the return type does
not satisfy \tcode{ForwardIterator}, repeated calls might not return equal values or
might not be well-defined; \tcode{begin} should be called at most once for such a range.
\exitnote

\rSec2[range.sized]{Sized ranges}

\pnum
The \tcode{SizedRange} concept specifies the requirements
of a \tcode{Range} type that knows its size in constant time with the
\tcode{size} function.

\begin{itemdecl}
template <class T>
concept @\oldtxt{bool}@ SizedRange =
  Range<T> &&
  !disable_sized_range<remove_cv_t<remove_reference_t<T>>> &&
  requires(T& t) {
    { @ranges@::size(t) } -> ConvertibleTo<difference_type_t<iterator_t<T>>>;
  };
\end{itemdecl}

\begin{itemdescr}
\pnum
Given an lvalue \tcode{t} of type \tcode{remove_reference_t<T>}, \tcode{SizedRange<T>} is satisfied only if:

\begin{itemize}
\item \tcode{ranges::size(t)} is \bigoh{1}, does not modify \tcode{t}, and is equal
to \tcode{ranges::distance(t)}.

\item If \tcode{iterator_t<T>} satisfies \tcode{ForwardIterator},
\tcode{size(t)} is well-defined regardless of the evaluation of
\tcode{begin(t)}. \enternote \tcode{size(t)} is otherwise not required be
well-defined after evaluating \tcode{begin(t)}. For a \tcode{SizedRange}
whose iterator type does not model \tcode{ForwardIterator}, for
example, \tcode{size(t)} might only be well-defined if evaluated before
the first call to \tcode{begin(t)}. \exitnote
\end{itemize}

\pnum
\enternote The \tcode{disable_sized_range} predicate provides a mechanism to enable use
of range types with the library that meet the syntactic requirements but do
not in fact satisfy \tcode{SizedRange}. A program that instantiates a library template
that requires a \tcode{Range} with such a range type \tcode{R} is ill-formed with no
diagnostic required unless
\tcode{disable_sized_range<remove_cv_t<remove_reference_t<R>{>}{>}} evaluates
to \tcode{true}~(\cxxref{structure.requirements}). \exitnote
\end{itemdescr}

\rSec2[range.view]{Views}

\pnum
The \tcode{View} concept specifies the requirements of a
\tcode{Range} type that has constant time copy, move and assignment operators; that
is, the cost of these operations is not proportional to the number of elements in
the \tcode{View}.

\pnum
\enterexample
Examples of \tcode{View}s are:

\begin{itemize}
\item A \tcode{Range} type that wraps a pair of iterators.

\item A \tcode{Range} type that holds its elements by \tcode{shared_ptr}
and shares ownership with all its copies.

\item A \tcode{Range} type that generates its elements on demand.
\end{itemize}

A container~(\cxxref{containers}) is not a \tcode{View} since copying the
container copies the elements, which cannot be done in constant time.
\exitexample

\begin{itemdecl}
template <class T>
constexpr bool @\placeholder{view-predicate}@ // \expos
  = @\seebelow@;

template <class T>
concept @\oldtxt{bool}@ View =
  Range<T> &&
  Semiregular<T> &&
  @\placeholder{view-predicate}@<T>;
\end{itemdecl}

\begin{itemdescr}
\pnum
Since the difference between \tcode{Range} and \tcode{View} is largely semantic, the
two are differentiated with the help of the \tcode{enable_view}
trait. Users may specialize \tcode{enable_view}
to derive from \tcode{true_type} or \tcode{false_type}.

\pnum
For a type \tcode{T}, the value of \tcode{\placeholder{view-predicate}<T>} shall be:
\begin{itemize}
\item If \tcode{enable_view<T>} has a member type \tcode{type}, \tcode{enable_view<T>::type::value};
\item Otherwise, if \tcode{T} is derived from \tcode{view_base}, \tcode{true};
\item Otherwise, if \tcode{T} is an instantiation of class template
\tcode{initializer_list}~(\cxxref{support.initlist}),
\tcode{set}~(\cxxref{set}),
\tcode{multiset}~(\cxxref{multiset}),
\tcode{unordered_set}~(\cxxref{unord.set}), or
\tcode{unordered_multiset}~(\cxxref{unord.multiset}), \tcode{false};
\item Otherwise, if both \tcode{T} and \tcode{const T} satisfy \tcode{Range} and
\tcode{reference_t<iterator_t<T>{>}} is not the same type as \tcode{reference_t<iterator_t<const T>{>}},
\tcode{false}; \enternote Deep \tcode{const}-ness implies element ownership, whereas shallow \tcode{const}-ness
implies reference semantics. \exitnote
\item Otherwise, \tcode{true}.
\end{itemize}
\end{itemdescr}

\rSec2[range.common]{Common ranges}

\ednote{We've renamed ``\tcode{BoundedRange}'' to ``\tcode{CommonRange}''. The authors believe
this is a better name than ``\tcode{ClassicRange}'', which LEWG weakly preferred. The reason is
that the iterator and sentinel of a Common range have the same type in \textit{common}.
A non-Common range can be turned into a Common range with the help of \tcode{common_iterator}.
P0789 ``Range Adaptors and Utilities'' will be proposing a \tcode{view::common} adaptor that
does precisely that.}

\pnum
The \oldtxt{\tcode{BoundedRange}}\newtxt{\tcode{CommonRange}} concept specifies requirements
of a \tcode{Range} type for which \tcode{begin} and \tcode{end} return objects of
the same type. \enternote The standard containers~(\cxxref{containers})
satisfy \oldtxt{\tcode{BoundedRange}}\newtxt{\tcode{CommonRange}}.\exitnote

\begin{codeblock}
template <class T>
concept @\oldtxt{bool}@ @\oldtxt{BoundedRange}\newtxt{CommonRange}@ =
  Range<T> && Same<iterator_t<T>, sentinel_t<T>>;
\end{codeblock}

\rSec2[range.input]{Input ranges}

\pnum
The \tcode{InputRange} concept specifies requirements of
a \tcode{Range} type for which \tcode{begin} returns a type
that satisfies \tcode{InputIterator}~(\ref{range.iterators.input}).

\begin{codeblock}
template <class T>
concept @\oldtxt{bool}@ InputRange =
  Range<T> && InputIterator<iterator_t<T>>;
\end{codeblock}

\rSec2[range.output]{Output ranges}

\pnum
The \tcode{OutputRange} concept specifies requirements of
a \tcode{Range} type for which \tcode{begin} returns a type that satisfies
\tcode{OutputIterator}~(\ref{range.iterators.output}).

\begin{codeblock}
template <class R, class T>
concept @\oldtxt{bool}@ OutputRange =
  Range<R> && OutputIterator<iterator_t<R>, T>;
\end{codeblock}

\rSec2[range.forward]{Forward ranges}

\pnum
The \tcode{ForwardRange} concept specifies requirements of an
\tcode{InputRange} type for which \tcode{begin} returns a type that satisfies
\tcode{ForwardIterator}~(\ref{range.iterators.forward}).

\begin{codeblock}
template <class T>
concept @\oldtxt{bool}@ ForwardRange =
  InputRange<T> && ForwardIterator<iterator_t<T>>;
\end{codeblock}

\rSec2[range.bidirectional]{Bidirectional ranges}

\pnum
The \tcode{BidirectionalRange} concept specifies requirements of a
\tcode{ForwardRange} type for which \tcode{begin} returns a type that satisfies
\tcode{BidirectionalIterator}~(\ref{range.iterators.bidirectional}).

\begin{codeblock}
template <class T>
concept @\oldtxt{bool}@ BidirectionalRange =
  ForwardRange<T> && BidirectionalIterator<iterator_t<T>>;
\end{codeblock}

\rSec2[range.random.access]{Random access ranges}

\pnum
The \tcode{RandomAccessRange} concept specifies requirements of a
\tcode{BidirectionalRange} type for which \tcode{begin} returns a type that satisfies
\tcode{RandomAccessIterator}~(\ref{range.iterators.random.access}).

\begin{codeblock}
template <class T>
concept @\oldtxt{bool}@ RandomAccessRange =
  BidirectionalRange<T> && RandomAccessIterator<iterator_t<T>>;
\end{codeblock}

\rSec1[dangling.wrappers]{Dangling wrapper}

\rSec2[range.dangling.wrap]{Class template \tcode{dangling}}

\pnum
\indexlibrary{\idxcode{dangling}}%
Class template \tcode{dangling} is a wrapper for an object that refers to another object whose
lifetime may have ended. It is used by algorithms that accept rvalue ranges and return iterators.

\begin{codeblock}
namespace std@\newtxt{::ranges}@ { @\oldtxt{namespace experimental \{ namespace ranges \{ inline namespace v1 \{}@
  template <CopyConstructible T>
  class dangling {
  public:
    constexpr dangling() requires DefaultConstructible<T>;
    constexpr dangling(T t);
    constexpr T get_unsafe() const;
  private:
    T value; // \expos
  };

  template <Range R>
  using safe_iterator_t =
    conditional_t<is_lvalue_reference@\newtxt{_v}@<R>@\oldtxt{::value}@,
      iterator_t<R>,
      dangling<iterator_t<R>>>;
}}@\oldtxt{\}\}}@
\end{codeblock}

\rSec3[range.dangling.wrap.ops]{\tcode{dangling} operations}

\rSec4[range.dangling.wrap.op.const]{\tcode{dangling} constructors}

\indexlibrary{\idxcode{dangling}!\idxcode{dangling}}%
\begin{itemdecl}
constexpr dangling() requires DefaultConstructible<T>;
\end{itemdecl}

\begin{itemdescr}
\pnum
\effects Constructs a \tcode{dangling}, value-initializing \tcode{value}.
\end{itemdescr}

\indexlibrary{\idxcode{dangling}!\idxcode{dangling}}%
\begin{itemdecl}
constexpr dangling(T t);
\end{itemdecl}

\begin{itemdescr}
\pnum
\effects Constructs a \tcode{dangling}, initializing \tcode{value} with \tcode{t}.
\end{itemdescr}

\rSec4[range.dangling.wrap.op.get]{\tcode{dangling::get_unsafe}}

\indexlibrary{\idxcode{get_unsafe}!\idxcode{dangling}}%
\indexlibrary{\idxcode{dangling}!\idxcode{get_unsafe}}%
\begin{itemdecl}
constexpr T get_unsafe() const;
\end{itemdecl}

\begin{itemdescr}
\pnum
\returns \tcode{value}.
\end{itemdescr}

%!TEX root = std.tex
\rSec0[algorithms]{Algorithms library}

\rSec1[algorithms.general]{General}

\pnum
This Clause describes components that \Cpp programs may use to perform
algorithmic operations on containers (Clause~\cxxref{containers}) and other sequences.

\pnum
The following subclauses describe components for
non-modifying sequence operation,
modifying sequence operations,
sorting and related operations,
and algorithms from the ISO C library,
as summarized in Table~\ref{tab:algorithms.summary}.

\begin{libsumtab}{Algorithms library summary}{tab:algorithms.summary}
\ref{alg.nonmodifying} & Non-modifying sequence operations  &           \\
\ref{alg.modifying.operations} & Mutating sequence operations & \tcode{<\newtxt{experimental/ranges_v1/}algorithm>} \\
\ref{alg.sorting} & Sorting and related operations      &           \\ \hline
\ref{alg.c.library} & C library algorithms          & \tcode{<cstdlib>} \\ \hline
\end{libsumtab}

\synopsis{Header \tcode{<\newtxt{experimental/ranges_v1/}algorithm>} synopsis}
\begin{addedblock}
\indexlibrary{\idxhdr{experimental/ranges_v1/algorithm}}%

\begin{codeblock}
#include <initializer_list>

namespace std { @\newtxt{namespace experimental \{ namespace ranges_v1 \{}@
  namespace tag {
    // \ref{alg.tagspec}, tag specifiers~(See \ref{taggedtup.tagged}):
    struct in@\oldtxt{put}@;
    struct in@\oldtxt{put}@1;
    struct in@\oldtxt{put}@2;
    struct out@\oldtxt{put}@;
    struct out@\oldtxt{put}@1;
    struct out@\oldtxt{put}@2;
    struct fun;
    struct min;
    struct max;
    struct begin;
    struct end;
  }

  // \ref{alg.nonmodifying}, non-modifying sequence operations:
  template<InputIterator I, Sentinel<I> S, class Proj = identity,
      IndirectCallablePredicate<Projected<I, Proj>> Pred>
    bool all_of(I first, S last, Pred pred, Proj proj = Proj{});

  template<InputRange Rng, class Proj = identity,
      IndirectCallablePredicate<Projected<IteratorType<Rng>, Proj>> Pred>
    bool all_of(Rng&& rng, Pred pred, Proj proj = Proj{});

  template<InputIterator I, Sentinel<I> S, class Proj = identity,
      IndirectCallablePredicate<Projected<I, Proj>> Pred>
    bool any_of(I first, S last, Pred pred, Proj proj = Proj{});

  template<InputRange Rng, class Proj = identity,
      IndirectCallablePredicate<Projected<IteratorType<Rng>, Proj>> Pred>
    bool any_of(Rng&& rng, Pred pred, Proj proj = Proj{});

  template<InputIterator I, Sentinel<I> S, class Proj = identity,
      IndirectCallablePredicate<Projected<I, Proj>> Pred>
    bool none_of(I first, S last, Pred pred, Proj proj = Proj{});

  template<InputRange Rng, class Proj = identity,
      IndirectCallablePredicate<Projected<IteratorType<Rng>, Proj>> Pred>
    bool none_of(Rng&& rng, Pred pred, Proj proj = Proj{});

  template<InputIterator I, Sentinel<I> S, class Proj = identity,
      IndirectCallable<Projected<I, Proj>> Fun>
    tagged_pair<tag::in(I), tag::fun(Fun)>
      for_each(I first, S last, Fun f, Proj proj = Proj{});

  template<InputRange Rng, class Proj = identity,
      IndirectCallable<Projected<IteratorType<Rng>, Proj>> Fun>
    tagged_pair<tag::in(@\oldtxt{IteratorType}\newtxt{safe_iterator_t}@<Rng>), tag::fun(Fun)>
      for_each(Rng&@\newtxt{\&}@ rng, Fun f, Proj proj = Proj{});

  template<InputIterator I, Sentinel<I> S, class T, class Proj = identity>
    requires IndirectCallableRelation<equal_to<>, Projected<I, Proj>, const T *>@\newtxt{()}@
    I find(I first, S last, const T& value, Proj proj = Proj{});

  template<InputRange Rng, class T, class Proj = identity>
    requires IndirectCallableRelation<equal_to<>, Projected<IteratorType<Rng>, Proj>, const T *>@\newtxt{()}@
    @\oldtxt{IteratorType}\newtxt{safe_iterator_t}@<Rng>
      find(Rng&@\newtxt{\&}@ rng, const T& value, Proj proj = Proj{});

  template<InputIterator I, Sentinel<I> S, class Proj = identity,
      IndirectCallablePredicate<Projected<I, Proj>> Pred>
    I find_if(I first, S last, Pred pred, Proj proj = Proj{});

  template<InputRange Rng, class Proj = identity,
      IndirectCallablePredicate<Projected<IteratorType<Rng>, Proj>> Pred>
    @\oldtxt{IteratorType}\newtxt{safe_iterator_t}@<Rng>
      find_if(Rng&@\newtxt{\&}@ rng, Pred pred, Proj proj = Proj{});

  template<InputIterator I, Sentinel<I> S, class Proj = identity,
      IndirectCallablePredicate<Projected<I, Proj>> Pred>
    I find_if_not(I first, S last, Pred pred, Proj proj = Proj{});

  template<InputRange Rng, class Proj = identity,
      IndirectCallablePredicate<Projected<IteratorType<Rng>, Proj>> Pred>
    @\oldtxt{IteratorType}\newtxt{safe_iterator_t}@<Rng>
      find_if_not(Rng&@\newtxt{\&}@ rng, Pred pred, Proj proj = Proj{});

  template<ForwardIterator I1, Sentinel<I1> S1, ForwardIterator I2,
      Sentinel<I2> S2, class Proj = identity,
      IndirectCallableRelation<I2, Projected<I1, Proj>> Pred = equal_to<>>
    I1
      find_end(I1 first1, S1 last1, I2 first2, S2 last2,
               Pred pred = Pred{}, Proj proj = Proj{});

  template<ForwardRange Rng1, ForwardRange Rng2, class Proj = identity,
      IndirectCallableRelation<IteratorType<Rng2>,
        Projected<IteratorType<Rng>, Proj>> Pred = equal_to<>>
    @\oldtxt{IteratorType}\newtxt{safe_iterator_t}@<Rng1>
      find_end(Rng1&@\newtxt{\&}@ rng1, Rng2&& rng2, Pred pred = Pred{}, Proj proj = Proj{});

  template<InputIterator I1, Sentinel<I1> S1, ForwardIterator I2, Sentinel<I2> S2,
      class Proj1 = identity, class Proj2 = identity,
      IndirectCallablePredicate<Projected<I1, Proj1>, Projected<I2, Proj2>> Pred = equal_to<>>
    I1
      find_first_of(I1 first1, S1 last1, I2 first2, S2 last2,
                    Pred pred = Pred{},
                    Proj1 proj1 = Proj1{}, Proj2 proj2 = Proj2{});

  template<InputRange Rng1, ForwardRange Rng2, class Proj1 = identity,
      class Proj2 = identity,
      IndirectCallablePredicate<Projected<IteratorType<Rng1>, Proj1>,
        Projected<IteratorType<Rng2>, Proj2>> Pred = equal_to<>>
    @\oldtxt{IteratorType}\newtxt{safe_iterator_t}@<Rng1>
      find_first_of(Rng1&@\newtxt{\&}@ rng1, Rng2&& rng2,
                    Pred pred = Pred{},
                    Proj1 proj1 = Proj1{}, Proj2 proj2 = Proj2{});

  template<ForwardIterator I, Sentinel<I> S, class Proj = identity,
      IndirectCallableRelation<Projected<I, Proj>> Pred = equal_to<>>
    I
      adjacent_find(I first, S last, Pred pred = Pred{},
                    Proj proj = Proj{});

  template<ForwardRange Rng, class Proj = identity,
      IndirectCallableRelation<Projected<IteratorType<Rng>, Proj>> Pred = equal_to<>>
    @\oldtxt{IteratorType}\newtxt{safe_iterator_t}@<Rng>
      adjacent_find(Rng&@\newtxt{\&}@ rng, Pred pred = Pred{}, Proj proj = Proj{});

  template<InputIterator I, Sentinel<I> S, class T, class Proj = identity>
    requires IndirectCallableRelation<equal_to<>, Projected<I, Proj>, const T *>@\newtxt{()}@
    DifferenceType<I>
      count(I first, S last, const T& value, Proj proj = Proj{});

  template<InputRange Rng, class T, class Proj = identity>
    requires IndirectCallableRelation<equal_to<>, Projected<IteratorType<Rng>, Proj>, const T *>@\newtxt{()}@
    DifferenceType<IteratorType<Rng>>
      count(Rng&& rng, const T& value, Proj proj = Proj{});

  template<InputIterator I, Sentinel<I> S, class Proj = identity,
      IndirectCallablePredicate<Projected<I, Proj>> Pred>
    DifferenceType<I>
      count_if(I first, S last, Pred pred, Proj proj = Proj{});

  template<InputRange Rng, class Proj = identity,
      IndirectCallablePredicate<Projected<IteratorType<Rng>, Proj>> Pred>
    DifferenceType<IteratorType<Rng>>
      count_if(Rng&& rng, Pred pred, Proj proj = Proj{});

  template<InputIterator I1, Sentinel<I1> S1, WeakInputIterator I2,
      class Proj1 = identity, class Proj2 = identity,
      IndirectCallablePredicate<Projected<I1, Proj1>, Projected<I2, Proj2>> Pred = equal_to<>>
    tagged_pair<tag::in1(I1), tag::in2(I2)>
      mismatch(I1 first1, S1 last1, I2 first2, Pred pred = Pred{},
               Proj1 proj1 = Proj1{}, Proj2 proj2 = Proj2{});

  template<InputRange Rng1, WeakInputIterator I2,
      class Proj1 = identity, class Proj2 = identity,
      IndirectCallablePredicate<Projected<IteratorType<Rng1>, Proj1>,
        Projected<I2, Proj2>> Pred = equal_to<>>
    tagged_pair<tag::in1(@\oldtxt{IteratorType}\newtxt{safe_iterator_t}@<Rng1>), tag::in2(I2)>
      mismatch(Rng1&@\newtxt{\&}@ rng1, I2 first2, Pred pred = Pred{},
               Proj1 proj1 = Proj1{}, Proj2 proj2 = Proj2{});

  template<InputIterator I1, Sentinel<I1> S1, InputIterator I2, Sentinel<I2> S2,
      class Proj1 = identity, class Proj2 = identity,
      IndirectCallablePredicate<Projected<I1, Proj1>, Projected<I2, Proj2>> Pred = equal_to<>>
    tagged_pair<tag::in1(I1), tag::in2(I2)>
      mismatch(I1 first1, S1 last1, I2 first2, S2 last2, Pred pred = Pred{},
               Proj1 proj1 = Proj1{}, Proj2 proj2 = Proj2{});

  template<InputRange Rng1, InputRange Rng2,
      class Proj1 = identity, class Proj2 = identity,
      IndirectCallablePredicate<Projected<IteratorType<Rng1>, Proj1>,
        Projected<IteratorType<Rng2>, Proj2>> Pred = equal_to<>>
    tagged_pair<tag::in1(@\oldtxt{IteratorType}\newtxt{safe_iterator_t}@<Rng1>),
                tag::in2(@\oldtxt{IteratorType}\newtxt{safe_iterator_t}@<Rng2>)>
      mismatch(Rng1&@\newtxt{\&}@ rng1, Rng2&@\newtxt{\&}@ rng2, Pred pred = Pred{},
               Proj1 proj1 = Proj1{}, Proj2 proj2 = Proj2{});

  template<InputIterator I1, Sentinel<I1> S1, WeakInputIterator I2,
      class Pred = equal_to<>, class Proj1 = identity, class Proj2 = identity>
    requires IndirectlyComparable<I1, I2, Pred, Proj1, Proj2>@\newtxt{()}@
    bool equal(I1 first1, S1 last1,
               I2 first2, Pred pred = Pred{},
               Proj1 proj1 = Proj1{}, Proj2 proj2 = Proj2{});

  template<InputRange Rng1, WeakInputIterator I2, class Pred = equal_to<>,
      class Proj1 = identity, class Proj2 = identity>
    requires IndirectlyComparable<IteratorType<Rng1>, I2, Pred, Proj1, Proj2>@\newtxt{()}@
    bool equal(Rng1&& rng1, I2 first2, Pred pred = Pred{},
               Proj1 proj1 = Proj1{}, Proj2 proj2 = Proj2{});

  template<InputIterator I1, Sentinel<I1> S1, InputIterator I2, Sentinel<I2> S2,
      class Pred = equal_to<>, class Proj1 = identity, class Proj2 = identity>
    requires IndirectlyComparable<I1, I2, Pred, Proj1, Proj2>@\newtxt{()}@
    bool equal(I1 first1, S1 last1, I2 first2, S2 last2,
               Pred pred = Pred{},
               Proj1 proj1 = Proj1{}, Proj2 proj2 = Proj2{});

  template<InputRange Rng1, InputRange Rng2, class Pred = equal_to<>,
      class Proj1 = identity, class Proj2 = identity>
    requires IndirectlyComparable<IteratorType<Rng1>, IteratorType<Rng2>, Pred, Proj1, Proj2>@\newtxt{()}@
    bool equal(Rng1&& rng1, Rng2&& rng2, Pred pred = Pred{},
               Proj1 proj1 = Proj1{}, Proj2 proj2 = Proj2{});

  template<ForwardIterator I1, Sentinel<I1> S1, ForwardIterator I2,
      class Pred = equal_to<>, class Proj1 = identity, class Proj2 = identity>
    requires IndirectlyComparable<I1, I2, Pred, Proj1, Proj2>@\newtxt{()}@
    bool is_permutation(I1 first1, S1 last1, I2 first2,
                        Pred pred = Pred{},
                        Proj1 proj1 = Proj1{}, Proj2 proj2 = Proj2{});

  template<ForwardRange Rng1, ForwardIterator I2, class Pred = equal_to<>,
      class Proj1 = identity, class Proj2 = identity>
    requires IndirectlyComparable<IteratorType<Rng1>, I2, Pred, Proj1, Proj2>@\newtxt{()}@
    bool is_permutation(Rng1&& rng1, I2 first2, Pred pred = Pred{},
                        Proj1 proj1 = Proj1{}, Proj2 proj2 = Proj2{});

  template<ForwardIterator I1, Sentinel<I1> S1, ForwardIterator I2,
      Sentinel<I2> S2, class Pred = equal_to<>, class Proj1 = identity,
      class Proj2 = identity>
    requires IndirectlyComparable<I1, I2, Pred, Proj1, Proj2>@\newtxt{()}@
    bool is_permutation(I1 first1, S1 last1, I2 first2, S2 last2,
                        Pred pred = Pred{},
                        Proj1 proj1 = Proj1{}, Proj2 proj2 = Proj2{});

  template<ForwardRange Rng1, ForwardRange Rng2, class Pred = equal_to<>,
      class Proj1 = identity, class Proj2 = identity>
    requires IndirectlyComparable<IteratorType<Rng1>, IteratorType<Rng2>, Pred, Proj1, Proj2>@\newtxt{()}@
    bool is_permutation(Rng1&& rng1, Rng2&& rng2, Pred pred = Pred{},
                        Proj1 proj1 = Proj1{}, Proj2 proj2 = Proj2{});

  template<ForwardIterator I1, Sentinel<I1> S1, ForwardIterator I2,
      Sentinel<I2> S2, class Pred = equal_to<>,
      class Proj1 = identity, class Proj2 = identity>
    requires IndirectlyComparable<I1, I2, Pred, Proj1, Proj2>@\newtxt{()}@
    I1
      search(I1 first1, S1 last1, I2 first2, S2 last2,
             Pred pred = Pred{},
             Proj1 proj1 = Proj1{}, Proj2 proj2 = Proj2{});

  template<ForwardRange Rng1, ForwardRange Rng2, class Pred = equal_to<>,
      class Proj1 = identity, class Proj2 = identity>
    requires IndirectlyComparable<IteratorType<Rng1>, IteratorType<Rng2>, Pred, Proj1, Proj2>@\newtxt{()}@
    @\oldtxt{IteratorType}\newtxt{safe_iterator_t}@<Rng1>
      search(Rng1&@\newtxt{\&}@ rng1, Rng2&& rng2, Pred pred = Pred{},
             Proj1 proj1 = Proj1{}, Proj2 proj2 = Proj2{});

  template<ForwardIterator I, Sentinel<I> S, class T,
      class Pred = equal_to<>, class Proj = identity>
    requires IndirectlyComparable<I1, const T*, Pred, Proj>@\newtxt{()}@
    I
      search_n(I first, S last, DifferenceType<I> count,
               const T& value, Pred pred = Pred{},
               Proj proj = Proj{});

  template<ForwardRange Rng, class T, class Pred = equal_to<>,
      class Proj = identity>
    requires IndirectlyComparable<IteratorType<Rng1>, const T*, Pred, Proj>@\newtxt{()}@
    @\oldtxt{IteratorType}\newtxt{safe_iterator_t}@<Rng>
      search_n(Rng&@\newtxt{\&}@ rng, DifferenceType<IteratorType<Rng>> count,
               const T& value, Pred pred = Pred{}, Proj proj = Proj{});

  // \ref{alg.modifying.operations}, modifying sequence operations:
  // \ref{alg.copy}, copy:
  template<InputIterator I, Sentinel<I> S, WeaklyIncrementable O>
    requires IndirectlyCopyable<I, O>@\newtxt{()}@
    tagged_pair<tag::in(I), tag::out(O)>
      copy(I first, S last, O result);

  template<InputRange Rng, WeaklyIncrementable O>
    requires IndirectlyCopyable<IteratorType<Rng>, O>@\newtxt{()}@
    tagged_pair<tag::in(@\oldtxt{IteratorType}\newtxt{safe_iterator_t}@<Rng>), tag::out(O)>
      copy(Rng&@\newtxt{\&}@ rng, O result);

  template<WeakInputIterator I, WeaklyIncrementable O>
    requires IndirectlyCopyable<I, O>@\newtxt{()}@
    tagged_pair<tag::in(I), tag::out(O)>
      copy_n(I first, @\oldtxt{iterator_distance_t}\newtxt{DifferenceType}@<I> n, O result);

  template<InputIterator I, Sentinel<I> S, WeaklyIncrementable O, class Proj = identity,
      IndirectCallablePredicate<Projected<I, Proj>> Pred>
    requires IndirectlyCopyable<I, O>@\newtxt{()}@
    tagged_pair<tag::in(I), tag::out(O)>
      copy_if(I first, S last, O result, Pred pred, Proj proj = Proj{});

  template<InputRange Rng, WeaklyIncrementable O, class Proj = identity,
      IndirectCallablePredicate<Projected<IteratorType<Rng>, Proj>> Pred>
    requires IndirectlyCopyable<IteratorType<Rng>, O>@\newtxt{()}@
    tagged_pair<tag::in(@\oldtxt{IteratorType}\newtxt{safe_iterator_t}@<Rng>), tag::out(O)>
      copy_if(Rng&@\newtxt{\&}@ rng, O result, Pred pred, Proj proj = Proj{});

  template<BidirectionalIterator I1, Sentinel<I1> S1, BidirectionalIterator I2>
    requires IndirectlyCopyable<I1, I2>@\newtxt{()}@
    tagged_pair<tag::in@\oldtxt{1}@(I1), tag::@\oldtxt{in2}\newtxt{out}@(I2)>
      copy_backward(I1 first, I1 last, I2 result);

  template<BidirectionalRange Rng, BidirectionalIterator I>
    requires IndirectlyCopyable<IteratorType<Rng>, I>@\newtxt{()}@
    tagged_pair<tag::in@\oldtxt{1}@(@\oldtxt{IteratorType}\newtxt{safe_iterator_t}@<Rng>), tag::@\oldtxt{in2}\newtxt{out}@(I)>
      copy_backward(Rng&@\newtxt{\&}@ rng, I result);

  // \ref{alg.move}, move:
  template<InputIterator I, Sentinel<I> S, WeaklyIncrementable O>
    requires IndirectlyMovable<I, O>@\newtxt{()}@
    tagged_pair<tag::in(I), tag::out(O)>
      move(I first, S last, O result);

  template<InputRange Rng, WeaklyIncrementable O>
    requires IndirectlyMovable<IteratorType<Rng>, O>@\newtxt{()}@
    tagged_pair<tag::in(@\oldtxt{IteratorType}\newtxt{safe_iterator_t}@<Rng>), tag::out(O)>
      move(Rng&@\newtxt{\&}@ rng, O result);

  template<BidirectionalIterator I1, Sentinel<I1> S1, BidirectionalIterator I2>
    requires IndirectlyMovable<I1, I2>@\newtxt{()}@
    tagged_pair<tag::in@\oldtxt{1}@(I1), tag::@\oldtxt{in2}\newtxt{out}@(I2)>
      move_backward(I1 first, I1 last, I2 result);

  template<BidirectionalRange Rng, BidirectionalIterator I>
    requires IndirectlyMovable<IteratorType<Rng>, I>@\newtxt{()}@
    tagged_pair<tag::in@\oldtxt{1}@(@\oldtxt{IteratorType}\newtxt{safe_iterator_t}@<Rng>), tag::@\oldtxt{in2}\newtxt{out}@(I)>
      move_backward(Rng&@\newtxt{\&}@ rng, I result);

  // \ref{alg.swap}, swap:
  template<ForwardIterator I1, Sentinel<I1> S1, ForwardIterator I2>
    requires IndirectlySwappable<I1, I2>@\newtxt{()}@
    tagged_pair<tag::in1(I1), tag::in2(I2)>
      swap_ranges(I1 first1, S1 last1, I2 first2);

  template<ForwardRange Rng, ForwardIterator I>
    requires IndirectlySwappable<IteratorType<Rng>, I>@\newtxt{()}@
    tagged_pair<tag::in1(@\oldtxt{IteratorType}\newtxt{safe_iterator_t}@<Rng>), tag::in2(I)>
      swap_ranges(Rng&@\newtxt{\&}@ rng1, I first2);

  template<ForwardIterator I1, Sentinel<I1> S1, ForwardIterator I2, Sentinel<I2> S2>
    requires IndirectlySwappable<I1, I2>@\newtxt{()}@
    tagged_pair<tag::in1(I1), tag::in2(I2)>
      swap_ranges(I1 first1, S1 last1, I2 first2, S2 last2);

  template<ForwardRange Rng1, ForwardRange Rng2>
    requires IndirectlySwappable<IteratorType<Rng1>, IteratorType<Rng2>>@\newtxt{()}@
    tagged_pair<tag::in1(@\oldtxt{IteratorType}\newtxt{safe_iterator_t}@<Rng1>), tag::in2(@\oldtxt{IteratorType}\newtxt{safe_iterator_t}@<Rng2>)>
      swap_ranges(Rng1&@\newtxt{\&}@ rng1, Rng2&@\newtxt{\&}@ rng2);

  template<InputIterator I, Sentinel<I> S, class Proj = identity,
      IndirectCallable<Projected<I, Proj>> F,
      WeakOutputIterator<IndirectCallableResultType<F, Projected<I, Proj>>> O>
    tagged_pair<tag::in(I), tag::out(O)>
      transform(I first, S last, O result, F op, Proj proj = Proj{});

  template<InputRange Rng, class Proj = identity,
      IndirectCallable<Projected<IteratorType<Rng>, Proj>> F,
      WeakOutputIterator<IndirectCallableResultType<F,
        Projected<IteratorType<Rng>, Proj>>> O>
    tagged_pair<tag::in(@\oldtxt{IteratorType}\newtxt{safe_iterator_t}@<Rng>), tag::out(O)>
      transform(Rng&@\newtxt{\&}@ rng, O result, F op, Proj proj = Proj{});

  template<InputIterator I1, Sentinel<I1> S1, WeakInputIterator I2,
      class Proj1 = identity, class Proj2 = identity,
      IndirectCallable<Projected<I1, Proj1>, Projected<I2, Proj2>> F,
      WeakOutputIterator<IndirectCallableResultType<F, Projected<I1, Proj1>,
        Projected<I2, Proj2>>> O>
    tagged_tuple<tag::in1(I1), tag::in2(I2), tag::out(O)>
      transform(I1 first1, S1 last1, I2 first2, O result,
                F binary_op, Proj1 proj1 = Proj1{}, Proj2 proj2 = Proj2{});

  template<InputRange Rng, WeakInputIterator I,
      class Proj1 = identity, class Proj2 = identity,
      IndirectCallable<Projected<IteratorType<Rng>, Proj1>, Projected<I, Proj2>> F,
      WeakOutputIterator<IndirectCallableResultType<F,
        Projected<IteratorType<Rng>, Proj1>, Projected<I, Proj2>>> O>
    tagged_tuple<tag::in1(@\oldtxt{IteratorType}\newtxt{safe_iterator_t}@<Rng>), tag::in2(I), tag::out(O)>
      transform(Rng&@\newtxt{\&}@ rng1, I first2, O result,
                F binary_op, Proj1 proj1 = Proj1{}, Proj2 proj2 = Proj2{});

  template<InputIterator I1, Sentinel<I1> S1, InputIterator I2, Sentinel<I2> S2,
      class Proj1 = identity, class Proj2 = identity,
      IndirectCallable<Projected<I1, Proj1>, Projected<I2, Proj2>> F,
      WeakOutputIterator<IndirectCallableResultType<F, Projected<I1, Proj1>,
        Projected<I2, Proj2>>> O>
    tagged_tuple<tag::in1(I1), tag::in2(I2), tag::out(O)>
      transform(I1 first1, S1 last1, I2 first2, S2 last2, O result,
              F binary_op, Proj1 proj1 = Proj1{}, Proj2 proj2 = Proj2{});

  template<InputRange Rng1, InputRange Rng2,
      class Proj1 = identity, class Proj2 = identity,
      IndirectCallable<Projected<IteratorType<Rng1>, Proj1>,
        Projected<IteratorType<Rng2>, Proj2>> F,
      WeakOutputIterator<IndirectCallableResultType<F,
        Projected<IteratorType<Rng1>, Proj1>, Projected<IteratorType<Rng2>, Proj2>>> O>
    tagged_tuple<tag::in1(@\oldtxt{IteratorType}\newtxt{safe_iterator_t}@<Rng1>),
                 tag::in2(@\oldtxt{IteratorType}\newtxt{safe_iterator_t}@<Rng2>),
                 tag::out(O)>
      transform(Rng1&@\newtxt{\&}@ rng1, Rng2&@\newtxt{\&}@ rng2, O result,
                F binary_op, Proj1 proj1 = Proj1{}, Proj2 proj2 = Proj2{});

  \end{codeblock}
  \ednote{REVIEW: In the Palo Alto proposal, \tcode{replace} requires only InputIterators.
      In C++14, it requires Forward.}
  \begin{codeblock}

  template<ForwardIterator I, Sentinel<I> S, class T1, Semiregular T2, class Proj = identity>
    requires Writable<I, T2>@\newtxt{()}@ &&
      IndirectCallableRelation<equal_to<>, Projected<I, Proj>, const T1 *>@\newtxt{()}@
    I
      replace(I first, S last, const T1& old_value, const T2& new_value, Proj proj = Proj{});

  template<ForwardRange Rng, class T1, Semiregular T2, class Proj = identity>
    requires Writable<IteratorType<Rng>, T2>@\newtxt{()}@ &&
      IndirectCallableRelation<equal_to<>, Projected<IteratorType<Rng>, Proj>, const T1 *>@\newtxt{()}@
    @\oldtxt{IteratorType}\newtxt{safe_iterator_t}@<Rng>
      replace(Rng&@\newtxt{\&}@ rng, const T1& old_value, const T2& new_value);

  template<ForwardIterator I, Sentinel<I> S, Semiregular T, class Proj = identity,
      IndirectCallablePredicate<Projected<I, Proj>> Pred>
    requires Writable<I, T>@\newtxt{()}@
    I
      replace_if(I first, S last, Pred pred, const T& new_value, Proj proj = Proj{});

  template<ForwardRange Rng, Semiregular T, class Proj = identity,
      IndirectCallablePredicate<Projected<IteratorType<Rng>, Proj>> Pred>
    requires Writable<IteratorType<Rng>, T>@\newtxt{()}@
    @\oldtxt{IteratorType}\newtxt{safe_iterator_t}@<Rng>
      replace_if(Rng&@\newtxt{\&}@ rng, Pred pred, const T& new_value, Proj proj = Proj{});

  template<InputIterator I, Sentinel<I> S, class T1, Semiregular T2, WeakOutputIterator<T2> O,
      class Proj = identity>
    requires IndirectlyCopyable<I, O>@\newtxt{()}@ &&
      IndirectCallableRelation<equal_to<>, Projected<I, Proj>, const T1 *>@\newtxt{()}@
    tagged_pair<tag::in(I), tag::out(O)>
      replace_copy(I first, S last, O result, const T1& old_value, const T2& new_value,
                   Proj proj = Proj{});

  template<InputRange Rng, class T1, Semiregular T2, WeakOutputIterator<T2> O,
      class Proj = identity>
    requires IndirectlyCopyable<IteratorType<Rng>, O>@\newtxt{()}@ &&
      IndirectCallableRelation<equal_to<>, Projected<IteratorType<Rng>, Proj>, const T1 *>@\newtxt{()}@
    tagged_pair<tag::in(@\oldtxt{IteratorType}\newtxt{safe_iterator_t}@<Rng>), tag::out(O)>
      replace_copy(Rng&@\newtxt{\&}@ rng, O result, const T1& old_value, const T2& new_value,
                   Proj proj = Proj{});

  template<InputIterator I, Sentinel<I> S, Semiregular T, WeakOutputIterator<T> O,
      class Proj = identity, IndirectCallablePredicate<Projected<I, Proj>> Pred>
    requires IndirectlyCopyable<I, O>@\newtxt{()}@
    tagged_pair<tag::in(I), tag::out(O)>
      replace_copy_if(I first, S last, O result, Pred pred, const T& new_value,
                      Proj proj = Proj{});

  template<InputRange Rng, Semiregular T, WeakOutputIterator<T> O, class Proj = identity,
      IndirectCallablePredicate<Projected<IteratorType<Rng>, Proj>> Pred>
    requires IndirectlyCopyable<IteratorType<Rng>, O>@\newtxt{()}@
    tagged_pair<tag::in(@\oldtxt{IteratorType}\newtxt{safe_iterator_t}@<Rng>), tag::out(O)>
      replace_copy_if(Rng&@\newtxt{\&}@ rng, O result, Pred pred, const T& new_value,
                      Proj proj = Proj{});

  \end{codeblock}
  \ednote{REVIEW: N3351 only requires WeaklyIncrementable for fill and generate}
  \begin{codeblock}
  template<Semiregular T, OutputIterator<T> O, Sentinel<O> S>
    O fill(O first, S last, const T& value);

  template<Semiregular T, OutputRange<T> Rng>
    @\oldtxt{IteratorType}\newtxt{safe_iterator_t}@<Rng>
      fill(Rng&@\newtxt{\&}@ rng, const T& value);

  template<Semiregular T, WeakOutputIterator<T> O>
    O fill_n(O first, DifferenceType<O> n, const T& value);

  template<Function F, OutputIterator<ResultType<F>> O,
      Sentinel<O> S>
    O generate(O first, S last, F gen);

  template<Function F, OutputRange<ResultType<F>> Rng>
    @\oldtxt{IteratorType}\newtxt{safe_iterator_t}@<Rng>
      generate(Rng&@\newtxt{\&}@ rng, F gen);

  template<Function F, WeakOutputIterator<ResultType<F>> O>
    O generate_n(O first, @\oldtxt{Distance}\newtxt{Difference}@Type<O> n, F gen);

  template<ForwardIterator I, Sentinel<I> S, class T, class Proj = identity>
    requires Permutable<I>@\newtxt{()}@ &&
      IndirectCallableRelation<equal_to<>, Projected<I, Proj>, const T *>@\newtxt{()}@
    I remove(I first, S last, const T& value, Proj proj = Proj{});

  template<ForwardRange Rng, class T, class Proj = identity>
    requires Permutable<IteratorType<Rng>>@\newtxt{()}@ &&
      IndirectCallableRelation<equal_to<>, Projected<IteratorType<Rng>, Proj>, const T *>@\newtxt{()}@
    @\oldtxt{IteratorType}\newtxt{safe_iterator_t}@<Rng>
      remove(Rng&@\newtxt{\&}@ rng, const T& value, Proj proj = Proj{});

  template<ForwardIterator I, Sentinel<I> S, class Proj = identity,
      IndirectCallablePredicate<Projected<I, Proj>> Pred>
    requires Permutable<I>@\newtxt{()}@
    I remove_if(I first, S last, Pred pred, Proj proj = Proj{});

  template<ForwardRange Rng, class Proj = identity,
      IndirectCallablePredicate<Projected<IteratorType<Rng>, Proj>> Pred>
    requires Permutable<IteratorType<Rng>>@\newtxt{()}@
    @\oldtxt{IteratorType}\newtxt{safe_iterator_t}@<Rng>
      remove_if(Rng&@\newtxt{\&}@ rng, Pred pred, Proj proj = Proj{});

  template<InputIterator I, Sentinel<I> S, WeaklyIncrementable O, class T,
      class Proj = identity>
    requires IndirectlyCopyable<I, O>@\newtxt{()}@ &&
      IndirectCallableRelation<equal_to<>, Projected<I, Proj>, const T *>@\newtxt{()}@
    tagged_pair<tag::in(I), tag::out(O)>
      remove_copy(I first, S last, O result, const T& value, Proj proj = Proj{});

  template<InputRange Rng, WeaklyIncrementable O, class T, class Proj = identity>
    requires IndirectlyCopyable<IteratorType<Rng>, O>@\newtxt{()}@ &&
      IndirectCallableRelation<equal_to<>, Projected<IteratorType<Rng>, Proj>, const T *>@\newtxt{()}@
    tagged_pair<tag::in(@\oldtxt{IteratorType}\newtxt{safe_iterator_t}@<Rng>), tag::out(O)>
      remove_copy(Rng&@\newtxt{\&}@ rng, O result, const T& value, Proj proj = Proj{});

  template<InputIterator I, Sentinel<I> S, WeaklyIncrementable O,
      class Proj = identity, IndirectCallablePredicate<Projected<I, Proj>> Pred>
    requires IndirectlyCopyable<I, O>@\newtxt{()}@
    tagged_pair<tag::in(I), tag::out(O)>
      remove_copy_if(I first, S last, O result, Pred pred, Proj proj = Proj{});

  template<InputRange Rng, WeaklyIncrementable O, class Proj = identity,
      IndirectCallablePredicate<Projected<IteratorType<Rng>, Proj>> Pred>
    requires IndirectlyCopyable<IteratorType<Rng>, O>@\newtxt{()}@
    tagged_pair<tag::in(@\oldtxt{IteratorType}\newtxt{safe_iterator_t}@<Rng>), tag::out(O)>
      remove_copy_if(Rng&@\newtxt{\&}@ rng, O result, Pred pred, Proj proj = Proj{});

  template<ForwardIterator I, Sentinel<I> S, class Proj = identity,
      IndirectCallableRelation<Projected<I, Proj>> R = equal_to<>>
    requires Permutable<I>@\newtxt{()}@
    I unique(I first, S last, R comp = R{}, Proj proj = Proj{});

  template<ForwardRange Rng, class Proj = identity,
      IndirectCallableRelation<Projected<IteratorType<Rng>, Proj>> R = equal_to<>>
    requires Permutable<IteratorType<Rng>>@\newtxt{()}@
    @\oldtxt{IteratorType}\newtxt{safe_iterator_t}@<Rng>
      unique(Rng&@\newtxt{\&}@ rng, R comp = R{}, Proj proj = Proj{});

  template<InputIterator I, Sentinel<I> S, WeaklyIncrementable O,
      class Proj = identity, IndirectCallableRelation<Projected<I, Proj>> R = equal_to<>>
    requires IndirectlyCopyable<I, O>@\newtxt{() \&\& (ForwardIterator<I>() ||}@
      @\newtxt{ForwardIterator<O>() || Copyable<ValueType<I>{}>())}@
    tagged_pair<tag::in(I), tag::out(O)>
      unique_copy(I first, S last, O result, R comp = R{}, Proj proj = Proj{});

  template<InputRange Rng, WeaklyIncrementable O, class Proj = identity,
      IndirectCallableRelation<Projected<IteratorType<Rng>, Proj>> R = equal_to<>>
    requires IndirectlyCopyable<IteratorType<Rng>, O>@\newtxt{() \&\&}@
      @\newtxt{(ForwardIterator<IteratorType<Rng>{}>() || ForwardIterator<O>() ||}@
       @\newtxt{Copyable<ValueType<IteratorType<Rng>{}>{}>())}@
    tagged_pair<tag::in(@\oldtxt{IteratorType}\newtxt{safe_iterator_t}@<Rng>), tag::out(O)>
      unique_copy(Rng&@\newtxt{\&}@ rng, O result, R comp = R{}, Proj proj = Proj{});

  template<BidirectionalIterator I, Sentinel<I> S>
    requires Permutable<I>@\newtxt{()}@
    I reverse(I first, S last);

  template<BidirectionalRange Rng>
    requires Permutable<IteratorType<Rng>>@\newtxt{()}@
    @\oldtxt{IteratorType}\newtxt{safe_iterator_t}@<Rng>
      reverse(Rng&@\newtxt{\&}@ rng);

  template<BidirectionalIterator I, Sentinel<I> S, WeaklyIncrementable O>
    requires IndirectlyCopyable<I, O>@\newtxt{()}@
    tagged_pair<tag::in(I), tag::out(O)> reverse_copy(I first, S last, O result);

  template<BidirectionalRange Rng, WeaklyIncrementable O>
    requires IndirectlyCopyable<IteratorType<Rng>, O>@\newtxt{()}@
    tagged_pair<tag::in(@\oldtxt{IteratorType}\newtxt{safe_iterator_t}@<Rng>), tag::out(O)>
      reverse_copy(Rng&@\newtxt{\&}@ rng, O result);

  \end{codeblock}
  \ednote{Could return a \tcode{range} instead of a \tcode{pair}.
    See Future Work annex~(\ref{future.iterator_range}).}
  \begin{codeblock}
  template<ForwardIterator I, Sentinel<I> S>
    requires Permutable<I>@\newtxt{()}@
    tagged_pair<tag::begin(I), tag::end(I)>
      rotate(I first, I middle, S last);

  template<ForwardRange Rng>
    requires Permutable<IteratorType<Rng>>@\newtxt{()}@
    tagged_pair<tag::begin(@\oldtxt{IteratorType}\newtxt{safe_iterator_t}@<Rng>),
                tag::end(@\oldtxt{IteratorType}\newtxt{safe_iterator_t}@<Rng>)>
      rotate(Rng&@\newtxt{\&}@ rng, IteratorType<Rng> middle);

  template<ForwardIterator I, Sentinel<I> S, WeaklyIncrementable O>
    requires IndirectlyCopyable<I, O>@\newtxt{()}@
    tagged_pair<tag::in(I), tag::out(O)>
      rotate_copy(I first, I middle, S last, O result);

  template<ForwardRange Rng, WeaklyIncrementable O>
    requires IndirectlyCopyable<IteratorType<Rng>, O>@\newtxt{()}@
    tagged_pair<tag::in(@\oldtxt{IteratorType}\newtxt{safe_iterator_t}@<Rng>), tag::out(O)>
      rotate_copy(Rng&@\newtxt{\&}@ rng, IteratorType<Rng> middle, O result);

  // \ref{alg.random.shuffle}, shuffle:
  template<RandomAccessIterator I, Sentinel<I> S, class Gen>
    requires Permutable<I>@\newtxt{()}@ && Convertible@\newtxt{To}@<ResultType<Gen>, DifferenceType<I>>@\newtxt{()}@ &&
      UniformRandomNumberGenerator<remove_reference_t<Gen>>@\newtxt{()}@
    I shuffle(I first, S last, Gen&& g);

  template<RandomAccessRange Rng, class Gen>
    requires Permutable<I>@\newtxt{()}@ && Convertible@\newtxt{To}@<ResultType<Gen>, DifferenceType<I>>@\newtxt{()}@ &&
      UniformRandomNumberGenerator<remove_reference_t<Gen>>
    @\oldtxt{IteratorType}\newtxt{safe_iterator_t}@<Rng>
      shuffle(Rng&@\newtxt{\&}@ rng, Gen&& g);

  // \ref{alg.partitions}, partitions:
  template<InputIterator I, Sentinel<I> S, class Proj = identity,
      IndirectCallablePredicate<Projected<I, Proj>> Pred>
    bool is_partitioned(I first, S last, Pred pred, Proj proj = Proj{});

  template<InputRange Rng, class Proj = identity,
      IndirectCallablePredicate<Projected<IteratorType<Rng>, Proj>> Pred>
    bool
      is_partitioned(Rng&& rng, Pred pred, Proj proj = Proj{});

  template<ForwardIterator I, Sentinel<I> S, class Proj = identity,
      IndirectCallablePredicate<Projected<I, Proj>> Pred>
    requires Permutable<I>@\newtxt{()}@
    I partition(I first, S last, Pred pred, Proj proj = Proj{});

  template<ForwardRange Rng, class Proj = identity,
      IndirectCallablePredicate<Projected<IteratorType<Rng>, Proj>> Pred>
    requires Permutable<IteratorType<Rng>>@\newtxt{()}@
    @\oldtxt{IteratorType}\newtxt{safe_iterator_t}@<Rng>
      partition(Rng&@\newtxt{\&}@ rng, Pred pred, Proj proj = Proj{});

  template<BidirectionalIterator I, Sentinel<I> S, class Proj = identity,
      IndirectCallablePredicate<Projected<I, Proj>> Pred>
    requires Permutable<I>@\newtxt{()}@
    I stable_partition(I first, S last, Pred pred, Proj proj = Proj{});

  template<BidirectionalRange Rng, class Proj = identity,
      IndirectCallablePredicate<Projected<IteratorType<Rng>, Proj>> Pred>
    requires Permutable<IteratorType<Rng>>@\newtxt{()}@
    @\oldtxt{IteratorType}\newtxt{safe_iterator_t}@<Rng>
      stable_partition(Rng&@\newtxt{\&}@ rng, Pred pred, Proj proj = Proj{});

  template<InputIterator I, Sentinel<I> S, WeaklyIncrementable O1, WeaklyIncrementable O2,
      class Proj = identity, IndirectCallablePredicate<Projected<I, Proj>> Pred>
    requires IndirectlyCopyable<I, O1>@\newtxt{()}@ && IndirectlyCopyable<I, O2>@\newtxt{()}@
    tagged_tuple<tag::in(I), tag::out1(O1), tag::out2(O2)>
      partition_copy(I first, S last, O1 out_true, O2 out_false, Pred pred,
                     Proj proj = Proj{});

  template<InputRange Rng, WeaklyIncrementable O1, WeaklyIncrementable O2,
      class Proj = identity,
      IndirectCallablePredicate<Projected<IteratorType<Rng>, Proj>> Pred>
    requires IndirectlyCopyable<IteratorType<Rng>, O1>@\newtxt{()}@ &&
      IndirectlyCopyable<IteratorType<Rng>, O2>@\newtxt{()}@
    tagged_tuple<tag::in(@\oldtxt{IteratorType}\newtxt{safe_iterator_t}@<Rng>), tag::out1(O1), tag::out2(O2)>
      partition_copy(Rng&@\newtxt{\&}@ rng, O1 out_true, O2 out_false, Pred pred, Proj proj = Proj{});

  \end{codeblock}
    \ednote{A new algorithm, needed by stable_partition.}
  \begin{codeblock}
  template<InputIterator I, Sentinel<I> S, WeaklyIncrementable O1, WeaklyIncrementable O2,
      class Proj = identity,
      IndirectCallablePredicate<Projected<I, Proj>> Pred>
    requires IndirectlyMovable<I, O1>@\newtxt{()}@ && IndirectlyMovable<I, O2>@\newtxt{()}@
    tagged_tuple<tag::in(I), tag::out1(O1), tag::out2(O2)>
      partition_move(I first, S last, O1 out_true, O2 out_false, Pred pred,
                     Proj proj = Proj{});

  template<InputRange Rng, WeaklyIncrementable O1, WeaklyIncrementable O2,
      class Proj = identity,
      IndirectCallablePredicate<Projected<IteratorType<Rng>, Proj>> Pred>
    requires IndirectlyMovable<IteratorType<Rng>, O1>@\newtxt{()}@ &&
      IndirectlyMovable<IteratorType<Rng>, O2>@\newtxt{()}@
    tagged_tuple<tag::in(@\oldtxt{IteratorType}\newtxt{safe_iterator_t}@<Rng>), tag::out1(O1), tag::out2(O2)>
      partition_move(Rng&@\newtxt{\&}@ rng, O1 out_true, O2 out_false, Pred pred,
                     Proj proj = Proj{});

  template<ForwardIterator I, Sentinel<I> S, class Proj = identity,
      IndirectCallablePredicate<Projected<I, Proj>> Pred>
    I partition_point(I first, S last, Pred pred, Proj proj = Proj{});

  template<ForwardRange Rng, class Proj = identity,
      IndirectCallablePredicate<Projected<IteratorType<Rng>, Proj>> Pred>
    @\oldtxt{IteratorType}\newtxt{safe_iterator_t}@<Rng>
      partition_point(Rng&@\newtxt{\&}@ rng, Pred pred, Proj proj = Proj{});

  // \ref{alg.sorting}, sorting and related operations:
  // \ref{alg.sort}, sorting:
  template<RandomAccessIterator I, Sentinel<I> S, class Comp = less<>,
      class Proj = identity>
    requires Sortable<I, Comp, Proj>@\newtxt{()}@
    I sort(I first, S last, Comp comp = Comp{}, Proj proj = Proj{});

  template<RandomAccessRange Rng, class Comp = less<>, class Proj = identity>
    requires Sortable<IteratorType<Rng>, Comp, Proj>@\newtxt{()}@
    @\oldtxt{IteratorType}\newtxt{safe_iterator_t}@<Rng>
      sort(Rng&@\newtxt{\&}@ rng, Comp comp = Comp{}, Proj proj = Proj{});

  template<RandomAccessIterator I, Sentinel<I> S, class Comp = less<>,
      class Proj = identity>
    requires Sortable<I, Comp, Proj>@\newtxt{()}@
    I stable_sort(I first, S last, Comp comp = Comp{}, Proj proj = Proj{});

  template<RandomAccessRange Rng, class Comp = less<>, class Proj = identity>
    requires Sortable<IteratorType<Rng>, Comp, Proj>@\newtxt{()}@
    @\oldtxt{IteratorType}\newtxt{safe_iterator_t}@<Rng>
      stable_sort(Rng&@\newtxt{\&}@ rng, Comp comp = Comp{}, Proj proj = Proj{});

  template<RandomAccessIterator I, Sentinel<I> S, class Comp = less<>,
      class Proj = identity>
    requires Sortable<I, Comp, Proj>@\newtxt{()}@
    I partial_sort(I first, I middle, S last, Comp comp = Comp{}, Proj proj = Proj{});

  template<RandomAccessRange Rng, class Comp = less<>, class Proj = identity>
    requires Sortable<IteratorType<Rng>, Comp, Proj>@\newtxt{()}@
    @\oldtxt{IteratorType}\newtxt{safe_iterator_t}@<Rng>
      partial_sort(Rng&@\newtxt{\&}@ rng, IteratorType<Rng> middle, Comp comp = Comp{},
                   Proj proj = Proj{});

  template<InputIterator I1, Sentinel<I> S1, RandomAccessIterator I2, Sentinel<I> S2,
      class R = less<>, class Proj = identity>
    requires IndirectlyCopyable<I1, I2>@\newtxt{()}@ && Sortable<I2, Comp, Proj>@\newtxt{()}@
    I2
      partial_sort_copy(I1 first, S1 last, I2 result_first, S2 result_last,
                        Comp comp = Comp{}, Proj proj = Proj{});

  template<InputRange Rng1, RandomAccessRange Rng2, class R = less<>,
      class Proj = identity>
    requires IndirectlyCopyable<IteratorType<Rng1>, IteratorType<Rng2>>@\newtxt{()}@ &&
        Sortable<IteratorType<Rng2>, Comp, Proj>@\newtxt{()}@
    @\oldtxt{IteratorType}\newtxt{safe_iterator_t}@<Rng2>
      partial_sort_copy(Rng1&@\newtxt{\&}@ rng, Rng2&@\newtxt{\&}@ result_rng, Comp comp = Comp{},
                        Proj proj = Proj{});

  template<ForwardIterator I, Sentinel<I> S, class Proj = identity,
      IndirectCallableStrictWeakOrder<Projected<I, Proj>> Comp = less<>>
    bool is_sorted(I first, S last, Comp comp = Comp{}, Proj proj = Proj{});

  template<ForwardRange Rng, class Proj = identity,
      IndirectCallableStrictWeakOrder<Projected<IteratorType<Rng>, Proj>> Comp = less<>>
    bool
      is_sorted(Rng&& rng, Comp comp = Comp{}, Proj proj = Proj{});

  template<ForwardIterator I, Sentinel<I> S, class Proj = identity,
      IndirectCallableStrictWeakOrder<Projected<I, Proj>> Comp = less<>>
    I is_sorted_until(I first, S last, Comp comp = Comp{}, Proj proj = Proj{});

  template<ForwardRange Rng, class Proj = identity,
      IndirectCallableStrictWeakOrder<Projected<IteratorType<Rng>, Proj>> Comp = less<>>
    @\oldtxt{IteratorType}\newtxt{safe_iterator_t}@<Rng>
      is_sorted_until(Rng&@\newtxt{\&}@ rng, Comp comp = Comp{}, Proj proj = Proj{});

  template<RandomAccessIterator I, Sentinel<I> S, class Comp = less<>,
      class Proj = identity>
    requires Sortable<I, Comp, Proj>@\newtxt{()}@
    I nth_element(I first, I nth, S last, Comp comp, Proj proj = Proj{});

  template<RandomAccessRange Rng, class Comp = less<>, class Proj = identity>
    requires Sortable<IteratorType<Rng>, Comp, Proj>@\newtxt{()}@
    @\oldtxt{IteratorType}\newtxt{safe_iterator_t}@<Rng>
      nth_element(Rng&@\newtxt{\&}@ rng, IteratorType<Rng> nth, Comp comp, Proj proj = Proj{});

  // \ref{alg.binary.search}, binary search:
  template<ForwardIterator I, Sentinel<I> S, TotallyOrdered T, class Proj = identity,
      IndirectCallableStrictWeakOrder<const T *, Projected<I, Proj>> Comp = less<>>
    I
      lower_bound(I first, S last, const T& value, Comp comp = Comp{},
                  Proj proj = Proj{});

  template<ForwardRange Rng, TotallyOrdered T, class Proj = identity,
      IndirectCallableStrictWeakOrder<const T *, Projected<IteratorType<Rng>, Proj>> Comp = less<>>
    @\oldtxt{IteratorType}\newtxt{safe_iterator_t}@<Rng>
      lower_bound(Rng&@\newtxt{\&}@ rng, const T& value, Comp comp = Comp{}, Proj proj = Proj{});

  template<ForwardIterator I, Sentinel<I> S, TotallyOrdered T, class Proj = identity,
      IndirectCallableStrictWeakOrder<const T *, Projected<I, Proj>> Comp = less<>>
    I
      upper_bound(I first, S last, const T& value, Comp comp = Comp{}, Proj proj = Proj{});

  template<ForwardRange Rng, TotallyOrdered T, class Proj = identity,
      IndirectCallableStrictWeakOrder<const T *, Projected<IteratorType<Rng>, Proj>> Comp = less<>>
    @\oldtxt{IteratorType}\newtxt{safe_iterator_t}@<Rng>
      upper_bound(Rng&@\newtxt{\&}@ rng, const T& value, Comp comp = Comp{}, Proj proj = Proj{});

  \end{codeblock}
  \ednote{This could return a \tcode{range} instead of a \tcode{pair}.
    See the Future Work annex~(\ref{future.iterator_range}).}
  \begin{codeblock}
  template<ForwardIterator I, Sentinel<I> S, TotallyOrdered T, class Proj = identity,
      IndirectCallableStrictWeakOrder<const T *, Projected<I, Proj>> Comp = less<>>
    tagged_pair<tag::begin(I), tag::end(I)>
      equal_range(I first, S last, const T& value, Comp comp = Comp{}, Proj proj = Proj{});

  template<ForwardRange Rng, TotallyOrdered T, class Proj = identity,
      IndirectCallableStrictWeakOrder<const T *, Projected<IteratorType<Rng>, Proj>> Comp = less<>>
    tagged_pair<tag::begin(@\oldtxt{IteratorType}\newtxt{safe_iterator_t}@<Rng>),
                tag::end(@\oldtxt{IteratorType}\newtxt{safe_iterator_t}@<Rng>)>
      equal_range(Rng&@\newtxt{\&}@ rng, const T& value, Comp comp = Comp{}, Proj proj = Proj{});

  template<ForwardIterator I, Sentinel<I> S, TotallyOrdered T, class Proj = identity,
      IndirectCallableStrictWeakOrder<const T *, Projected<I, Proj>> Comp = less<>>
    bool
      binary_search(I first, S last, const T& value, Comp comp = Comp{},
                    Proj proj = Proj{});

  template<ForwardRange Rng, TotallyOrdered T, class Proj = identity,
      IndirectCallableStrictWeakOrder<const T *, Projected<IteratorType<Rng>, Proj>> Comp = less<>>
    bool
      binary_search(Rng&@\newtxt{\&}@ rng, const T& value, Comp comp = Comp{},
                    Proj proj = Proj{});

  // \ref{alg.merge}, merge:
  \end{codeblock}
    \ednote{REVIEW: Why does the Palo Alto TR require \tcode{Incrementable} instead of
      \tcode{WeaklyIncrementable}?}
  \begin{codeblock}
  template<InputIterator I1, Sentinel<I1> S1, InputIterator I2, Sentinel<I2> S2,
      Incrementable O, class Comp = less<>, class Proj1 = identity,
      class Proj2 = identity>
    requires Mergeable<I1, I2, O, Comp, Proj1, Proj2>@\newtxt{()}@
    tagged_tuple<tag::in1(I1), tag::in2(I2), tag::out(O)>
      merge(I1 first1, S1 last1, I2 first2, S2 last2, O result,
            Comp comp = Comp{}, Proj1 proj1 = Proj1{}, Proj2 proj2 = Proj2{});

  template<InputRange Rng1, InputRange Rng2, Incrementable O, class Comp = less<>,
      class Proj1 = identity, class Proj2 = identity>
    requires Mergeable<IteratorType<Rng1>, IteratorType<Rng2>, O, Comp, Proj1, Proj2>@\newtxt{()}@
    tagged_tuple<tag::in1(@\oldtxt{IteratorType}\newtxt{safe_iterator_t}@<Rng1>),
                 tag::in2(@\oldtxt{IteratorType}\newtxt{safe_iterator_t}@<Rng2>),
                 tag::out(O)>
      merge(Rng1&@\newtxt{\&}@ rng1, Rng2&@\newtxt{\&}@ rng2, O result,
            Comp comp = Comp{}, Proj1 proj1 = Proj1{}, Proj2 proj2 = Proj2{});

  \end{codeblock}
    \ednote{A new algorithm, needed by \tcode{inplace_merge} and \tcode{stable_sort}.}
  \begin{codeblock}
  template<InputIterator I1, Sentinel<I1> S1, InputIterator I2, Sentinel<I2> S2,
      Incrementable O, class Comp = less<>, class Proj1 = identity,
      class Proj2 = identity>
    requires MergeMovable<I1, I2, O, Comp, Proj1, Proj2>@\newtxt{()}@
    tagged_tuple<tag::in1(I1), tag::in2(I2), tag::out(O)>
      merge_move(I1 first1, S1 last1, I2 first2, S2 last2, O result,
                 Comp comp = Comp{}, Proj1 proj1 = Proj1{}, Proj2 proj2 = Proj2{});

  template<InputRange Rng1, InputRange Rng2, Incrementable O, class Comp = less<>,
      class Proj1 = identity, class Proj2 = identity>
    requires MergeMovable<IteratorType<Rng1>, IteratorType<Rng2>, O, Comp, Proj1, Proj2>@\newtxt{()}@
    tagged_tuple<tag::in1(@\oldtxt{IteratorType}\newtxt{safe_iterator_t}@<Rng1>),
                 tag::in2(@\oldtxt{IteratorType}\newtxt{safe_iterator_t}@<Rng2>),
                 tag::out(O)>
      merge_move(Rng1&@\newtxt{\&}@ rng1, Rng2&@\newtxt{\&}@ rng2, O result,
                 Comp comp = Comp{}, Proj1 proj1 = Proj1{}, Proj2 proj2 = Proj2{});

  template<BidirectionalIterator I, Sentinel<I> S, class Comp = less<>,
      class Proj = identity>
    requires Sortable<I, Comp, Proj>@\newtxt{()}@
    I
      inplace_merge(I first, I middle, S last, Comp comp = Comp{}, Proj proj = Proj{});

  template<BidirectionalRange Rng, class Comp = less<>, class Proj = identity>
    requires Sortable<IteratorType<Rng>, Comp, Proj>@\newtxt{()}@
    @\oldtxt{IteratorType}\newtxt{safe_iterator_t}@<Rng>
      inplace_merge(Rng&@\newtxt{\&}@ rng, IteratorType<Rng> middle, Comp comp = Comp{},
                    Proj proj = Proj{});

  // \ref{alg.set.operations}, set operations:
  template<InputIterator I1, Sentinel<I1> S1, InputIterator I2, Sentinel<I2> S2,
      class Proj1 = identity, class Proj2 = identity,
      IndirectCallableStrictWeakOrder<Projected<I1, Proj1>, Projected<I2, Proj2>> Comp = less<>>
    bool
      includes(I1 first1, S1 last1, I2 first2, S2 last2, Comp comp = Comp{},
               Proj1 proj1 = Proj1{}, Proj2 proj2 = Proj2{});

  template<InputRange Rng1, InputRange Rng2, class Proj1 = identity,
      class Proj2 = identity,
      IndirectCallableStrictWeakOrder<Projected<IteratorType<Rng1>, Proj1>,
        Projected<IteratorType<Rng2>, Proj2>> Comp = less<>>
    bool
      includes(Rng1&& rng1, Rng2&& rng2, Comp comp = Comp{},
               Proj1 proj1 = Proj1{}, Proj2 proj2 = Proj2{});

  template<InputIterator I1, Sentinel<I1> S1, InputIterator I2, Sentinel<I2> S2,
      WeaklyIncrementable O, class Comp = less<>, class Proj1 = identity, class Proj2 = identity>
    requires Mergeable<I1, I2, O, Comp, Proj1, Proj2>@\newtxt{()}@
    tagged_tuple<tag::in1(I1), tag::in2(I2), tag::out(O)>
      set_union(I1 first1, S1 last1, I2 first2, S2 last2, O result, Comp comp = Comp{},
                Proj1 proj1 = Proj1{}, Proj2 proj2 = Proj2{});

  template<InputRange Rng1, InputRange Rng2, WeaklyIncrementable O,
      class Comp = less<>, class Proj1 = identity, class Proj2 = identity>
    requires Mergeable<IteratorType<Rng1>, IteratorType<Rng2>, O, Comp, Proj1, Proj2>@\newtxt{()}@
    tagged_tuple<tag::in1(@\oldtxt{IteratorType}\newtxt{safe_iterator_t}@<Rng1>),
                 tag::in2(@\oldtxt{IteratorType}\newtxt{safe_iterator_t}@<Rng2>),
                 tag::out(O)>
      set_union(Rng1&@\newtxt{\&}@ rng1, Rng2&@\newtxt{\&}@ rng2, O result, Comp comp = Comp{},
                Proj1 proj1 = Proj1{}, Proj2 proj2 = Proj2{});

  template<InputIterator I1, Sentinel<I1> S1, InputIterator I2, Sentinel<I2> S2,
      WeaklyIncrementable O, class Comp = less<>, class Proj1 = identity, class Proj2 = identity>
    requires Mergeable<I1, I2, O, Comp, Proj1, Proj2>@\newtxt{()}@
    O
      set_intersection(I1 first1, S1 last1, I2 first2, S2 last2, O result,
                       Comp comp = Comp{}, Proj1 proj1 = Proj1{}, Proj2 proj2 = Proj2{});

  template<InputRange Rng1, InputRange Rng2, WeaklyIncrementable O,
      class Comp = less<>, class Proj1 = identity, class Proj2 = identity>
    requires Mergeable<IteratorType<Rng1>, IteratorType<Rng2>, O, Comp, Proj1, Proj2>@\newtxt{()}@
    O
      set_intersection(Rng1&& rng1, Rng2&& rng2, O result,
                       Comp comp = Comp{}, Proj1 proj1 = Proj1{}, Proj2 proj2 = Proj2{});

  template<InputIterator I1, Sentinel<I1> S1, InputIterator I2, Sentinel<I2> S2,
      WeaklyIncrementable O, class Comp = less<>, class Proj1 = identity, class Proj2 = identity>
    requires Mergeable<I1, I2, O, Comp, Proj1, Proj2>@\newtxt{()}@
    tagged_pair<tag::in1(I1), tag::out(O)>
      set_difference(I1 first1, S1 last1, I2 first2, S2 last2, O result,
                     Comp comp = Comp{}, Proj1 proj1 = Proj1{}, Proj2 proj2 = Proj2{});

  template<InputRange Rng1, InputRange Rng2, WeaklyIncrementable O,
      class Comp = less<>, class Proj1 = identity, class Proj2 = identity>
    requires Mergeable<IteratorType<Rng1>, IteratorType<Rng2>, O, Comp, Proj1, Proj2>@\newtxt{()}@
    tagged_pair<tag::in1(@\oldtxt{IteratorType}\newtxt{safe_iterator_t}@<Rng1>), tag::out(O)>
      set_difference(Rng1&@\newtxt{\&}@ rng1, Rng2&& rng2, O result,
                     Comp comp = Comp{}, Proj1 proj1 = Proj1{}, Proj2 proj2 = Proj2{});

  template<InputIterator I1, Sentinel<I1> S1, InputIterator I2, Sentinel<I2> S2,
      WeaklyIncrementable O, class Comp = less<>, class Proj1 = identity, class Proj2 = identity>
    requires Mergeable<I1, I2, O, Comp, Proj1, Proj2>@\newtxt{()}@
    tagged_tuple<tag::in1(I1), tag::in2(I2), tag::out(O)>
      set_symmetric_difference(I1 first1, S1 last1, I2 first2, S2 last2, O result,
                               Comp comp = Comp{}, Proj1 proj1 = Proj1{},
                               Proj2 proj2 = Proj2{});

  template<InputRange Rng1, InputRange Rng2, WeaklyIncrementable O,
      class Comp = less<>, class Proj1 = identity, class Proj2 = identity>
    requires Mergeable<IteratorType<Rng1>, IteratorType<Rng2>, O, Comp, Proj1, Proj2>@\newtxt{()}@
    tagged_tuple<tag::in1(@\oldtxt{IteratorType}\newtxt{safe_iterator_t}@<Rng1>),
                 tag::in2(@\oldtxt{IteratorType}\newtxt{safe_iterator_t}@<Rng2>),
                 tag::out(O)>
      set_symmetric_difference(Rng1&@\newtxt{\&}@ rng1, Rng2&@\newtxt{\&}@ rng2, O result, Comp comp = Comp{},
                               Proj1 proj1 = Proj1{}, Proj2 proj2 = Proj2{});

  // \ref{alg.heap.operations}, heap operations:
  template<RandomAccessIterator I, Sentinel<I> S, class Comp = less<>,
      class Proj = identity>
    requires Sortable<I, Comp, Proj>@\newtxt{()}@
    I push_heap(I first, S last, Comp comp = Comp{}, Proj proj = Proj{});

  template<RandomAccessRange Rng, class Comp = less<>, class Proj = identity>
    requires Sortable<IteratorType<Rng>, Comp, Proj>@\newtxt{()}@
    @\oldtxt{IteratorType}\newtxt{safe_iterator_t}@<Rng>
      push_heap(Rng&@\newtxt{\&}@ rng, Comp comp = Comp{}, Proj proj = Proj{});

  template<RandomAccessIterator I, Sentinel<I> S, class Comp = less<>,
      class Proj = identity>
    requires Sortable<I, Comp, Proj>@\newtxt{()}@
    I pop_heap(I first, S last, Comp comp = Comp{}, Proj proj = Proj{});

  template<RandomAccessRange Rng, class Comp = less<>, class Proj = identity>
    requires Sortable<IteratorType<Rng>, Comp, Proj>@\newtxt{()}@
    @\oldtxt{IteratorType}\newtxt{safe_iterator_t}@<Rng>
      pop_heap(Rng&@\newtxt{\&}@ rng, Comp comp = Comp{}, Proj proj = Proj{});

  template<RandomAccessIterator I, Sentinel<I> S, class Comp = less<>,
      class Proj = identity>
    requires Sortable<I, Comp, Proj>@\newtxt{()}@
    I make_heap(I first, S last, Comp comp = Comp{}, Proj proj = Proj{});

  template<RandomAccessRange Rng, class Comp = less<>, class Proj = identity>
    requires Sortable<IteratorType<Rng>, Comp, Proj>@\newtxt{()}@
    @\oldtxt{IteratorType}\newtxt{safe_iterator_t}@<Rng>
      make_heap(Rng&@\newtxt{\&}@ rng, Comp comp = Comp{}, Proj proj = Proj{});

  template<RandomAccessIterator I, Sentinel<I> S, class Comp = less<>,
      class Proj = identity>
    requires Sortable<I, Comp, Proj>@\newtxt{()}@
    I sort_heap(I first, S last, Comp comp = Comp{}, Proj proj = Proj{});

  template<RandomAccessRange Rng, class Comp = less<>, class Proj = identity>
    requires Sortable<IteratorType<Rng>, Comp, Proj>@\newtxt{()}@
    @\oldtxt{IteratorType}\newtxt{safe_iterator_t}@<Rng>
      sort_heap(Rng&@\newtxt{\&}@ rng, Comp comp = Comp{}, Proj proj = Proj{});

  template<RandomAccessIterator I, Sentinel<I> S, class Proj = identity,
      IndirectCallableStrictWeakOrder<Projected<I, Proj>> Comp = less<>>
    bool is_heap(I first, S last, Comp comp = Comp{}, Proj proj = Proj{});

  template<RandomAccessRange Rng, class Proj = identity,
      IndirectCallableStrictWeakOrder<Projected<IteratorType<Rng>, Proj>> Comp = less<>>
    bool
      is_heap(Rng&& rng, Comp comp = Comp{}, Proj proj = Proj{});

  template<RandomAccessIterator I, Sentinel<I> S, class Proj = identity,
      IndirectCallableStrictWeakOrder<Projected<I, Proj>> Comp = less<>>
    I is_heap_until(I first, S last, Comp comp = Comp{}, Proj proj = Proj{});

  template<RandomAccessRange Rng, class Proj = identity,
      IndirectCallableStrictWeakOrder<Projected<IteratorType<Rng>, Proj>> Comp = less<>>
    @\oldtxt{IteratorType}\newtxt{safe_iterator_t}@<Rng>
      is_heap_until(Rng&@\newtxt{\&}@ rng, Comp comp = Comp{}, Proj proj = Proj{});

  // \ref{alg.min.max}, minimum and maximum:
  template<TotallyOrdered T>
    constexpr const T& min(const T& a, const T& b);

  template<class T, class Comp>
    requires StrictWeakOrder<FunctionType<Comp>, T>@\newtxt{()}@
    constexpr const T& min(const T& a, const T& b, Comp comp);

\end{codeblock}
\ednote{REVIEW: The Palo Alto report returns by const reference here but the
  current standard returns by value.}
\begin{codeblock}
  template<TotallyOrdered T>
    requires Semiregular<T>@\newtxt{()}@
    constexpr T min(initializer_list<T> t);

  template<InputRange Rng>
    requires TotallyOrdered<ValueType<IteratorType<Rng>>>() &&
      Semiregular<ValueType<IteratorType<Rng>>>@\newtxt{()}@
    ValueType<IteratorType<Rng>>
      min(Rng&& rng);

  template<Semiregular T, class Comp>
    requires StrictWeakOrder<FunctionType<Comp>, T>@\newtxt{()}@
    constexpr T min(initializer_list<T> t, Comp comp);

  template<InputRange Rng,
      IndirectCallableStrictWeakOrder<IteratorType<Rng>> Comp>
    requires Semiregular<ValueType<IteratorType<Rng>>>@\newtxt{()}@
    ValueType<IteratorType<Rng>>
      min(Rng&& rng, Comp comp);

  template<TotallyOrdered T>
    constexpr const T& max(const T& a, const T& b);

  template<class T, class Comp>
    requires StrictWeakOrder<FunctionType<Comp>, T>@\newtxt{()}@
    constexpr const T& max(const T& a, const T& b, Comp comp);

  template<TotallyOrdered T>
    requires Semiregular<T>@\newtxt{()}@
    constexpr T max(initializer_list<T> t);

  template<InputRange Rng>
    requires TotallyOrdered<ValueType<IteratorType<Rng>>>() &&
      Semiregular<ValueType<IteratorType<Rng>>>@\newtxt{()}@
    ValueType<IteratorType<Rng>>
      max(Rng&& rng);

  template<Semiregular T, class Comp>
    requires StrictWeakOrder<FunctionType<Comp>, T>@\newtxt{()}@
    constexpr T max(initializer_list<T> t, Comp comp);

  template<InputRange Rng,
      IndirectCallableStrictWeakOrder<IteratorType<Rng>> Comp>
    requires Semiregular<ValueType<IteratorType<Rng>>>@\newtxt{()}@
    ValueType<IteratorType<Rng>>
      max(Rng&& rng, Comp comp);

  template<TotallyOrdered T>
    constexpr tagged_pair<tag::min(const T&), tag::max(const T&)>
      minmax(const T& a, const T& b);

  template<class T, class Comp>
    requires StrictWeakOrder<FunctionType<Comp>, T>@\newtxt{()}@
    constexpr tagged_pair<tag::min(const T&), tag::max(const T&)>
      minmax(const T& a, const T& b, Comp comp);

  template<TotallyOrdered T>
    requires Semiregular<T>@\newtxt{()}@
    constexpr tagged_pair<tag::min(T), tag::max(T)>
      minmax(initializer_list<T> t);

  template<InputRange Rng>
    requires TotallyOrdered<ValueType<IteratorType<Rng>>>() &&
      Semiregular<ValueType<IteratorType<Rng>>>@\newtxt{()}@
    tagged_pair<tag::min(ValueType<IteratorType<Rng>>), tag::max(ValueType<IteratorType<Rng>>)>
      minmax(Rng&& rng);

  template<Semiregular T, class Comp>
    requires StrictWeakOrder<FunctionType<Comp>, T>@\newtxt{()}@
    constexpr tagged_pair<tag::min(T), tag::max(T)>
      minmax(initializer_list<T> t, Comp comp);

  template<InputRange Rng,
      IndirectCallableStrictWeakOrder<IteratorType<Rng>> Comp>
    requires Semiregular<ValueType<IteratorType<Rng>>>@\newtxt{()}@
    tagged_pair<tag::min(ValueType<IteratorType<Rng>>), tag::max(ValueType<IteratorType<Rng>>)>
      minmax(Rng&& rng, Comp comp);

  template<ForwardIterator I, Sentinel<I> S, class Proj = identity,
      IndirectCallableStrictWeakOrder<Projected<I, Proj>> Comp = less<>>
    I min_element(I first, S last, Comp comp = Comp{}, Proj proj = Proj{});

  template<ForwardRange Rng, class Proj = identity,
      IndirectCallableStrictWeakOrder<Projected<IteratorType<Rng>, Proj>> Comp = less<>>
    @\oldtxt{IteratorType}\newtxt{safe_iterator_t}@<Rng>
      min_element(Rng&@\newtxt{\&}@ rng, Comp comp = Comp{}, Proj proj = Proj{});

  template<ForwardIterator I, Sentinel<I> S, class Proj = identity,
      IndirectCallableStrictWeakOrder<Projected<I, Proj>> Comp = less<>>
    I max_element(I first, S last, Comp comp = Comp{}, Proj proj = Proj{});

  template<ForwardRange Rng, class Proj = identity,
      IndirectCallableStrictWeakOrder<Projected<IteratorType<Rng>, Proj>> Comp = less<>>
    @\oldtxt{IteratorType}\newtxt{safe_iterator_t}@<Rng>
      max_element(Rng&@\newtxt{\&}@ rng, Comp comp = Comp{}, Proj proj = Proj{});

  template<ForwardIterator I, Sentinel<I> S, class Proj = identity,
      IndirectCallableStrictWeakOrder<Projected<I, Proj>> Comp = less<>>
    tagged_pair<tag::min(I), tag::max(I)>
      minmax_element(I first, S last, Comp comp = Comp{}, Proj proj = Proj{});

  template<ForwardRange Rng, class Proj = identity,
      IndirectCallableStrictWeakOrder<Projected<IteratorType<Rng>, Proj>> Comp = less<>>
    tagged_pair<tag::min(@\oldtxt{IteratorType}\newtxt{safe_iterator_t}@<Rng>),
                tag::max(@\oldtxt{IteratorType}\newtxt{safe_iterator_t}@<Rng>)>
      minmax_element(Rng&@\newtxt{\&}@ rng, Comp comp = Comp{}, Proj proj = Proj{});

  template<InputIterator I1, Sentinel<I1> S1, InputIterator I2, Sentinel<I2> S2,
      class Proj1 = identity, class Proj2 = identity,
      IndirectCallableStrictWeakOrder<Projected<I1, Proj1>, Projected<I2, Proj2>> Comp = less<>>
    bool
      lexicographical_compare(I1 first1, S1 last1, I2 first2, S2 last2,
                              Comp comp = Comp{}, Proj1 proj1 = Proj1{}, Proj2 proj2 = Proj2{});

  template<InputRange Rng1, InputRange Rng2, class Proj1 = identity,
      class Proj2 = identity,
      IndirectCallableStrictWeakOrder<Projected<IteratorType<Rng1>, Proj1>,
        Projected<IteratorType<Rng2>, Proj2>> Comp = less<>>
    bool
      lexicographical_compare(Rng1&& rng1, Rng2&& rng2, Comp comp = Comp{},
                              Proj1 proj1 = Proj1{}, Proj2 proj2 = Proj2{});

  // \ref{alg.permutation.generators}, permutations:
  template<BidirectionalIterator I, Sentinel<I> S, class Comp = less<>,
      class Proj = identity>
    requires Sortable<I, Comp, Proj>@\newtxt{()}@
    bool next_permutation(I first, S last, Comp comp = Comp{}, Proj proj = Proj{});

  template<BidirectionalRange Rng, class Comp = less<>,
      class Proj = identity>
    requires Sortable<IteratorType<Rng>, Comp, Proj>@\newtxt{()}@
    bool
      next_permutation(Rng&& rng, Comp comp = Comp{}, Proj proj = Proj{});

  template<BidirectionalIterator I, Sentinel<I> S, class Comp = less<>,
      class Proj = identity>
    requires Sortable<I, Comp, Proj>@\newtxt{()}@
    bool prev_permutation(I first, S last, Comp comp = Comp{}, Proj proj = Proj{});

  template<BidirectionalRange Rng, class Comp = less<>,
      class Proj = identity>
    requires Sortable<IteratorType<Rng>, Comp, Proj>@\newtxt{()}@
    bool
      prev_permutation(Rng&& rng, Comp comp = Comp{}, Proj proj = Proj{});
}@\newtxt{\}\}}@
\end{codeblock}
\end{addedblock}

\pnum
All of the algorithms are separated from the particular implementations of data structures and are
parameterized by iterator types.
Because of this, they can work with program-defined data structures, as long
as these data structures have iterator types satisfying the assumptions on the algorithms.

\pnum
For purposes of determining the existence of data races, algorithms shall
not modify objects referenced through an iterator argument unless the
specification requires such modification.

\ednote{The following paragraphs are removed because they are redundant; these requirements
are now enforced in code by the requires clauses.}

\begin{removedblock}
\pnum
Throughout this Clause, the names of template parameters
are used to express type requirements.
If an algorithm's template parameter is
\tcode{InputIterator},
\tcode{InputIterator1},
or
\tcode{InputIterator2},
the actual template argument shall satisfy the
requirements of an input iterator~(\ref{input.iterators}).
If an algorithm's template parameter is
\tcode{OutputIterator},
\tcode{OutputIterator1},
or
\tcode{OutputIterator2},
the actual template argument shall satisfy the requirements
of an output iterator~(\ref{output.iterators}).
If an algorithm's template parameter is
\tcode{ForwardIterator},
\tcode{ForwardIterator1},
or
\tcode{ForwardIterator2},
the actual template argument shall satisfy the requirements
of a forward iterator~(\ref{forward.iterators}).
If an algorithm's template parameter is
\tcode{BidirectionalIterator},
\tcode{Bidirectional\-Iterator1},
or
\tcode{BidirectionalIterator2},
the actual template argument shall satisfy the requirements
of a bidirectional iterator~(\ref{bidirectional.iterators}).
If an algorithm's template parameter is
\tcode{RandomAccessIterator},
\tcode{Random\-AccessIterator1},
or
\tcode{RandomAccessIterator2},
the actual template argument shall satisfy the requirements
of a random-access iterator~(\ref{random.access.iterators}).

\pnum
If an algorithm's
\synopsis{Effects}
section says that a value pointed to by any iterator passed
as an argument is modified, then that algorithm has an additional
type requirement:
The type of that argument shall satisfy the requirements
of a mutable iterator~(\ref{iterator.requirements}).
\enternote
This requirement does not affect arguments that are declared as
\tcode{OutputIterator},
\tcode{OutputIterator1},
or
\tcode{OutputIterator2},
because output iterators must always be mutable.
\exitnote
\end{removedblock}

\pnum
Both in-place and copying versions are provided for certain
algorithms.\footnote{The decision whether to include a copying version was
usually based on complexity considerations. When the cost of doing the operation
dominates the cost of copy, the copying version is not included. For example,
\tcode{sort_copy} is not included because the cost of sorting is much more
significant, and users might as well do \tcode{copy} followed by \tcode{sort}.}
When such a version is provided for \textit{algorithm} it is called
\textit{algorithm\tcode{_copy}}. Algorithms that take predicates end with the
suffix \tcode{_if} (which follows the suffix \tcode{_copy}).

\begin{removedblock}
\pnum
The
\tcode{Predicate}
parameter is used whenever an algorithm expects a function object~(\ref{function.objects})
that, when applied to the result
of dereferencing the corresponding iterator, returns a value testable as
\tcode{true}.
In other words, if an algorithm
takes
\tcode{Predicate pred}
as its argument and \tcode{first}
as its iterator argument, it should work correctly in the
construct
\tcode{pred(*first)} contextually converted to \tcode{bool} (Clause~\cxxref{conv}).
The function object
\tcode{pred}
shall not apply any non-constant
function through the dereferenced iterator.

\pnum
The
\tcode{BinaryPredicate}
parameter is used whenever an algorithm expects a function object that when applied to
the result of dereferencing two corresponding iterators or to dereferencing an
iterator and type
\tcode{T}
when
\tcode{T}
is part of the signature returns a value testable as
\tcode{true}.
In other words, if an algorithm takes
\tcode{BinaryPredicate binary_pred}
as its argument and \tcode{first1} and \tcode{first2} as
its iterator arguments, it should work correctly in
the construct
\tcode{binary_pred(*first1, *first2)} contextually converted to \tcode{bool} (Clause~\cxxref{conv}).
\tcode{BinaryPredicate}
always takes the first
iterator's \tcode{value_type}
as its first argument, that is, in those cases when
\tcode{T value}
is part of the signature, it should work
correctly in the
construct \tcode{binary_pred(*first1, value)} contextually converted to \tcode{bool} (Clause~\cxxref{conv}).
\tcode{binary_pred} shall not
apply any non-constant function through the dereferenced iterators.
\end{removedblock}

\begin{addedblock}
\pnum
\enternote
Projections and predicates are typically used as follows:

\begin{codeblock}
auto&& proj_ = @\oldtxt{callable}\newtxt{\xname{as_function}}@(proj); @\newtxt{// see \ref{functiontype.indirectcallables}}@
auto&& pred_ = @\oldtxt{callable}\newtxt{\xname{as_function}}@(pred);
if(pred_(proj_(*first))) // ...
\end{codeblock}
\exitnote
\end{addedblock}

\pnum
\enternote
Unless otherwise specified, algorithms that take function objects as arguments
are permitted to copy those function objects freely. Programmers for whom object
identity is important should consider using a wrapper class that points to a
noncopied implementation object such as \tcode{reference_wrapper<T>}~(\cxxref{refwrap}), or some equivalent solution.
\exitnote

\begin{removedblock}
\pnum
When the description of an algorithm gives an expression such as
\tcode{*first == value}
for a condition, the expression shall evaluate to
either true or false in boolean contexts.
\end{removedblock}

\pnum
In the description of the algorithms operators
\tcode{+}
and
\tcode{-}
are used for some of the iterator categories for which
they do not have to be defined.
In these cases the semantics of
\tcode{a+n}
is the same as that of

\begin{codeblock}
X tmp = a;
advance(tmp, n);
return tmp;
\end{codeblock}

and that of
\tcode{b-a}
is the same as of

\begin{codeblock}
return distance(a, b);
\end{codeblock}

\begin{addedblock}
\pnum
In the description of algorithm return values, sentinel values are sometimes
returned where an iterator is expected. In these cases, the semantics are as
if the sentinel is converted into an iterator as follows:

\begin{codeblock}
I tmp = first;
while(tmp != last)
  ++tmp;
return tmp;
\end{codeblock}

\pnum
Overloads of algorithms that take \tcode{Range} arguments~(\ref{ranges.range})
behave as if they are implemented by calling \tcode{begin} and \tcode{end} on
the \tcode{Range} and dispatching to the overload that takes separate
iterator and sentinel arguments.
\end{addedblock}

\ednote{Before [alg.nonmodifying], insert the following section. All subsequent sections should be
renumbered as appropriate (but they aren't here for the purposes of the review).}

\begin{addedblock}
\section*{25.?? Tag specifiers\hfill[alg.tagspec]}\label{alg.tagspec}
\addcontentsline{toc}{section}{25.?? Tag specifiers}
\setcounter{Paras}{0}

\begin{itemdecl}
namespace tag {
  struct in { /* @\impdef@ */ };
  struct in1 { /* @\impdef@ */ };
  struct in2 { /* @\impdef@ */ };
  struct out { /* @\impdef@ */ };
  struct out1 { /* @\impdef@ */ };
  struct out2 { /* @\impdef@ */ };
  struct fun { /* @\impdef@ */ };
  struct min { /* @\impdef@ */ };
  struct max { /* @\impdef@ */ };
  struct begin { /* @\impdef@ */ };
  struct end { /* @\impdef@ */ };
}
\end{itemdecl}

\begin{itemdescr}
\pnum In the following description, let \tcode{$X$} be the name of a type in the \tcode{tag}
namespace above.

\pnum \tcode{tag::$X$} is a tag specifier~(\ref{taggedtup.tagged}) such that
\tcode{\textit{TAGGET}($D$, tag::$X$, $N$)} names a tagged getter~(\ref{taggedtup.tagged})
with DerivedCharacteristic \tcode{$D$}, ElementIndex \tcode{$N$}, and ElementName \tcode{$X$}.

\pnum \enterexample \tcode{tag::in} is a type such that \tcode{\textit{TAGGET}($D$, tag::in, $N$)}
names a type with the following interface:

\begin{codeblock}
struct @\xname{input_getter}@ {
  constexpr decltype(auto) in() &       { return get<@$N$@>(static_cast<@$D$@&>(*this)); }
  constexpr decltype(auto) in() &&      { return get<@$N$@>(static_cast<@$D$@&&>(*this)); }
  constexpr decltype(auto) in() const & { return get<@$N$@>(static_cast<const @$D$@&>(*this)); }
};
\end{codeblock}
\exitexample
\end{itemdescr}
\end{addedblock}

%%\setcounter{section}{3}

\rSec1[alg.nonmodifying]{Non-modifying sequence operations}

\rSec2[alg.all_of]{All of}

\indexlibrary{\idxcode{all_of}}%
\begin{removedblock}
\begin{itemdecl}
template<class InputIterator, class Predicate>
  bool all_of(InputIterator first, InputIterator last, Predicate pred);
\end{itemdecl}
\end{removedblock}
\begin{addedblock}
\begin{itemdecl}
template<InputIterator I, Sentinel<I> S, class Proj = identity,
    IndirectCallablePredicate<Projected<I, Proj>> Pred>
  bool all_of(I first, S last, Pred pred, Proj proj = Proj{});

template<InputRange Rng, class Proj = identity,
    IndirectCallablePredicate<Projected<IteratorType<Rng>, Proj>> Pred>
  bool all_of(Rng&& rng, Pred pred, Proj proj = Proj{});
\end{itemdecl}
\end{addedblock}

\begin{itemdescr}
\pnum
\returns \tcode{true} if
\range{first}{last} is empty or if
\changed{\tcode{pred(*i)}}{\tcode{\textit{INVOKE}(pred, \textit{INVOKE}(proj, *i))}}
is \tcode{true} for every iterator \tcode{i} in the range \range{first}{last},
and \tcode{false} otherwise.

\pnum
\complexity At most \tcode{last - first} applications of the predicate
\added{ and \tcode{last - first} applications of the projection}.
\end{itemdescr}

\rSec2[alg.any_of]{Any of}

\indexlibrary{\idxcode{any_of}}%
\begin{removedblock}
\begin{itemdecl}
template<class InputIterator, class Predicate>
  bool any_of(InputIterator first, InputIterator last, Predicate pred);
\end{itemdecl}
\end{removedblock}
\begin{addedblock}
\begin{itemdecl}
template<InputIterator I, Sentinel<I> S, class Proj = identity,
    IndirectCallablePredicate<Projected<I, Proj>> Pred>
  bool any_of(I first, S last, Pred pred, Proj proj = Proj{});

template<InputRange Rng, class Proj = identity,
    IndirectCallablePredicate<Projected<IteratorType<Rng>, Proj>> Pred>
  bool any_of(Rng&& rng, Pred pred, Proj proj = Proj{});
\end{itemdecl}
\end{addedblock}

\begin{itemdescr}
\pnum
\returns \tcode{false} if \range{first}{last} is empty or
if there is no iterator \tcode{i} in the range
\range{first}{last} such that
\changed{\tcode{pred(*i)}}{\tcode{\textit{INVOKE}(pred, \textit{INVOKE}(proj, *i))}}
is \tcode{true}, and \tcode{true} otherwise.

\pnum
\complexity At most \tcode{last - first} applications of the predicate
\added{ and \tcode{last - first} applications of the projection}.
\end{itemdescr}

\rSec2[alg.none_of]{None of}

\indexlibrary{\idxcode{none_of}}%
\begin{removedblock}
\begin{itemdecl}
template<class InputIterator, class Predicate>
  bool none_of(InputIterator first, InputIterator last, Predicate pred);
\end{itemdecl}
\end{removedblock}
\begin{addedblock}
\begin{itemdecl}
template<InputIterator I, Sentinel<I> S, class Proj = identity,
    IndirectCallablePredicate<Projected<I, Proj>> Pred>
  bool none_of(I first, S last, Pred pred, Proj proj = Proj{});

template<InputRange Rng, class Proj = identity,
    IndirectCallablePredicate<Projected<IteratorType<Rng>, Proj>> Pred>
  bool none_of(Rng&& rng, Pred pred, Proj proj = Proj{});
\end{itemdecl}
\end{addedblock}

\begin{itemdescr}
\pnum
\returns \tcode{true} if
\range{first}{last} is empty or if
\changed{\tcode{pred(*i)}}{\tcode{\textit{INVOKE}(pred, \textit{INVOKE}(proj, *i))}}
is \tcode{false} for every iterator \tcode{i} in the range \range{first}{last},
and \tcode{false} otherwise.

\pnum
\complexity At most \tcode{last - first} applications of the predicate
\added{ and \tcode{last - first} applications of the projection}.
\end{itemdescr}

\rSec2[alg.foreach]{For each}

\indexlibrary{\idxcode{for_each}}%
\begin{removedblock}
\begin{itemdecl}
template<class InputIterator, class Function>
  Function for_each(InputIterator first, InputIterator last, Function f);
\end{itemdecl}
\end{removedblock}
\begin{addedblock}
\begin{itemdecl}
template<InputIterator I, Sentinel<I> S, class Proj = identity,
    IndirectCallable<Projected<I, Proj>> Fun>
  tagged_pair<tag::in(I), tag::fun(Fun)>
    for_each(I first, S last, Fun f, Proj proj = Proj{});

template<InputRange Rng, class Proj = identity,
    IndirectCallable<Projected<IteratorType<Rng>, Proj>> Fun>
  tagged_pair<tag::in(@\oldtxt{IteratorType}\newtxt{safe_iterator_t}@<Rng>), tag::fun(Fun)>
    for_each(Rng&@\newtxt{\&}@ rng, Fun f, Proj proj = Proj{});
\end{itemdecl}
\end{addedblock}

\begin{itemdescr}
\begin{removedblock}
\pnum
\requires \tcode{Function} shall meet the requirements of
\tcode{MoveConstructible}~(\changed{Table~\cxxref{moveconstructible}}{
\ref{concepts.lib.object.moveconstructible}}).
\enternote \tcode{Function} need not meet the requirements of
\tcode{CopyConstructible}~(\changed{Table~\cxxref{copyconstructible}}{
\ref{concepts.lib.object.copyconstructible}}).\exitnote
\end{removedblock}

\pnum
\effects
\changed{Applies
\tcode{f} to the result of dereferencing every iterator}{Calls
\tcode{\textit{INVOKE}(f, \textit{INVOKE}(proj, *i))} for every iterator
\tcode{i}} in the range
\range{first}{last},
starting from
\tcode{first}
and proceeding to
\tcode{last - 1}.
\enternote If the \changed{type of \tcode{first} satisfies the
requirements of a mutable iterator}{result of
\tcode{\textit{INVOKE}(proj, *i)} is a mutable reference}, \tcode{f} may apply
nonconstant functions\removed{ through the dereferenced iterator}.\exitnote

\pnum
\returns
\changed{\tcode{std::move(f)}}{\tcode{\{last, std::move(f)\}}}.

\pnum
\complexity
Applies \tcode{f}\added{ and \tcode{proj}}
exactly
\tcode{last - first}
times.

\pnum
\notes
If \tcode{f} returns a result, the result is ignored.
\end{itemdescr}

\rSec2[alg.find]{Find}

\indexlibrary{\idxcode{find}}%
\indexlibrary{\idxcode{find_if}}%
\indexlibrary{\idxcode{find_if_not}}%
\begin{removedblock}
\begin{itemdecl}
template<class InputIterator, class T>
  InputIterator find(InputIterator first, InputIterator last,
                     const T& value);

template<class InputIterator, class Predicate>
  InputIterator find_if(InputIterator first, InputIterator last,
                        Predicate pred);
template<class InputIterator, class Predicate>
  InputIterator find_if_not(InputIterator first, InputIterator last,
                            Predicate pred);
\end{itemdecl}
\end{removedblock}
\begin{addedblock}
\begin{itemdecl}
template<InputIterator I, Sentinel<I> S, class T, class Proj = identity>
  requires IndirectCallableRelation<equal_to<>, Projected<I, Proj>, const T *>@\newtxt{()}@
  I find(I first, S last, const T& value, Proj proj = Proj{});

template<InputRange Rng, class T, class Proj = identity>
  requires IndirectCallableRelation<equal_to<>, Projected<IteratorType<Rng>, Proj>, const T *>@\newtxt{()}@
  @\oldtxt{IteratorType}\newtxt{safe_iterator_t}@<Rng>
    find(Rng&@\newtxt{\&}@ rng, const T& value, Proj proj = Proj{});

template<InputIterator I, Sentinel<I> S, class Proj = identity,
    IndirectCallablePredicate<Projected<I, Proj>> Pred>
  I find_if(I first, S last, Pred pred, Proj proj = Proj{});

template<InputRange Rng, class Proj = identity,
    IndirectCallablePredicate<Projected<IteratorType<Rng>, Proj>> Pred>
  @\oldtxt{IteratorType}\newtxt{safe_iterator_t}@<Rng>
    find_if(Rng&@\newtxt{\&}@ rng, Pred pred, Proj proj = Proj{});

template<InputIterator I, Sentinel<I> S, class Proj = identity,
    IndirectCallablePredicate<Projected<I, Proj>> Pred>
  I find_if_not(I first, S last, Pred pred, Proj proj = Proj{});

template<InputRange Rng, class Proj = identity,
    IndirectCallablePredicate<Projected<IteratorType<Rng>, Proj>> Pred>
  @\oldtxt{IteratorType}\newtxt{safe_iterator_t}@<Rng>
    find_if_not(Rng&@\newtxt{\&}@ rng, Pred pred, Proj proj = Proj{});
\end{itemdecl}
\end{addedblock}

\begin{itemdescr}
\pnum
\returns
The first iterator
\tcode{i}
in the range
\range{first}{last}
for which the following corresponding
conditions hold:
\changed{\tcode{*i == value}, \tcode{pred(*i) != false}, \tcode{pred(*i) == false}}
{\tcode{\textit{INVOKE}(proj, *i) == value},
\tcode{\textit{INVOKE}(pred, \textit{INVOKE}(proj, *i)) != false},
\tcode{\textit{INVOKE}(pred, \textit{INVOKE}(proj, *i)) == false}}.
Returns \tcode{last} if no such iterator is found.

\pnum
\complexity
At most
\tcode{last - first}
applications of the corresponding predicate\added{ and projection}.
\end{itemdescr}

\rSec2[alg.find.end]{Find end}

\indexlibrary{\idxcode{find_end}}%
\begin{removedblock}
\begin{itemdecl}
template<class ForwardIterator1, class ForwardIterator2>
  ForwardIterator1
    find_end(ForwardIterator1 first1, ForwardIterator1 last1,
             ForwardIterator2 first2, ForwardIterator2 last2);

template<class ForwardIterator1, class ForwardIterator2,
         class BinaryPredicate>
  ForwardIterator1
    find_end(ForwardIterator1 first1, ForwardIterator1 last1,
             ForwardIterator2 first2, ForwardIterator2 last2,
             BinaryPredicate pred);
\end{itemdecl}
\end{removedblock}
\begin{addedblock}
\begin{itemdecl}
template<ForwardIterator I1, Sentinel<I1> S1, ForwardIterator I2,
    Sentinel<I2> S2, class Proj = identity,
    IndirectCallableRelation<I2, Projected<I1, Proj>> Pred = equal_to<>>
  I1
    find_end(I1 first1, S1 last1, I2 first2, S2 last2,
             Pred pred = Pred{}, Proj proj = Proj{});

template<ForwardRange Rng1, ForwardRange Rng2,
    class Proj = identity,
    IndirectCallableRelation<IteratorType<Rng2>,
      Projected<IteratorType<Rng>, Proj>> Pred = equal_to<>>
  @\oldtxt{IteratorType}\newtxt{safe_iterator_t}@<Rng1>
    find_end(Rng1&@\newtxt{\&}@ rng1, Rng2&& rng2, Pred pred = Pred{}, Proj proj = Proj{});
\end{itemdecl}
\end{addedblock}

\begin{itemdescr}
\pnum
\effects
Finds a subsequence of equal values in a sequence.

\pnum
\returns
The last iterator
\tcode{i}
in the range \range{first1}{last1 - (last2 - first2)}
such that for every non-negative integer
\tcode{n < (last2 - first2)},
the following \removed{corresponding }condition\removed{s} hold\added{s}:
\tcode{\removed{*(i + n) == *(first2 + n), pred(*(i + n), *(first2 + n)) != false}
\added{\textit{INVOKE}(pred, \textit{INVOKE}(proj, *(i + n)), *(first2 + n)) != false}}.
Returns \tcode{last1}
if
\range{first2}{last2} is empty or if
no such iterator is found.

\pnum
\complexity
At most
\tcode{(last2 - first2) * (last1 - first1 - (last2 - first2) + 1)}
applications of the corresponding predicate\added{ and projection}.
\end{itemdescr}

\rSec2[alg.find.first.of]{Find first}

\indexlibrary{\idxcode{find_first_of}}%
\begin{removedblock}
\begin{itemdecl}
template<class InputIterator, class ForwardIterator>
  InputIterator
    find_first_of(InputIterator first1, InputIterator last1,
                  ForwardIterator first2, ForwardIterator last2);

template<class InputIterator, class ForwardIterator,
          class BinaryPredicate>
  InputIterator
    find_first_of(InputIterator first1, InputIterator last1,
                  ForwardIterator first2, ForwardIterator last2,
                  BinaryPredicate pred);
\end{itemdecl}
\end{removedblock}
\begin{addedblock}
\begin{itemdecl}
template<InputIterator I1, Sentinel<I1> S1, ForwardIterator I2, Sentinel<I2> S2,
    class Proj1 = identity, class Proj2 = identity,
    IndirectCallablePredicate<Projected<I1, Proj1>, Projected<I2, Proj2>> Pred = equal_to<>>
  I1
    find_first_of(I1 first1, S1 last1, I2 first2, S2 last2, Pred pred = Pred{},
                  Proj1 proj1 = Proj1{}, Proj2 proj2 = Proj2{});

template<InputRange Rng1, ForwardRange Rng2, class Proj1 = identity,
    class Proj2 = identity,
    IndirectCallablePredicate<Projected<IteratorType<Rng1>, Proj1>,
      Projected<IteratorType<Rng2>, Proj2>> Pred = equal_to<>>
  @\oldtxt{IteratorType}\newtxt{safe_iterator_t}@<Rng1>
    find_first_of(Rng1&@\newtxt{\&}@ rng1, Rng2&& rng2, Pred pred = Pred{},
                  Proj1 proj1 = Proj1{}, Proj2 proj2 = Proj2{});
\end{itemdecl}
\end{addedblock}

\begin{itemdescr}
\pnum
\effects
Finds an element that matches one of a set of values.

\pnum
\returns
The first iterator
\tcode{i}
in the range \range{first1}{last1}
such that for some
iterator
\tcode{j}
in the range \range{first2}{last2}
the following condition\removed{s} hold\added{s}:
\tcode{\removed{*i == *j, pred(*i,*j) != false}
\added{\textit{INVOKE}(pred,}\brk\added{ \textit{INVOKE}(proj1, *i), \textit{INVOKE}(proj2, *j)) != false}}.
Returns \tcode{last1}
if \range{first2}{last2} is empty or if
no such iterator is found.

\pnum
\complexity
At most
\tcode{(last1-first1) * (last2-first2)}
applications of the corresponding predicate\added{ and the two projections}.
\end{itemdescr}

\rSec2[alg.adjacent.find]{Adjacent find}

\indexlibrary{\idxcode{adjacent_find}}%
\begin{removedblock}
\begin{itemdecl}
template<class ForwardIterator>
  ForwardIterator adjacent_find(ForwardIterator first, ForwardIterator last);

template<class ForwardIterator, class BinaryPredicate>
  ForwardIterator adjacent_find(ForwardIterator first, ForwardIterator last,
                              BinaryPredicate pred);
\end{itemdecl}
\end{removedblock}
\begin{addedblock}
\begin{itemdecl}
template<ForwardIterator I, Sentinel<I> S, class Proj = identity,
    IndirectCallableRelation<Projected<I, Proj>> Pred = equal_to<>>
  I
    adjacent_find(I first, S last, Pred pred = Pred{},
                  Proj proj = Proj{});

template<ForwardRange Rng, class Proj = identity,
    IndirectCallableRelation<Projected<IteratorType<Rng>, Proj>> Pred = equal_to<>>
  @\oldtxt{IteratorType}\newtxt{safe_iterator_t}@<Rng>
    adjacent_find(Rng&@\newtxt{\&}@ rng, Pred pred = Pred{}, Proj proj = Proj{});
\end{itemdecl}
\end{addedblock}

\begin{itemdescr}
\pnum
\returns
The first iterator
\tcode{i}
such that both
\tcode{i}
and
\tcode{i + 1}
are in
the range
\range{first}{last}
for which
the following corresponding condition\removed{s} hold\added{s}:
\tcode{\removed{*i == *(i + 1), pred(*i, *(i + 1)) != false}
\added{\textit{INVOKE}(pred, \textit{INVOKE}(proj, *i), \textit{INVOKE}(proj, *(i + 1))) != false}}.
Returns \tcode{last}
if no such iterator is found.

\pnum
\complexity
For a nonempty range, exactly
\tcode{min((i - first) + 1, (last - first) - 1)}
applications of the corresponding predicate, where \tcode{i} is
\tcode{adjacent_find}'s
return value\added{, and no more than twice as many applications of the projection}.
\end{itemdescr}

\rSec2[alg.count]{Count}

\indexlibrary{\idxcode{count}}%
\indexlibrary{\idxcode{count_if}}%
\begin{removedblock}
\begin{itemdecl}
template<class InputIterator, class T>
    typename iterator_traits<InputIterator>::difference_type
       count(InputIterator first, InputIterator last, const T& value);

template<class InputIterator, class Predicate>
    typename iterator_traits<InputIterator>::difference_type
      count_if(InputIterator first, InputIterator last, Predicate pred);
\end{itemdecl}
\end{removedblock}
\begin{addedblock}
\begin{itemdecl}
template<InputIterator I, Sentinel<I> S, class T, class Proj = identity>
  requires IndirectCallableRelation<equal_to<>, Projected<I, Proj>, const T *>@\newtxt{()}@
  DifferenceType<I>
    count(I first, S last, const T& value, Proj proj = Proj{});

template<InputRange Rng, class T, class Proj = identity>
  requires IndirectCallableRelation<equal_to<>, Projected<IteratorType<Rng>, Proj>, const T *>@\newtxt{()}@
  DifferenceType<IteratorType<Rng>>
    count(Rng&& rng, const T& value, Proj proj = Proj{});

template<InputIterator I, Sentinel<I> S, class Proj = identity,
    IndirectCallablePredicate<Projected<I, Proj>> Pred>
  DifferenceType<I>
    count_if(I first, S last, Pred pred, Proj proj = Proj{});

template<InputRange Rng, class Proj = identity,
    IndirectCallablePredicate<Projected<IteratorType<Rng>, Proj>> Pred>
  DifferenceType<IteratorType<Rng>>
    count_if(Rng&& rng, Pred pred, Proj proj = Proj{});
\end{itemdecl}
\end{addedblock}

\begin{itemdescr}
\pnum
\effects
Returns the number of iterators
\tcode{i}
in the range \range{first}{last}
for which the following corresponding
conditions hold:
\tcode{\changed{*i == value, pred(*i) != false}
{\textit{INVOKE}(proj, *i) == value, \textit{INVOKE}(pred, \textit{INVOKE}(proj, *i)) != false}}.

\pnum
\complexity
Exactly
\tcode{last - first}
applications of the corresponding predicate\added{ and projection}.
\end{itemdescr}

\rSec2[mismatch]{Mismatch}

\indexlibrary{\idxcode{mismatch}}%
\begin{removedblock}
\begin{itemdecl}
template<class InputIterator1, class InputIterator2>
  pair<InputIterator1, InputIterator2>
      mismatch(InputIterator1 first1, InputIterator1 last1,
               InputIterator2 first2);

template<class InputIterator1, class InputIterator2,
          class BinaryPredicate>
  pair<InputIterator1, InputIterator2>
      mismatch(InputIterator1 first1, InputIterator1 last1,
               InputIterator2 first2, BinaryPredicate pred);

template<class InputIterator1, class InputIterator2>
  pair<InputIterator1, InputIterator2>
    mismatch(InputIterator1 first1, InputIterator1 last1,
             InputIterator2 first2, InputIterator2 last2);

template<class InputIterator1, class InputIterator2,
           class BinaryPredicate>
  pair<InputIterator1, InputIterator2>
    mismatch(InputIterator1 first1, InputIterator1 last1,
             InputIterator2 first2, InputIterator2 last2,
             BinaryPredicate pred);
\end{itemdecl}
\end{removedblock}
\begin{addedblock}
\begin{itemdecl}
template<InputIterator I1, Sentinel<I1> S1, WeakInputIterator I2,
    class Proj1 = identity, class Proj2 = identity
    IndirectCallablePredicate<Projected<I1, Proj1>, Projected<I2, Proj2>> Pred = equal_to<>>
  tagged_pair<tag::in1(I1), tag::in2(I2)>
    mismatch(I1 first1, S1 last1, I2 first2, Pred pred = Pred{},
             Proj1 proj1 = Proj1{}, Proj2 proj2 = Proj2{});

template<InputRange Rng1, WeakInputIterator I2,
    class Proj1 = identity, class Proj2 = identity
    IndirectCallablePredicate<Projected<IteratorType<Rng1>, Proj1>,
      Projected<I2, Proj2>> Pred = equal_to<>>
  tagged_pair<tag::in1(@\oldtxt{IteratorType}\newtxt{safe_iterator_t}@<Rng1>), tag::in2(I2)>
    mismatch(Rng1&@\newtxt{\&}@ rng1, I2 first2, Pred pred = Pred{},
             Proj1 proj1 = Proj1{}, Proj2 proj2 = Proj2{});

template<InputIterator I1, Sentinel<I1> S1, InputIterator I2, Sentinel<I2> S2,
    class Proj1 = identity, class Proj2 = identity,
    IndirectCallablePredicate<Projected<I1, Proj1>, Projected<I2, Proj2>> Pred = equal_to<>>
  tagged_pair<tag::in1(I1), tag::in2(I2)>
    mismatch(I1 first1, S1 last1, I2 first2, S2 last2, Pred pred = Pred{},
             Proj1 proj1 = Proj1{}, Proj2 proj2 = Proj2{});

template<InputRange Rng1, InputRange Rng2,
    class Proj1 = identity, class Proj2 = identity,
    IndirectCallablePredicate<Projected<IteratorType<Rng1>, Proj1>,
      Projected<IteratorType<Rng2>, Proj2>> Pred = equal_to<>>
  tagged_pair<tag::in1(@\oldtxt{IteratorType}\newtxt{safe_iterator_t}@<Rng1>), tag::in2(@\oldtxt{IteratorType}\newtxt{safe_iterator_t}@<Rng2>)>
    mismatch(Rng1&@\newtxt{\&}@ rng1, Rng2&@\newtxt{\&}@ rng2, Pred pred = Pred{},
             Proj1 proj1 = Proj1{}, Proj2 proj2 = Proj2{});
\end{itemdecl}
\end{addedblock}

\begin{itemdescr}
\pnum
\remarks If \tcode{last2} was not given in the argument list, it denotes
\tcode{first2 + (last1 - first1)} below.

\pnum
\returns
A pair of iterators
\tcode{i}
and
\tcode{j}
such that
\tcode{j == first2 + (i - first1)}
and
\tcode{i}
is the first iterator
in the range \range{first1}{last1}
for which the following corresponding conditions hold:

\begin{itemize}
\item \tcode{j} is in the range \tcode{[first2, last2)}.
\begin{removedblock}
\item \tcode{!(*i == *(first2 + (i - first1)))}
\end{removedblock}
\item \tcode{\changed{pred(*i, *(first2 + (i - first1))) == false}
{\textit{INVOKE}(pred, \textit{INVOKE}(proj1, *i), \textit{INVOKE}(proj2, *(first2 + (i - first1)))) == false}}
\end{itemize}

Returns the pair
\tcode{first1 + min(last1 - first1, last2 - first2)}
and
\tcode{first2 + min(last1 - first1, last2 - first2)}
if such an iterator
\tcode{i}
is not found.

\pnum
\complexity
At most
\tcode{last1 - first1}
applications of the corresponding predicate\added{ and both projections}.
\end{itemdescr}

\rSec2[alg.equal]{Equal}

\indexlibrary{\idxcode{equal}}%
\begin{removedblock}
\begin{itemdecl}
template<class InputIterator1, class InputIterator2>
  bool equal(InputIterator1 first1, InputIterator1 last1,
             InputIterator2 first2);

template<class InputIterator1, class InputIterator2,
          class BinaryPredicate>
  bool equal(InputIterator1 first1, InputIterator1 last1,
             InputIterator2 first2, BinaryPredicate pred);

template<class InputIterator1, class InputIterator2>
  bool equal(InputIterator1 first1, InputIterator1 last1,
             InputIterator2 first2, InputIterator2 last2);

template<class InputIterator1, class InputIterator2,
           class BinaryPredicate>
  bool equal(InputIterator1 first1, InputIterator1 last1,
             InputIterator2 first2, InputIterator2 last2,
             BinaryPredicate pred);
\end{itemdecl}
\end{removedblock}
\begin{addedblock}
\begin{itemdecl}
template<InputIterator I1, Sentinel<I1> S1, WeakInputIterator I2,
    class Pred = equal_to<>, class Proj1 = identity, class Proj2 = identity>
  requires IndirectlyComparable<I1, I2, Pred, Proj1, Proj2>@\newtxt{()}@
  bool equal(I1 first1, S1 last1,
             I2 first2, Pred pred = Pred{},
             Proj1 proj1 = Proj1{}, Proj2 proj2 = Proj2{});

template<InputRange Rng1, WeakInputIterator I2, class Pred = equal_to<>,
    class Proj1 = identity, class Proj2 = identity>
  requires IndirectlyComparable<IteratorType<Rng1>, I2, Pred, Proj1, Proj2>@\newtxt{()}@
  bool equal(Rng1&& rng1, I2 first2, Pred pred = Pred{},
             Proj1 proj1 = Proj1{}, Proj2 proj2 = Proj2{});

template<InputIterator I1, Sentinel<I1> S1, InputIterator I2, Sentinel<I2> S2,
    class Pred = equal_to<>, class Proj1 = identity, class Proj2 = identity>
  requires IndirectlyComparable<I1, I2, Pred, Proj1, Proj2>@\newtxt{()}@
  bool equal(I1 first1, S1 last1, I2 first2, S2 last2,
             Pred pred = Pred{},
             Proj1 proj1 = Proj1{}, Proj2 proj2 = Proj2{});

template<InputRange Rng1, InputRange Rng2, class Pred = equal_to<>,
    class Proj1 = identity, class Proj2 = identity>
  requires IndirectlyComparable<IteratorType<Rng1>, IteratorType<Rng2>, Pred, Proj1, Proj2>@\newtxt{()}@
  bool equal(Rng1&& rng1, Rng2&& rng2, Pred pred = Pred{},
             Proj1 proj1 = Proj1{}, Proj2 proj2 = Proj2{});
\end{itemdecl}
\end{addedblock}

\begin{itemdescr}
\pnum
\remarks If \tcode{last2} was not given in the argument list, it denotes
\tcode{first2 + (last1 - first1)} below.

\pnum
\returns
If
\tcode{last1 - first1 != last2 - first2},
return
\tcode{false}.
Otherwise return
\tcode{true}
if for every iterator
\tcode{i}
in the range \range{first1}{last1}
the following\removed{ corresponding} condition\removed{s} hold\added{s}:
\tcode{\changed{*i == *(first2 + (i - first1)), pred(*i, *(first2 + (i - first1))) != false}
{\textit{INVOKE}(pred, \textit{INVOKE}(proj1, *i), \textit{INVOKE}(proj2, *(first2 + (i - first1)))) != false}}.
Otherwise, returns
\tcode{false}.

\pnum
\complexity
No applications of the corresponding predicate\added{ and projections} if
\changed{
\tcode{InputIterator1}
and
\tcode{InputIterator2}
meet the requirements of random access iterators}{
\oldtxt{\tcode{I1} and \tcode{S1} model} \tcode{SizedIteratorRange\newtxt{<I1,S1>()}} \newtxt{is satisfied}, and
\oldtxt{\tcode{I2} and \tcode{S2} model} \tcode{SizedIteratorRange\newtxt{<I2,S2>()}} \newtxt{is satisfied},
}
and
\tcode{last1 - first1 != last2 - first2}.
Otherwise, at most
\tcode{min(last1 - first1, last2 - first2)}
applications of the corresponding predicate\added{ and projections}.
\end{itemdescr}

\rSec2[alg.is_permutation]{Is permutation}

\indexlibrary{\idxcode{is_permutation}}%
\begin{removedblock}
\begin{itemdecl}
template<class ForwardIterator1, class ForwardIterator2>
  bool is_permutation(ForwardIterator1 first1, ForwardIterator1 last1,
                      ForwardIterator2 first2);
template<class ForwardIterator1, class ForwardIterator2,
                 class BinaryPredicate>
  bool is_permutation(ForwardIterator1 first1, ForwardIterator1 last1,
                      ForwardIterator2 first2, BinaryPredicate pred);
template<class ForwardIterator1, class ForwardIterator2>
  bool is_permutation(ForwardIterator1 first1, ForwardIterator1 last1,
                      ForwardIterator2 first2, ForwardIterator2 last2);
template<class ForwardIterator1, class ForwardIterator2,
                 class BinaryPredicate>
  bool is_permutation(ForwardIterator1 first1, ForwardIterator1 last1,
                      ForwardIterator2 first2, ForwardIterator2 last2,
                      BinaryPredicate pred);
\end{itemdecl}
\end{removedblock}
\begin{addedblock}
\begin{itemdecl}
template<ForwardIterator I1, Sentinel<I1> S1, ForwardIterator I2,
    class Pred = equal_to<>, class Proj1 = identity, class Proj2 = identity>
  requires IndirectlyComparable<I1, I2, Pred, Proj1, Proj2>@\newtxt{()}@
  bool is_permutation(I1 first1, S1 last1, I2 first2,
                      Pred pred = Pred{},
                      Proj1 proj1 = Proj1{}, Proj2 proj2 = Proj2{});

template<ForwardRange Rng1, ForwardIterator I2, class Pred = equal_to<>,
    class Proj1 = identity, class Proj2 = identity>
  requires IndirectlyComparable<IteratorType<Rng1>, I2, Pred, Proj1, Proj2>@\newtxt{()}@
  bool is_permutation(Rng1&& rng1, I2 first2, Pred pred = Pred{},
                      Proj1 proj1 = Proj1{}, Proj2 proj2 = Proj2{});

template<ForwardIterator I1, Sentinel<I1> S1, ForwardIterator I2,
    Sentinel<I2> S2, class Pred = equal_to<>, class Proj1 = identity,
    class Proj2 = identity>
  requires IndirectlyComparable<I1, I2, Pred, Proj1, Proj2>@\newtxt{()}@
  bool is_permutation(I1 first1, S1 last1, I2 first2, S2 last2,
                      Pred pred = Pred{},
                      Proj1 proj1 = Proj1{}, Proj2 proj2 = Proj2{});

template<ForwardRange Rng1, ForwardRange Rng2, class Pred = equal_to<>,
    class Proj1 = identity, class Proj2 = identity>
  requires IndirectlyComparable<IteratorType<Rng1>, IteratorType<Rng2>, Pred, Proj1, Proj2>@\newtxt{()}@
  bool is_permutation(Rng1&& rng1, Rng2&& rng2, Pred pred = Pred{},
                      Proj1 proj1 = Proj1{}, Proj2 proj2 = Proj2{});
\end{itemdecl}
\end{addedblock}

\begin{itemdescr}
\begin{removedblock}
\pnum
\requires \tcode{ForwardIterator1} and \tcode{ForwardIterator2} shall have the same
value type. The comparison function shall be an equivalence relation.
\end{removedblock}

\pnum
\remarks If \tcode{last2} was not given in the argument list, it denotes
\tcode{first2 + (last1 - first1)} below.

\pnum
\returns If \tcode{last1 - first1 != last2 - first2}, return \tcode{false}.
Otherwise return \tcode{true} if there exists a permutation of the elements in the
range \range{first2}{first2 + (last1 - first1)}, beginning with
\tcode{\changed{ForwardIterator2}{I2} begin}, such that
\tcode{equal(first1, last1, begin\added{, pred, proj1, proj2})} returns \tcode{true}
\removed{ or
\tcode{equal(first1, last1, begin, pred)} returns \tcode{true}}; otherwise, returns
\tcode{false}.

\pnum
\complexity
No applications of the corresponding predicate\added{ and projections} if
\changed{
\tcode{ForwardIterator1}
and
\tcode{ForwardIter\-ator2}
meet the requirements of random access iterators}{
\oldtxt{\tcode{I1} and \tcode{S1} model} \tcode{SizedIterator\-Range\newtxt{<I1,S1>()}} \newtxt{is satisfied}, and
\oldtxt{\tcode{I2} and \tcode{S2} model} \tcode{SizedIteratorRange\newtxt{<I2,S2>()}} \newtxt{is satisfied},
} and \tcode{last1 - first1 != last2 - first2}.
Otherwise, exactly \tcode{distance(first1, last1)} applications of the
corresponding predicate\added{ and projections} if
\tcode{equal(\brk{}first1, last1, first2, last2\added{, pred, proj1, proj2})}
would return \tcode{true}\removed{ if \tcode{pred} was not given in the argument list
or \tcode{equal(first1, last1, first2, last2, pred)} would return \tcode{true} if pred
was given in the argument list}; otherwise, at
worst \bigoh{N^2}, where $N$ has the value \tcode{distance(first1, last1)}.
\end{itemdescr}

\rSec2[alg.search]{Search}

\indexlibrary{\idxcode{search}}%
\begin{removedblock}
\begin{itemdecl}
template<class ForwardIterator1, class ForwardIterator2>
  ForwardIterator1
    search(ForwardIterator1 first1, ForwardIterator1 last1,
           ForwardIterator2 first2, ForwardIterator2 last2);

template<class ForwardIterator1, class ForwardIterator2,
         class BinaryPredicate>
  ForwardIterator1
    search(ForwardIterator1 first1, ForwardIterator1 last1,
           ForwardIterator2 first2, ForwardIterator2 last2,
           BinaryPredicate pred);
\end{itemdecl}
\end{removedblock}
\begin{addedblock}
\begin{itemdecl}
template<ForwardIterator I1, Sentinel<I1> S1, ForwardIterator I2,
    Sentinel<I2> S2, class Pred = equal_to<>,
    class Proj1 = identity, class Proj2 = identity>
  requires IndirectlyComparable<I1, I2, Pred, Proj1, Proj2>@\newtxt{()}@
  I1
    search(I1 first1, S1 last1, I2 first2, S2 last2,
           Pred pred = Pred{},
           Proj1 proj1 = Proj1{}, Proj2 proj2 = Proj2{});

template<ForwardRange Rng1, ForwardRange Rng2, class Pred = equal_to<>,
    class Proj1 = identity, class Proj2 = identity>
  requires IndirectlyComparable<IteratorType<Rng1>, IteratorType<Rng2>, Pred, Proj1, Proj2>@\newtxt{()}@
  @\oldtxt{IteratorType}\newtxt{safe_iterator_t}@<Rng1>
    search(Rng1&@\newtxt{\&}@ rng1, Rng2&& rng2, Pred pred = Pred{},
           Proj1 proj1 = Proj1{}, Proj2 proj2 = Proj2{});
\end{itemdecl}
\end{addedblock}

\begin{itemdescr}
\pnum
\effects
Finds a subsequence of equal values in a sequence.

\pnum
\returns
The first iterator
\tcode{i}
in the range \range{first1}{last1 - (last2-first2)}
such that for every non-negative integer
\tcode{n}
less than
\tcode{last2 - first2}
the following\removed{ corresponding} condition\removed{s} hold\added{s}:
\tcode{\changed{*(i + n) == *(first2 + n), pred(*(i + n), *(first2 + n)) != false}
{\textit{INVOKE}(pred, \textit{INVOKE}(proj1, *(i + n)), \textit{INVOKE}(proj2, *(first2 + n))) != false}}.
Returns \tcode{first1}
if \range{first2}{last2} is empty,
otherwise returns \tcode{last1}
if no such iterator is found.

\pnum
\complexity
At most
\tcode{(last1 - first1) * (last2 - first2)}
applications of the corresponding predicate\added{ and projections}.
\end{itemdescr}

\indexlibrary{\idxcode{search_n}}%
\begin{removedblock}
\begin{itemdecl}
template<class ForwardIterator, class Size, class T>
  ForwardIterator
    search_n(ForwardIterator first, ForwardIterator last, Size count,
           const T& value);

template<class ForwardIterator, class Size, class T,
         class BinaryPredicate>
  ForwardIterator
    search_n(ForwardIterator first, ForwardIterator last, Size count,
           const T& value, BinaryPredicate pred);
\end{itemdecl}
\end{removedblock}
\begin{addedblock}
\begin{itemdecl}
template<ForwardIterator I, Sentinel<I> S, class T,
    class Pred = equal_to<>, class Proj = identity>
  requires IndirectlyComparable<I1, const T*, Pred, Proj>@\newtxt{()}@
  I
    search_n(I first, S last, DifferenceType<I> count,
             const T& value, Pred pred = Pred{},
             Proj proj = Proj{});

template<ForwardRange Rng, class T, class Pred = equal_to<>,
    class Proj = identity>
  requires IndirectlyComparable<IteratorType<Rng1>, const T*, Pred, Proj>@\newtxt{()}@
  @\oldtxt{IteratorType}\newtxt{safe_iterator_t}@<Rng>
    search_n(Rng&@\newtxt{\&}@ rng, DifferenceType<IteratorType<Rng>> count,
             const T& value, Pred pred = Pred{}, Proj proj = Proj{});
\end{itemdecl}
\end{addedblock}

\begin{itemdescr}
\begin{removedblock}
\pnum
\requires
The type
\tcode{Size}
shall be convertible to integral type~(\cxxref{conv.integral}, \cxxref{class.conv}).
\end{removedblock}

\pnum
\effects
Finds a subsequence of equal values in a sequence.

\pnum
\returns
The first iterator
\tcode{i}
in the range \range{first}{last-count}
such that for every non-negative integer
\tcode{n}
less than
\tcode{count}
the following\removed{ corresponding} condition\removed{s} hold\added{s}:
\tcode{\changed{*(i + n) == value, pred(*(i + n),value) != false}
{\textit{INVOKE}(pred, \textit{INVOKE}(proj, *(i + n)), value) != false}}.
Returns \tcode{last}
if no such iterator is found.

\pnum
\complexity
At most
\tcode{last - first}
applications of the corresponding predicate\added{ and projection}.
\end{itemdescr}

\rSec1[alg.modifying.operations]{Mutating sequence operations}

\rSec2[alg.copy]{Copy}

\indexlibrary{\idxcode{copy}}%
\begin{removedblock}
\begin{itemdecl}
template<class InputIterator, class OutputIterator>
  OutputIterator copy(InputIterator first, InputIterator last,
                      OutputIterator result);
\end{itemdecl}
\end{removedblock}
\begin{addedblock}
\begin{itemdecl}
template<InputIterator I, Sentinel<I> S, WeaklyIncrementable O>
  requires IndirectlyCopyable<I, O>@\newtxt{()}@
  tagged_pair<tag::in(I), tag::out(O)>
    copy(I first, S last, O result);

template<InputRange Rng, WeaklyIncrementable O>
  requires IndirectlyCopyable<IteratorType<Rng>, O>@\newtxt{()}@
  tagged_pair<tag::in(@\oldtxt{IteratorType}\newtxt{safe_iterator_t}@<Rng>), tag::out(O)>
    copy(Rng&@\newtxt{\&}@ rng, O result);
\end{itemdecl}
\end{addedblock}

\begin{itemdescr}
\pnum
\effects Copies elements in the range \range{first}{last} into the range
\range{result}{result + (last - first)} starting from \tcode{first} and
proceeding to \tcode{last}. For each non-negative integer
\tcode{n < (last - first)}, performs \tcode{*(result + n) = *(first + n)}.

\pnum
\returns \tcode{\changed{result + (last - first)}{\{last, result + (last - first)\}}}.

\pnum
\requires \tcode{result} shall not be in the range \range{first}{last}.

\pnum
\complexity Exactly \tcode{last - first} assignments.
\end{itemdescr}

\indexlibrary{\idxcode{copy_n}}%
\begin{removedblock}
\begin{itemdecl}
template<class InputIterator, class Size, class OutputIterator>
  OutputIterator copy_n(InputIterator first, Size n,
                        OutputIterator result);
\end{itemdecl}
\end{removedblock}
\begin{addedblock}
\begin{itemdecl}
template<WeakInputIterator I, WeaklyIncrementable O>
  requires IndirectlyCopyable<I, O>@\newtxt{()}@
  tagged_pair<tag::in(I), tag::out(O)>
    copy_n(I first, @\oldtxt{iterator_distance_t}\newtxt{DifferenceType}@<I> n, O result);
\end{itemdecl}
\end{addedblock}

\begin{itemdescr}
\pnum
\effects For each non-negative integer
$i < n$, performs \tcode{*(result + i) = *(first + i)}.

\pnum
\returns \tcode{\changed{result + n}{\{first + n, result + n\}}}.

\pnum
\complexity Exactly \tcode{n} assignments.
\end{itemdescr}

\indexlibrary{\idxcode{copy_n}}%
\begin{removedblock}
\begin{itemdecl}
template<class InputIterator, class OutputIterator, class Predicate>
  OutputIterator copy_if(InputIterator first, InputIterator last,
                         OutputIterator result, Predicate pred);
\end{itemdecl}
\end{removedblock}
\begin{addedblock}
\begin{itemdecl}
template<InputIterator I, Sentinel<I> S, WeaklyIncrementable O, class Proj = identity,
    IndirectCallablePredicate<Projected<I, Proj>> Pred>
  requires IndirectlyCopyable<I, O>@\newtxt{()}@
  tagged_pair<tag::in(I), tag::out(O)>
    copy_if(I first, S last, O result, Pred pred, Proj proj = Proj{});

template<InputRange Rng, WeaklyIncrementable O, class Proj = identity,
    IndirectCallablePredicate<Projected<IteratorType<Rng>, Proj>> Pred>
  requires IndirectlyCopyable<IteratorType<Rng>, O>@\newtxt{()}@
  tagged_pair<tag::in(@\oldtxt{IteratorType}\newtxt{safe_iterator_t}@<Rng>), tag::out(O)>
    copy_if(Rng&@\newtxt{\&}@ rng, O result, Pred pred, Proj proj = Proj{});
\end{itemdecl}
\end{addedblock}

\begin{itemdescr}
\pnum
\requires The ranges \range{first}{last} and \range{result}{result + (last - first)} shall not overlap.

\pnum
\effects Copies all of the elements referred to by the iterator \tcode{i} in the range \range{first}{last}
for which \tcode{\changed{pred(*i)}{\textit{INVOKE}(pred, \textit{INVOKE}(proj, *i))}} is \tcode{true}.

\pnum
\returns \changed{The end of the resulting range}{\tcode{\{last, result + (last - first)\}}}.

\pnum
\complexity Exactly \tcode{last - first} applications of the corresponding predicate.

\pnum
\remarks Stable~(\cxxref{algorithm.stable}).
\end{itemdescr}


\indexlibrary{\idxcode{copy_backward}}%
\begin{removedblock}
\begin{itemdecl}
template<class BidirectionalIterator1, class BidirectionalIterator2>
  BidirectionalIterator2
    copy_backward(BidirectionalIterator1 first,
                  BidirectionalIterator1 last,
                  BidirectionalIterator2 result);
\end{itemdecl}
\end{removedblock}
\begin{addedblock}
\begin{itemdecl}
template<BidirectionalIterator I1, Sentinel<I1> S1, BidirectionalIterator I2>
  requires IndirectlyCopyable<I1, I2>@\newtxt{()}@
  tagged_pair<tag::in@\oldtxt{1}@(I1), tag::@\oldtxt{in2}\newtxt{out}@(I2)>
    copy_backward(I1 first, I1 last, I2 result);

template<BidirectionalRange Rng, BidirectionalIterator I>
  requires IndirectlyCopyable<IteratorType<Rng>, I>@\newtxt{()}@
  tagged_pair<tag::in@\oldtxt{1}@(@\oldtxt{IteratorType}\newtxt{safe_iterator_t}@<Rng>), tag::@\oldtxt{in2}\newtxt{out}@(I)>
    copy_backward(Rng&@\newtxt{\&}@ rng, I result);
\end{itemdecl}
\end{addedblock}

\begin{itemdescr}
\pnum
\effects
Copies elements in the range \range{first}{last}
into the
range \range{result - (last-first)}{result}
starting from
\tcode{last - 1}
and proceeding to \tcode{first}.\footnote{\tcode{copy_backward}
should be used instead of copy when \tcode{last}
is in
the range
\range{result - (last - first)}{result}.}
For each positive integer
\tcode{n <= (last - first)},
performs
\tcode{*(result - n) = *(last - n)}.

\pnum
\requires
\tcode{result}
shall not be in the range
\brange{first}{last}.

\pnum
\returns
\tcode{\changed{result - (last - first)}{\{last, result - (last - first)\}}}.

\pnum
\complexity
Exactly
\tcode{last - first}
assignments.
\end{itemdescr}

\rSec2[alg.move]{Move}

\indexlibrary{move\tcode{move}}%
\begin{removedblock}
\begin{itemdecl}
template<class InputIterator, class OutputIterator>
  OutputIterator move(InputIterator first, InputIterator last,
                      OutputIterator result);
\end{itemdecl}
\end{removedblock}
\begin{addedblock}
\begin{itemdecl}
template<InputIterator I, Sentinel<I> S, WeaklyIncrementable O>
  requires IndirectlyMovable<I, O>@\newtxt{()}@
  tagged_pair<tag::in(I), tag::out(O)>
    move(I first, S last, O result);

template<InputRange Rng, WeaklyIncrementable O>
  requires IndirectlyMovable<IteratorType<Rng>, O>@\newtxt{()}@
  tagged_pair<tag::in(@\oldtxt{IteratorType}\newtxt{safe_iterator_t}@<Rng>), tag::out(O)>
    move(Rng&@\newtxt{\&}@ rng, O result);
\end{itemdecl}
\end{addedblock}

\begin{itemdescr}
\pnum
\effects
Moves elements in the range \range{first}{last}
into the range \range{result}{result + (last - first)}
starting from first and proceeding to last.
For each non-negative integer
\tcode{n < (last-first)},
performs
\tcode{*(result + n)} \tcode{= std::move(*(first + n))}.

\pnum
\returns
\tcode{\changed{result + (last - first)}{\{last, result + (last - first)\}}}.

\pnum
\requires
\tcode{result}
shall not be in the range
\range{first}{last}.

\pnum
\complexity
Exactly
\tcode{last - first}
move assignments.
\end{itemdescr}

\indexlibrary{\idxcode{move_backward}}%
\begin{removedblock}
\begin{itemdecl}
template<class BidirectionalIterator1, class BidirectionalIterator2>
  BidirectionalIterator2
    move_backward(BidirectionalIterator1 first,
                  BidirectionalIterator1 last,
                  BidirectionalIterator2 result);
\end{itemdecl}
\end{removedblock}
\begin{addedblock}
\begin{itemdecl}
template<BidirectionalIterator I1, Sentinel<I1> S1, BidirectionalIterator I2>
  requires IndirectlyMovable<I1, I2>@\newtxt{()}@
  tagged_pair<tag::in@\oldtxt{1}@(I1), tag::@\oldtxt{in2}\newtxt{out}@(I2)>
    move_backward(I1 first, I1 last, I2 result);

template<BidirectionalRange Rng, BidirectionalIterator I>
  requires IndirectlyMovable<IteratorType<Rng>, I>@\newtxt{()}@
  tagged_pair<tag::in@\oldtxt{1}@(@\oldtxt{IteratorType}\newtxt{safe_iterator_t}@<Rng>), tag::@\oldtxt{in2}\newtxt{out}@(I)>
    move_backward(Rng&@\newtxt{\&}@ rng, I result);
\end{itemdecl}
\end{addedblock}

\begin{itemdescr}
\pnum
\effects
Moves elements in the range \range{first}{last}
into the
range \range{result - (last-first)}{result}
starting from
\tcode{last - 1}
and proceeding to first.\footnote{\tcode{move_backward}
should be used instead of move when last
is in
the range
\range{result - (last - first)}{result}.}
For each positive integer
\tcode{n <= (last - first)},
performs
\tcode{*(result - n) = std::move(*(last - n))}.

\pnum
\requires
\tcode{result}
shall not be in the range
\brange{first}{last}.

\pnum
\returns
\tcode{\changed{result - (last - first)}{\{last, result - (last - first)\}}}.

\pnum
\complexity
Exactly
\tcode{last - first}
assignments.
\end{itemdescr}

\rSec2[alg.swap]{swap}

\indexlibrary{\idxcode{swap_ranges}}%
\begin{removedblock}
\begin{itemdecl}
template<class ForwardIterator1, class ForwardIterator2>
  ForwardIterator2
    swap_ranges(ForwardIterator1 first1, ForwardIterator1 last1,
                ForwardIterator2 first2);
\end{itemdecl}
\end{removedblock}
\begin{addedblock}
\begin{itemdecl}
template<ForwardIterator I1, Sentinel<I1> S1, ForwardIterator I2>
  requires IndirectlySwappable<I1, I2>@\newtxt{()}@
  tagged_pair<tag::in1(I1), tag::in2(I2)>
    swap_ranges(I1 first1, S1 last1, I2 first2);

template<ForwardRange Rng, ForwardIterator I>
  requires IndirectlySwappable<IteratorType<Rng>, I>@\newtxt{()}@
  tagged_pair<tag::in1(@\oldtxt{IteratorType}\newtxt{safe_iterator_t}@<Rng>), tag::in2(I)>
    swap_ranges(Rng&@\newtxt{\&}@ rng1, I first2);

template<ForwardIterator I1, Sentinel<I1> S1, ForwardIterator I2, Sentinel<I2> S2>
  requires IndirectlySwappable<I1, I2>@\newtxt{()}@
  tagged_pair<tag::in1(I1), tag::in2(I2)>
    swap_ranges(I1 first1, S1 last1, I2 first2, S2 last2);

template<ForwardRange Rng1, ForwardRange Rng2>
  requires IndirectlySwappable<IteratorType<Rng1>, IteratorType<Rng2>>@\newtxt{()}@
  tagged_pair<tag::in1(@\oldtxt{IteratorType}\newtxt{safe_iterator_t}@<Rng1>), tag::in2(@\oldtxt{IteratorType}\newtxt{safe_iterator_t}@<Rng2>)>
    swap_ranges(Rng1&@\newtxt{\&}@ rng1, Rng2&@\newtxt{\&}@ rng2);
\end{itemdecl}
\end{addedblock}

\begin{itemdescr}
\pnum
\effects
\added{For the first two overloads, let \tcode{last2} be \tcode{first2 + (last1 - first1)}. }
For each non-negative integer \tcode{\removed{n < (last1 - first1)}}
\tcode{\added{n < min(last1 - first1, last2 - first2)}}
performs:
\tcode{swap(*(first1 + n), \brk{}*(first2 + n))}.

\pnum
\requires
The two ranges \range{first1}{last1}
and
\removed{\range{first2}{first2 + (last1 - first1)}}\brk{}\added{\range{first2}{last2}}
shall not overlap.
\tcode{*(first1 + n)} shall be swappable with~(\ref{concepts.lib.corelang.swappable})
\tcode{*(first2 + n)}.

\pnum
\returns
\tcode{\changed{first2 + (last1 - first1)}{\{first1 + n, first2 + n\}}}\added{, where
\tcode{n} is }\tcode{\added{min(last1 - first1, last2 - first2)}}.

\pnum
\complexity
Exactly
\tcode{\changed{last1 - first1}{min(last1 - first1, last2 - first2)}}
swaps.
\end{itemdescr}

\begin{removedblock}
\indexlibrary{\idxcode{iter_swap}}%
\begin{itemdecl}
template<class ForwardIterator1, class ForwardIterator2>
  void iter_swap(ForwardIterator1 a, ForwardIterator2 b);
\end{itemdecl}


\begin{itemdescr}
\pnum
\effects
\tcode{swap(*a, *b)}.

\pnum
\requires
\tcode{a} and \tcode{b} shall be dereferenceable. \tcode{*a} shall be
swappable with~(\ref{concepts.lib.corelang.swappable}) \tcode{*b}.
\end{itemdescr}
\end{removedblock}

\rSec2[alg.transform]{Transform}

\indexlibrary{\idxcode{transform}}%
\begin{removedblock}
\begin{itemdecl}
template<class InputIterator, class OutputIterator,
         class UnaryOperation>
  OutputIterator
    transform(InputIterator first, InputIterator last,
              OutputIterator result, UnaryOperation op);

template<class InputIterator1, class InputIterator2,
         class OutputIterator, class BinaryOperation>
  OutputIterator
    transform(InputIterator1 first1, InputIterator1 last1,
              InputIterator2 first2, OutputIterator result,
              BinaryOperation binary_op);
\end{itemdecl}
\end{removedblock}
\begin{addedblock}
\begin{itemdecl}
template<InputIterator I, Sentinel<I> S, class Proj = identity,
    IndirectCallable<Projected<I, Proj>> F,
    WeakOutputIterator<IndirectCallableResultType<F, Projected<I, Proj>>> O>
  tagged_pair<tag::in(I), tag::out(O)>
    transform(I first, S last, O result, F op, Proj proj = Proj{});

template<InputRange Rng, class Proj = identity,
    IndirectCallable<Projected<IteratorType<Rng>, Proj>> F,
    WeakOutputIterator<IndirectCallableResultType<F,
      Projected<IteratorType<Rng>, Proj>>> O>
  tagged_pair<tag::in(@\oldtxt{IteratorType}\newtxt{safe_iterator_t}@<Rng>), tag::out(O)>
    transform(Rng&@\newtxt{\&}@ rng, O result, F op, Proj proj = Proj{});

template<InputIterator I1, Sentinel<I1> S1, WeakInputIterator I2,
    class Proj1 = identity, class Proj2 = identity,
    IndirectCallable<Projected<I1, Proj1>, Projected<I2, Proj2>> F,
    WeakOutputIterator<IndirectCallableResultType<F, Projected<I1, Proj1>,
      Projected<I2, Proj2>>> O>
  tagged_tuple<tag::in1(I1), tag::in2(I2), tag::out(O)>
    transform(I1 first1, S1 last1, I2 first2, O result,
              F binary_op, Proj1 proj1 = Proj1{}, Proj2 proj2 = Proj2{});

template<InputRange Rng, WeakInputIterator I,
    class Proj1 = identity, class Proj2 = identity,
    IndirectCallable<Projected<IteratorType<Rng>, Proj1>, Projected<I, Proj2>> F,
    WeakOutputIterator<IndirectCallableResultType<F,
      Projected<IteratorType<Rng>, Proj1>, Projected<I, Proj2>>> O>
  tagged_tuple<tag::in1(@\oldtxt{IteratorType}\newtxt{safe_iterator_t}@<Rng>), tag::in2(I), tag::out(O)>
    transform(Rng&@\newtxt{\&}@ rng1, I first2, O result,
              F binary_op, Proj1 proj1 = Proj1{}, Proj2 proj2 = Proj2{});

template<InputIterator I1, Sentinel<I1> S1, InputIterator I2, Sentinel<I2> S2,
    class Proj1 = identity, class Proj2 = identity,
    IndirectCallable<Projected<I1, Proj1>, Projected<I2, Proj2>> F,
    WeakOutputIterator<IndirectCallableResultType<F, Projected<I1, Proj1>,
      Projected<I2, Proj2>>> O>
  tagged_tuple<tag::in1(I1), tag::in2(I2), tag::out(O)>
    transform(I1 first1, S1 last1, I2 first2, S2 last2, O result,
              F binary_op, Proj1 proj1 = Proj1{}, Proj2 proj2 = Proj2{});

template<InputRange Rng1, InputRange Rng2,
    class Proj1 = identity, class Proj2 = identity,
    IndirectCallable<Projected<IteratorType<Rng1>, Proj1>,
      Projected<IteratorType<Rng2>, Proj2>> F,
    WeakOutputIterator<IndirectCallableResultType<F,
      Projected<IteratorType<Rng1>, Proj1>, Projected<IteratorType<Rng2>, Proj2>>> O>
  tagged_tuple<tag::in1(@\oldtxt{IteratorType}\newtxt{safe_iterator_t}@<Rng1>),
               tag::in2(@\oldtxt{IteratorType}\newtxt{safe_iterator_t}@<Rng2>),
               tag::out(O)>
    transform(Rng1&@\newtxt{\&}@ rng1, Rng2&@\newtxt{\&}@ rng2, O result,
              F binary_op, Proj1 proj1 = Proj1{}, Proj2 proj2 = Proj2{});
\end{itemdecl}
\end{addedblock}

\begin{itemdescr}
\begin{addedblock}
\pnum
For binary transforms that do not take \tcode{last2}, let \tcode{last2}
be \tcode{first2 + (last1 - first1)}. Let $N$ be \tcode{(last1 - first1)}
for unary transforms, or \tcode{min(last1 - first1, last2 - first2) for binary
transforms.}
\end{addedblock}

\pnum
\effects
Assigns through every iterator
\tcode{i}
in the range
\range{result}{result + \changed{(last1 - first1)}{$N$}}
a new
corresponding value equal to
\tcode{\changed{op(*(first1 + (i - result)))}{\textit{INVOKE}(op, \textit{INVOKE}(proj, *(first1 + (i - result))))}}
or
\tcode{\changed{binary_op(*(first1 + (i - result)), *(first2 + (i - result)))}
{\textit{INVOKE}(\brk{}binary_op, \textit{INVOKE}(proj1, *(first1 + (i - result))), \textit{INVOKE}(proj2, *(first2 + (i - result))))}}.

\pnum
\requires
\tcode{op} and \tcode{binary_op}
shall not invalidate iterators or subranges, or modify elements in the ranges
\crange{first1}{\changed{last1}{first1 + $N$}},
\crange{first2}{first2 + \changed{(last1 - first1)}{$N$}},
and
\crange{result}
{result + \changed{(last1 - first1)}{$N$}}.\footnote{The use of fully
closed ranges is intentional.}

\pnum
\returns
\tcode{\changed{result + (last1 - first1)}{\{first1 + $N$, result + $N$\}}}
\added{ or }\tcode{\added{make_tagged_tuple}}\brk\tcode{\added{<tag::in1, tag::in2, tag::out>}}\tcode{\added{(first1 + $N$, first2 + $N$, result + $N$)}}.

\pnum
\complexity
Exactly
\tcode{\changed{last1 - first1}{$N$}}
applications of
\tcode{op} or \tcode{binary_op}.

\pnum
\notes
\tcode{result} may be equal to \tcode{first\added{1}}
in case of unary transform,
or to \tcode{first1} or \tcode{first2}
in case of binary transform.
\end{itemdescr}

\rSec2[alg.replace]{Replace}

\indexlibrary{\idxcode{replace}}%
\indexlibrary{\idxcode{replace_if}}%
\begin{removedblock}
\begin{itemdecl}
template<class ForwardIterator, class T>
  void replace(ForwardIterator first, ForwardIterator last,
               const T& old_value, const T& new_value);

template<class ForwardIterator, class Predicate, class T>
  void replace_if(ForwardIterator first, ForwardIterator last,
                  Predicate pred, const T& new_value);
\end{itemdecl}
\end{removedblock}
\begin{addedblock}
\begin{itemdecl}
template<ForwardIterator I, Sentinel<I> S, class T1, Semiregular T2, class Proj = identity>
  requires Writable<I, T2>@\newtxt{()}@ &&
    IndirectCallableRelation<equal_to<>, Projected<I, Proj>, const T1 *>@\newtxt{()}@
  I
    replace(I first, S last, const T1& old_value, const T2& new_value, Proj proj = Proj{});

template<ForwardRange Rng, class T1, Semiregular T2, class Proj = identity>
  requires Writable<IteratorType<Rng>, T2>@\newtxt{()}@ &&
    IndirectCallableRelation<equal_to<>, Projected<IteratorType<Rng>, Proj>, const T1 *>@\newtxt{()}@
  @\oldtxt{IteratorType}\newtxt{safe_iterator_t}@<Rng>
    replace(Rng&@\newtxt{\&}@ rng, const T1& old_value, const T2& new_value);

template<ForwardIterator I, Sentinel<I> S, Semiregular T, class Proj = identity,
    IndirectCallablePredicate<Projected<I, Proj>> Pred>
  requires Writable<I, T>@\newtxt{()}@
  I
    replace_if(I first, S last, Pred pred, const T& new_value, Proj proj = Proj{});

template<ForwardRange Rng, Semiregular T, class Proj = identity,
    IndirectCallablePredicate<Projected<IteratorType<Rng>, Proj>> Pred>
  requires Writable<IteratorType<Rng>, T>@\newtxt{()}@
  @\oldtxt{IteratorType}\newtxt{safe_iterator_t}@<Rng>
    replace_if(Rng&@\newtxt{\&}@ rng, Pred pred, const T& new_value, Proj proj = Proj{});
\end{itemdecl}
\end{addedblock}

\begin{itemdescr}
\begin{removedblock}
\pnum
\requires
The expression
\tcode{*first = new_value}
shall be valid.
\end{removedblock}

\pnum
\effects
Substitutes elements referred by the iterator
\tcode{i}
in the range \range{first}{last}
with \tcode{new_value},
when the following corresponding conditions hold:
\tcode{\changed{*i == old_value}{\textit{INVOKE}(proj, *i) == old_value}},
\tcode{\changed{pred(*i) != false}{\textit{INVOKE}(pred, \textit{INVOKE}(proj, *i)) != false}}.

\begin{addedblock}
\pnum
\returns
\tcode{last}.
\end{addedblock}

\pnum
\complexity
Exactly
\tcode{last - first}
applications of the corresponding predicate\added{ and projection}.
\end{itemdescr}

\indexlibrary{\idxcode{replace_copy}}%
\indexlibrary{\idxcode{replace_copy_if}}%
\begin{removedblock}
\begin{itemdecl}
template<class InputIterator, class OutputIterator, class T>
  OutputIterator
    replace_copy(InputIterator first, InputIterator last,
                 OutputIterator result,
                 const T& old_value, const T& new_value);

template<class InputIterator, class OutputIterator, class Predicate, class T>
  OutputIterator
    replace_copy_if(InputIterator first, InputIterator last,
                    OutputIterator result,
                    Predicate pred, const T& new_value);
\end{itemdecl}
\end{removedblock}
\begin{addedblock}
\begin{itemdecl}
template<InputIterator I, Sentinel<I> S, class T1, Semiregular T2, WeakOutputIterator<T2> O,
    class Proj = identity>
  requires IndirectlyCopyable<I, O>@\newtxt{()}@ &&
    IndirectCallableRelation<equal_to<>, Projected<I, Proj>, const T1 *>@\newtxt{()}@
  tagged_pair<tag::in(I), tag::out(O)>
    replace_copy(I first, S last, O result, const T1& old_value, const T2& new_value,
                 Proj proj = Proj{});

template<InputRange Rng, class T1, Semiregular T2, WeakOutputIterator<T2> O,
    class Proj = identity>
  requires IndirectlyCopyable<IteratorType<Rng>, O>@\newtxt{()}@ &&
    IndirectCallableRelation<equal_to<>, Projected<IteratorType<Rng>, Proj>, const T1 *>@\newtxt{()}@
  tagged_pair<tag::in(@\oldtxt{IteratorType}\newtxt{safe_iterator_t}@<Rng>), tag::out(O)>
    replace_copy(Rng&@\newtxt{\&}@ rng, O result, const T1& old_value, const T2& new_value,
                 Proj proj = Proj{});

template<InputIterator I, Sentinel<I> S, Semiregular T, WeakOutputIterator<T> O,
    class Proj = identity, IndirectCallablePredicate<Projected<I, Proj>> Pred>
  requires IndirectlyCopyable<I, O>@\newtxt{()}@
  tagged_pair<tag::in(I), tag::out(O)>
    replace_copy_if(I first, S last, O result, Pred pred, const T& new_value,
                    Proj proj = Proj{});

template<InputRange Rng, Semiregular T, WeakOutputIterator<T> O, class Proj = identity,
    IndirectCallablePredicate<Projected<IteratorType<Rng>, Proj>> Pred>
  requires IndirectlyCopyable<IteratorType<Rng>, O>@\newtxt{()}@
  tagged_pair<tag::in(@\oldtxt{IteratorType}\newtxt{safe_iterator_t}@<Rng>), tag::out(O)>
    replace_copy_if(Rng&@\newtxt{\&}@ rng, O result, Pred pred, const T& new_value,
                    Proj proj = Proj{});
\end{itemdecl}
\end{addedblock}

\begin{itemdescr}
\pnum
\requires
\removed{The results of the expressions
\tcode{*first}
and
\tcode{new_value}
shall be writable to the
\tcode{result}
output iterator.}
The ranges
\range{first}{last}
and
\range{result}{result + (last - first)}
shall not overlap.

\pnum
\effects
Assigns to every iterator
\tcode{i}
in the
range
\range{result}{result + (last - first)}
either
\tcode{new_value}
or
\tcode{*\brk(first + (i - result))}
depending on whether the following corresponding conditions hold:

\begin{removedblock}
\begin{codeblock}
*(first + (i - result)) == old_value
pred(*(first + (i - result))) != false
\end{codeblock}
\end{removedblock}
\begin{addedblock}
\begin{codeblock}
@\textit{INVOKE}@(proj, *(first + (i - result))) == old_value
@\textit{INVOKE}@(pred, @\textit{INVOKE}@(proj, *(first + (i - result)))) != false
\end{codeblock}
\end{addedblock}

\pnum
\returns
\tcode{\{last, result + (last - first)\}}.

\pnum
\complexity
Exactly
\tcode{last - first}
applications of the corresponding predicate\added{ and projection}.
\end{itemdescr}

\rSec2[alg.fill]{Fill}

\indexlibrary{\idxcode{fill}}%
\indexlibrary{\idxcode{fill_n}}%
\begin{removedblock}
\begin{itemdecl}
template<class ForwardIterator, class T>
  void fill(ForwardIterator first, ForwardIterator last, const T& value);

template<class OutputIterator, class Size, class T>
  OutputIterator fill_n(OutputIterator first, Size n, const T& value);
\end{itemdecl}
\end{removedblock}
\begin{addedblock}
\begin{itemdecl}
template<Semiregular T, OutputIterator<T> O, Sentinel<O> S>
  O fill(O first, S last, const T& value);

template<Semiregular T, OutputRange<T> Rng>
  @\oldtxt{IteratorType}\newtxt{safe_iterator_t}@<Rng>
    fill(Rng&@\newtxt{\&}@ rng, const T& value);

template<Semiregular T, WeakOutputIterator<T> O>
  O fill_n(O first, DifferenceType<O> n, const T& value);
\end{itemdecl}
\end{addedblock}

\begin{itemdescr}
\begin{removedblock}
\pnum
\requires
The expression
\tcode{value}
shall be writable to the output iterator. The type
\tcode{Size}
shall be convertible to an integral type~(\cxxref{conv.integral}, \cxxref{class.conv}).
\end{removedblock}

\pnum
\effects
\changed{The first algorithm}{\tcode{fill}} assigns \tcode{value} through all the
iterators in the range \range{first}{last}. \changed{The second algorithm}{\tcode{fill_n}}
assigns \tcode{value} through all the iterators in the range \range{first}{first + n}
if \tcode{n} is positive, otherwise it does nothing.

\pnum
\returns \added{\tcode{fill} returns \tcode{last}. }\tcode{fill_n} returns \tcode{first + n}
for non-negative values of \tcode{n} and \tcode{first} for negative values.

\pnum
\complexity
Exactly
\tcode{last - first},
\tcode{n}, or 0 assignments, respectively.
\end{itemdescr}

\rSec2[alg.generate]{Generate}

\indexlibrary{\idxcode{generate}}%
\indexlibrary{\idxcode{generate_n}}%
\begin{removedblock}
\begin{itemdecl}
template<class ForwardIterator, class Generator>
  void generate(ForwardIterator first, ForwardIterator last,
                Generator gen);

template<class OutputIterator, class Size, class Generator>
  OutputIterator generate_n(OutputIterator first, Size n, Generator gen);
\end{itemdecl}
\end{removedblock}
\begin{addedblock}
\begin{itemdecl}
template<Function F, OutputIterator<ResultType<F>> O,
    Sentinel<O> S>
  O generate(O first, S last, F gen);

template<Function F, OutputRange<ResultType<F>> Rng>
  @\oldtxt{IteratorType}\newtxt{safe_iterator_t}@<Rng>
    generate(Rng&@\newtxt{\&}@ rng, F gen);

template<Function F, WeakOutputIterator<ResultType<F>> O>
  O generate_n(O first, @\oldtxt{Distance}\newtxt{Difference}@Type<O> n, F gen);
\end{itemdecl}
\end{addedblock}

\begin{itemdescr}
\pnum
\effects
\changed{The first algorithm}{\tcode{generate}} invokes the function object \tcode{gen}
and assigns the return value of \tcode{gen} through all the iterators in the range
\range{first}{last}. \changed{The second algorithm}{\tcode{generate_n}} invokes the
function object \tcode{gen} and assigns the return value of \tcode{gen} through all
the iterators in the range \range{first}{first + n} if \tcode{n} is positive,
otherwise it does nothing.

\begin{removedblock}
\pnum
\requires
\tcode{gen} takes no arguments,
\tcode{Size}
shall be convertible to an integral type~(\cxxref{conv.integral}, \cxxref{class.conv}).
\end{removedblock}

\pnum
\returns \added{\tcode{generate} returns \tcode{last}. }\tcode{generate_n} returns
\tcode{first + n} for non-negative values of \tcode{n} and \tcode{first} for negative values.

\pnum
\complexity
Exactly
\tcode{last - first},
\tcode{n}, or 0
invocations of \tcode{gen} and assignments, respectively.
\end{itemdescr}

\rSec2[alg.remove]{Remove}

\indexlibrary{\idxcode{remove}}%
\indexlibrary{\idxcode{remove_if}}%
\begin{removedblock}
\end{removedblock}
\begin{addedblock}
\begin{itemdecl}
template<ForwardIterator I, Sentinel<I> S, class T, class Proj = identity>
  requires Permutable<I>@\newtxt{()}@ &&
    IndirectCallableRelation<equal_to<>, Projected<I, Proj>, const T *>@\newtxt{()}@
  I remove(I first, S last, const T& value, Proj proj = Proj{});

template<ForwardRange Rng, class T, class Proj = identity>
  requires Permutable<IteratorType<Rng>>@\newtxt{()}@ &&
    IndirectCallableRelation<equal_to<>, Projected<IteratorType<Rng>, Proj>, const T *>@\newtxt{()}@
  @\oldtxt{IteratorType}\newtxt{safe_iterator_t}@<Rng>
    remove(Rng&@\newtxt{\&}@ rng, const T& value, Proj proj = Proj{});

template<ForwardIterator I, Sentinel<I> S, class Proj = identity,
    IndirectCallablePredicate<Projected<I, Proj>> Pred>
  requires Permutable<I>@\newtxt{()}@
  I remove_if(I first, S last, Pred pred, Proj proj = Proj{});

template<ForwardRange Rng, class Proj = identity,
    IndirectCallablePredicate<Projected<IteratorType<Rng>, Proj>> Pred>
  requires Permutable<IteratorType<Rng>>@\newtxt{()}@
  @\oldtxt{IteratorType}\newtxt{safe_iterator_t}@<Rng>
    remove_if(Rng&@\newtxt{\&}@ rng, Pred pred, Proj proj = Proj{});
\end{itemdecl}
\end{addedblock}

\begin{itemdescr}
\begin{removedblock}
\pnum
\requires
The type of
\tcode{*first}
shall satisfy the \tcode{MoveAssignable}
requirements (Table~\cxxref{moveassignable}).
\end{removedblock}

\pnum
\effects
Eliminates all the elements referred to by iterator
\tcode{i}
in the range \range{first}{last}
for which the following corresponding conditions hold:
\tcode{\changed{*i == value}{\textit{INVOKE}(proj, *i) == value}},
\tcode{\changed{pred(*i) != false}{\textit{INVOKE}(pred, \textit{INVOKE}(proj, *i)) != false}}.

\pnum
\returns
The end of the resulting range.

\pnum
\remarks Stable~(\cxxref{algorithm.stable}).

\pnum
\complexity
Exactly
\tcode{last - first}
applications of the corresponding predicate\added{ and projection}.

\pnum
\realnote each element in the range \range{ret}{last}, where \tcode{ret} is
the returned value, has a valid but unspecified state, because the algorithms
can eliminate elements by moving from elements that were originally
in that range.
\end{itemdescr}

\indexlibrary{\idxcode{remove_copy}}%
\indexlibrary{\idxcode{remove_copy_if}}%
\begin{removedblock}
\begin{itemdecl}
template<class InputIterator, class OutputIterator, class T>
  OutputIterator
    remove_copy(InputIterator first, InputIterator last,
                OutputIterator result, const T& value);

template<class InputIterator, class OutputIterator, class Predicate>
  OutputIterator
    remove_copy_if(InputIterator first, InputIterator last,
                   OutputIterator result, Predicate pred);
\end{itemdecl}
\end{removedblock}
\begin{addedblock}
\begin{itemdecl}
template<InputIterator I, Sentinel<I> S, WeaklyIncrementable O, class T,
    class Proj = identity>
  requires IndirectlyCopyable<I, O>@\newtxt{()}@ &&
    IndirectCallableRelation<equal_to<>, Projected<I, Proj>, const T *>@\newtxt{()}@
  tagged_pair<tag::in(I), tag::out(O)>
    remove_copy(I first, S last, O result, const T& value, Proj proj = Proj{});

template<InputRange Rng, WeaklyIncrementable O, class T, class Proj = identity>
  requires IndirectlyCopyable<IteratorType<Rng>, O>@\newtxt{()}@ &&
    IndirectCallableRelation<equal_to<>, Projected<IteratorType<Rng>, Proj>, const T *>@\newtxt{()}@
  tagged_pair<tag::in(@\oldtxt{IteratorType}\newtxt{safe_iterator_t}@<Rng>), tag::out(O)>
    remove_copy(Rng&@\newtxt{\&}@ rng, O result, const T& value, Proj proj = Proj{});

template<InputIterator I, Sentinel<I> S, WeaklyIncrementable O,
    class Proj = identity, IndirectCallablePredicate<Projected<I, Proj>> Pred>
  requires IndirectlyCopyable<I, O>@\newtxt{()}@
  tagged_pair<tag::in(I), tag::out(O)>
    remove_copy_if(I first, S last, O result, Pred pred, Proj proj = Proj{});

template<InputRange Rng, WeaklyIncrementable O, class Proj = identity,
    IndirectCallablePredicate<Projected<IteratorType<Rng>, Proj>> Pred>
  requires IndirectlyCopyable<IteratorType<Rng>, O>@\newtxt{()}@
  tagged_pair<tag::in(@\oldtxt{IteratorType}\newtxt{safe_iterator_t}@<Rng>), tag::out(O)>
    remove_copy_if(Rng&@\newtxt{\&}@ rng, O result, Pred pred, Proj proj = Proj{});
\end{itemdecl}
\end{addedblock}

\begin{itemdescr}
\pnum
\requires
The ranges
\range{first}{last}
and
\range{result}{result + (last - first)}
shall not overlap.\removed{ The expression \tcode{*result = *first} shall be valid.}

\pnum
\effects
Copies all the elements referred to by the iterator
\tcode{i}
in the range
\range{first}{last}
for which the following corresponding conditions do not hold:
\tcode{\changed{*i == value}{\textit{INVOKE}(proj, *i) == value}},
\tcode{\changed{pred(*i) != false}{\textit{INVOKE}(pred, \textit{INVOKE}(proj, *i)) != false}}.

\pnum
\returns
\changed{T}{A pair consisting of \tcode{last} and t}he end of the resulting range.

\pnum
\complexity
Exactly
\tcode{last - first}
applications of the corresponding predicate\added{ and projection}.

\pnum
\remarks Stable~(\cxxref{algorithm.stable}).
\end{itemdescr}

\rSec2[alg.unique]{Unique}

\indexlibrary{\idxcode{unique}}%
\begin{removedblock}
\begin{itemdecl}
template<class ForwardIterator>
  ForwardIterator unique(ForwardIterator first, ForwardIterator last);

template<class ForwardIterator, class BinaryPredicate>
  ForwardIterator unique(ForwardIterator first, ForwardIterator last,
                         BinaryPredicate pred);
\end{itemdecl}
\end{removedblock}
\begin{addedblock}
\begin{itemdecl}
template<ForwardIterator I, Sentinel<I> S, class Proj = identity,
    IndirectCallableRelation<Projected<I, Proj>> R = equal_to<>>
  requires Permutable<I>@\newtxt{()}@
  I unique(I first, S last, R comp = R{}, Proj proj = Proj{});

template<ForwardRange Rng, class Proj = identity,
    IndirectCallableRelation<Projected<IteratorType<Rng>, Proj>> R = equal_to<>>
  requires Permutable<IteratorType<Rng>>@\newtxt{()}@
  @\oldtxt{IteratorType}\newtxt{safe_iterator_t}@<Rng>
    unique(Rng&@\newtxt{\&}@ rng, R comp = R{}, Proj proj = Proj{});
\end{itemdecl}
\end{addedblock}

\begin{itemdescr}
\pnum
\effects
For a nonempty range, eliminates all but the first element from every
consecutive group of equivalent elements referred to by the iterator
\tcode{i}
in the range
\range{first + 1}{last}
for which the following conditions hold:
\tcode{\changed{*(i - 1) == *i}{\textit{INVOKE}(proj, *(i - 1)) == \textit{INVOKE}(proj, *i)}}
or
\tcode{\changed{pred(*(i - 1), *i) != false}{\textit{INVOKE}(pred, \textit{INVOKE}(proj, *(i - 1)), \textit{INVOKE}(proj, *i)) != false}}.

\begin{removedblock}
\pnum
\requires
The comparison function shall be an equivalence relation.
The type of \tcode{*first} shall satisfy the
\tcode{MoveAssignable} requirements (Table~\cxxref{moveassignable}).
\end{removedblock}

\pnum
\returns
The end of the resulting range.

\pnum
\complexity
For nonempty ranges, exactly
\tcode{(last - first) - 1}
applications of the corresponding predicate\added{ and no more than twice as many
applications of the projection}.
\end{itemdescr}

\indexlibrary{\idxcode{unique_copy}}%
\begin{removedblock}
\begin{itemdecl}
template<class InputIterator, class OutputIterator>
  OutputIterator
    unique_copy(InputIterator first, InputIterator last,
                OutputIterator result);

template<class InputIterator, class OutputIterator,
         class BinaryPredicate>
  OutputIterator
    unique_copy(InputIterator first, InputIterator last,
                OutputIterator result, BinaryPredicate pred);
\end{itemdecl}
\end{removedblock}
\begin{addedblock}
\begin{itemdecl}
template<InputIterator I, Sentinel<I> S, WeaklyIncrementable O,
    class Proj = identity, IndirectCallableRelation<Projected<I, Proj>> R = equal_to<>>
  requires IndirectlyCopyable<I, O>@\newtxt{() \&\& (ForwardIterator<I>() ||}@
    @\newtxt{ForwardIterator<O>() || Copyable<ValueType<I>{}>())}@
  tagged_pair<tag::in(I), tag::out(O)>
    unique_copy(I first, S last, O result, R comp = R{}, Proj proj = Proj{});

template<InputRange Rng, WeaklyIncrementable O, class Proj = identity,
    IndirectCallableRelation<Projected<IteratorType<Rng>, Proj>> R = equal_to<>>
  requires IndirectlyCopyable<IteratorType<Rng>, O>@\newtxt{() \&\&}@
    @\newtxt{(ForwardIterator<IteratorType<Rng>{}>() || ForwardIterator<O>() ||}@
     @\newtxt{Copyable<ValueType<IteratorType<Rng>{}>{}>())}@
  tagged_pair<tag::in(@\oldtxt{IteratorType}\newtxt{safe_iterator_t}@<Rng>), tag::out(O)>
    unique_copy(Rng&@\newtxt{\&}@ rng, O result, R comp = R{}, Proj proj = Proj{});
\end{itemdecl}
\end{addedblock}

\begin{itemdescr}
\pnum
\requires
\removed{The comparison function shall be an equivalence relation.}
The ranges
\range{first}{last}
and
\range{result}{result+(last-first)}
shall not overlap.\removed{ The expression
\tcode{*result = *first}
shall be valid.} \oldoldtxt{If neither
\tcode{InputIterator}
nor
\tcode{OutputIterator}
meets the requirements of forward iterator then the value type of
\tcode{InputIterator}
shall be \tcode{CopyConstructible} (Table~\cxxref{copyconstructible}) and
\tcode{CopyAssignable} (Table~\cxxref{copyassignable}).
Otherwise \tcode{CopyConstructible} is not required.}

\pnum
\effects
Copies only the first element from every consecutive group of equal elements referred to by
the iterator
\tcode{i}
in the range
\range{first}{last}
for which the following corresponding conditions hold:
\tcode{\changed{*i == *(i - 1)}{\textit{INVOKE}(proj, *i) == \textit{INVOKE}(proj, *(i - 1))}}
or
\tcode{\changed{pred(*i, *(i - 1)) != false}{\textit{INVOKE}(pred, \textit{INVOKE}(proj, *i), \textit{INVOKE}(proj, *(i - 1))) != false}}.

\pnum
\returns
\changed{T}{A pair consisting of \tcode{last} and t}he end of the resulting range.

\pnum
\complexity
For nonempty ranges, exactly
\tcode{last - first - 1}
applications of the corresponding predicate\added{ and no more than twice as many
applications of the projection}.
\end{itemdescr}

\rSec2[alg.reverse]{Reverse}

\indexlibrary{\idxcode{reverse}}%
\begin{removedblock}
\begin{itemdecl}
template<class BidirectionalIterator>
  void reverse(BidirectionalIterator first, BidirectionalIterator last);
\end{itemdecl}
\end{removedblock}
\begin{addedblock}
\begin{itemdecl}
template<BidirectionalIterator I, Sentinel<I> S>
  requires Permutable<I>@\newtxt{()}@
  I reverse(I first, S last);

template<BidirectionalRange Rng>
  requires Permutable<IteratorType<Rng>>@\newtxt{()}@
  @\oldtxt{IteratorType}\newtxt{safe_iterator_t}@<Rng>
    reverse(Rng&@\newtxt{\&}@ rng);
\end{itemdecl}
\end{addedblock}

\begin{itemdescr}
\pnum
\effects
For each non-negative integer
\tcode{i < (last - first)/2},
applies
\tcode{iter_swap}
to all pairs of iterators
\tcode{first + i, (last - i) - 1}.

\begin{removedblock}
\pnum
\requires
\tcode{*first} shall be swappable~(\ref{concepts.lib.corelang.swappable}).
\end{removedblock}

\begin{addedblock}
\pnum
\returns \tcode{last}.
\end{addedblock}

\pnum
\complexity
Exactly
\tcode{(last - first)/2}
swaps.
\end{itemdescr}

\indexlibrary{\idxcode{reverse_copy}}%
\begin{removedblock}
\begin{itemdecl}
template<class BidirectionalIterator, class OutputIterator>
  OutputIterator
    reverse_copy(BidirectionalIterator first,
                 BidirectionalIterator last, OutputIterator result);
\end{itemdecl}
\end{removedblock}
\begin{addedblock}
\begin{itemdecl}
template<BidirectionalIterator I, Sentinel<I> S, WeaklyIncrementable O>
  requires IndirectlyCopyable<I, O>@\newtxt{()}@
  tagged_pair<tag::in(I), tag::out(O)> reverse_copy(I first, S last, O result);

template<BidirectionalRange Rng, WeaklyIncrementable O>
  requires IndirectlyCopyable<IteratorType<Rng>, O>@\newtxt{()}@
  tagged_pair<tag::in(@\oldtxt{IteratorType}\newtxt{safe_iterator_t}@<Rng>), tag::out(O)>
    reverse_copy(Rng&@\newtxt{\&}@ rng, O result);
\end{itemdecl}
\end{addedblock}

\begin{itemdescr}
\pnum
\effects
Copies the range
\range{first}{last}
to the range
\range{result}{result+(last-first)}
such that
for every non-negative integer
\tcode{i < (last - first)}
the following assignment takes place:
\tcode{*(result + (last - first) - 1 - i) = *(first + i)}.

\pnum
\requires
The ranges
\range{first}{last}
and
\range{result}{result+(last-first)}
shall not overlap.

\pnum
\returns
\tcode{\changed{result + (last - first)}{\{last, result + (last - first)\}}}.

\pnum
\complexity
Exactly
\tcode{last - first}
assignments.
\end{itemdescr}

\rSec2[alg.rotate]{Rotate}

\indexlibrary{\idxcode{rotate}}%
\begin{removedblock}
\begin{itemdecl}
template<class ForwardIterator>
  ForwardIterator rotate(ForwardIterator first, ForwardIterator middle,
              ForwardIterator last);
\end{itemdecl}
\end{removedblock}
\begin{addedblock}
\begin{itemdecl}
template<ForwardIterator I, Sentinel<I> S>
  requires Permutable<I>@\newtxt{()}@
  tagged_pair<tag::begin(I), tag::end(I)> rotate(I first, I middle, S last);

template<ForwardRange Rng>
  requires Permutable<IteratorType<Rng>>@\newtxt{()}@
  tagged_pair<tag::begin(@\oldtxt{IteratorType}\newtxt{safe_iterator_t}@<Rng>), tag::end(@\oldtxt{IteratorType}\newtxt{safe_iterator_t}@<Rng>)>
    rotate(Rng&@\newtxt{\&}@ rng, IteratorType<Rng> middle);
\end{itemdecl}
\end{addedblock}

\begin{itemdescr}
\pnum
\effects
For each non-negative integer
\tcode{i < (last - first)},
places the element from the position
\tcode{first + i}
into position
\tcode{first + (i + (last - middle)) \% (last - first)}.

\pnum
\returns \tcode{\changed{first + (last - middle)}{\{first + (last - middle), last\}}}.

\pnum
\notes
This is a left rotate.

\pnum
\requires
\range{first}{middle}
and
\range{middle}{last}
shall be valid ranges.
\removed{
\tcode{ForwardIterator} shall satisfy the requirements of
\tcode{ValueSwappable}~(\ref{concepts.lib.corelang.swappable}). The type of \tcode{*first} shall satisfy
the requirements of \tcode{MoveConstructible}
(Table~\cxxref{moveconstructible}) and the
requirements of
\tcode{MoveAssignable}
(Table~\cxxref{moveassignable}).}

\pnum
\complexity
At most
\tcode{last - first}
swaps.
\end{itemdescr}

\indexlibrary{\idxcode{rotate_copy}}%
\begin{removedblock}
\begin{itemdecl}
template<class ForwardIterator, class OutputIterator>
  OutputIterator
    rotate_copy(ForwardIterator first, ForwardIterator middle,
                ForwardIterator last, OutputIterator result);
\end{itemdecl}
\end{removedblock}
\begin{addedblock}
\begin{itemdecl}
template<ForwardIterator I, Sentinel<I> S, WeaklyIncrementable O>
  requires IndirectlyCopyable<I, O>@\newtxt{()}@
  tagged_pair<tag::in(I), tag::out(O)>
    rotate_copy(I first, I middle, S last, O result);

template<ForwardRange Rng, WeaklyIncrementable O>
  requires IndirectlyCopyable<IteratorType<Rng>, O>@\newtxt{()}@
  tagged_pair<tag::in(@\oldtxt{IteratorType}\newtxt{safe_iterator_t}@<Rng>), tag::out(O)>
    rotate_copy(Rng&@\newtxt{\&}@ rng, IteratorType<Rng> middle, O result);
\end{itemdecl}
\end{addedblock}

\begin{itemdescr}
\pnum
\effects
Copies the range
\range{first}{last}
to the range
\range{result}{result + (last - first)}
such that for each non-negative integer
\tcode{i < (last - first)}
the following assignment takes place:
\tcode{*(result + i) =  *(first +
(i + (middle - first)) \% (last - first))}.

\pnum
\returns
\tcode{\changed{result + (last - first)}{\{last, result + (last - first)\}}}.

\pnum
\requires
The ranges
\range{first}{last}
and
\range{result}{result + (last - first)}
shall not overlap.

\pnum
\complexity
Exactly
\tcode{last - first}
assignments.
\end{itemdescr}

\rSec2[alg.random.shuffle]{Shuffle}

\indexlibrary{\idxcode{shuffle}}%
\begin{removedblock}
\begin{itemdecl}
template<class RandomAccessIterator, class UniformRandomNumberGenerator>
  void shuffle(RandomAccessIterator first,
                      RandomAccessIterator last,
                      UniformRandomNumberGenerator&& g);
\end{itemdecl}
\end{removedblock}
\begin{addedblock}
\begin{itemdecl}
template<RandomAccessIterator I, Sentinel<I> S, class Gen>
  requires Permutable<I>@\newtxt{()}@ && Convertible@\newtxt{To}@<ResultType<Gen>, DifferenceType<I>>@\newtxt{()}@ &&
    UniformRandomNumberGenerator<remove_reference_t<Gen>>@\newtxt{()}@
  I shuffle(I first, S last, Gen&& g);

template<RandomAccessRange Rng, class Gen>
  requires Permutable<I>@\newtxt{()}@ && Convertible@\newtxt{To}@<ResultType<Gen>, DifferenceType<I>>@\newtxt{()}@ &&
    UniformRandomNumberGenerator<remove_reference_t<Gen>>@\newtxt{()}@
  @\oldtxt{IteratorType}\newtxt{safe_iterator_t}@<Rng>
    shuffle(Rng&@\newtxt{\&}@ rng, Gen&& g);
\end{itemdecl}
\end{addedblock}

\begin{itemdescr}
\pnum
\effects
Permutes the elements in the range
\range{first}{last}
such that each possible permutation of those elements has equal probability of appearance.

\begin{removedblock}
\pnum
\requires
\tcode{RandomAccessIterator} shall satisfy the requirements of
\tcode{ValueSwappable}~(\ref{concepts.lib.corelang.swappable}).
The type
\tcode{UniformRandomNumberGenerator} shall meet the requirements of a uniform
random number generator~(\cxxref{rand.req.urng}) type whose return type is
convertible to
\tcode{iterator_traits<Random\-Access\-Itera\-tor>::difference_type}.
\end{removedblock}

\pnum
\complexity
Exactly
\tcode{(last - first) - 1}
swaps.

\begin{addedblock}
\pnum
\returns \tcode{last}
\end{addedblock}

\pnum
\notes
To the extent that the implementation of this function makes use of random
numbers, the object \tcode{g} shall serve as the implementation's source of
randomness.

\end{itemdescr}

\rSec2[alg.partitions]{Partitions}

\indexlibrary{\idxcode{is_partitioned}}%
\begin{removedblock}
\begin{itemdecl}
template<class InputIterator, class Predicate>
  bool is_partitioned(InputIterator first, InputIterator last, Predicate pred);
\end{itemdecl}
\end{removedblock}
\begin{addedblock}
\begin{itemdecl}
template<InputIterator I, Sentinel<I> S, class Proj = identity,
    IndirectCallablePredicate<Projected<I, Proj>> Pred>
  bool is_partitioned(I first, S last, Pred pred, Proj proj = Proj{});

template<InputRange Rng, class Proj = identity,
    IndirectCallablePredicate<Projected<IteratorType<Rng>, Proj>> Pred>
  bool
    is_partitioned(Rng&& rng, Pred pred, Proj proj = Proj{});
\end{itemdecl}
\end{addedblock}

\begin{itemdescr}
\begin{removedblock}
\pnum
\requires \tcode{InputIterator}'s value type shall be convertible to \tcode{Predicate}'s argument type.
\end{removedblock}

\pnum
\returns \tcode{true} if
\range{first}{last} is empty or if
\range{first}{last} is partitioned by \tcode{pred}\added{ and \tcode{proj}}, i.e. if all
\changed{elements that satisfy \tcode{pred} appear}{iterators \tcode{i} for which
\tcode{\textit{INVOKE}(pred, \textit{INVOKE}(proj, *i)) != false} come} before those that do not\added{,
for every \tcode{i} in \range{first}{last}}.

\pnum
\complexity Linear. At most \tcode{last - first} applications of \tcode{pred}\added{ and \tcode{proj}}.
\end{itemdescr}

\indexlibrary{\idxcode{partition}}%
\begin{removedblock}
\begin{itemdecl}
template<class ForwardIterator, class Predicate>
  ForwardIterator
    partition(ForwardIterator first,
              ForwardIterator last, Predicate pred);
\end{itemdecl}
\end{removedblock}
\begin{addedblock}
\begin{itemdecl}
template<ForwardIterator I, Sentinel<I> S, class Proj = identity,
    IndirectCallablePredicate<Projected<I, Proj>> Pred>
  requires Permutable<I>@\newtxt{()}@
  I partition(I first, S last, Pred pred, Proj proj = Proj{});

template<ForwardRange Rng, class Proj = identity,
    IndirectCallablePredicate<Projected<IteratorType<Rng>, Proj>> Pred>
  requires Permutable<IteratorType<Rng>>@\newtxt{()}@
  @\oldtxt{IteratorType}\newtxt{safe_iterator_t}@<Rng>
    partition(Rng&@\newtxt{\&}@ rng, Pred pred, Proj proj = Proj{});
\end{itemdecl}
\end{addedblock}

\begin{itemdescr}
\pnum
\begin{removedblock}
\effects Places all the elements in the range \range{first}{last} that satisfy
\tcode{pred} before all the elements that do not satisfy it.
\end{removedblock}

\begin{addedblock}
\effects Permutes the elements in the range \range{first}{last} such that there exists and iterator \tcode{i}
such that for every iterator \tcode{j} in the range \range{first}{i}
\tcode{\textit{INVOKE}(pred, \textit{INVOKE}(proj, *j)) != false}, and for every iterator \tcode{k} in the
range \range{i}{last}, \tcode{\textit{INVOKE}(pred, \textit{INVOKE}(proj, *k)) == false}
\end{addedblock}

\pnum
\returns An iterator \tcode{i} such that for every iterator \tcode{j} in the range \range{first}{i}
\tcode{\changed{pred(*j) != false}{\textit{INVOKE}(pred, \textit{INVOKE}(proj, *j)) != false}},
and for every iterator \tcode{k} in the range \range{i}{last},
\tcode{\changed{pred(*k) == false}{\textit{INVOKE}(pred, \textit{INVOKE}(proj, *k)) == false}}.

\begin{removedblock}
\pnum
\requires
\tcode{ForwardIterator} shall satisfy the requirements of
\tcode{ValueSwappable}~(\ref{concepts.lib.corelang.swappable}).
\end{removedblock}

\pnum
\complexity If \changed{ForwardIterator}{I} meets the requirements for a BidirectionalIterator, at most
\tcode{(last - first) / 2} swaps are done; otherwise at most \tcode{last - first} swaps
are done. Exactly \tcode{last - first} applications of the predicate\added{ and projection} are done.
\end{itemdescr}

\indexlibrary{\idxcode{stable_partition}}%
\begin{removedblock}
\begin{itemdecl}
template<class BidirectionalIterator, class Predicate>
  BidirectionalIterator
    stable_partition(BidirectionalIterator first,
                     BidirectionalIterator last, Predicate pred);
\end{itemdecl}
\end{removedblock}
\begin{addedblock}
\begin{itemdecl}
template<BidirectionalIterator I, Sentinel<I> S, class Proj = identity,
    IndirectCallablePredicate<Projected<I, Proj>> Pred>
  requires Permutable<I>@\newtxt{()}@
  I stable_partition(I first, S last, Pred pred, Proj proj = Proj{});

template<BidirectionalRange Rng, class Proj = identity,
    IndirectCallablePredicate<Projected<IteratorType<Rng>, Proj>> Pred>
  requires Permutable<IteratorType<Rng>>@\newtxt{()}@
  @\oldtxt{IteratorType}\newtxt{safe_iterator_t}@<Rng>
    stable_partition(Rng&@\newtxt{\&}@ rng, Pred pred, Proj proj = Proj{});
\end{itemdecl}
\end{addedblock}

\begin{itemdescr}
\pnum
\begin{removedblock}
\effects
Places all the elements in the range
\range{first}{last}
that satisfy \tcode{pred} before all the
elements that do not satisfy it.
\end{removedblock}

\begin{addedblock}
\effects Permutes the elements in the range \range{first}{last} such that there exists and iterator \tcode{i}
such that for every iterator \tcode{j} in the range \range{first}{i}
\tcode{\textit{INVOKE}(pred, \textit{INVOKE}(proj, *j)) != false}, and for every iterator \tcode{k} in the
range \range{i}{last}, \tcode{\textit{INVOKE}(pred, \textit{INVOKE}(proj, *k)) == false}
\end{addedblock}

\pnum
\returns
An iterator
\tcode{i}
such that for every iterator
\tcode{j}
in the range
\range{first}{i},
\tcode{\changed{pred(*j) != false}{\textit{INVOKE}(pred, \textit{INVOKE}(proj, *j)) != false}},
and for every iterator
\tcode{k}
in the range
\range{i}{last},
\tcode{\changed{pred(*k) == false}{\textit{INVOKE}(pred, \textit{INVOKE}(proj, *k)) == false}}.
The relative order of the elements in both groups is preserved.

\begin{removedblock}
\pnum
\requires
\tcode{BidirectionalIterator} shall satisfy the requirements of
\tcode{ValueSwappable}~(\ref{concepts.lib.corelang.swappable}). The type
of \tcode{*first} shall satisfy the requirements of
\tcode{MoveConstructible} (Table~\cxxref{moveconstructible}) and of
\tcode{MoveAssignable} (Table~\cxxref{moveassignable}).
\end{removedblock}

\pnum
\complexity
At most
\tcode{(last - first) * log(last - first)}
swaps, but only linear number of swaps if there is enough extra memory.
Exactly
\tcode{last - first}
applications of the predicate\added{ and projection}.
\end{itemdescr}

\indexlibrary{\idxcode{partition_copy}}%
\begin{removedblock}
\begin{itemdecl}
template<class InputIterator, class OutputIterator1,
          class OutputIterator2, class Predicate>
  pair<OutputIterator1, OutputIterator2>
  partition_copy(InputIterator first, InputIterator last,
                 OutputIterator1 out_true, OutputIterator2 out_false,
                 Predicate pred);
\end{itemdecl}
\end{removedblock}
\begin{addedblock}
\begin{itemdecl}
template<InputIterator I, Sentinel<I> S, WeaklyIncrementable O1, WeaklyIncrementable O2,
    class Proj = identity, IndirectCallablePredicate<Projected<I, Proj>> Pred>
  requires IndirectlyCopyable<I, O1>@\newtxt{()}@ && IndirectlyCopyable<I, O2>@\newtxt{()}@
  tagged_tuple<tag::in(I), tag::out1(O1), tag::out2(O2)>
    partition_copy(I first, S last, O1 out_true, O2 out_false, Pred pred,
                   Proj proj = Proj{});

template<InputRange Rng, WeaklyIncrementable O1, WeaklyIncrementable O2,
    class Proj = identity,
    IndirectCallablePredicate<Projected<IteratorType<Rng>, Proj>> Pred>
  requires IndirectlyCopyable<IteratorType<Rng>, O1>@\newtxt{()}@ &&
    IndirectlyCopyable<IteratorType<Rng>, O2>@\newtxt{()}@
  tagged_tuple<tag::in(@\oldtxt{IteratorType}\newtxt{safe_iterator_t}@<Rng>), tag::out1(O1), tag::out2(O2)>
    partition_copy(Rng&@\newtxt{\&}@ rng, O1 out_true, O2 out_false, Pred pred, Proj proj = Proj{});
\end{itemdecl}
\end{addedblock}

\begin{itemdescr}
\pnum
\requires \removed{\tcode{InputIterator}'s value type shall be \tcode{CopyAssignable}, and shall be
writable to the \tcode{out_true} and \tcode{out_false} \tcode{OutputIterator}s, and shall be
convertible to \tcode{Predicate}'s argument type. }The input range shall not overlap with
either of the output ranges.

\pnum
\effects For each iterator \tcode{i} in \range{first}{last}, copies \tcode{*i} to the output range
beginning with \tcode{out_true} if
\tcode{\changed{pred(*i)}{\textit{INVOKE}(pred, \textit{INVOKE}(proj, *i))}} is \tcode{true}, or to
the output range beginning with \tcode{out_false} otherwise.

\pnum
\returns A \changed{pair}{tuple} \tcode{p} such that\added{ \tcode{get<0>(p)} is \tcode{last}},
\tcode{\changed{p.first}{get<1>(p)}} is the end of the output range beginning at \tcode{out_true}
and \tcode{\changed{p.second}{get<2>(p)}} is the end of the output range beginning at \tcode{out_false}.

\pnum
\complexity Exactly \tcode{last - first} applications of \tcode{pred}\added{ and \tcode{proj}}.
\end{itemdescr}


\begin{addedblock}
\indexlibrary{\idxcode{partition_move}}%
\begin{itemdecl}
template<InputIterator I, Sentinel<I> S, WeaklyIncrementable O1, WeaklyIncrementable O2,
    class Proj = identity,
    IndirectCallablePredicate<Projected<I, Proj>> Pred>
  requires IndirectlyMovable<I, O1>@\newtxt{()}@ && IndirectlyMovable<I, O2>@\newtxt{()}@
  tagged_tuple<tag::in(I), tag::out1(O1), tag::out2(O2)>
    partition_move(I first, S last, O1 out_true, O2 out_false, Pred pred,
                   Proj proj = Proj{});

template<InputRange Rng, WeaklyIncrementable O1, WeaklyIncrementable O2,
    class Proj = identity,
    IndirectCallablePredicate<Projected<IteratorType<Rng>, Proj>> Pred>
  requires IndirectlyMovable<IteratorType<Rng>, O1>@\newtxt{()}@ &&
    IndirectlyMovable<IteratorType<Rng>, O2>@\newtxt{()}@
  tagged_tuple<tag::in(@\oldtxt{IteratorType}\newtxt{safe_iterator_t}@<Rng>), tag::out1(O1), tag::out2(O2)>
    partition_move(Rng&@\newtxt{\&}@ rng, O1 out_true, O2 out_false, Pred pred,
                   Proj proj = Proj{});
\end{itemdecl}

\begin{itemdescr}
\pnum
\requires The input range shall not overlap with either of the output ranges.

\pnum
\effects For each iterator \tcode{i} in \range{first}{last}, moves \tcode{*i} to the output range
beginning with \tcode{out_true} if
\tcode{\textit{INVOKE}(pred, \textit{INVOKE}(proj, *i))} is \tcode{true}, or to
the output range beginning with \tcode{out_false} otherwise.

\pnum
\returns A tuple \tcode{p} such that \tcode{get<0>(p)} is \tcode{last},
\tcode{get<1>(p)} is the end of the output range beginning at \tcode{out_true}
and \tcode{get<2>(p)} is the end of the output range beginning at \tcode{out_false}.

\pnum
\complexity Exactly \tcode{last - first} applications of \tcode{pred} and \tcode{proj}.
\end{itemdescr}
\end{addedblock}


\indexlibrary{\idxcode{partition_point}}%
\begin{removedblock}
\begin{itemdecl}
template<class ForwardIterator, class Predicate>
  ForwardIterator partition_point(ForwardIterator first,
                                  ForwardIterator last,
                                  Predicate pred);
\end{itemdecl}
\end{removedblock}
\begin{addedblock}
\begin{itemdecl}
template<ForwardIterator I, Sentinel<I> S, class Proj = identity,
    IndirectCallablePredicate<Projected<I, Proj>> Pred>
  I partition_point(I first, S last, Pred pred, Proj proj = Proj{});

template<ForwardRange Rng, class Proj = identity,
    IndirectCallablePredicate<Projected<IteratorType<Rng>, Proj>> Pred>
  @\oldtxt{IteratorType}\newtxt{safe_iterator_t}@<Rng>
    partition_point(Rng&@\newtxt{\&}@ rng, Pred pred, Proj proj = Proj{});
\end{itemdecl}
\end{addedblock}

\begin{itemdescr}
\pnum
\requires \removed{\tcode{ForwardIterator}'s value type shall be convertible to \tcode{Predicate}'s argument
type. }\range{first}{last} shall be partitioned by \tcode{pred}\added{ and \tcode{proj}}, i.e.
\removed{all elements that satisfy \tcode{pred} shall appear before those that do not}
\added{there should be an iterator \tcode{mid} such that }
\tcode{\added{all_of(first, mid, pred, proj)}}\added{ and }\tcode{\added{none_of(mid, last, pred, proj)}}
\added{ are both true}.

\pnum
\returns An iterator \tcode{mid} such that \tcode{all_of(first, mid, pred\added{, proj})} and
\tcode{none_of(mid, last, pred\added{, proj})} are both true.

\pnum
\complexity \bigoh{log(last - first)} applications of \tcode{pred}\added{ and \tcode{proj}}.
\end{itemdescr}


\rSec1[alg.sorting]{Sorting and related operations}

\pnum
All the operations in~\ref{alg.sorting} \changed{have two versions: one that takes a function object of type
\tcode{Compare}
and one that uses an
\tcode{operator<}}{take an optional binary callable predicate of type \tcode{Comp} that defaults to \tcode{less<>}}.

\pnum
\tcode{\changed{Compare}{Comp}}
is \changed{a function object
type~(\ref{function.objects})}{a callable object~(\cxxref{func.require})}. The return value of \changed{the
function call operation}{the \tcode{\textit{INVOKE}} operation} applied to
an object of type \tcode{\changed{Compare}{Comp}}, when contextually converted to
\tcode{bool} (Clause~\cxxref{conv}),
yields \tcode{true} if the first argument of the call
is less than the second, and
\tcode{false}
otherwise.
\tcode{\changed{Compare}{Comp} comp}
is used throughout for algorithms assuming an ordering relation.
It is assumed that
\tcode{comp}
will not apply any non-constant function through the dereferenced iterator.

\pnum
\removed{
For all algorithms that take
\tcode{Compare},
there is a version that uses
\tcode{operator<}
instead.
That is,
\tcode{comp(*i, *j) != false}
defaults to
\tcode{*i < *j != false}.
For algorithms other than those described in~\ref{alg.binary.search} to work correctly,
\tcode{comp} has to induce a strict weak ordering on the values.}

\ednote{REVIEW: The above (struck) sentence implies that the binary search algorithms do not
require a strict weak ordering relation, but the ``Palo Alto report'' is clear that they do.
Which is it?}

\ednote{The following description of ``strict weak order'' has moved to the definition of the
\tcode{StrictWeakOrder} concept~(\ref{concepts.lib.functions.strictweakorder}).}

\begin{removedblock}
\pnum
The term
\techterm{strict}
refers to the
requirement of an irreflexive relation (\tcode{!comp(x, x)} for all \tcode{x}),
and the term
\techterm{weak}
to requirements that are not as strong as
those for a total ordering,
but stronger than those for a partial
ordering.
If we define
\tcode{equiv(a, b)}
as
\tcode{!comp(a, b) \&\& !comp(b, a)},
then the requirements are that
\tcode{comp}
and
\tcode{equiv}
both be transitive relations:

\begin{itemize}
\item
\tcode{comp(a, b) \&\& comp(b, c)}
implies
\tcode{comp(a, c)}
\item
\tcode{equiv(a, b) \&\& equiv(b, c)}
implies
\tcode{equiv(a, c)}
\enternote
Under these conditions, it can be shown that
\begin{itemize}
\item
\tcode{equiv}
is an equivalence relation
\item
\tcode{comp}
induces a well-defined relation on the equivalence
classes determined by
\tcode{equiv}
\item
The induced relation is a strict total ordering.
\exitnote
\end{itemize}
\end{itemize}
\end{removedblock}

\pnum
A sequence is
\techterm{sorted with respect to a comparator\added{ and projection}}
\tcode{comp}\added{ and \tcode{proj}} if for every iterator
\tcode{i}
pointing to the sequence and every non-negative integer
\tcode{n}
such that
\tcode{i + n}
is a valid iterator pointing to an element of the sequence,
\tcode{\changed{comp(*(i + n), *i) == false}{\textit{INVOKE}(comp, \textit{INVOKE}(proj, *(i + n)), \textit{INVOKE}(proj, *i)) == false}}.

\pnum
A sequence
\range{start}{finish}
is
\techterm{partitioned with respect to an expression}
\tcode{f(e)}
if there exists an integer
\tcode{n}
such that for all
\tcode{0 <= i < distance(start, finish)},
\tcode{f(*(start + i))}
is true if and only if
\tcode{i < n}.

\pnum
In the descriptions of the functions that deal with ordering relationships we frequently use a notion of
equivalence to describe concepts such as stability.
The equivalence to which we refer is not necessarily an
\tcode{operator==},
but an equivalence relation induced by the strict weak ordering.
That is, two elements
\tcode{a}
and
\tcode{b}
are considered equivalent if and only if
\tcode{!(a < b) \&\& !(b < a)}.

\rSec2[alg.sort]{Sorting}

\rSec3[sort]{\tcode{sort}}

\indexlibrary{\idxcode{sort}}%
\begin{removedblock}
\begin{itemdecl}
template<class RandomAccessIterator>
  void sort(RandomAccessIterator first, RandomAccessIterator last);

template<class RandomAccessIterator, class Compare>
  void sort(RandomAccessIterator first, RandomAccessIterator last,
            Compare comp);
\end{itemdecl}
\end{removedblock}
\begin{addedblock}
\begin{itemdecl}
template<RandomAccessIterator I, Sentinel<I> S, class Comp = less<>,
    class Proj = identity>
  requires Sortable<I, Comp, Proj>@\newtxt{()}@
  I sort(I first, S last, Comp comp = Comp{}, Proj proj = Proj{});

template<RandomAccessRange Rng, class Comp = less<>, class Proj = identity>
  requires Sortable<IteratorType<Rng>, Comp, Proj>@\newtxt{()}@
  @\oldtxt{IteratorType}\newtxt{safe_iterator_t}@<Rng>
    sort(Rng&@\newtxt{\&}@ rng, Comp comp = Comp{}, Proj proj = Proj{});
\end{itemdecl}
\end{addedblock}

\begin{itemdescr}
\pnum
\effects
Sorts the elements in the range
\range{first}{last}.

\begin{removedblock}
\pnum
\requires
\tcode{RandomAccessIterator} shall satisfy the requirements of
\tcode{ValueSwappable}~(\ref{concepts.lib.corelang.swappable}). The type
of \tcode{*first} shall satisfy the requirements of
\tcode{MoveConstructible} (Table~\cxxref{moveconstructible}) and of
\tcode{MoveAssignable} (Table~\cxxref{moveassignable}).
\end{removedblock}

\pnum
\complexity
\bigoh{N\log(N)}
(where
\tcode{$N$ == last - first})
comparisons.
\end{itemdescr}

\rSec3[stable.sort]{\tcode{stable_sort}}

\indexlibrary{\idxcode{stable_sort}}%
\begin{removedblock}
\begin{itemdecl}
template<class RandomAccessIterator>
  void stable_sort(RandomAccessIterator first, RandomAccessIterator last);

template<class RandomAccessIterator, class Compare>
  void stable_sort(RandomAccessIterator first, RandomAccessIterator last,
                   Compare comp);
\end{itemdecl}
\end{removedblock}
\begin{addedblock}
\begin{itemdecl}
template<RandomAccessIterator I, Sentinel<I> S, class Comp = less<>,
    class Proj = identity>
  requires Sortable<I, Comp, Proj>@\newtxt{()}@
  I stable_sort(I first, S last, Comp comp = Comp{}, Proj proj = Proj{});

template<RandomAccessRange Rng, class Comp = less<>, class Proj = identity>
  requires Sortable<IteratorType<Rng>, Comp, Proj>@\newtxt{()}@
  @\oldtxt{IteratorType}\newtxt{safe_iterator_t}@<Rng>
    stable_sort(Rng&@\newtxt{\&}@ rng, Comp comp = Comp{}, Proj proj = Proj{});
\end{itemdecl}
\end{addedblock}

\begin{itemdescr}
\pnum
\effects
Sorts the elements in the range \range{first}{last}.

\begin{removedblock}
\pnum
\requires
\tcode{RandomAccessIterator} shall satisfy the requirements of
\tcode{ValueSwappable}~(\ref{concepts.lib.corelang.swappable}). The type
of \tcode{*first} shall satisfy the requirements of
\tcode{MoveConstructible} (Table~\cxxref{moveconstructible}) and of
\tcode{MoveAssignable} (Table~\cxxref{moveassignable}).
\end{removedblock}

\pnum
\complexity
It does at most $N \log^2(N)$
(where
\tcode{$N$ == last - first})
comparisons; if enough extra memory is available, it is
$N \log(N)$.

\pnum
\remarks Stable~(\cxxref{algorithm.stable}).
\end{itemdescr}

\rSec3[partial.sort]{\tcode{partial_sort}}

\indexlibrary{\idxcode{partial_sort}}%
\begin{removedblock}
\begin{itemdecl}
template<class RandomAccessIterator>
  void partial_sort(RandomAccessIterator first,
                    RandomAccessIterator middle,
                    RandomAccessIterator last);

template<class RandomAccessIterator, class Compare>
  void partial_sort(RandomAccessIterator first,
                    RandomAccessIterator middle,
                    RandomAccessIterator last,
                    Compare comp);
\end{itemdecl}
\end{removedblock}
\begin{addedblock}
\begin{itemdecl}
template<RandomAccessIterator I, Sentinel<I> S, class Comp = less<>,
    class Proj = identity>
  requires Sortable<I, Comp, Proj>@\newtxt{()}@
  I partial_sort(I first, I middle, S last, Comp comp = Comp{}, Proj proj = Proj{});

template<RandomAccessRange Rng, class Comp = less<>, class Proj = identity>
  requires Sortable<IteratorType<Rng>, Comp, Proj>@\newtxt{()}@
  @\oldtxt{IteratorType}\newtxt{safe_iterator_t}@<Rng>
    partial_sort(Rng&@\newtxt{\&}@ rng, IteratorType<Rng> middle, Comp comp = Comp{},
                 Proj proj = Proj{});
\end{itemdecl}
\end{addedblock}

\begin{itemdescr}
\pnum
\effects
Places the first
\tcode{middle - first}
sorted elements from the range
\range{first}{last}
into the range
\range{first}{middle}.
The rest of the elements in the range
\range{middle}{last}
are placed in an unspecified order.
\indextext{unspecified}%

\begin{removedblock}
\pnum
\requires
\tcode{RandomAccessIterator} shall satisfy the requirements of
\tcode{ValueSwappable}~(\ref{concepts.lib.corelang.swappable}). The type
of \tcode{*first} shall satisfy the requirements of
\tcode{MoveConstructible} (Table~\cxxref{moveconstructible}) and of
\tcode{MoveAssignable} (Table~\cxxref{moveassignable}).
\end{removedblock}

\pnum
\complexity
It takes approximately
\tcode{(last - first) * log(middle - first)}
comparisons.
\end{itemdescr}

\rSec3[partial.sort.copy]{\tcode{partial_sort_copy}}

\indexlibrary{\idxcode{partial_sort_copy}}%
\begin{removedblock}
\begin{itemdecl}
template<class InputIterator, class RandomAccessIterator>
  RandomAccessIterator
    partial_sort_copy(InputIterator first, InputIterator last,
                      RandomAccessIterator result_first,
                      RandomAccessIterator result_last);

template<class InputIterator, class RandomAccessIterator,
         class Compare>
  RandomAccessIterator
    partial_sort_copy(InputIterator first, InputIterator last,
                      RandomAccessIterator result_first,
                      RandomAccessIterator result_last,
                      Compare comp);
\end{itemdecl}
\end{removedblock}
\begin{addedblock}
\begin{itemdecl}
template<InputIterator I1, Sentinel<I> S1, RandomAccessIterator I2, Sentinel<I> S2,
    class R = less<>, class Proj = identity>
  requires IndirectlyCopyable<I1, I2>@\newtxt{()}@ && Sortable<I2, Comp, Proj>@\newtxt{()}@
  I2
    partial_sort_copy(I1 first, S1 last, I2 result_first, S2 result_last,
                      Comp comp = Comp{}, Proj proj = Proj{});

template<InputRange Rng1, RandomAccessRange Rng2, class R = less<>,
    class Proj = identity>
  requires IndirectlyCopyable<IteratorType<Rng1>, IteratorType<Rng2>>@\newtxt{()}@ &&
      Sortable<IteratorType<Rng2>, Comp, Proj>@\newtxt{()}@
  @\oldtxt{IteratorType}\newtxt{safe_iterator_t}@<Rng2>
    partial_sort_copy(Rng1&@\newtxt{\&}@ rng, Rng2&@\newtxt{\&}@ result_rng, Comp comp = Comp{},
                      Proj proj = Proj{});
\end{itemdecl}
\end{addedblock}

\begin{itemdescr}
\pnum
\effects
Places the first
\tcode{min(last - first, result_last - result_first)}
sorted elements into the range
\range{result_first}{result_first + min(last - first, result_last - result_first)}.

\pnum
\returns
The smaller of:
\tcode{result_last} or
\tcode{result_first + (last - first)}.

\begin{removedblock}
\pnum
\requires
\tcode{RandomAccessIterator} shall satisfy the requirements of
\tcode{ValueSwappable}~(\ref{concepts.lib.corelang.swappable}). The type
of \tcode{*result_first} shall satisfy the requirements of
\tcode{MoveConstructible} (Table~\cxxref{moveconstructible}) and of
\tcode{Move\-Assignable} (Table~\cxxref{moveassignable}).
\end{removedblock}

\pnum
\complexity
Approximately
\tcode{(last - first) * log(min(last - first, result_last - result_first))}
comparisons.
\end{itemdescr}

\rSec3[is.sorted]{\tcode{is_sorted}}

\indexlibrary{\idxcode{is_sorted}}%
\begin{removedblock}
\begin{itemdecl}
template<class ForwardIterator>
  bool is_sorted(ForwardIterator first, ForwardIterator last);
\end{itemdecl}
\end{removedblock}
\begin{addedblock}
\begin{itemdecl}
template<ForwardIterator I, Sentinel<I> S, class Proj = identity,
    IndirectCallableStrictWeakOrder<Projected<I, Proj>> Comp = less<>>
  bool is_sorted(I first, S last, Comp comp = Comp{}, Proj proj = Proj{});

template<ForwardRange Rng, class Proj = identity,
    IndirectCallableStrictWeakOrder<Projected<IteratorType<Rng>, Proj>> Comp = less<>>
  bool
    is_sorted(Rng&& rng, Comp comp = Comp{}, Proj proj = Proj{});
\end{itemdecl}
\end{addedblock}

\begin{itemdescr}
\pnum
\returns \tcode{is_sorted_until(first, last\added{, comp, proj}) == last}
\end{itemdescr}

\begin{removedblock}
\indexlibrary{\idxcode{is_sorted}}%
\begin{itemdecl}
template<class ForwardIterator, class Compare>
  bool is_sorted(ForwardIterator first, ForwardIterator last,
    Compare comp);
\end{itemdecl}

\begin{itemdescr}
\pnum
\returns \tcode{is_sorted_until(first, last, comp) == last}
\end{itemdescr}
\end{removedblock}

\indexlibrary{\idxcode{is_sorted_until}}%
\begin{removedblock}
\begin{itemdecl}
template<class ForwardIterator>
  ForwardIterator is_sorted_until(ForwardIterator first, ForwardIterator last);
template<class ForwardIterator, class Compare>
  ForwardIterator is_sorted_until(ForwardIterator first, ForwardIterator last,
    Compare comp);
\end{itemdecl}
\end{removedblock}
\begin{addedblock}
\begin{itemdecl}
template<ForwardIterator I, Sentinel<I> S, class Proj = identity,
    IndirectCallableStrictWeakOrder<Projected<I, Proj>> Comp = less<>>
  I is_sorted_until(I first, S last, Comp comp = Comp{}, Proj proj = Proj{});

template<ForwardRange Rng, class Proj = identity,
    IndirectCallableStrictWeakOrder<Projected<IteratorType<Rng>, Proj>> Comp = less<>>
  @\oldtxt{IteratorType}\newtxt{safe_iterator_t}@<Rng>
    is_sorted_until(Rng&@\newtxt{\&}@ rng, Comp comp = Comp{}, Proj proj = Proj{});
\end{itemdecl}
\end{addedblock}

\begin{itemdescr}
\pnum
\returns If \tcode{distance(first, last) < 2}, returns
\tcode{last}. Otherwise, returns
the last iterator \tcode{i} in \crange{first}{last} for which the
range \range{first}{i} is sorted.

\pnum
\complexity Linear.
\end{itemdescr}

\rSec2[alg.nth.element]{Nth element}

\indexlibrary{\idxcode{nth_element}}%
\begin{removedblock}
\begin{itemdecl}
template<class RandomAccessIterator>
  void nth_element(RandomAccessIterator first, RandomAccessIterator nth,
                   RandomAccessIterator last);

template<class RandomAccessIterator, class Compare>
  void nth_element(RandomAccessIterator first, RandomAccessIterator nth,
                   RandomAccessIterator last,  Compare comp);
\end{itemdecl}
\end{removedblock}
\begin{addedblock}
\begin{itemdecl}
template<RandomAccessIterator I, Sentinel<I> S, class Comp = less<>,
    class Proj = identity>
  requires Sortable<I, Comp, Proj>@\newtxt{()}@
  I nth_element(I first, I nth, S last, Comp comp, Proj proj = Proj{});

template<RandomAccessRange Rng, class Comp = less<>, class Proj = identity>
  requires Sortable<IteratorType<Rng>, Comp, Proj>@\newtxt{()}@
  @\oldtxt{IteratorType}\newtxt{safe_iterator_t}@<Rng>
    nth_element(Rng&@\newtxt{\&}@ rng, IteratorType<Rng> nth, Comp comp, Proj proj = Proj{});
\end{itemdecl}
\end{addedblock}

\begin{itemdescr}
\pnum
After
\tcode{nth_element}
the element in the position pointed to by \tcode{nth}
is the element that would be
in that position if the whole range were sorted, unless \tcode{nth == last}.
Also for every iterator
\tcode{i}
in the range
\range{first}{nth}
and every iterator
\tcode{j}
in the range
\range{nth}{last}
it holds that:\changed{
\tcode{!(*j < *i)}
or
\tcode{comp(*j, *i) == false}}{
\tcode{\textit{INVOKE}(comp, \textit{INVOKE}(proj, *j), \textit{INVOKE}(proj, *i)) == false}}.

\begin{removedblock}
\pnum
\requires
\tcode{RandomAccessIterator} shall satisfy the requirements of
\tcode{ValueSwappable}~(\ref{concepts.lib.corelang.swappable}). The type
of \tcode{*first} shall satisfy the requirements of
\tcode{MoveConstructible} (Table~\cxxref{moveconstructible}) and of
\tcode{MoveAssignable} (Table~\cxxref{moveassignable}).
\end{removedblock}

\pnum
\complexity
Linear on average.
\end{itemdescr}

\rSec2[alg.binary.search]{Binary search}

\pnum
All of the algorithms in this section are versions of binary search
and assume that the sequence being searched is partitioned with respect to
an expression formed by binding the search key to an argument of the
\removed{implied or explicit} comparison function\added{ and projection}.
They work on non-random access iterators minimizing the number of comparisons,
which will be logarithmic for all types of iterators.
They are especially appropriate for random access iterators,
because these algorithms do a logarithmic number of steps
through the data structure.
For non-random access iterators they execute a linear number of steps.

\rSec3[lower.bound]{\tcode{lower_bound}}

\indexlibrary{\idxcode{lower_bound}}%
\begin{removedblock}
\begin{itemdecl}
template<class ForwardIterator, class T>
  ForwardIterator
    lower_bound(ForwardIterator first, ForwardIterator last,
                const T& value);

template<class ForwardIterator, class T, class Compare>
  ForwardIterator
    lower_bound(ForwardIterator first, ForwardIterator last,
                const T& value, Compare comp);
\end{itemdecl}
\end{removedblock}
\begin{addedblock}
\begin{itemdecl}
template<ForwardIterator I, Sentinel<I> S, TotallyOrdered T, class Proj = identity,
    IndirectCallableStrictWeakOrder<const T *, Projected<I, Proj>> Comp = less<>>
  I
    lower_bound(I first, S last, const T& value, Comp comp = Comp{},
                Proj proj = Proj{});

template<ForwardRange Rng, TotallyOrdered T, class Proj = identity,
    IndirectCallableStrictWeakOrder<const T *, Projected<IteratorType<Rng>, Proj>> Comp = less<>>
  @\oldtxt{IteratorType}\newtxt{safe_iterator_t}@<Rng>
    lower_bound(Rng&@\newtxt{\&}@ rng, const T& value, Comp comp = Comp{}, Proj proj = Proj{});
\end{itemdecl}
\end{addedblock}

\begin{itemdescr}
\pnum
\requires
The elements
\tcode{e}
of
\range{first}{last}
shall be partitioned with respect to the expression
\tcode{\changed{e < value}{\textit{INVOKE}(comp, \textit{INVOKE}(proj, e), value)}}
\removed{or
\tcode{comp(e, value)}}.

\pnum
\returns
The furthermost iterator
\tcode{i}
in the range
\crange{first}{last}
such that for every iterator
\tcode{j}
in the range
\range{first}{i}
the following corresponding conditions hold:
\removed{\tcode{*j < value}
or}
\tcode{\changed{comp(*j, value) != false}{\textit{INVOKE}(comp, \textit{INVOKE}(proj, *j), value) != false}}.

\pnum
\complexity
At most
$\log_2(\tcode{last - first}) + \bigoh{1}$
\changed{comparisons}{applications of the comparison function and projection}.
\end{itemdescr}

\rSec3[upper.bound]{\tcode{upper_bound}}

\indexlibrary{\idxcode{upper_bound}}%
\begin{removedblock}
\begin{itemdecl}
template<class ForwardIterator, class T>
  ForwardIterator
    upper_bound(ForwardIterator first, ForwardIterator last,
                const T& value);

template<class ForwardIterator, class T, class Compare>
  ForwardIterator
    upper_bound(ForwardIterator first, ForwardIterator last,
                const T& value, Compare comp);
\end{itemdecl}
\end{removedblock}
\begin{addedblock}
\begin{itemdecl}
template<ForwardIterator I, Sentinel<I> S, TotallyOrdered T, class Proj = identity,
    IndirectCallableStrictWeakOrder<const T *, Projected<I, Proj>> Comp = less<>>
  I
    upper_bound(I first, S last, const T& value, Comp comp = Comp{}, Proj proj = Proj{});

template<ForwardRange Rng, TotallyOrdered T, class Proj = identity,
    IndirectCallableStrictWeakOrder<const T *, Projected<IteratorType<Rng>, Proj>> Comp = less<>>
  @\oldtxt{IteratorType}\newtxt{safe_iterator_t}@<Rng>
    upper_bound(Rng&@\newtxt{\&}@ rng, const T& value, Comp comp = Comp{}, Proj proj = Proj{});
\end{itemdecl}
\end{addedblock}

\begin{itemdescr}
\pnum
\requires
The elements
\tcode{e}
of
\range{first}{last}
shall be partitioned with respect to the expression
\tcode{\changed{!(value < e)}{!\textit{INVOKE}(comp, value, \textit{INVOKE}(proj, e))}}
\removed{
or
\tcode{!comp(\brk{}value, e)}}.

\pnum
\returns
The furthermost iterator
\tcode{i}
in the range
\crange{first}{last}
such that for every iterator
\tcode{j}
in the range
\range{first}{i}
the following corresponding conditions hold:
\removed{\tcode{!(value < *j)}
or}
\tcode{\changed{comp(value, *j) == false}{\textit{INVOKE}(comp, value, \textit{INVOKE}(proj, *j)) == false}}.

\pnum
\complexity
At most
$\log_2(\tcode{last - first}) + \bigoh{1}$
\changed{comparisons}{applications of the comparison function and projection}.
\end{itemdescr}

\rSec3[equal.range]{\tcode{equal_range}}

\indexlibrary{\idxcode{equal_range}}%
\begin{removedblock}
\begin{itemdecl}
template<class ForwardIterator, class T>
  pair<ForwardIterator, ForwardIterator>
    equal_range(ForwardIterator first,
                ForwardIterator last, const T& value);

template<class ForwardIterator, class T, class Compare>
  pair<ForwardIterator, ForwardIterator>
    equal_range(ForwardIterator first,
                ForwardIterator last, const T& value,
                Compare comp);
\end{itemdecl}
\end{removedblock}
\begin{addedblock}
\begin{itemdecl}
template<ForwardIterator I, Sentinel<I> S, TotallyOrdered T, class Proj = identity,
    IndirectCallableStrictWeakOrder<const T *, Projected<I, Proj>> Comp = less<>>
  tagged_pair<tag::begin(I), tag::end(I)>
    equal_range(I first, S last, const T& value, Comp comp = Comp{}, Proj proj = Proj{});

template<ForwardRange Rng, TotallyOrdered T, class Proj = identity,
    IndirectCallableStrictWeakOrder<const T *, Projected<IteratorType<Rng>, Proj>> Comp = less<>>
  tagged_pair<tag::begin(@\oldtxt{IteratorType}\newtxt{safe_iterator_t}@<Rng>),
              tag::end(@\oldtxt{IteratorType}\newtxt{safe_iterator_t}@<Rng>)>
    equal_range(Rng&@\newtxt{\&}@ rng, const T& value, Comp comp = Comp{}, Proj proj = Proj{});
\end{itemdecl}
\end{addedblock}

\begin{itemdescr}
\pnum
\requires
The elements
\tcode{e}
of
\range{first}{last}
shall be partitioned with respect to the expressions
\tcode{\changed{e < value}{\textit{INVOKE}(comp, \textit{INVOKE}(proj, e), value)}}
and
\tcode{\changed{!(value < e)}{!\textit{INVOKE}(comp, value, \textit{INVOKE}(proj, e))}}
\removed{or
\tcode{comp(e, value)}
and
\tcode{!comp(value, e)}}.
Also, for all elements
\tcode{e}
of
\tcode{[first, last)},
\tcode{\changed{e < value}{\textit{INVOKE}(comp, \textit{INVOKE}(proj, e), value)}}
shall imply
\tcode{\changed{!(value < e)}{!\textit{INVOKE}(comp, value, \textit{INVOKE}(proj, e))}}
\removed{or
\tcode{comp(e, value)}
shall imply
\tcode{!comp(value, e)}}.

\pnum
\returns
\begin{removedblock}
\begin{codeblock}
make_pair(lower_bound(first, last, value),
          upper_bound(first, last, value))
\end{codeblock}
or
\end{removedblock}
\begin{codeblock}
@\changed{make_pair(}{\{}@lower_bound(first, last, value, comp@\added{, proj}@),
           upper_bound(first, last, value, comp@\added{, proj}@)@\changed{)}{\}}@
\end{codeblock}

\pnum
\complexity
At most
$2 * \log_2(\tcode{last - first}) + \bigoh{1}$
\changed{comparisons}{applications of the comparison function and projection}.
\end{itemdescr}

\rSec3[binary.search]{\tcode{binary_search}}

\indexlibrary{\idxcode{binary_search}}%
\begin{removedblock}
\begin{itemdecl}
template<class ForwardIterator, class T>
  bool binary_search(ForwardIterator first, ForwardIterator last,
                     const T& value);

template<class ForwardIterator, class T, class Compare>
  bool binary_search(ForwardIterator first, ForwardIterator last,
                     const T& value, Compare comp);
\end{itemdecl}
\end{removedblock}
\begin{addedblock}
\begin{itemdecl}
template<ForwardIterator I, Sentinel<I> S, TotallyOrdered T, class Proj = identity,
    IndirectCallableStrictWeakOrder<const T *, Projected<I, Proj>> Comp = less<>>
  bool
    binary_search(I first, S last, const T& value, Comp comp = Comp{},
                  Proj proj = Proj{});

template<ForwardRange Rng, TotallyOrdered T, class Proj = identity,
    IndirectCallableStrictWeakOrder<const T *, Projected<IteratorType<Rng>, Proj>> Comp = less<>>
  bool
    binary_search(Rng&@\newtxt{\&}@ rng, const T& value, Comp comp = Comp{},
                  Proj proj = Proj{});
\end{itemdecl}
\end{addedblock}

\begin{itemdescr}
\pnum
\requires
The elements
\tcode{e}
of
\range{first}{last}
are partitioned with respect to the expressions
\tcode{\changed{e < value}{\textit{INVOKE}(comp, \textit{INVOKE}(proj, e), value)}}
and
\tcode{\changed{!(value < e)}{!\textit{INVOKE}(comp, value, \textit{INVOKE}(proj, e))}}
\removed{or
\tcode{comp(e, value)}
and
\tcode{!comp(value, e)}}.
Also, for all elements
\tcode{e}
of
\tcode{[first, last)},
\tcode{\changed{e < value}{\textit{INVOKE}(comp, \textit{INVOKE}(proj, e), value)}}
shall imply
\tcode{\changed{!(value < e)}{!\textit{INVOKE}(comp, value, \textit{INVOKE}(proj, e))}}
\removed{or
\tcode{comp(e, value)}
shall imply
\tcode{!comp(value, e)}}.

\pnum
\returns
\tcode{true}
if there is an iterator
\tcode{i}
in the range
\range{first}{last}
that satisfies the corresponding conditions:
\tcode{\changed{!(*i < value) \&\& !(value < *i)}{
\textit{INVOKE}(comp, \textit{INVOKE}(proj, *i), value) == false \&\&
\textit{INVOKE}(comp, value, \textit{INVOKE}(proj, *i)) == false}}
\removed{or}
\tcode{\removed{comp(*i, value) == false \&\& comp(value, *i) == false}}.

\pnum
\complexity
At most
$\log_2(\tcode{last - first}) + \bigoh{1}$
\changed{comparisons}{applications of the comparison function and projection}.
\end{itemdescr}

\rSec2[alg.merge]{Merge}

\indexlibrary{\idxcode{merge}}%
\begin{removedblock}
\begin{itemdecl}
template<class InputIterator1, class InputIterator2,
         class OutputIterator>
  OutputIterator
    merge(InputIterator1 first1, InputIterator1 last1,
          InputIterator2 first2, InputIterator2 last2,
          OutputIterator result);

template<class InputIterator1, class InputIterator2,
         class OutputIterator, class Compare>
  OutputIterator
    merge(InputIterator1 first1, InputIterator1 last1,
          InputIterator2 first2, InputIterator2 last2,
          OutputIterator result, Compare comp);
\end{itemdecl}
\end{removedblock}
\begin{addedblock}
\begin{itemdecl}
template<InputIterator I1, Sentinel<I1> S1, InputIterator I2, Sentinel<I2> S2,
    Incrementable O, class Comp = less<>, class Proj1 = identity,
    class Proj2 = identity>
  requires Mergeable<I1, I2, O, Comp, Proj1, Proj2>@\newtxt{()}@
  tagged_tuple<tag::in1(I1), tag::in2(I2), tag::out(O)>
    merge(I1 first1, S1 last1, I2 first2, S2 last2, O result,
          Comp comp = Comp{}, Proj1 proj1 = Proj1{}, Proj2 proj2 = Proj2{});

template<InputRange Rng1, InputRange Rng2, Incrementable O, class Comp = less<>,
    class Proj1 = identity, class Proj2 = identity>
  requires Mergeable<IteratorType<Rng1>, IteratorType<Rng2>, O, Comp, Proj1, Proj2>@\newtxt{()}@
  tagged_tuple<tag::in1(@\oldtxt{IteratorType}\newtxt{safe_iterator_t}@<Rng1>),
               tag::in2(@\oldtxt{IteratorType}\newtxt{safe_iterator_t}@<Rng2>),
               tag::out(O)>
    merge(Rng1&@\newtxt{\&}@ rng1, Rng2&@\newtxt{\&}@ rng2, O result,
          Comp comp = Comp{}, Proj1 proj1 = Proj1{}, Proj2 proj2 = Proj2{});
\end{itemdecl}
\end{addedblock}

\begin{itemdescr}
\pnum
\effects\ Copies all the elements of the two ranges \range{first1}{last1} and
\range{first2}{last2} into the range \range{result}{result_last}, where \tcode{result_last}
is \tcode{result + (last1 - first1) + (last2 - first2)}, such that the resulting range satisfies
\tcode{\removed{is_sorted(result, result_last)}}\removed{ or}
\ednote{TODO The following postcondition isn't well-formed:}
\tcode{is_sorted(result, result_last, comp)}\removed{, respectively}.

\pnum
\requires The ranges \range{first1}{last1} and \range{first2}{last2} shall be
sorted with respect to \tcode{\removed{operator<}}\removed{ or} \tcode{comp}\added{, \tcode{proj1}, and \tcode{proj2}}.
The resulting range shall not overlap with either of the original ranges.

\pnum
\returns
\tcode{\removed{result + (last1 - first1) + (last2 - first2)}\added{
make_tagged_tuple<tag::in1, tag::in2,}}{\\}\tcode{\added{tag::out>(last1, last2, result + (last1 - first1) + (last2 - first2))}}.

\pnum
\complexity
At most
\tcode{(last1 - first1) + (last2 - first2) - 1}
\changed{comparisons}{applications of the comparison function and each projection}.

\pnum
\remarks Stable~(\cxxref{algorithm.stable}).
\end{itemdescr}

\begin{addedblock}
\indexlibrary{\idxcode{merge_move}}%
\begin{itemdecl}
template<InputIterator I1, Sentinel<I1> S1, InputIterator I2, Sentinel<I2> S2,
    Incrementable O, class Comp = less<>, class Proj1 = identity,
    class Proj2 = identity>
  requires MergeMovable<I1, I2, O, Comp, Proj1, Proj2>@\newtxt{()}@
  tagged_tuple<tag::in1(I1), tag::in2(I2), tag::out(O)>
    merge_move(I1 first1, S1 last1, I2 first2, S2 last2, O result,
               Comp comp = Comp{}, Proj1 proj1 = Proj1{}, Proj2 proj2 = Proj2{});

template<InputRange Rng1, InputRange Rng2, Incrementable O, class Comp = less<>,
    class Proj1 = identity, class Proj2 = identity>
  requires MergeMovable<IteratorType<Rng1>, IteratorType<Rng2>, O, Comp, Proj1, Proj2>@\newtxt{()}@
  tagged_tuple<tag::in1(@\oldtxt{IteratorType}\newtxt{safe_iterator_t}@<Rng1>),
               tag::in2(@\oldtxt{IteratorType}\newtxt{safe_iterator_t}@<Rng2>),
               tag::out(O)>
    merge_move(Rng1&@\newtxt{\&}@ rng1, Rng2&@\newtxt{\&}@ rng2, O result,
               Comp comp = Comp{}, Proj1 proj1 = Proj1{}, Proj2 proj2 = Proj2{});
\end{itemdecl}

\begin{itemdescr}
\pnum
\effects\ Moves all the elements of the two ranges \range{first1}{last1} and
\range{first2}{last2} into the range \range{result}{result_last}, where \tcode{result_last}
is \tcode{result + (last1 - first1) + (last2 - first2)}, such that the resulting range satisfies
\ednote{TODO The following postcondition isn't well-formed:}
\tcode{is_sorted(result, result_last, comp)}.

\pnum
\requires The ranges \range{first1}{last1} and \range{first2}{last2} shall be
sorted with respect to \tcode{comp}, \tcode{proj1}, and \tcode{proj2}.
The resulting range shall not overlap with either of the original ranges.

\pnum
\returns
\tcode{make_tagged_tuple<tag::in1, tag::in2, tag::out>(last1, last2, result + (last1 - first1) + (last2 - first2))}.

\pnum
\complexity
At most
\tcode{(last1 - first1) + (last2 - first2) - 1}
applications of the comparison function and each projection.

\pnum
\remarks Stable~(\cxxref{algorithm.stable}).
\end{itemdescr}
\end{addedblock}

\indexlibrary{\idxcode{inplace_merge}}%
\begin{removedblock}
\begin{itemdecl}
template<class BidirectionalIterator>
  void inplace_merge(BidirectionalIterator first,
                     BidirectionalIterator middle,
                     BidirectionalIterator last);

template<class BidirectionalIterator, class Compare>
  void inplace_merge(BidirectionalIterator first,
                     BidirectionalIterator middle,
                     BidirectionalIterator last, Compare comp);
\end{itemdecl}
\end{removedblock}
\begin{addedblock}
\begin{itemdecl}
template<BidirectionalIterator I, Sentinel<I> S, class Comp = less<>,
    class Proj = identity>
  requires Sortable<I, Comp, Proj>@\newtxt{()}@
  I
    inplace_merge(I first, I middle, S last, Comp comp = Comp{}, Proj proj = Proj{});

template<BidirectionalRange Rng, class Comp = less<>, class Proj = identity>
  requires Sortable<IteratorType<Rng>, Comp, Proj>@\newtxt{()}@
  @\oldtxt{IteratorType}\newtxt{safe_iterator_t}@<Rng>
    inplace_merge(Rng&@\newtxt{\&}@ rng, IteratorType<Rng> middle, Comp comp = Comp{},
                  Proj proj = Proj{});
\end{itemdecl}
\end{addedblock}

\begin{itemdescr}
\pnum
\effects
Merges two sorted consecutive ranges
\range{first}{middle}
and
\range{middle}{last},
putting the result of the merge into the range
\range{first}{last}.
The resulting range will be in non-decreasing order;
that is, for every iterator
\tcode{i}
in
\range{first}{last}
other than
\tcode{first},
the condition
\removed{\tcode{*i < *(i - 1)}
or, respectively,}
\tcode{\changed{comp(*i, *(i - 1))}{
\textit{INVOKE}(comp, \textit{INVOKE}(proj, *i), \textit{INVOKE}(proj, *(i - 1)))}}
will be false.

\pnum
\requires
The ranges \range{first}{middle} and \range{middle}{last} shall be
sorted with respect to \tcode{\removed{operator<}}\removed{ or} \tcode{comp}\added{ and
\tcode{proj}}.
\removed{\tcode{BidirectionalIterator} shall satisfy the requirements of
\tcode{ValueSwappable}~(\ref{concepts.lib.corelang.swappable}). The type
of \tcode{*first} shall satisfy the requirements of
\tcode{MoveConstructible} (Table~\cxxref{moveconstructible}) and of
\tcode{MoveAssignable} (Table~\cxxref{moveassignable}).}

\begin{addedblock}
\pnum
\returns \tcode{last}
\end{addedblock}

\pnum
\complexity
When enough additional memory is available,
\tcode{(last - first) - 1}
\changed{comparisons}{applications of the comparison function and projection}.
If no additional memory is available, an algorithm with complexity
$N \log(N)$
(where
\tcode{N}
is equal to
\tcode{last - first})
may be used.

\pnum
\remarks Stable~(\cxxref{algorithm.stable}).
\end{itemdescr}

\rSec2[alg.set.operations]{Set operations on sorted structures}

\pnum
This section defines all the basic set operations on sorted structures.
They also work with
\tcode{multiset}s~(\cxxref{multiset})
containing multiple copies of equivalent elements.
The semantics of the set operations are generalized to
\tcode{multiset}s
in a standard way by defining
\tcode{set_union()}
to contain the maximum number of occurrences of every element,
\tcode{set_intersection()}
to contain the minimum, and so on.

\rSec3[includes]{\tcode{includes}}

\indexlibrary{\idxcode{includes}}%
\begin{removedblock}
\begin{itemdecl}
template<class InputIterator1, class InputIterator2>
  bool includes(InputIterator1 first1, InputIterator1 last1,
                InputIterator2 first2, InputIterator2 last2);

template<class InputIterator1, class InputIterator2, class Compare>
  bool includes(InputIterator1 first1, InputIterator1 last1,
                InputIterator2 first2, InputIterator2 last2,
                Compare comp);
\end{itemdecl}
\end{removedblock}
\begin{addedblock}
\begin{itemdecl}
template<InputIterator I1, Sentinel<I1> S1, InputIterator I2, Sentinel<I2> S2,
    class Proj1 = identity, class Proj2 = identity,
    IndirectCallableStrictWeakOrder<Projected<I1, Proj1>, Projected<I2, Proj2>> Comp = less<>>
  bool
    includes(I1 first1, S1 last1, I2 first2, S2 last2, Comp comp = Comp{},
             Proj1 proj1 = Proj1{}, Proj2 proj2 = Proj2{});

template<InputRange Rng1, InputRange Rng2, class Proj1 = identity,
    class Proj2 = identity,
    IndirectCallableStrictWeakOrder<Projected<IteratorType<Rng1>, Proj1>,
      Projected<IteratorType<Rng2>, Proj2>> Comp = less<>>
  bool
    includes(Rng1&& rng1, Rng2&& rng2, Comp comp = Comp{},
             Proj1 proj1 = Proj1{}, Proj2 proj2 = Proj2{});
\end{itemdecl}
\end{addedblock}

\begin{itemdescr}
\pnum
\returns
\tcode{true}
if \range{first2}{last2} is empty or
if every element in the range
\range{first2}{last2}
is contained in the range
\range{first1}{last1}.
Returns
\tcode{false}
otherwise.

\pnum
\complexity
At most
\tcode{2 * ((last1 - first1) + (last2 - first2)) - 1}
\changed{comparisons}{applications of the comparison function and projections}.
\end{itemdescr}

\rSec3[set.union]{\tcode{set_union}}

\indexlibrary{\idxcode{set_union}}%
\begin{removedblock}
\begin{itemdecl}
template<class InputIterator1, class InputIterator2,
         class OutputIterator>
  OutputIterator
    set_union(InputIterator1 first1, InputIterator1 last1,
              InputIterator2 first2, InputIterator2 last2,
              OutputIterator result);

template<class InputIterator1, class InputIterator2,
         class OutputIterator, class Compare>
  OutputIterator
    set_union(InputIterator1 first1, InputIterator1 last1,
              InputIterator2 first2, InputIterator2 last2,
              OutputIterator result, Compare comp);
\end{itemdecl}
\end{removedblock}
\begin{addedblock}
\begin{itemdecl}
template<InputIterator I1, Sentinel<I1> S1, InputIterator I2, Sentinel<I2> S2,
    WeaklyIncrementable O, class Comp = less<>, class Proj1 = identity, class Proj2 = identity>
  requires Mergeable<I1, I2, O, Comp, Proj1, Proj2>@\newtxt{()}@
  tagged_tuple<tag::in1(I1), tag::in2(I2), tag::out(O)>
    set_union(I1 first1, S1 last1, I2 first2, S2 last2, O result, Comp comp = Comp{},
              Proj1 proj1 = Proj1{}, Proj2 proj2 = Proj2{});

template<InputRange Rng1, InputRange Rng2, WeaklyIncrementable O,
    class Comp = less<>, class Proj1 = identity, class Proj2 = identity>
  requires Mergeable<IteratorType<Rng1>, IteratorType<Rng2>, O, Comp, Proj1, Proj2>@\newtxt{()}@
  tagged_tuple<tag::in1(@\oldtxt{IteratorType}\newtxt{safe_iterator_t}@<Rng1>),
               tag::in2(@\oldtxt{IteratorType}\newtxt{safe_iterator_t}@<Rng2>),
               tag::out(O)>
    set_union(Rng1&@\newtxt{\&}@ rng1, Rng2&@\newtxt{\&}@ rng2, O result, Comp comp = Comp{},
              Proj1 proj1 = Proj1{}, Proj2 proj2 = Proj2{});
\end{itemdecl}
\end{addedblock}

\begin{itemdescr}
\pnum
\effects
Constructs a sorted union of the elements from the two ranges;
that is, the set of elements that are present in one or both of the ranges.

\pnum
\requires
The resulting range shall not overlap with either of the original ranges.

\pnum
\returns
\removed{The end of the constructed range}
\tcode{\added{make_tagged_tuple<tag::in1, tag::in2, tag::out>(last1, last2, result + $n$)}}\added{, where \tcode{$n$} is
the number of elements in the constructed range}.

\pnum
\complexity
At most
\tcode{2 * ((last1 - first1) + (last2 - first2)) - 1}
\changed{comparisons}{applications of the comparison function and projections}.

\pnum
\notes If \range{first1}{last1} contains $m$ elements that are equivalent to
each other and \range{first2}{last2} contains $n$ elements that are equivalent
to them, then all $m$ elements from the first range shall be copied to the output
range, in order, and then $\max(n - m, 0)$ elements from the second range shall
be copied to the output range, in order.
\end{itemdescr}

\rSec3[set.intersection]{\tcode{set_intersection}}

\indexlibrary{\idxcode{set_intersection}}%
\begin{removedblock}
\begin{itemdecl}
template<class InputIterator1, class InputIterator2,
         class OutputIterator>
  OutputIterator
    set_intersection(InputIterator1 first1, InputIterator1 last1,
                     InputIterator2 first2, InputIterator2 last2,
                     OutputIterator result);

template<class InputIterator1, class InputIterator2,
         class OutputIterator, class Compare>
  OutputIterator
    set_intersection(InputIterator1 first1, InputIterator1 last1,
                     InputIterator2 first2, InputIterator2 last2,
                     OutputIterator result, Compare comp);
\end{itemdecl}
\end{removedblock}
\begin{addedblock}
\begin{itemdecl}
template<InputIterator I1, Sentinel<I1> S1, InputIterator I2, Sentinel<I2> S2,
    WeaklyIncrementable O, class Comp = less<>, class Proj1 = identity, class Proj2 = identity>
  requires Mergeable<I1, I2, O, Comp, Proj1, Proj2>@\newtxt{()}@
  O
    set_intersection(I1 first1, S1 last1, I2 first2, S2 last2, O result,
                     Comp comp = Comp{}, Proj1 proj1 = Proj1{}, Proj2 proj2 = Proj2{});

template<InputRange Rng1, InputRange Rng2, WeaklyIncrementable O,
    class Comp = less<>, class Proj1 = identity, class Proj2 = identity>
  requires Mergeable<IteratorType<Rng1>, IteratorType<Rng2>, O, Comp, Proj1, Proj2>@\newtxt{()}@
  O
    set_intersection(Rng1&& rng1, Rng2&& rng2, O result,
                     Comp comp = Comp{}, Proj1 proj1 = Proj1{}, Proj2 proj2 = Proj2{});
\end{itemdecl}
\end{addedblock}

\begin{itemdescr}
\pnum
\effects
Constructs a sorted intersection of the elements from the two ranges;
that is, the set of elements that are present in both of the ranges.

\pnum
\requires
The resulting range shall not overlap with either of the original ranges.

\pnum
\returns
The end of the constructed range.

\pnum
\complexity
At most
\tcode{2 * ((last1 - first1) + (last2 - first2)) - 1}
\changed{comparisons}{applications of the comparison function and projections}.

\pnum
\notes If \range{first1}{last1} contains $m$ elements that are equivalent to
each other and \range{first2}{last2} contains $n$ elements that are equivalent
to them, the first $\min(m, n)$ elements shall be copied from the first range
to the output range, in order.
\end{itemdescr}

\rSec3[set.difference]{\tcode{set_difference}}

\indexlibrary{\idxcode{set_difference}}%
\begin{removedblock}
\begin{itemdecl}
template<class InputIterator1, class InputIterator2,
         class OutputIterator>
  OutputIterator
    set_difference(InputIterator1 first1, InputIterator1 last1,
                   InputIterator2 first2, InputIterator2 last2,
                   OutputIterator result);

template<class InputIterator1, class InputIterator2,
         class OutputIterator, class Compare>
  OutputIterator
    set_difference(InputIterator1 first1, InputIterator1 last1,
                   InputIterator2 first2, InputIterator2 last2,
                   OutputIterator result, Compare comp);
\end{itemdecl}
\end{removedblock}
\begin{addedblock}
\begin{itemdecl}
template<InputIterator I1, Sentinel<I1> S1, InputIterator I2, Sentinel<I2> S2,
    WeaklyIncrementable O, class Comp = less<>, class Proj1 = identity, class Proj2 = identity>
  requires Mergeable<I1, I2, O, Comp, Proj1, Proj2>@\newtxt{()}@
  tagged_pair<tag::in1(I1), tag::out(O)>
    set_difference(I1 first1, S1 last1, I2 first2, S2 last2, O result,
                   Comp comp = Comp{}, Proj1 proj1 = Proj1{}, Proj2 proj2 = Proj2{});

template<InputRange Rng1, InputRange Rng2, WeaklyIncrementable O,
    class Comp = less<>, class Proj1 = identity, class Proj2 = identity>
  requires Mergeable<IteratorType<Rng1>, IteratorType<Rng2>, O, Comp, Proj1, Proj2>@\newtxt{()}@
  tagged_pair<tag::in1(@\oldtxt{IteratorType}\newtxt{safe_iterator_t}@<Rng1>), tag::out(O)>
    set_difference(Rng1&@\newtxt{\&}@ rng1, Rng2&& rng2, O result,
                   Comp comp = Comp{}, Proj1 proj1 = Proj1{}, Proj2 proj2 = Proj2{});
\end{itemdecl}
\end{addedblock}

\begin{itemdescr}
\pnum
\effects
Copies the elements of the range
\range{first1}{last1}
which are not present in the range
\range{first2}{last2}
to the range beginning at
\tcode{result}.
The elements in the constructed range are sorted.

\pnum
\requires
The resulting range shall not overlap with either of the original ranges.

\pnum
\returns
\removed{The end of the constructed range}\tcode{\added{\{last1, result + $n$\}}}\added{, where $n$
is the number of elements in the constructed range}.

\pnum
\complexity
At most
\tcode{2 * ((last1 - first1) + (last2 - first2)) - 1}
\changed{comparisons}{applications of the comparison function and projections}.

\pnum
\notes
If
\range{first1}{last1}
contains $m$
elements that are equivalent to each other and
\range{first2}{last2}
contains $n$
elements that are equivalent to them, the last
$\max(m - n, 0)$
elements from
\range{first1}{last1}
shall be copied to the output range.
\end{itemdescr}

\rSec3[set.symmetric.difference]{\tcode{set_symmetric_difference}}

\indexlibrary{\idxcode{set_symmetric_difference}}%
\begin{removedblock}
\begin{itemdecl}
template<class InputIterator1, class InputIterator2,
         class OutputIterator>
  OutputIterator
    set_symmetric_difference(InputIterator1 first1, InputIterator1 last1,
                             InputIterator2 first2, InputIterator2 last2,
                             OutputIterator result);

template<class InputIterator1, class InputIterator2,
         class OutputIterator, class Compare>
  OutputIterator
    set_symmetric_difference(InputIterator1 first1, InputIterator1 last1,
                             InputIterator2 first2, InputIterator2 last2,
                             OutputIterator result, Compare comp);
\end{itemdecl}
\end{removedblock}
\begin{addedblock}
\begin{itemdecl}
template<InputIterator I1, Sentinel<I1> S1, InputIterator I2, Sentinel<I2> S2,
    WeaklyIncrementable O, class Comp = less<>, class Proj1 = identity, class Proj2 = identity>
  requires Mergeable<I1, I2, O, Comp, Proj1, Proj2>@\newtxt{()}@
  tagged_tuple<tag::in1(I1), tag::in2(I2), tag::out(O)>
    set_symmetric_difference(I1 first1, S1 last1, I2 first2, S2 last2, O result,
                             Comp comp = Comp{}, Proj1 proj1 = Proj1{},
                             Proj2 proj2 = Proj2{});

template<InputRange Rng1, InputRange Rng2, WeaklyIncrementable O,
    class Comp = less<>, class Proj1 = identity, class Proj2 = identity>
  requires Mergeable<IteratorType<Rng1>, IteratorType<Rng2>, O, Comp, Proj1, Proj2>@\newtxt{()}@
  tagged_tuple<tag::in1(@\oldtxt{IteratorType}\newtxt{safe_iterator_t}@<Rng1>),
               tag::in2(@\oldtxt{IteratorType}\newtxt{safe_iterator_t}@<Rng2>),
               tag::out(O)>
    set_symmetric_difference(Rng1&@\newtxt{\&}@ rng1, Rng2&@\newtxt{\&}@ rng2, O result, Comp comp = Comp{},
                             Proj1 proj1 = Proj1{}, Proj2 proj2 = Proj2{});
\end{itemdecl}
\end{addedblock}

\begin{itemdescr}
\pnum
\effects
Copies the elements of the range
\range{first1}{last1}
that are not present in the range
\range{first2}{last2},
and the elements of the range
\range{first2}{last2}
that are not present in the range
\range{first1}{last1}
to the range beginning at
\tcode{result}.
The elements in the constructed range are sorted.

\pnum
\requires
The resulting range shall not overlap with either of the original ranges.

\pnum
\returns
\removed{The end of the constructed range}
\tcode{\added{make_tagged_tuple<tag::in1, tag::in2, tag::out>(last1, last2, result + $n$)}}\added{, where \tcode{$n$} is
the number of elements in the constructed range}.

\pnum
\complexity
At most
\tcode{2 * ((last1 - first1) + (last2 - first2)) - 1}
\changed{comparisons}{applications of the comparison function and projections}.

\pnum
\notes
If \range{first1}{last1} contains $m$ elements that are equivalent to each other and
\range{first2}{last2} contains $n$ elements that are equivalent to them, then
$|m - n|$ of those elements shall be copied to the output range: the last
$m - n$ of these elements from \range{first1}{last1} if $m > n$, and the last
$n - m$ of these elements from \range{first2}{last2} if $m < n$.
\end{itemdescr}

\rSec2[alg.heap.operations]{Heap operations}

\pnum
A
\techterm{heap}
is a particular organization of elements in a range between two random access iterators
\range{a}{b}.
Its two key properties are:

\begin{description}
\item{(1)} There is no element greater than
\tcode{*a}
in the range and
\item{(2)} \tcode{*a}
may be removed by
\tcode{pop_heap()},
or a new element added by
\tcode{push_heap()},
in
$\mathcal{O}(\log(N))$
time.
\end{description}

\pnum
These properties make heaps useful as priority queues.

\pnum
\tcode{make_heap()}
converts a range into a heap and
\tcode{sort_heap()}
turns a heap into a sorted sequence.

\rSec3[push.heap]{\tcode{push_heap}}

\indexlibrary{\idxcode{push_heap}}%
\begin{removedblock}
\begin{itemdecl}
template<class RandomAccessIterator>
  void push_heap(RandomAccessIterator first, RandomAccessIterator last);

template<class RandomAccessIterator, class Compare>
  void push_heap(RandomAccessIterator first, RandomAccessIterator last,
                 Compare comp);
\end{itemdecl}
\end{removedblock}
\begin{addedblock}
\begin{itemdecl}
template<RandomAccessIterator I, Sentinel<I> S, class Comp = less<>,
    class Proj = identity>
  requires Sortable<I, Comp, Proj>@\newtxt{()}@
  I push_heap(I first, S last, Comp comp = Comp{}, Proj proj = Proj{});

template<RandomAccessRange Rng, class Comp = less<>, class Proj = identity>
  requires Sortable<IteratorType<Rng>, Comp, Proj>@\newtxt{()}@
  @\oldtxt{IteratorType}\newtxt{safe_iterator_t}@<Rng>
    push_heap(Rng&@\newtxt{\&}@ rng, Comp comp = Comp{}, Proj proj = Proj{});
\end{itemdecl}
\end{addedblock}

\begin{itemdescr}
\pnum
\effects
Places the value in the location
\tcode{last - 1}
into the resulting heap
\range{first}{last}.

\pnum
\requires
The range
\range{first}{last - 1}
shall be a valid heap.
\removed{The type of \tcode{*first} shall satisfy
the \tcode{MoveConstructible} requirements
(Table~\cxxref{moveconstructible}) and the
\tcode{MoveAssignable} requirements
(Table~\cxxref{moveassignable})}.

\begin{addedblock}
\pnum
\returns \tcode{last}
\end{addedblock}

\pnum
\complexity
At most
\tcode{log(last - first)}
\changed{comparisons}{applications of the comparison function and projection}.
\end{itemdescr}

\rSec3[pop.heap]{\tcode{pop_heap}}

\indexlibrary{\idxcode{pop_heap}}%
\begin{removedblock}
\begin{itemdecl}
template<class RandomAccessIterator>
  void pop_heap(RandomAccessIterator first, RandomAccessIterator last);

template<class RandomAccessIterator, class Compare>
  void pop_heap(RandomAccessIterator first, RandomAccessIterator last,
                Compare comp);
\end{itemdecl}
\end{removedblock}
\begin{addedblock}
\begin{itemdecl}
template<RandomAccessIterator I, Sentinel<I> S, class Comp = less<>,
    class Proj = identity>
  requires Sortable<I, Comp, Proj>@\newtxt{()}@
  I pop_heap(I first, S last, Comp comp = Comp{}, Proj proj = Proj{});

template<RandomAccessRange Rng, class Comp = less<>, class Proj = identity>
  requires Sortable<IteratorType<Rng>, Comp, Proj>@\newtxt{()}@
  @\oldtxt{IteratorType}\newtxt{safe_iterator_t}@<Rng>
    pop_heap(Rng&@\newtxt{\&}@ rng, Comp comp = Comp{}, Proj proj = Proj{});
\end{itemdecl}
\end{addedblock}

\begin{itemdescr}
\pnum
\requires
The range
\range{first}{last}
shall be a valid non-empty heap.
\removed{\tcode{RandomAccessIterator} shall satisfy the requirements of
\tcode{ValueSwappable}~(\ref{concepts.lib.corelang.swappable}). The type
of \tcode{*first} shall satisfy the requirements of
\tcode{MoveConstructible} (Table~\cxxref{moveconstructible}) and of
\tcode{MoveAssignable} (Table~\cxxref{moveassignable}).}

\pnum
\effects
Swaps the value in the location \tcode{first}
with the value in the location
\tcode{last - 1}
and makes
\range{first}{last - 1}
into a heap.

\begin{addedblock}
\pnum
\returns \tcode{last}
\end{addedblock}

\pnum
\complexity
At most
\tcode{2 * log(last - first)}
\changed{comparisons}{applications of the comparison function and projection}.
\end{itemdescr}

\rSec3[make.heap]{\tcode{make_heap}}

\indexlibrary{\idxcode{make_heap}}%
\begin{removedblock}
\begin{itemdecl}
template<class RandomAccessIterator>
  void make_heap(RandomAccessIterator first, RandomAccessIterator last);

template<class RandomAccessIterator, class Compare>
  void make_heap(RandomAccessIterator first, RandomAccessIterator last,
                 Compare comp);
\end{itemdecl}
\end{removedblock}
\begin{addedblock}
\begin{itemdecl}
template<RandomAccessIterator I, Sentinel<I> S, class Comp = less<>,
    class Proj = identity>
  requires Sortable<I, Comp, Proj>@\newtxt{()}@
  I make_heap(I first, S last, Comp comp = Comp{}, Proj proj = Proj{});

template<RandomAccessRange Rng, class Comp = less<>, class Proj = identity>
  requires Sortable<IteratorType<Rng>, Comp, Proj>@\newtxt{()}@
  @\oldtxt{IteratorType}\newtxt{safe_iterator_t}@<Rng>
    make_heap(Rng&@\newtxt{\&}@ rng, Comp comp = Comp{}, Proj proj = Proj{});
\end{itemdecl}
\end{addedblock}

\begin{itemdescr}
\pnum
\effects
Constructs a heap out of the range
\range{first}{last}.

\begin{removedblock}
\pnum
\requires The type of \tcode{*first} shall satisfy
the \tcode{MoveConstructible} requirements
(Table~\cxxref{moveconstructible}) and the
\tcode{MoveAssignable} requirements
(Table~\cxxref{moveassignable}).
\end{removedblock}

\begin{addedblock}
\pnum
\returns \tcode{last}
\end{addedblock}

\pnum
\complexity
At most
\tcode{3 * (last - first)}
\changed{comparisons}{applications of the comparison function and projection}.
\end{itemdescr}

\rSec3[sort.heap]{\tcode{sort_heap}}

\indexlibrary{\idxcode{sort_heap}}%
\begin{removedblock}
\begin{itemdecl}
template<class RandomAccessIterator>
  void sort_heap(RandomAccessIterator first, RandomAccessIterator last);

template<class RandomAccessIterator, class Compare>
  void sort_heap(RandomAccessIterator first, RandomAccessIterator last,
                 Compare comp);
\end{itemdecl}
\end{removedblock}
\begin{addedblock}
\begin{itemdecl}
template<RandomAccessIterator I, Sentinel<I> S, class Comp = less<>,
    class Proj = identity>
  requires Sortable<I, Comp, Proj>@\newtxt{()}@
  I sort_heap(I first, S last, Comp comp = Comp{}, Proj proj = Proj{});

template<RandomAccessRange Rng, class Comp = less<>, class Proj = identity>
  requires Sortable<IteratorType<Rng>, Comp, Proj>@\newtxt{()}@
  @\oldtxt{IteratorType}\newtxt{safe_iterator_t}@<Rng>
    sort_heap(Rng&@\newtxt{\&}@ rng, Comp comp = Comp{}, Proj proj = Proj{});
\end{itemdecl}
\end{addedblock}

\begin{itemdescr}
\pnum
\effects
Sorts elements in the heap
\range{first}{last}.

\pnum
\requires The range \range{first}{last} shall be a valid heap.
\removed{\tcode{RandomAccessIterator} shall satisfy the requirements of
\tcode{ValueSwappable}~(\ref{concepts.lib.corelang.swappable}). The type
of \tcode{*first} shall satisfy the requirements of
\tcode{MoveConstructible} (Table~\cxxref{moveconstructible}) and of
\tcode{MoveAssignable} (Table~\cxxref{moveassignable}).}

\begin{addedblock}
\pnum
\returns \tcode{last}
\end{addedblock}

\pnum
\complexity
At most $N \log(N)$
comparisons (where
\tcode{N == last - first}).
\end{itemdescr}

\rSec3[is.heap]{\tcode{is_heap}}

\indexlibrary{\idxcode{is_heap}}%
\begin{removedblock}
\begin{itemdecl}
  template<class RandomAccessIterator>
    bool is_heap(RandomAccessIterator first, RandomAccessIterator last);
\end{itemdecl}
\end{removedblock}
\begin{addedblock}
\begin{itemdecl}
template<RandomAccessIterator I, Sentinel<I> S, class Proj = identity,
    IndirectCallableStrictWeakOrder<Projected<I, Proj>> Comp = less<>>
  bool is_heap(I first, S last, Comp comp = Comp{}, Proj proj = Proj{});

template<RandomAccessRange Rng, class Proj = identity,
    IndirectCallableStrictWeakOrder<Projected<IteratorType<Rng>, Proj>> Comp = less<>>
  bool
    is_heap(Rng&& rng, Comp comp = Comp{}, Proj proj = Proj{});
\end{itemdecl}
\end{addedblock}

\begin{itemdescr}
\pnum
\returns \tcode{is_heap_until(first, last\added{, comp, proj}) == last}
\end{itemdescr}

\begin{removedblock}
\indexlibrary{\idxcode{is_heap}}%
\begin{itemdecl}
  template<class RandomAccessIterator, class Compare>
    bool is_heap(RandomAccessIterator first, RandomAccessIterator last, Compare comp);
\end{itemdecl}

\begin{itemdescr}
\pnum
\returns \tcode{is_heap_until(first, last, comp) == last}
\end{itemdescr}
\end{removedblock}

\indexlibrary{\idxcode{is_heap_until}}%
\begin{removedblock}
\begin{itemdecl}
  template<class RandomAccessIterator>
    RandomAccessIterator is_heap_until(RandomAccessIterator first, RandomAccessIterator last);
  template<class RandomAccessIterator, class Compare>
    RandomAccessIterator is_heap_until(RandomAccessIterator first, RandomAccessIterator last,
      Compare comp);
\end{itemdecl}
\end{removedblock}
\begin{addedblock}
\begin{itemdecl}
template<RandomAccessIterator I, Sentinel<I> S, class Proj = identity,
    IndirectCallableStrictWeakOrder<Projected<I, Proj>> Comp = less<>>
  I is_heap_until(I first, S last, Comp comp = Comp{}, Proj proj = Proj{});

template<RandomAccessRange Rng, class Proj = identity,
    IndirectCallableStrictWeakOrder<Projected<IteratorType<Rng>, Proj>> Comp = less<>>
  @\oldtxt{IteratorType}\newtxt{safe_iterator_t}@<Rng>
    is_heap_until(Rng&@\newtxt{\&}@ rng, Comp comp = Comp{}, Proj proj = Proj{});
\end{itemdecl}
\end{addedblock}

\begin{itemdescr}
\pnum
\returns If \tcode{distance(first, last) < 2}, returns
\tcode{last}. Otherwise, returns
the last iterator \tcode{i} in \crange{first}{last} for which the
range \range{first}{i} is a heap.

\pnum
\complexity Linear.
\end{itemdescr}

\rSec2[alg.min.max]{Minimum and maximum}

\indexlibrary{\idxcode{min}}%
\begin{removedblock}
\begin{itemdecl}
template<class T> constexpr const T& min(const T& a, const T& b);
template<class T, class Compare>
  constexpr const T& min(const T& a, const T& b, Compare comp);
\end{itemdecl}
\end{removedblock}
\begin{addedblock}
\begin{itemdecl}
template<TotallyOrdered T>
  constexpr const T& min(const T& a, const T& b);

template<class T, class Comp>
  requires StrictWeakOrder<FunctionType<Comp>, T>@\newtxt{()}@
  constexpr const T& min(const T& a, const T& b, Comp comp);
\end{itemdecl}
\end{addedblock}

\begin{itemdescr}
\begin{removedblock}
\pnum
\requires
Type
\tcode{T}
is
\tcode{LessThanComparable} (Table~\cxxref{lessthancomparable}).
\end{removedblock}

\pnum
\returns
The smaller value.

\pnum
\notes
Returns the first argument when the arguments are equivalent.
\end{itemdescr}

\indexlibrary{\idxcode{min}}%
\begin{removedblock}
\begin{itemdecl}
template<class T>
  constexpr T min(initializer_list<T> t);
template<class T, class Compare>
  constexpr T min(initializer_list<T> t, Compare comp);
\end{itemdecl}
\end{removedblock}
\begin{addedblock}
\begin{itemdecl}
template<TotallyOrdered T>
  requires Semiregular<T>@\newtxt{()}@
  constexpr T min(initializer_list<T> rng);

template<InputRange Rng>
  requires TotallyOrdered<ValueType<IteratorType<Rng>>>() &&
    Semiregular<ValueType<IteratorType<Rng>>>@\newtxt{()}@
  ValueType<IteratorType<Rng>>
    min(Rng&& rng);

template<Semiregular T, class Comp>
  requires StrictWeakOrder<FunctionType<Comp>, T>@\newtxt{()}@
  constexpr T min(initializer_list<T> rng, Comp comp);

template<InputRange Rng,
    IndirectCallableStrictWeakOrder<IteratorType<Rng>> Comp>
  requires Semiregular<ValueType<IteratorType<Rng>>>@\newtxt{()}@
  ValueType<IteratorType<Rng>>
    min(Rng&& rng, Comp comp);
\end{itemdecl}
\end{addedblock}

\begin{itemdescr}
\pnum
\requires \removed{\tcode{T} is \tcode{LessThanComparable} and \tcode{CopyConstructible} and
\tcode{t.size() > 0}}\tcode{\added{distance(begin(rng), end(rng)) > 0}}.

\pnum
\returns The smallest value in the initializer_list\added{ or range}.

\pnum
\remarks Returns a copy of the leftmost argument when several arguments are equivalent to the smallest.\
\end{itemdescr}

\indexlibrary{\idxcode{max}}%
\begin{removedblock}
\begin{itemdecl}
template<class T> constexpr const T& max(const T& a, const T& b);
template<class T, class Compare>
  constexpr const T& max(const T& a, const T& b, Compare comp);
\end{itemdecl}
\end{removedblock}
\begin{addedblock}
\begin{itemdecl}
template<TotallyOrdered T>
  constexpr const T& max(const T& a, const T& b);

template<class T, class Comp>
  requires StrictWeakOrder<FunctionType<Comp>, T>@\newtxt{()}@
  constexpr const T& max(const T& a, const T& b, Comp comp);
\end{itemdecl}
\end{addedblock}

\begin{itemdescr}
\begin{removedblock}
\pnum
\requires
Type
\tcode{T}
is
\tcode{LessThanComparable} (Table~\cxxref{lessthancomparable}).
\end{removedblock}

\pnum
\returns
The larger value.

\pnum
\notes
Returns the first argument when the arguments are equivalent.
\end{itemdescr}

\indexlibrary{\idxcode{max}}%
\begin{removedblock}
\begin{itemdecl}
template<class T>
  constexpr T max(initializer_list<T> t);
template<class T, class Compare>
  constexpr T max(initializer_list<T> t, Compare comp);
\end{itemdecl}
\end{removedblock}
\begin{addedblock}
\begin{itemdecl}
template<TotallyOrdered T>
  requires Semiregular<T>@\newtxt{()}@
  constexpr T max(initializer_list<T> rng);

template<InputRange Rng>
  requires TotallyOrdered<ValueType<IteratorType<Rng>>>() &&
    Semiregular<ValueType<IteratorType<Rng>>>@\newtxt{()}@
  ValueType<IteratorType<Rng>>
    max(Rng&& rng);

template<Semiregular T, class Comp>
  requires StrictWeakOrder<FunctionType<Comp>, T>@\newtxt{()}@
  constexpr T max(initializer_list<T> rng, Comp comp);

template<InputRange Rng,
    IndirectCallableStrictWeakOrder<IteratorType<Rng>> Comp>
  requires Semiregular<ValueType<IteratorType<Rng>>>@\newtxt{()}@
  ValueType<IteratorType<Rng>>
    max(Rng&& rng, Comp comp);
\end{itemdecl}
\end{addedblock}

\begin{itemdescr}
\pnum
\requires \removed{\tcode{T} is \tcode{LessThanComparable} and \tcode{CopyConstructible} and
\tcode{t.size() > 0}}\tcode{\added{distance(begin(rng), end(rng)) > 0}}.

\pnum
\returns The largest value in the initializer_list\added{ or range}.

\pnum
\remarks Returns a copy of the leftmost argument when several arguments are equivalent to the largest.
\end{itemdescr}

\indexlibrary{\idxcode{minmax}}%
\begin{removedblock}
\begin{itemdecl}
template<class T> constexpr pair<const T&, const T&> minmax(const T& a, const T& b);
template<class T, class Compare>
  constexpr pair<const T&, const T&> minmax(const T& a, const T& b, Compare comp);
\end{itemdecl}
\end{removedblock}
\begin{addedblock}
\begin{itemdecl}
template<TotallyOrdered T>
  constexpr tagged_pair<tag::min(const T&), tag::max(const T&)>
    minmax(const T& a, const T& b);

template<class T, class Comp>
  requires StrictWeakOrder<FunctionType<Comp>, T>@\newtxt{()}@
  constexpr tagged_pair<tag::min(const T&), tag::max(const T&)>
    minmax(const T& a, const T& b, Comp comp);
\end{itemdecl}
\end{addedblock}

\begin{itemdescr}
\begin{removedblock}
\pnum
\requires
Type
\tcode{T}
shall be
\tcode{LessThanComparable} (Table~\cxxref{lessthancomparable}).
\end{removedblock}

\pnum
\returns
\tcode{\changed{pair<const T\&, const T\&>(}{\{}b, a\changed{)}{\}}} if \tcode{b} is smaller
than \tcode{a}, and
\tcode{\changed{pair<const T\&, const T\&>(}{\{}a, b\changed{)}{\}}} otherwise.

\pnum
\notes
Returns \tcode{\changed{pair<const T\&, const T\&>(}{\{}a, b\changed{)}{\}}} when the arguments are equivalent.

\pnum
\complexity
Exactly one comparison.
\end{itemdescr}

\indexlibrary{\idxcode{minmax}}%
\begin{removedblock}
\begin{itemdecl}
template<class T>
  constexpr pair<T, T> minmax(initializer_list<T> t);
template<class T, class Compare>
  constexpr pair<T, T> minmax(initializer_list<T> t, Compare comp);
\end{itemdecl}
\end{removedblock}
\begin{addedblock}
\begin{itemdecl}
template<TotallyOrdered T>
  requires Semiregular<T>@\newtxt{()}@
  constexpr tagged_pair<tag::min(T), tag::max(T)>
    minmax(initializer_list<T> rng);

template<InputRange Rng>
  requires TotallyOrdered<ValueType<IteratorType<Rng>>>() &&
    Semiregular<ValueType<IteratorType<Rng>>>@\newtxt{()}@
  tagged_pair<tag::min(ValueType<IteratorType<Rng>>), tag::max(ValueType<IteratorType<Rng>>)>
    minmax(Rng&& rng);

template<Semiregular T, class Comp>
  requires StrictWeakOrder<FunctionType<Comp>, T>@\newtxt{()}@
  constexpr tagged_pair<tag::min(T), tag::max(T)>
    minmax(initializer_list<T> rng, Comp comp);

template<InputRange Rng,
    IndirectCallableStrictWeakOrder<IteratorType<Rng>> Comp>
  requires Semiregular<ValueType<IteratorType<Rng>>>@\newtxt{()}@
  tagged_pair<tag::min(ValueType<IteratorType<Rng>>), tag::max(ValueType<IteratorType<Rng>>)>
    minmax(Rng&& rng, Comp comp);
\end{itemdecl}
\end{addedblock}

\begin{itemdescr}
\pnum
\requires \removed{\tcode{T} is \tcode{LessThanComparable} and \tcode{CopyConstructible} and
\tcode{t.size() > 0}}\tcode{\added{distance(begin(rng), end(rng)) > 0}}.

\pnum
\returns \tcode{\changed{pair<T, T>(}{\{}x, y\changed{)}{\}}}, where \tcode{x} has the smallest and \tcode{y} has the
largest value in the initializer list\added{ or range}.

\pnum
\remarks \tcode{x} is a copy of the leftmost argument when several arguments are equivalent to
the smallest. \tcode{y} is a copy of the rightmost argument when several arguments are
equivalent to the largest.

\pnum
\complexity At most \tcode{\changed{(3/2) * t.size()}{(3/2) * distance(begin(rng), end(rng))}}
applications of the corresponding predicate.
\end{itemdescr}

\indexlibrary{\idxcode{min_element}}%
\begin{removedblock}
\begin{itemdecl}
template<class ForwardIterator>
  ForwardIterator min_element(ForwardIterator first, ForwardIterator last);

template<class ForwardIterator, class Compare>
  ForwardIterator min_element(ForwardIterator first, ForwardIterator last,
                            Compare comp);
\end{itemdecl}
\end{removedblock}
\begin{addedblock}
\begin{itemdecl}
template<ForwardIterator I, Sentinel<I> S, class Proj = identity,
    IndirectCallableStrictWeakOrder<Projected<I, Proj>> Comp = less<>>
  I min_element(I first, S last, Comp comp = Comp{}, Proj proj = Proj{});

template<ForwardRange Rng, class Proj = identity,
    IndirectCallablStrictWeakOrder<Projected<IteratorType<Rng>, Proj>> Comp = less<>>
  @\oldtxt{IteratorType}\newtxt{safe_iterator_t}@<Rng>
    min_element(Rng&@\newtxt{\&}@ rng, Comp comp = Comp{}, Proj proj = Proj{});
\end{itemdecl}
\end{addedblock}

\begin{itemdescr}
\pnum
\returns
The first iterator
\tcode{i}
in the range
\range{first}{last}
such that for every iterator
\tcode{j}
in the range
\range{first}{last}
the following corresponding conditions hold:
\removed{\tcode{!(*j < *i)}
or}
\tcode{\changed{comp(*j, *i) == false}{
\textit{INVOKE}(comp, \textit{INVOKE}(proj, *j), \textit{INVOKE}(proj, *i)) == false}}.
Returns
\tcode{last}
if
\tcode{first == last}.

\pnum
\complexity
Exactly
\tcode{max((last - first) - 1, 0)}
applications of the \changed{corresponding comparisons}{comparison function and
\newtxt{exactly twice as many applications of the} projection}.
\end{itemdescr}

\indexlibrary{\idxcode{max_element}}%
\begin{removedblock}
\begin{itemdecl}
template<class ForwardIterator>
  ForwardIterator max_element(ForwardIterator first, ForwardIterator last);
template<class ForwardIterator, class Compare>
  ForwardIterator max_element(ForwardIterator first, ForwardIterator last,
                            Compare comp);
\end{itemdecl}
\end{removedblock}
\begin{addedblock}
\begin{itemdecl}
template<ForwardIterator I, Sentinel<I> S, class Proj = identity,
    IndirectCallableStrictWeakOrder<Projected<I, Proj>> Comp = less<>>
  I max_element(I first, S last, Comp comp = Comp{}, Proj proj = Proj{});

template<ForwardRange Rng, class Proj = identity,
    IndirectCallableStrictWeakOrder<Projected<IteratorType<Rng>, Proj>> Comp = less<>>
  @\oldtxt{IteratorType}\newtxt{safe_iterator_t}@<Rng>
    max_element(Rng&@\newtxt{\&}@ rng, Comp comp = Comp{}, Proj proj = Proj{});
\end{itemdecl}
\end{addedblock}

\begin{itemdescr}
\pnum
\returns
The first iterator
\tcode{i}
in the range
\range{first}{last}
such that for every iterator
\tcode{j}
in the range
\range{first}{last}
the following corresponding conditions hold:
\removed{\tcode{!(*i < *j)}
or}
\tcode{\changed{comp(*i, *j) == false}{\textit{INVOKE}(comp, \textit{INVOKE}(proj, *i), \textit{INVOKE}(proj, *j)) == false}}.
Returns
\tcode{last}
if
\tcode{first == last}.

\pnum
\complexity
Exactly
\tcode{max((last - first) - 1, 0)}
applications of the \changed{corresponding comparisons}{comparison function and
\newtxt{exactly twice as many applications of the} projection}.
\end{itemdescr}

\indexlibrary{\idxcode{minmax_element}}%
\begin{removedblock}
\begin{itemdecl}
template<class ForwardIterator>
  pair<ForwardIterator, ForwardIterator>
    minmax_element(ForwardIterator first, ForwardIterator last);
template<class ForwardIterator, class Compare>
  pair<ForwardIterator, ForwardIterator>
    minmax_element(ForwardIterator first, ForwardIterator last, Compare comp);
\end{itemdecl}
\end{removedblock}
\begin{addedblock}
\begin{itemdecl}
template<ForwardIterator I, Sentinel<I> S, class Proj = identity,
    IndirectCallableStrictWeakOrder<Projected<I, Proj>> Comp = less<>>
  tagged_pair<tag::min(I), tag::max(I)>
    minmax_element(I first, S last, Comp comp = Comp{}, Proj proj = Proj{});

template<ForwardRange Rng, class Proj = identity,
    IndirectCallableStrictWeakOrder<Projected<IteratorType<Rng>, Proj>> Comp = less<>>
  tagged_pair<tag::min(@\oldtxt{IteratorType}\newtxt{safe_iterator_t}@<Rng>),
              tag::max(@\oldtxt{IteratorType}\newtxt{safe_iterator_t}@<Rng>)>
    minmax_element(Rng&@\newtxt{\&}@ rng, Comp comp = Comp{}, Proj proj = Proj{});
\end{itemdecl}
\end{addedblock}

\begin{itemdescr}
\pnum
\returns
\tcode{\changed{make_pair(}{\{}first, first\changed{)}{\}}} if \range{first}{last} is empty, otherwise
\tcode{\changed{make_pair(}{\{}m, M\changed{)}{\}}}, where \tcode{m} is
the first iterator in \range{first}{last} such that no iterator in the range refers to a smaller
element, and where \tcode{M} is the last iterator in \range{first}{last} such that no iterator
in the range refers to a larger element.

\pnum
\complexity
At most
$max(\lfloor{\frac{3}{2}} (N-1)\rfloor, 0)$
applications of the \changed{corresponding predicate}{comparison function and
\newtxt{at most twice as many applications of the} projection},
where $N$ is \tcode{distance(first, last)}.
\end{itemdescr}

\rSec2[alg.lex.comparison]{Lexicographical comparison}

\indexlibrary{\idxcode{lexicographical_compare}}%
\begin{removedblock}
\begin{itemdecl}
template<class InputIterator1, class InputIterator2>
  bool
    lexicographical_compare(InputIterator1 first1, InputIterator1 last1,
                            InputIterator2 first2, InputIterator2 last2);

template<class InputIterator1, class InputIterator2, class Compare>
  bool
    lexicographical_compare(InputIterator1 first1, InputIterator1 last1,
                            InputIterator2 first2, InputIterator2 last2,
                            Compare comp);
\end{itemdecl}
\end{removedblock}
\begin{addedblock}
\begin{itemdecl}
template<InputIterator I1, Sentinel<I1> S1, InputIterator I2, Sentinel<I2> S2,
    class Proj1 = identity, class Proj2 = identity,
    IndirectCallableStrictWeakOrder<Projected<I1, Proj1>, Projected<I2, Proj2>> Comp = less<>>
  bool
    lexicographical_compare(I1 first1, S1 last1, I2 first2, S2 last2,
                            Comp comp = Comp{}, Proj1 proj1 = Proj1{}, Proj2 proj2 = Proj2{});

template<InputRange Rng1, InputRange Rng2, class Proj1 = identity,
    class Proj2 = identity,
    IndirectCallableStrictWeakOrder<Projected<IteratorType<Rng1>, Proj1>,
      Projected<IteratorType<Rng2>, Proj2>> Comp = less<>>
  bool
    lexicographical_compare(Rng1&& rng1, Rng2&& rng2, Comp comp = Comp{},
                            Proj1 proj1 = Proj1{}, Proj2 proj2 = Proj2{});
\end{itemdecl}
\end{addedblock}

\begin{itemdescr}
\pnum
\returns
\tcode{true}
if the sequence of elements defined by the range
\range{first1}{last1}
is lexicographically less than the sequence of elements defined by the range
\range{first2}{last2} and
\tcode{false}
otherwise.

\pnum
\complexity
At most
\tcode{2*min((last1 - first1), (last2 - first2))}
applications of the corresponding comparison\added{ and projection}.

\pnum
\notes
If two sequences have the same number of elements and their corresponding
elements are equivalent, then neither sequence is lexicographically
less than the other.
If one sequence is a prefix of the other, then the shorter sequence is
lexicographically less than the longer sequence.
Otherwise, the lexicographical comparison of the sequences yields the same
result as the comparison of the first corresponding pair of
elements that are not equivalent.

\begin{removedblock}
\begin{codeblock}
for ( ; first1 != last1 && first2 != last2 ; ++first1, ++first2) {
  if (*first1 < *first2) return true;
  if (*first2 < *first1) return false;
}
return first1 == last1 && first2 != last2;
\end{codeblock}
\end{removedblock}
\begin{addedblock}
\begin{codeblock}
using namespace placeholders;
auto&& cmp1 = bind(comp, bind(proj1, _1), bind(proj2, _2));
auto&& cmp2 = bind(comp, bind(proj2, _1), bind(proj1, _2));
for ( ; first1 != last1 && first2 != last2 ; ++first1, ++first2) {
  if (cmp1(*first1, *first2)) return true;
  if (cmp2(*first2, *first1)) return false;
}
return first1 == last1 && first2 != last2;
\end{codeblock}
\end{addedblock}

\pnum
\remarks\ An empty sequence is lexicographically less than any non-empty sequence, but
not less than any empty sequence.

\end{itemdescr}

\rSec2[alg.permutation.generators]{Permutation generators}

\indexlibrary{\idxcode{next_permutation}}%
\begin{removedblock}
\begin{itemdecl}
template<class BidirectionalIterator>
  bool next_permutation(BidirectionalIterator first,
                        BidirectionalIterator last);

template<class BidirectionalIterator, class Compare>
  bool next_permutation(BidirectionalIterator first,
                        BidirectionalIterator last, Compare comp);
\end{itemdecl}
\end{removedblock}
\begin{addedblock}
\begin{itemdecl}
template<BidirectionalIterator I, Sentinel<I> S, class Comp = less<>,
    class Proj = identity>
  requires Sortable<I, Comp, Proj>@\newtxt{()}@
  bool next_permutation(I first, S last, Comp comp = Comp{}, Proj proj = Proj{});

template<BidirectionalRange Rng, class Comp = less<>,
    class Proj = identity>
  requires Sortable<IteratorType<Rng>, Comp, Proj>@\newtxt{()}@
  bool
    next_permutation(Rng&& rng, Comp comp = Comp{}, Proj proj = Proj{});
\end{itemdecl}
\end{addedblock}

\begin{itemdescr}
\pnum
\effects
Takes a sequence defined by the range
\range{first}{last}
and transforms it into the next permutation.
The next permutation is found by assuming that the set of all permutations is
lexicographically sorted with respect to
\removed{\tcode{operator<}
or} \tcode{comp}\added{ and \tcode{proj}}.
If such a permutation exists, it returns
\tcode{true}.
Otherwise, it transforms the sequence into the smallest permutation,
that is, the ascendingly sorted one, and returns
\tcode{false}.

\begin{removedblock}
\pnum
\requires
\tcode{BidirectionalIterator} shall satisfy the requirements of
\tcode{ValueSwappable}~(\ref{concepts.lib.corelang.swappable}).
\end{removedblock}

\pnum
\complexity
At most
\tcode{(last - first)/2}
swaps.
\end{itemdescr}

\indexlibrary{\idxcode{prev_permutation}}%
\begin{removedblock}
\begin{itemdecl}
template<class BidirectionalIterator>
  bool prev_permutation(BidirectionalIterator first,
                        BidirectionalIterator last);

template<class BidirectionalIterator, class Compare>
  bool prev_permutation(BidirectionalIterator first,
                        BidirectionalIterator last, Compare comp);
\end{itemdecl}
\end{removedblock}
\begin{addedblock}
\begin{itemdecl}
template<BidirectionalIterator I, Sentinel<I> S, class Comp = less<>,
    class Proj = identity>
  requires Sortable<I, Comp, Proj>@\newtxt{()}@
  bool prev_permutation(I first, S last, Comp comp = Comp{}, Proj proj = Proj{});

template<BidirectionalRange Rng, class Comp = less<>,
    class Proj = identity>
  requires Sortable<IteratorType<Rng>, Comp, Proj>@\newtxt{()}@
  bool
    prev_permutation(Rng&& rng, Comp comp = Comp{}, Proj proj = Proj{});
\end{itemdecl}
\end{addedblock}

\begin{itemdescr}
\pnum
\effects
Takes a sequence defined by the range
\range{first}{last}
and transforms it into the previous permutation.
The previous permutation is found by assuming that the set of all permutations is
lexicographically sorted with respect to
\removed{\tcode{operator<}
or} \tcode{comp}\added{ and \tcode{proj}}.

\pnum
\returns
\tcode{true}
if such a permutation exists.
Otherwise, it transforms the sequence into the largest permutation,
that is, the descendingly sorted one, and returns
\tcode{false}.

\begin{removedblock}
\pnum
\requires
\tcode{BidirectionalIterator} shall satisfy the requirements of
\tcode{ValueSwappable}~(\ref{concepts.lib.corelang.swappable}).
\end{removedblock}

\pnum
\complexity
At most
\tcode{(last - first)/2}
swaps.
\end{itemdescr}

\rSec1[alg.c.library]{C library algorithms}

\pnum
Table~\ref{tab:algorithms.hdr.cstdlib} describes some of the contents of the header \tcode{<cstdlib>}.

\begin{libsyntab3}{cstdlib}{tab:algorithms.hdr.cstdlib}
\type   & \tcode{size_t}  &         \\ \hline
\functions  & \tcode{bsearch} & \tcode{qsort} \\
\end{libsyntab3}

\pnum
The contents are the same as the Standard C library header
\tcode{<stdlib.h>}
with the following exceptions:

\pnum
The function signature:

\begin{codeblock}
bsearch(const void *, const void *, size_t, size_t,
  int (*)(const void *, const void *));
\end{codeblock}

is replaced by the two declarations:

\begin{codeblock}
extern "C" void* bsearch(const void* key, const void* base,
                         size_t nmemb, size_t size,
                         int (*compar)(const void*, const void*));
extern "C++" void* bsearch(const void* key, const void* base,
                           size_t nmemb, size_t size,
                           int (*compar)(const void*, const void*));
\end{codeblock}

both of which have the same behavior as the original declaration.

\pnum
The function signature:

\begin{codeblock}
qsort(void *, size_t, size_t,
  int (*)(const void *, const void *));
\end{codeblock}

is replaced by the two declarations:

\begin{codeblock}
extern "C" void qsort(void* base, size_t nmemb, size_t size,
                      int (*compar)(const void*, const void*));
extern "C++" void qsort(void* base, size_t nmemb, size_t size,
                        int (*compar)(const void*, const void*));
\end{codeblock}

both of which have the same behavior as the original declaration. The behavior is
undefined unless the objects in the array pointed to by \tcode{base} are of trivial type.

\enternote
Because the function argument \tcode{compar()} may throw an exception,
\tcode{bsearch()}
and
\tcode{qsort()}
are allowed to propagate the exception~(\cxxref{res.on.exception.handling}).
\exitnote

\xref
ISO C 7.10.5.

} % \color{addclr}
