%!TEX root = P0896.tex

\setcounter{chapter}{30}
\rSec0[thread]{Thread support library}

[...]

\setcounter{section}{1}
\rSec1[thread.req]{Requirements}

[...]

\setcounter{subsection}{5}
\rSec2[thread.decaycopy]{\tcode{decay_copy}}

\pnum
In several places in this Clause the operation
\indextext{DECAY_COPY@\tcode{\placeholder{DECAY_COPY}}}%
\indexlibrary{DECAY_COPY@\tcode{\placeholder{DECAY_COPY}}}%
{\tcode{\placeholdernc{DECAY_COPY}(x)}} is used. All
such uses mean call the function \tcode{decay_copy(x)} and use the
result, where \tcode{decay_copy} is defined as follows:

\begin{codeblock}
template<class T>
  @\newnewtxt{constexpr}@ decay_t<T> decay_copy(T&& v)
    @\newnewtxt{noexcept(is_nothrow_constructible_v<decay_t<T>, T>)}@
  { return std::forward<T>(v); }
\end{codeblock}

[...]
