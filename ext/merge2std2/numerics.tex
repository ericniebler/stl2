%!TEX root = P0896.tex

\rSec0[numerics]{Numerics library}

\setcounter{section}{6}
\rSec1[numarray]{Numeric arrays}

\ednote{In the paragraph that specifies \tcode{valarray}'s iterators,
cross-reference "contiguous iterator" to [iterator.concept.contiguous] instead of
[iterator.requirements.general]. (Is this too subtle a means to require
implementations to specialize \tcode{ranges::iterator_category}?)}

\setcounter{subsection}{9}
\rSec2[valarray.range]{\tcode{valarray} range access}

\pnum
In the \tcode{begin} and \tcode{end} function templates that follow, \unspec{1}
is a type that meets the requirements of a mutable random access
iterator~(\cxxref{random.access.iterators})
and of a contiguous iterator\iref{iterator.concept.contiguous}
whose \tcode{value_type} is the template
parameter \tcode{T} and whose \tcode{reference} type is \tcode{T\&}. \unspec{2} is a
type that meets the requirements of a constant random access
iterator~(\cxxref{random.access.iterators})
and of a contiguous iterator\iref{iterator.concept.contiguous}
whose \tcode{value_type} is the template
parameter \tcode{T} and whose \tcode{reference} type is \tcode{const T\&}.

\pnum [...]
