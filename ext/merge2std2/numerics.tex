%!TEX root = P0896.tex

\rSec0[numerics]{Numerics library}

[...]

\setcounter{section}{7}
\rSec1[numarray]{Numeric arrays}

[...]

\setcounter{subsection}{9}
\rSec2[valarray.range]{\tcode{valarray} range access}

\pnum
In the \tcode{begin} and \tcode{end} function templates that follow, \unspec{1}
is a type that meets the requirements of a mutable \removed{random access
iterator} \added{\oldconcept{RandomAccessIterator}}\iref{random.access.iterators}
\removed{and of a contiguous iterator\iref{iterator.requirements.general}}
\added{and models \libconcept{ContiguousIterator}\iref{iterator.concept.contiguous},}
whose \tcode{value_type} is the template
parameter \tcode{T} and whose \tcode{reference} type is \tcode{T\&}. \unspec{2} is a
type that meets the requirements of a constant \removed{random access
iterator\iref{random.access.iterators}} \added{\oldconcept{RandomAccessIterator}}
\removed{and of a contiguous iterator\iref{iterator.requirements.general}}
\added{and models \libconcept{ContiguousIterator},}
whose \tcode{value_type} is the template
parameter \tcode{T} and whose \tcode{reference} type is \tcode{const T\&}.

[...]
