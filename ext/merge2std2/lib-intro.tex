%!TEX root = P0896.tex
\newpage
\setcounter{chapter}{14}
\setcounter{table}{16}
\rSec0[library]{Library introduction}

[...]

\rSec1[library.general]{General}

[...]

\ednote{Insert a new row in \tref{library.categories} for the ranges library:}

\begin{libsumtabbase}{Library categories}{tab:library.categories}{Clause}{Category}
Clause \cxxref{language.support}  &   &   Language support library    \\
\ref{concepts}                    &   &   Concepts library            \\
Clause \cxxref{diagnostics}       &   &   Diagnostics library         \\
\ref{utilities}                   &   &   General utilities library   \\
\ref{strings}                     &   &   Strings library             \\
Clause \cxxref{localization}      &   &   Localization library        \\
\ref{containers}                  &   &   Containers library          \\
\ref{iterators}                   &   &   Iterators library           \\
\newtxt{\ref{range}}              &   &   \newtxt{Ranges library}     \\
\ref{algorithms}                  &   &   Algorithms library          \\
\ref{numerics}                    &   &   Numerics library            \\
Clause \cxxref{input.output}      &   &   Input/output library        \\
Clause \cxxref{re}                &   &   Regular expressions library \\
Clause \cxxref{atomics}           &   &   Atomic operations library   \\
Clause \cxxref{thread}            &   &   Thread support library      \\
\end{libsumtabbase}

[...]

\setcounter{Paras}{8}
\pnum
The containers\iref{containers}, iterators\iref{iterators},
\newtxt{ranges\iref{range}}, and algorithms\iref{algorithms}
libraries provide a \Cpp{} program with access
to a subset of the most widely used algorithms and data structures.

\setcounter{section}{2}
\rSec1[definitions]{Definitions}

[...]

\begin{addedblock}
\setcounter{subsection}{17}
\definition{projection}{defns.projection}
\indexdefn{projection}%
\defncontext{function object argument} transformation \oldtxt{which} \newtxt{that}
an algorithm applies before inspecting the values of elements

\begin{example}
\begin{codeblock}
std::pair<int, const char*> pairs[] = {{2, "foo"}, {1, "bar"}, {0, "baz"}};
std::ranges::sort(pairs, std::less<>{}, [](auto const& p) { return p.first; });
\end{codeblock}
sorts the pairs in increasing order of their \tcode{first} members:
\begin{codeblock}
{{0, "baz"}, {1, "bar"}, {2, "foo"}}
\end{codeblock}
\end{example}
\end{addedblock}

[...]

\setcounter{section}{4}
\rSec1[requirements]{Library-wide requirements}

[...]

\setcounter{subsection}{1}
\setcounter{subsubsection}{1}
\rSec3[headers]{Headers}

\ednote{Add header \tcode{<ranges>} to \tref{cpp.library.headers}:}

\begin{multicolfloattable}{\Cpp{} library headers}{tab:cpp.library.headers}
{llll}
\tcode{<algorithm>} \\
\tcode{<any>} \\
\tcode{<array>} \\
\tcode{<atomic>} \\
\tcode{<bit>} \\
\tcode{<bitset>} \\
\tcode{<charconv>} \\
\tcode{<chrono>} \\
\tcode{<codecvt>} \\
\tcode{<compare>} \\
\tcode{<complex>} \\
\tcode{<concepts>} \\
\tcode{<condition_variable>} \\
\tcode{<contract>} \\
\tcode{<deque>} \\
\tcode{<exception>} \\
\tcode{<execution>} \\
\tcode{<filesystem>} \\
\columnbreak
\tcode{<forward_list>} \\
\tcode{<fstream>} \\
\tcode{<functional>} \\
\tcode{<future>} \\
\tcode{<initializer_list>} \\
\tcode{<iomanip>} \\
\tcode{<ios>} \\
\tcode{<iosfwd>} \\
\tcode{<iostream>} \\
\tcode{<istream>} \\
\tcode{<iterator>} \\
\tcode{<limits>} \\
\tcode{<list>} \\
\tcode{<locale>} \\
\tcode{<map>} \\
\tcode{<memory>} \\
\tcode{<memory_resource>} \\
\tcode{<mutex>} \\
\columnbreak
\tcode{<new>} \\
\tcode{<numeric>} \\
\tcode{<optional>} \\
\tcode{<ostream>} \\
\tcode{<queue>} \\
\tcode{<random>} \\
\tcode{\newtxt{<ranges>}} \\
\tcode{<ratio>} \\
\tcode{<regex>} \\
\tcode{<scoped_allocator>} \\
\tcode{<set>} \\
\tcode{<shared_mutex>} \\
\tcode{<span>} \\
\tcode{<sstream>} \\
\tcode{<stack>} \\
\tcode{<stdexcept>} \\
\tcode{<streambuf>} \\
\columnbreak
\tcode{<string>} \\
\tcode{<string_view>} \\
\tcode{<strstream>} \\
\tcode{<syncstream>} \\
\tcode{<system_error>} \\
\tcode{<thread>} \\
\tcode{<tuple>} \\
\tcode{<typeindex>} \\
\tcode{<typeinfo>} \\
\tcode{<type_traits>} \\
\tcode{<unordered_map>} \\
\tcode{<unordered_set>} \\
\tcode{<utility>} \\
\tcode{<valarray>} \\
\tcode{<variant>} \\
\tcode{<vector>} \\
\tcode{<version>} \\
\end{multicolfloattable}

[...]

\rSec3[allocator.requirements]{\oldconcept{Allocator} requirements}

[...]

\setcounter{Paras}{4}
\pnum
An allocator type \tcode{X} shall satisfy the
\oldconcept{CopyConstructible} requirements (Table \cxxref{copyconstructible}).
The \tcode{X::pointer}, \tcode{X::const_pointer}, \tcode{X::void_pointer}, and
\tcode{X::const_void_pointer} types shall satisfy the
\oldconcept{Nullable\-Pointer} requirements (Table \cxxref{nullablepointer}).
No constructor,
comparison function, copy operation, move operation, or swap operation on
these pointer types shall exit via an exception. \tcode{X::pointer} and \tcode{X::const_pointer} shall also
satisfy the requirements for
a random access iterator\iref{random.access.iterators} and \oldtxt{of
a contiguous iterator\iref{iterator.requirements.general}.}
\newtxt{the additional requirement that}
\begin{codeblock}
@\newtxt{addressof(*(a + n)) == addressof(*a) + n}@
\end{codeblock}
\newtxt{when both \tcode{a} and \tcode{(a + n)} are valid dereferenceable
pointer values for some integral value \tcode{n}.}
