%!TEX root = std.tex

\rSec0[intro]{Description of Instructions}

[stmt.ranged] in the Ranges TS is written as a set of editorial instructions
relative to C++14 that update the wording for the range-based for loop.
This paper proposes alterations to those editorial instructions in the form of
--- you guessed it --- even more editorial instructions. In an attempt to provide
clarity of presentation, this paper uses \emph{five} distinct formatting styles
to represent text with different properties:
\begin{itemize}
\item Text that is the same in \Cpp 14 and in the Ranges TS is presented in a
plain style without adornment.

\item {\color{remclr} Text which the TS strikes from \Cpp 14 is red in color.}

\item \added{Text which the TS adds to \Cpp 14 is cyan in color.}

\item \oldtxt{Text which \emph{this} paper proposes to strike from the \emph{TS} is
purple and struck-through.}

\item \newtxt{Text which this paper proposes to add to the TS is gold and
underlined.}
\end{itemize}

\setcounter{chapter}{5}
\setcounter{section}{1}

\rSec2[stmt.ranged]{The range-based \tcode{for} statement}

\ednote{Modify [stmt.ranged] to use a formulation similar to the C++17 FDIS:}

\begin{quote}
\pnum
\oldoldtxt{For a}\newnewtxt{The} range-based \tcode{for} statement \oldoldtxt{of the form}

\begin{ncbnf}
\terminal{for (} for-range-declaration \terminal{:} \oldoldtxt{expression}\newnewtxt{for-range-initializer} \terminal{)} statement
\end{ncbnf}
%
\begin{removedblock}
\oldoldtxt{let \grammarterm{range-init} be equivalent to the expression surrounded by parentheses}

\begin{bnf}
\oldoldtxt{\terminal{(} expression \terminal{)}}
\end{bnf}
%
\oldoldtxt{and for a range-based for statement of the form}

\begin{bnf}
\oldoldtxt{\terminal{for} \terminal{(} for-range-declaration \terminal{:} braced-init-list \terminal{)} statement}
\end{bnf}
%
\oldoldtxt{let \grammarterm{range-init} be equivalent to the \grammarterm{braced-init-list}.}
\end{removedblock}
%
\oldoldtxt{In each case, a range-based for statement} is equivalent to

\begin{removedblock}
\begin{ncbnftab}
\terminal{\{}\br
\>\terminal{auto \&\&__range =} range-init \terminal{;}\br
\>\terminal{for ( auto __begin =} begin-expr \terminal{,}\br
\>\terminal{{ }{ }{ }{ }{ }{ }{ }{ }{ }{ }{ }__end =} end-expr \terminal{;}\br
\>\>\terminal{ __begin != __end;}\br
\>\>\terminal{ ++__begin ) \{}\br
\>\>for-range-declaration \terminal{= *__begin;}\br
\>\>statement\br
\>\terminal{\}}\br
\terminal{\}}
\end{ncbnftab}
\end{removedblock}

\begin{addedblock}
\begin{ncbnftab}
\terminal{\{}\br
\>\terminal{auto \&\&__range =} \oldoldtxt{range-init}\newnewtxt{for-range-initializer} \terminal{;}\br
\>\terminal{auto __begin =} begin-expr \terminal{;}\br
\>\terminal{auto __end =} end-expr \terminal{;}\br
\>\terminal{for ( ; __begin != __end; ++__begin ) \{}\br
\>\>for-range-declaration \terminal{= *__begin;}\br
\>\>statement\br
\>\terminal{\}}\br
\terminal{\}}
\end{ncbnftab}
\end{addedblock}

where

\begin{itemize}
\item
\newnewtxt{if the \grammarterm{for-range-initializer} is an \grammarterm{expression},
it is regarded as if it were surrounded by parentheses (so that a comma operator
cannot be reinterpreted as delimiting two \grammarterm{init-declarator}{s});}

\item
\tcode{__range}, \tcode{__begin}, and \tcode{__end} are variables defined for
exposition only; and \oldoldtxt{\tcode{_RangeT} is the type of the
\grammarterm{}{expression}, and \placeholder{begin-expr} and \placeholder{end-expr}
are determined as follows:}

\item
\newnewtxt{\placeholder{begin-expr} and \placeholder{end-expr} are determined as follows:}

\begin{itemize}
\item if \oldoldtxt{\tcode{_RangeT}}\newnewtxt{the \grammarterm{for-range-initializer}} is an \newnewtxt{an expression of}
array type \newnewtxt{\tcode{R}}, \placeholder{begin-expr} and \placeholder{end-expr} are
\tcode{__range} and \tcode{__range + __bound}, respectively, where \tcode{__bound} is
the array bound. If \tcode{\oldoldtxt{_RangeT}\newnewtxt{R}} is an array of unknown \oldoldtxt{size}\newnewtxt{bound} or an array of
incomplete type, the program is ill-formed;

\item if \oldoldtxt{\tcode{_RangeT}}\newnewtxt{the \grammarterm{for-range-initializer}} is \oldoldtxt{a}\newnewtxt{an expression of}
class type \newnewtxt{\tcode{C}}, the \grammarterm{unqualified-id}{s}
\tcode{begin} and \tcode{end} are looked up in the scope of \oldoldtxt{class \tcode{\mbox{_RangeT}}}\newnewtxt{\tcode{C}}
as if by class member access lookup~(\stdcxxref{basic.lookup.classref}), and if either
(or both) finds at least one declaration, \placeholder{begin-expr} and
\placeholder{end-expr} are \tcode{__range.begin()} and \tcode{__range.end()},
respectively;

\item otherwise, \placeholder{begin-expr} and \placeholder{end-expr} are \tcode{begin(__range)}
and \tcode{end(__range)}, respectively, where \tcode{begin} and \tcode{end} are looked
up in the associated namespaces~(\stdcxxref{basic.lookup.argdep}).
\enternote Ordinary unqualified lookup~(\stdcxxref{basic.lookup.unqual}) is not
performed. \exitnote
\end{itemize}
\end{itemize}

\enterexample
\begin{codeblock}
int array[5] = { 1, 2, 3, 4, 5 };
for (int& x : array)
  x *= 2;
\end{codeblock}
\exitexample%
\indextext{statement!iteration|)}

\pnum
In the \grammarterm{decl-specifier-seq} of a \grammarterm{for-range-declaration},
each \grammarterm{decl-specifier} shall be either a \grammarterm{type-specifier}
or \tcode{constexpr}. The \grammarterm{decl-specifier-seq} shall not define a
class or enumeration.
\end{quote}
