%!TEX root = D0970.tex

\setcounter{chapter}{20}
\rSec0[std2]{Standard Library, Version 2}
\setcounter{SectionDepthBase}{1}
\setcounter{section}{3}

\ednote{The following changes are suggested for P0896.}

\rSec0[std2.ranges]{Ranges library}

\setcounter{subsection}{3}

\rSec1[std2.range.access]{Range access}

\pnum
In addition to being available via inclusion of the \tcode{<std2/range>}
header, the customization point objects in \ref{std2.range.access} are
available when \tcode{<std2/iterator>} is included.

\rSec2[std2.range.access.begin]{\tcode{begin}}

\pnum
The name \tcode{begin} denotes a customization point
 object~(\cxxref{customization.point.object}). The expression
 \tcode{::std2::begin(E)} for some subexpression \tcode{E} is
 expression-equivalent to:

\begin{itemize}
\item
  \removed{\tcode{::std2::begin(static_cast<const T\&>(E))} if \tcode{E} is an
  rvalue of type \tcode{T}. This usage is deprecated. \enternote This deprecated
  usage exists so that \tcode{::std2::begin(E)} behaves similarly to
  \tcode{std::begin(E)} as defined in ISO/IEC 14882 when \tcode{E} is an rvalue.
  \exitnote}

\item
  \removed{Otherwise, }\tcode{(E) + 0} if \tcode{E} has array
  type~(\cxxref{basic.compound})\added{ and is an lvalue}.

\item
  Otherwise, \added{if \tcode{E} is an lvalue,}
  \tcode{\textit{DECAY_COPY}((E).begin())} if it is a valid expression and its
  type \tcode{I} meets the syntactic requirements of \tcode{Iterator<I>}. If
  \tcode{Iterator} is not satisfied, the program is ill-formed with no
  diagnostic required.

\item
  Otherwise, \tcode{\textit{DECAY_COPY}(begin(E))} if it is a valid expression
  and its type \tcode{I} meets the syntactic requirements of \tcode{Iterator<I>}
  with overload resolution performed in a context that includes the
  \added{following} declaration\added{s:}
  \begin{codeblock}
  template <class T> void begin(T&@\added{\&}@) = delete;
  @\added{template <class T> void begin(std::initializer_list<T>\&\&) = delete;}@
  \end{codeblock}
  and does not include a declaration of \tcode{::std2::begin}. If
  \tcode{Iterator} is not satisfied, the program is ill-formed with no
  diagnostic required.

\item
  Otherwise, \tcode{::std2::begin(E)} is ill-formed.
\end{itemize}

\pnum
\enternote Whenever \tcode{::std2::begin(E)} is a valid expression, its
type satisfies \tcode{Iterator}. \exitnote

\rSec2[std2.range.access.end]{\tcode{end}}
\pnum
The name \tcode{end} denotes a customization point
object~(\cxxref{customization.point.object}). The expression
\tcode{::std2::end(E)} for some subexpression \tcode{E} is expression-equivalent to:

\begin{itemize}
\item
  \removed{\tcode{::std2::end(static_cast<const T\&>(E))} if \tcode{E} is an rvalue of
  type \tcode{T}. This usage is deprecated.
  \enternote This deprecated usage exists so that
  \tcode{::std2::end(E)} behaves similarly to \tcode{std::end(E)}
  as defined in ISO/IEC 14882 when \tcode{E} is an rvalue. \exitnote}

\item
  \removed{Otherwise, }\tcode{(E) + extent_v<T>} if \tcode{E} has array
  type~(\cxxref{basic.compound}) \changed{\tcode{T}}{and is an lvalue}.

\item
  Otherwise, \added{if \tcode{E} is an lvalue, }
  \tcode{\textit{DECAY_COPY}((E).end())} if it is a valid expression and its
  type \tcode{S} meets the syntactic requirements of
  \tcode{Sentinel<\brk{}S, decltype(\brk{}::std2::\brk{}begin(E))>}. If
  \tcode{Sentinel} is not satisfied, the program is ill-formed with
  no diagnostic required.

\item
  Otherwise, \tcode{\textit{DECAY_COPY}(end(E))} if it is a valid expression and
  its type \tcode{S} meets the syntactic requirements of
  \tcode{Sentinel<\brk{}S, decltype(\brk{}::std2::\brk{}begin(E))>} with
  overload resolution performed in a context that includes the \added{following}
  declaration\added{s:}
  \begin{codeblock}
  template <class T> void end(T&@\added{\&}@) = delete;
  @\added{template <class T> void end(std::initializer_list<T>\&\&) = delete;}@
  \end{codeblock}
  and does not include a declaration of \tcode{::std2::end}. If \tcode{Sentinel}
  is not satisfied, the program is ill-formed with no diagnostic required.

\item
  Otherwise, \tcode{::std2::end(E)} is ill-formed.
\end{itemize}

\pnum
\enternote Whenever \tcode{::std2::end(E)} is a valid expression, the
types of \tcode{::std2::end(E)} and \tcode{::std2::\brk{}begin(E)} satisfy
\tcode{Sentinel}. \exitnote

\rSec2[std2.range.access.cbegin]{\tcode{cbegin}}
\pnum
The name \tcode{cbegin} denotes a customization point
object~(\cxxref{customization.point.object}). The expression
\tcode{::std2::\brk{}cbegin(E)} for some subexpression \tcode{E} of type
\tcode{T} is expression-equivalent to\added{:}
\begin{itemize}
\item
  \tcode{::std2::begin(static_cast<const T\&>(E))} \added{if \tcode{E} is an lvalue}.
\item
  \added{Otherwise, \tcode{::std2::begin(static_cast<const T\&\&>(E))}.}
\end{itemize}

\pnum
\removed{Use of \tcode{::std2::cbegin(E)} with rvalue \tcode{E} is deprecated.
\enternote This deprecated usage exists so that \tcode{::std2::cbegin(E)}
behaves similarly to \tcode{std::cbegin(E)} as defined in ISO/IEC 14882 when
\tcode{E} is an rvalue. \exitnote}

\pnum
\enternote Whenever \tcode{::std2::cbegin(E)} is a valid expression, its
type satisfies \tcode{Iterator}. \exitnote

\rSec2[std2.range.access.cend]{\tcode{cend}}
\pnum
The name \tcode{cend} denotes a customization point
object~(\cxxref{customization.point.object}). The expression
\tcode{::std2::cend(E)} for some subexpression \tcode{E} of type \tcode{T}
is expression-equivalent to\added{:}
\begin{itemize}
\item
  \tcode{::std2::end(static_cast<const T\&>(E))} \added{if \tcode{E} is an lvalue}.
\item
  \added{Otherwise, \tcode{::std2::end(static_cast<const T\&\&>(E))}.}
\end{itemize}

\pnum
\removed{Use of \tcode{::std2::cend(E)} with rvalue \tcode{E} is deprecated.
\enternote This deprecated usage exists so that \tcode{::std2::\brk{}cend(E)}
behaves similarly to \tcode{std::cend(E)} as defined in ISO/IEC 14882 when
\tcode{E} is an rvalue. \exitnote}

\pnum
\enternote Whenever \tcode{::std2::cend(E)} is a valid expression, the
types of \tcode{::std2::cend(E)} and \tcode{::std2::\brk{}cbegin(E)} satisfy
\tcode{Sentinel}. \exitnote

\rSec2[std2.range.access.rbegin]{\tcode{rbegin}}
\ednote{This changes \tcode{rbegin} and \tcode{rend} into proper customization
points, with ``\tcode{rbegin}'' and ``\tcode{rend}'' looked up via
argument-dependent lookup. The idea is to support types for which reverse
iterators can be implemented more efficiently than with
\tcode{reverse_iterator}, and which might want to overload \tcode{rbegin} and
\tcode{rend} for rvalue arguments. A simple example might be a
\tcode{reverse_subrange} type, which would want to overload \tcode{rbegin} and
\tcode{rend} to return the unmodified underlying iterator and sentinel (as
opposed to \tcode{begin} which would return \tcode{reverse_iterator}s).}

\pnum
The name \tcode{rbegin} denotes a customization point
object~(\cxxref{customization.point.object}). The expression
\tcode{::std2::rbegin(E)} for some subexpression \tcode{E} is
expression-equivalent to:

\begin{itemize}
\item
  \removed{\tcode{::std2::rbegin(static_cast<const T\&>(E))} if \tcode{E} is an
  rvalue of type \tcode{T}. This usage is deprecated. \enternote This deprecated
  usage exists so that \tcode{::std2::rbegin(E)} behaves similarly to
  \tcode{std::rbegin(E)} as defined in ISO/IEC 14882 when \tcode{E} is an
  rvalue. \exitnote}

\item
  \changed{Otherwise}{If \tcode{E} is an lvalue},
  \tcode{\textit{DECAY_COPY}((E).rbegin())} if it is a valid expression and its
  type \tcode{I} meets the syntactic requirements of \tcode{Iterator<I>}. If
  \tcode{Iterator} is not satisfied, the program is ill-formed with no
  diagnostic required.

\item
  \added{Otherwise, \tcode{\textit{DECAY_COPY}(rbegin(E))} if it is a valid
  expression and its type \tcode{I} meets the syntactic requirements
  of \tcode{Iterator<I>} with overload resolution performed in a context that
  includes the following declaration:}
  \begin{addedblock}
  \begin{codeblock}
  template <class T> void rbegin(T&&) = delete;
  \end{codeblock}
  \end{addedblock}
  \added{and does not include a declaration of \tcode{::std2::rbegin}. If
  \tcode{Iterator} is not satisfied, the program is ill-formed with no
  diagnostic required.}

\item
  Otherwise, \tcode{make_reverse_iterator(::std2::end(E))} if both
  \tcode{::std2::begin(E)} and \tcode{::std2::\brk{}end(\brk{}E)} are valid
  expressions of the same type \tcode{I} which meets the syntactic requirements
  of \tcode{Bi\-direct\-ional\-Iterator<I>}~(\ref{std2.iterators.bidirectional}).

\item
  Otherwise, \tcode{::std2::rbegin(E)} is ill-formed.
\end{itemize}

\pnum
\enternote Whenever \tcode{::std2::rbegin(E)} is a valid expression, its
type satisfies \tcode{Iterator}. \exitnote

\rSec2[std2.range.access.rend]{\tcode{rend}}
\pnum
The name \tcode{rend} denotes a customization point
object~(\cxxref{customization.point.object}). The expression
\tcode{::std2::rend(E)} for some subexpression \tcode{E} is
expression-equivalent to:

\begin{itemize}
\item
  \removed{\tcode{::std2::rend(static_cast<const T\&>(E))} if \tcode{E} is an
  rvalue of type \tcode{T}. This usage is deprecated. \enternote This deprecated
  usage exists so that \tcode{::std2::rend(E)} behaves similarly to
  \tcode{std::rend(E)} as defined in ISO/IEC 14882 when \tcode{E} is an rvalue.
  \exitnote}

\item
  \changed{Otherwise}{If \tcode{E} is an lvalue},
  \tcode{\textit{DECAY_COPY}((E).rend())} if it is a valid expression and its
  type \tcode{S} meets the syntactic requirements of
  \tcode{Sentinel<\brk{}S, decltype(\brk{}::std2::\brk{}rbegin(E))>}. If
  \tcode{Sentinel} is not satisfied, the program is ill-formed with
  no diagnostic required.

\item
  \added{Otherwise, \tcode{\textit{DECAY_COPY}(rend(E))} if it is a valid
  expression and its type \tcode{S} meets the syntactic requirements
  of \tcode{Sentinel<S, decltype(\brk{}::std2::\brk{}rbegin(E))>} with overload
  resolution performed in a context that includes the following declaration:}
  \begin{addedblock}
  \begin{codeblock}
  template <class T> void rend(T&&) = delete;
  \end{codeblock}
  \end{addedblock}
  \added{and does not include a declaration of \tcode{::std2::rend}. If
  \tcode{Sentinel} is not satisfied, the program is ill-formed with no
  diagnostic required.}

\item
  Otherwise, \tcode{make_reverse_iterator(::std2\colcol{}begin(E))} if both
  \tcode{::std2::begin(E)} and \tcode{::std2\colcol{}end(\brk{}E)} are valid
  expressions of the same type \tcode{I} which meets the syntactic requirements
  of \tcode{Bi\-dir\-ect\-ion\-al\-It\-er\-at\-or<I>}~(\ref{std2.iterators.bidirectional}).

\item
  Otherwise, \tcode{::std2::rend(E)} is ill-formed.
\end{itemize}

\pnum
\enternote Whenever \tcode{::std2::rend(E)} is a valid expression, the
types of \tcode{::std2::\brk{}rend(E)} and \tcode{::std2::\brk{}rbegin(E)} satisfy
\tcode{Sentinel}. \exitnote

\rSec2[std2.range.access.crbegin]{\tcode{crbegin}}
\pnum
The name \tcode{crbegin} denotes a customization point
object~(\cxxref{customization.point.object}). The expression
\tcode{::std2::\brk{}crbegin(E)} for some subexpression \tcode{E} of type
\tcode{T} is expression-equivalent to\added{:}

\begin{itemize}
\item
  \tcode{::std2::\brk{}rbegin(static_cast<const T\&>(E))}\added{ if \tcode{E}
  is an lvalue}.
\item
  \added{Otherwise, \tcode{::std2::\brk{}rbegin(static_cast<const T\&\&>(E))}.}
\end{itemize}

\pnum
\removed{Use of \tcode{::std2::crbegin(E)} with rvalue \tcode{E} is deprecated.
\enternote This deprecated usage exists so that \tcode{::std2::crbegin(E)}
behaves similarly to \tcode{std::crbegin(E)} as defined in ISO/IEC 14882 when
\tcode{E} is an rvalue. \exitnote}

\pnum
\enternote Whenever \tcode{::std2::crbegin(E)} is a valid expression, its
type satisfies \tcode{Iterator}. \exitnote

\rSec2[std2.range.access.crend]{\tcode{crend}}
\pnum
The name \tcode{crend} denotes a customization point
object~(\cxxref{customization.point.object}). The expression
\tcode{::std2::crend(E)} for some subexpression \tcode{E} of type \tcode{T}
is expression-equivalent to\added{:}

\begin{itemize}
\item
  \tcode{::std2::\brk{}rend(static_cast<const T\&>(E))}\added{ if \tcode{E}
  is an lvalue}.
\item
  \added{Otherwise, \tcode{::std2::\brk{}rend(static_cast<const T\&\&>(E))}.}
\end{itemize}

\pnum
\removed{Use of \tcode{::std2::crend(E)} with rvalue \tcode{E} is deprecated.
\enternote This deprecated usage exists so that \tcode{::std2::crend(E)}
behaves similarly to \tcode{std::crend(E)} as defined in ISO/IEC 14882 when
\tcode{E} is an rvalue. \exitnote}

\pnum
\enternote Whenever \tcode{::std2::crend(E)} is a valid expression, the
types of \tcode{::std2::crend(E)} and \tcode{::std2::\brk{}crbegin(\brk{}E)} satisfy
\tcode{Sentinel}. \exitnote

\rSec1[std2.range.primitives]{Range primitives}

\pnum
In addition to being available via inclusion of the \tcode{<std2/range>}
header, the customization point objects in \ref{std2.range.primitives} are
available when \tcode{<std2/iterator>} is included.

\rSec2[std2.range.primitives.size]{\tcode{size}}
\pnum
The name \tcode{size} denotes a customization point
object~(\cxxref{customization.point.object}). The expression
\tcode{::std2::size(E)} for some subexpression \tcode{E} with type
\tcode{T} is expression-equivalent to:

\begin{itemize}
\item
  \tcode{\textit{DECAY_COPY}(extent_v<T>)} if \tcode{T} is an array
  type~(\cxxref{basic.compound}).

\item
  Otherwise,
  \tcode{\textit{DECAY_COPY}(\removed{static_cast<const T\&>(}E\removed{)}.size())}
  if it is a valid expression and its type \tcode{I} satisfies
  \tcode{Integral<I>} and
  \tcode{disable_\-sized_\-range<\added{remove_cvref_t<}T\added{>}>}~(\ref{std2.ranges.sized})
  is \tcode{false}.

\item
  Otherwise,
  \tcode{\textit{DECAY_COPY}(size(\removed{static_cast<const T\&>(}E\removed{)}))}
  if it is a valid expression and its type \tcode{I} satisfies \tcode{Integral<I>}
  with overload resolution performed in a context that includes the
  \added{following} declaration\added{:}
  \begin{codeblock}
  template <class T> void size(@\removed{const}@ T&@\added{\&}@) = delete;
  \end{codeblock}
  and does not include a declaration of \tcode{::std2::size}, and
  \tcode{disable_\-sized_\-range<\added{remove_cvref_t<}T\added{>}>} is \tcode{false}.

\item
  Otherwise,
  \tcode{\textit{DECAY_COPY}(::std2::\removed{c}end(E) - ::std2::\removed{c}begin(E))},
  except that \tcode{E} is only evaluated once, if it is a valid expression and
  the types \tcode{I} and \tcode{S} of \tcode{::std2::\removed{c}begin(E)} and
  \tcode{::std2\colcol{}\removed{c}end(\brk{}E)} meet the syntactic requirements
  of \tcode{SizedSentinel<S, I>}~(\ref{std2.iterators.sizedsentinel}) and
  \tcode{Forward\-Iter\-at\-or<I>}. If \tcode{SizedSentinel} and
  \tcode{Forward\-Iter\-at\-or} are not satisfied, the program is ill-formed
  with no diagnostic required.

\item
  Otherwise, \tcode{::std2::size(E)} is ill-formed.
\end{itemize}

\pnum
\enternote Whenever \tcode{::std2::size(E)} is a valid expression, its
type satisfies \tcode{Integral}. \exitnote

\rSec2[std2.range.primitives.empty]{\tcode{empty}}
\pnum
The name \tcode{empty} denotes a customization point
object~(\cxxref{customization.point.object}). The expression
\tcode{::std2::empty(E)} for some subexpression \tcode{E} is
expression-equivalent to:

\begin{itemize}
\item
  \tcode{bool((E).empty())} if it is a valid expression.

\item
  Otherwise, \tcode{::std2::size(E) == 0} if it is a valid expression.

\item
  Otherwise, \tcode{bool(::std2::begin(E) == ::std2::end(E))},
  except that \tcode{E} is only evaluated once, if it is a valid expression and the type of
  \tcode{::std2::begin(E)} satisfies \tcode{ForwardIterator}.

\item
  Otherwise, \tcode{::std2::empty(E)} is ill-formed.
\end{itemize}

\pnum
\enternote Whenever \tcode{::std2::empty(E)} is a valid expression, it
has type \tcode{bool}. \exitnote

\rSec2[std2.range.primitives.data]{\tcode{data}}
\pnum
The name \tcode{data} denotes a customization point
object~(\cxxref{customization.point.object}). The expression
\tcode{::std2::data(E)} for some subexpression \tcode{E} is
expression-equivalent to:

\begin{itemize}
\item
  \removed{\tcode{::std2::data(static_cast<const T\&>(E))} if \tcode{E} is an
  rvalue of type \tcode{T}. This usage is deprecated. \enternote
  This deprecated usage exists so that \tcode{::std2::data(E)} behaves
  similarly to \tcode{std::data(E)} as defined in the \Cpp Working
  Paper when \tcode{E} is an rvalue. \exitnote}

\item
  \changed{Otherwise}{If \tcode{E} is an lvalue},
  \tcode{\textit{DECAY_COPY}((E).data())} if it is a valid expression of pointer
  to object type.

\item
  Otherwise, \tcode{::std2::begin(E)} if it is a valid expression of pointer to
  object type.

\item
  Otherwise, \tcode{::std2::data(E)} is ill-formed.
\end{itemize}

\pnum
\enternote Whenever \tcode{::std2::data(E)} is a valid expression, it
has pointer to object type. \exitnote

\rSec2[std2.range.primitives.cdata]{\tcode{cdata}}
\pnum
The name \tcode{cdata} denotes a customization point
object~(\cxxref{customization.point.object}). The expression
\tcode{::std2::cdata(E)} for some subexpression \tcode{E} of type \tcode{T}
is expression-equivalent to\added{:}
\begin{itemize}
\item
  \tcode{::std2::data(static_cast<const T\&>(E))}\added{ if \tcode{E} is an
  lvalue}.
\item
  \added{Otherwise, \tcode{::std2::data(static_cast<const T\&\&>(E))}.}
\end{itemize}

\pnum
\removed{Use of \tcode{::std2::cdata(E)} with rvalue \tcode{E} is deprecated.
\enternote This deprecated usage exists so that \tcode{::std2::cdata(E)}
has behavior consistent with \tcode{::std2::data(E)} when \tcode{E} is
an rvalue. \exitnote}

\pnum
\enternote Whenever \tcode{::std2::cdata(E)} is a valid expression, it
has pointer to object type. \exitnote

\rSec1[std2.ranges.requirements]{Range requirements}

\ednote{As in P0896.}

\rSec1[std2.dangling.wrappers]{Dangling wrapper}

\rSec2[std2.dangling.wrap]{Class template \tcode{dangling}}

\pnum
\indexlibrary{\idxcode{dangling}}%
Class template \tcode{dangling} is a wrapper for an object that refers to another object whose
lifetime may have ended. It is used by algorithms that accept rvalue ranges and return iterators.

\begin{codeblock}
namespace std2 { inline namespace v1 {
  template <CopyConstructible T>
  class dangling {
  public:
    constexpr dangling() requires DefaultConstructible<T>;
    constexpr dangling(T t);
    constexpr T get_unsafe() const;
  private:
    T value; // \expos
  };

  template <Range R>
  using safe_iterator_t =
    conditional_t<is_lvalue_reference_v<R>,
      iterator_t<R>,
      dangling<iterator_t<R>>>;
}}
\end{codeblock}

\rSec3[std2.dangling.wrap.ops]{\tcode{dangling} operations}

\rSec4[std2.dangling.wrap.op.const]{\tcode{dangling} constructors}

\indexlibrary{\idxcode{dangling}!\idxcode{dangling}}%
\begin{itemdecl}
constexpr dangling() requires DefaultConstructible<T>;
\end{itemdecl}

\begin{itemdescr}
\pnum
\effects Constructs a \tcode{dangling}, value-initializing \tcode{value}.
\end{itemdescr}

\indexlibrary{\idxcode{dangling}!\idxcode{dangling}}%
\begin{itemdecl}
constexpr dangling(T t);
\end{itemdecl}

\begin{itemdescr}
\pnum
\effects Constructs a \tcode{dangling}, initializing \tcode{value} with \tcode{t}.
\end{itemdescr}

\rSec4[std2.dangling.wrap.op.get]{\tcode{dangling::get_unsafe}}

\indexlibrary{\idxcode{get_unsafe}!\idxcode{dangling}}%
\indexlibrary{\idxcode{dangling}!\idxcode{get_unsafe}}%
\begin{itemdecl}
constexpr T get_unsafe() const;
\end{itemdecl}

\begin{itemdescr}
\pnum
\returns \tcode{value}.
\end{itemdescr}

