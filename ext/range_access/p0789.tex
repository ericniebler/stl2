%!TEX root = D0970.tex

\part[Changes to P0789 R2]{Changes to P0789 R2\hfill[P0789]}\label{P0789}

\setcounter{chapter}{9}
\rSec0[ranges]{Ranges library}

\ednote{The following changes are suggested for P0789.}

\ednote{Change section ``Header \tcode{<experimental/ranges/range>} synopsis''
[range.synopsis], as follows:}

\begin{codeblock}
namespace std { namespace experimental { namespace ranges { inline namespace v1 {
  // ... as before

  enum class subrange_kind : bool { unsized, sized };
  // \ref{ranges.subrange}:
  template <Iterator I, Sentinel<I> S = I, subrange_kind K = see below >
      requires K == subrange_kind::sized || !SizedSentinel<S, I>
  class subrange;

  @\added{template <class I, class S, subrange_kind K>}@
    @\added{constexpr I begin(subrange<I, S, K>\&\& r);}@

  @\added{template <class I, class S, subrange_kind K>}@
    @\added{constexpr S end(subrange<I, S, K>\&\& r);}@

  // ... as before
}}}}
\end{codeblock}

\setcounter{section}{5}
\rSec1[ranges.requirements]{Range requirements}

\setcounter{subsection}{1}
\rSec2[std2.ranges.range]{Ranges}
\ednote{The equivalent change should be made in P0896 also.}

\pnum
The \tcode{Range} concept defines the requirements of a type that allows
iteration over its elements by providing a \tcode{begin} iterator and an
\tcode{end} sentinel.
\enternote Most algorithms requiring this concept simply forward to an
\tcode{Iterator}-based algorithm by calling \tcode{begin} and \tcode{end}. \exitnote

\begin{itemdecl}
template <class T>
concept bool @\changed{Range}{\textit{range-impl}}@ = @\added{// \expos}@
  requires(T&& t) {
    ranges::begin(@\added{std::forward<T>(}@t@\added{)}@); // not necessarily equality-preserving (see below)
    ranges::end(@\added{std::forward<T>(}@t@\added{)}@);
  };

@\added{template <class T>}@
@\added{concept bool Range =}@
  @\added{\textit{range-impl}<T\&>;}@

@\added{template <class T>}@
@\added{concept bool \textit{forwarding-range} = // \expos}@
  @\added{Range<T> \&\& \textit{range-impl}<T>;}@
\end{itemdecl}

\begin{itemdescr}
\pnum
Given an \changed{lvalue \tcode{t} of type \tcode{remove_reference_t<T>},
\tcode{Range<T>}}{expression \tcode{E} such that \tcode{decltype((E))} is \tcode{T},
\tcode{\textit{range-impl}<T>}} is satisfied only if

\begin{itemize}
\item \range{\added{ranges::}begin(\changed{t}{E})}{\added{ranges::}end(\changed{t}{E})} denotes a range.

\item Both \tcode{\added{ranges::}begin(\changed{t}{E})} and \tcode{\added{ranges::}end(\changed{t}{E})} are amortized constant time
and non-modifying. \enternote \tcode{\added{ranges::}begin(\changed{t}{E})} and \tcode{\added{ranges::}end(\changed{t}{E})} do not require
implicit expression variations~(\cxxref{concepts.lib.general.equality}). \exitnote

\item If \changed{\tcode{iterator_t<T>}}{the type of \tcode{ranges::begin(E)}} satisfies \tcode{ForwardIterator},
\tcode{\added{ranges::}begin(\changed{t}{E})} is equality preserving.
\end{itemize}

\begin{addedblock}
\pnum
Given an expression \tcode{E} such that \tcode{decltype((E))} is \tcode{T},
\tcode{\textit{forwarding-range}<T>} is satisfied only if
\begin{itemize}
\item
  The expressions \tcode{ranges::begin(E)} and \tcode{ranges::begin(static_cast<T\&>(E))}
  are expression-equivalent.

\item
  The expressions \tcode{ranges::end(E)} and \tcode{ranges::end(static_cast<T\&>(E))}
  are expression-equivalent.
\end{itemize}

\end{addedblock}
\end{itemdescr}

\pnum \enternote
Equality preservation of both \tcode{begin} and \tcode{end} enables passing a \tcode{Range}
whose iterator type satisfies \tcode{ForwardIterator}
to multiple algorithms and
making multiple passes over the range by repeated calls to \tcode{begin} and \tcode{end}.
Since \tcode{begin} is not required to be equality preserving when the return type does
not satisfy \tcode{ForwardIterator}, repeated calls might not return equal values or
might not be well-defined; \tcode{begin} should be called at most once for such a range.
\exitnote

\setcounter{subsection}{10}
\rSec2[ranges.viewable]{Viewable ranges}

\pnum The \tcode{ViewableRange} concept specifies the requirements of a \tcode{Range}
type that can be converted to a \tcode{View} safely.

\begin{codeblock}
template <class T>
concept bool ViewableRange =
  Range<T> && (@\changed{is_lvalue_reference_v<T>}{\textit{forwarding-range}<T>}@ || View<decay_t<T>>); @\removed{// \seebelow}@
\end{codeblock}

\begin{removedblock}
\pnum
There need not be any subsumption relationship between \tcode{ViewableRange<T>}
and \tcode{is_lvalue_reference_v<T>}.
\end{removedblock}

\rSec1[ranges.utilities]{Range utilities}

\setcounter{subsection}{1}
\rSec2[ranges.subranges]{Sub-ranges}

\rSec3[ranges.subrange]{\tcode{subrange}}

\begin{codeblock}
namespace std { namespace experimental { namespace ranges { inline namespace v1 {
  // ... as before

  @\added{template <class T, class U>}@
  @\added{concept bool \textit{not-same-as} = // \expos}@
    @\added{!Same<remove_cvref_t<T>, remove_cvref_t<U>{>};}@

  template <Iterator I, Sentinel<I> S = I, subrange_kind K = @\seebelow@>
    requires K == subrange_kind::sized || !SizedSentinel<S, I>
  class subrange : public view_interface<subrange<I, S, K>> {
  private:
    static constexpr bool StoreSize =
      K == subrange_kind::sized && !SizedSentinel<S, I>; // \expos
    I begin_ {}; // \expos
    S end_ {}; // \expos
    difference_type_t<I> size_ = 0; // \expos; only present when StoreSize is true
  public:
    using iterator = I;
    using sentinel = S;

    subrange() = default;

    constexpr subrange(I i, S s) requires !StoreSize;

    constexpr subrange(I i, S s, difference_type_t<I> n)
      requires K == subrange_kind::sized;

    @\removed{template <ConvertibleTo<I> X, ConvertibleTo<S> Y, subrange_kind Z>}@
    @\removed{constexpr subrange(subrange<X, Y, Z> r)}@
      @\removed{requires !StoreSize || Z == subrange_kind::sized;}@

    @\removed{template <ConvertibleTo<I> X, ConvertibleTo<S> Y, subrange_kind Z>}@
    @\removed{constexpr subrange(subrange<X, Y, Z> r, difference_type_t<I> n)}@
      @\removed{requires K == subrange_kind::sized;}@

    @\added{template <\textit{not-same-as}<subrange> R>}@
    @\added{requires \textit{forwarding-range}<R> \&\&}@
      @\added{ConvertibleTo<iterator_t<R>, I> \&\& ConvertibleTo<sentinel_t<R>, S>}@
    @\added{constexpr subrange(R\&\& r) requires !StoreSize || SizedRange<R>;}@

    @\added{template <\textit{forwarding-range} R>}@
    @\added{requires ConvertibleTo<iterator_t<R>, I> \&\& ConvertibleTo<sentinel_t<R>, S>}@
    @\added{constexpr subrange(R\&\& r, difference_type_t<I> n)}@
      @\added{requires K == subrange_kind::sized;}@

    template <@\changed{\textit{pair-like-convertible-to}<I, S>}{\textit{not-same-as}<subrange>}@ PairLike>
      @\added{requires \textit{pair-like-convertible-to}<PairLike, I, S>}@
    constexpr subrange(PairLike&& r) requires !StoreSize;

    template <@\textit{pair-like-convertible-to}@<I, S> PairLike>
    constexpr subrange(PairLike&& r, difference_type_t<I> n)
      requires K == subrange_kind::sized;

    @\removed{template <Range R>}@
      @\removed{requires ConvertibleTo<iterator_t<R>, I> \&\& ConvertibleTo<sentinel_t<R>, S>}@
    @\removed{constexpr subrange(R\& r) requires !StoreSize || SizedRange<R>;}@

    template <@\changed{\textit{pair-like-convertible-from}<const I\&, const S\&>}{\textit{not-same-as}<subrange>}@ PairLike>
      @\added{requires \textit{pair-like-convertible-from}<PairLike, const I\&, const S\&>}@
    constexpr operator PairLike() const;

    constexpr I begin() const;
    constexpr S end() const;
    constexpr bool empty() const;
    constexpr difference_type_t<I> size() const
      requires K == subrange_kind::sized;
    [[nodiscard]] constexpr subrange next(difference_type_t<I> n = 1) const;
    [[nodiscard]] constexpr subrange prev(difference_type_t<I> n = 1) const
      requires BidirectionalIterator<I>;
    constexpr subrange& advance(difference_type_t<I> n);
  };

  @\added{template <class I, class S, subrange_kind K>}@
    @\added{constexpr I begin(subrange<I, S, K>\&\& r);}@

  @\added{template <class I, class S, subrange_kind K>}@
    @\added{constexpr S end(subrange<I, S, K>\&\& r);}@

  template <Iterator I, Sentinel<I> S>
  subrange(I, S, difference_type_t<I>) -> subrange<I, S, subrange_kind::sized>;

  template <@\textit{iterator-sentinel-pair}@ P>
  subrange(P) ->
    subrange<tuple_element_t<0, P>, tuple_element_t<1, P>>;

  template <@\textit{iterator-sentinel-pair}@ P>
  subrange(P, difference_type_t<tuple_element_t<0, P>>) ->
    subrange<tuple_element_t<0, P>, tuple_element_t<1, P>, subrange_kind::sized>;

  @\removed{template <Iterator I, Sentinel<I> S, subrange_kind K>}@
  @\removed{subrange(subrange<I, S, K>, difference_type_t<I>) ->}@
    @\removed{subrange<I, S, subrange_kind::sized>;}@

  @\removed{template <Range R>}@
  @\removed{subrange(R\&) -> subrange<iterator_t<R>, sentinel_t<R>>;}@

  @\removed{template <SizedRange R>}@
  @\removed{subrange(R\&) -> subrange<iterator_t<R>, sentinel_t<R>, subrange_kind::sized>;}@

  @\added{template <\textit{forwarding-range} R>}@
  @\added{subrange(R\&\&) -> subrange<iterator_t<R>, sentinel_t<R>{>};}@

  @\added{template <\textit{forwarding-range} R>}@
    @\added{requires SizedRange<R>}@
  @\added{subrange(R\&\&) -> subrange<iterator_t<R>, sentinel_t<R>, subrange_kind::sized>;}@

  @\added{template <\textit{forwarding-range} R>}@
  @\added{subrange(R\&\&, difference_type_t<iterator_t<R>{>}) ->}@
    @\added{subrange<iterator_t<R>, sentinel_t<R>, subrange_kind::sized>;}@

  // ... as before

  @\added{template <Range R>}@
    @\added{using safe_subrange_t =}@
      @\added{conditional_t<\textit{forwarding-range}<R>,}@
        @\added{subrange<iterator_t<R>{>},}@
        @\added{dangling<subrange<iterator_t<R>{>}>{>};}@
}}}}
\end{codeblock}

\pnum
The default value for \tcode{subrange}'s third (non-type) template parameter is:
\begin{itemize}
\item If \tcode{SizedSentinel<S, I>} is satisfied, \tcode{subrange_kind::sized}.
\item Otherwise, \tcode{subrange_kind::unsized}.
\end{itemize}

\rSec4[ranges.subrange.ctor]{\tcode{subrange} constructors}

\indexlibrary{\idxcode{subrange}!\idxcode{subrange}}%
\begin{itemdecl}
constexpr subrange(I i, S s) requires !StoreSize;
\end{itemdecl}

\begin{itemdescr}
\pnum
\effects Initializes \tcode{begin_} with \tcode{i} and \tcode{end_} with
\tcode{s}.
\end{itemdescr}

\indexlibrary{\idxcode{subrange}!\idxcode{subrange}}%
\begin{itemdecl}
constexpr subrange(I i, S s, difference_type_t<I> n)
  requires K == subrange_kind::sized;
\end{itemdecl}

\begin{itemdescr}
\pnum
\requires \tcode{n == distance(i, s)}.

\pnum
\effects Initializes \tcode{begin_} with \tcode{i}, \tcode{end_} with
\tcode{s}. If \tcode{StoreSize} is \tcode{true}, initializes \tcode{size_} with
\tcode{n}.
\end{itemdescr}

\begin{removedblock}
\begin{itemdecl}
template <ConvertibleTo<I> X, ConvertibleTo<S> Y, subrange_kind Z>
constexpr subrange(subrange<X, Y, Z> r)
  requires !StoreSize || Z == subrange_kind::sized;
\end{itemdecl}

\begin{itemdescr}
\pnum
\effects Equivalent to:
\begin{itemize}
\item If \tcode{StoreSize} is \tcode{true},
\tcode{subrange\{r.begin(), r.end(), r.size()\}}.
\item Otherwise, \tcode{subrange\{r.begin(), r.end()\}}.
\end{itemize}
\end{itemdescr}

\begin{itemdecl}
template <ConvertibleTo<I> X, ConvertibleTo<S> Y, subrange_kind Z>
constexpr subrange(subrange<X, Y, Z> r, difference_type_t<I> n)
  requires K == subrange_kind::sized;
\end{itemdecl}

\begin{itemdescr}
\pnum
\effects Equivalent to \tcode{subrange\{r.begin(), r.end(), n\}}.
\end{itemdescr}
\end{removedblock}

\begin{addedblock}
\indexlibrary{\idxcode{subrange}!\idxcode{subrange}}%
\begin{itemdecl}
template <@\textit{not-same-as}@<subrange> R>
  requires @\textit{forwarding-range}@<R> &&
    ConvertibleTo<iterator_t<R>, I> && ConvertibleTo<sentinel_t<R>, S>
constexpr subrange(R&& r) requires !StoreSize || SizedRange<R>;
\end{itemdecl}

\begin{itemdescr}
\pnum
\effects Equivalent to:
\begin{itemize}
\item If \tcode{StoreSize} is \tcode{true},
\tcode{subrange\{ranges::begin(r), ranges::end(r), ranges::size(r)\}}.
\item Otherwise, \tcode{subrange\{ranges::begin(r), ranges::end(r)\}}.
\end{itemize}
\end{itemdescr}

\indexlibrary{\idxcode{subrange}!\idxcode{subrange}}%
\begin{itemdecl}
template <@\textit{forwarding-range}@ R>
  requires ConvertibleTo<iterator_t<R>, I> && ConvertibleTo<sentinel_t<R>, S>
constexpr subrange(R&& r, difference_type_t<I> n)
  requires K == subrange_kind::sized;
\end{itemdecl}

\begin{itemdescr}
\pnum
\effects Equivalent to \tcode{subrange\{ranges::begin(r), ranges::end(r), n\}}.
\end{itemdescr}
\end{addedblock}

\indexlibrary{\idxcode{subrange}!\idxcode{subrange}}%
\begin{itemdecl}
template <@\changed{\textit{pair-like-convertible-to}<I, S>}{\textit{not-same-as}<subrange>}@ PairLike>
  @\added{requires \textit{pair-like-convertible-to}<PairLike, I, S>}@
constexpr subrange(PairLike&& r) requires !StoreSize;
\end{itemdecl}

\begin{itemdescr}
\pnum
\effects Equivalent to:
\begin{codeblock}
subrange{get<0>(std::forward<PairLike>(r)), get<1>(std::forward<PairLike>(r))}
\end{codeblock}
\end{itemdescr}

\indexlibrary{\idxcode{subrange}!\idxcode{subrange}}%
\begin{itemdecl}
template <@\textit{pair-like-convertible-to}@<I, S> PairLike>
constexpr subrange(PairLike&& r, difference_type_t<I> n)
  requires K == subrange_kind::sized;
\end{itemdecl}

\begin{itemdescr}
\pnum
\effects Equivalent to:
\begin{codeblock}
subrange{get<0>(std::forward<PairLike>(r)), get<1>(std::forward<PairLike>(r)), n}
\end{codeblock}
\end{itemdescr}

\begin{removedblock}
\begin{itemdecl}
template <Range R>
  requires ConvertibleTo<iterator_t<R>, I> && ConvertibleTo<sentinel_t<R>, S>
constexpr subrange(R& r) requires !StoreSize || SizedRange<R>;
\end{itemdecl}

\begin{itemdescr}
\pnum
\effects Equivalent to:
\begin{itemize}
\item If \tcode{StoreSize} is \tcode{true},
\tcode{subrange\{ranges::begin(r), ranges::end(r), distance(r)\}}.
\item Otherwise,
\tcode{subrange\{ranges::begin(r), ranges::end(r)\}}.
\end{itemize}
\end{itemdescr}
\end{removedblock}

\rSec4[ranges.subrange.ops]{\tcode{subrange} operators}

\indexlibrary{\idxcode{operator \textit{PairLike}}!\idxcode{subrange}}%
\begin{itemdecl}
template <@\changed{\textit{pair-like-convertible-from}<const I\&, const S\&>}{\textit{not-same-as}<subrange>}@ PairLike>
  @\added{requires \textit{pair-like-convertible-from}<PairLike, const I\&, const S\&>}@
constexpr operator PairLike() const;
\end{itemdecl}

\begin{itemdescr}
\pnum
\effects Equivalent to: \tcode{return PairLike(begin_, end_);}.
\end{itemdescr}

\rSec4[ranges.subrange.accessors]{\tcode{subrange} accessors}

\indexlibrary{\idxcode{begin}!\idxcode{subrange}}%
\begin{itemdecl}
constexpr I begin() const;
\end{itemdecl}

\begin{itemdescr}
\pnum
\effects Equivalent to: \tcode{return begin_;}.
\end{itemdescr}

\indexlibrary{\idxcode{end}!\idxcode{subrange}}%
\begin{itemdecl}
constexpr S end() const;
\end{itemdecl}

\begin{itemdescr}
\pnum
\effects Equivalent to: \tcode{return end_;}.
\end{itemdescr}

\indexlibrary{\idxcode{empty}!\idxcode{subrange}}%
\begin{itemdecl}
constexpr bool empty() const;
\end{itemdecl}

\begin{itemdescr}
\pnum
\effects Equivalent to: \tcode{return begin_ == end_;}.
\end{itemdescr}

\indexlibrary{\idxcode{size}!\idxcode{subrange}}%
\begin{itemdecl}
constexpr difference_type_t<I> size() const
  requires K == subrange_kind::sized;
\end{itemdecl}

\begin{itemdescr}
\pnum
\effects Equivalent to:
\begin{itemize}
\item It \tcode{StoreSize} is \tcode{true}, \tcode{return size_;}.
\item Otherwise, \tcode{return end_ - begin_;}.
\end{itemize}
\end{itemdescr}

\indexlibrary{\idxcode{next}!\idxcode{subrange}}%
\begin{itemdecl}
[[nodiscard]] constexpr subrange next(difference_type_t<I> n = 1) const;
\end{itemdecl}

\begin{itemdescr}
\pnum
\effects Equivalent to:
\begin{codeblock}
auto tmp = *this;
tmp.advance(n);
return tmp;
\end{codeblock}

\pnum
\enternote If \tcode{ForwardIterator<I>} is not satisfied, \tcode{next} may
invalidate \tcode{*this}. \exitnote
\end{itemdescr}

\indexlibrary{\idxcode{prev}!\idxcode{subrange}}%
\begin{itemdecl}
[[nodiscard]] constexpr subrange prev(difference_type_t<I> n = 1) const
  requires BidirectionalIterator<I>;
\end{itemdecl}

\begin{itemdescr}
\pnum
\effects Equivalent to:
\begin{codeblock}
auto tmp = *this;
tmp.advance(-n);
return tmp;
\end{codeblock}
\end{itemdescr}

\indexlibrary{\idxcode{advance}!\idxcode{subrange}}%
\begin{itemdecl}
constexpr subrange& advance(difference_type_t<I> n);
\end{itemdecl}

\begin{itemdescr}
\pnum
\effects Equivalent to:
\begin{itemize}
\item If \tcode{StoreSize} is \tcode{true},
\begin{codeblock}
size_ -= n - ranges::advance(begin_, n, end_);
return *this;
\end{codeblock}
\item Otherwise,
\begin{codeblock}
ranges::advance(begin_, n, end_);
return *this;
\end{codeblock}
\end{itemize}
\end{itemdescr}

\rSec4[ranges.subrange.nonmember]{\tcode{subrange} non-member functions}

\begin{addedblock}
\indexlibrary{\idxcode{begin}!\idxcode{subrange}}%
\begin{itemdecl}
template <class I, class S, subrange_kind K>
  constexpr I begin(subrange<I, S, K>&& r);
\end{itemdecl}

\begin{itemdescr}
\pnum
\effects Equivalent to:
\begin{codeblock}
return r.begin();
\end{codeblock}
\end{itemdescr}

\indexlibrary{\idxcode{end}!\idxcode{subrange}}%
\begin{itemdecl}
template <class I, class S, subrange_kind K>
  constexpr S end(subrange<I, S, K>&& r);
\end{itemdecl}

\begin{itemdescr}
\pnum
\effects Equivalent to:
\begin{codeblock}
return r.end();
\end{codeblock}
\end{itemdescr}
\end{addedblock}

\indexlibrary{\idxcode{get}!\idxcode{subrange}}%
\begin{itemdecl}
template <std::size_t N, class I, class S, subrange_kind K>
  requires N < 2
constexpr auto get(const subrange<I, S, K>& r);
\end{itemdecl}

\begin{itemdescr}
\pnum
\effects Equivalent to:
\begin{codeblock}
if constexpr (N == 0)
  return r.begin();
else
  return r.end();
\end{codeblock}
\end{itemdescr}

\rSec1[ranges.adaptors]{Range adaptors}

\setcounter{subsection}{3}
\rSec2[ranges.adaptors.all]{\tcode{view::all}}

\pnum
The purpose of \tcode{view::all} is to return a \tcode{View} that includes all
elements of the \tcode{Range} passed in.

\pnum
The name \tcode{view::all} denotes a range adaptor
object~(\ref{ranges.adaptor.object}). The
expression \tcode{view::all(E)} for some subexpression \tcode{E} is
expression-equivalent to:

\begin{itemize}
\item \tcode{\textit{DECAY_COPY}(E)} if the decayed type of \tcode{E}
satisfies the concept \tcode{View}.
\item \tcode{subrange\{E\}} if \changed{\tcode{E}
is an lvalue and has a type that satisfies concept \tcode{Range}}{that
expression is well-formed}.
\item Otherwise, \tcode{view::all(E)} is ill-formed.
\end{itemize}

\remark Whenever \tcode{view::all(E)} is a valid expression, it is a prvalue
whose type satisfies \tcode{View}.

\rSec0[algorithms]{Algorithms library}

\ednote{Some of the algorithms in the Ranges TS (\tcode{rotate} and \tcode{equal_range})
actually return subranges, but they do so using \tcode{tagged_pair}. With the addition
of a proper \tcode{subrange} type, we suggest changing these algorithms to return
\tcode{subrange}.}

\rSec1[algorithms.general]{General}

\ednote{Change the \tcode{<experimental/ranges/algorithm>} synopsis as follows:}

\begin{codeblock}
#include <initializer_list>

namespace std { namespace experimental { namespace ranges { inline namespace v1 {
  // ... as before

  template <ForwardIterator I, Sentinel<I> S>
    requires Permutable<I>
    @\removed{tagged_pair<tag::begin(I), tag::end(I)>}@
    @\added{subrange<I>}@
      rotate(I first, I middle, S last);

  template <ForwardRange Rng>
    requires Permutable<iterator_t<Rng>>
    @\removed{tagged_pair<tag::begin(safe_iterator_t<Rng>),}@
                @\removed{tag::end(safe_iterator_t<Rng>)>}@
    @\added{safe_subrange_t<Rng>}@
      rotate(Rng&& rng, iterator_t<Rng> middle);

  // ... as before

  template <ForwardIterator I, Sentinel<I> S, class T, class Proj = identity,
      IndirectStrictWeakOrder<const T*, projected<I, Proj>> Comp = less<>>
    @\removed{tagged_pair<tag::begin(I), tag::end(I)>}@
    @\added{subrange<I>}@
      equal_range(I first, S last, const T& value, Comp comp = Comp{}, Proj proj = Proj{});

  template <ForwardRange Rng, class T, class Proj = identity,
      IndirectStrictWeakOrder<const T*, projected<iterator_t<Rng>, Proj>> Comp = less<>>
    @\removed{tagged_pair<tag::begin(safe_iterator_t<Rng>),}@
                @\removed{tag::end(safe_iterator_t<Rng>)>}@
    @\added{safe_subrange_t<Rng>}@
      equal_range(Rng&& rng, const T& value, Comp comp = Comp{}, Proj proj = Proj{});

  // ... as before
}}}}
\end{codeblock}

\setcounter{section}{3}

\rSec1[alg.modifying.operations]{Mutating sequence operations}

\setcounter{subsection}{10}

\rSec2[alg.rotate]{Rotate}

\indexlibrary{\idxcode{rotate}}%
\begin{itemdecl}
template <ForwardIterator I, Sentinel<I> S>
  requires Permutable<I>
  @\removed{tagged_pair<tag::begin(I), tag::end(I)>}@
  @\added{subrange<I>}@
    rotate(I first, I middle, S last);

template <ForwardRange Rng>
  requires Permutable<iterator_t<Rng>>
  @\removed{tagged_pair<tag::begin(safe_iterator_t<Rng>),}@
              @\removed{tag::end(safe_iterator_t<Rng>)>}@
  @\added{safe_subrange_t<Rng>}@
    rotate(Rng&& rng, iterator_t<Rng> middle);
\end{itemdecl}

\begin{itemdescr}
\pnum
\effects
For each non-negative integer
\tcode{i < (last - first)},
places the element from the position
\tcode{first + i}
into position
\tcode{first + (i + (last - middle)) \% (last - first)}.

\pnum
\returns \tcode{\{first + (last - middle), last\}}.

\pnum
\notes
This is a left rotate.

\pnum
\requires
\range{first}{middle}
and
\range{middle}{last}
shall be valid ranges.

\pnum
\complexity
At most
\tcode{last - first}
swaps.
\end{itemdescr}

\rSec1[alg.sorting]{Sorting and related operations}

\setcounter{subsection}{2}

\rSec2[alg.binary.search]{Binary search}

\setcounter{subsubsection}{2}

\rSec3[equal.range]{\tcode{equal_range}}

\indexlibrary{\idxcode{equal_range}}%
\begin{itemdecl}
template <ForwardIterator I, Sentinel<I> S, class T, class Proj = identity,
    IndirectStrictWeakOrder<const T*, projected<I, Proj>> Comp = less<>>
  @\removed{tagged_pair<tag::begin(I), tag::end(I)>}@
  @\added{subrange<I>}@
    equal_range(I first, S last, const T& value, Comp comp = Comp{}, Proj proj = Proj{});

template <ForwardRange Rng, class T, class Proj = identity,
    IndirectStrictWeakOrder<const T*, projected<iterator_t<Rng>, Proj>> Comp = less<>>
  @\removed{tagged_pair<tag::begin(safe_iterator_t<Rng>),}@
              @\removed{tag::end(safe_iterator_t<Rng>)>}@
  @\added{safe_subrange_t<Rng>}@
    equal_range(Rng&& rng, const T& value, Comp comp = Comp{}, Proj proj = Proj{});
\end{itemdecl}

\begin{itemdescr}
\pnum
\requires
The elements
\tcode{e}
of
\range{first}{last}
shall be partitioned with respect to the expressions
\tcode{invoke(comp, invoke(proj, e), value)}
and
\tcode{!invoke(\brk{}comp, value, invoke(proj, e))}.
Also, for all elements
\tcode{e}
of
\tcode{[first, last)},
\tcode{invoke(comp, invoke(proj, e), value)}
shall imply \\
\tcode{!invoke(\brk{}comp, value, invoke(proj, e))}.

\pnum
\returns
\begin{codeblock}
{lower_bound(first, last, value, comp, proj),
 upper_bound(first, last, value, comp, proj)}
\end{codeblock}

\pnum
\complexity
At most
$2 * \log_2(\tcode{last - first}) + \bigoh{1}$
applications of the comparison function and projection.
\end{itemdescr}
