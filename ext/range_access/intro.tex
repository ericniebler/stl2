\rSec0[intro]{General}

\begin{quote}
``Begin at the beginning, the King said, very gravely, and go on till you come to the end: then stop.''
\begin{flushright}
\textemdash \textit{Lewis Carroll}
\end{flushright}
\end{quote}

\begin{addedblock}
\rSec1[intro.history]{Revision history}

\rSec2[intro.history.rev1]{Revision 1}
This paper has been rebased on P0789R3~(\cite{P0789}) and P0896R1~(\cite{P0896})
(and hence C++20).
\end{addedblock}

\rSec1[intro.scope]{Scope}

\pnum
This document suggests improvements to the range access customization points
(\tcode{begin}, \tcode{end}, \textit{et.al.}) of ISO/IEC TS 21425:2017, otherwise
known as the Ranges TS. The improvements suggested here apply to P0896 R1,
``Merging the Ranges TS''~(\cite{P0896}), and to P0789 R3, ``Range Adaptors and
Utilities''~(\cite{P0789}).

\rSec1[intro.problem]{Problems with \tcode{std::ranges::begin}}

\pnum
For the sake of compatibility with \tcode{std::begin} and ease of migration,
\tcode{std::\oldtxt{experimental::}ranges::begin} accepted rvalues and treated them the
same as \tcode{const} lvalues. This behavior was deprecated because it is
fundamentally unsound: any iterator returned by such an overload is highly
likely to dangle after the full expression that contained the invocation of
\tcode{begin}.

\pnum
Another problem, and one that until recently seemed unrelated to the design of
\tcode{begin}, was that algorithms that return iterators will wrap those
iterators in \tcode{std::\oldtxt{experimental::}ranges::dangling<>} if the range passed
to them is an rvalue. This ignores the fact that for some range types ---
\newtxt{\tcode{std::span}, \tcode{std::string_view}, and} P0789's \tcode{subrange},
in particular --- the iterator's validity does not
depend on the range's lifetime at all. In the case where \oldtxt{a p}\newtxt{an }rvalue
\oldtxt{\tcode{subrange<>}}\newtxt{of one of the above types} is passed to an algorithm,
returning a wrapped iterator is totally unnecessary.

\pnum
The author believed that to fix the problem with \tcode{subrange} and
\tcode{dangling} would require the addition of a new trait to give the authors
of range types a way to say whether its iterators can safely outlive the range.
That felt like a hack\oldtxt{, and that feeling was reinforced by the author's inability
to pick a name for such a trait that was sufficiently succint and clear}.

\rSec1[intro.design]{Suggested Design}

\pnum
We recognized that by removing the deprecated default support for rvalues from
the range access customization points, we made design space for range authors to
opt-in to this behavior for their range types, thereby communicating to the
algorithms that an iterator can safely outlive its range type. This eliminates
the need for \tcode{dangling} when passing an rvalue \newtxt{\tcode{std::span}, 
\tcode{std::string_view}, or} \tcode{subrange}, an important usage scenario.

\pnum
This improved design would be both safer and more expressive: users should be
unable to pass to \oldtxt{\tcode{std2}}\ \newtxt{\tcode{std::ranges}}\tcode{::begin}
any rvalue range unless its result is guaranteed to not dangle.

\pnum
The mechanics of this change are subtle. There are two typical ways for making
a type satisfy the \tcode{Range} concept:

\begin{enumerate}
\item Give the type \tcode{begin()} and \tcode{end()} member functions
(typically not lvalue reference-qualified), as below:
\begin{codeblock}
struct Buffer {
  char* begin();
  const char* begin() const;
  char* end();
  const char* end() const;
};
\end{codeblock}
\item Define \tcode{begin} and \tcode{end} as free functions, typically
overloaded for \tcode{const} and non-\tcode{const} lvalue references, as shown
below:
\begin{codeblock}
struct Buffer { /*...*/ };

char* begin(Buffer&);
const char* begin(const Buffer&);
char* end(Buffer&);
const char* end(const Buffer&);
\end{codeblock}
\end{enumerate}

\pnum
These approaches offer few clues as to whether iterators yielded from this range
will remain valid even the range itself has been destroyed. With the first,
\tcode{Buffer\{\}.begin()} compiles successfully. Likewise, with the second,
\tcode{begin(Buffer\{\})} is also well-formed. Neither yields any useful
information.

\pnum
The design presented in this paper takes a two-pronged approach:

\begin{enumerate}
\item \tcode{\oldtxt{std2}\newtxt{std::ranges}::begin(E)} never considers
\tcode{E.begin()} unless \tcode{E} is an lvalue.
\item \tcode{\oldtxt{std2}\newtxt{std::ranges}::begin(E)} will consider an overload of
\tcode{begin(E)} found by ADL, looked up in a context that (a) does not include
\tcode{\oldtxt{std2}\newtxt{std::ranges}::begin}, and (b) includes the following
declaration:
\begin{codeblock}
// "Poison pill" overload:
template <class T>
void begin(T&&) = delete;
\end{codeblock}
\end{enumerate}

This approach gives \tcode{\oldtxt{std2}\newtxt{std::ranges}::begin} the property that,
for some rvalue expression \tcode{E} of type \tcode{T}, the expression
\tcode{\oldtxt{std2}\newtxt{std::ranges}::begin(E)}
will not compile unless there is a free function \tcode{begin} findable by
ADL that specifically accepts rvalues of type \tcode{T}, and that overload is
prefered by partial ordering over the general \tcode{void begin(T\&\&)} ``poison
pill'' overload.

\pnum
This design has the following benefits:

\begin{itemize}
\item No iterator returned from \tcode{\oldtxt{std2}\newtxt{std::ranges}::begin(E)}
can dangle, even if \tcode{E} is an rvalue expression.
\item Authors of simple view types for which iterators may safely outlive the
range (like P0789's \tcode{subrange<>}) may denote such support by providing an
overload of \tcode{begin} that accepts rvalues.
\end{itemize}

\pnum
Once \tcode{\oldtxt{std2}\newtxt{std::ranges}::begin}, \tcode{end}, and friends have
been redefined as described above, the \tcode{safe_iterator_t} alias template
can be redefined to only wrap an iterator in \tcode{dangling<>} for a
\tcode{Range} type \tcode{R} if
\tcode{\oldtxt{std2}\newtxt{std::ranges}::begin(std::declval<R>())} is ill-formed. In
code:

\begin{codeblock}
template <Range R@\oldtxt{, class = void}@>
struct __safe_iterator {
  using type = dangling<iterator_t<R>>;
};
template <class R>
  @\newtxt{requires requires (R\&\& r) \{ std::ranges::begin((R\&\&) r); \}}@
struct __safe_iterator<R@\oldtxt{, void_t<decltype(std2::begin(declval<R>()))>}@> {
  using type = iterator_t<R>;
};

template <Range R>
using safe_iterator_t = typename __safe_iterator<R>::type;
\end{codeblock}

Now algorithms that accept \tcode{Range} parameters by forwarding reference and
that return iterators into that range can simply declare their return type as
\tcode{safe_iterator_t<R>} and have that iterator wrapped only if it can dangle.

\rSec1[intro.refs]{References}

\pnum
The following referenced documents are indispensable for the
application of this document. For dated references, only the
edition cited applies. For undated references, the latest edition
of the referenced document (including any amendments) applies.

\begin{itemize}
\item ISO/IEC 14882:2017, \doccite{Programming Languages - \Cpp}
\item ISO/IEC TS 21425:2017, \doccite{Technical Specification - \Cpp Extensions for Ranges}
\end{itemize}

ISO/IEC 14882:2017 is herein called the \defn{C\Rplus\Rplus\xspace Standard} and
ISO/IEC TS 21425:2017 is called the \defn{Ranges TS}.
