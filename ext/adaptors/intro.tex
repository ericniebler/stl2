\rSec0[intro]{General}

\begin{quote}
``Adopt your own view and adapt with others' views.''
\begin{flushright}
\textemdash \textit{Mohammed Sekouty}
\end{flushright}
\end{quote}

\rSec1[intro.scope]{Scope}

\pnum
This document provides extensions to the Ranges TS~\cite{ranges-ts} to
support the creation of pipelines of range transformations. In particular,
changes and extensions to the Ranges TS include:

\begin{itemize}
\item A \tcode{subrange} type that stores an
iterator/sentinel pair and satisfies the requirements of the \tcode{View} concept.
\item A \tcode{view::all} range adaptor that turns a \tcode{Range} into a
\tcode{View} while respecting memory safety.
\item A \tcode{view::filter} range adaptor that accepts a \tcode{Range} and a
\tcode{Predicate} and returns a \tcode{View} of the underlying range that skips
those elements that fail to satisfy the predicate.
\item A \tcode{view::transform} range adaptor that accepts a \tcode{Range} and a
unary \tcode{Invocable} and produces a view that applies the invocable to each
element of the underlying range.
\item A \tcode{view::iota} range that takes a \tcode{WeaklyIncrementable} and
yields a range of elements produced by incrementing the initial element
monotonically. Optionally, it takes an upper bound at which to stop.
\item A \tcode{view::empty} range that creates an empty range of a certain
element type.
\item A \tcode{view::single} range that creates a range of cardinality 1 with
the specified element.
\item A \tcode{view::join} range adaptor takes a range of ranges,
and lazily flattens the ranges into one range.
\item A \tcode{view::split} range adaptor takes a range and a delimiter,
and lazily splits the range into subranges on the delimiter. The delimiter may
be either an element or a subrange.
\item A \tcode{view::counted} range adaptor that takes an iterator and a count
of elements, and returns a range of that many elements starting at the one
denoted by the iterator.
\item A \tcode{view::common} range adaptor that takes a range for which the
iterator and sentinel types differ, and returns a range for which the iterator
and sentinel types are the same.
\item A \tcode{view::reverse} range adaptor that takes a bidirectional range and
returns a new range that iterates the elements in reverse order.
\end{itemize}

\rSec1[intro.design]{Design Considerations}

\pnum
The Ranges position paper, N4128 ``Ranges for the Standard Library,
Revision 1''~(\cite{N4128}), contains extensive motivation and design
considerations. That paper explains why the ranges design distinguishes between
``\tcode{Range}'' and ``\tcode{View}'' (called ``\tcode{Iterable}'' and
``\tcode{Range}'' in that paper). This section calls out specific parts of the
adaptors and utilities design that might be of particular interest. 

\rSec2[intro.filter]{The \tcode{filter} adaptor is not \tcode{const}-iterable}

\pnum
N4128 \S\-3.3.10 discusses how just because a type \tcode{T} satisfies \tcode{Range} does
not imply that the type \tcode{const T} satisfies \tcode{Range}. It gives the
example of an \tcode{istream_range}, which reads each value from a stream and
stores it in a private cache. Since the range is mutated while it is iterated,
its \tcode{begin} and \tcode{end} member functions cannot be \tcode{const}. The
\tcode{filter} adaptor is a similar case, but it is not immediately obvious why.

\pnum
According to the semantic requirements of the \tcode{Range} concept, \tcode{begin}
and \tcode{end} must be amortized constant-time operations. That means that
repeated calls to \tcode{begin} or \tcode{end} on the same range will be fast,
a property that a great many adaptors take advantage of, freeing them from the
need to pessimistically cache the results of these operations themselves.

\pnum
The \tcode{filter} view, which skips elements that fail to satisfy a predicate,
needs to do an \bigoh{N} probe to find the first element that satisfies the
predicate so that \tcode{begin} can return it. The options are:

\begin{enumerate}
\item \textbf{Compute this position on adaptor construction.} This solution has
multiple problems. First, constructing an adaptor should be \bigoh{1}. Doing an
\bigoh{N} probe obviously conflicts with that. Second, the probe would return a
position in the source range, but when the \tcode{filter} view is copied, the
iterator becomes invalid, lest it be left pointing to an element in the source
range. That means that copies and moves of the \tcode{filter} view would need to
invalidate the cache and perform \textit{another} \bigoh{N} probe to find the first
element of the filtered range. \bigoh{N} copy and move operations make it difficult
to reason about the cost of building adaptor pipelines.
\item \textbf{Recompute the first position on each invocation of \tcode{begin}.} This is
obviously in conflict with the semantic requirements of the \tcode{Range} concept,
which specifies that \tcode{begin} is amortized constant-time.
\item \textbf{Compute the first position once in \tcode{begin} on demand and cache
the result, with synchronization.} Taking a lock in the \tcode{begin} member
function in order to update an internal cache permits that operation to be
\tcode{const} while satisfying [res.on.data.races], but obviously incurs overhead
and violates the ``don't pay for what you don't use'' mantra.
\item \textbf{Compute the first position once in \tcode{begin} on demand and cache
the result, without synchronization.} The downside of this approach is that
\tcode{begin} cannot be \tcode{const} without violating [res.on.data.races].
\end{enumerate}

\pnum
None of these are great options, and this particular design point has been
discussed at extensive length~(see
\href{https://github.com/ericniebler/range-v3/issues/385}{range-v3\#385})
in the context of the \tcode{filter} view and an assortment of other adaptors
that are unable to provide \tcode{const} iteration. The general consensus is
that option (4) above is the least bad option, and is consistent with the
perspective that adaptors are lazily-evaluated algorithms: some algorithms can
be implemented without the need to maintain mutable state. Others cannot.

\rSec2[intro.join]{The \tcode{join} view is only sometimes \tcode{const}-iterable}

\pnum
As with the \tcode{filter} view, the \tcode{join} view must maintain internal
state as it is being iterated. Since the \tcode{join} view takes a range of
ranges and presents a flattened view, it uses two iterators to denote each
position: an iterator into the outer range and an iterator into the inner range.

\pnum
If the outer range is generating the inner ranges on the fly (that is, if
dereferencing the outer iterator yields a prvalue inner range), that range must
be stored somewhere while it is being iterated. The obvious place to store it
is within the \tcode{join_view} object itself. Each time the outer iterator is
incremented, this inner range object must be updated. This makes the
\tcode{join_view} non-\tcode{const}-iterable, just like the \tcode{filter} view.

\pnum
However, if the result of dereferencing the outer iterator is a glvalue, then
we know the inner range object is reified in memory somewhere. Rather than store
a copy of the inner range object within the \tcode{join_view}, we can simply
assume the inner range will persist long enough for the inner iterator to
traverse it. Additionally, we can dereference the outer iterator whenever we
need to access the inner range object.

\pnum
For this reason, the \tcode{join} view is \textit{sometimes}
\tcode{const}-iterable, and the constraints on the \tcode{const} overloads of
its \tcode{begin} and \tcode{end} member functions reflect that.

\rSec2[intro.reverse]{The \tcode{reverse} view is only sometimes \tcode{const}-iterable}

\pnum
As with \tcode{filter}, the \tcode{reverse} view needs to cache the end of the
range so that \tcode{begin} can return it in amortized \bigoh{1}. The exception
is when adapting a \tcode{CommonRange}; that is, a range for which \tcode{end}
returns an iterator. As a result, the \tcode{reverse} view is only
\tcode{const}-iterable when adapting a \tcode{CommonRange}.

\rSec2[intro.iota.deduction]{\tcode{iota} view type deduction}

\pnum
The \tcode{iota} view takes an incrementable and (optionally) an upper bound,
and returns a range of all the elements reachable from the start (inclusive) to
the bound (exclusive). The bound defaults to an unreachable sentinel, yielding
an infinite range.

\pnum
The bound need not have the same type as the iterable, which permits \tcode{iota}
to work with iterator/sentinel pairs. However, that also opens the door to
integral signed/unsigned mismatch bugs, like \tcode{view::iota(0, v.size())},
where \tcode{0} is a (signed) \tcode{int} and \tcode{v.size()} is an (unsigned)
\tcode{std::size_t}.

\pnum
The deduction guides as currently specified permit the bound to have a different
type as the incrementable \textit{unless} both the incrementable and the bound
are integral types with different signedness.

\rSec2[intro.iota.indices]{\tcode{iota(N)} is an infinite range}

\pnum
There appears to be an expectation among some programmers that a single-argument
invocation of \tcode{iota} such as \tcode{view::iota(10)} produces a 10 element
range: \tcode{0} through \tcode{9} inclusive. This leads to bug reports such as
\href{https://github.com/ericniebler/range-v3/issues/277}{range-v3\#277}, where
the behavior of the \tcode{iota} adaptor is compared unfavorably to similar
facilities in other languages, which provide the desired (by the submitter)
behavior.

\pnum
There are two reasons for the behavior as specified:

\begin{itemize}
\item Consistency with the \tcode{std::iota} numeric algorithm, which accepts a
single incrementable value and fills a range with that value and its successors, and
\item Compatability with non-Integral incrementables. When the incrementable is
non-Integral, as with an iterator, it is nonsensical to interpret the single
argument as an upper bound, since there is no ``zero'' iterator that can be
treated as the lower bound.
\end{itemize}

\pnum
So what to do about the confusion about \tcode{view::iota(10)}? We see three
possibilities:

\begin{enumerate}
\item Disallow it.
\item Disallow it for Integral arguments.
\item Permit it and educate users.
\end{enumerate}

The authors have opted for (3), to permit the usage. We further note that an
adaptor that works \textit{only} with Integral types (\tcode{view::indices},
perhaps) could make a different choice about the interpretation of a
single-argument form.


\rSec1[intro.refs]{References}

\pnum
The following referenced documents are indispensable for the
application of this document. For dated references, only the
edition cited applies. For undated references, the latest edition
of the referenced document (including any amendments) applies.

\begin{itemize}
\item ISO/IEC 14882:2017, \doccite{Programming Languages - \Cpp}
\item ISO/IEC TS 21425:2017, \doccite{Technical Specification - \Cpp Extensions for Ranges}
\end{itemize}

ISO/IEC 14882:2017 is herein called the \defn{C\Rplus\Rplus\xspace Standard} and
ISO/IEC TS 21425:2017 is called the \defn{Ranges TS}.

\rSec1[intro.compliance]{Implementation compliance}

\pnum
Conformance requirements for this specification are the same as those
defined in \ref{intro.compliance} in the \Cpp Standard.
\enternote
Conformance is defined in terms of the behavior of programs.
\exitnote

\begin{removedblock}
\rSec1[intro.namespaces]{Namespaces, headers, and modifications to standard classes}

\pnum
Since the extensions described in this document are experimental additions to the Ranges TS,
everything defined herein is declared within namespace \tcode{std::experimental::ranges::v1}.

\pnum
Unless otherwise specified, references to entities described in this
document or the Ranges TS are assumed to be qualified with \tcode{::std::experimental::ranges::}, and
references to entities described in the International Standard are assumed to be
qualified with \tcode{::std::}.
\end{removedblock}