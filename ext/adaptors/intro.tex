\rSec0[intro]{General}

\begin{quote}
``Adopt your own view and adapt with others' views.''
\begin{flushright}
\textemdash \textit{Mohammed Sekouty}
\end{flushright}
\end{quote}

\rSec1[intro.scope]{Scope}

\ednote{For motivation and design considerations, please refer to N4128,
``Ranges for the Standard Library, Revision 1''~(\cite{N4128}).}

\pnum
This document provides extensions to the Ranges TS~\cite{ranges-ts} to
support the creation of pipelines of range transformations. In particular,
changes and extensions to the Ranges TS include:

\begin{itemize}
\item \changed{An \tcode{iterator_range}}{A \tcode{subrange}} type that stores an
iterator/sentinel pair and satisfies the requirements of the \tcode{View} concept.
{\color{remclr}
\item \removed{A \tcode{sized_iterator_range} type that stores an iterator/sentinel pair
and a size, and satisfies the requirements of both the \tcode{View} and
\tcode{SizedRange} concepts.}}
\item A \tcode{view::all} range adaptor that turns a \tcode{Range} into a
\tcode{View} while respecting memory safety.
\item A \tcode{view::filter} range adaptor that accepts a \tcode{Range} and a
\tcode{Predicate} and returns a \tcode{View} of the underlying range that skips
those elements that fail to satisfy the predicate.
\item A \tcode{view::transform} range adaptor that accepts a \tcode{Range} and a
unary \tcode{Invocable} and produces a view that applies the invocable to each
element of the underlying range.
\item A \tcode{view::iota} range that takes a \tcode{WeaklyIncrementable} and
yields a range of elements produced by incrementing the initial element
monotonically. Optionally, it takes an upper bound at which to stop.
\item A \tcode{view::emtpy} range that creates an empty range of a certain
element type.
\item A \tcode{view::single} range that creates a range of cardinality 1 with
the specified element.
\item A \tcode{view::join} range adaptor takes a range of ranges,
and lazily flattens the ranges into one range.
\item A \tcode{view::split} range adaptor takes a range and a delimiter,
and lazily splits the range into subranges on the delimiter. The delimiter may
be either an element or a subrange.
\end{itemize}

\rSec1[intro.refs]{References}

\pnum
The following referenced documents are indispensable for the
application of this document. For dated references, only the
edition cited applies. For undated references, the latest edition
of the referenced document (including any amendments) applies.

\begin{itemize}
\item ISO/IEC 14882:2017, \doccite{Programming Languages - \Cpp}
\item JTC1/SC22/WG21 N4685, \doccite{Technical Specification - \Cpp Extensions for Ranges}
\end{itemize}

ISO/IEC 14882:2017 is herein called the \defn{C\Rplus\Rplus\xspace Standard} and N4685 is called
the \defn{Ranges TS}.

\rSec1[intro.compliance]{Implementation compliance}

\pnum
Conformance requirements for this specification are the same as those
defined in \ref{intro.compliance} in the \Cpp Standard.
\enternote
Conformance is defined in terms of the behavior of programs.
\exitnote

\rSec1[intro.namespaces]{Namespaces, headers, and modifications to standard classes}

\pnum
Since the extensions described in this document are experimental additions to the Ranges TS,
everything defined herein is declared within namespace \tcode{std::experimental::ranges::v1}.

\pnum
Unless otherwise specified, references to other entities described in this
document are assumed to be qualified with \tcode{std::experimental::ranges::}, and
references to entities described in the International Standard are assumed to be
qualified with \tcode{std::}.
