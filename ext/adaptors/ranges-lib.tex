%!TEX root = D0789R1.tex

\setcounter{chapter}{9}
\rSec0[ranges]{Ranges library}

\ednote{To the section ``Header \tcode{<experimental/ranges/range>} synopsis''
10.3 [range.synopsis], add the following:}

\begin{codeblock}
namespace std { namespace experimental { namespace ranges { inline namespace v1 {
  @\added{// \ref{ranges.viewable}:}@
  @\added{template <class T>}@
  @\added{concept bool ViewableRange = \seebelow;}@

  // \ref{ranges.view_interface}:
  template <class D>
  class view_interface;

  @\removed{// 10.7.2.1:}@
  @\removed{template <Iterator I, Sentinel<I> S = I>}@
  @\removed{class iterator_range;}@

  @\removed{// 10.7.2.2:}@
  @\removed{template <Iterator I, Sentinel<I> S = I>}@
  @\removed{class sized_iterator_range;}@

  @\added{enum class subrange_kind : bool \{ unsized, sized \}; }@

  @\added{// \ref{ranges.subrange}:}@
  @\added{template <Iterator I, Sentinel<I> S = I, subrange_kind Sized = \seebelow>}@
  @\added{class subrange;}@

  // \ref{ranges.adaptors.all}:
  namespace view { inline constexpr @\unspec@ all = @\unspec@; }

  template <@\changed{Range}{ViewableRange}@ R>
    @\removed{requires is_lvalue_reference_v<R> || View<decay_t<R>>}@
  using all_view = @\removed{decay_t<}@decltype(view::all(declval<R>()))@\removed{>}@;

  // \ref{ranges.adaptors.filter_view}:
  template <InputRange R, IndirectUnaryPredicate<iterator_t<R>> Pred>
    requires View<R>
  class filter_view;

  namespace view { inline constexpr @\unspec@ filter = @\unspec@; }

  // \ref{ranges.adaptors.transform_view}:
  template <InputRange R, CopyConstructible F>
    requires View<R> && Invocable<F&, reference_t<iterator_t<R>>>
  class transform_view;

  namespace view { inline constexpr @\unspec@ transform = @\unspec@; }

  // \ref{ranges.adaptors.iota_view}:
  template <WeaklyIncrementable I, Semiregular Bound = unreachable>
    requires WeaklyEqualityComparable<I, Bound>
  class iota_view;

  namespace view { inline constexpr @\unspec@ iota = @\unspec@; }

  // \ref{ranges.adaptors.join_view}:
  template <InputRange R>
    requires View<R> && InputRange<reference_t<iterator_t<R>>> &&
      (is_reference_v<reference_t<iterator_t<R>>> ||
        View<value_type_t<iterator_t<R>>>)
  class join_view;

  namespace view { inline constexpr @\unspec@ join = @\unspec@; }

  // \ref{ranges.adaptors.empty_view}:
  template <class T>
    requires @\changed{requires \{ *(T*)nullptr; \}}{is_object_v<T>}@
  class empty_view;

  namespace view {
    template <class T>
    inline constexpr empty_view<T> empty {};
  }

  // \ref{ranges.adaptors.single_view}:
  template <CopyConstructible T>
  class single_view;

  namespace view { inline constexpr @\unspec@ single = @\unspec@; }

  // \expos
  template <class R>
  concept bool @\textit{tiny-range}@ = @\seebelow@;

  // \ref{ranges.adaptors.split_view}:
  template <InputRange Rng, ForwardRange Pattern>
    requires View<Rng> && View<Pattern> &&
        IndirectlyComparable<iterator_t<Rng>, iterator_t<Pattern>> &&
        (ForwardRange<Rng> || @\textit{tiny-range}@<Pattern>)
  @\changed{struct}{class}@ split_view;

  namespace view { inline constexpr @\unspec@ split = @\unspec@; }

  // \ref{ranges.adaptors.counted}:
  namespace view { inline constexpr @\unspec@ counted = @\unspec@; }

  @\added{// \ref{ranges.adaptors.bounded_view}:}@
  @\added{template <View Rng>}@
    @\added{requires !BoundedRange<Rng>}@
  @\added{class bounded_view;}@

  @\added{namespace view \{ inline constexpr \unspec bounded = \unspec; \}}@

  @\added{// \ref{ranges.adaptors.reverse_view}:}@
  @\added{template <View Rng>}@
    @\added{requires BidirectionalRange<Rng>}@
  @\added{class reverse_view;}@

  @\added{namespace view \{ inline constexpr \unspec reverse = \unspec; \}}@
}}}}

namespace std {
  template <class I, class S@\added{, ranges::subrange_kind B}@>
    struct tuple_size<ranges::@\changed{iterator_range}{subrange}@<I, S@\added{, B}@>>
      : std::integral_constant<size_t, 2> {};
  template <class I, class S@\added{, ranges::subrange_kind B}@>
    struct tuple_element<0, ranges::@\changed{iterator_range}{subrange}@<I, S@\added{, B}@>> {
      using type = I;
    };
  template <class I, class S@\added{, ranges::subrange_kind B}@>
    struct tuple_element<1, ranges::@\changed{iterator_range}{subrange}@<I, S@\added{, B}@>> {
      using type = S;
    };

  @\removed{template <class I, class S>}@
    @\removed{struct tuple_size<ranges::sized_iterator_range<I, S>{>}}@
      @\removed{: std::integral_constant<size_t, 3> \{\};}@
  @\removed{template <class I, class S>}@
    @\removed{struct tuple_element<0, ranges::sized_iterator_range<I, S>{>} \{}@
      @\removed{using type = I;}@
    @\removed{\};}@
  @\removed{template <class I, class S>}@
    @\removed{struct tuple_element<1, ranges::sized_iterator_range<I, S>{>} \{}@
      @\removed{using type = S;}@
    @\removed{\};}@
  @\removed{template <class I, class S>}@
    @\removed{struct tuple_element<2, ranges::sized_iterator_range<I, S>{>} \{}@
      @\removed{using type = ranges::difference_type_t<I>;}@
    @\removed{\};}@
}
\end{codeblock}

\ednote{After Ranges TS subclause 10.6 [ranges.requirements], insert a new subclause
10.7, ``Range utilities'' with stable name [ranges.utilities]}

\setcounter{section}{6}
\setcounter{subsection}{10}
{\color{addclr}
\rSec2[ranges.viewable]{Viewable ranges}
\pnum The \tcode{ViewableRange} concept specifies the requirements of a \tcode{Range}
type that can be converted to a \tcode{View} safely.

\begin{codeblock}
template <class T>
concept bool ViewableRange =
  Range<T> && (is_lvalue_reference_v<T> || View<decay_t<T>>); // \seebelow
\end{codeblock}

\pnum
There need not be any subsumption relationship between \tcode{ViewableRange<T>}
and \tcode{is_lvalue_reference_v<T>}.
}

\rSec1[ranges.utilities]{Range utilities}

\pnum
The components in this section are general utilities for representing and manipulating
ranges.

\rSec2[ranges.view_interface]{View interface}

\pnum
The \tcode{view_interface} is a helper for defining \tcode{View}-like types that offer a
container-like interface. It is parameterized with the type that inherits from it.

\indexlibrary{\idxcode{view_interface}}%
\begin{codeblock}
namespace std { namespace experimental { namespace ranges { inline namespace v1
{
  // \expos
  template <Range R>
  struct @\textit{range-common-iterator-impl}@ {
    using type = common_iterator<iterator_t<R>, sentinel_t<R>>;
  };
  template <BoundedRange R>
  struct @\textit{range-common-iterator-impl}@<R> {
    using type = iterator_t<R>;
  };
  template <Range R>
    using @\textit{range-common-iterator}@ =
      typename @\textit{range-common-iterator-impl}@<R>::type;

  template <class D>
  class view_interface : @\added{public}@ view_base {
  private:
    constexpr D& derived() noexcept { // \expos
      return static_cast<D&>(*this);
    }
    constexpr const D& derived() const noexcept { // \expos
      return static_cast<const D&>(*this);
    }
  public:
    constexpr bool empty() const requires ForwardRange<const D>;
    constexpr explicit operator bool() const
      requires @\changed{ForwardRange<const D>}{requires \{ ranges::empty(derived()); \}}@;
    @\removed{constexpr bool operator!() const requires ForwardRange<const D>;}@

    constexpr auto size() const requires ForwardRange<const D> &&
      SizedSentinel<sentinel_t<const D>, iterator_t<const D>>;

    constexpr decltype(auto) front() requires ForwardRange<D>;
    constexpr decltype(auto) front() const requires ForwardRange<const D>;

    constexpr decltype(auto) back()
      requires BidirectionalRange<D> && BoundedRange<D>;
    constexpr decltype(auto) back() const
      requires BidirectionalRange<const D> && BoundedRange<const D>;

    template <RandomAccessRange R = D>
      constexpr decltype(auto) operator[](difference_type_t<iterator_t<R>> n);
    template <RandomAccessRange R = const D>
      constexpr decltype(auto) operator[](difference_type_t<iterator_t<R>> n) const;

    template <RandomAccessRange R = D>
        requires SizedRange<R>
      constexpr decltype(auto) at(difference_type_t<iterator_t<R>> n);
    template <RandomAccessRange R = const D>
        requires SizedRange<R>
      constexpr decltype(auto) at(difference_type_t<iterator_t<R>> n) const;

    template <ForwardRange C>
        requires !View<C> && @\removed{MoveConstructible<C> \&\&}@
          ConvertibleTo<@\changed{value_type_t}{reference_t}@<iterator_t<const D>>, value_type_t<iterator_t<C>>> &&
          Constructible<C, @\textit{range-common-iterator}@<const D>, @\textit{range-common-iterator}@<const D>>
      operator C () const;
  };
}}}}
\end{codeblock}

\pnum
The template parameter for \tcode{view_interface} may be an incomplete type.

\rSec3[ranges.view_interface.accessors]{\tcode{view_interface} accessors}

\indexlibrary{\idxcode{empty}!\idxcode{view_interface}}%
\begin{itemdecl}
constexpr bool empty() const requires ForwardRange<const D>;
\end{itemdecl}

\begin{itemdescr}
\pnum
\effects Equivalent to: \tcode{return ranges::begin(derived()) == ranges::end(derived());}.
\end{itemdescr}

\indexlibrary{\idxcode{operator bool}!\idxcode{view_interface}}%
\begin{itemdecl}
constexpr explicit operator bool() const
  requires @\changed{ForwardRange<const D>}{requires \{ ranges::empty(derived()); \}}@;
\end{itemdecl}

\begin{itemdescr}
\pnum
\removed{\returns \tcode{!empty()}.}
\added{\effects Equivalent to: \tcode{return !ranges::empty(derived());}}
\end{itemdescr}

\begin{removedblock}
\indexlibrary{\idxcode{operator!}!\idxcode{view_interface}}%
\begin{itemdecl}
constexpr bool operator!() const requires ForwardRange<const D>;
\end{itemdecl}

\begin{itemdescr}
\pnum
\returns \tcode{empty()}.
\end{itemdescr}
\end{removedblock}

\indexlibrary{\idxcode{size}!\idxcode{view_interface}}%
\begin{itemdecl}
constexpr auto size() const requires ForwardRange<const D> &&
SizedSentinel<sentinel_t<const D>, iterator_t<const D>>;
\end{itemdecl}

\begin{itemdescr}
\pnum
\effects Equivalent to: \tcode{return ranges::end(derived()) - ranges::begin(derived());}.
\end{itemdescr}

\indexlibrary{\idxcode{front}!\idxcode{view_interface}}%
\begin{itemdecl}
constexpr decltype(auto) front() requires ForwardRange<D>;
constexpr decltype(auto) front() const requires ForwardRange<const D>;
\end{itemdecl}

\begin{itemdescr}
\pnum
\requires \tcode{!empty()}.

\pnum
\effects Equivalent to: \tcode{return *ranges::begin(derived());}.
\end{itemdescr}

\indexlibrary{\idxcode{back}!\idxcode{view_interface}}%
\begin{itemdecl}
constexpr decltype(auto) back()
requires BidirectionalRange<D> && BoundedRange<D>;
constexpr decltype(auto) back() const
requires BidirectionalRange<const D> && BoundedRange<const D>;
\end{itemdecl}

\begin{itemdescr}
\pnum
\requires \tcode{!empty()}.

\pnum
\effects Equivalent to: \tcode{return *prev(ranges::end(derived()));}.
\end{itemdescr}

\indexlibrary{\idxcode{operator[]}!\idxcode{view_interface}}%
\begin{itemdecl}
template <RandomAccessRange R = D>
constexpr decltype(auto) operator[](difference_type_t<iterator_t<R>> n);
template <RandomAccessRange R = const D>
constexpr decltype(auto) operator[](difference_type_t<iterator_t<R>> n) const;
\end{itemdecl}

\begin{itemdescr}
\pnum
\requires \tcode{ranges::begin(derived()) + n} is well-formed.

\pnum
\effects Equivalent to: \tcode{return ranges::begin(derived())[n];}.
\end{itemdescr}

\indexlibrary{\idxcode{at}!\idxcode{view_interface}}%
\begin{itemdecl}
template <RandomAccessRange R = D>
  requires SizedRange<R>
constexpr decltype(auto) at(difference_type_t<iterator_t<R>> n);
template <RandomAccessRange R = const D>
  requires SizedRange<R>
constexpr decltype(auto) at(difference_type_t<iterator_t<R>> n) const;
\end{itemdecl}

\begin{itemdescr}
\pnum
\effects Equivalent to: \tcode{return \changed{ranges::begin(derived())}{derived()}[n];}.

\pnum
\throws \tcode{out_of_range} if \tcode{n < 0 || n >= ranges::size(derived())}.
\end{itemdescr}

\indexlibrary{\idxcode{at}!\idxcode{view_interface}}%
\begin{itemdecl}
template <ForwardRange C>
  requires !View<C> && @\removed{MoveConstructible<C> \&\&}@
    ConvertibleTo<@\changed{value_type_t}{reference_t}@<iterator_t<const D>>, value_type_t<iterator_t<C>>> &&
    Constructible<C, @\textit{range-common-iterator}@<const D>, @\textit{range-common-iterator}@<const D>>
operator C () const;
\end{itemdecl}

\begin{itemdescr}
\pnum
\effects Equivalent to:
\begin{codeblock}
using I = @\textit{range-common-iterator}@<R>;
return C@\changed{\{}{(}@I{ranges::begin(derived())}, I{ranges::end(derived())}@\changed{\}}{)}@;
\end{codeblock}
\end{itemdescr}

\rSec2[ranges.subranges]{Sub-ranges}

\pnum
\changed{The \tcode{iterator_range} and \tcode{sized_iterator_range} classes bundle together an
iterator and a sentinel into a single object that satisfies the \tcode{View} concept.
\tcode{sized_iterator_range} additionally stores the range's size and satisfies the
\tcode{SizedRange} concept.}{The \tcode{subrange} class template bundles together an
iterator and a sentinel into a single object that satisfies the \tcode{View} object. Additionally,
it satisfies the \tcode{SizedRange} concept when the final template parameter is \tcode{true}.}

\ednote{Remove sections [ranges.iterator_range] and [ranges.sized_iterator_range] and replace
with the following:}

\rSec3[ranges.subrange]{\tcode{subrange}}

\indexlibrary{\idxcode{subrange}}%
\begin{addedblock}
\begin{codeblock}
namespace std { namespace experimental { namespace ranges { inline namespace v1 {
  template <class T>
  concept bool @\textit{pair-like}@ = // \expos
    requires {
      { tuple_size<T>::value } -> const Same<size_t>&;
      requires tuple_size<T>::value == 2;
      typename tuple_element_t<0, T>;
      typename tuple_element_t<1, T>;
    };

  template <class T, class U, class V>
  concept bool @\textit{pair-like-convertible-to}@ = // \expos
    @\textit{pair-like}@<T> && !Range<T> &&
    ConvertibleTo<tuple_element_t<0, T>, U> &&
    ConvertibleTo<tuple_element_t<1, T>, V>;

  template <class T, class U, class V>
  concept bool @\textit{pair-like-convertible-from}@ = // \expos
    @\textit{pair-like}@<T> && !Range<T> &&
    Constructible<T, U, V> &&
    ConvertibleTo<U, tuple_element_t<0, T>> &&
    ConvertibleTo<V, tuple_element_t<1, T>>;

  template <class T>
  concept bool @\textit{iterator-sentinel-pair}@ = // \expos
    @\textit{pair-like}@<T> && !Range<T> &&
    Sentinel<tuple_element_t<1, T>, tuple_element_t<0, T>>;

  template <Iterator I, Sentinel<I> S = I, subrange_kind Sized = @\seebelow@>
  struct subrange : view_interface<subrange<I, S, Sized>> {
  private:
    static constexpr bool StoreSize = bool(Sized) && !SizedSentinel<S, I>; // \expos
    I begin_ {}; // \expos
    S end_ {}; // \expos
    difference_type_t<I> size_ = 0; // \expos; only present when StoreSize is true
  public:
    using iterator = I;
    using sentinel = S;

    subrange() = default;

    constexpr subrange(I i, S s) requires !StoreSize;

    constexpr subrange(I i, S s, difference_type_t<I> n) requires bool(Sized);

    template <ConvertibleTo<I> X, ConvertibleTo<S> Y, subrange_kind B>
    constexpr subrange(subrange<X, Y, B> r) requires !StoreSize || bool(B);

    template <ConvertibleTo<I> X, ConvertibleTo<S> Y, subrange_kind B>
    constexpr subrange(subrange<X, Y, B> r, difference_type_t<I> n) requires bool(Sized);

    template <@\textit{pair-like-convertible-to}@<I, S> PairLike>
    constexpr subrange(PairLike r) requires !StoreSize;

    template <@\textit{pair-like-convertible-to}@<I, S> PairLike>
    constexpr subrange(PairLike r, difference_type_t<I> n) requires bool(Sized);

    template <Range R>
      requires ConvertibleTo<iterator_t<R>, I> && ConvertibleTo<sentinel_t<R>, S>
    constexpr subrange(R& r) requires !StoreSize || SizedRange<R>;

    template <@\textit{pair-like-convertible-from}@<const I&, const S&> PairLike>
    constexpr operator PairLike() const;

    constexpr I begin() const;
    constexpr S end() const;
    constexpr bool empty() const;
    constexpr auto size() const requires bool(Sized);
    [[nodiscard]] constexpr subrange next(difference_type_t<I> n = 1) const;
    [[nodiscard]] constexpr subrange prev(difference_type_t<I> n = 1) const;
    constexpr subrange& advance(difference_type_t<I> n);
  };

  template <Iterator I, Sentinel<I> S>
  subrange(I, S, difference_type_t<I>) -> subrange<I, S, subrange_kind::sized>;

  template <@\textit{iterator-sentinel-pair}@ P>
  subrange(P) ->
    subrange<tuple_element_t<0, P>, tuple_element_t<1, P>>;

  template <@\textit{iterator-sentinel-pair}@ P>
  subrange(P, difference_type_t<tuple_element_t<0, P>>) ->
    subrange<tuple_element_t<0, P>, tuple_element_t<1, P>, subrange_kind::sized>;

  template <Iterator I, Sentinel<I> S, subrange_kind B>
  subrange(subrange<I, S, B>, difference_type_t<I>) -> subrange<I, S, subrange_kind::sized>;

  template <Range R>
  subrange(R&) -> subrange<iterator_t<R>, sentinel_t<R>>;

  template <SizedRange R>
  subrange(R&) -> subrange<iterator_t<R>, sentinel_t<R>, subrange_kind::sized>;

  template <std::size_t N, class I, class S, subrange_kind Sized>
    requires N < 2
  constexpr auto get(const subrange<I, S, Sized>& r);
}}}}
\end{codeblock}

\pnum
The default value for \tcode{subrange}'s third (non-type) template parameter is:
\begin{itemize}
\item If \tcode{SizedSentinel<S, I>} is satisfied, \tcode{subrange_kind::sized}.
\item Otherwise, \tcode{subrange_kind::unsized}.
\end{itemize}

\rSec4[ranges.subrange.ctor]{\tcode{subrange} constructors}

\indexlibrary{\idxcode{subrange}!\idxcode{subrange}}%
\begin{itemdecl}
constexpr subrange(I i, S s) requires !StoreSize;
\end{itemdecl}

\begin{itemdescr}
\pnum
\effects Initializes \tcode{begin_} with \tcode{i} and \tcode{end_} with
\tcode{s}.
\end{itemdescr}

\indexlibrary{\idxcode{subrange}!\idxcode{subrange}}%
\begin{itemdecl}
constexpr subrange(I i, S s, difference_type_t<I> n) requires bool(Sized);
\end{itemdecl}

\begin{itemdescr}
\pnum
\requires \tcode{n == distance(i, s)}.

\pnum
\effects Initializes \tcode{begin_} with \tcode{i}, \tcode{end_} with
\tcode{s}. If \tcode{StoreSize} is \tcode{true}, initializes \tcode{size_} with
\tcode{n}.
\end{itemdescr}

\indexlibrary{\idxcode{subrange}!\idxcode{subrange}}%
\begin{itemdecl}
template <ConvertibleTo<I> X, ConvertibleTo<S> Y, subrange_kind B>
constexpr subrange(subrange<X, Y, B> r) requires !StoreSize || bool(B);
\end{itemdecl}

\begin{itemdescr}
\pnum
\effects Equivalent to:
\begin{itemize}
\item If \tcode{StoreSize} is \tcode{true},
\tcode{subrange\{r.begin(), r.end(), r.size()\}}.
\item Otherwise, \tcode{subrange\{r.begin(), r.end()\}}.
\end{itemize}
\end{itemdescr}

\indexlibrary{\idxcode{subrange}!\idxcode{subrange}}%
\begin{itemdecl}
template <ConvertibleTo<I> X, ConvertibleTo<S> Y, subrange_kind B>
constexpr subrange(subrange<X, Y, B> r, difference_type_t<I> n) requires bool(Sized);
\end{itemdecl}

\begin{itemdescr}
\pnum
\requires \tcode{bool(B)} is \tcode{false} or \tcode{n == r.size()}.

\pnum
\effects Equivalent to \tcode{subrange\{r.begin(), r.end(), n\}}.
\end{itemdescr}

\indexlibrary{\idxcode{subrange}!\idxcode{subrange}}%
\begin{itemdecl}
template <@\textit{pair-like-convertible-to}@<I, S> PairLike>
constexpr subrange(PairLike r) requires !StoreSize;
\end{itemdecl}

\begin{itemdescr}
\pnum
\effects Equivalent to
\tcode{subrange\{get<0>(std::move(r)), get<1>(std::move(r))\}}.
\end{itemdescr}

\indexlibrary{\idxcode{subrange}!\idxcode{subrange}}%
\begin{itemdecl}
template <@\textit{pair-like-convertible-to}@<I, S> PairLike>
constexpr subrange(PairLike r, difference_type_t<I> n) requires bool(Sized);
\end{itemdecl}

\begin{itemdescr}
\pnum
\effects Equivalent to
\tcode{subrange\{get<0>(std::move(r)), get<1>(std::move(r)), n\}}.
\end{itemdescr}

\indexlibrary{\idxcode{subrange}!\idxcode{subrange}}%
\begin{itemdecl}
template <Range R>
  requires ConvertibleTo<iterator_t<R>, I> && ConvertibleTo<sentinel_t<R>, S>
constexpr subrange(R& r) requires !StoreSize || SizedRange<R>;
\end{itemdecl}

\begin{itemdescr}
\pnum
\effects Equivalent to:
\begin{itemize}
\item If \tcode{StoreSize} is \tcode{true},
\tcode{subrange\{ranges::begin(r), ranges::end(r), ranges::size(r)\}}.
\item Otherwise,
\tcode{subrange\{ranges::begin(r), ranges::end(r)\}}.
\end{itemize}
\end{itemdescr}

\rSec4[ranges.subrange.ops]{\tcode{subrange} operators}

\indexlibrary{\idxcode{operator \textit{PairLike}}!\idxcode{subrange}}%
\begin{itemdecl}
template <@\textit{pair-like-convertible-from}@<const I&, const S&> PairLike>
constexpr operator PairLike() const;
\end{itemdecl}

\begin{itemdescr}
\pnum
\effects Equivalent to: \tcode{return PairLike\{begin_, end_\};}.
\end{itemdescr}

\rSec4[ranges.subrange.accessors]{\tcode{subrange} accessors}

\indexlibrary{\idxcode{begin}!\idxcode{subrange}}%
\begin{itemdecl}
constexpr I begin() const;
\end{itemdecl}

\begin{itemdescr}
\pnum
\effects Equivalent to: \tcode{return begin_;}.
\end{itemdescr}

\indexlibrary{\idxcode{end}!\idxcode{subrange}}%
\begin{itemdecl}
constexpr I end() const;
\end{itemdecl}

\begin{itemdescr}
\pnum
\effects Equivalent to: \tcode{return end_;}.
\end{itemdescr}

\indexlibrary{\idxcode{empty}!\idxcode{subrange}}%
\begin{itemdecl}
constexpr bool empty() const;
\end{itemdecl}

\begin{itemdescr}
\pnum
\effects Equivalent to: \tcode{return begin_ == end_;}.
\end{itemdescr}

\indexlibrary{\idxcode{size}!\idxcode{subrange}}%
\begin{itemdecl}
constexpr auto size() const requires bool(Sized);
\end{itemdecl}

\begin{itemdescr}
\pnum
\effects Equivalent to:
\begin{itemize}
\item It \tcode{StoreSize} is \tcode{true}, \tcode{return size_;}.
\item Otherwise, \tcode{return end_ - begin_;}.
\end{itemize}
\end{itemdescr}

\indexlibrary{\idxcode{next}!\idxcode{subrange}}%
\begin{itemdecl}
[[nodiscard]] constexpr subrange next(difference_type_t<I> n = 1) const;
\end{itemdecl}

\begin{itemdescr}
\pnum
\effects Equivalent to:
\begin{codeblock}
auto tmp = *this;
tmp.advance(n);
return tmp;
\end{codeblock}
\end{itemdescr}

\indexlibrary{\idxcode{prev}!\idxcode{subrange}}%
\begin{itemdecl}
[[nodiscard]] constexpr subrange prev(difference_type_t<I> n = 1) const;
\end{itemdecl}

\begin{itemdescr}
\pnum
\effects Equivalent to:
\begin{codeblock}
auto tmp = *this;
tmp.advance(-n);
return tmp;
\end{codeblock}
\end{itemdescr}

\indexlibrary{\idxcode{advance}!\idxcode{subrange}}%
\begin{itemdecl}
constexpr subrange& advance(difference_type_t<I> n);
\end{itemdecl}

\begin{itemdescr}
\pnum
\effects Equivalent to:
\begin{itemize}
\item If \tcode{StoreSize} is \tcode{true},
\begin{codeblock}
size_ -= n - ranges::advance(begin_, n, end_);
return *this;
\end{codeblock}
\item Otherwise,
\begin{codeblock}
ranges::advance(begin_, n, end_);
return *this;
\end{codeblock}
\end{itemize}
\end{itemdescr}

\rSec4[ranges.subrange.nonmember]{\tcode{subrange} non-member functions}

\indexlibrary{\idxcode{get}!\idxcode{subrange}}%
\begin{itemdecl}
template <std::size_t N, class I, class S, subrange_kind Sized>
  requires N < 2
constexpr auto get(const subrange<I, S, Sized>& r) noexcept;
\end{itemdecl}

\begin{itemdescr}
\pnum
\effects Equivalent to:
\begin{codeblock}
if constexpr (N == 0)
  return r.begin();
else
  return r.end();
\end{codeblock}
\end{itemdescr}
\end{addedblock}

\rSec1[ranges.adaptors]{Range adaptors}

\pnum
This section defines \techterm{range adaptors}, which are utilities that transform a \tcode{Range}
into a \tcode{View} with custom behaviors. These adaptors can be chained to create pipelines of
range transformations that evaluate lazily as the resulting view is iterated.

\pnum
Range adaptors are declared in namespace \tcode{std::experimental::ranges::v1::view}.

\pnum
The bitwise or operator is overloaded for the purpose of creating adaptor chain pipelines. The
adaptors also support function call syntax with equivalent semantics.

\pnum
\enterexample
\begin{codeblock}
vector<int> ints{0,1,2,3,4,5};
auto even = [](int i){ return 0 == i % 2; };
auto square = [](int i) { return i * i; };
for (int i : ints | view::filter(even) | view::transform(square)) {
  cout << i << ' '; // prints: 0 4 16
}
\end{codeblock}
\exitexample

\rSec2[ranges.adaptor.object]{Range adaptor objects}

\begin{addedblock}
\pnum
A \techterm{range adaptor closure object} is a unary function object that accepts
a \tcode{ViewableRange} as an argument and returns a \tcode{View}. For a range
adaptor closure object \tcode{C} and an expression \tcode{R} such that
\tcode{decltype((R))} satisfies \tcode{ViewableRange}, the following expressions
are equivalent and return a \tcode{View}:

\begin{codeblock}
C(R)
R | C
\end{codeblock}

Given an additional range adaptor closure objects \tcode{D},
the expression \tcode{C | D} is well-formed and produces another range adaptor
closure object such that the following two expressions are equivalent:

\begin{codeblock}
R | C | D
R | (C | D)
\end{codeblock}
\end{addedblock}

\pnum
A \techterm{range adaptor object} is a customization point object~(\cxxref{??})
that accepts a \changed{\tcode{Range}}{\tcode{ViewableRange}} as its first
argument and \removed{that} returns a \tcode{View}.

\pnum
If the adaptor accepts only one argument, then \changed{the following alternate
syntaxes are semantically equivalent:}{it is a range adaptor closure object.}

\begin{removedblock}
\begin{codeblock}
@\textit{adaptor}@(rng)
rng | @\textit{adaptor}@
\end{codeblock}
\end{removedblock}

\pnum
If the adaptor accepts more than one argument, then the following \changed{alternate
syntaxes}{expressions} are \removed{semantically} equivalent:

\begin{codeblock}
@\textit{adaptor}@(rng, args...)
@\added{\textit{adaptor}(args...)(rng)}@
rng | @\textit{adaptor}@(args...)
\end{codeblock}

\begin{addedblock}
In this case, \tcode{\textit{adaptor}(args...)} is a range adaptor closure object.
\end{addedblock}

\begin{removedblock}
\pnum
The first argument to a range adaptor shall be either an lvalue
\tcode{Range} or a \tcode{View}.
\end{removedblock}

\rSec2[ranges.adaptor.semiregular_wrapper]{Semiregular wrapper}

\pnum
Many of the types in this section are specified in terms of an exposition-only helper
called \tcode{\textit{semiregular}<T>}. This type behaves exactly like \tcode{optional<T>}
with the following exceptions:

\begin{itemize}
\item \tcode{\textit{semiregular}<T>} constrains its argument with \tcode{CopyConstructible<T>}.
\item If \tcode{T} satisfies \tcode{DefaultConstructble}, the default constructor of
\tcode{\textit{semiregular}<T>} is equivalent to:
\begin{codeblock}
constexpr @\textit{semiregular}@()
  noexcept(is_nothrow_default_constructible<T>::value)
: @\textit{semiregular}@{in_place} {}
\end{codeblock}
\item If the syntactic requirements of \tcode{Assignable<T\&, const T\&>} are not satisfied, the
copy assignment operator is equivalent to:
\begin{codeblock}
constexpr @\textit{semiregular}@& operator=(const @\textit{semiregular}@& that)
  noexcept(is_nothrow_copy_constructible<T>::value) {
  if (that) emplace(*that);
  else reset();
  return *this;
}
\end{codeblock}
\item If the syntactic requirements of \tcode{Assignable<T\&, T>} are not satisfied, the
move assignment operator is equivalent to:
\begin{codeblock}
constexpr @\textit{semiregular}@& operator=(@\textit{semiregular}@&& that)
  noexcept(is_nothrow_move_constructible<T>::value) {
  if (that) emplace(std::move(*that));
  else reset();
  return *this;
}
\end{codeblock}
\end{itemize}

{\color{addclr}
\rSec2[ranges.adaptor.simple_view]{Simple views}

\pnum
Many of the types in this section are specified in terms of an exposition-only
Boolean variable template called \tcode{\textit{simple-view}<T>}, defined as follows:

\begin{codeblock}
template <class R>
concept bool @\textit{\xname{simple-view}}@ =
  View<R> && View<const R> &&
  Same<iterator_t<R>, iterator_t<const R>> &&
  Same<sentinel_t<R>, sentinel_t<const R>>;

template <class R>
  constexpr bool @\textit{simple-view}@ = false;

template <@\textit{\xname{simple-view}}@ R>
  constexpr bool @\textit{simple-view}@<R> = true;
\end{codeblock}
}

\rSec2[ranges.adaptors.all]{\tcode{view::all}}

\pnum
The purpose of \tcode{view::all} is to return a \tcode{View} that includes all
elements of the \tcode{Range} passed in.

\pnum
The name \tcode{view::all} denotes a range adaptor
object~(\ref{ranges.adaptor.object}). \changed{Given an expression \tcode{E} and
a type \tcode{T} such that \tcode{decltype((E))} is \tcode{T}, then t}{T}he
expression \tcode{view::all(E)} for some subexpression \tcode{E} is
expression-equivalent to:

\begin{itemize}
\item \tcode{\textit{DECAY_COPY}(E)} if the \added{decayed} type of \tcode{E}
satisfies the concept \tcode{View}.
{\color{remclr}
\item \removed{\tcode{sized_iterator_range\{E\}} if \tcode{E} is an lvalue and has
a type that satisfies concept \tcode{SizedRange}.}}
\item \changed{\tcode{iterator_range\{E\}}}{\tcode{subrange\{E\}}} if \tcode{E}
is an lvalue and has a type that satisfies concept \tcode{Range}.
\item Otherwise, \tcode{view::all(E)} is ill-formed.
\end{itemize}

\begin{addedblock}
\remark Whenever \tcode{view::all(E)} is a valid expression, it is a prvalue
whose type satisfies \tcode{View}.
\end{addedblock}

\rSec2[ranges.adaptors.filter_view]{Class template \tcode{filter_view}}

\pnum
The purpose of \tcode{filter_view} is to present a view of an underlying
sequence without the elements that fail to satisfy a predicate.

\pnum
\enterexample
\begin{codeblock}
vector<int> is{ 0, 1, 2, 3, 4, 5, 6 };
filter_view evens{is, [](int i) { return 0 == i % 2; }};
for (int i : evens)
  cout << i << ' '; // prints: 0 2 4 6
\end{codeblock}
\exitexample

\begin{codeblock}
namespace std { namespace experimental { namespace ranges { inline namespace v1 {
  template <InputRange R, IndirectUnaryPredicate<iterator_t<R>> Pred>
    requires View<R>
  class filter_view : @\added{public}@ view_interface<filter_view<R, Pred>> {
  private:
    R base_; // \expos
    @\textit{semiregular}@<Pred> pred_; // \expos
  public:
    filter_view() = default;
    constexpr filter_view(R base, Pred pred);
    template <InputRange O>
      requires @\changed{(is_lvalue_reference_v<O> || View<decay_t<O>{>})}{ViewableRange<O>}@ &&
        Constructible<R, all_view<O>>
    constexpr filter_view(O&& o, Pred pred);

    constexpr R base() const;

    class iterator;
    class sentinel;

    constexpr iterator begin();
    constexpr sentinel end();
    constexpr iterator end() requires BoundedRange<R>;
  };

  template <InputRange R, CopyConstructible Pred>
    requires IndirectUnaryPredicate<Pred, iterator_t<R>> &&
    @\changed{(is_lvalue_reference_v<R> || View<decay_t<R>{>})}{ViewableRange<R>}@
  filter_view(R&&, Pred) -> filter_view<all_view<R>, Pred>;
}}}}
\end{codeblock}

\rSec3[ranges.adaptors.filter_view.ops]{\tcode{filter_view} operations}
\rSec4[ranges.adaptors.filter_view.ctor]{\tcode{filter_view} constructors}

\indexlibrary{\idxcode{filter_view}!\idxcode{filter_view}}%
\begin{itemdecl}
constexpr filter_view(R base, Pred pred);
\end{itemdecl}

\begin{itemdescr}
\pnum
\effects Initializes \tcode{base_} with \tcode{std::move(base)} and initializes
\tcode{pred_} with \tcode{std::move(pred)}.
\end{itemdescr}

\indexlibrary{\idxcode{filter_view}!\idxcode{filter_view}}%
\begin{itemdecl}
template <InputRange O>
requires @\changed{(is_lvalue_reference_v<O> || View<decay_t<O>{>})}{ViewableRange<O>}@ &&
  Constructible<R, all_view<O>>
constexpr filter_view(O&& o, Pred pred);
\end{itemdecl}

\begin{itemdescr}
\pnum
\effects Initializes \tcode{base_} with \tcode{view::all(std::forward<O>(o))}
and initializes \tcode{pred_} with \tcode{std::move(pred)}.
\end{itemdescr}

\rSec4[ranges.adaptors.filter_view.conv]{\tcode{filter_view} conversion}

\indexlibrary{\idxcode{base}!\idxcode{filter_view}}%
\begin{itemdecl}
constexpr R base() const;
\end{itemdecl}

\begin{itemdescr}
\pnum
\returns \tcode{base_}.
\end{itemdescr}

\rSec4[ranges.adaptors.filter_view.begin]{\tcode{filter_view} range begin}

\indexlibrary{\idxcode{begin}!\idxcode{filter_view}}%
\begin{itemdecl}
constexpr iterator begin();
\end{itemdecl}

\begin{itemdescr}
\pnum
\effects Equivalent to:
\begin{codeblock}
return {*this, ranges::find_if(base_, ref(*pred_))};
\end{codeblock}

\pnum
\remarks In order to provide the amortized constant time complexity required by
the \tcode{Range} concept, this function caches the result within the
\tcode{filter_view} for use on subsequent calls.
\end{itemdescr}

\rSec4[ranges.adaptors.filter_view.end]{\tcode{filter_view} range end}

\indexlibrary{\idxcode{end}!\idxcode{filter_view}}%
\begin{itemdecl}
constexpr sentinel end();
\end{itemdecl}

\begin{itemdescr}
\pnum
\returns \tcode{sentinel\{*this\}}.
\end{itemdescr}

\begin{itemdecl}
constexpr iterator end() requires BoundedRange<R>;
\end{itemdecl}

\begin{itemdescr}
\pnum
\returns \tcode{iterator\{*this, ranges::end(base_)\}}.
\end{itemdescr}

\rSec3[ranges.adaptors.filter_view.iterator]{Class template \tcode{filter_view::iterator}}

\indexlibrary{\idxcode{iterator}!\idxcode{filter_view}}%
\begin{codeblock}
namespace std { namespace experimental { namespace ranges { inline namespace v1 {
  template <class R, class Pred>
  class filter_view<R, Pred>::iterator {
  private:
    iterator_t<R> current_ {}; // \expos
    filter_view* parent_ = nullptr; // \expos
  public:
    using iterator_category = @\seebelow@;
    using value_type = value_type_t<iterator_t<R>>;
    using difference_type = difference_type_t<iterator_t<R>>;

    iterator() = default;
    constexpr iterator(filter_view& parent, iterator_t<R> current);

    constexpr iterator_t<R> base() const;
    constexpr reference_t<iterator_t<R>> operator*() const;

    constexpr iterator& operator++();
    constexpr void operator++(int);
    constexpr iterator operator++(int) requires ForwardRange<R>;

    constexpr iterator& operator--() requires BidirectionalRange<R>;
    constexpr iterator operator--(int) requires BidirectionalRange<R>;

    friend constexpr bool operator==(const iterator& x, const iterator& y)
      requires EqualityComparable<iterator_t<R>>;
    friend constexpr bool operator!=(const iterator& x, const iterator& y)
      requires EqualityComparable<iterator_t<R>>;

    friend constexpr rvalue_reference_t<iterator_t<R>> iter_move(const iterator& i)
      noexcept(@\seebelow@);
    friend constexpr void iter_swap(const iterator& x, const iterator& y)
      noexcept(@\seebelow@) requires IndirectlySwappable<iterator_t<R>>;
  };
}}}}
\end{codeblock}

\pnum
The type \tcode{filter_view<R>::iterator::iterator_category} is defined as follows:
\begin{itemize}
\item If \tcode{R} satisfies \tcode{BidirectionalRange<R>}, then \tcode{iterator_category}
is an alias for \tcode{ranges::bidirectional_iterator_tag}.
\item If \tcode{R} satisfies \tcode{ForwardRange<R>}, then \tcode{iterator_category}
is an alias for \tcode{ranges::forward_iterator_tag}.
\item Otherwise, \tcode{iterator_category} is an alias for
\tcode{ranges::input_iterator_tag}.
\end{itemize}

\rSec4[ranges.adaptors.filter_view.iterator.ops]{\tcode{filter_view::iterator} operations}
\rSec5[ranges.adaptors.filter_view.iterator.ctor]{\tcode{filter_view::iterator} constructors}

\indexlibrary{\idxcode{iterator}!\idxcode{filter_view::iterator}}%
\begin{itemdecl}
constexpr iterator(filter_view& parent, iterator_t<R> current);
\end{itemdecl}

\begin{itemdescr}
\pnum
\effects Initializes \tcode{current_} with \tcode{current} and \tcode{parent_}
with \tcode{\&parent}.
\end{itemdescr}

\rSec5[ranges.adaptors.filter_view.iterator.conv]{\tcode{filter_view::iterator} conversion}

\indexlibrary{\idxcode{base}!\idxcode{filter_view::iterator}}%
\begin{itemdecl}
constexpr iterator_t<R> base() const;
\end{itemdecl}

\begin{itemdescr}
\pnum
\returns \tcode{current_}.
\end{itemdescr}

\rSec5[ranges.adaptors.filter_view.iterator.star]{\tcode{filter_view::iterator::operator*}}

\indexlibrary{\idxcode{operator*}!\idxcode{filter_view::iterator}}%
\begin{itemdecl}
constexpr reference_t<iterator_t<R>> operator*() const;
\end{itemdecl}

\begin{itemdescr}
\pnum
\returns \tcode{*current_}.
\end{itemdescr}

\rSec5[ranges.adaptors.filter_view.iterator.inc]{\tcode{filter_view::iterator::operator++}}

\indexlibrary{\idxcode{operator++}!\idxcode{filter_view::iterator}}%
\begin{itemdecl}
constexpr iterator& operator++();
\end{itemdecl}

\begin{itemdescr}
\pnum
\effects Equivalent to:
\begin{codeblock}
current_ = find_if(++current_, ranges::end(parent_->base_), ref(*parent_->pred_));
return *this;
\end{codeblock}
\end{itemdescr}

\indexlibrary{\idxcode{operator++}!\idxcode{filter_view::iterator}}%
\begin{itemdecl}
constexpr void operator++(int);
\end{itemdecl}

\begin{itemdescr}
\pnum
\effects Equivalent to \tcode{(void)++*this}.
\end{itemdescr}

\indexlibrary{\idxcode{operator++}!\idxcode{filter_view::iterator}}%
\begin{itemdecl}
constexpr iterator operator++(int) requires ForwardRange<R>;
\end{itemdecl}

\begin{itemdescr}
\pnum
\effects Equivalent to:
\begin{codeblock}
auto tmp = *this;
++*this;
return tmp;
\end{codeblock}
\end{itemdescr}

\rSec5[ranges.adaptors.filter_view.iterator.dec]{\tcode{filter_view::iterator::operator\dcr}}

\indexlibrary{\idxcode{operator\dcr}!\idxcode{filter_view::iterator}}%
\begin{itemdecl}
constexpr iterator& operator--() requires BidirectionalRange<R>;
\end{itemdecl}

\begin{itemdescr}
\pnum
\effects Equivalent to:
\begin{codeblock}
do
--current_;
while(invoke(*parent_->pred_, *current_));
return *this;
\end{codeblock}
\end{itemdescr}

\indexlibrary{\idxcode{operator\dcr}!\idxcode{filter_view::iterator}}%
\begin{itemdecl}
constexpr iterator operator--(int) requires BidirectionalRange<R>;
\end{itemdecl}

\begin{itemdescr}
\pnum
\effects Equivalent to:
\begin{codeblock}
auto tmp = *this;
--*this;
return tmp;
\end{codeblock}
\end{itemdescr}

\rSec5[ranges.adaptors.filter_view.iterator.comp]{\tcode{filter_view::iterator} comparisons}

\indexlibrary{\idxcode{operator==}!\idxcode{filter_view::iterator}}%
\begin{itemdecl}
friend constexpr bool operator==(const iterator& x, const iterator& y)
requires EqualityComparable<iterator_t<R>>;
\end{itemdecl}

\begin{itemdescr}
\pnum
\returns \tcode{x.current_ == y.current_}.
\end{itemdescr}

\indexlibrary{\idxcode{operator!=}!\idxcode{filter_view::iterator}}%
\begin{itemdecl}
friend constexpr bool operator!=(const iterator& x, const iterator& y)
requires EqualityComparable<iterator_t<R>>;
\end{itemdecl}

\begin{itemdescr}
\pnum
\returns \tcode{!(x == y)}.
\end{itemdescr}

\rSec4[ranges.adaptors.filter_view.iterator.nonmember]{\tcode{filter_view::iterator} non-member functions}

\indexlibrary{\idxcode{iter_move}!\idxcode{filter_view::iterator}}%
\begin{itemdecl}
friend constexpr rvalue_reference_t<iterator_t<R>> iter_move(const iterator& i)
noexcept(@\seebelow@);
\end{itemdecl}

\begin{itemdescr}
\pnum
\returns \tcode{ranges::iter_move(i.current_)}.

\pnum
\remarks The expression in \tcode{noexcept} is equivalent to:
\begin{codeblock}
noexcept(ranges::iter_move(i.current_))
\end{codeblock}
\end{itemdescr}

\indexlibrary{\idxcode{iter_swap}!\idxcode{filter_view::iterator}}%
\begin{itemdecl}
friend constexpr void iter_swap(const iterator& x, const iterator& y)
noexcept(@\seebelow@) requires IndirectlySwappable<iterator_t<R>>;
\end{itemdecl}

\begin{itemdescr}
\pnum
\effects Equivalent to \tcode{ranges::iter_swap(x.current_, y.current_)}.

\pnum
\remarks The expression in \tcode{noexcept} is equivalent to:
\begin{codeblock}
noexcept(ranges::iter_swap(x.current_, y.current_))
\end{codeblock}
\end{itemdescr}

\rSec3[ranges.adaptors.filter_view.sentinel]{Class template \tcode{filter_view::sentinel}}

\indexlibrary{\idxcode{sentinel}!\idxcode{filter_view}}%
\begin{codeblock}
namespace std { namespace experimental { namespace ranges { inline namespace v1 {
  template <class R, class Pred>
  class filter_view<R, Pred>::sentinel {
  private:
    sentinel_t<R> end_; // \expos
  public:
    sentinel() = default;
    explicit constexpr sentinel(filter_view& parent);

    constexpr sentinel_t<R> base() const;

    friend constexpr bool operator==(const iterator& x, const sentinel& y);
    friend constexpr bool operator==(const sentinel& x, const iterator& y);
    friend constexpr bool operator!=(const iterator& x, const sentinel& y);
    friend constexpr bool operator!=(const sentinel& x, const iterator& y);
  };
}}}}
\end{codeblock}

\rSec4[ranges.adaptors.filter_view.sentinel.ctor]{\tcode{filter_view::sentinel} constructors}

\indexlibrary{\idxcode{sentinel}!\idxcode{filter_view::sentinel}}%
\begin{itemdecl}
explicit constexpr sentinel(filter_view& parent);
\end{itemdecl}

\begin{itemdescr}
\pnum
\effects Initializes \tcode{end_} with \tcode{ranges::end(parent)}.
\end{itemdescr}

\rSec4[ranges.adaptors.filter_view.sentinel.conv]{\tcode{filter_view::sentinel} conversion}

\indexlibrary{\idxcode{base}!\idxcode{filter_view::sentinel}}%
\begin{itemdecl}
constexpr sentinel_t<R> base() const;
\end{itemdecl}

\begin{itemdescr}
\pnum
\returns \tcode{end_}.
\end{itemdescr}

\rSec4[ranges.adaptors.filter_view.sentinel.comp]{\tcode{filter_view::sentinel} comparison}

\indexlibrary{\idxcode{operator==}!\idxcode{filter_view::sentinel}}%
\begin{itemdecl}
friend constexpr bool operator==(const iterator& x, const sentinel& y);
\end{itemdecl}

\begin{itemdescr}
\pnum
\returns \tcode{x.current_ == y.end_}.
\end{itemdescr}

\indexlibrary{\idxcode{operator==}!\idxcode{filter_view::sentinel}}%
\begin{itemdecl}
friend constexpr bool operator==(const sentinel& x, const iterator& y);
\end{itemdecl}

\begin{itemdescr}
\pnum
\returns \tcode{y == x}.
\end{itemdescr}

\indexlibrary{\idxcode{operator!=}!\idxcode{filter_view::sentinel}}%
\begin{itemdecl}
friend constexpr bool operator!=(const iterator& x, const sentinel& y);
\end{itemdecl}

\begin{itemdescr}
\pnum
\returns \tcode{!(x == y)}.
\end{itemdescr}

\indexlibrary{\idxcode{operator!=}!\idxcode{filter_view::sentinel}}%
\begin{itemdecl}
friend constexpr bool operator!=(const sentinel& x, const iterator& y);
\end{itemdecl}

\begin{itemdescr}
\pnum
\returns \tcode{!(y == x)}.
\end{itemdescr}


\rSec2[ranges.adaptors.filter]{\tcode{view::filter}}

\pnum
The name \tcode{view::filter} denotes a range adaptor
object~(\ref{ranges.adaptor.object}). Let \tcode{E} and \tcode{P} be expressions such
that types \tcode{T} and \tcode{U} are \tcode{decltype((E))} and \tcode{decltype((P))}
respectively. Then the expression \tcode{view::filter(E, P)} is expression-equivalent to:

\begin{itemize}
\item \tcode{filter_view\{E, P\}} if
\tcode{InputRange<T> \&\& IndirectUnaryPredicate<decay_t<U>, iterator_t<T>{>}}
is satisfied.
\item Otherwise, \tcode{view::filter(E, P)} is ill-formed.
\end{itemize}

\rSec2[ranges.adaptors.transform_view]{Class template \tcode{transform_view}}

\pnum
The purpose of \tcode{transform_view} is to present a view of an underlying
sequence after applying a transformation function to each element.

\pnum
\enterexample
\begin{codeblock}
vector<int> is{ 0, 1, 2, 3, 4 };
transform_view squares{is, [](int i) { return i * i; }};
for (int i : squares)
  cout << i << ' '; // prints: 0 1 4 9 16
\end{codeblock}
\exitexample

\begin{codeblock}
namespace std { namespace experimental { namespace ranges { inline namespace v1 {
  template <InputRange R, CopyConstructible F>
    requires View<R> && Invocable<F&, reference_t<iterator_t<R>>>
  class transform_view : @\added{public}@ view_interface<transform_view<R, F>> {
  private:
    R base_; // \expos
    @\textit{semiregular}@<F> fun_; // \expos
    template <bool Const>
      struct @\xname{iterator}@; // \expos
    template <bool Const>
      struct @\xname{sentinel}@; // \expos
  public:
    transform_view() = default;
    constexpr transform_view(R base, F fun);
    template <InputRange O>
      requires @\changed{(is_lvalue_reference_v<O> || View<decay_t<O>{>})}{ViewableRange<O>}@ &&
          Constructible<R, all_view<O>>
    constexpr transform_view(O&& o, F fun);

    using iterator = @\xname{iterator}@<false>;
    using sentinel = @\xname{sentinel}@<false>;
    using const_iterator = @\xname{iterator}@<true>;
    using const_sentinel = @\xname{sentinel}@<true>;

    constexpr R base() const;

    constexpr iterator begin();
    constexpr const_iterator begin() const requires Range<const R> &&
      Invocable<const F&, reference_t<iterator_t<const R>>>;

    constexpr sentinel end();
    constexpr const_sentinel end() const requires Range<const R> &&
      Invocable<const F&, reference_t<iterator_t<const R>>>;
    constexpr iterator end() requires BoundedRange<R>;
    constexpr const_iterator end() const requires BoundedRange<const R> &&
      Invocable<const F&, reference_t<iterator_t<const R>>>;

    constexpr auto size() requires SizedRange<R>;
    constexpr auto size() const requires SizedRange<const R>;
  };

  template <class R, class F>
  transform_view(R&& r, F fun) -> transform_view<all_view<R>, F>;
}}}}
\end{codeblock}

\rSec3[ranges.adaptors.transform_view.ops]{\tcode{transform_view} operations}
\rSec4[ranges.adaptors.transform_view.ctor]{\tcode{transform_view} constructors}

\indexlibrary{\idxcode{transform_view}!\idxcode{transform_view}}%
\begin{itemdecl}
constexpr transform_view(R base, F fun);
\end{itemdecl}

\begin{itemdescr}
\pnum
\effects Initializes \tcode{base_} with \tcode{std::move(base)} and initializes
\tcode{fun_} with \tcode{std::move(fun)}.
\end{itemdescr}

\indexlibrary{\idxcode{transform_view}!\idxcode{transform_view}}%
\begin{itemdecl}
template <InputRange O>
requires @\changed{(is_lvalue_reference_v<O> || View<decay_t<O>{>})}{ViewableRange<O>}@ &&
    Constructible<R, all_view<O>>
constexpr transform_view(O&& o, F fun);
\end{itemdecl}

\begin{itemdescr}
\pnum
\effects Initializes \tcode{base_} with \tcode{view::all(std::forward<O>(o))}
and initializes \tcode{fun_} with \tcode{std::move(fun)}.
\end{itemdescr}

\rSec4[ranges.adaptors.transform_view.conv]{\tcode{transform_view} conversion}

\indexlibrary{\idxcode{base}!\idxcode{transform_view}}%
\begin{itemdecl}
constexpr R base() const;
\end{itemdecl}

\begin{itemdescr}
\pnum
\returns \tcode{base_}.
\end{itemdescr}

\rSec4[ranges.adaptors.transform_view.begin]{\tcode{transform_view} range begin}

\indexlibrary{\idxcode{begin}!\idxcode{transform_view}}%
\begin{itemdecl}
constexpr iterator begin();
constexpr const_iterator begin() const
requires Range<const R> && Invocable<const F&, reference_t<iterator_t<const R>>>;
\end{itemdecl}

\begin{itemdescr}
\pnum
\effects Equivalent to:
\begin{codeblock}
return {*this, ranges::begin(base_)};
\end{codeblock}
\end{itemdescr}

\rSec4[ranges.adaptors.transform_view.end]{\tcode{transform_view} range end}

\indexlibrary{\idxcode{end}!\idxcode{transform_view}}%
\begin{itemdecl}
constexpr sentinel end();
constexpr const_sentinel end() const requires Range<const R> &&
  Invocable<const F&, reference_t<iterator_t<const R>>>;
\end{itemdecl}

\begin{itemdescr}
\pnum
\effects Equivalent to \tcode{sentinel\{ranges::end(base_)\}} and
\tcode{const_sentinel\{ranges::end(base_)\}} for the first and second overload,
respectively.
\end{itemdescr}

\indexlibrary{\idxcode{end}!\idxcode{transform_view}}%
\begin{itemdecl}
constexpr iterator end() requires BoundedRange<R>;
constexpr const_iterator end() const requires BoundedRange<R> &&
  Invocable<const F&, reference_t<iterator_t<const R>>>;
\end{itemdecl}

\begin{itemdescr}
\pnum
\effects Equivalent to:
\begin{codeblock}
return {*this, ranges::end(base_)};
\end{codeblock}
\end{itemdescr}

\rSec4[ranges.adaptors.transform_view.size]{\tcode{transform_view} range size}

\indexlibrary{\idxcode{size}!\idxcode{transform_view}}%
\begin{itemdecl}
constexpr auto size() requires SizedRange<R>;
constexpr auto size() const requires SizedRange<const R>;
\end{itemdecl}

\begin{itemdescr}
\pnum
\returns \tcode{ranges::size(base_)}.
\end{itemdescr}

\rSec3[ranges.adaptors.transform_view.iterator]{Class template \tcode{transform_view::\xname{iterator}}}

\pnum
\tcode{transform_view<R, F>::\xname{iterator}} is an exposition-only type.

\begin{codeblock}
namespace std { namespace experimental { namespace ranges { inline namespace v1 {
  template <class R, class F>
  template <bool Const>
  class transform_view<R, F>::@\xname{iterator}@ { // \expos
  private:
    using Parent = conditional_t<Const, const transform_view, transform_view>;
    using Base = conditional_t<Const, const R, R>;
    iterator_t<Base> current_ {};
    Parent* parent_ = nullptr;
  public:
    using iterator_category = iterator_category_t<iterator_t<Base>>;
    using value_type = remove_const_t<remove_reference_t<
        invoke_result_t<F&, reference_t<iterator_t<Base>>>>>;
    using difference_type = difference_type_t<iterator_t<Base>>;

    @\xname{iterator}@() = default;
    constexpr @\xname{iterator}@(Parent& parent, iterator_t<Base> current);
    constexpr @\xname{iterator}@(@\xname{iterator}@<!Const> i)
      requires Const && ConvertibleTo<iterator_t<R>, iterator_t<Base>>;

    constexpr iterator_t<Base> base() const;
    constexpr decltype(auto) operator*() const;

    constexpr @\xname{iterator}@& operator++();
    constexpr void operator++(int);
    constexpr @\xname{iterator}@ operator++(int) requires ForwardRange<Base>;

    constexpr @\xname{iterator}@& operator--() requires BidirectionalRange<Base>;
    constexpr @\xname{iterator}@ operator--(int) requires BidirectionalRange<Base>;

    constexpr @\xname{iterator}@& operator+=(difference_type n)
      requires RandomAccessRange<Base>;
    constexpr @\xname{iterator}@& operator-=(difference_type n)
      requires RandomAccessRange<Base>;
    constexpr decltype(auto) operator[](difference_type n) const
      requires RandomAccessRange<Base>;

    friend constexpr bool operator==(const @\xname{iterator}@& x, const @\xname{iterator}@& y)
      requires EqualityComparable<iterator_t<Base>>;
    friend constexpr bool operator!=(const @\xname{iterator}@& x, const @\xname{iterator}@& y)
      requires EqualityComparable<iterator_t<Base>>;

    friend constexpr bool operator<(const @\xname{iterator}@& x, const @\xname{iterator}@& y)
      requires RandomAccessRange<Base>;
    friend constexpr bool operator>(const @\xname{iterator}@& x, const @\xname{iterator}@& y)
      requires RandomAccessRange<Base>;
    friend constexpr bool operator<=(const @\xname{iterator}@& x, const @\xname{iterator}@& y)
      requires RandomAccessRange<Base>;
    friend constexpr bool operator>=(const @\xname{iterator}@& x, const @\xname{iterator}@& y)
      requires RandomAccessRange<Base>;

    friend constexpr @\xname{iterator}@ operator+(@\xname{iterator}@ i, difference_type n)
      requires RandomAccessRange<Base>;
    friend constexpr @\xname{iterator}@ operator+(difference_type n, @\xname{iterator}@ i)
      requires RandomAccessRange<Base>;

    friend constexpr @\xname{iterator}@ operator-(@\xname{iterator}@ i, difference_type n)
      requires RandomAccessRange<Base>;
    friend constexpr difference_type operator-(const @\xname{iterator}@& x, const @\xname{iterator}@& y)
      requires RandomAccessRange<Base>;

    friend constexpr decltype(auto) iter_move(const @\xname{iterator}@& i)
      noexcept(@\seebelow@);
    friend constexpr void iter_swap(const @\xname{iterator}@& x, const @\xname{iterator}@& y)
      noexcept(@\seebelow@) requires IndirectlySwappable<iterator_t<Base>>;
  };
}}}}
\end{codeblock}

\rSec4[ranges.adaptors.transform_view.iterator.ops]{\tcode{transform_view::\xname{iterator}} operations}
\rSec5[ranges.adaptors.transform_view.iterator.ctor]{\tcode{transform_view::\xname{iterator}} constructors}

\indexlibrary{\idxcode{\xname{iterator}}!\idxcode{transform_view::\xname{iterator}}}
\begin{itemdecl}
constexpr @\xname{iterator}@(Parent& parent, iterator_t<Base> current);
\end{itemdecl}

\begin{itemdescr}
\pnum
\effects Initializes \tcode{current_} with \tcode{current} and
initializes \tcode{parent_} with \tcode{\&parent}.
\end{itemdescr}

\indexlibrary{\idxcode{\xname{iterator}}!\idxcode{transform_view::\xname{iterator}}}%
\begin{itemdecl}
constexpr @\xname{iterator}@(@\xname{iterator}@<!Const> i)
requires Const && ConvertibleTo<iterator_t<R>, iterator_t<Base>>;
\end{itemdecl}

\begin{itemdescr}
\pnum
\effects Initializes \tcode{parent_} with \tcode{i.parent_} and \tcode{current_}
with \tcode{i.current_}.
\end{itemdescr}

\rSec5[ranges.adaptors.transform_view.iterator.conv]{\tcode{transform_view::\xname{iterator}} conversion}

\indexlibrary{\idxcode{base}!\idxcode{transform_view::\xname{iterator}}}
\begin{itemdecl}
constexpr iterator_t<Base> base() const;
\end{itemdecl}

\begin{itemdescr}
\pnum
\returns \tcode{current_}.
\end{itemdescr}

\rSec5[ranges.adaptors.transform_view.iterator.star]{\tcode{transform_view::\xname{iterator}::operator*}}

\indexlibrary{\idxcode{operator*}!\idxcode{transform_view::\xname{iterator}}}
\begin{itemdecl}
constexpr decltype(auto) operator*() const;
\end{itemdecl}

\begin{itemdescr}
\pnum
\returns \tcode{invoke(*parent_->fun_, *current_)}.
\end{itemdescr}

\rSec5[ranges.adaptors.transform_view.iterator.inc]{\tcode{transform_view::\xname{iterator}::operator++}}

\indexlibrary{\idxcode{operator++}!\idxcode{transform_view::\xname{iterator}}}
\begin{itemdecl}
constexpr @\xname{iterator}@& operator++();
\end{itemdecl}

\begin{itemdescr}
\pnum
\effects Equivalent to:
\begin{codeblock}
++current_;
return *this;
\end{codeblock}
\end{itemdescr}

\indexlibrary{\idxcode{operator++}!\idxcode{transform_view::\xname{iterator}}}
\begin{itemdecl}
constexpr void operator++(int);
\end{itemdecl}

\begin{itemdescr}
\pnum
\effects Equivalent to:
\begin{codeblock}
++current_;
\end{codeblock}
\end{itemdescr}

\indexlibrary{\idxcode{operator++}!\idxcode{transform_view::\xname{iterator}}}
\begin{itemdecl}
constexpr @\xname{iterator}@ operator++(int) requires ForwardRange<Base>;
\end{itemdecl}

\begin{itemdescr}
\pnum
\effects Equivalent to:
\begin{codeblock}
auto tmp = *this;
++*this;
return tmp;
\end{codeblock}
\end{itemdescr}

\rSec5[ranges.adaptors.transform_view.iterator.dec]{\tcode{transform_view::\xname{iterator}::operator\dcr}}

\indexlibrary{\idxcode{operator\dcr}!\idxcode{transform_view::\xname{iterator}}}
\begin{itemdecl}
constexpr @\xname{iterator}@& operator--() requires BidirectionalRange<Base>;
\end{itemdecl}

\begin{itemdescr}
\pnum
\effects Equivalent to:
\begin{codeblock}
--current_;
return *this;
\end{codeblock}
\end{itemdescr}

\indexlibrary{\idxcode{operator\dcr}!\idxcode{transform_view::\xname{iterator}}}
\begin{itemdecl}
constexpr @\xname{iterator}@ operator--(int) requires BidirectionalRange<Base>;
\end{itemdecl}

\begin{itemdescr}
\pnum
\effects Equivalent to:
\begin{codeblock}
auto tmp = *this;
--*this;
return tmp;
\end{codeblock}
\end{itemdescr}

\rSec5[ranges.adaptors.transform_view.iterator.adv]{\tcode{transform_view::\xname{iterator}} advance}

\indexlibrary{\idxcode{operator+=}!\idxcode{transform_view::\xname{iterator}}}
\begin{itemdecl}
constexpr @\xname{iterator}@& operator+=(difference_type n)
requires RandomAccessRange<Base>;
\end{itemdecl}

\begin{itemdescr}
\pnum
\effects Equivalent to:
\begin{codeblock}
current_ += n;
return *this;
\end{codeblock}
\end{itemdescr}

\indexlibrary{\idxcode{operator-=}!\idxcode{transform_view::\xname{iterator}}}%
\begin{itemdecl}
constexpr @\xname{iterator}@& operator-=(difference_type n)
requires RandomAccessRange<Base>;
\end{itemdecl}

\begin{itemdescr}
\pnum
\effects Equivalent to:
\begin{codeblock}
current_ -= n;
return *this;
\end{codeblock}
\end{itemdescr}

\rSec5[ranges.adaptors.transform_view.iterator.idx]{\tcode{transform_view::\xname{iterator}} index}

\indexlibrary{\idxcode{operator[]}!\idxcode{transform_view::\xname{iterator}}}
\begin{itemdecl}
constexpr decltype(auto) operator[](difference_type n) const
requires RandomAccessRange<Base>;
\end{itemdecl}

\begin{itemdescr}
\pnum
\effects Equivalent to:
\begin{codeblock}
return invoke(*parent_->fun_, current_[n]);
\end{codeblock}
\end{itemdescr}

\rSec4[ranges.adaptors.transform_view.iterator.comp]{\tcode{transform_view::\xname{iterator}} comparisons}

\indexlibrary{\idxcode{operator==}!\idxcode{transform_view::\xname{iterator}}}
\begin{itemdecl}
friend constexpr bool operator==(const @\xname{iterator}@& x, const @\xname{iterator}@& y)
requires EqualityComparable<iterator_t<Base>>;
\end{itemdecl}

\begin{itemdescr}
\pnum
\returns \tcode{x.current_ == y.current_}.
\end{itemdescr}

\indexlibrary{\idxcode{operator!=}!\idxcode{transform_view::\xname{iterator}}}%
\begin{itemdecl}
friend constexpr bool operator!=(const @\xname{iterator}@& x, const @\xname{iterator}@& y)
requires EqualityComparable<iterator_t<Base>>;
\end{itemdecl}

\begin{itemdescr}
\pnum
\returns \tcode{!(x == y)}.
\end{itemdescr}

\indexlibrary{\idxcode{operator<}!\idxcode{transform_view::\xname{iterator}}}%
\begin{itemdecl}
friend constexpr bool operator<(const @\xname{iterator}@& x, const @\xname{iterator}@& y)
requires RandomAccessRange<Base>;
\end{itemdecl}

\begin{itemdescr}
\pnum
\returns \tcode{x.current_ < y.current_}.
\end{itemdescr}

\indexlibrary{\idxcode{operator>}!\idxcode{transform_view::\xname{iterator}}}%
\begin{itemdecl}
friend constexpr bool operator>(const @\xname{iterator}@& x, const @\xname{iterator}@& y)
requires RandomAccessRange<Base>;
\end{itemdecl}

\begin{itemdescr}
\pnum
\returns \tcode{y < x}.
\end{itemdescr}

\indexlibrary{\idxcode{operator<=}!\idxcode{transform_view::\xname{iterator}}}%
\begin{itemdecl}
friend constexpr bool operator<=(const @\xname{iterator}@& x, const @\xname{iterator}@& y)
requires RandomAccessRange<Base>;
\end{itemdecl}

\begin{itemdescr}
\pnum
\returns \tcode{!(y < x)}.
\end{itemdescr}

\indexlibrary{\idxcode{operator>=}!\idxcode{transform_view::\xname{iterator}}}%
\begin{itemdecl}
friend constexpr bool operator>=(const @\xname{iterator}@& x, const @\xname{iterator}@& y)
requires RandomAccessRange<Base>;
\end{itemdecl}

\begin{itemdescr}
\pnum
\returns \tcode{!(x < y)}.
\end{itemdescr}

\rSec4[ranges.adaptors.transform_view.iterator.nonmember]{\tcode{transform_view::\xname{iterator}} non-member functions}

\indexlibrary{\idxcode{operator+}!\idxcode{transform_view::\xname{iterator}}}
\begin{itemdecl}
friend constexpr @\xname{iterator}@ operator+(@\xname{iterator}@ i, difference_type n)
requires RandomAccessRange<Base>;
friend constexpr @\xname{iterator}@ operator+(difference_type n, @\xname{iterator}@ i)
requires RandomAccessRange<Base>;
\end{itemdecl}

\begin{itemdescr}
\pnum
\returns \tcode{\xname{iterator}\{*i.parent_, i.current_ + n\}}.
\end{itemdescr}

\indexlibrary{\idxcode{operator-}!\idxcode{transform_view::\xname{iterator}}}%
\begin{itemdecl}
friend constexpr @\xname{iterator}@ operator-(@\xname{iterator}@ i, difference_type n)
requires RandomAccessRange<Base>;
\end{itemdecl}

\begin{itemdescr}
\pnum
\returns \tcode{\xname{iterator}\{*i.parent_, i.current_ - n\}}.
\end{itemdescr}

\indexlibrary{\idxcode{operator-}!\idxcode{transform_view::\xname{iterator}}}%
\begin{itemdecl}
friend constexpr difference_type operator-(const @\xname{iterator}@& x, const @\xname{iterator}@& y)
requires RandomAccessRange<Base>;
\end{itemdecl}

\begin{itemdescr}
\pnum
\returns \tcode{x.current_ - y.current_}.
\end{itemdescr}

\indexlibrary{\idxcode{iter_move}!\idxcode{transform_view::\xname{iterator}}}
\begin{itemdecl}
friend constexpr decltype(auto) iter_move(const @\xname{iterator}@& i)
noexcept(@\seebelow@);
\end{itemdecl}

\begin{itemdescr}
\pnum
\effects Equivalent to:
\begin{itemize}
\item If the expression \tcode{*i} is an lvalue, then
\tcode{std::move(*i)}.
\item Otherwise, \tcode{*i}.
\end{itemize}

\pnum
\remarks The expression in the \tcode{noexcept} is equivalent to:
\begin{codeblock}
noexcept(invoke(*i.parent_->fun_, *i.current_))
\end{codeblock}
\end{itemdescr}

\indexlibrary{\idxcode{iter_swap}!\idxcode{transform_view::\xname{iterator}}}
\begin{itemdecl}
friend constexpr void iter_swap(const @\xname{iterator}@& x, const @\xname{iterator}@& y)
noexcept(@\seebelow@) requires IndirectlySwappable<iterator_t<Base>>;
\end{itemdecl}

\begin{itemdescr}
\pnum
\effects Equivalent to \tcode{ranges::iter_swap(x.current_, y.current_)}.

\pnum
\remarks The expression in the \tcode{noexcept} is equivalent to:
\begin{codeblock}
noexcept(ranges::iter_swap(x.current_, y.current_))
\end{codeblock}
\end{itemdescr}


\rSec3[ranges.adaptors.transform_view.sentinel]{Class template \tcode{transform_view::\xname{sentinel}}}

\pnum
\tcode{transform_view<R, F>::\xname{sentinel}} is an exposition-only type.

\begin{codeblock}
namespace std { namespace experimental { namespace ranges { inline namespace v1 {
  template <class R, class F>
  template <bool Const>
  class transform_view<R, F>::@\xname{sentinel}@ {
  private:
    using Parent = conditional_t<Const, const transform_view, transform_view>;
    using Base = conditional_t<Const, const R, R>;
    sentinel_t<Base> end_ {};
  public:
    @\xname{sentinel}@() = default;
    explicit constexpr @\xname{sentinel}@(sentinel_t<Base> end);
    constexpr @\xname{sentinel}@(@\xname{sentinel}@<!Const> i)
      requires Const && ConvertibleTo<sentinel_t<R>, sentinel_t<Base>>;

    constexpr sentinel_t<Base> base() const;

    friend constexpr bool operator==(const @\xname{iterator}@<Const>& x, const @\xname{sentinel}@& y);
    friend constexpr bool operator==(const @\xname{sentinel}@& x, const @\xname{iterator}@<Const>& y);
    friend constexpr bool operator!=(const @\xname{iterator}@<Const>& x, const @\xname{sentinel}@& y);
    friend constexpr bool operator!=(const @\xname{sentinel}@& x, const @\xname{iterator}@<Const>& y);

    friend constexpr difference_type_t<iterator_t<Base>>
      operator-(const @\xname{iterator}@<Const>& x, const @\xname{sentinel}@& y)
        requires SizedSentinel<sentinel_t<Base>, iterator_t<Base>>;
    friend constexpr difference_type_t<iterator_t<Base>>
      operator-(const @\xname{sentinel}@& y, const @\xname{iterator}@<Const>& x)
        requires SizedSentinel<sentinel_t<Base>, iterator_t<Base>>;
  };
}}}}
\end{codeblock}

\rSec3[ranges.adaptors.transform_view.sentinel.ctor]{\tcode{transform_view::\xname{sentinel} constructors}}

\indexlibrary{\idxcode{\xname{sentinel}}!\idxcode{transform_view::\xname{sentinel}}}
\begin{itemdecl}
explicit constexpr @\xname{sentinel}@(sentinel_t<Base> end);
\end{itemdecl}

\begin{itemdescr}
\pnum
\effects Initializes \tcode{end_} with \tcode{end}.
\end{itemdescr}

\indexlibrary{\idxcode{\xname{sentinel}}!\idxcode{transform_view::\xname{sentinel}}}
\begin{itemdecl}
constexpr @\xname{sentinel}@(@\xname{sentinel}@<!Const> i)
requires Const && ConvertibleTo<sentinel_t<R>, sentinel_t<Base>>;
\end{itemdecl}

\begin{itemdescr}
\pnum
\effects Initializes \tcode{end_} with \tcode{i.end_}.
\end{itemdescr}

\rSec3[ranges.adaptors.transform_view.sentinel.conv]{\tcode{transform_view::\xname{sentinel} conversion}}

\indexlibrary{\idxcode{base}!\idxcode{transform_view::\xname{sentinel}}}
\begin{itemdecl}
constexpr sentinel_t<Base> base() const;
\end{itemdecl}

\begin{itemdescr}
\pnum
\returns \tcode{end_}.
\end{itemdescr}

\rSec3[ranges.adaptors.transform_view.sentinel.comp]{\tcode{transform_view::\xname{sentinel} comparison}}

\indexlibrary{\idxcode{operator==}!\idxcode{transform_view::\xname{sentinel}}}
\begin{itemdecl}
friend constexpr bool operator==(const @\xname{iterator}@<Const>& x, const @\xname{sentinel}@& y);
\end{itemdecl}

\begin{itemdescr}
\pnum
\returns \tcode{x.current_ == y.end_}.
\end{itemdescr}

\indexlibrary{\idxcode{operator==}!\idxcode{transform_view::\xname{sentinel}}}
\begin{itemdecl}
friend constexpr bool operator==(const @\xname{sentinel}@& x, const @\xname{iterator}@<Const>& y);
\end{itemdecl}

\begin{itemdescr}
\pnum
\returns \tcode{y == x}.
\end{itemdescr}

\indexlibrary{\idxcode{operator!=}!\idxcode{transform_view::\xname{sentinel}}}%
\begin{itemdecl}
friend constexpr bool operator!=(const @\xname{iterator}@<Const>& x, const @\xname{sentinel}@& y);
\end{itemdecl}

\begin{itemdescr}
\pnum
\returns \tcode{!(x == y)}.
\end{itemdescr}

\indexlibrary{\idxcode{operator!=}!\idxcode{transform_view::\xname{sentinel}}}%
\begin{itemdecl}
friend constexpr bool operator!=(const @\xname{sentinel}@& x, const @\xname{iterator}@<Const>& y);
\end{itemdecl}

\begin{itemdescr}
\pnum
\returns \tcode{!(y == x)}.
\end{itemdescr}

\rSec3[ranges.adaptors.transform_view.sentinel.nonmember]{\tcode{transform_view::\xname{sentinel}} non-member functions}

\indexlibrary{\idxcode{operator!=}!\idxcode{transform_view::\xname{sentinel}}}%
\begin{itemdecl}
friend constexpr difference_type_t<iterator_t<Base>>
operator-(const @\xname{iterator}@<Const>& x, const @\xname{sentinel}@& y)
  requires SizedSentinel<sentinel_t<Base>, iterator_t<Base>>;
\end{itemdecl}

\begin{itemdescr}
\pnum
\returns \tcode{x.current_ - y.end_}.
\end{itemdescr}

\indexlibrary{\idxcode{operator!=}!\idxcode{transform_view::\xname{sentinel}}}%
\begin{itemdecl}
friend constexpr difference_type_t<iterator_t<Base>>
operator-(const @\xname{sentinel}@& y, const @\xname{iterator}@<Const>& x)
  requires SizedSentinel<sentinel_t<Base>, iterator_t<Base>>;
\end{itemdecl}

\begin{itemdescr}
\pnum
\returns \tcode{x.end_ - y.current_}.
\end{itemdescr}

\rSec2[ranges.adaptors.transform]{\tcode{view::transform}}

\pnum
The name \tcode{view::transform} denotes a range adaptor
object~(\ref{ranges.adaptor.object}). Let \tcode{E} and \tcode{F} be expressions such
that types \tcode{T} and \tcode{U} are \tcode{decltype((E))} and \tcode{decltype((F))}
respectively. Then the expression \tcode{view::transform(E, F)} is expression-equivalent to:

\begin{itemize}
\item \tcode{transform_view\{E, F\}} if
\tcode{InputRange<T> \&\& CopyConstructible<decay_t<U>{>} \&\& Invocable<decay_t<U>\&, reference_t<iterator_t<T>{>}>}
is satisfied.
\item Otherwise, \tcode{view::transform(E, F)} is ill-formed.
\end{itemize}

\rSec2[ranges.adaptors.iota_view]{Class template \tcode{iota_view}}

\pnum
The purpose of \tcode{iota_view} is to generate a sequence of elements by
monotonically incrementing an initial value.

\ednote{The following definition of \tcode{iota_view} presumes the resolution of
stl2\#507 (\url{https://github.com/ericniebler/stl2/issues/507}).}

\pnum
\enterexample
\begin{codeblock}
iota_view indices{1, 10};
for (int i : squares)
  cout << i << ' '; // prints: 1 2 3 4 5 6 7 8 9
\end{codeblock}
\exitexample

\begin{codeblock}
namespace std { namespace experimental { namespace ranges { inline namespace v1 {
  // \expos
  template <class I>
  concept bool @\textit{decrementable}@ = @\seebelow@;
  // \expos
  template <class I>
  concept bool @\textit{advanceable}@ = @\seebelow@;

  template <WeaklyIncrementable I, class Bound = unreachable>
    requires WeaklyEqualityComparable<I, Bound>
  @\changed{struct}{class}@ iota_view : @\added{public}@ view_interface<iota_view<I, Bound>> {
  private:
    I value_ {}; // \expos
    @\changed{I}{Bound}@ bound_ {}; // \expos
  public:
    iota_view() = default;
    constexpr explicit iota_view(I value)@\added{;} \removed{requires Same<Bound, unreachable>}@
    constexpr iota_view(I value, Bound bound); // \seebelow

    struct iterator;
    struct sentinel;

    constexpr iterator begin() const;
    constexpr sentinel end() const;
    constexpr iterator end() const requires Same<I, Bound>;

    constexpr auto size() const requires @\seebelow@;
  };

  template <WeaklyIncrementable I>
  @\removed{explicit}@ iota_view(I) -> iota_view<I>;

  template <Incrementable I>
  iota_view(I, I) -> iota_view<I, I>;

  template <WeaklyIncrementable I, Semiregular Bound>
    requires WeaklyEqualityComparable<I, Bound> && !ConvertibleTo<Bound, I>
  iota_view(I, Bound) -> iota_view<I, Bound>;
}}}}
\end{codeblock}

\pnum
The exposition-only \tcode{\textit{decrementable}} concept is equivalent to:
\begin{codeblock}
template <class I>
concept bool @\textit{decrementable}@ =
Incrementable<I> && requires(I i) {
  { --i } -> Same<I>&;
  { i-- } -> Same<I>&&;
};
\end{codeblock}

\pnum
When an object is in the domain of both pre- and post-decrement,
the object is said to be \techterm{decrementable}.

\pnum
Let \tcode{a} and \tcode{b} be incrementable and decrementable objects of type
\tcode{I}. \tcode{\textit{decrementable}<I>} is satisfied only if
\begin{itemize}
\item \tcode{\&(--a) == \&a};
\item If \tcode{bool(a == b)} then \tcode{bool(a-- == b)}.
\item If \tcode{bool(a == b)} then \tcode{bool((a--, a) == --b)}.
\item If \tcode{bool(a == b)} then \tcode{bool(--(++a) == b)} and
\tcode{bool(++(--a) == b)}.
\end{itemize}

\pnum
The exposition-only \tcode{\textit{advanceable}} concept is equivalent to:
\begin{codeblock}
template <class I>
concept bool @\textit{advanceable}@ =
@\textit{decrementable}@<I> && StrictTotallyOrdered<I> &&
  requires { typename difference_type_t<I>; } &&
  requires(I a, const I b, const difference_type_t<I> n) {
    { a += n } -> Same<I>&;
    { b + n } -> Same<I>&&;
    { n + b } -> Same<I>&&;
    { a -= n } -> Same<I>&;
    { b - n } -> Same<I>&&;
    { b - b } -> Same<difference_type_t<I>>&&;
  };
\end{codeblock}

Let \tcode{a} and \tcode{b} be objects of type \tcode{I} such that \tcode{b} is
reachable from \tcode{a}. Let \tcode{n} be the smallest number of applications
of \tcode{++a} necessary to make \tcode{bool(a == b)} be \tcode{true}. Then if
\tcode{n} is representable by \tcode{difference_type_t<I>},
\tcode{\textit{advanceable}<I>} is satisfied only if:
\begin{itemize}
\item \tcode{(a += n)} is equal to \tcode{b}.
\item \tcode{\&(a += n)} is equal to \tcode{\&a}.
\item \tcode{(a + n)} is equal to \tcode{(a += n)}.
\item For any two positive integers \tcode{x} and \tcode{y}, if
\tcode{a + (x + y)} is valid, then \tcode{a + (x + y)} is equal to
\tcode{(a + x) + y}.
\item \tcode{a + 0} is equal to \tcode{a}.
\item If \tcode{(a + (n - 1))} is valid, then \tcode{a + n} is equal to
\tcode{++(a + (n - 1))}.
\item \tcode{(b += -n)} is equal to \tcode{a}.
\item \tcode{(b -= n)} is equal to \tcode{a}.
\item \tcode{\&(b -= n)} is equal to \tcode{\&b}.
\item \tcode{(b - n)} is equal to \tcode{(b -= n)}.
\item \tcode{b - a} is equal to \tcode{n}.
\item \tcode{a - b} is equal to \tcode{-n}.
\item \tcode{a <= b}.
\end{itemize}

\rSec3[ranges.adaptors.iota_view.ops]{\tcode{iota_view} operations}
\rSec4[ranges.adaptors.iota_view.ctor]{\tcode{iota_view} constructors}

\indexlibrary{\idxcode{iota_view}!\idxcode{iota_view}}%
\begin{itemdecl}
constexpr explicit iota_view(I value)@\added{;} \removed{requires Same<Bound, unreachable>}@
\end{itemdecl}

\begin{itemdescr}
\begin{addedblock}
\pnum
\requires \tcode{Bound\{\}} is reachable from \tcode{value}.
\end{addedblock}

\pnum
\effects Initializes \tcode{value_} with \tcode{value}.
\end{itemdescr}

\indexlibrary{\idxcode{iota_view}!\idxcode{iota_view}}%
\begin{itemdecl}
constexpr iota_view(I value, Bound bound);
\end{itemdecl}

\begin{itemdescr}
\pnum
\requires \tcode{bound} is reachable from \tcode{value}.

\pnum
\effects Initializes \tcode{value_} with \tcode{value} and \tcode{bound_} with
\tcode{bound}.

\pnum
\remarks This constructor does not contribute a function template to the overload
set used when resolving a placeholder for a deduced class
type~(\cxxref{over.match.class.deduct}).
\end{itemdescr}

\rSec4[ranges.adaptors.iota_view.begin]{\tcode{iota_view} range begin}

\indexlibrary{\idxcode{begin}!\idxcode{iota_view}}%
\begin{itemdecl}
constexpr iterator begin() const;
\end{itemdecl}

\begin{itemdescr}
\pnum
\returns \tcode{iterator\{value_\}}.
\end{itemdescr}

\rSec4[ranges.adaptors.iota_view.end]{\tcode{iota_view} range end}

\indexlibrary{\idxcode{end}!\idxcode{iota_view}}%
\begin{itemdecl}
constexpr sentinel end() const;
\end{itemdecl}

\begin{itemdescr}
\pnum
\returns \tcode{sentinel\{bound_\}}.
\end{itemdescr}

\indexlibrary{\idxcode{end}!\idxcode{iota_view}}%
\begin{itemdecl}
constexpr iterator end() const requires Same<I, Bound>;
\end{itemdecl}

\begin{itemdescr}
\pnum
\returns \tcode{iterator\{bound_\}}.
\end{itemdescr}

\rSec4[ranges.adaptors.iota_view.size]{\tcode{iota_view} range size}

\indexlibrary{\idxcode{size}!\idxcode{iota_view}}%
\begin{itemdecl}
constexpr auto size() const requires @\seebelow@;
\end{itemdecl}

\begin{itemdescr}
\pnum
\returns \tcode{bound_ - value_}.

\pnum
\remarks The expression in the \tcode{requires} clause is equivalent to:
\begin{codeblock}
(Same<I, Bound> && @\textit{advanceable}@<I>) ||
(Integral<I> && Integral<Bound>) ||
SizedSentinel<Bound, I>
\end{codeblock}
\end{itemdescr}

\rSec3[ranges.adaptors.iota_view.iterator]{Class \tcode{iota_view::iterator}}

\begin{codeblock}
namespace std { namespace experimental { namespace ranges { inline namespace v1 {
  template <class I, class Bound>
  struct iota_view<I, Bound>::iterator {
  private:
    I value_ {}; // \expos
  public:
    using iterator_category = @\seebelow@;
    using value_type = I;
    using difference_type = difference_type_t<I>;

    iterator() = default;
    explicit constexpr iterator(I value);

    constexpr I operator*() const noexcept(is_nothrow_copy_constructible_v<I>);

    constexpr iterator& operator++();
    constexpr void operator++(int);
    constexpr iterator operator++(int) requires Incrementable<I>;

    constexpr iterator& operator--() requires @\textit{decrementable}@<I>;
    constexpr iterator operator--(int) requires @\textit{decrementable}@<I>;

    constexpr iterator& operator+=(difference_type n)
      requires @\textit{advanceable}@<I>;
    constexpr iterator& operator-=(difference_type n)
      requires @\textit{advanceable}@<I>;
    constexpr I operator[](difference_type n) const
      requires @\textit{advanceable}@<I>;

    friend constexpr bool operator==(const iterator& x, const iterator& y)
      requires EqualityComparable<I>;
    friend constexpr bool operator!=(const iterator& x, const iterator& y)
      requires EqualityComparable<I>;

    friend constexpr bool operator<(const iterator& x, const iterator& y)
      requires StrictTotallyOrdered<I>;
    friend constexpr bool operator>(const iterator& x, const iterator& y)
      requires StrictTotallyOrdered<I>;
    friend constexpr bool operator<=(const iterator& x, const iterator& y)
      requires StrictTotallyOrdered<I>;
    friend constexpr bool operator>=(const iterator& x, const iterator& y)
      requires StrictTotallyOrdered<I>;

    friend constexpr iterator operator+(iterator i, difference_type n)
      requires @\textit{advanceable}@<I>;
    friend constexpr iterator operator+(difference_type n, iterator i)
      requires @\textit{advanceable}@<I>;

    friend constexpr iterator operator-(iterator i, difference_type n)
      requires @\textit{advanceable}@<I>;
    friend constexpr difference_type operator-(const iterator& x, const iterator& y)
      requires @\textit{advanceable}@<I>;
  };
}}}}
\end{codeblock}

\pnum
\tcode{iota_view<I, Bound>::iterator::iterator_category} is defined as follows:
\begin{itemize}
\item If \tcode{I} satisfies \tcode{\textit{advanceable}}, then
\tcode{iterator_category} is \tcode{ranges::random_access_iterator_tag}.
\item Otherwise, if \tcode{I} satisfies \tcode{\textit{decrementable}}, then
\tcode{iterator_category} is \tcode{ranges::bidirectional_iterator_tag}.
\item Otherwise, if \tcode{I} satisfies \tcode{Incrementable}, then
\tcode{iterator_category} is \tcode{ranges::forward_iterator_tag}.
\item Otherwise, \tcode{iterator_category} is \tcode{ranges::input_iterator_tag}.
\end{itemize}

\pnum
\enternote Overloads for \tcode{iter_move} and \tcode{iter_swap} are omitted
intentionally.\exitnote

\rSec4[ranges.adaptors.iota_view.iterator.ops]{\tcode{iota_view::iterator} operations}
\rSec5[ranges.adaptors.iota_view.iterator.ctor]{\tcode{iota_view::iterator} constructors}

\indexlibrary{\idxcode{iterator}!\idxcode{iota_view::iterator}}
\begin{itemdecl}
explicit constexpr iterator(I value);
\end{itemdecl}

\begin{itemdescr}
\pnum
\effects Initializes \tcode{value_} with \tcode{value}.
\end{itemdescr}

\rSec5[ranges.adaptors.iota_view.iterator.star]{\tcode{iota_view::iterator::operator*}}

\indexlibrary{\idxcode{operator*}!\idxcode{iota_view::iterator}}
\begin{itemdecl}
constexpr I operator*() const noexcept(is_nothrow_copy_constructible_v<I>);
\end{itemdecl}

\begin{itemdescr}
\pnum
\returns \tcode{value_}.

\pnum
\enternote The \tcode{noexcept} clause is needed by the default \tcode{iter_move}
implementation.\exitnote
\end{itemdescr}

\rSec5[ranges.adaptors.iota_view.iterator.inc]{\tcode{iota_view::iterator::operator++}}

\indexlibrary{\idxcode{operator++}!\idxcode{iota_view::iterator}}
\begin{itemdecl}
constexpr iterator& operator++();
\end{itemdecl}

\begin{itemdescr}
\pnum
\effects Equivalent to:
\begin{codeblock}
++value_;
return *this;
\end{codeblock}
\end{itemdescr}

\indexlibrary{\idxcode{operator++}!\idxcode{iota_view::iterator}}
\begin{itemdecl}
constexpr void operator++(int);
\end{itemdecl}

\begin{itemdescr}
\pnum
\effects Equivalent to \tcode{++*this}.
\end{itemdescr}

\indexlibrary{\idxcode{operator++}!\idxcode{iota_view::iterator}}
\begin{itemdecl}
constexpr iterator operator++(int) requires Incrementable<I>;
\end{itemdecl}

\begin{itemdescr}
\pnum
\effects Equivalent to:
\begin{codeblock}
auto tmp = *this;
++*this;
return tmp;
\end{codeblock}
\end{itemdescr}

\rSec5[ranges.adaptors.iota_view.iterator.dec]{\tcode{iota_view::iterator::operator\dcr}}

\indexlibrary{\idxcode{operator\dcr}!\idxcode{iota_view::iterator}}
\begin{itemdecl}
constexpr iterator& operator--() requires @\textit{decrementable}@<I>;
\end{itemdecl}

\begin{itemdescr}
\pnum
\effects Equivalent to:
\begin{codeblock}
--value_;
return *this;
\end{codeblock}
\end{itemdescr}

\indexlibrary{\idxcode{operator\dcr}!\idxcode{iota_view::iterator}}
\begin{itemdecl}
constexpr iterator operator--(int) requires @\textit{decrementable}@<I>;
\end{itemdecl}

\begin{itemdescr}
\pnum
\effects Equivalent to:
\begin{codeblock}
auto tmp = *this;
--*this;
return tmp;
\end{codeblock}
\end{itemdescr}

\rSec5[ranges.adaptors.iota_view.iterator.adv]{\tcode{iota_view::iterator} advance}

\indexlibrary{\idxcode{operator+=}!\idxcode{iota_view::iterator}}
\begin{itemdecl}
constexpr iterator& operator+=(difference_type n)
requires @\textit{advanceable}@<I>;
\end{itemdecl}

\begin{itemdescr}
\pnum
\effects Equivalent to:
\begin{codeblock}
value_ += n;
return *this;
\end{codeblock}
\end{itemdescr}

\indexlibrary{\idxcode{operator-=}!\idxcode{iota_view::iterator}}
\begin{itemdecl}
constexpr iterator& operator-=(difference_type n)
requires @\textit{advanceable}@<I>;
\end{itemdecl}

\begin{itemdescr}
\pnum
\effects Equivalent to:
\begin{codeblock}
value_ -= n;
return *this;
\end{codeblock}
\end{itemdescr}

\rSec5[ranges.adaptors.iota_view.iterator.idx]{\tcode{iota_view::iterator} index}

\indexlibrary{\idxcode{operator[]}!\idxcode{iota_view::iterator}}
\begin{itemdecl}
constexpr I operator[](difference_type n) const
requires @\textit{advanceable}@<I>;
\end{itemdecl}

\begin{itemdescr}
\pnum
\returns \tcode{value_ + n}.
\end{itemdescr}

\rSec4[ranges.adaptors.iota_view.iterator.cmp]{\tcode{iota_view::iterator} comparisons}

\indexlibrary{\idxcode{operator==}!\idxcode{iota_view::iterator}}
\begin{itemdecl}
friend constexpr bool operator==(const iterator& x, const iterator& y)
requires EqualityComparable<I>;
\end{itemdecl}

\begin{itemdescr}
\pnum
\returns \tcode{x.value_ == y.value_}.
\end{itemdescr}

\indexlibrary{\idxcode{operator!=}!\idxcode{iota_view::iterator}}
\begin{itemdecl}
friend constexpr bool operator!=(const iterator& x, const iterator& y)
requires EqualityComparable<I>;
\end{itemdecl}

\begin{itemdescr}
\pnum
\returns \tcode{!(x == y)}.
\end{itemdescr}

\indexlibrary{\idxcode{operator<}!\idxcode{iota_view::iterator}}
\begin{itemdecl}
friend constexpr bool operator<(const iterator& x, const iterator& y)
requires StrictTotallyOrdered<I>;
\end{itemdecl}

\begin{itemdescr}
\pnum
\returns \tcode{x.value_ < y.value_}.
\end{itemdescr}

\indexlibrary{\idxcode{operator>}!\idxcode{iota_view::iterator}}
\begin{itemdecl}
friend constexpr bool operator>(const iterator& x, const iterator& y)
requires StrictTotallyOrdered<I>;
\end{itemdecl}

\begin{itemdescr}
\pnum
\returns \tcode{y < x}.
\end{itemdescr}

\indexlibrary{\idxcode{operator<=}!\idxcode{iota_view::iterator}}
\begin{itemdecl}
friend constexpr bool operator<=(const iterator& x, const iterator& y)
requires StrictTotallyOrdered<I>;
\end{itemdecl}

\begin{itemdescr}
\pnum
\returns \tcode{!(y < x)}.
\end{itemdescr}

\indexlibrary{\idxcode{operator>=}!\idxcode{iota_view::iterator}}
\begin{itemdecl}
friend constexpr bool operator>=(const iterator& x, const iterator& y)
requires StrictTotallyOrdered<I>;
\end{itemdecl}

\begin{itemdescr}
\pnum
\returns \tcode{!(x < y)}.
\end{itemdescr}

\rSec4[ranges.adaptors.iota_view.iterator.nonmember]{\tcode{iota_view::iterator} non-member functions}

\indexlibrary{\idxcode{operator+}!\idxcode{iota_view::iterator}}
\begin{itemdecl}
friend constexpr iterator operator+(iterator i, difference_type n)
requires @\textit{advanceable}@<I>;
\end{itemdecl}

\begin{itemdescr}
\pnum
\returns \tcode{iterator\{*i + n\}}.
\end{itemdescr}

\indexlibrary{\idxcode{operator+}!\idxcode{iota_view::iterator}}
\begin{itemdecl}
friend constexpr iterator operator+(difference_type n, iterator i)
requires @\textit{advanceable}@<I>;
\end{itemdecl}

\begin{itemdescr}
\pnum
\returns \tcode{i + n}.
\end{itemdescr}

\indexlibrary{\idxcode{operator-}!\idxcode{iota_view::iterator}}
\begin{itemdecl}
friend constexpr iterator operator-(iterator i, difference_type n)
requires @\textit{advanceable}@<I>;
\end{itemdecl}

\begin{itemdescr}
\pnum
\returns \tcode{i + -n}.
\end{itemdescr}

\indexlibrary{\idxcode{operator-}!\idxcode{iota_view::iterator}}
\begin{itemdecl}
friend constexpr difference_type operator-(const iterator& x, const iterator& y)
requires @\textit{advanceable}@<I>;
\end{itemdecl}

\begin{itemdescr}
\pnum
\returns \tcode{*x - *y}.
\end{itemdescr}

\rSec3[ranges.adaptors.iota_view.sentinel]{Class \tcode{iota_view::sentinel}}

\begin{codeblock}
namespace std { namespace experimental { namespace ranges { inline namespace v1 {
  template <class I, class Bound>
  struct iota_view<I, Bound>::sentinel {
  private:
    Bound bound_ {}; // \expos
  public:
    sentinel() = default;
    constexpr explicit sentinel(Bound bound);

    friend constexpr bool operator==(const iterator& x, const sentinel& y);
    friend constexpr bool operator==(const sentinel& x, const iterator& y);
    friend constexpr bool operator!=(const iterator& x, const sentinel& y);
    friend constexpr bool operator!=(const sentinel& x, const iterator& y);
  };
}}}}
\end{codeblock}

\rSec4[ranges.adaptors.iota_view.sentinel.ctor]{\tcode{iota_view::sentinel} constructors}

\indexlibrary{\idxcode{sentinel}!\idxcode{iota_view::sentinel}}
\begin{itemdecl}
constexpr explicit sentinel(Bound bound);
\end{itemdecl}

\begin{itemdescr}
\pnum
\effects Initializes \tcode{bound_} with \tcode{bound}.
\end{itemdescr}

\rSec4[ranges.adaptors.iota_view.sentinel.cmp]{\tcode{iota_view::sentinel} comparisons}

\indexlibrary{\idxcode{operator==}!\idxcode{iota_view::sentinel}}
\begin{itemdecl}
friend constexpr bool operator==(const iterator& x, const sentinel& y);
\end{itemdecl}

\begin{itemdescr}
\pnum
\returns \tcode{x.value_ == y.bound_}.
\end{itemdescr}

\indexlibrary{\idxcode{operator==}!\idxcode{iota_view::sentinel}}
\begin{itemdecl}
friend constexpr bool operator==(const sentinel& x, const iterator& y);
\end{itemdecl}

\begin{itemdescr}
\pnum
\returns \tcode{y == x}.
\end{itemdescr}

\indexlibrary{\idxcode{operator!=}!\idxcode{iota_view::sentinel}}
\begin{itemdecl}
friend constexpr bool operator!=(const iterator& x, const sentinel& y);
\end{itemdecl}

\begin{itemdescr}
\pnum
\returns \tcode{!(x == y)}.
\end{itemdescr}

\indexlibrary{\idxcode{operator!=}!\idxcode{iota_view::sentinel}}
\begin{itemdecl}
friend constexpr bool operator!=(const sentinel& x, const iterator& y);
\end{itemdecl}

\begin{itemdescr}
\pnum
\returns \tcode{!(y == x)}.
\end{itemdescr}

\rSec2[ranges.adaptors.iota]{\tcode{view::iota}}

\pnum
The name \tcode{view::iota} denotes a customization point object~(\cxxref{???}).
Let \tcode{E} and \tcode{F} be expressions such that their un-\cv\ qualified types
are \tcode{I} and \tcode{J} respectively. Then the expression \tcode{view::iota(E)}
is expression-equivalent to:

\begin{itemize}
\item \tcode{iota_view\{E\}} if \tcode{WeaklyIncrementable<I>} is satisfied.
\item Otherwise, \tcode{view::iota(E)} is ill-formed.
\end{itemize}

\pnum
The expression \tcode{view::iota(E, F)} is expression-equivalent to:

\begin{itemize}
\item \tcode{iota_view\{E, F\}} if either of the following sets of constraints
is satisfied:
\begin{itemize}
\item \begin{codeblock}
Incrementable<I> && Same<I, J>
\end{codeblock}

\item \begin{codeblock}
WeaklyIncrementable<I> && Semiregular<J> &&
WeaklyEqualityComparable<I, J> && !ConvertibleTo<J, I>
\end{codeblock}
\end{itemize}
\item Otherwise, \tcode{view::iota(E, F)} is ill-formed.
\end{itemize}

\rSec2[ranges.adaptors.take_view]{Class template \tcode{take_view}}

\pnum
The purpose of \tcode{take_view} is to produce a range of the first
\tcode{\textit{N}} elements from another range.

\pnum
\enterexample
\begin{codeblock}
vector<int> is{0,1,2,3,4,5,6,7,8,9};
take_view few{is, 5};
for (int i : few)
  cout << i << ' '; // prints: 0 1 2 3 4
\end{codeblock}
\exitexample

\begin{codeblock}
namespace std { namespace experimental { namespace ranges { inline namespace v1 {
  template <InputRange R>
    requires View<R>
  @\changed{struct}{class}@ take_view : @\added{public}@ view_interface<take_view<R>> {
  private:
    R base_ {}; // \expos
    difference_type_t<iterator_t<R>> count_ {}; // \expos
    template <bool Const>
      struct @\xname{sentinel}@; // \expos
  public:
    take_view() = default;
    constexpr take_view(R base, difference_type_t<iterator_t<R>> count);
    template <InputRange O>
      requires @\changed{(is_lvalue_reference_v<O> || View<decay_t<O>{>})}{ViewableRange<O>}@ &&
          Constructible<R, all_view<O>>
    constexpr take_view(O&& o, difference_type_t<iterator_t<R>> count);

    constexpr R base() const;

    constexpr auto begin();
    constexpr auto begin() const requires Range<const R>;
    constexpr auto begin() requires RandomAccessRange<R> && SizedRange<R>;
    constexpr auto begin() const
      requires RandomAccessRange<const R> && SizedRange<const R>;

    constexpr auto end();
    constexpr auto end() const requires Range<const R>;
    constexpr auto end() requires RandomAccessRange<R> && SizedRange<R>;
    constexpr auto end() const
      requires RandomAccessRange<const R> && SizedRange<const R>;

    constexpr auto size() requires SizedRange<R>;
    constexpr auto size() const requires SizedRange<const R>;

    using iterator = iterator_t<take_view>;
    using const_iterator = @\seebelow@;
    using sentinel = sentinel_t<take_view>;
    using const_sentinel = @\seebelow@;
  };

  template <InputRange R>
  take_view(R&& base, difference_type_t<iterator_t<R>> n)
    -> take_view<all_view<R>>;
}}}}
\end{codeblock}

\pnum
\tcode{take_view<R>::const_iterator} is defined as follows:
\begin{itemize}
\item If \tcode{const R} satisfies \tcode{Range} then \tcode{const_iterator} is
an alias for \tcode{iterator_t<const take_view>}.
\item Otherwise, there is no type \tcode{take_view<R>::const_iterator}.
\end{itemize}

\pnum
\tcode{take_view<R>::const_sentinel} is defined as follows:
\begin{itemize}
\item If \tcode{const R} satisfies \tcode{Range} then \tcode{const_sentinel} is
an alias for \tcode{sentinel_t<const take_view>}.
\item Otherwise, there is no type \tcode{take_view<R>::const_sentinel}.
\end{itemize}

\rSec3[ranges.adaptors.take_view.ops]{\tcode{take_view} operations}
\rSec4[ranges.adaptors.take_view.ctor]{\tcode{take_view} constructors}

\indexlibrary{\idxcode{take_view}!\idxcode{take_view}}%
\begin{itemdecl}
constexpr take_view(R base, difference_type_t<iterator_t<R>> count);
\end{itemdecl}

\begin{itemdescr}
\pnum
\effects Initializes \tcode{base_} with \tcode{std::move(base)} and initializes
\tcode{count_} with \tcode{count}.
\end{itemdescr}

\indexlibrary{\idxcode{take_view}!\idxcode{take_view}}%
\begin{itemdecl}
template <InputRange O>
requires @\changed{(is_lvalue_reference_v<O> || View<decay_t<O>{>})}{ViewableRange<O>}@ &&
    Constructible<R, all_view<O>>
constexpr take_view(O&& o, difference_type_t<iterator_t<R>> count);
\end{itemdecl}

\begin{itemdescr}
\pnum
\effects Initializes \tcode{base_} with \tcode{view::all(std::forward<O>(o))}
and initializes \tcode{count_} with \tcode{count}.
\end{itemdescr}

\rSec4[ranges.adaptors.take_view.conv]{\tcode{take_view} conversion}

\indexlibrary{\idxcode{base}!\idxcode{take_view}}%
\begin{itemdecl}
constexpr R base() const;
\end{itemdecl}

\begin{itemdescr}
\pnum
\returns \tcode{base_}.
\end{itemdescr}

\rSec4[ranges.adaptors.take_view.begin]{\tcode{take_view} range begin}

\indexlibrary{\idxcode{begin}!\idxcode{take_view}}%
\begin{itemdecl}
constexpr auto begin();
constexpr auto begin() const requires Range<const R>;
\end{itemdecl}

\begin{itemdescr}
\pnum
\effects Equivalent to:
\begin{codeblock}
return make_counted_iterator(ranges::begin(base_), count_);
\end{codeblock}
\end{itemdescr}

\indexlibrary{\idxcode{begin}!\idxcode{take_view}}%
\begin{itemdecl}
constexpr auto begin() requires RandomAccessRange<R> && SizedRange<R>;
constexpr auto begin() const
requires RandomAccessRange<const R> && SizedRange<const R>;
\end{itemdecl}

\begin{itemdescr}
\pnum
\effects Equivalent to:
\begin{codeblock}
return ranges::begin(base_);
\end{codeblock}
\end{itemdescr}

\rSec4[ranges.adaptors.take_view.end]{\tcode{take_view} range end}

\indexlibrary{\idxcode{end}!\idxcode{take_view}}%
\begin{itemdecl}
constexpr auto end();
constexpr auto end() const requires Range<const R>;
\end{itemdecl}

\begin{itemdescr}
\pnum
\effects Equivalent to \tcode{\xname{sentinel}<\changed{false}{\textit{simple-view}<R>}>\{ranges::end(base_)\}} and
\tcode{\xname{sentinel}<\ true>\{ranges::end(base_)\}} for the first and second
overload, respectively.
\end{itemdescr}

\indexlibrary{\idxcode{end}!\idxcode{take_view}}%
\begin{itemdecl}
constexpr auto end() requires RandomAccessRange<R> && SizedRange<R>;
constexpr auto end() const
requires RandomAccessRange<const R> && SizedRange<const R>;
\end{itemdecl}

\begin{itemdescr}
\pnum
\effects Equivalent to:
\begin{codeblock}
return ranges::begin(base_) + size();
\end{codeblock}
\end{itemdescr}

\rSec4[ranges.adaptors.take_view.size]{\tcode{take_view} range size}

\indexlibrary{\idxcode{size}!\idxcode{take_view}}%
\begin{itemdecl}
constexpr auto size() requires SizedRange<R>;
constexpr auto size() const requires SizedRange<const R>;
\end{itemdecl}

\begin{itemdescr}
\pnum
\effects Equivalent to
\tcode{ranges::size(base_) < count_ ? ranges::size(base_) : count_}, except with
only one call to \tcode{ranges::size(base_)}.
\end{itemdescr}

\rSec3[ranges.adaptors.take_view.sentinel]{Class template \tcode{take_view::\xname{sentinel}}}

\pnum
\tcode{take_view<R>::\xname{sentinel}} is an exposition-only type.

\begin{codeblock}
namespace std { namespace experimental { namespace ranges { inline namespace v1 {
  template <class R>
  template <bool Const>
  class take_view<R>::@\xname{sentinel}@ { // \expos
  private:
    using Parent = conditional_t<Const, const take_view, take_view>;
    using Base = conditional_t<Const, const R, R>;
    sentinel_t<Base> end_ {};
    using CI = counted_iterator<iterator_t<Base>>;
  public:
    @\xname{sentinel}@() = default;
    constexpr explicit @\xname{sentinel}@(sentinel_t<Base> end);
    constexpr @\xname{sentinel}@(@\xname{sentinel}@<!Const> s)
      requires Const && ConvertibleTo<sentinel_t<R>, sentinel_t<Base>>;

    constexpr sentinel_t<Base> base() const;

    friend constexpr bool operator==(const @\xname{sentinel}@& x, const CI& y)
      requires EqualityComparable<iterator_t<Base>>;
    friend constexpr bool operator==(const CI& x, const @\xname{sentinel}@& y)
      requires EqualityComparable<iterator_t<Base>>;
    friend constexpr bool operator!=(const @\xname{sentinel}@& x, const CI& y)
      requires EqualityComparable<iterator_t<Base>>;
    friend constexpr bool operator!=(const CI& x, const @\xname{sentinel}@& y)
      requires EqualityComparable<iterator_t<Base>>;
  };
}}}}
\end{codeblock}

\rSec4[ranges.adaptors.take_view.sentinel.ops]{\tcode{take_view::\xname{sentinel}} operations}
\rSec5[ranges.adaptors.take_view.sentinel.ctor]{\tcode{take_view::\xname{sentinel}} constructors}

\indexlibrary{\idxcode{\xname{sentinel}}!\idxcode{take_view::\xname{sentinel}}}
\begin{itemdecl}
constexpr explicit @\xname{sentinel}@(sentinel_t<Base> end);
\end{itemdecl}

\begin{itemdescr}
\pnum
\effects Initializes \tcode{end_} with \tcode{end}.
\end{itemdescr}

\indexlibrary{\idxcode{\xname{sentinel}}!\idxcode{take_view::\xname{sentinel}}}%
\begin{itemdecl}
constexpr @\xname{sentinel}@(@\xname{sentinel}@<!Const> s)
requires Const && ConvertibleTo<sentinel_t<R>, sentinel_t<Base>>;
\end{itemdecl}

\begin{itemdescr}
\pnum
\effects Initializes \tcode{end_} with \tcode{s.end_}.
\end{itemdescr}

\rSec5[ranges.adaptors.take_view.sentinel.conv]{\tcode{take_view::\xname{sentinel}} conversion}

\indexlibrary{\idxcode{base}!\idxcode{take_view::\xname{sentinel}}}
\begin{itemdecl}
constexpr sentinel_t<Base> base() const;
\end{itemdecl}

\begin{itemdescr}
\pnum
\returns \tcode{end_}.
\end{itemdescr}

\rSec4[ranges.adaptors.take_view.sentinel.comp]{\tcode{take_view::\xname{sentinel}} comparisons}

\indexlibrary{\idxcode{operator==}!\idxcode{take_view::\xname{sentinel}}}
\begin{itemdecl}
friend constexpr bool operator==(const @\xname{sentinel}@& x, const CI& y)
requires EqualityComparable<iterator_t<Base>>;
\end{itemdecl}

\begin{itemdescr}
\pnum
\returns \tcode{y.count() == 0 || y.base() == x.end_}.
\end{itemdescr}

\begin{itemdecl}
friend constexpr bool operator==(const CI& x, const @\xname{sentinel}@& y)
requires EqualityComparable<iterator_t<Base>>;
\end{itemdecl}

\begin{itemdescr}
\pnum
\returns \tcode{y == x}.
\end{itemdescr}

\indexlibrary{\idxcode{operator!=}!\idxcode{take_view::\xname{iterator}}}%
\begin{itemdecl}
friend constexpr bool operator!=(const @\xname{sentinel}@& x, const CI& y)
requires EqualityComparable<iterator_t<Base>>;
\end{itemdecl}

\begin{itemdescr}
\pnum
\returns \tcode{!(x == y)}.
\end{itemdescr}

\begin{itemdecl}
friend constexpr bool operator!=(const CI& x, const @\xname{sentinel}@& y)
requires EqualityComparable<iterator_t<Base>>;
\end{itemdecl}

\begin{itemdescr}
\pnum
\returns \tcode{!(y == x)}.
\end{itemdescr}

\rSec2[ranges.adaptors.take]{\tcode{view::take}}

\pnum
The name \tcode{view::take} denotes a range adaptor
object~(\ref{ranges.adaptor.object}). Let \tcode{E} and \tcode{F} be expressions such
that type \tcode{T} is \tcode{decltype((E))}. Then the expression
\tcode{view::take(E, F)} is expression-equivalent to:

\begin{itemize}
\item \tcode{take_view\{E, F\}} if \tcode{InputRange<T>} is satisfied and if \tcode{F}
is implicitly convertible to \tcode{difference_type_t<iterator_t<T>{>}}.
\item Otherwise, \tcode{view::take(E, F)} is ill-formed.
\end{itemize}

\rSec2[ranges.adaptors.join_view]{Class template \tcode{join_view}}

\pnum
The purpose of \tcode{join_view} is to flatten a range of ranges into a
range.

\pnum
\enterexample
\begin{codeblock}
vector<string> ss{"hello", " ", "world", "!"};
join_view greeting{ss};
for (char ch : greeting)
  cout << ch; // prints: hello world!
\end{codeblock}
\exitexample

\begin{codeblock}
namespace std { namespace experimental { namespace ranges { inline namespace v1 {
  template <InputRange R>
      requires View<R> && InputRange<reference_t<iterator_t<R>>> &&
          (is_reference_v<reference_t<iterator_t<R>>> ||
          View<value_type_t<iterator_t<R>>>)
  class join_view : @\added{public}@ view_interface<join_view<R>> {
  private:
    using InnerRng = reference_t<iterator_t<R>>; // \expos
    template <bool Const>
      struct @\xname{iterator}@; // \expos
    template <bool Const>
      struct @\xname{sentinel}@; // \expos

    R base_ {}; // \expos
    all_view<InnerRng> inner_ {}; // \expos, only present when !is_reference_v<InnerRng>
  public:
    join_view() = default;
    constexpr explicit join_view(R base);

    template <InputRange O>
        requires @\changed{(is_lvalue_reference_v<O> || View<decay_t<O>{>})}{ViewableRange<O>}@ &&
            Constructible<R, all_view<O>>
      constexpr explicit join_view(O&& o);

    using iterator = @\xname{iterator}@<@\changed{false}{\textit{simple-view}<R>}@>;
    using sentinel = @\xname{sentinel}@<@\changed{false}{\textit{simple-view}<R>}@>;
    using const_iterator = @\xname{iterator}@<true>;
    using const_sentinel = @\xname{sentinel}@<true>;

    constexpr iterator begin();

    constexpr const_iterator begin() const requires InputRange<const R> &&
        is_reference_v<reference_t<iterator_t<const R>>>;

    constexpr sentinel end();

    constexpr const_sentinel end() const requires InputRange<const R> &&
        is_reference_v<reference_t<iterator_t<const R>>>;

    constexpr iterator end() requires ForwardRange<R> &&
        is_reference_v<InnerRng> && ForwardRange<InnerRng> &&
        BoundedRange<R> && BoundedRange<InnerRng>;

    constexpr const_iterator end() const requires ForwardRange<const R> &&
        is_reference_v<reference_t<iterator_t<const R>>> &&
        ForwardRange<reference_t<iterator_t<const R>>> &&
        BoundedRange<const R> && BoundedRange<reference_t<iterator_t<const R>>>;
  };

  template <InputRange R>
      requires InputRange<reference_t<iterator_t<R>>> &&
          (is_reference_v<reference_t<iterator_t<R>>> ||
          View<value_type_t<iterator_t<R>>>)
    explicit join_view(R&&) -> join_view<all_view<R>>;
}}}}
\end{codeblock}

\rSec3[ranges.adaptors.join_view.ops]{\tcode{join_view} operations}
\rSec4[ranges.adaptors.join_view.ctor]{\tcode{join_view} constructors}

\indexlibrary{\idxcode{join_view}!\idxcode{join_view}}%
\begin{itemdecl}
explicit constexpr join_view(R base);
\end{itemdecl}

\begin{itemdescr}
\pnum
\effects Initializes \tcode{base_} with \tcode{std::move(base)}.
\end{itemdescr}

\indexlibrary{\idxcode{join_view}!\idxcode{join_view}}%
\begin{itemdecl}
template <InputRange O>
  requires @\changed{(is_lvalue_reference_v<O> || View<decay_t<O>{>})}{ViewableRange<O>}@ &&
      Constructible<R, all_view<O>>
constexpr explicit join_view(O&& o);
\end{itemdecl}

\begin{itemdescr}
\pnum
\effects Initializes \tcode{base_} with \tcode{view::all(std::forward<O>(o))}.
\end{itemdescr}

\rSec4[ranges.adaptors.join_view.begin]{\tcode{join_view} range begin}

\indexlibrary{\idxcode{begin}!\idxcode{join_view}}%
\begin{itemdecl}
constexpr iterator begin();
constexpr const_iterator begin() const requires InputRange<const R> &&
  is_reference_v<reference_t<iterator_t<const R>>>;
\end{itemdecl}

\begin{itemdescr}
\pnum
\returns \tcode{\{*this, ranges::begin(base_)\}}.
\end{itemdescr}

\rSec4[ranges.adaptors.join_view.end]{\tcode{join_view} range end}

\indexlibrary{\idxcode{end}!\idxcode{join_view}}%
\begin{itemdecl}
constexpr sentinel end();
constexpr const_sentinel end() const requires InputRange<const R> &&
  is_reference_v<reference_t<iterator_t<const R>>>;
\end{itemdecl}

\begin{itemdescr}
\pnum
\effects Equivalent to \tcode{sentinel\{*this\}} and
\tcode{const_sentinel\{*this\}} for the first and second overload,
respectively.
\end{itemdescr}

\indexlibrary{\idxcode{end}!\idxcode{join_view}}%
\begin{itemdecl}
constexpr iterator end() requires ForwardRange<R> &&
  is_reference_v<InnerRng> && ForwardRange<InnerRng> &&
  BoundedRange<R> && BoundedRange<InnerRng>;
constexpr const_iterator end() const requires ForwardRange<const R> &&
  is_reference_v<reference_t<iterator_t<const R>>> &&
  ForwardRange<reference_t<iterator_t<const R>>> &&
  BoundedRange<const R> && BoundedRange<reference_t<iterator_t<const R>>>;
\end{itemdecl}

\begin{itemdescr}
\pnum
\returns \tcode{\{*this, ranges::end(base_)\}}.
\end{itemdescr}

\rSec3[ranges.adaptors.join_view.iterator]{Class template \tcode{join_view::\xname{iterator}}}

\pnum
\tcode{join_view::\xname{iterator}} is an exposition-only type.

\begin{codeblock}
namespace std { namespace experimental { namespace ranges { inline namespace v1 {
template <class R>
  template <bool Const>
  struct join_view<R>::@\xname{iterator}@ {
  private:
    using Base = conditional_t<Const, const R, R>;
    using Parent = conditional_t<Const, const join_view, join_view>;

    iterator_t<Base> outer_ {};
    iterator_t<reference_t<iterator_t<Base>>> inner_ {};
    Parent* parent_ {};

    constexpr void satisfy_();
  public:
    using iterator_category = @\seebelow@;
    using value_type = value_type_t<iterator_t<reference_t<iterator_t<Base>>>>;
    using difference_type = @\seebelow@;

    @\xname{iterator}@() = default;
    constexpr @\xname{iterator}@(Parent& parent, iterator_t<R> outer);
    constexpr @\xname{iterator}@(@\xname{iterator}@<!Const> i) requires Const &&
        ConvertibleTo<iterator_t<R>, iterator_t<Base>> &&
        ConvertibleTo<iterator_t<InnerRng>,
            iterator_t<reference_t<iterator_t<Base>>>>;

    constexpr decltype(auto) operator*() const;

    constexpr @\xname{iterator}@& operator++();
    constexpr void operator++(int);
    constexpr @\xname{iterator}@ operator++(int)
        requires is_reference_v<reference_t<iterator_t<Base>>> &&
            ForwardRange<Base> &&
            ForwardRange<reference_t<iterator_t<Base>>>;

    constexpr @\xname{iterator}@& operator--();
        requires is_reference_v<reference_t<iterator_t<Base>>> &&
            BidirectionalRange<Base> &&
            BidirectionalRange<reference_t<iterator_t<Base>>>;

    constexpr @\xname{iterator}@ operator--(int)
        requires is_reference_v<reference_t<iterator_t<Base>>> &&
            BidirectionalRange<Base> &&
            BidirectionalRange<reference_t<iterator_t<Base>>>;

    friend constexpr bool operator==(const @\xname{iterator}@& x, const @\xname{iterator}@& y)
    requires is_reference_v<reference_t<iterator_t<Base>>> &&
        EqualityComparable<iterator_t<Base>> &&
        EqualityComparable<iterator_t<reference_t<iterator_t<Base>>>>;

    friend constexpr bool operator!=(const @\xname{iterator}@& x, const @\xname{iterator}@& y)
    requires is_reference_v<reference_t<iterator_t<Base>>> &&
        EqualityComparable<iterator_t<Base>> &&
        EqualityComparable<iterator_t<reference_t<iterator_t<Base>>>>;

    friend constexpr decltype(auto) iter_move(const @\xname{iterator}@& i)
        noexcept(@\seebelow@);

    friend constexpr void iter_swap(const @\xname{iterator}@& x, const @\xname{iterator}@& y)
        noexcept(@\seebelow@);
  };
}}}}
\end{codeblock}

\pnum
\tcode{join_view<R>::iterator::iterator_category} is defined as follows:
\begin{itemize}
\item If \tcode{Base} satisfies \tcode{BidirectionalRange}, and if
\tcode{is_reference_v<reference_t<iterator_t<Base>{>}>} is \tcode{true}, and if
\tcode{reference_t<iterator_t<Base>{>}} satisfies \tcode{BidirectionalRange}, then
\tcode{iterator_category} is \tcode{ranges::bidirectional_iterator_tag}.
\item Otherwise, if \tcode{Base} satisfies \tcode{ForwardRange}, and if
\tcode{is_reference_v<reference_t<iterator_t<Base>{>}>} is \tcode{true}, and if
\tcode{reference_t<iterator_t<Base>{>}} satisfies \tcode{ForwardRange}, then
\tcode{iterator_category} is \tcode{ranges::forward_iterator_tag}.
\item Otherwise, \tcode{iterator_category} is \tcode{ranges::input_iterator_tag}.
\end{itemize}

\pnum
\tcode{join_view<R>::iterator::difference_type} is an alias for:
\begin{codeblock}
common_type_t<
difference_type_t<iterator_t<Base>>,
difference_type_t<iterator_t<reference_t<iterator_t<Base>>>>>
\end{codeblock}

\pnum
The \tcode{join_view<R>::iterator::satisfy_()} function is equivalent to:
\begin{codeblock}
for (; outer_ != ranges::end(parent_->base_); ++outer_) {
  auto&& inner = @\textit{inner-range-update}@;
  inner_ = ranges::begin(inner);
  if (inner_ != ranges::end(inner))
    return;
}
if constexpr (is_reference_v<reference_t<iterator_t<Base>>>)
  inner_ = iterator_t<reference_t<iterator_t<Base>>>{};
\end{codeblock}

where \tcode{\textit{inner-range-update}} is equivalent to:
\begin{itemize}
\item If \tcode{is_reference_v<reference_t<iterator_t<Base>{>}>} is \tcode{true},
\tcode{*outer_}.
\item Otherwise,
\begin{codeblock}
[this](auto&& x) -> decltype(auto) {
return (parent_->inner_ = view::all(x));
}(*outer_)
\end{codeblock}
\end{itemize}

\rSec4[ranges.adaptors.join_view.iterator.ops]{\tcode{join_view::\xname{iterator}} operations}
\rSec5[ranges.adaptors.join_view.iterator.ctor]{\tcode{join_view::\xname{iterator}} constructors}

\indexlibrary{\idxcode{\xname{iterator}}!\idxcode{join_view::\xname{iterator}}}
\begin{itemdecl}
constexpr @\xname{iterator}@(Parent& parent, iterator_t<R> outer)
\end{itemdecl}

\begin{itemdescr}
\pnum
\effects Initializes \tcode{outer_} with \tcode{outer} and initializes
\tcode{parent_} with \tcode{\&parent}; then calls \tcode{satisfy_()}.
\end{itemdescr}

\indexlibrary{\idxcode{\xname{iterator}}!\idxcode{join_view::\xname{iterator}}}
\begin{itemdecl}
constexpr @\xname{iterator}@(@\xname{iterator}@<!Const> i) requires Const &&
  ConvertibleTo<iterator_t<R>, iterator_t<Base>> &&
  ConvertibleTo<iterator_t<InnerRng>,
      iterator_t<reference_t<iterator_t<Base>>>>;
\end{itemdecl}

\begin{itemdescr}
\pnum
\effects Initializes \tcode{outer_} with \tcode{i.outer_}, initializes
\tcode{inner_} with \tcode{i.inner_}, and initializes \tcode{parent_} with
\tcode{i.parent_}.
\end{itemdescr}

\rSec5[ranges.adaptors.join_view.iterator.star]{\tcode{join_view::iterator::operator*}}

\indexlibrary{\idxcode{operator*}!\idxcode{join_view::\xname{iterator}}}
\begin{itemdecl}
constexpr decltype(auto) operator*() const;
\end{itemdecl}

\begin{itemdescr}
\pnum
\returns \tcode{*inner_}.
\end{itemdescr}

\rSec5[ranges.adaptors.join_view.iterator.inc]{\tcode{join_view::iterator::operator++}}

\indexlibrary{\idxcode{operator++}!\idxcode{join_view::\xname{iterator}}}
\begin{itemdecl}
constexpr @\xname{iterator}@& operator++();
\end{itemdecl}

\begin{itemdescr}
\pnum
\effects Equivalent to:
\begin{codeblock}
if (++inner_ == ranges::end(@\textit{inner-range}@)) {
  ++outer_;
  satisfy_();
}
return *this;
\end{codeblock}

where \tcode{\textit{inner-range}} is equivalent to:
\begin{itemize}
\item If \tcode{is_reference_v<reference_t<iterator_t<Base>{>}>} is \tcode{true}, \tcode{*outer_}.
\item Otherwise, \tcode{parent_->inner_}.
\end{itemize}
\end{itemdescr}

\indexlibrary{\idxcode{operator++}!\idxcode{join_view::\xname{iterator}}}
\begin{itemdecl}
constexpr void operator++(int);
\end{itemdecl}

\begin{itemdescr}
\pnum
\effects Equivalent to:
\begin{codeblock}
(void)++*this;
\end{codeblock}
\end{itemdescr}

\indexlibrary{\idxcode{operator++}!\idxcode{join_view::\xname{iterator}}}
\begin{itemdecl}
constexpr @\xname{iterator}@ operator++(int)
  requires is_reference_v<reference_t<iterator_t<Base>>> &&
      ForwardRange<Base> &&
      ForwardRange<reference_t<iterator_t<Base>>>;
\end{itemdecl}

\begin{itemdescr}
\pnum
\effects Equivalent to:
\begin{codeblock}
auto tmp = *this;
++*this;
return tmp;
\end{codeblock}
\end{itemdescr}

\rSec5[ranges.adaptors.join_view.iterator.dec]{\tcode{join_view::iterator::operator\dcr}}

\indexlibrary{\idxcode{operator\dcr}!\idxcode{join_view::\xname{iterator}}}
\begin{itemdecl}
constexpr @\xname{iterator}@& operator--();
  requires is_reference_v<reference_t<iterator_t<Base>>> &&
      BidirectionalRange<Base> &&
      BidirectionalRange<reference_t<iterator_t<Base>>>;
\end{itemdecl}

\begin{itemdescr}
\pnum
\effects Equivalent to:
\begin{codeblock}
if (outer_ == ranges::end(parent_->base_))
  inner_ = ranges::end(*--outer_);
while (inner_ == ranges::begin(*outer_))
  inner_ = ranges::end(*--outer_);
--inner_;
return *this;
\end{codeblock}
\end{itemdescr}

\indexlibrary{\idxcode{operator\dcr}!\idxcode{join_view::\xname{iterator}}}
\begin{itemdecl}
constexpr @\xname{iterator}@ operator--(int)
  requires is_reference_v<reference_t<iterator_t<Base>>> &&
      BidirectionalRange<Base> &&
      BidirectionalRange<reference_t<iterator_t<Base>>>;
\end{itemdecl}

\begin{itemdescr}
\pnum
\effects Equivalent to:
\begin{codeblock}
auto tmp = *this;
--*this;
return tmp;
\end{codeblock}
\end{itemdescr}

\rSec4[ranges.adaptors.join_view.iterator.comp]{\tcode{join_view::\xname{iterator}} comparisons}

\indexlibrary{\idxcode{operator==}!\idxcode{join_view::\xname{iterator}}}
\begin{itemdecl}
friend constexpr bool operator==(const @\xname{iterator}@& x, const @\xname{iterator}@& y)
  requires is_reference_v<reference_t<iterator_t<Base>>> &&
      EqualityComparable<iterator_t<Base>> &&
      EqualityComparable<iterator_t<reference_t<iterator_t<Base>>>>;
\end{itemdecl}

\begin{itemdescr}
\pnum
\returns \tcode{x.outer_ == y.outer_ \&\& x.inner_ == y.inner_}.
\end{itemdescr}

\indexlibrary{\idxcode{operator!=}!\idxcode{join_view::\xname{iterator}}}
\begin{itemdecl}
friend constexpr bool operator!=(const @\xname{iterator}@& x, const @\xname{iterator}@& y)
  requires is_reference_v<reference_t<iterator_t<Base>>> &&
      EqualityComparable<iterator_t<Base>> &&
      EqualityComparable<iterator_t<reference_t<iterator_t<Base>>>>;
\end{itemdecl}

\begin{itemdescr}
\pnum
\returns \tcode{!(x == y)}.
\end{itemdescr}

\rSec4[ranges.adaptors.join_view.iterator.nonmember]{\tcode{join_view::\xname{iterator}} non-member functions}

\indexlibrary{\idxcode{iter_move}!\idxcode{join_view::\xname{iterator}}}
\begin{itemdecl}
friend constexpr decltype(auto) iter_move(const @\xname{iterator}@& i)
  noexcept(@\seebelow@);
\end{itemdecl}

\begin{itemdescr}
\pnum
\returns \tcode{ranges::iter_move(i.inner_)}.

\pnum
\remarks The expression in the \tcode{noexcept} clause is equivalent to:
\begin{codeblock}
noexcept(ranges::iter_move(i.inner_))
\end{codeblock}
\end{itemdescr}

\indexlibrary{\idxcode{iter_swap}!\idxcode{join_view::\xname{iterator}}}
\begin{itemdecl}
friend constexpr void iter_swap(const @\xname{iterator}@& x, const @\xname{iterator}@& y)
  noexcept(@\seebelow@);
\end{itemdecl}

\begin{itemdescr}
\pnum
\returns \tcode{ranges::iter_swap(x.inner_, y.inner_)}.

\pnum
\remarks The expression in the \tcode{noexcept} clause is equivalent to:
\begin{codeblock}
noexcept(ranges::iter_swap(x.inner_, y.inner_))
\end{codeblock}
\end{itemdescr}

\rSec3[ranges.adaptors.join_view.sentinel]{Class template \tcode{join_view::\xname{sentinel}}}

\pnum
\tcode{join_view::\xname{sentinel}} is an exposition-only type.

\begin{codeblock}
namespace std { namespace experimental { namespace ranges { inline namespace v1 {
  template <class R>
  template <bool Const>
  struct join_view<R>::@\xname{sentinel}@ {
  private:
    using Base = conditional_t<Const, const R, R>;
    using Parent = conditional_t<Const, const join_view, join_view>;
    sentinel_t<Base> end_ {};
  public:
    @\xname{sentinel}@() = default;

    constexpr explicit @\xname{sentinel}@(Parent& parent);
    constexpr @\xname{sentinel}@(@\xname{sentinel}@<!Const> s) requires Const &&
        ConvertibleTo<sentinel_t<R>, sentinel_t<Base>>;

    friend constexpr bool operator==(const @\xname{iterator}@<Const>& x, const @\xname{sentinel}@& y);
    friend constexpr bool operator==(const @\xname{sentinel}@& x, const @\xname{iterator}@<Const>& y);
    friend constexpr bool operator!=(const @\xname{iterator}@<Const>& x, const @\xname{sentinel}@& y);
    friend constexpr bool operator!=(const @\xname{sentinel}@& x, const @\xname{iterator}@<Const>& y);
  };
}}}}
\end{codeblock}

\rSec4[ranges.adaptors.join_view.sentinel.ops]{\tcode{join_view::\xname{sentinel}} operations}
\rSec5[ranges.adaptors.join_view.sentinel.ctor]{\tcode{join_view::\xname{sentinel}} constructors}

\indexlibrary{\idxcode{\xname{sentinel}}!\idxcode{join_view::\xname{sentinel}}}
\begin{itemdecl}
constexpr explicit @\xname{sentinel}@(Parent& parent);
\end{itemdecl}

\begin{itemdescr}
\pnum
\effects Initializes \tcode{end_} with \tcode{ranges::end(parent.base_)}.
\end{itemdescr}

\indexlibrary{\idxcode{\xname{sentinel}}!\idxcode{join_view::\xname{sentinel}}}
\begin{itemdecl}
constexpr @\xname{sentinel}@(@\xname{sentinel}@<!Const> s) requires Const &&
  ConvertibleTo<sentinel_t<R>, sentinel_t<Base>>;
\end{itemdecl}

\begin{itemdescr}
\pnum
\effects Initializes \tcode{end_} with \tcode{s.end_}.
\end{itemdescr}

\rSec4[ranges.adaptors.join_view.sentinel.comp]{\tcode{join_view::\xname{sentinel}} comparisons}

\indexlibrary{\idxcode{operator==}!\idxcode{join_view::\xname{sentinel}}}
\begin{itemdecl}
friend constexpr bool operator==(const @\xname{iterator}@<Const>& x, const @\xname{sentinel}@& y);
\end{itemdecl}

\begin{itemdescr}
\pnum
\returns \tcode{x.outer_ == y.end_}.
\end{itemdescr}

\indexlibrary{\idxcode{operator==}!\idxcode{join_view::\xname{sentinel}}}
\begin{itemdecl}
friend constexpr bool operator==(const @\xname{sentinel}@& x, const @\xname{iterator}@<Const>& y);
\end{itemdecl}

\begin{itemdescr}
\pnum
\returns \tcode{y == x}.
\end{itemdescr}

\indexlibrary{\idxcode{operator!=}!\idxcode{join_view::\xname{sentinel}}}
\begin{itemdecl}
friend constexpr bool operator!=(const @\xname{iterator}@<Const>& x, const @\xname{sentinel}@& y);
\end{itemdecl}

\begin{itemdescr}
\pnum
\returns \tcode{!(x == y)}.
\end{itemdescr}

\indexlibrary{\idxcode{operator!=}!\idxcode{join_view::\xname{sentinel}}}
\begin{itemdecl}
friend constexpr bool operator!=(const @\xname{sentinel}@& x, const @\xname{iterator}@<Const>& y);
\end{itemdecl}

\begin{itemdescr}
\pnum
\returns \tcode{!(y == x)}.
\end{itemdescr}

\rSec2[ranges.adaptors.join]{\tcode{view::join}}

\pnum
The name \tcode{view::join} denotes a range adaptor
object~(\ref{ranges.adaptor.object}). Let \tcode{E} be an expression such
that type \tcode{T} is \tcode{decltype((E))}. Then the expression
\tcode{view::join(E)} is expression-equivalent to:

\begin{itemize}
\item \tcode{join_view\{E\}} if the following is satisfied:
\begin{codeblock}
InputRange<T> &&
InputRange<reference_t<iterator_t<T>>> &&
(is_reference_v<reference_t<iterator_t<T>>> ||
View<value_type_t<iterator_t<T>>)
\end{codeblock}
\item Otherwise, \tcode{view::join(E)} is ill-formed.
\end{itemize}

\rSec2[ranges.adaptors.empty_view]{Class template \tcode{empty_view}}

\pnum
The purpose of \tcode{empty_view} is to produce an empty range of elements of
a particular type.

\pnum
\enterexample
\begin{codeblock}
empty_view<int> e;
static_assert(ranges::empty(e));
static_assert(0 == e.size());
\end{codeblock}
\exitexample

\begin{codeblock}
namespace std { namespace experimental { namespace ranges { inline namespace v1 {
  template <class T>
    requires @\changed{requires \{ *(T*)nullptr; \}}{is_object_v<T>}@
  class empty_view : @\added{public}@ view_interface<empty_view<T>> {
  public:
    @\removed{empty_view() = default;}@

    using iterator = T*;
    using const_iterator = T*;
    using sentinel = T*;
    using const_sentinel = T*;

    constexpr static T* begin() noexcept;
    constexpr static T* end() noexcept;
    constexpr static ptrdiff_t size() noexcept;
    constexpr static T* data() noexcept;
  };
}}}}
\end{codeblock}

\rSec3[ranges.adaptors.empty_view.ops]{\tcode{empty_view} operations}
\rSec4[ranges.adaptors.empty_view.begin]{\tcode{empty_view} begin}

\indexlibrary{\idxcode{begin}!\idxcode{empty_view}}%
\begin{itemdecl}
constexpr static T* begin() noexcept;
\end{itemdecl}

\begin{itemdescr}
\pnum
\returns \tcode{nullptr}.
\end{itemdescr}

\rSec4[ranges.adaptors.empty_view.end]{\tcode{empty_view} end}

\indexlibrary{\idxcode{end}!\idxcode{empty_view}}%
\begin{itemdecl}
constexpr static T* end() noexcept;
\end{itemdecl}

\begin{itemdescr}
\pnum
\returns \tcode{nullptr}.
\end{itemdescr}

\rSec4[ranges.adaptors.empty_view.size]{\tcode{empty_view} size}

\indexlibrary{\idxcode{size}!\idxcode{empty_view}}%
\begin{itemdecl}
constexpr static ptrdiff_t size() noexcept;
\end{itemdecl}

\begin{itemdescr}
\pnum
\returns \tcode{0}.
\end{itemdescr}

\rSec4[ranges.adaptors.empty_view.data]{\tcode{empty_view} data}

\indexlibrary{\idxcode{data}!\idxcode{empty_view}}%
\begin{itemdecl}
constexpr static T* data() noexcept;
\end{itemdecl}

\begin{itemdescr}
\pnum
\returns \tcode{nullptr}.
\end{itemdescr}

\rSec2[ranges.adaptors.single_view]{Class template \tcode{single_view}}

\pnum
The purpose of \tcode{single_view} is to produce a range that contains exactly
one element of a specified value.

\pnum
\enterexample
\begin{codeblock}
single_view s{4};
for (int i : s)
  cout << i; // prints 4
\end{codeblock}
\exitexample

\begin{codeblock}
namespace std { namespace experimental { namespace ranges { inline namespace v1 {
  template <CopyConstructible T>
  class single_view : @\added{public}@ view_interface<single_view<T>> {
  private:
    @\textit{semiregular}@<T> value_; // \expos
  public:
    single_view() = default;
    constexpr explicit single_view(const T& t);
    constexpr explicit single_view(T&& t);
    template <class... Args>
      requires Constructible<T, Args...>
    constexpr single_view(in_place_t, Args&&... args);

    using iterator = const T*;
    using const_iterator = const T*;
    using sentinel = const T*;
    using const_sentinel = const T*;

    constexpr const T* begin() const noexcept;
    constexpr const T* end() const noexcept;
    constexpr static ptrdiff_t size() noexcept;
    constexpr const T* data() const noexcept;
  };

  template <class T>
  requires CopyConstructible<decay_t<T>>
  explicit single_view(T&&) -> single_view<decay_t<T>>;
}}}}
\end{codeblock}

\rSec3[ranges.adaptors.single_view.ops]{\tcode{single_view} operations}
\rSec4[ranges.adaptors.single_view.ctor]{\tcode{single_view} constructors}

\indexlibrary{\idxcode{single_view}!\idxcode{single_view}}%
\begin{itemdecl}
constexpr explicit single_view(const T& t);
\end{itemdecl}

\begin{itemdescr}
\pnum
\effects Initializes \tcode{value_} with \tcode{t}.
\end{itemdescr}

\indexlibrary{\idxcode{single_view}!\idxcode{single_view}}%
\begin{itemdecl}
constexpr explicit single_view(T&& t);
\end{itemdecl}

\begin{itemdescr}
\pnum
\effects Initializes \tcode{value_} with \tcode{std::move(t)}.
\end{itemdescr}

\indexlibrary{\idxcode{single_view}!\idxcode{single_view}}%
\begin{itemdecl}
template <class... Args>
constexpr single_view(in_place_t, Args&&... args);
\end{itemdecl}

\begin{itemdescr}
\pnum
\effects Initializes \tcode{value_} as if by
\tcode{value_\{in_place, std::forward<Args>(args)...\}}.
\end{itemdescr}

\rSec4[ranges.adaptors.single_view.begin]{\tcode{single_view} begin}

\indexlibrary{\idxcode{begin}!\idxcode{empty_view}}%
\begin{itemdecl}
constexpr const T* begin() const noexcept;
\end{itemdecl}

\begin{itemdescr}
\pnum
\requires \tcode{bool(value_)}

\pnum
\returns \tcode{value_.operator->()}.
\end{itemdescr}

\rSec4[ranges.adaptors.single_view.end]{\tcode{single_view} end}

\indexlibrary{\idxcode{end}!\idxcode{empty_view}}%
\begin{itemdecl}
constexpr const T* end() const noexcept;
\end{itemdecl}

\begin{itemdescr}
\pnum
\requires \tcode{bool(value_)}

\pnum
\returns \tcode{value_.operator->() + 1}.
\end{itemdescr}

\rSec4[ranges.adaptors.single_view.size]{\tcode{single_view} size}

\indexlibrary{\idxcode{size}!\idxcode{empty_view}}%
\begin{itemdecl}
constexpr static ptrdiff_t size() noexcept;
\end{itemdecl}

\begin{itemdescr}
\pnum
\requires \tcode{bool(value_)}

\pnum
\returns \tcode{1}.
\end{itemdescr}

\rSec4[ranges.adaptors.single_view.data]{\tcode{single_view} data}

\indexlibrary{\idxcode{data}!\idxcode{single_view}}%
\begin{itemdecl}
constexpr const T* data() const noexcept;
\end{itemdecl}

\begin{itemdescr}
\pnum
\requires \tcode{bool(value_)}

\pnum
\returns \tcode{begin()}.
\end{itemdescr}

\rSec2[ranges.adaptors.single]{\tcode{view::single}}

\pnum
The name \tcode{view::single} denotes a customization point object~(\cxxref{???}).
Let \tcode{E} be an expression such that its un-\cv\ qualified type
is \tcode{I}. Then the expression \tcode{view::single(E)} is expression-equivalent
to:

\begin{itemize}
\item \tcode{single_view\{E\}} if \tcode{CopyConstructible<I>} is satisfied.
\item Otherwise, \tcode{view::single(E)} is ill-formed.
\end{itemize}

\rSec2[ranges.adaptors.split_view]{Class template \tcode{split_view}}

\pnum
The \tcode{split_view} takes a range and a delimiter, and splits the range into
subranges on the delimiter. The delimiter can be a single element or a range of
elements.

\pnum
\enterexample
\begin{codeblock}
string str{"the quick brown fox"};
split_view sentence{str, ' '};
for (auto word : sentence) {
  for (char ch : word)
    cout << ch;
  cout << " *";
}
// The above prints: the *quick *brown *fox *
\end{codeblock}
\exitexample

\begin{codeblock}
namespace std { namespace experimental { namespace ranges { inline namespace v1 {
  // \expos
  template <class R>
  concept bool @\textit{tiny-range}@ =
    SizedRange<R> && requires {
      requires remove_reference_t<R>::size() <= 1;
    };

  template <InputRange Rng, ForwardRange Pattern>
    requires View<Rng> && View<Pattern> &&
        IndirectlyComparable<iterator_t<Rng>, iterator_t<Pattern>> &&
        (ForwardRange<Rng> || @\textit{tiny-range}@<Pattern>)
  @\changed{struct}{class}@ split_view {
  private:
    Rng base_ {}; // expos
    Pattern pattern_ {}; // expos
    iterator_t<Rng> current_ {}; // expos, only present if !ForwardRange<Rng>
    template <bool Const> struct @\xname{outer_iterator}@; // expos
    template <bool Const> struct @\xname{outer_sentinel}@; // expos
    template <bool Const> struct @\xname{inner_iterator}@; // expos
    template <bool Const> struct @\xname{inner_sentinel}@; // expos
  public:
    split_view() = default;
    constexpr split_view(Rng base, Pattern pattern);

    template <InputRange O, ForwardRange P>
      requires @\changed{(is_lvalue_reference_v<O> || View<decay_t<O>{>})}{ViewableRange<O>}@ &&
          @\changed{(is_lvalue_reference_v<P> || View<decay_t<P>{>})}{ViewableRange<P>}@ &&
          Constructible<Rng, all_view<O>> &&
          Constructible<Pattern, all_view<P>>
    constexpr split_view(O&& o, P&& p);

    template <InputRange O>
      requires @\changed{(is_lvalue_reference_v<O> || View<decay_t<O>{>})}{ViewableRange<O>}@ &&
          Constructible<Rng, all_view<O>> &&
          Constructible<Pattern, single_view<value_type_t<iterator_t<O>>>>
    constexpr split_view(O&& o, value_type_t<iterator_t<O>> e);

    using iterator = @\xname{outer_iterator}@<@\changed{false}{\textit{simple-view}<R>}@>;
    using sentinel = @\xname{outer_sentinel}@<@\changed{false}{\textit{simple-view}<R>}@>;
    using const_iterator = @\xname{outer_iterator}@<true>;
    using const_sentinel = @\xname{outer_sentinel}@<true>;

    constexpr iterator begin();
    constexpr iterator begin() requires ForwardRange<Rng>;
    constexpr const_iterator begin() const requires ForwardRange<@\added{const}@ Rng>;

    constexpr sentinel end()
    constexpr const_sentinel end() const requires ForwardRange<@\added{const}@ Rng>;

    constexpr iterator end()
      requires ForwardRange<Rng> && BoundedRange<Rng>;
    constexpr const_iterator end() const
      requires ForwardRange<@\added{const}@ Rng> && BoundedRange<@\added{const}@ Rng>;
  };

  template <InputRange O, ForwardRange P>
    requires @\changed{(is_lvalue_reference_v<O> || View<decay_t<O>{>})}{ViewableRange<O>}@ &&
      @\changed{(is_lvalue_reference_v<P> || View<decay_t<P>{>})}{ViewableRange<P>}@ &&
      IndirectlyComparable<iterator_t<O>, iterator_t<P>> &&
      (ForwardRange<O> || _TinyRange<P>)
  split_view(O&&, P&&) -> split_view<all_view<O>, all_view<P>>;

  template <InputRange O>
    requires @\changed{(is_lvalue_reference_v<O> || View<decay_t<O>{>})}{ViewableRange<O>}@ &&
      IndirectlyComparable<iterator_t<Rng>, const value_type_t<iterator_t<Rng>>*> &&
      CopyConstructible<value_type_t<iterator_t<O>>>
  split_view(O&&, value_type_t<iterator_t<O>>)
    -> split_view<all_view<O>, single_view<value_type_t<iterator_t<O>>>>;
}}}}
\end{codeblock}

\rSec3[ranges.adaptors.split_view.ops]{\tcode{split_view} operations}
\rSec4[ranges.adaptors.split_view.ctor]{\tcode{split_view} constructors}

\indexlibrary{\idxcode{split_view}!\idxcode{split_view}}%
\begin{itemdecl}
constexpr split_view(Rng base, Pattern pattern);
\end{itemdecl}

\begin{itemdescr}
\pnum
\effects Initializes \tcode{base_} with \tcode{std::move(base)} and initializes
\tcode{pattern_} with \tcode{std::move(pattern)}.
\end{itemdescr}

\indexlibrary{\idxcode{split_view}!\idxcode{split_view}}%
\begin{itemdecl}
template <InputRange O, ForwardRange P>
requires @\changed{(is_lvalue_reference_v<O> || View<decay_t<O>{>})}{ViewableRange<O>}@ &&
    @\changed{(is_lvalue_reference_v<P> || View<decay_t<P>{>})}{ViewableRange<P>}@ &&
    Constructible<Rng, all_view<O>> &&
    Constructible<Pattern, all_view<P>>
constexpr split_view(O&& o, P&& p);
\end{itemdecl}

\begin{itemdescr}
\pnum
\effects Delegates to
\tcode{split_view\{view::all(std::forward<O>(o)), view::all(std::forward<P>(p))\}}.
\end{itemdescr}

\indexlibrary{\idxcode{split_view}!\idxcode{split_view}}%
\begin{itemdecl}
template <InputRange O>
requires @\changed{(is_lvalue_reference_v<O> || View<decay_t<O>{>})}{ViewableRange<O>}@ &&
    Constructible<Rng, all_view<O>> &&
    Constructible<Pattern, single_view<value_type_t<iterator_t<O>>>>
constexpr split_view(O&& o, value_type_t<iterator_t<O>> e);
\end{itemdecl}

\begin{itemdescr}
\pnum
\effects Delegates to
\tcode{split_view\{view::all(std::forward<O>(o)), single_view\{std::move(e)\}\}}.
\end{itemdescr}

\rSec4[ranges.adaptors.split_view.begin]{\tcode{split_view} range begin}

\indexlibrary{\idxcode{begin}!\idxcode{split_view}}%
\begin{itemdecl}
constexpr iterator begin();
\end{itemdecl}

\begin{itemdescr}
\pnum
\effects Equivalent to:
\begin{codeblock}
current_ = ranges::begin(base_);
return iterator{*this};
\end{codeblock}
\end{itemdescr}

\indexlibrary{\idxcode{begin}!\idxcode{split_view}}%
\begin{itemdecl}
constexpr iterator begin() requires ForwardRange<Rng>;
constexpr const_iterator begin() const requires ForwardRange<Rng>;
\end{itemdecl}

\begin{itemdescr}
\pnum
\returns \tcode{\{*this, ranges::begin(base_)\}}.
\end{itemdescr}

\rSec4[ranges.adaptors.split_view.end]{\tcode{split_view} range end}

\indexlibrary{\idxcode{end}!\idxcode{split_view}}%
\begin{itemdecl}
constexpr sentinel end()
constexpr const_sentinel end() const requires ForwardRange<Rng>;
\end{itemdecl}

\begin{itemdescr}
\pnum
\effects Equivalent to \tcode{sentinel\{*this\}} and
\tcode{const_sentinel\{*this\}} for the first and second overloads,
respectively.
\end{itemdescr}

\indexlibrary{\idxcode{end}!\idxcode{split_view}}%
\begin{itemdecl}
constexpr iterator end()
requires ForwardRange<Rng> && BoundedRange<Rng>;
constexpr const_iterator end() const
requires ForwardRange<Rng> && BoundedRange<Rng>;
\end{itemdecl}

\begin{itemdescr}
\pnum
\returns \tcode{\{*this, ranges::end(base_)\}}.
\end{itemdescr}

\rSec3[ranges.adaptors.split_view.outer_iterator]{Class template \tcode{split_view::\xname{outer_iterator}}}

\pnum
\enternote \tcode{split_view::\xname{outer_iterator}} is an exposition-only type.\exitnote

\begin{codeblock}
namespace std { namespace experimental { namespace ranges { inline namespace v1 {
  template <class Rng, class Pattern>
  template <bool Const>
  struct split_view<Rng, Pattern>::@\xname{outer_iterator}@ {
  private:
    using Base = conditional_t<Const, const Rng, Rng>;
    using Parent = conditional_t<Const, const split_view, split_view>;
    iterator_t<Base> current_ {}; // Only present if ForwardRange<Rng> is satisfied
    Parent* parent_ = nullptr;
  public:
    using iterator_category = @\seebelow@;
    using difference_type = difference_type_t<iterator_t<Base>>;
    struct value_type;

    @\xname{outer_iterator}@() = default;
    constexpr explicit @\xname{outer_iterator}@(Parent& parent);
    constexpr @\xname{outer_iterator}@(Parent& parent, iterator_t<Base> current)
      requires ForwardRange<Base>;
    constexpr @\xname{outer_iterator}@(@\xname{outer_iterator}@<!Const> i) requires Const &&
      ConvertibleTo<iterator_t<Rng>, iterator_t<Base>>;

    constexpr value_type operator*() const;

    constexpr @\xname{outer_iterator}@& operator++();
    constexpr void operator++(int);
    constexpr @\xname{outer_iterator}@ operator++(int) requires ForwardRange<Base>;

    friend constexpr bool operator==(const @\xname{outer_iterator}@& x, const @\xname{outer_iterator}@& y)
      requires ForwardRange<Base>;
    friend constexpr bool operator!=(const @\xname{outer_iterator}@& x, const @\xname{outer_iterator}@& y)
      requires ForwardRange<Base>;
  };
}}}}
\end{codeblock}

\pnum
\tcode{split_view<Rng, Pattern>::\xname{outer_iterator}::iterator_category} is defines
as follows:
\begin{itemize}
\item If \tcode{\xname{outer_iterator}::Base} satisfies \tcode{ForwardRange}, then
\tcode{iterator_category} is \tcode{ranges::forward_iterator_tag}.
\item Otherwise, \tcode{iterator_category} is \tcode{ranges::input_iterator_tag}.
\end{itemize}

\rSec3[ranges.adaptors.split_view.outer_iterator.ops]{\tcode{split_view::\xname{outer_iterator}} operations}
\rSec4[ranges.adaptors.split_view.outer_iterator.ctor]{\tcode{split_view::\xname{outer_iterator}} constructors}

\indexlibrary{\idxcode{\xname{outer_iterator}}!\idxcode{split_view::\xname{outer_iterator}}}%
\begin{itemdecl}
constexpr explicit @\xname{outer_iterator}@(Parent& parent);
\end{itemdecl}

\begin{itemdescr}
\pnum
\effects Initializes \tcode{parent_} with \tcode{\&parent}.
\end{itemdescr}

\indexlibrary{\idxcode{\xname{outer_iterator}}!\idxcode{split_view::\xname{outer_iterator}}}%
\begin{itemdecl}
constexpr @\xname{outer_iterator}@(Parent& parent, iterator_t<Base> current)
requires ForwardRange<Base>;
\end{itemdecl}

\begin{itemdescr}
\pnum
\effects Initializes \tcode{parent_} with \tcode{\&parent} and \tcode{current_}
with \tcode{current}.
\end{itemdescr}

\indexlibrary{\idxcode{\xname{outer_iterator}}!\idxcode{split_view::\xname{outer_iterator}}}%
\begin{itemdecl}
constexpr @\xname{outer_iterator}@(@\xname{outer_iterator}@<!Const> i) requires Const &&
ConvertibleTo<iterator_t<Rng>, iterator_t<Base>>;
\end{itemdecl}

\begin{itemdescr}
\pnum
\effects Initializes \tcode{parent_} with \tcode{i.parent_} and \tcode{current_}
with \tcode{i.current_}.
\end{itemdescr}

\rSec4[ranges.adaptors.split_view.outer_iterator.star]{\tcode{split_view::\xname{outer_iterator}::operator*}}

\indexlibrary{\idxcode{operator*}!\idxcode{split_view::\xname{outer_iterator}}}%
\begin{itemdecl}
constexpr value_type operator*() const;
\end{itemdecl}

\begin{itemdescr}
\pnum
\returns \tcode{value_type\{*this\}}.
\end{itemdescr}

\rSec4[ranges.adaptors.split_view.outer_iterator.inc]{\tcode{split_view::\xname{outer_iterator}::operator++}}

\indexlibrary{\idxcode{operator++}!\idxcode{split_view::\xname{outer_iterator}}}%
\begin{itemdecl}
constexpr @\xname{outer_iterator}@& operator++();
\end{itemdecl}

\begin{itemdescr}
\pnum
\effects Equivalent to:
\begin{codeblock}
auto const end = ranges::end(parent_->base_);
if (@\textit{current}@ == end) return *this;
auto const [pbegin, pend] = @\changed{iterator_range}{subrange}@{parent_->pattern_};
do {
  auto [b,p] = mismatch(@\textit{current}@, end, pbegin, pend);
  if (p != pend) continue; // The pattern didn't match
  @\textit{current}@ = @\textit{bump}@(b, pbegin, pend, end); // skip the pattern
  break;
} while (++@\textit{current}@ != end);
return *this;
\end{codeblock}
\end{itemdescr}

Where \tcode{\textit{current}} is equivalent to:
\begin{itemize}
\item If \tcode{Rng} satisfies \tcode{ForwardRange}, \tcode{current_}.
\item Otherwise, \tcode{parent_->current_}.
\end{itemize}

and \tcode{\textit{bump}(b, x, y, e)} is equivalent to:
\begin{itemize}
\item If \tcode{Rng} satisfies \tcode{ForwardRange}, \tcode{next(b, (int)(x == y), e)}.
\item Otherwise, \tcode{b}.
\end{itemize}

\indexlibrary{\idxcode{operator++}!\idxcode{split_view::\xname{outer_iterator}}}%
\begin{itemdecl}
constexpr void operator++(int);
\end{itemdecl}

\begin{itemdescr}
\pnum
\effects Equivalent to \tcode{(void)++*this}.
\end{itemdescr}

\indexlibrary{\idxcode{operator++}!\idxcode{split_view::\xname{outer_iterator}}}%
\begin{itemdecl}
constexpr @\xname{outer_iterator}@ operator++(int) requires ForwardRange<Base>;
\end{itemdecl}

\begin{itemdescr}
\pnum
\effects Equivalent to:
\begin{codeblock}
auto tmp = *this;
++*this;
return tmp;
\end{codeblock}
\end{itemdescr}

\rSec4[ranges.adaptors.split_view.outer_iterator.nonmember]{\tcode{split_view::\xname{outer_iterator}} non-member functions}

\indexlibrary{\idxcode{operator==}!\idxcode{split_view::\xname{outer_iterator}}}%
\begin{itemdecl}
friend constexpr bool operator==(const @\xname{outer_iterator}@& x, const @\xname{outer_iterator}@& y)
requires ForwardRange<Base>;
\end{itemdecl}

\begin{itemdescr}
\pnum
\effects Equivalent to:
\begin{codeblock}
return x.current_ == y.current_;
\end{codeblock}
\end{itemdescr}

\indexlibrary{\idxcode{operator!=}!\idxcode{split_view::\xname{outer_iterator}}}%
\begin{itemdecl}
friend constexpr bool operator!=(const @\xname{outer_iterator}@& x, const @\xname{outer_iterator}@& y)
requires ForwardRange<Base>;
\end{itemdecl}

\begin{itemdescr}
\pnum
\effects Equivalent to:
\begin{codeblock}
return !(x == y);
\end{codeblock}
\end{itemdescr}

\rSec3[ranges.adaptors.split_view.outer_sentinel]{Class template \tcode{split_view::\xname{outer_sentinel}}}

\pnum
\enternote \tcode{split_view::\xname{outer_sentinel}} is an exposition-only type.\exitnote

\begin{codeblock}
namespace std { namespace experimental { namespace ranges { inline namespace v1 {
  template <class Rng, class Pattern>
  template <bool Const>
  struct split_view<Rng, Pattern>::@\xname{outer_sentinel}@ {
  private:
    using Base = conditional_t<Const, const Rng, Rng>;
    using Parent = conditional_t<Const, const split_view, split_view>;
    sentinel_t<Base> end_;
  public:
    @\xname{outer_sentinel}@() = default;
    constexpr explicit @\xname{outer_sentinel}@(Parent& parent);

    friend constexpr bool operator==(const @\xname{outer_iterator}@<Const>& x, const @\xname{outer_sentinel}@& y);
    friend constexpr bool operator==(const @\xname{outer_sentinel}@& x, const @\xname{outer_iterator}@<Const>& y);
    friend constexpr bool operator!=(const @\xname{outer_iterator}@<Const>& x, const @\xname{outer_sentinel}@& y);
    friend constexpr bool operator!=(const @\xname{outer_sentinel}@& x, const @\xname{outer_iterator}@<Const>& y);
  };
}}}}
\end{codeblock}

\rSec4[ranges.adaptors.split_view.outer_sentinel.ctor]{\tcode{split_view::\xname{outer_sentinel}} constructors}

\indexlibrary{\idxcode{\xname{outer_sentinel}}!\idxcode{split_view::\xname{outer_sentinel}}}%
\begin{itemdecl}
constexpr explicit @\xname{outer_sentinel}@(Parent& parent);
\end{itemdecl}

\begin{itemdescr}
\pnum
\effects Initializes \tcode{end_} with \tcode{ranges::end(parent.base_)}.
\end{itemdescr}

\rSec4[ranges.adaptors.split_view.outer_sentinel.nonmember]{\tcode{split_view::\xname{outer_sentinel}} non-member functions}

\indexlibrary{\idxcode{operator==}!\idxcode{split_view::\xname{outer_sentinel}}}%
\begin{itemdecl}
friend constexpr bool operator==(const @\xname{outer_iterator}@<Const>& x, const @\xname{outer_sentinel}@& y);
\end{itemdecl}

\begin{itemdescr}
\pnum
\effects Equivalent to:
\begin{codeblock}
return @\textit{current}@(x) == y.end_;
\end{codeblock}

Where \tcode{\textit{current}(x)} is equivalent to:
\begin{itemize}
\item If \tcode{Rng} satisfies \tcode{ForwardRange}, \tcode{x.current_}.
\item Otherwise, \tcode{x.parent_->current_}.
\end{itemize}
\end{itemdescr}

\indexlibrary{\idxcode{operator==}!\idxcode{split_view::\xname{outer_sentinel}}}%
\begin{itemdecl}
friend constexpr bool operator==(const @\xname{outer_sentinel}@& x, const @\xname{outer_iterator}@<Const>& y);
\end{itemdecl}

\begin{itemdescr}
\pnum
\effects Equivalent to:
\begin{codeblock}
return y == x;
\end{codeblock}
\end{itemdescr}

\indexlibrary{\idxcode{operator!=}!\idxcode{split_view::\xname{outer_sentinel}}}%
\begin{itemdecl}
friend constexpr bool operator!=(const @\xname{outer_iterator}@<Const>& x, const @\xname{outer_sentinel}@& y);
\end{itemdecl}

\begin{itemdescr}
\pnum
\effects Equivalent to:
\begin{codeblock}
return !(x == y);
\end{codeblock}
\end{itemdescr}

\indexlibrary{\idxcode{operator!=}!\idxcode{split_view::\xname{outer_sentinel}}}%
\begin{itemdecl}
friend constexpr bool operator!=(const @\xname{outer_sentinel}@& x, const @\xname{outer_iterator}@<Const>& y);
\end{itemdecl}

\begin{itemdescr}
\pnum
\effects Equivalent to:
\begin{codeblock}
return !(y == x);
\end{codeblock}
\end{itemdescr}

\rSec3[ranges.adaptors.split_view.outer_iterator.value_type]{Class \tcode{split_view::\xname{outer_iterator}::value_type}}

\pnum
\enternote \tcode{split_view::\xname{outer_iterator}::value_type} is an exposition-only type.\exitnote

\begin{codeblock}
namespace std { namespace experimental { namespace ranges { inline namespace v1 {
  template <class Rng, class Pattern>
  template <bool Const>
  struct split_view<Rng, Pattern>::@\xname{outer_iterator}@<Const>::value_type {
  private:
    @\xname{outer_iterator}@ i_ {};
  public:
    value_type() = default;
    constexpr explicit value_type(@\xname{outer_iterator}@ i);

    using iterator = @\xname{inner_iterator}@<Const>;
    using sentinel = @\xname{inner_sentinel}@<Const>;
    using const_iterator = @\xname{inner_iterator}@<Const>;
    using const_sentinel = @\xname{inner_sentinel}@<Const>;

    constexpr iterator begin() const;
    constexpr sentinel end() const;
  };
}}}}
\end{codeblock}

\rSec4[ranges.adaptors.split_view.outer_iterator.value_type.ctor]{\tcode{split_view::\xname{outer_iterator}::value_type} constructors}

\indexlibrary{\idxcode{value_type}!\idxcode{split_view::\xname{outer_iterator}::value_type}}%
\begin{itemdecl}
constexpr explicit value_type(@\xname{outer_iterator}@ i);
\end{itemdecl}

\begin{itemdescr}
\pnum
\effects Initializes \tcode{i_} with \tcode{i}.
\end{itemdescr}

\rSec4[ranges.adaptors.split_view.outer_iterator.value_type.begin]{\tcode{split_view::\xname{outer_iterator}::value_type} range begin}

\indexlibrary{\idxcode{begin}!\idxcode{split_view::\xname{outer_iterator}::value_type}}%
\begin{itemdecl}
constexpr iterator begin() const;
\end{itemdecl}

\begin{itemdescr}
\pnum
\returns \tcode{iterator\{i_\}}.
\end{itemdescr}

\rSec4[ranges.adaptors.split_view.outer_iterator.value_type.end]{\tcode{split_view::\xname{outer_iterator}::value_type} range end}

\indexlibrary{\idxcode{end}!\idxcode{split_view::\xname{outer_iterator}::value_type}}%
\begin{itemdecl}
constexpr sentinel end() const;
\end{itemdecl}

\begin{itemdescr}
\pnum
\returns \tcode{sentinel\{\}}.
\end{itemdescr}

\rSec3[ranges.adaptors.split_view.inner_iterator]{Class template \tcode{split_view::\xname{inner_iterator}}}

\pnum
\enternote \tcode{split_view::\xname{inner_iterator}} is an exposition-only type.\exitnote

\pnum
In the definition of \tcode{split_view<Rng, Pattern>::\xname{inner_iterator}} below,
\tcode{\textit{current}(i)} is equivalent to:
\begin{itemize}
\item If \tcode{Rng} satisfies \tcode{ForwardRange}, \tcode{i.current_}.
\item Otherwise, \tcode{i.parent_->current_}.
\end{itemize}

\begin{codeblock}
namespace std { namespace experimental { namespace ranges { inline namespace v1 {
  template <class Rng, class Pattern>
  template <bool Const>
  struct split_view<Rng, Pattern>::@\xname{inner_iterator}@ {
  private:
    using Base = conditional_t<Const, const Rng, Rng>;
    @\xname{outer_iterator}@<Const> i_ {};
    bool zero_ = false;
  public:
    using iterator_category = iterator_category_t<@\xname{outer_iterator}@<Const>>;
    using difference_type = difference_type_t<iterator_t<Base>>;
    using value_type = value_type_t<iterator_t<Base>>;

    @\xname{inner_iterator}@() = default;
    constexpr explicit @\xname{inner_iterator}@(@\xname{outer_iterator}@<Const> i);

    constexpr decltype(auto) operator*() const;

    constexpr @\xname{inner_iterator}@& operator++();
    constexpr void operator++(int);
    constexpr @\xname{inner_iterator}@ operator++(int) requires ForwardRange<Base>;

    friend constexpr bool operator==(const @\xname{inner_iterator}@& x, const @\xname{inner_iterator}@& y)
      requires ForwardRange<Base>;
    friend constexpr bool operator!=(const @\xname{inner_iterator}@& x, const @\xname{inner_iterator}@& y)
      requires ForwardRange<Base>;

    friend constexpr decltype(auto) iter_move(const @\xname{inner_iterator}@& i)
      noexcept(@\seebelow@);
    friend constexpr void iter_swap(const @\xname{inner_iterator}@& x, const @\xname{inner_iterator}@& y)
      noexcept(@\seebelow@) requires IndirectlySwappable<iterator_t<Base>>;
  };
}}}}
\end{codeblock}

\rSec4[ranges.adaptors.split_view.inner_iterator.ctor]{\tcode{split_view::\xname{inner_iterator}} constructors}

\indexlibrary{\idxcode{\xname{inner_iterator}}!\idxcode{split_view::\xname{inner_iterator}}}%
\begin{itemdecl}
constexpr explicit @\xname{inner_iterator}@(@\xname{outer_iterator}@<Const> i);
\end{itemdecl}

\begin{itemdescr}
\pnum
\effects Initializes \tcode{i_} with \tcode{i}.
\end{itemdescr}

\rSec4[ranges.adaptors.split_view.inner_iterator.star]{\tcode{split_view::\xname{inner_iterator}::operator*}}

\indexlibrary{\idxcode{operator*}!\idxcode{split_view::\xname{inner_iterator}}}%
\begin{itemdecl}
constexpr decltype(auto) operator*() const;
\end{itemdecl}

\begin{itemdescr}
\pnum
\returns \tcode{*\textit{current}(i_)}.
\end{itemdescr}

\rSec4[ranges.adaptors.split_view.inner_iterator.inc]{\tcode{split_view::\xname{inner_iterator}::operator++}}

\indexlibrary{\idxcode{operator++}!\idxcode{split_view::\xname{inner_iterator}}}%
\begin{itemdecl}
constexpr decltype(auto) operator++() const;
\end{itemdecl}

\begin{itemdescr}
\pnum
\effects Equivalent to:
\begin{codeblock}
++@\textit{current}@(i_);
zero_ = true;
return *this;
\end{codeblock}
\end{itemdescr}

\indexlibrary{\idxcode{operator++}!\idxcode{split_view::\xname{inner_iterator}}}%
\begin{itemdecl}
constexpr void operator++(int);
\end{itemdecl}

\begin{itemdescr}
\pnum
\effects Equivalent to \tcode{(void)++*this}.
\end{itemdescr}

\indexlibrary{\idxcode{operator++}!\idxcode{split_view::\xname{inner_iterator}}}%
\begin{itemdecl}
constexpr @\xname{inner_iterator}@ operator++(int) requires ForwardRange<Base>;
\end{itemdecl}

\begin{itemdescr}
\pnum
\effects Equivalent to:
\begin{codeblock}
auto tmp = *this;
++*this;
return tmp;
\end{codeblock}
\end{itemdescr}

\rSec4[ranges.adaptors.split_view.inner_iterator.comp]{\tcode{split_view::\xname{inner_iterator}} comparisons}

\indexlibrary{\idxcode{operator==}!\idxcode{split_view::\xname{inner_iterator}}}%
\begin{itemdecl}
friend constexpr bool operator==(const @\xname{inner_iterator}@& x, const @\xname{inner_iterator}@& y)
requires ForwardRange<Base>;
\end{itemdecl}

\begin{itemdescr}
\pnum
\effects Equivalent to:
\begin{codeblock}
return x.i_.current_ == y.i_.current_;
\end{codeblock}
\end{itemdescr}

\indexlibrary{\idxcode{operator!=}!\idxcode{split_view::\xname{inner_iterator}}}%
\begin{itemdecl}
friend constexpr bool operator!=(const @\xname{inner_iterator}@& x, const @\xname{inner_iterator}@& y)
requires ForwardRange<Base>;
\end{itemdecl}

\begin{itemdescr}
\pnum
\effects Equivalent to:
\begin{codeblock}
return !(x == y);
\end{codeblock}
\end{itemdescr}

\rSec4[ranges.adaptors.split_view.inner_iterator.nonmember]{\tcode{split_view::\xname{inner_iterator}} non-member functions}

\indexlibrary{\idxcode{iter_move}!\idxcode{split_view::\xname{inner_iterator}}}%
\begin{itemdecl}
friend constexpr decltype(auto) iter_move(const @\xname{inner_iterator}@& i)
noexcept(@\seebelow@);
\end{itemdecl}

\begin{itemdescr}
\pnum
\returns \tcode{ranges::iter_move(\textit{current}(i.i_))}.

\pnum
\remarks The expression in the \tcode{noexcept} clause is equivalent to:
\begin{codeblock}
noexcept(ranges::iter_move(@\textit{current}@(i.i_)))
\end{codeblock}
\end{itemdescr}

\indexlibrary{\idxcode{iter_swap}!\idxcode{split_view::\xname{inner_iterator}}}%
\begin{itemdecl}
friend constexpr void iter_swap(const @\xname{inner_iterator}@& x, const @\xname{inner_iterator}@& y)
noexcept(@\seebelow@) requires IndirectlySwappable<iterator_t<Base>>;
\end{itemdecl}

\begin{itemdescr}
\pnum
\effects Equivalent to \tcode{ranges::iter_swap(\textit{current}(x.i_), \textit{current}(y.i_))}.

\pnum
\remarks The expression in the \tcode{noexcept} clause is equivalent to:
\begin{codeblock}
noexcept(ranges::iter_swap(@\textit{current}@(x.i_), @\textit{current}@(y.i_)))
\end{codeblock}
\end{itemdescr}

\rSec3[ranges.adaptors.split_view.inner_sentinel]{Class template \tcode{split_view::\xname{inner_sentinel}}}

\pnum
\enternote \tcode{split_view::\xname{inner_sentinel}} is an exposition-only type.\exitnote

\begin{codeblock}
namespace std { namespace experimental { namespace ranges { inline namespace v1 {
  template <class Rng, class Pattern>
  template <bool Const>
  struct split_view<Rng, Pattern>::@\xname{inner_sentinel}@ {
    friend constexpr bool operator==(const @\xname{inner_iterator}@<Const>& x, @\xname{inner_sentinel}@);
    friend constexpr bool operator==(@\xname{inner_sentinel}@ x, const @\xname{inner_iterator}@<Const>& y);
    friend constexpr bool operator!=(const @\xname{inner_iterator}@<Const>& x, @\xname{inner_sentinel}@ y);
    friend constexpr bool operator!=(@\xname{inner_sentinel}@ x, const @\xname{inner_iterator}@<Const>& y);
  };
}}}}
\end{codeblock}

\rSec4[ranges.adaptors.split_view.inner_sentinel.comp]{\tcode{split_view::\xname{inner_sentinel}} comparisons}

\indexlibrary{\idxcode{operator==}!\idxcode{split_view::\xname{inner_sentinel}}}%
\begin{itemdecl}
friend constexpr bool operator==(const @\xname{inner_iterator}@<Const>& x, @\xname{inner_sentinel}@)
\end{itemdecl}

\begin{itemdescr}
\pnum
\effects Equivalent to:
\begin{codeblock}
auto cur = x.i_.current();
auto end = ranges::end(x.i_.parent_->base_);
if (cur == end) return true;
auto [pcur, pend] = @\changed{iterator_range}{subrange}@{x.i_.parent_->pattern_};
if (pcur == pend) return x.zero_;
do {
  if (*cur != *pcur) return false;
  if (++pcur == pend) return true;
} while (++cur != end);
return false;
\end{codeblock}
\end{itemdescr}

\indexlibrary{\idxcode{operator==}!\idxcode{split_view::\xname{inner_sentinel}}}%
\begin{itemdecl}
friend constexpr bool operator==(@\xname{inner_sentinel}@ x, const @\xname{inner_iterator}@<Const>& y);
\end{itemdecl}

\begin{itemdescr}
\pnum
\effects Equivalent to:
\begin{codeblock}
return y == x;
\end{codeblock}
\end{itemdescr}

\indexlibrary{\idxcode{operator!=}!\idxcode{split_view::\xname{inner_sentinel}}}%
\begin{itemdecl}
friend constexpr bool operator!=(const @\xname{inner_iterator}@<Const>& x, @\xname{inner_sentinel}@ y);
\end{itemdecl}

\begin{itemdescr}
\pnum
\effects Equivalent to:
\begin{codeblock}
return !(x == y);
\end{codeblock}
\end{itemdescr}

\indexlibrary{\idxcode{operator!=}!\idxcode{split_view::\xname{inner_sentinel}}}%
\begin{itemdecl}
friend constexpr bool operator!=(@\xname{inner_sentinel}@ x, const @\xname{inner_iterator}@<Const>& y);
\end{itemdecl}

\begin{itemdescr}
\pnum
\effects Equivalent to:
\begin{codeblock}
return !(y == x);
\end{codeblock}
\end{itemdescr}

\rSec2[ranges.adaptors.split]{\tcode{view::split}}

\pnum
The name \tcode{view::split} denotes a range adaptor object~(\cxxref{???}).
Let \tcode{E} and \tcode{F} be expressions such that their types are \tcode{T}
and \tcode{U} respectively. Then the expression \tcode{view::split(E, F)} is
expression-equivalent to:

\begin{itemize}
\item \tcode{split_view\{E, F\}} if either of the following sets of requirements
is satisfied:
\begin{itemize}
\item \begin{codeblock}
InputRange<T> && ForwardRange<U> &&
@\changed{(is_lvalue_reference_v<T> || View<decay_t<T>{>})}{ViewableRange<T>}@ &&
@\changed{(is_lvalue_reference_v<U> || View<decay_t<U>{>})}{ViewableRange<U>}@ &&
IndirectlyComparable<iterator_t<T>, iterator_t<U>> &&
(ForwardRange<T> || @\textit{tiny-range}@<U>)
\end{codeblock}

\item \begin{codeblock}
InputRange<T> && @\changed{(is_lvalue_reference_v<T> || View<decay_t<T>{>})}{ViewableRange<T>}@ &&
IndirectlyComparable<iterator_t<T>, const value_type_t<iterator_t<T>>*> &&
CopyConstructible<value_type_t<iterator_t<T>>> &&
ConvertibleTo<U, value_type_t<iterator_t<T>>>
\end{codeblock}
\end{itemize}
\item Otherwise, \tcode{view::split(E, F)} is ill-formed.
\end{itemize}

\rSec2[ranges.adaptors.counted]{\tcode{view::counted}}

\pnum
The name \tcode{view::counted} denotes a customization point object~(\cxxref{???}).
Let \tcode{E} and \tcode{F} be expressions such that their decayed types are \tcode{T}
and \tcode{U} respectively. Then the expression \tcode{view::counted(E, F)} is
expression-equivalent to:

\begin{itemize}
\item \tcode{\changed{iterator_range}{subrange}\{E, E + F\}} is \tcode{T} is a pointer to an object type,
and if \tcode{U} is implicitly convertible to \tcode{ptrdiff_t}.
\item Otherwise, \tcode{\changed{iterator_range}{subrange}\{\removed{make_}counted_iterator(E,
\changed{F}{static_cast<difference_type_t<\ T>{>}(F)}), default_sentinel\{\}\}}
if \tcode{It\-erator<T> \&\& ConvertibleTo<U, difference_type_t<T>{>}} is satisfied.
\item Otherwise, \tcode{view::counted(E, F)} is ill-formed.
\end{itemize}

\begin{addedblock}
\rSec2[ranges.adaptors.bounded_view]{Class template \tcode{bounded_view}}

\pnum
The \tcode{bounded_view} takes a range which has different types for its iterator
and sentinel and turns it into an equivalent range where the iterator and sentinel
have the same type.

\pnum
\remark \tcode{bounded_view} is useful for calling legacy algorithms that expect
a range's iterator and sentinel types to be the same.

\pnum
\enterexample
\begin{codeblock}
// Legacy algorithm:
template <class ForwardIterator>
size_t count(ForwardIterator first, ForwardIterator last);

template <ForwardRange R>
void my_algo(R&& r) {
  auto&& bounded = bounded_view{r};
  auto cnt = count(bounded.begin(), bounded.end());
  // ...
}
\end{codeblock}
\exitexample

\begin{codeblock}
namespace std { namespace experimental { namespace ranges { inline namespace v1 {
  template <View Rng>
    requires !BoundedRange<Rng>
  class bounded_view : public view_interface<bounded_view<Rng>> {
  private:
    Rng base_; // \expos
  public:
    bounded_view() = default;

    explicit constexpr bounded_view(Rng rng);

    template <ViewableRange O>
      requires !BoundedRange<O> && Constructible<Rng, all_view<O>>
    explicit constexpr bounded_view(O&& o);

    constexpr Rng base() const;

    constexpr auto begin();
    constexpr auto begin() const requires Range<const Rng>;

    constexpr auto begin()
      requires RandomAccessRange<Rng> && SizedRange<Rng>;
    constexpr auto begin() const
      requires RandomAccessRange<const Rng> && SizedRange<const Rng>;

    constexpr auto end();
    constexpr auto end() const requires Range<const Rng>;

    constexpr auto end()
      requires RandomAccessRange<Rng> && SizedRange<Rng>;
    constexpr auto end() const
      requires RandomAccessRange<const Rng> && SizedRange<Rng>;

    constexpr auto size() const requires SizedRange<const Rng>;
  };

  template <ViewableRange O>
    requires !BoundedRange<O>
  bounded_view(O&&) -> bounded_view<all_view<O>>;
}}}}
\end{codeblock}

\rSec3[ranges.adaptors.bounded_view.ops]{\tcode{bounded_view} operations}
\rSec4[ranges.adaptors.bounded_view.ctor]{\tcode{bounded_view} constructors}

\indexlibrary{\idxcode{bounded_view}!\idxcode{bounded_view}}%
\begin{itemdecl}
explicit constexpr bounded_view(Rng base);
\end{itemdecl}

\begin{itemdescr}
\pnum
\effects Initializes \tcode{base_} with \tcode{std::move(base)}.
\end{itemdescr}

\indexlibrary{\idxcode{bounded_view}!\idxcode{bounded_view}}%
\begin{itemdecl}
template <ViewableRange O>
  requires !BoundedRange<O> && Constructible<Rng, all_view<O>>
explicit constexpr bounded_view(O&& o);
\end{itemdecl}

\begin{itemdescr}
\pnum
\effects Initializes \tcode{base_} with \tcode{view::all(std::forward<O>(o))}.
\end{itemdescr}

\rSec4[ranges.adaptors.bounded_view.conv]{\tcode{bounded_view} conversion}

\indexlibrary{\idxcode{base}!\idxcode{bounded_view}}%
\begin{itemdecl}
constexpr Rng base() const;
\end{itemdecl}

\begin{itemdescr}
\pnum
\returns \tcode{base_}.
\end{itemdescr}

\rSec4[ranges.adaptors.bounded_view.begin]{\tcode{bounded_view} begin}

\indexlibrary{\idxcode{begin}!\idxcode{bounded_view}}%
\begin{itemdecl}
constexpr auto begin();
constexpr auto begin() const requires Range<const Rng>;
\end{itemdecl}

\begin{itemdescr}
\pnum
\effects Equivalent to:
\begin{codeblock}
return common_iterator<iterator_t<Rng>, sentinel_t<Rng>>(ranges::begin(base_));
\end{codeblock}
and
\begin{codeblock}
return common_iterator<iterator_t<const Rng>, sentinel_t<const Rng>>(ranges::begin(base_));
\end{codeblock}
for the first and second overloads, respectively.
\end{itemdescr}

\indexlibrary{\idxcode{begin}!\idxcode{bounded_view}}%
\begin{itemdecl}
constexpr auto begin()
  requires RandomAccessRange<Rng> && SizedRange<Rng>;
constexpr auto begin() const
  requires RandomAccessRange<const Rng> && SizedRange<const Rng>;
\end{itemdecl}

\begin{itemdescr}
\pnum
\effects Equivalent to:
\begin{codeblock}
return ranges::begin(base_);
\end{codeblock}
\end{itemdescr}

\rSec4[ranges.adaptors.bounded_view.end]{\tcode{bounded_view} end}

\indexlibrary{\idxcode{end}!\idxcode{bounded_view}}%
\begin{itemdecl}
constexpr auto end();
constexpr auto end() const requires Range<const Rng>;
\end{itemdecl}

\begin{itemdescr}
\pnum
\effects Equivalent to:
\begin{codeblock}
return common_iterator<iterator_t<Rng>, sentinel_t<Rng>>(ranges::end(base_));
\end{codeblock}
and
\begin{codeblock}
return common_iterator<iterator_t<const Rng>, sentinel_t<const Rng>>(ranges::end(base_));
\end{codeblock}
for the first and second overloads, respectively.
\end{itemdescr}

\indexlibrary{\idxcode{end}!\idxcode{bounded_view}}%
\begin{itemdecl}
constexpr auto end()
  requires RandomAccessRange<Rng> && SizedRange<Rng>;
constexpr auto end() const
  requires RandomAccessRange<const Rng> && SizedRange<const Rng>;
\end{itemdecl}

\begin{itemdescr}
\pnum
\effects Equivalent to:
\begin{codeblock}
return ranges::begin(base_) + ranges::size(base_);
\end{codeblock}
\end{itemdescr}

\rSec4[ranges.adaptors.bounded_view.size]{\tcode{bounded_view} size}

\indexlibrary{\idxcode{size}!\idxcode{bounded_view}}%
\begin{itemdecl}
constexpr auto size() const requires SizedRange<const Rng>;
\end{itemdecl}

\begin{itemdescr}
\pnum
\effects Equivalent to: \tcode{return ranges::size(base_);}.
\end{itemdescr}

\rSec2[ranges.adaptors.bounded]{\tcode{view::bounded}}

\pnum
The name \tcode{view::bounded} denotes a range adaptor
object~(\ref{ranges.adaptor.object}).
Let \tcode{E} be an expression such that \tcode{U} is \tcode{decltype((E))}.
Then the expression \tcode{view::bounded(E)} is expression-equivalent to:

\begin{itemize}
\item If \tcode{ViewableRange<U> \&\& BoundedRange<U>} is satisfied,
\tcode{view::all(E)}.
\item Otherwise, if \tcode{ViewableRange<U>} is satisfied,
\tcode{bounded_view\{E\}}.
\item Otherwise, \tcode{view::bounded(E)} is ill-formed.
\end{itemize}

\rSec2[ranges.adaptors.reverse_view]{Class template \tcode{reverse_view}}

\pnum
The \tcode{reverse_view} takes a bidirectional range and produces another range
that iterates the same elements in reverse order.

\pnum
\enterexample
\begin{codeblock}
vector<int> is {0,1,2,3,4};
reverse_view rv {is};
for (int i : rv)
  cout << i << ' '; // prints: 4 3 2 1 0
\end{codeblock}
\exitexample

\begin{codeblock}
namespace std { namespace experimental { namespace ranges { inline namespace v1 {
  template <View Rng>
    requires BidirectionalRange<Rng>
  class reverse_view : public view_interface<reverse_view<Rng>> {
  private:
    Rng base_; // \expos
  public:
    reverse_view() = default;

    explicit constexpr reverse_view(Rng rng);

    template <ViewableRange O>
      requires BidirectionalRange<O> && Constructible<Rng, all_view<O>>
    explicit constexpr reverse_view(O&& o);

    constexpr Rng base() const;

    constexpr auto begin();
    constexpr auto begin() requires BoundedRange<Rng>;
    constexpr auto begin() const requires BoundedRange<const Rng>;

    constexpr auto end();
    constexpr auto end() const requires BoundedRange<const Rng>;

    constexpr auto size() const requires SizedRange<const Rng>;
  };

  template <ViewableRange O>
    requires BidirectionalRange<O>
  reverse_view(O&&) -> reverse_view<all_view<O>>;
}}}}
\end{codeblock}

\rSec3[ranges.adaptors.reverse_view.ops]{\tcode{reverse_view} operations}
\rSec4[ranges.adaptors.reverse_view.ctor]{\tcode{reverse_view} constructors}

\indexlibrary{\idxcode{reverse_view}!\idxcode{reverse_view}}%
\begin{itemdecl}
explicit constexpr reverse_view(Rng base);
\end{itemdecl}

\begin{itemdescr}
\pnum
\effects Initializes \tcode{base_} with \tcode{std::move(base)}.
\end{itemdescr}

\indexlibrary{\idxcode{reverse_view}!\idxcode{reverse_view}}%
\begin{itemdecl}
template <ViewableRange O>
  requires !BoundedRange<O> && Constructible<Rng, all_view<O>>
explicit constexpr reverse_view(O&& o);
\end{itemdecl}

\begin{itemdescr}
\pnum
\effects Initializes \tcode{base_} with \tcode{view::all(std::forward<O>(o))}.
\end{itemdescr}

\rSec4[ranges.adaptors.reverse_view.conv]{\tcode{reverse_view} conversion}

\indexlibrary{\idxcode{base}!\idxcode{reverse_view}}%
\begin{itemdecl}
constexpr Rng base() const;
\end{itemdecl}

\begin{itemdescr}
\pnum
\returns \tcode{base_}.
\end{itemdescr}

\rSec4[ranges.adaptors.reverse_view.begin]{\tcode{reverse_view} begin}

\indexlibrary{\idxcode{begin}!\idxcode{reverse_view}}%
\begin{itemdecl}
constexpr auto begin();
\end{itemdecl}

\begin{itemdescr}
\pnum
\effects Equivalent to:
\begin{codeblock}
return reverse_iterator{ranges::next(ranges::begin(base_), ranges::end(base_))};
\end{codeblock}

\pnum
\remarks In order to provide the amortized constant time complexity required by
the \tcode{Range} concept, this function caches the result within the
\tcode{reverse_view} for use on subsequent calls.
\end{itemdescr}

\indexlibrary{\idxcode{begin}!\idxcode{reverse_view}}%
\begin{itemdecl}
constexpr auto begin() requires BoundedRange<Rng>;
constexpr auto begin() const requires BoundedRange<const Rng>;
\end{itemdecl}

\begin{itemdescr}
\pnum
\effects Equivalent to:
\begin{codeblock}
return reverse_iterator{ranges::end(base_)};
\end{codeblock}
\end{itemdescr}

\rSec4[ranges.adaptors.reverse_view.end]{\tcode{reverse_view} end}

\indexlibrary{\idxcode{end}!\idxcode{reverse_view}}%
\begin{itemdecl}
constexpr auto end() requires BoundedRange<Rng>;
constexpr auto end() const requires BoundedRange<const Rng>;
\end{itemdecl}

\begin{itemdescr}
\pnum
\effects Equivalent to:
\begin{codeblock}
return reverse_iterator{ranges::begin(base_)};
\end{codeblock}
\end{itemdescr}

\rSec4[ranges.adaptors.reverse_view.size]{\tcode{reverse_view} size}

\indexlibrary{\idxcode{size}!\idxcode{reverse_view}}%
\begin{itemdecl}
constexpr auto size() const requires SizedRange<const Rng>;
\end{itemdecl}

\begin{itemdescr}
\pnum
\effects Equivalent to:
\begin{codeblock}
return ranges::size(base_);
\end{codeblock}
\end{itemdescr}

\rSec2[ranges.adaptors.reverse]{\tcode{view::reverse}}

\pnum
The name \tcode{view::reverse} denotes a range adaptor
object~(\ref{ranges.adaptor.object}).
Let \tcode{E} be an expression such that \tcode{U} is \tcode{decltype((E))}.
Then the expression \tcode{view::reverse(E)} is expression-equivalent to:

\begin{itemize}
\item If \tcode{ViewableRange<U> \&\& BidirectionalRange<U>} is satisfied,
\tcode{reverse_view\{E\}}.
\item Otherwise, \tcode{view::reverse(E)} is ill-formed.
\end{itemize}

\end{addedblock}
