\rSec0[intro]{General}
\rSec1[intro.change.log]{Change Log}
\rSec2[intro.change.log.minor]{Minor changes}

The following changes are editorial in nature, or were added to the discussion
in LEWG.

\begin{itemize}
\item Minor typographical changes (e.g. \tcode{s/typename/class/g}).
\item Moved \doccite{Annex B Reference implementation} to \ref{intro.compliance}.
\item All \tcode{NoThrow} iterators delimited by \tcode{Sentinel}s are now delimited by
      \tcode{\placeholder{no-throw-sentinel}}s.
\end{itemize}

\rSec2[intro.change.log.major]{Major changes}

The following changes affect the proposal's direction, as requested by LEWG.

\begin{itemize}
\item Re-specified all concepts as exposition-only.
\item Removed \doccite{Annex A Compatibility features}, which deprecated the three-legged memory
      specialisations.
\item Changed the \tcode{uninitialized_copy_n} and \tcode{uninitialized_move_n} so that they take a
      full output range, rather than half an output range. This was not strictly directed by LEWG,
      but it is in line with the direction to remove the three-legged algorithms.
\item Added editor notes to preserve the requirements from \doccite{Annex A} that affect the
      memory specialisations in namespace \tcode{std}.
\end{itemize}

\rSec1[intro.scope]{Scope}

\pnum
This proposal describes functions compatible with the Ranges TS. These extensions change and add to
the existing memory specialisation library facilities found in C++17.

\rSec1[intro.motivation]{Motivation}

\pnum
\doccite{N3351 A Concept Design for the STL} (Stroustrup and Sutton, 2012), also known as the
`Palo Alto Report', serves as the basis for the Ranges TS. Both the Palo Alto Report and the Ranges
TS focus on the algorithms located in \tcode{<algorithm>}, and are not extended to revise the
algorithms found in \tcode{<memory>} and \tcode{<numeric>}. P1033 seeks to include those algorithms
found in \tcode{<memory>} alongside the algorithms revised by the Ranges TS.

\rSec1[intro.compliance]{Implementation compliance}

\pnum
Similarly to the Ranges TS, conformance requirements are the same as those described in
\ref{intro.compliance} in the \Cpp Standard.
\enternote
Conformance is defined in terms of the behaviour of programs.
\exitnote

\pnum
A reference implementation can be found in the master branch of Casey Carter's Ranges TS reference
implementation. For a closer inspection, please visit
\href{https://github.com/CaseyCarter/cmcstl2/tree/master/include/stl2/detail/memory}{the reference implementation}.

\rSec1[intro.namespace]{Namespaces, headers, and modifications to standard classes}

\pnum
All components described in this document are declared in an unspecified namespace.

\ednote{The following text is taken from the \Cpp Standard and edited to reflect the fact that much
of this document is suggesting parallel constrained facilities that are specified as diffs against
the existing unconstrained facilities in namespace \tcode{std}.}

\pnum
The International Standard, ISO/IEC 14882, together with ISO/IEC 19217:2015 (the Concepts TS) and
ISO/IEC TS 21425:2017 (the Ranges TS), provide important context and specification for this paper.
This document is written as a set of changes against the C++ Working Paper, N4741. Sections are
copied verbatim and modified so as to define similar but different components in namespace \tcode{std}.
Effort was made to keep chapter and section numbers the same as in N4741 for the sake of easy
cross-referencing with the understanding that section numbers will change in the final draft of
ISO/IEC 14882 that this proposal is integrated with.

\pnum
References to other entities described in this document are assumed to be qualified with
\tcode{::std::ranges::}, and references to entities described in the International Standard are
assumed to be qualified with \tcode{::std::}.

\pnum
P1033 uses diff formatting to show how the proposed algorithms differ from the existing algorithms
in the International Standard. \added{Turquoise formatting} indicates text added to
\tcode{::std::ranges}, and \removed{red formatting} indicates where the proposed algorithms diverge
from their International Standard forebears. Except where expressly noted, no changes are proposed
to the existing algorithms.

\pnum
This document specifically describes changes to \tcode{<memory>}. As such, new content can be found
in the \tcode{<ranges/memory>} header.

\begin{floattable}{Proposal headers}{tab:intro.headers}{l}
\topline
\tcode{<ranges/memory>}\\
\bottomline
\end{floattable}
