\rSec0[intro.scope]{Scope}

\begin{quote}
``Eventually, all things merge into one, and a river runs through it.''
\begin{flushright}
\textemdash \textit{Norman Maclean}
\end{flushright}
\end{quote}

\pnum
This document proposes changes to the components of namespace \tcode{std} and
namespace \tcode{std::ranges} to deepen the integration of the components of the
Ranges TS into the Standard Library.

\rSec0[intro.refs]{Normative References}

\pnum
The following referenced documents are indispensable for the
application of this document. For dated references, only the
edition cited applies. For undated references, the latest edition
of the referenced document (including any amendments) applies.

\begin{itemize}
\item ISO/IEC 14882, \doccite{Programming Languages - \Cpp}
\item ISO/IEC TS 21425:2017, \doccite{Technical Specification - \Cpp Extensions for Ranges}
\end{itemize}

ISO/IEC 14882 is herein called the \defn{C\Rplus\Rplus\xspace Standard} and
ISO/IEC TS 21425:2017 is called the \defn{Ranges TS}.

\rSec0[intro]{General Principles}

\rSec1[intro.goals]{Goals}

\pnum
The goal of this paper is to reduce the needless duplication of functionality
that would come from naively dropping the Ranges TS into the \tcode{std::ranges}
namespace. We also aim to simplify the job of authoring an iterator type that
works with both existing generic code, which expects types to conform to the
iterator requirements tables~([iterator.requirements]) as well as to satisfy
the requirements of the iterator concepts as defined in the Ranges TS.

This paper builds a bridge between the pre-concepts STL and the newer
constrained facilities.

\rSec1[intro.methedology]{Methodology}

\rSec2[intro.background]{Background}

\pnum
When the STL was first written, there was no way test an expression to see if
it was well-formed without causing a hard error at compile-time. As a result,
iterators needed to explicitly tell generic code what concept they satisfied
via a \tcode{::iterator_category} typedef, either within a specialization of
\tcode{std::iterator_traits} or nested within the iterator type itself.

\pnum
With the addition of generalized SFINAE for expressions~(\cite{expr-sfinae}),
and especially with the addition of concepts, generic code no longer strictly
needs the iterator to declare which concept is satisfied; instead, generic
code can inspect which required expressions the type supports. Iterator tag types
are needed only to disambiguate between concepts that differ only in semantics,
or to opt out of accidental conformance.

\pnum
For legacy reasons, in both the working draft and in the Ranges TS, generic
algorithms still require iterator types to declare their category via tag. This
paper proposes doing away with that requirement, using syntactic conformance to
infer the tag type for legacy code that still requires it. New code will simply
use \textit{requires-clauses} and the iterator concepts from the Ranges TS to
constrain algorithms and select implementations.

\pnum
A great deal of the machinery in the Ranges TS duplicates existing functionality
in Standard Library. This made sense when the Ranges TS was seen as the
beginning of a fork of the Standard Library itself. That duplication now seems
like bad design.

\pnum
By leveraging the compiler's new ability to infer concept satisfaction and by
reusing existing library functionality, we can make it easier for users to author
iterator types that meet the expectations of old generic code, while also
addressing the shortcomings of the STL that the Ranges TS aimed to address.

\rSec2[intro.strategy]{Implementation Strategy}

\rSec3[intro.iterator_tags]{Iterator tags}

\pnum
The Ranges TS defines a parallel set of iterator tag types, from
\tcode{input_iterator_tag} and \tcode{output_iterator_tag}, through
\tcode{random_access_iterator_tag}. This paper proposes to remove those
tag types and simply reuse the existing ones, with the addition of a new
\tcode{contiguous_iterator_tag} type that represents a refinement of
random-access iterators.

\rSec3[intro.iterator_concept]{\tcode{iterator_concept}}

\pnum
There are several differences between the iterator concepts as defined in the
C\Rplus\Rplus\xspace Standard and the Ranges TS. To accomodate the
fact that a single type might satisfy different iterator concepts in the
different standards -- and the fact that there will always be times when it is
necessary to explicitly specify conformance with a tag type -- this paper
proposes to add an additional optional member to \tcode{std::iterator_traits}
named \tcode{::iterator_concept}.

\pnum
As always, \tcode{std::iterator_traits<I>::iterator_category} specifies which of
the requirements tables in [iterator.requirements] the type \tcode{I} purports
to satisfy, whereas \tcode{std::iterator_traits<I>::iterator_concept}, when
present, is used to opt-in or -out of satisfaction of concepts as defined in the
Ranges TS.

\rSec3[intro.spec_assoc_types]{Specifying associated types}

\pnum
The Ranges TS defines three different customization points --
\tcode{ranges::value_type<>}, \tcode{ranges::difference_type<>}, and
\tcode{ranges::iterator_category<>} -- for specifying the associated types for
the \tcode{Readable}, \tcode{Weakly\-Incrementable}, and \tcode{InputIterator}
concepts, respectively.

\pnum
With the addition of the optional \tcode{iterator_concept} nested typedef of
\tcode{std::iterator_traits<>}, the \tcode{ranges::iterator_category<>}
customization point is no longer necessary. Should users need to non-intrusively
specify an iterator's tag, they may simply specialize \tcode{std::iterator_traits}
as they always have.

\pnum
For the sake of clarity and consistency with the naming of
\tcode{std::iterator_traits}, we suggest renaming
\tcode{std::ranges::value_type<I>::type} to \tcode{std::ranges::readable_traits<I>::value_type}.
Likewise, \tcode{std::\-ranges::\-difference_type<I>::type} is renamed to
\tcode{std::ranges::incrementable_traits<I>::difference_type}.

\pnum
The primary \tcode{std::iterator_traits<>} template uses
\tcode{std::ranges::readable_traits<>} and
\tcode{std::\-ranges::\-incrementable_traits<>} for computing \tcode{::value_type}
and \tcode{::difference_type}, respectively. That way, users need only specify
these traits once.

\rSec3[intro.use_assoc_types]{Using associated types}

\pnum
To specify an iterator's \tcode{value_type}, users may specialize either
\tcode{std::iterator_traits<>}, \tcode{std::\-ranges::\-readable_traits<>}, or both.
If they specialize \tcode{iterator_traits<>} and not \tcode{readable_traits<>},
then \tcode{readable_traits<I>::value_type} will not reflect the value type they
specified in \tcode{iterator_traits}. Generic code that uses
\tcode{typename std::ranges::readable_traits<I>::value_type} directly is
most likely wrong when \tcode{I} is an iterator type.

\pnum
One possible solution would be to simply remove \tcode{std::ranges::readable_traits}
and tell people to use \tcode{iterator_traits} to specify the value type of their
\tcode{Readable} types. However, there are readable types that are not iterators;
for example, \tcode{std::optional<int>}. Needing to specialize something called
``\tcode{iterator_traits}'' and specify a \tcode{difference_type} for something
that is not incrementable would be strange, as would telling users to create a
specialization of \tcode{iterator_traits} that lacks some of the traditional
typedefs.

\pnum
Instead, we propose to use the alias template \tcode{value_type_t<>} -- renamed
to \tcode{iter_value_t} and promoted to namespace \tcode{std} -- from the
Ranges TS, and to make it smarter. Rather than immediately dispatching to either
\tcode{iterator_traits<I>} or \tcode{readable_traits<I>}, \tcode{iter_value_t<I>}
first tests whether \tcode{iterator_traits<I>} has selected a user-specialization.
If so, \tcode{iter_value_t<I>} is an alias for
\tcode{std::iterator_traits<I>::value_type}; otherwise, it is an alias for
\tcode{std::ranges::readable_traits<I>::value_type}.

\pnum
The \tcode{difference_type_t} alias from the Ranges TS, renamed to
\tcode{std::iter_difference_t}, gets the same treatment.

\pnum
Testing whether the user specialized a trait is a matter of giving the primary
template some testable property that user specializations would lack. For instance,
the primary template \tcode{std::iterator_traits<I>} might define a hidden nested
typedef \tcode{__unspecialized} that is an alias for \tcode{I}. Any specialization
\tcode{std::iterator_traits<I>} that either lacks that typedef or has one that
is not an alias for \tcode{I} is necessarily a user-defined specialization.

\rSec3[intro.naming]{Names of iterator associated types}

\pnum
As already mentioned above, the aliases \tcode{std::ranges::value_type_t} and
\tcode{std::ranges::difference_type_t} are renamed to \tcode{std::iter_value_t}
and \tcode{std::iter_difference_t}. Here is the complete list of iterator
associated types, and their old and new names:

\begin{libsumtabbase}{Iterator associated types}{tab:intro.assoc_types_names}{Old name}{New name}
\tcode{std::ranges::difference_type_t}  &   & \tcode{std::iter_difference_t}          \\
\tcode{std::ranges::value_type_t}       &   & \tcode{std::iter_value_t}               \\
\tcode{std::ranges::reference_t}        &   & \tcode{std::iter_reference_t}           \\
\tcode{std::ranges::rvalue_reference_t} &   & \tcode{std::iter_rvalue_reference_t}    \\
\end{libsumtabbase}

\pnum
The new names correct two problems with the old names: the term ``\tcode{type}''
and the suffix ``\tcode{_t}'' both carry the same semantic information, and
names like ``value type'' and ``reference'' are overly general. The prefix
``\tcode{iter_}'' already denotes ``of iterators'', as in \tcode{iter_swap}.

\rSec2[intro.spec]{Method of specification}

\pnum
The changes suggested by this paper are presented as a set of diffs against
the working draft as ammended by P0896 and P0944.

\pnum
Where the text of the Ranges TS or the working draft needs to be updated,
the text is presented with change markings: \removed{red strikethrough} for
removed text and \added{blue underline} for added text.
