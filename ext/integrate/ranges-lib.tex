%!TEX root = std.tex
\rSec0[ranges]{Ranges library}

\rSec1[ranges.general]{General}

\pnum
This clause describes components for dealing with ranges of elements.

\pnum
The following subclauses describe
range and view requirements, and
components for
range primitives
as summarized in Table~\ref{tab:ranges.lib.summary}.

\begin{libsumtab}{Ranges library summary}{tab:ranges.lib.summary}
  \removed{\ref{iterators}}    & \removed{Iterators}         & \removed{\tcode{<range>}} \\
  \ref{ranges.access}       & Range access      & \added{\tcode{<range>}} \\
  \ref{ranges.primitives}   & Range primitives  & \\
  \ref{ranges.requirements} & Requirements      & \\
  \removed{\ref{algorithms}}   & \removed{Algorithms}        & \\
\end{libsumtab}

\rSec1[ranges.decaycopy]{decay_copy}

\ednote{... as in P0896R1.}

\rSec1[ranges.synopsis]{Header \tcode{<range>} synopsis}

\ednote{Remove the \tcode<range> synopsis from P0896R1 and replace it with the
following. Note that everything except the range access points and range
primitives gets promoted to namespace \tcode{std}.}

\indexlibrary{\idxhdr{range}}%
\begin{codeblock}
#include <iterator>
#include <initializer_list>

namespace std {
  namespace ranges {
    inline namespace @\unspec@ {
      // \ref{ranges.access}, range access:
      inline constexpr @\unspec@ begin = @\unspec@;
      inline constexpr @\unspec@ end = @\unspec@;
      inline constexpr @\unspec@ cbegin = @\unspec@;
      inline constexpr @\unspec@ cend = @\unspec@;
      inline constexpr @\unspec@ rbegin = @\unspec@;
      inline constexpr @\unspec@ rend = @\unspec@;
      inline constexpr @\unspec@ crbegin = @\unspec@;
      inline constexpr @\unspec@ crend = @\unspec@;

      // \ref{ranges.primitives}, range primitives:
      inline constexpr @\unspec@ size = @\unspec@;
      inline constexpr @\unspec@ empty = @\unspec@;
      inline constexpr @\unspec@ data = @\unspec@;
      inline constexpr @\unspec@ cdata = @\unspec@;
    }
  }

  template <class T>
  using iterator_t = decltype(ranges::begin(declval<T&>()));

  template <class T>
  using sentinel_t = decltype(ranges::end(declval<T&>()));

  template <class>
  constexpr bool disable_sized_range = false;

  template <class T>
  struct enable_view { };

  struct view_base { };

  // \ref{ranges.requirements}, range requirements:

  // \ref{ranges.range}, Range:
  template <class T>
  concept Range = @\seebelow@;

  // \ref{ranges.sized}, SizedRange:
  template <class T>
  concept SizedRange = @\seebelow@;

  // \ref{ranges.view}, View:
  template <class T>
  concept View = @\seebelow@;

  // \ref{ranges.common}, CommonRange:
  template <class T>
  concept CommonRange = @\seebelow@;

  // \ref{ranges.input}, InputRange:
  template <class T>
  concept InputRange = @\seebelow@;

  // \ref{ranges.output}, OutputRange:
  template <class R, class T>
  concept OutputRange = @\seebelow@;

  // \ref{ranges.forward}, ForwardRange:
  template <class T>
  concept ForwardRange = @\seebelow@;

  // \ref{ranges.bidirectional}, BidirectionalRange:
  template <class T>
  concept BidirectionalRange = @\seebelow@;

  // \ref{ranges.random.access}, RandomAccessRange:
  template <class T>
  concept RandomAccessRange = @\seebelow@;
}
\end{codeblock}

\ednote{Remove 29.4, [ranges.iterators], from P0896R1.}

\rSec1[ranges.access]{Range access}

\begin{addedblock}
\pnum
In addition to begin available via inclusion of the \tcode{<range>} header,
the customization point objects in \ref{ranges.access} are available when
\tcode{<iterator>} is included.
\end{addedblock}

\ednote{...otherwise, as in P0896R1.}

\rSec1[ranges.primitives]{Range primitives}

\begin{addedblock}
\pnum
In addition to begin available via inclusion of the \tcode{<range>} header,
the customization point objects in \ref{ranges.primitives} are available when
\tcode{<iterator>} is included.
\end{addedblock}

\ednote{...otherwise, as in P0896R1.}

\rSec1[ranges.requirements]{Range requirements}

\rSec2[ranges.requirements.general]{General}

\ednote{...as in P0896R1.}

\rSec2[ranges.range]{Ranges}

\ednote{...as in P0896R1.}

\rSec2[ranges.sized]{Sized ranges}
\ednote{Change the definition of the \tcode{SizedRange} concept as follows:}

\pnum
The \tcode{SizedRange} concept specifies the requirements
of a \tcode{Range} type that knows its size in constant time with the
\tcode{size} function.

\begin{itemdecl}
template <class T>
concept SizedRange =
  Range<T> &&
  !disable_sized_range<@remove_cv\added{ref}_t<\removed{remove_reference_t<}R>\removed{>}@> &&
  requires(T& t) {
    { ranges::size(t) } -> ConvertibleTo<@\changed{difference_type_t}{iter_difference_t}@<iterator_t<T>>>;
  };
\end{itemdecl}

\begin{itemdescr}
\pnum
Given an lvalue \tcode{t} of type \tcode{remove_reference_t<T>},
\tcode{SizedRange<T>} is satisfied only if:

\begin{itemize}
\item \tcode{ranges::size(t)} is \bigoh{1}, does not modify \tcode{t}, and is equal
to \tcode{ranges::distance(t)}.

\item If \tcode{iterator_t<T>} satisfies \tcode{ForwardIterator},
\tcode{size(t)} is well-defined regardless of the evaluation of
\tcode{begin(t)}. \enternote \tcode{size(t)} is otherwise not required be
well-defined after evaluating \tcode{begin(t)}. For a \tcode{SizedRange}
whose iterator type does not model \tcode{ForwardIterator}, for
example, \tcode{size(t)} might only be well-defined if evaluated before
the first call to \tcode{begin(t)}. \exitnote
\end{itemize}

\pnum
\enternote The \tcode{disable_sized_range} predicate provides a mechanism to enable use
of range types with the library that meet the syntactic requirements but do
not in fact satisfy \tcode{SizedRange}. A program that instantiates a library template
that requires a \tcode{Range} with such a range type \tcode{R} is ill-formed with no
diagnostic required unless
\tcode{disable_sized_range<remove_cv\added{ref}_t<\removed{remove_reference_t<}R>\removed{>}>} evaluates
to \tcode{true}~(\cxxref{structure.requirements}). \exitnote
\end{itemdescr}

\rSec2[ranges.view]{Views}

\ednote{...as in P0896R1.}

\rSec2[ranges.common]{Common ranges}

\ednote{...as in P0896R1.}

\rSec2[ranges.input]{Input ranges}

\ednote{...as in P0896R1.}

\rSec2[ranges.output]{Output ranges}

\ednote{...as in P0896R1.}

\rSec2[ranges.forward]{Forward ranges}

\ednote{...as in P0896R1.}

\rSec2[ranges.bidirectional]{Bidirectional ranges}

\ednote{...as in P0896R1.}

\rSec2[ranges.random.access]{Random access ranges}

\ednote{...as in P0896R1.}

\rSec2[ranges.contiguous]{Contiguous ranges}

\ednote{...as in P0944.}

\rSec1[dangling.wrappers]{Dangling wrapper}

\rSec2[dangling.wrap]{Class template \tcode{dangling}}

\pnum
\indexlibrary{\idxcode{dangling}}%
Class template \tcode{dangling} is a wrapper for an object that refers to another object whose
lifetime may have ended. It is used by algorithms that accept rvalue ranges and return iterators.

\begin{codeblock}
namespace std { namespace ranges {
  template <CopyConstructible T>
  class dangling {
  public:
    constexpr dangling() requires DefaultConstructible<T>;
    constexpr dangling(T t);
    constexpr T get_unsafe() const;
  private:
    T value; // \expos
  };

  template <Range R>
  using safe_iterator_t =
    conditional_t<is_lvalue_reference_v<R>,
      iterator_t<R>,
      dangling<iterator_t<R>>>;
}}
\end{codeblock}

\rSec3[dangling.wrap.ops]{\tcode{dangling} operations}

\rSec4[dangling.wrap.op.const]{\tcode{dangling} constructors}

\indexlibrary{\idxcode{dangling}!\idxcode{dangling}}%
\begin{itemdecl}
constexpr dangling() requires DefaultConstructible<T>;
\end{itemdecl}

\begin{itemdescr}
\pnum
\effects Constructs a \tcode{dangling}, value-initializing \tcode{value}.
\end{itemdescr}

\indexlibrary{\idxcode{dangling}!\idxcode{dangling}}%
\begin{itemdecl}
constexpr dangling(T t);
\end{itemdecl}

\begin{itemdescr}
\pnum
\effects Constructs a \tcode{dangling}, initializing \tcode{value} with
\tcode{\added{std::move(}t\added{)}}.
\end{itemdescr}

\rSec4[dangling.wrap.op.get]{\tcode{dangling::get_unsafe}}

\indexlibrary{\idxcode{get_unsafe}!\idxcode{dangling}}%
\indexlibrary{\idxcode{dangling}!\idxcode{get_unsafe}}%
\begin{itemdecl}
constexpr T get_unsafe() const;
\end{itemdecl}

\begin{itemdescr}
\pnum
\returns \tcode{value}.
\end{itemdescr}

\ednote{Remove 29.9, [ranges.algorithms], from P0896R1.}
