%!TEX root = stdconcepts.tex
\newpage
\setcounter{chapter}{19}
\setcounter{table}{14}
\rSec0[library]{Library introduction}

\rSec1[library.general]{General}

\ednote{Modify Table \ref{tab:library.categories} as follows (note that the
consequent renumbering of the clauses following the newly-inserted "Concepts
library" is NOT depicted here or in the remainder of this document for ease of
review):}

\begin{libsumtabbase}{Library categories}{tab:library.categories}{Clause}{Category}
Clause \cxxref{language.support}  &   &   Language support library    \\
\added{\ref{concepts.lib}} &   &   \added{Concepts library}    \\
Clause \cxxref{diagnostics}       &   &   Diagnostics library         \\
Clause \cxxref{utilities}         &   &   General utilities library   \\
Clause \cxxref{strings}           &   &   Strings library             \\
Clause \cxxref{localization}      &   &   Localization library        \\
Clause \cxxref{containers}        &   &   Containers library          \\
Clause \cxxref{iterators}         &   &   Iterators library           \\
Clause \cxxref{algorithms}        &   &   Algorithms library          \\
Clause \cxxref{numerics}          &   &   Numerics library            \\
Clause \cxxref{input.output}      &   &   Input/output library        \\
Clause \cxxref{re}                &   &   Regular expressions library \\
Clause \cxxref{atomics}           &   &   Atomic operations library   \\
Clause \cxxref{thread}            &   &   Thread support library      \\
\end{libsumtabbase}

\ednote{Add a new paragraph between paragraphs 4 and 5:}

\setcounter{Paras}{4}
\begin{addedblock}
\pnum
The concepts library (\ref{concepts.lib}) describes library components
that \Cpp programs may use to perform compile-time validation of template
parameters and perform function dispatch based on properties of types.
\end{addedblock}

\setcounter{section}{2}
\rSec1[definitions]{Definitions}

\ednote{Add a new definition for "expression-equivalent":}

{\color{addclr}
\setcounter{subsection}{10}
\definition{expression-equivalent}{defns.expression.equivalent}
\indexdefn{expression-equivalent}%
relationship that exists between two expressions \tcode{E1} and \tcode{E2} such that
\begin{itemize}
\item
\tcode{E1} and \tcode{E2} have the same effects,

\item
\tcode{noexcept(E1) == noexcept(E2)}, and

\item
\tcode{E1} is a constant subexpression if and only if \tcode{E2} is a constant subexpression
\end{itemize}
} %% \color{newclr}

\rSec1[description]{Method of description (Informative)}

\rSec2[structure]{Structure of each clause}

\setcounter{subsubsection}{1}
\rSec3[structure.summary]{Summary}

\ednote{Add a new bullet to the list in paragraph 2:}

\setcounter{Paras}{1}
\pnum
The contents of the summary and the detailed specifications include:

\begin{itemize}
\item macros
\item values
\item types
\item classes and class templates
\item functions and function templates
\item objects
\item \added{concepts}
\end{itemize}

\rSec3[structure.requirements]{Requirements}

\ednote{Modify paragraph 1 as follows:}

\pnum
\indextext{requirements}%
Requirements describe constraints that shall be met by a \Cpp{} program that extends the standard library.
Such extensions are generally one of the following:

\begin{itemize}
\item Template arguments
\item Derived classes
\item Containers, iterators, and algorithms that meet an interface convention \added{or satisfy a concept}
\end{itemize}

\ednote{Modify paragraph 4 as follows:}

\setcounter{Paras}{3}
\pnum
Requirements are stated in terms of well-defined expressions that define valid terms of
the types that satisfy the requirements. For every set of well-defined expression
requirements there is \newnewtxt{either a named concept or}
a table that specifies an initial set of the valid expressions and
their semantics. Any generic algorithm~(\cxxref{algorithms}) that uses the
well-defined expression requirements is described in terms of the valid expressions for
its template type parameters.

\ednote{Add new paragraphs after the existing paragraphs:}

\setcounter{Paras}{6}
\begin{addedblock}
\pnum
Required operations of any concept defined in this document need not be
total functions; that is, some arguments to a required operation may
result in the required semantics failing to be satisfied. \enterexample
The required \tcode{<} operator of the \tcode{StrictTotallyOrdered}
concept~(\ref{concepts.lib.compare.stricttotallyordered}) does not meet the
semantic requirements of that concept when operating on NaNs.\exitexample
This does not affect whether a type satisfies the concept.

\pnum
A declaration may explicitly impose requirements through its associated
constraints~(\cxxref{temp.constr.decl}). When the associated constraints
refer to a concept~(\cxxref{temp.concept}), additional semantic requirements are
imposed on the use of the declaration.
\end{addedblock}

\rSec2[conventions]{Other conventions}

\rSec3[type.descriptions]{Type descriptions}

\ednote{Add a new subclause after [character.seq]:}
\setcounter{paragraph}{5}
\begin{addedblock}
\rSec4[customization.point.object]{Customization Point Object types}

\pnum
A \techterm{customization point object} is a function object~(\ref{function.objects}) with a
literal class type that interacts with user-defined types while
enforcing semantic requirements on that interaction.

\pnum
The type of a customization point object shall satisfy
\tcode{Semiregular}~(\ref{concepts.lib.object.semiregular}).

\pnum
All instances of a specific customization point object type shall
be equal~(\ref{concepts.lib.general.equality}).

\pnum
The type of a customization point object \tcode{T} shall satisfy
\tcode{Invocable<const T\newtxt{\&}, Args...>}~(\ref{concepts.lib.callable.invocable}) when the types of
\tcode{Args...} meet the requirements specified in that
customization point object's definition. \oldtxt{Otherwise}\newtxt{When the types of
\tcode{Args...} do not meet the customization point object's requirements}, \tcode{T}
shall not have a function call operator that participates in
overload resolution.

\pnum
Each customization point object type constrains its return type
to satisfy a particular concept.

\pnum
\oldtxt{The library defines several named customization point objects.
In every translation unit where such a name is defined, it shall
refer to the same instance of the customization point object.}

\pnum
\enternote Many of the customization point objects in the library
evaluate function call expressions with an unqualified name which
results in a call to a user-defined function found by argument
dependent name lookup~(\cxxref{basic.lookup.argdep}). To preclude
such an expression resulting in a call to unconstrained functions
with the same name in namespace \tcode{std}, customization point
objects specify that lookup for these expressions is performed in
a context that includes deleted overloads matching the signatures
of overloads defined in namespace \tcode{std}. When the deleted
overloads are viable, user-defined overloads must be more
specialized~(\cxxref{temp.func.order}) or more
constrained~(\cxxref{temp.constr.order}) to be used by a
customization point object. \exitnote
\end{addedblock}

\rSec1[requirements]{Library-wide requirements}

\setcounter{subsection}{1}
\setcounter{subsubsection}{1}
\rSec3[headers]{Headers}

\ednote{Add header \tcode{<concepts>} to Table \ref{tab:cpp.library.headers}}

\begin{floattable}{\Cpp{} library headers}{tab:cpp.library.headers}
{llll}
\topline
\tcode{<algorithm>} &
\tcode{<fstream>} &
\tcode{<new>} &
\tcode{<string_view>} \\

\tcode{<any>} &
\tcode{<functional>} &
\tcode{<numeric>} &
\tcode{<strstream>} \\

\tcode{<array>} &
\tcode{<future>} &
\tcode{<optional>} &
\tcode{<syncstream>} \\

\tcode{<atomic>} &
\tcode{<initializer_list>} &
\tcode{<ostream>} &
\tcode{<system_error>} \\

\tcode{<bitset>} &
\tcode{<iomanip>} &
\tcode{<queue>} &
\tcode{<thread>} \\

\tcode{<charconv>} &
\tcode{<ios>} &
\tcode{<random>} &
\tcode{<tuple>} \\

\tcode{<chrono>} &
\tcode{<iosfwd>} &
\tcode{<ratio>} &
\tcode{<type_traits>} \\

\tcode{<codecvt>} &
\tcode{<iostream>} &
\tcode{<regex>} &
\tcode{<typeindex>} \\

\tcode{<compare>} &
\tcode{<istream>} &
\tcode{<scoped_allocator>} &
\tcode{<typeinfo>} \\

\tcode{<complex>} &
\tcode{<iterator>} &
\tcode{<set>} &
\tcode{<unordered_map>} \\

\added{\tcode{<concepts>}} &
\tcode{<limits>} &
\tcode{<shared_mutex>} &
\tcode{<unordered_set>} \\

\tcode{<condition_variable>} &
\tcode{<list>} &
\tcode{<sstream>} &
\tcode{<utility>} \\

\tcode{<deque>} &
\tcode{<locale>} &
\tcode{<stack>} &
\tcode{<valarray>} \\

\tcode{<exception>} &
\tcode{<map>} &
\tcode{<stdexcept>} &
\tcode{<variant>} \\

\tcode{<execution>} &
\tcode{<memory>} &
\tcode{<streambuf>} & \\

\tcode{<filesystem>} &
\tcode{<memory_resource>} &
\tcode{<vector>} & \\

\tcode{<forward_list>} &
\tcode{<mutex>} &
\tcode{<string>} & \\
\end{floattable}

\setcounter{subsection}{3}
\rSec2[constraints]{Constraints on programs}

\setcounter{subsubsection}{7}
\rSec3[res.on.functions]{Other functions}
\ednote{Modify paragraph 2 as follows:}

\setcounter{Paras}{1}
\pnum
In particular, the effects are undefined in the following cases:

\begin{itemize}
\item
for replacement functions~(\cxxref{new.delete}), if the installed replacement function does not
implement the semantics of the applicable
\required
paragraph.
\item
for handler functions~(\cxxref{new.handler}, \cxxref{terminate.handler}),
if the installed handler function does not implement the semantics of the applicable
\required
paragraph
\item
for types used as template arguments when instantiating a template component,
if the operations on the type do not implement the semantics of the applicable
\emph{Requirements}
subclause~(\cxxref{allocator.requirements}, \cxxref{container.requirements}, \cxxref{iterator.requirements},
\cxxref{algorithms.requirements}, \cxxref{numeric.requirements}).
Operations on such types can report a failure by throwing an exception
unless otherwise specified.
\item
if any replacement function or handler function or destructor operation exits via an exception,
unless specifically allowed
in the applicable
\required
paragraph.
\item
if an incomplete type~(\cxxref{basic.types}) is used as a template
argument when instantiating a template component \added{or evaluating a concept}, unless specifically
allowed for that component.
\end{itemize}

\ednote{Add a new subclause after [res.on.required]:}
\begin{addedblock}
\setcounter{subsubsection}{11}
\rSec3[res.on.requirements]{Semantic requirements}
\pnum
If the semantic requirements of a declaration's
constraints~(\ref{structure.requirements}) are not satisfied at the point of
use, the program is ill-formed, no diagnostic required.
\end{addedblock}
