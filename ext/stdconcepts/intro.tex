\rSec0[intro]{Introduction}

Proposal \href{https://wg21.link/p0802r0}%
{P0802R0 "Applying Concepts to the Standard Library"} and the LEWG discussion
thereof captured in \href{https://wg21.link/p0872r0}%
{P0872R0 "Discussion Summary: Applying Concepts to the Standard Library"} call
for a proposal to insert the concepts library from the Ranges TS into the C++20
WD. This is that proposal.

The motivating discussion from P0802R0 suggests that the Ranges TS can provide a
basis of concepts for use in other library work, so we can avoid a string of
proposals that all define small variations on common ideas:
\begin{quote}
How can the C++ Concepts core language feature be best applied to the standard
library?

It seems clear that the basis for introducing concepts to the standard library
must be the Ranges TS. That paper encapsulates the committee's knowledge and
experience with fundamental library concepts and how these library concepts can
be applied to improve the existing standard library. The Ranges TS has been
implemented and exposed to the C++ community for several years; any other
approach would be pure invention and speculation.

The Ranges TS has two separable components: a library of fundamental concepts
(TS Clauses 6 and 7), and revisions of existing library components (TS Clauses
8-12, also known as STL2). The characteristics of these two components are quite
different, so they should be considered and adopted separately.
\end{quote}

This proposal includes the "library of fundamental concepts," the "revisions of
existing library components" are in the sister proposal P0896. Again quoting P0802R0:
\begin{quote}
\textbf{Recommendation: Fundamental Library Concepts}

Ranges TS clause 7 (Concepts library) should adopted by the C++20 WP as soon as
a proposal can be prepared and processed by LEWG/LWG. We recommend that Casey
Carter and Eric Niebler lead this effort and that they be given sufficient
authority to include other fundamental material from the Ranges TS.

\textbf{Rationale:} The fundamental concepts are mature and well-known, as they are
based on standard library requirements that have been developed and refined from
C++98 onward. Because concepts are an entirely new core language feature, these
fundamental concepts can be defined in the standard library without breaking any
existing C++ code (modulo the usual namespace caveats). Furthermore, failure to
standardize these fundamental concepts quickly is likely to result in
proliferation of similar but subtly different user-supplied concepts, often with
the same names. Confusion seems inevitable under such circumstances.
\end{quote}

This document proposes the following parts of the Ranges TS for inclusion in C++20:
\begin{itemize}
\item The Concepts library (Clause 7) to be defined in namespace \tcode{std} inside
  a new \tcode{<concepts>} header
\item Portions of the utilities library which do not break existing code: the
  \tcode{identity} function object, changes to \tcode{common_type}  and the
  addition of the \tcode{common_reference} type trait
\item The numerics library (which consists of only the
  \tcode{UniformRandomBitGenerator} concept)
\end{itemize}
Some of the library concepts introduced share the names of requirement tables
defined in [utility.arg.requirements]; the names of those requirement tables are
changed to "make way".

\rSec1[intro.history]{Revision History}
\rSec2[intro.history.r3]{Revision 3}
\begin{itemize}
\item Editorial changes per LWG review.
\end{itemize}

\rSec2[intro.history.r2]{Revision 2}
\begin{itemize}
\item Add a feature test macro.
\end{itemize}

\rSec2[intro.history.r1]{Revision 1}
\begin{itemize}
\item Fix typo in \tcode{common_type} wording due to editorial error
  incorporating the PR for Ranges issue \#506.
\item Strike mentions of namespace \tcode{std2} from the library introduction.
\item Reformulate concepts \tcode{Swappable} and \tcode{SwappableWith} in terms
  of the \tcode{is_swappable} and \tcode{is_swappable_with} type traits.
\item Strike the specification of the \tcode{std2::swap} customization point object.
\end{itemize}

\rSec1[intro.macro]{Feature testing recommendations}
An implementation that supports the concepts library defined herein may indicate
that support by defining the feature test macro \tcode{__cpp_lib_concepts}, with
the value thereof to be determined by the date that this proposal is adopted by
WG21.

\rSec1[intro.cpp98]{Renaming "requirements tables"}

\pnum
\ednote{Before applying the changes in the remainder of this specification,
prepend the prefix "Cpp98" to uses of the named requirement sets below, and
change them to \textit{italic} non code font (for example,
\textit{Cpp98Destructible}):}
\begin{itemize}
\item \tcode{Allocator}
\item \tcode{BasicLockable}
\item \tcode{BinaryTypeTrait}
\item \tcode{Clock}
\item \tcode{CopyAssignable}
\item \tcode{CopyConstructible}
\item \tcode{DefaultConstructible}
\item \tcode{Destructible}
\item \tcode{EqualityComparable}
\item \tcode{Hash}
\item \tcode{Iterator}
\item \tcode{LessThanComparable}
\item \tcode{Lockable}
\item \tcode{MoveAssignable}
\item \tcode{MoveConstructible}
\item \tcode{NullablePointer}
\item \tcode{TimedLockable}
\item \tcode{TransformationTrait}
\item \tcode{TrivialClock}
\item \tcode{UnaryTypeTrait}
\item \tcode{ValueSwappable}
\end{itemize}
This document reuses many of these names for concept definitions. We recommend
that future uses of CamelCase with \tcode{code font} in the library specification
be reserved for concepts. Note that the "italic non-code-font" convention is
already used for some named sets of requirements: the allocator-aware container
requirements. This change establishes a uniform convention for all non-concept
requirement sets.

\ednote{What about "swappable"/"swappable with"/"swappable requirements"?}

\rSec1[intro.style]{Style of presentation}

\pnum
The remainder of this document is a technical specification in the form of
editorial instructions directing that changes be made to the text of the C++
working draft. The formatting of the text suggests the origin of each portion of
the wording.

Existing wording from the C++ working draft - included to provide context - is
presented without decoration.

\begin{addedblock}
Entire clauses / subclauses / paragraphs incorporated from the ISO/IEC 21425:2017
(the "Ranges TS") are presented in a distinct cyan color.
\end{addedblock}

\added{In-line additions of wording from the Ranges TS to the C++ working draft
are presented in cyan with underline.}

\removed{In-line bits of wording to be struck from the C++ working draft are
presented in red with strike-through.}

\newtxt{Wording to be added which is original to this document appears in gold
with underline.}

\oldtxt{Wording from the Ranges TS which IS NOT to be added to the C++ working
draft is presented in magenta with strikethrough.}

Ideally, these formatting conventions make it clear which wording comes from
which document in this three-way merge.
