\rSec0[intro]{Scope/Intro/Design discussion}

\ednote{FIXME: replace.}

\pnum
This document describes extensions to the \Cpp
Programming Language that
permit operations on ranges of data. These extensions include
changes and additions to the existing library facilities as well
as the extension of one core language facility. In particular,
changes and extensions to the Standard Library include:

\begin{itemize}
\item The formulation of the foundational and iterator concept requirements
using the syntax of the Concepts TS.
\item Analogues of the Standard Library algorithms specified in terms of the new
concepts.
\item The loosening of the algorithm constraints to permit the use of
\techterm{sentinels} to denote the end of a range and corresponding changes to algorithm
return types where necessary.
\item The addition of new concepts describing \techterm{range} and \techterm{view}
abstractions; that is, objects with a begin iterator and an end sentinel.
\item New algorithm overloads that take range objects.
\item Support of \techterm{callable objects} (as opposed to \techterm{function objects})
passed as arguments to the algorithms.
\item The addition of optional \techterm{projection} arguments to the algorithms to
permit on-the-fly data transformations.
\item Analogues of the iterator primitives and new primitives in support of the
addition of sentinels to the library.
\item Constrained analogues of the standard iterator adaptors and stream iterators
that satisfy the new iterator concepts.
\item New iterator adaptors (\tcode{counted_iterator} and \tcode{common_iterator}) and
sentinels (\tcode{unreachable}).
\end{itemize}

\rSec1[intro.stl1]{Renaming "requirements tables"}

\pnum
\ednote{Before applying the changes in the remainder of this specification, prepend the prefix "STL1" to uses of the names below in the Standard Library clauses:}
\begin{itemize}
\item \tcode{EqualityComparable}
\item \tcode{DefaultConstructible}
\item \tcode{MoveConstructible}
\item \tcode{CopyConstructible}
\item \tcode{MoveAssignable}
\item \tcode{CopyAssignable}
\item \tcode{Destructible}
\end{itemize}
This document reuses these names for concept definitions.

\ednote{FIXME: "swappable"/"swappable with"/"swappable requirements" need special treatment.}

\rSec1[intro.style]{Style of presentation}

\pnum
The remainder of this document is a technical specification in the form of
editorial instructions directing that changes be made to the text of the C++
working draft. The formatting of the text suggests the origin of each portion of
the wording.

Existing wording from the C++ working draft - included to provide context - is
presented without decoration.

\begin{addedblock}
Entire clauses / subclauses / paragraphs incorporated from the ISO/IEC 21425:2017
(the "Ranges TS") are presented in a distinct cyan color.
\end{addedblock}

\added{In-line additions of wording from the Ranges TS to the C++ working draft
are presented in cyan with underline.}

\removed{In-line bits of wording to be struck from the C++ working draft are
presented in red with strike-through.}

\newtxt{Wording to be added which is original to this document appears in gold
with underline.}

\oldtxt{Wording from the Ranges TS which IS NOT to be added to the C++ working
draft is presented in magenta with strikethrough.}

Ideally, these formatting conventions make it clear which wording comes from
which document in this three-way merge.
