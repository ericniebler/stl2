%!TEX root = stdconcepts.tex
\setcounter{chapter}{22}
\rSec0[utilities]{General utilities library}

\setcounter{section}{1}
\rSec1[utility]{Utility components}

\setcounter{section}{13}
\rSec1[function.objects]{Function objects}

\ednote{Add a new declaration to the \tcode{<functional>} synopsis:}
\rSec2[functional.syn]{Header \tcode{<functional>} synopsis}

\begin{codeblock}
  [...]
  template<> struct bit_xor<void>;
  template<> struct bit_not<void>;

  @\added{// \ref{func.identity}, identity:}@
  @\added{struct identity;}@

  // \cxxref{func.notfn}, function template not_fn
  template<class F> unspecified not_fn(F&& f);

  [...]
\end{codeblock}

\ednote{Add a new subclause before [func.not_fn]:}
\begin{addedblock}
\setcounter{subsection}{9}
\rSec2[func.identity]{Class \tcode{identity}}

\indexlibrary{\idxcode{identity}}%
\begin{itemdecl}
struct identity {
  template <class T>
    constexpr T&& operator()(T&& t) const noexcept;

  using is_transparent = @\unspec@;
};

template <class T>
  constexpr T&& operator()(T&& t) const noexcept;
\end{itemdecl}

\begin{itemdescr}
\pnum
\oldtxt{\returns}\newtxt{\effects
Equivalent to: \tcode{return}} \tcode{std::forward<T>(t);}
\end{itemdescr}
\end{addedblock}

\rSec1[meta]{Metaprogramming and type traits}

\setcounter{subsection}{1}
\rSec2[meta.type.synop]{Header \tcode{<type_traits>} synopsis}
\ednote{Add new declarations to the \tcode{<type_traits>} synopsis:}
\begin{codeblock}
  [...]
  template <class... T> struct common_type;
  @\added{template <class T, class U, template <class> class TQual, template <class> class UQual>}@
    @\added{struct basic_common_reference \{ \};}@
  @\added{template <class... T> struct common_reference;}@
  template<class T> struct underlying_type;
  [...]
  template <class... T>
    using common_type_t = typename common_type<T...>::type;
  @\added{template <class... T>}@
    @\added{using common_reference_t = typename common_reference<T...>::type;}@
  template<class T>
    using underlying_type_t = typename underlying_type<T>::type;
  [...]
\end{codeblock}

\setcounter{subsection}{7}
\setcounter{subsubsection}{5}
\setcounter{table}{49}
\rSec3[meta.trans.other]{Other transformations}

\ednote{Add new traits to Table~\ref{tab:type-traits.other}}

\begin{libreqtab2a}{Other transformations}{tab:type-traits.other}
\\ \topline
\lhdr{Template} &   \rhdr{Comments} \\ \capsep
\endfirsthead
\continuedcaption\\
\topline
\lhdr{Template} &   \rhdr{Comments} \\ \capsep
\endhead

... & ... \\ \rowsep

\tcode{template<class... T>} \tcode{struct common_type;}
 &
 Unless this trait is specialized (as specified in Note B, below),
 the member \tcode{type} shall be defined or omitted as specified in Note A, below.
 If it is omitted, there shall be no member \tcode{type}.
 Each type in the parameter pack \tcode{T} shall be
 complete, \cv{}~\tcode{void}, or an array of unknown bound. \\ \rowsep

\tcode{\added{template <class, class,}}\br
 \tcode{\added{  template <class> class,}}\br
 \tcode{\added{  template <class> class>}}\br
 \tcode{\added{struct}}\br
 \tcode{\added{  basic_common_reference;}}
 &
 \added{\oldtxt{The primary template shall have no member typedef
 \tcode{type}.} A program may specialize this trait if at least one
 template parameter in the specialization depends on a user-defined
 type. In such a specialization, a member typedef \tcode{type} may be
 defined or omitted. If it is omitted, there shall be no member
 \tcode{type}.
 \enternote Such specializations may be used to influence
 the result of }\tcode{\added{common_reference}}\added{.\exitnote} \\ \rowsep

\added{\tcode{template <class... T>}} \added{\tcode{struct common_reference;}}
 &
 \added{The member typedef \tcode{type} shall be defined or omitted
 as specified below. If it is omitted, there shall be no member
 \tcode{type}. Each type in the parameter pack \tcode{T} shall be
 complete or (possibly \cv) \tcode{void}.} \\ \rowsep

... & ... \\

\end{libreqtab2a}

\ednote{Insert this new paragraph before paragraph 3:}

\begin{addedblock}
\setcounter{Paras}{2}
\indexlibrary{\idxcode{common_type}}%
\pnum
Let \tcode{CREF(A)} be \tcode{add_lvalue_reference_t<const remove_reference_t<A>{}>}.
Let \tcode{XREF(A)} denote a unary template \tcode{T} such that
  \tcode{\oldtxt{T<remove_cvref_t<A>{}>}}\oldtxt{ denotes the same type as \tcode{A}}
  \tcode{\newtxt{T<U>}}\newtxt{ denotes the same type as \tcode{U} with the addition
  of \tcode{A}'s cv and reference qualifiers, for a type \tcode{U} such
  that }\tcode{\newtxt{is_same_v<U, remove_cvref_t<U>{}>}}\newtxt{ is \tcode{true}.}
Let \tcode{COPYCV(FROM, TO)} be an alias for type \tcode{TO} with the addition of
  \tcode{FROM}'s top-level cv-qualifiers.
  \enterexample \tcode{COPYCV(const int, volatile short)} is an alias for
  \tcode{const volatile short}. \exitexample
\oldtxt{Let \tcode{RREF_RES(Z)} be \tcode{remove_reference_t<Z>\&\&} if \tcode{Z} is a
  reference type or \tcode{Z} otherwise.}
Let \tcode{COND_RES(X, Y)} be
  \tcode{decltype(declval<bool>() ? declval<X(\&)()>()() : declval<Y(\&)()>()())}.
Given types \tcode{A} and \tcode{B}, let \tcode{X} be \tcode{remove_reference_t<A>},
 let \tcode{Y} be \tcode{remove_reference_t<B>}, and let \tcode{COMMON_REF(A, B)} be:
\begin{itemize}
\item If \tcode{A} and \tcode{B} are both lvalue reference types,
  \tcode{COMMON_REF(A, B)} is
  \tcode{COND_RES(COPYCV(X, Y) \&, COPYCV(Y, X) \&)} \newtxt{if that type exists
  and is a reference type}.
\item Otherwise, let \tcode{C} be
  \tcode{\oldtxt{RREF_RES(COMMON_REF(X\&, Y\&))}}
  \tcode{\newtxt{remove_reference_t<COMMON_REF(X\&, Y\&)>\&\&}}.
  If \tcode{A} and \tcode{B} are both rvalue reference types,
  \tcode{C} is well-formed, and
  \tcode{is_convertible_v<A, C> \&\& is_convertible_v<B, C>} is \tcode{true},
  then \tcode{COMMON_REF(A, B)} is \tcode{C}.
\item Otherwise, let \tcode{D} be
  \tcode{COMMON_REF(const X\&, Y\&)}. If \tcode{A} is an rvalue
  reference and \tcode{B} is an lvalue reference and \tcode{D} is
  well-formed and \tcode{is_convertible_v<A, D>} is
  \tcode{true}, then \tcode{COMMON_REF(A, B)} is \tcode{D}.
\item Otherwise, if \tcode{A} is an lvalue reference and \tcode{B}
  is an rvalue reference, then \tcode{COMMON_REF(A, B)} is
  \tcode{COMMON_REF(B, A)}.
\item Otherwise, \tcode{COMMON_REF(A, B)} is
  \oldtxt{\tcode{decay_t<COND_RES(CREF(A), CREF(B))>}} \newtxt{ill-formed}.
\end{itemize}

If any of the types computed above are ill-formed, then
\tcode{COMMON_REF(A, B)} is ill-formed.
\end{addedblock}

\ednote{Modify the following "Note A" paragraph as follows:}

\pnum
Note A: For the \tcode{common_type} trait applied to a parameter pack
\tcode{T} of types, the member \tcode{type} shall be either defined or not
present as follows:

\begin{itemize}
\item If \tcode{sizeof...(T)} is zero, there shall be no member \tcode{type}.

\item If \tcode{sizeof...(T)} is one, let \tcode{T0} denote the sole type
constituting the pack \tcode{T}.
The member \grammarterm{typedef-name} \tcode{type} shall denote the same
type, if any, as \tcode{common_type_t<T0, T0>};
otherwise there shall be no member \tcode{type}.

\item If \tcode{sizeof...(T)} is two,
let the first and second types constituting \tcode{T} be denoted
by \tcode{T1} and \tcode{T2}, respectively, and
let \tcode{D1} and \tcode{D2} denote
the same types as \tcode{decay_t<T1>} and \tcode{decay_t<T2>}, respectively.
  \begin{itemize}
  \item If \tcode{is_same_v<T1, D1>} is \tcode{false} or
     \tcode{is_same_v<T2, D2>} is \tcode{false},
     let \tcode{C} denote the same type, if any, as \tcode{common_type_t<D1, D2>}.

  \item
    \added{\enternote None of the following will apply if there is a
    specialization }\tcode{\added{common_type<D1, D2>}}\added{. \exitnote}

  \item Otherwise, \removed{let \tcode{C} denote the same type, if any, as} \added{if}
\begin{codeblock}
decay_t<decltype(false ? declval<D1>() : declval<D2>())>
\end{codeblock}
    \removed{\enternote This will not apply if there is a specialization \tcode{common_type<D1, D2>}. \exitnote}
    \added{denotes a valid type, let \tcode{C} denote its type.}

  \item \added{Otherwise, let \tcode{C} denote the same type
    as }\tcode{\added{decay_t<COND_RES(CREF(D1), CREF(D2))>}}\added{, if any.}
  \end{itemize}
In either case, the member \grammarterm{typedef-name} \tcode{type} shall denote
the same type, if any, as \tcode{C}.
Otherwise, there shall be no member \tcode{type}.

\item If \tcode{sizeof...(T)} is greater than two,
let \tcode{T1}, \tcode{T2}, and \tcode{R}, respectively,
denote the first, second, and (pack of) remaining types constituting \tcode{T}.
Let \tcode{C} denote the same type, if any, as \tcode{common_type_t<T1, T2>}.
If there is such a type \tcode{C}, the member \grammarterm{typedef-name} \tcode{type}
shall denote the same type, if any, as \tcode{common_type_t<C, R...>}.
Otherwise, there shall be no member \tcode{type}.
\end{itemize}

\ednote{Add new paragraphs following the paragaph that begins "Note B":}

\begin{addedblock}
\pnum
For the \tcode{common_reference} trait applied to a parameter pack \tcode{T} of
types, the member \tcode{type} shall be either defined or not present as follows:
\begin{itemize}
\item If \tcode{sizeof...(T)} is zero, there shall be no member \tcode{type}.

\item Otherwise, if \tcode{sizeof...(T)} is one, let \tcode{T0} denote the sole
  type in the pack \tcode{T}. The member typedef \tcode{type} shall denote the
  same type as \tcode{T0}.

\item Otherwise, if \tcode{sizeof...(T)} is two, let \tcode{T1} and \tcode{T2}
  denote the two types in the pack \tcode{T}. Then
  \begin{itemize}
  \item If \tcode{T1} and \tcode{T2} are reference types and
    \tcode{COMMON_REF(T1, T2)} is well-formed \oldtxt{and denotes a reference type}
    then the member typedef \tcode{type} denotes that type.

  \item Otherwise, if \tcode{basic_common_reference<remove_cvref_t<T1>, remove_cvref_t<T2>,
    XREF(T1), XREF(T2)>::type} is well-formed, then the member typedef
    \tcode{type} denotes that type.

  \item Otherwise, if \tcode{COND_RES(T1, T2)} is well-formed, then the
    member typedef \tcode{type} denotes that type.

  \item Otherwise, if \tcode{common_type_t<T1, T2>} is well-formed, then the
    member typedef \tcode{type} denotes that type.

  \item Otherwise, there shall be no member \tcode{type}.
  \end{itemize}

\item Otherwise, if \tcode{sizeof...(T)} is greater than two, let \tcode{T1},
  \tcode{T2}, and \tcode{Rest}, respectively, denote the first, second, and
  (pack of) remaining types comprising \tcode{T}. Let \tcode{C} be the type
  \tcode{common_reference_t<T1, T2>}. Then:
  \begin{itemize}
  \item If there is such a type \tcode{C}, the member typedef \tcode{type} shall
    denote the same type, if any, as \tcode{common_reference_t<C, Rest...>}.

  \item Otherwise, there shall be no member \tcode{type}.
  \end{itemize}
\end{itemize}

\pnum
Notwithstanding the provisions of \cxxref{meta.type.synop}, and
pursuant to \cxxref{namespace.std},
a program may specialize \tcode{basic_common_reference<T, U, TQual, UQual>}
for types \tcode{T} and \tcode{U} such that
\tcode{is_same_v<T, decay_t<T>{>}} and
\tcode{is_same_v<U, decay_t<U>{>}} are each \tcode{true}.
\enternote Such specializations are needed when only explicit conversions
are desired between the template arguments. \exitnote
Such a specialization need not have a member named \tcode{type},
but if it does, that member shall be a \grammarterm{typedef-name}
for an accessible and unambiguous type \tcode{C}
to which each of the types \tcode{TQual<T>} and \tcode{UQual<U>} is convertible.
Moreover, \tcode{basic_common_reference<T, U, TQual, UQual>::type} shall denote
the same type, if any, as does \tcode{basic_common_reference<U, T, UQual, TQual>::type}.
A program may not specialize \tcode{basic_common_reference} on the third or
fourth parameters, \tcode{TQual} or \tcode{UQual}. No diagnostic is required for
a violation of these rules.
\end{addedblock}
