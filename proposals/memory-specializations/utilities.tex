%!TEX root = std.tex
\setcounter{chapter}{22}
\rSec0[utilities]{General utilities library}

\setcounter{section}{9}
\rSec1[memory]{Memory}
\setcounter{subsection}{1}
\rSec2[memory.syn]{Header \tcode{<\added{ranges/}memory>} synopsis}

\begin{codeblock}
// \ref{specialized.algorithms} specialized algorithms:
\end{codeblock}
{\color{addclr}
\begin{codeblock}
template <class I>
concept @\xname{NoThrowInputIterator}@ = @\seebelow@; // \expos

template <class S, class I>
concept @\xname{NoThrowSentinel}@ = @\seebelow@; // \expos

template <class Rng>
concept @\xname{NoThrowInputRange}@ = @\seebelow@; // \expos

template <class I>
concept @\xname{NoThrowForwardIterator}@ = @\seebelow@; // \expos

template <class Rng>
concept @\xname{NoThrowForwardRange}@ = @\seebelow@; // \expos
\end{codeblock}
} %% \color{addclr}

{\color{remclr}
\begin{codeblock}
template <class ForwardIterator>
  void uninitialized_default_construct(ForwardIterator first, ForwardIterator last);
template <class ForwardIterator, class Size>
  ForwardIterator uninitialized_default_construct_n(ForwardIterator first, Size n);
\end{codeblock}
} %% \color{remclr}
{\color{addclr}
\begin{codeblock}
template <@\xname{NoThrowForwardIterator}@ I, Sentinel<I> S>
  requires
DefaultConstructible<value_type_t<I>>
  I uninitialized_default_construct(I first, S last);

template <@\xname{NoThrowForwardRange}@ Rng>
  requires
DefaultConstructible<value_type_t<iterator_t<Rng>>>
  safe_iterator_t<Rng> uninitialized_default_construct(Rng&& rng);

template <@\xname{NoThrowForwardIterator}@ I>
  requires
DefaultConstructible<value_type_t<I>>
  I uninitialized_default_construct_n(I first, difference_type_t<I> n);
\end{codeblock}
} %% \color{addclr}

{\color{remclr}
\begin{codeblock}
template <class ForwardIterator>
  void uninitialized_value_construct(ForwardIterator first, ForwardIterator last);
template <class ForwardIterator, class Size>
  ForwardIterator uninitialized_value_construct_n(ForwardIterator first, Size n);
\end{codeblock}
} %% \color{remclr}
{\color{addclr}
\begin{codeblock}
template <@\xname{NoThrowForwardIterator}@ I, Sentinel<I> S>
  requires
DefaultConstructible<value_type_t<I>>
  I uninitialized_value_construct(I first, S last);

template <@\xname{NoThrowForwardRange}@ Rng>
  requires
DefaultConstructible<value_type_t<iterator_t<Rng>>>
  safe_iterator_t<Rng> uninitialized_value_construct(Rng&& rng);

template <@\xname{NoThrowForwardIterator}@ I>
  requires
DefaultConstructible<value_type_t<I>>
  I uninitialized_value_construct_n(I first, difference_type_t<I> n);
\end{codeblock}
} %% \color{addclr}

{\color{remclr}
\begin{codeblock}
template <class InputIterator, class ForwardIterator>
  ForwardIterator uninitialized_copy(InputIterator first, InputIterator last,
                                     ForwardIterator result);
template <class InputIterator, class Size, class ForwardIterator>
  ForwardIterator uninitialized_copy_n(InputIterator first, Size n,
                                       ForwardIterator result);
\end{codeblock}
} %% \color{remclr}
{\color{addclr}
\begin{codeblock}
template <InputIterator I, Sentinel<I> S1, @\xname{NoThrowForwardIterator}@ O, @\xname{NoThrowSentinel}@<O> S2>
  requires
Constructible<value_type_t<O>, reference_t<I>>
  tagged_pair<tag::in(I), tag::out(O)>
uninitialized_copy(I ifirst, S1 ilast, O ofirst, S2 olast);

template <InputRange IRng, @\xname{NoThrowForwardRange}@ ORng>
  requires
Constructible<value_type_t<iterator_t<ORng>>, reference_t<iterator_t<IRng>>>
  tagged_pair<tag::in(safe_iterator_t<IRng>), tag::out(safe_iterator_t<ORng>)>
uninitialized_copy(IRng&& irng, ORng&& orng);

template <InputIterator I, @\xname{NoThrowForwardIterator}@ O>
  requires
Constructible<value_type_t<O>, reference_t<I>>
  tagged_pair<tag::in(I), tag::out(O)>
uninitialized_copy_n(I first, difference_type_t<I> n, O out)
\end{codeblock}
} %% \color{addclr}

{\color{remclr}
\begin{codeblock}
template <class InputIterator, class ForwardIterator>
  ForwardIterator uninitialized_move(InputIterator first, InputIterator last,
                                     ForwardIterator result);
template <class InputIterator, class Size, class ForwardIterator>
  pair<InputIterator, ForwardIterator>
    uninitialized_move_n(InputIterator first, Size n, ForwardIterator result);
\end{codeblock}
} %% \color{remclr}
{\color{addclr}
\begin{codeblock}
template <InputIterator I, Sentinel<I> S1, @\xname{NoThrowForwardIterator}@ O, @\xname{NoThrowSentinel}@<O> S2>
requires
  Constructible<value_type_t<O>, rvalue_reference_t<I>>
tagged_pair<tag::in(I), tag::out(O)>
  uninitialized_move(I ifirst, S1 ilast, O ofirst, S2 olast);

template <InputRange IRng, @\xname{NoThrowForwardRange}@ ORng>
  requires
Constructible<value_type_t<iterator_t<ORng>>, rvalue_reference_t<iterator_t<IRng>>>
  tagged_pair<tag::in(safe_iterator_t<IRng>), tag::out(safe_iterator_t<ORng>)>
uninitialized_move(IRng&& irng, ORng&& orng);

template <InputIterator I, @\xname{NoThrowForwardIterator}@ O>
  requires
Constructible<value_type_t<O>, rvalue_reference_t<I>>
  tagged_pair<tag::in(I), tag::out(O)>
uninitialized_move_n(I first, difference_type_t<I> n, O result);
\end{codeblock}
} %% \color{addclr}

{\color{remclr}
\begin{codeblock}
template <class ForwardIterator, class T>
  void uninitialized_fill(ForwardIterator first, ForwardIterator last,
                          const T& x);
template <class ForwardIterator, class Size, class T>
  ForwardIterator uninitialized_fill_n(ForwardIterator first, Size n, const T& x);
\end{codeblock}
} %% \color{remclr}
{\color{addclr}
\begin{codeblock}
template <@\xname{NoThrowForwardIterator}@ I, Sentinel<I> S, typename T>
  requires
Constructible<value_type_t<I>, const T&>
  I uninitialized_fill(I first, S last, const T& x);

template <@\xname{NoThrowForwardRange}@ Rng, typename T>
  requires
Constructible<value_type_t<iterator_t<Rng>>, const T&>
  safe_iterator_t<Rng> uninitialized_fill(Rng&& rng, const T& x);

template <@\xname{NoThrowForwardIterator}@ I, typename T>
  requires
Constructible<value_type_t<I>, const T&>
  I uninitialized_fill_n(I first, const difference_type_t<I> n, const T& x);
\end{codeblock}
} %% \color{addclr}
{\color{remclr}
\begin{codeblock}
template <class T>
  void destroy_at(T* location);
template <class ForwardIterator>
  void destroy(ForwardIterator first, ForwardIterator last);
template <class ForwardIterator, class Size>
  ForwardIterator destroy_n(ForwardIterator first, Size n);
\end{codeblock}
} %% \color{remclr}
{\color{addclr}
\begin{codeblock}
template <Destructible T>
  void destroy_at(T* location) noexcept;

template <@\xname{NoThrowInputIterator}@ I, @\xname{NoThrowSentinel}@<I> S>
  requires
Destructible<value_type_t<I>>
  I destroy(I first, S last) noexcept;

template <@\xname{NoThrowRange}@ Rng>
  requires
Destructible<value_type_t<iterator_t<Rng>>
  safe_iterator_t<Rng> destroy(Rng&& rng) noexcept;

template <@\xname{NoThrowInputIterator} I>
  requires
Destructible<value_type_t<I>>
  I destroy_n(I first, difference_type_t<I> n) noexcept;
\end{codeblock}
} %% \color{addclr}

\setcounter{subsection}{9}
\rSec2[specialized.algorithms]{Specialized algorithms}
\setcounter{Paras}{0}
\pnum
Throughout this sub-clause, the names of template parameters are used to express type requirements.
{\color{remclr}
\begin{itemize}
\item If an algorithm's template parameter is named InputIterator, the template argument shall satisfy the
requirements of an input iterator ~(\ref{input.iterators}).

\item If an algorithm's template parameter is named ForwardIterator, the template argument shall satisfy the
requirements of a forward iterator ~(\ref{forward.iterators}), and is required to have the property that no exceptions
are thrown from increment, assignment, comparison, or indirection through valid iterators.
\end{itemize}
} %% \color{remclr}

{\color{addclr}
\setcounter{Paras}{0}
\pnum This section defines the following expositional concepts:

\begin{itemdecl}
template <class I>
concept @\xname{NoThrowInputIterator}@ = // \expos
  InputIterator<I> &&
  is_lvalue_reference_v<reference_t<I>> &&
  Same<remove_cv_ref_t<reference_t<I>>, value_type_t<I>>;
\end{itemdecl}

\begin{itemdescr}
\pnum No exceptions are thrown from increment, copy, move, assignment, or indirection through valid
iterators.

\pnum
\enternote The distinction between \tcode{InputIterator} and
\tcode{\xname{NoThrowInputIterator}} is purely semantic.\exitnote
\end{itemdescr}

\begin{itemdecl}
template <class S, class I>
concept @\xname{NoThrowSentinel}@ = // \expos
  Sentinel<S, I>;
\end{itemdecl}

\begin{itemdescr}
\pnum No exceptions are thrown from comparisons between objects of type \tcode{I} and \tcode{S}.

\pnum
\enternote The distinction between \tcode{Sentinel} and
\tcode{\xname{NoThrowSentinel}} is purely semantic.\exitnote
\end{itemdescr}

\begin{itemdecl}
template <class Rng>
concept @\xname{NoThrowInputRange}@ = // \expos
  Range<Rng> &&
  @\xname{NoThrowInputIterator}@<iterator_t<Rng>> &&
  @\xname{NoThrowSentinel}@<sentinel_t<Rng>, iterator_t<Rng>>;
\end{itemdecl}

\begin{itemdescr}
\pnum No exceptions are thrown from calls to \tcode{begin} and \tcode{end} on an object of type
\tcode{Rng}.

\pnum
\enternote The distinction between \tcode{InputRange} and
\tcode{\xname{NoThrowInputRange}} is purely semantic.\exitnote
\end{itemdescr}

\begin{itemdecl}
template <class I>
concept @\xname{NoThrowForwardIterator}@ = // \expos
  @\xname{NoThrowInputIterator}@<I> &&
  @\xname{NoThrowSentinel@}<I, I> &&
  ForwardIterator<I>;
\end{itemdecl}

\begin{itemdescr}
\pnum
\enternote The distinction between \tcode{ForwardIterator} and
\tcode{\xname{NoThrowForwardIterator}} is purely semantic.\exitnote
\end{itemdescr}

\begin{itemdecl}
template <class Rng>
concept @\xname{NoThrowForwardRange}@ = // \expos
  @\xname{NoThrowForwardIterator}@<iterator_t<Rng>> &&
  @\xname{NoThrowInputRange}@<Rng> &&
  ForwardRange<Rng>;
\end{itemdecl}

\begin{itemdescr}
\pnum
\enternote The distinction between \tcode{ForwardRange} and
\tcode{\xname{NoThrowForwardRange}} is purely semantic.\exitnote
\end{itemdescr}
} %% \color{addclr}

Unless otherwise specified, if an exception is thrown in the following algorithms there are no effects.
\rSec3[uninitialized.construct.default]{\tcode{uninitialized_default_construct}}
{\color{remclr}
\begin{codeblock}
template <class ForwardIterator>
  void uninitialized_default_construct(ForwardIterator first, ForwardIterator last);
\end{codeblock}
} %% \color{remclr}

{\color{addclr}
\begin{codeblock}
template <@\xname{NoThrowForwardIterator}@ I, Sentinel<I> S>
  requires
DefaultConstructible<value_type_t<I>>
  I uninitialized_default_construct(I first, S last);
\end{codeblock}
} %% \color{addclr}

\setcounter{Paras}{0}
\pnum
\effects Equivalent to:
\begin{codeblock}
    for (; first != last; ++first)
      ::new @\added{(const_cast<void*>}@(static_cast<@\added{const volatile }@void*>(addressof(*first)))@\added{)}@
        @\changed{typename iterator_traits<ForwardIterator>::value_type}{remove_reference_t<reference_t<I{>}>}@;
    @\added{return first;}@
\end{codeblock}

\ednote{\tcode{const_cast<void*>} is necessary to ensure that \tcode{::operator new<void*>} (`True
Placement New') is called. The decision to cast \tcode{const}-ness away after calling
\tcode{addressof} is an alternative to preventing users from being unable to pass ranges that are
non-\tcode{const}.\\
\\
When \tcode{addressof(*i)} returns a \tcode{const T*}, this will not convert to \tcode{void*}, and
so no suitable overload of the True Placement New is found.\\
\\
It is noted that \tcode{const}-qualified means `do not modify', and that the \tcode{const_cast}
ignores this (and is thus lying). However, these algorithms are also claiming that they iterate
over objects of type \tcode{T}: nary a \tcode{T} is in the range. We present this as `the objects
in this range should be \tcode{const}', rather than `the memory here is \tcode{const}'.}

{\color{addclr}
\begin{codeblock}
template <@\xname{NoThrowForwardRange}@ Rng>
  requires
DefaultConstructible<value_type_t<iterator_t<Rng>>>
  safe_iterator_t<Rng> uninitialized_default_construct(Rng&& rng);
\end{codeblock}

\pnum
\effects Equivalent to:
\begin{codeblock}
        return uninitialized_default_construct(begin(rng), end(rng));
\end{codeblock}
} %% \color{addclr}

{\color{remclr}
\begin{codeblock}
template <class ForwardIterator, class Size>
  ForwardIterator uninitialized_default_construct_n(ForwardIterator first, Size n);
\end{codeblock}

\setcounter{Paras}{1}
\pnum
\effects Equivalent to:
\begin{codeblock}
        for (; n>0; (void)++first, --n)
          ::new (static_cast<void*>(addressof(*first)))
            typename iterator_traits<ForwardIterator>::value_type;
        return first;
\end{codeblock}
} %% \color{remclr}

{\color{addclr}

\begin{codeblock}
template <@\xname{NoThrowForwardIterator}@ I>
  requires
DefaultConstructible<value_type_t<I>>
  I uninitialized_default_construct_n(I first, difference_type_t<I> n);
\end{codeblock}

\pnum
\effects Equivalent to:
\begin{codeblock}
    return uninitialized_default_construct(make_counted_iterator(first, n),
      default_sentinel{}).base();
\end{codeblock}
} %% \color{addclr}

\rSec3[uninitialized.construct.value]{\tcode{uninitialized_value_construct}}
{\color{remclr}
\begin{codeblock}
template <class ForwardIterator>
  void uninitialized_value_construct(ForwardIterator first, ForwardIterator last);
\end{codeblock}
} %% \color{remclr}
{\color{addclr}
\begin{codeblock}
template <@\xname{NoThrowForwardIterator}@ I, Sentinel<I> S>
  requires
DefaultConstructible<value_type_t<I>>
  I uninitialized_value_construct(I first, S last);
\end{codeblock}
} %% \color{addclr}

\setcounter{Paras}{0}
\pnum
\effects Equivalent to:

\begin{codeblock}
    for (; first != last; ++first)
      ::new @\added{(const_cast<void*>}@(static_cast<@\added{const volatile }@void*>(addressof(*first)))@\added{)}@
        @\changed{typename iterator_traits<ForwardIterator>::value_type}{remove_reference_t<reference_t<I{>}>}@();
    @\added{return first;}@
\end{codeblock}

{\color{addclr}
\begin{codeblock}
template <@\xname{NoThrowForwardRange}@ Rng>
requires
  DefaultConstructible<value_type_t<iterator_t<Rng>>>
safe_iterator_t<Rng> uninitialized_value_construct(Rng&& rng);
\end{codeblock}

\pnum
\effects Equivalent to:
\begin{codeblock}
    return uninitialized_value_construct(begin(rng), end(rng));
\end{codeblock}
} %% \color{addclr}

{\color{remclr}
\begin{codeblock}
template <class ForwardIterator, class Size>
  ForwardIterator uninitialized_value_construct_n(ForwardIterator first, Size n);
\end{codeblock}
\setcounter{Paras}{1}
\pnum
\effects Equivalent to:
\begin{codeblock}
        for (; n>0; (void)++first, --n)
          ::new (static_cast<void*>(addressof(*first)))
            typename iterator_traits<ForwardIterator>::value_type();
        return first;
\end{codeblock}
} %% \color{remclr}

{\color{addclr}
\begin{codeblock}
template <@\xname{NoThrowForwardIterator}@ I>
  requires
DefaultConstructible<value_type_t<I>>
  I uninitialized_value_construct_n(I first, difference_type_t<I> n);
\end{codeblock}

\pnum
\effects Equivalent to:
\begin{codeblock}
    return uninitialized_value_construct(make_counted_iterator(first, n),
                                         default_sentinel{}).base();
\end{codeblock}
} %% \color{addclr}

\rSec3[uninitialized.copy]{\tcode{uninitialized_copy}}
{\color{remclr}
\begin{codeblock}
template <class InputIterator, class ForwardIterator>
  ForwardIterator uninitialized_copy(InputIterator first, InputIterator last,
                                     ForwardIterator result);
\end{codeblock}

\setcounter{Paras}{0}
\pnum
\effects As if by:
\begin{codeblock}
    for (; first != last; ++result, (void)++first)
      ::new (static_cast<void*>(addressof(*result)))
        typename iterator_traits<ForwardIterator>::value_type;
\end{codeblock}

\setcounter{Paras}{1}
\pnum
\returns \tcode{result}
} %% \color{remclr}

{\color{addclr}
\begin{codeblock}
template <InputIterator I, Sentinel<I> S1, @\xname{NoThrowForwardIterator}@ O, @\xname{NoThrowSentinel}@<O> S2>
  requires
Constructible<value_type_t<O>, reference_t<I>>
  tagged_pair<tag::in(I), tag::out(O)>
uninitialized_copy(I ifirst, S1 ilast, O ofirst, S2 olast);
\end{codeblock}

\setcounter{Paras}{0}
\pnum
\effects Equivalent to:
\begin{codeblock}
    for (; ifirst != ilast && ofirst != olast; ++ofirst, (void)++ifirst) {
      ::new (const_cast<void*>(static_cast<const volatile void*>(addressof(*ofirst))))
        remove_reference_t<reference_t<O>>(*ifirst);
    }
    return {ifirst, ofirst};
\end{codeblock}

\pnum
\requires \tcode{result} shall not be in the range \range{first}{last}.

\begin{codeblock}
template <InputRange IRng, @\xname{NoThrowForwardRange}@ ORng>
requires
  Constructible<value_type_t<iterator_t<ORng>>, reference_t<iterator_t<IRng>>>
tagged_pair<tag::in(safe_iterator_t<IRng>), tag::out(safe_iterator_t<ORng>O)>
  uninitialized_copy(IRng&& irng, ORng&& orng);
\end{codeblock}

\pnum
\effects Equivalent to:
\begin{codeblock}
    return uninitialized_copy(begin(irng), end(irng), begin(orng), end(orng));
\end{codeblock}

\pnum
\requires \tcode{result} shall not be in the range \range{begin(rng)}{end(rng))}.
} %% \color{addclr}

{\color{remclr}
\begin{codeblock}
template <class InputIterator, class Size, class ForwardIterator>
  ForwardIterator uninitialized_copy_n(InputIterator first, Size n,
                                       ForwardIterator result);
\end{codeblock}

\setcounter{Paras}{2}
\pnum
\effects As if by:
\begin{codeblock}
        for ( ; n > 0; ++result, (void) ++first, --n) {
          ::new (static_cast<void*>(addressof(*result)))
            typename iterator_traits<ForwardIterator>::value_type(*first);
        }
\end{codeblock}

\pnum
\returns \tcode{result}
} %% \color{remclr}

{\color{addclr}
\begin{codeblock}
template <InputIterator I, @\xname{NoThrowForwardIterator}@ O>
  requires
Constructible<value_type_t<O>, reference_t<I>>
  tagged_pair<tag::in(I), tag::out(O)> uninitialized_copy_n(I first, difference_type_t<I> n, O out);
\end{codeblock}

\setcounter{Paras}{4}
\pnum
\effects Equivalent to:
\begin{codeblock}
    auto t = uninitialized_copy(make_counted_iterator(first, n), default_sentinel{}).base();
    return {t.in().base(), t.out()};
\end{codeblock}

\pnum
\requires \tcode{result} shall not be in the range \range{first}{next(first, n)}.
} %% \color{addclr}

\rSec3[uninitialized.move]{\tcode{uninitialized_move}}
{\color{remclr}
\begin{codeblock}
template <class InputIterator, class ForwardIterator>
  ForwardIterator uninitialized_move(InputIterator first, InputIterator last,
                                     ForwardIterator result);
\end{codeblock}

\setcounter{Paras}{0}
\pnum
\effects Equivalent to:
\begin{codeblock}
    for (; first != last; (void)++result, ++first)
      ::new (static_cast<void*>(addressof(*result)))
        typename iterator_traits<ForwardIterator>::value_type(std::move(*first));
    return result;
\end{codeblock}
} %% \color{remclr}
{\color{addclr}
\begin{codeblock}
template <InputIterator I, Sentinel<I> S1, @\xname{NoThrowForwardIterator}@ O, @\xname{NoThrowSentinel}@<O> S2>
  requires
Constructible<value_type_t<O>, rvalue_reference_t<I>>
  tagged_pair<tag::in(I), tag::out(O)>
uninitialized_move(I ifirst, S1 ilast, O ofirst, S2 olast);
\end{codeblock}

\setcounter{Paras}{0}
\pnum
\effects Equivalent to:
\begin{codeblock}
    for (; ifirst != ilast && ofirst != olast; ++ofirst, (void)++ifirst) {
      ::new (const_cast<void*>(static_cast<const volatile void*>(addressof(*ofirst))))
        remove_reference_t<reference_t<O>>(std::move(*ifirst));
    }
    return {ifirst, ofirst};
\end{codeblock}
} %% color\addclr

\pnum
\remarks If an exception is thrown, some objects in the range \range{first}{last} are left in a valid but
unspecified state.

{\color{addclr}
\pnum
\requires \range{first}{last} shall not be in the range \range{first}{last}.

\begin{codeblock}
template <InputRange IRng, @\xname{NoThrowForwardRng}@ ORng>
  requires
Constructible<value_type_t<iterator_t<ORng>>, rvalue_reference_t<iterator_t<IRng>>>
  tagged_pair<tag::in(safe_iterator_t<IRng>), tag::out(safe_iterator_t<ORng>)>
uninitialized_move(IRng&& irng, ORng&& orng);
\end{codeblock}

\pnum
\effects Equivalent to:
\begin{codeblock}
    return uninitialized_move(begin(irng), end(irng), begin(orng), end(orng));
\end{codeblock}

\pnum
\remarks If an exception is thrown, some objects in the range \range{begin(rng)}{end(rng)} are left in a valid, but
unspecified state.

\pnum
\requires \tcode{orng} shall not be in the range \tcode{irng}.
} %% \color{addclr}

{\color{remclr}
\begin{codeblock}
template <class InputIterator, class Size, class ForwardIterator>
  pair<InputIterator, ForwardIterator>
    uninitialized_move_n(InputIterator first, Size n, ForwardIterator result);
\end{codeblock}

\setcounter{Paras}{2}
\pnum
\effects Equivalent to:
\begin{codeblock}
        for (; n > 0; ++result, (void) ++first, --n)
          ::new (static_cast<void*>(addressof(*result)))
            typename iterator_traits<ForwardIterator>::value_type(std::move(*first));
        return {first,result};
\end{codeblock}
} %% \color{remclr}

{\color{addclr}
\begin{codeblock}
template <InputIterator I, @\xname{NoThrowForwardIterator}@ O>
requires
  Constructible<value_type_t<O>, rvalue_reference_t<I>>
tagged_pair<tag::in(I), tag::out(O)>
  uninitialized_move_n(I first, difference_type_t<I> n, O result);
\end{codeblock}

\pnum
\effects Equivalent to:
\begin{codeblock}
    auto t = uninitialized_move(make_counted_iterator(first, n),
                                default_sentinel{}, result).base();
    return {t.in().base(), t.out()};
\end{codeblock}
} %% \color{addclr}

\setcounter{Paras}{3}
\pnum
\remarks If an exception is thrown, some objects in the range \range{first}{std::next(first, n)}
are left in a valid\added{,} but unspecified state.

{\color{addclr}
\setcounter{Paras}{5}
\pnum
\requires \tcode{result} shall not be in the range \range{first}{std::next(first, n)}.
} %% \color{addclr}

\rSec3[uninitialized.fill]{\tcode{uninitialized_fill}}
{\color{remclr}
\begin{codeblock}
template <class ForwardIterator, class T>
  void uninitialized_fill(ForwardIterator first, ForwardIterator last,
                          const T& x);
\end{codeblock}
} %% \color{remclr}

{\color{addclr}
\begin{codeblock}
template <@\xname{NoThrowForwardIterator}@ I, Sentinel<I> S, typename T>
requires
  Constructible<value_type_t<I>, const T&>
I uninitialized_fill(I first, S last, const T& x);
\end{codeblock}
} %% \color{addclr}

\pnum
\effects Equivalent to:
\begin{codeblock}
    for (; first != last; ++first) {
      ::new @\added{(const_cast<void*>}@(static_cast<@\added{const volatile }@void*>(addressof(*first)))@\added{)}@
        @\changed{typename iterator_traits<ForwardIterator>::value_type}{remove_reference_t<reference<I{>}>}@(x);
    }
    return first;
\end{codeblock}

{\color{addclr}
\begin{codeblock}
template <@\xname{NoThrowForwardRange}@ Rng, typename T>
requires
  Constructible<value_type_t<iterator_t<Rng>>, const T&>
safe_iterator_t<Rng> uninitialized_fill(Rng&& rng, const T& x)
\end{codeblock}

\pnum
\effects Equivalent to:
\begin{codeblock}
    return uninitialized_fill(begin(rng), end(rng), x);
\end{codeblock}
} %% \color{addclr}

{\color{remclr}
\begin{codeblock}
template <class ForwardIterator, class Size, class T>
  ForwardIterator uninitialized_fill_n(ForwardIterator first, Size n, const T& x);
\end{codeblock}

\setcounter{Paras}{1}
\pnum
\effects As if by:
\begin{codeblock}
        for (; n--; ++first)
          ::new (static_cast<void*>(addressof(*first)))
            typename iterator_traits<ForwardIterator>::value_type(x);
        return first;
\end{codeblock}
} %% \color{remclr}

{\color{addclr}
\begin{codeblock}
template <@\xname{NoThrowForwardIterator}@ I, typename T>
requires
  Constructible<value_type_t<I>, const T&>
I uninitialized_fill_n(I first, const difference_type_t<I> n, const T& x);
\end{codeblock}

\setcounter{Paras}{3}
\pnum
\effects Equivalent to:
\begin{codeblock}
    return uninitialized_fill(make_counted_iterator(first, n), default_sentinel{}, x).base();
\end{codeblock}
} %% \color{addclr}

\rSec3[specialized.destroy]{\tcode{destroy}}
{\color{remclr}
\begin{codeblock}
template <class T>
  void destroy_at(T* location);
\end{codeblock}
} %% \color{remclr}

{\color{addclr}
\begin{codeblock}
template <Destructible T>
void destroy_at(T* location) noexcept;
\end{codeblock}
} %% \color{addclr}

\setcounter{Paras}{0}
\pnum
\effects Equivalent to:
\begin{codeblock}
    location->~T();
\end{codeblock}

{\color{remclr}
\begin{codeblock}
template <class ForwardIterator>
  void destroy(ForwardIterator first, ForwardIterator last);
\end{codeblock}
} %% \color{remclr}

{\color{addclr}
\begin{codeblock}
template <__NoThrowInputIterator I, __NoThrowSentinel<I> S>
requires
  Destructible<value_type_t<I>>
I destroy(I first, S last) noexcept;
\end{codeblock}
} %% \color{addclr}

\pnum
\effects Equivalent to:
\begin{codeblock}
    for (; first != last; ++first)
      destroy_at(addressof(*first));
    @\added{return first;}@
\end{codeblock}

{\color{addclr}
\begin{codeblock}
template <__NoThrowInputRange Rng>
requires
  Destructible<value_type_t<iterator_t<Rng>>>
safe_iterator_t<Rng> destroy(Rng&& rng) noexcept;
\end{codeblock}

\pnum
\effects Equivalent to:
\begin{codeblock}
    return destroy(begin(rng), end(rng));
\end{codeblock}
} %% \color{addclr}

{\color{remclr}
\begin{codeblock}
template <class ForwardIterator, class Size>
  ForwardIterator destroy_n(ForwardIterator first, Size n);
\end{codeblock}

\setcounter{Paras}{2}
\pnum
\effects Equivalent to:
\begin{codeblock}
      for (; n > 0; (void)++first, --n)
          destroy_at(addressof(*first));
        return first;
\end{codeblock}
} %% \color{remclr}

{\color{addclr}
\begin{codeblock}
template <__NoThrowInputIterator I>
requires
  Destructible<value_type_t<I>
I destroy_n(I first, difference_type_t<I> n) noexcept;
\end{codeblock}

\pnum
\effects Equivalent to:
\begin{codeblock}
    return destroy(make_counted_iterator(first, n), default_sentinel{}).base();
\end{codeblock}
} %% \color{addclr}