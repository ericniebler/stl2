%!TEX root = p1033.tex
\normannex{depr}{Compatibility features}

\rSec1[depr.general]{General}

\pnum
This Clause describes features of this document that are specified for compatibility with existing
implementations.

\rSec1[depr.memory.range-and-a-half]{Memory specialisation range-and-a-half algorithms}
{\color{addclr}
\pnum
The following algorithm signatures are deemed unsafe, and are proposed to be deprecated.

\begin{codeblock}
template <InputIterator I, Sentinel<I> S, NoThrowForwardIterator O>
requires
  Constructible<value_type_t<O>, reference_t<I>> &&
tagged_pair<tag::in(I), tag::out(O)>
  uninitialized_copy(I first, S last, O result);
\end{codeblock}

\pnum
\effects Equivalent to:
\begin{codeblock}
    for (; first != last; ++result, (void)++first) {
      ::new (const_cast<void*>(static_cast<const volatile void*>(addressof(*result))))
        remove_reference_t<reference_t<O>>(*first);
    }
    return {first, result};
\end{codeblock}

\pnum
\requires \range{result}{next(result, distance(first, last))} shall not overlap with
\range{first}{last}.

\ednote{This paragraph applies to the algorithms already in the International Standard as well.}

\begin{codeblock}
template <InputRange Rng, NoThrowForwardIterator O>
  requires
Constructible<value_type_t<O>, reference_t<iterator_t<Rng>>>
  tagged_pair<tag::in(safe_iterator_t<Rng>), tag::out(O)>
uninitialized_copy(Rng&& rng, O result);
\end{codeblock}

\pnum
\effects Equivalent to:
\begin{codeblock}
   return uninitialized_copy(begin(rng), end(rng), result);
\end{codeblock}

\pnum
\requires \range{result}{next(result, distance(rng))} shall not overlap with
\range{begin(rng)}{end(rng)}.

\begin{codeblock}
template <InputIterator I, Sentinel<I> S, __NoThrowForwardIterator O>
  requires
Constructible<value_type_t<O>, rvalue_reference_t<I>>
  tagged_pair<tag::in(I), tag::out(O)>
uninitialized_move(I first, S last, O result);
\end{codeblock}

\pnum
\effects Equivalent to:
\begin{codeblock}
   for (; first != last; ++result, (void)++first) {
     ::new (const_cast<void*>(static_cast<const volatile void*>(addressof(*result))))
       remove_reference_t<reference_t<O>>(std::move(*first));
   }
   return {first, result};
\end{codeblock}

\pnum
\remarks If an exception is thrown, some objects in the range \range{first}{last} are left in a
valid but unspecified state.

\pnum
\requires \range{result}{next(result, distance(first, last))} shall not overlap with
\range{first}{last}.

\ednote{This paragraph applies to the algorithms already in the International Standard as well.}

\begin{codeblock}
template <InputRange Rng, NoThrowForwardIterator O>
   requires
Constructible<value_type_t<O>, reference_t<iterator_t<Rng>>>
   tagged_pair<tag::in(safe_iterator_t<Rng>), tag::out(O)>
uninitialized_copy(Rng&& rng, O result);
\end{codeblock}

\pnum
\effects Equivalent to:
\begin{codeblock}
   return uninitialized_copy(begin(rng), end(rng), result);
\end{codeblock}

\pnum
\remarks If an exception is thrown, some objects in the range \range{begin(rng)}{end(rng)} are left
in a valid but unspecified state.

\pnum
\requires \range{result}{next(result, distance(rng))} shall not overlap with
\range{begin(rng)}{end(rng)}.
} %% \color{addclr}
