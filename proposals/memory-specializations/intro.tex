\rSec0[intro]{General}
\rSec1[intro.scope]{Scope}

\pnum
This proposal describes functions compatible with the Ranges Technical
Specification~(\ref{intro.refs}) that shall not be ratified prior to the
publication of the TS. These extensions change and add to the existing
memory specialisation library facilities found in N4606~(\ref{intro.refs}).

\rSec1[intro.refs]{References}
This document requires some familiarity with the following documents.

\begin{itemize}
\item JTC1/SC22/WG21 N4606,   \doccite{Working Draft, Standard for Programming Language \Cpp}
\item ISO/IEC TS 19217:2015,  \doccite{Programming Languages - \Cpp Extensions for Concepts}
\item JTC1/SC22/WG21 D0459r0, \doccite{\Cpp Extensions for Ranges, Speculative Combined Proposal Document}
\end{itemize}

N4606 is herein called the \defn{\Cpp Standard Draft}, D0459r0 is called the \defn{Ranges TS Draft},
and ISO/IEC TS 19217:2015 is called the \defn{Concepts TS}.

\rSec1[intro.compliance]{Implementation compliance}

\pnum
Similarly to the Ranges TS, conformance requirements are the same as those described in
\ref{intro.compliance} in the \Cpp Standard Draft.
\enternote
Conformance is defined in terms of the behaviour of programs.
\exitnote

\rSec1[intro.namespace]{Namespaces, headers, and modifications to standard classes}

\pnum
All components described in this document are declared in namespace \tcode{std::experimental::ranges::v2}.

\ednote{The following text is taken from the
Ranges TS Draft and edited to reflect the fact that much of this document
is suggesting parallel constrained facilities that are specified as diffs against
the existing unconstrained facilities in namespace \tcode{std}.}

\pnum
The \Cpp Standard Draft, N4606, together with the Concepts TS and the Ranges
TS Draft, provide important context and specification for this paper. This
document is written as a set of changes against N4606. Sections are copied
verbatim and modified so as to define similar but different components in
namespace \tcode{std::experimental::ranges::v2}. Effort was made to keep
chapter and section numbers the same as in N4606 for the sake of easy
cross-referencing with the understanding that section numbers will change in
the final draft of a future, yet-to-be proposed \doccite{Ranges 2 TS}.

\pnum
Instructions to modify or add paragraphs are written as explicit instructions.
Modifications made to existing text from the International Standard use
\added{green colouring} to represent added text and \removed{red colouring} to
represent deleted text. Text in \newtxt{gold colouring} is used to denote text that
was added since the previous working draft of this proposal, and \oldtxt{purple colouring}
denotes text removed.

\pnum
References to other entities described in this document are assumed to be
qualified with \tcode{std::experimental::ranges::}, and references to entities
described in the \Cpp Standard Draft are assumed to be qualified with \tcode{std::}.

\pnum
This document specifically describes changes to \tcode{<memory>}. As such, new content can be found
in the \tcode{<experimental/ranges/memory>} header, as described in the Ranges TS Draft.

\begin{floattable}{Proposal headers}{tab:intro.headers}{l}
\topline
\tcode{<experimental/ranges/memory>}\\
\bottomline
\end{floattable}