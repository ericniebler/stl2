%!TEX root = stl2-ts.tex
\infannex{diff}{Compatibility}

\rSec1[diff.cpp]{\Cpp and Ranges}

\pnum
This section details the known breaking changes likely to effect user code when being ported to the
version of the Standard Library described in this document.

\rSec2[diff.cpp.algo_return]{Algorithm Return Types}

\pnum
The algorithms described in this document permit the type of the end sentinel to differ from the
type of the begin iterator. This is so that the algorithms can operate on ranges for which the
physical end position is not yet known.

\pnum
The physical end position of the input range is determined during the execution of many of the
algorithms. Rather than lose that potentially useful information, the design presented here has such
algorithms return the iterator position of the end of the range. In many cases, this is a breaking
change. Some algorithms that return iterators in today's STL are changed to return pairs, and
algorithms that return pairs today are changed to return tuples. This is likely to be the most
noticeable breaking change.

\pnum
Alternate designs that were less impactful were considered and dismissed. See Section 3.3.6 in
N4128~(\cite{niebler2014}) for a discussion of the issues.

\rSec2[diff.cpp.constraints]{Stronger Constraints}

\pnum
In this proposal, many algorithms and utilities get stricter type checking. For example, algorithms
constrained with \tcode{LessThanComparable} today are constrained by \tcode{StrictTotallyOrdered} in this
document. This concept requires types to provide \textit{all} the relational operators, not just
\tcode{operator<}.

\pnum
The use of coarser-grained, higher-level concepts in algorithm constraints is to make the type
checks more semantic in nature and less syntactic. It also has the benefit of being less verbose
while giving algorithm implementors greater implementation freedom. This approach is in contrast to
the previous effort to add concepts to the Standard Library in the C++0x timeframe, which saw a
proliferation of small, purely syntactic concepts and algorithm constraints that merely restated the
algorithms' implementation details more verbosely in the algorithms' function signatures.

\pnum
The potential for breakage must be carefully weighed against the integrity and complexity of the
constraints system. The coarseness of the concepts may need to change in response to real-world
usage.

\rSec2[diff.cpp.functional]{Constrained Functional Objects}

\pnum
The algorithm design described in this document assumes that the function objects
\tcode{std::equal_to} and \tcode{std::less} get constraints added to their function call operators.
(The former is constrained with \tcode{Equality\-Comparable} and the latter with
\tcode{StrictTotallyOrdered}). Similar constraints are added to the other function objects
in \tcode{<functional>}. As with the
coarsely-grained algorithm constraints, these function object constraints are likely to cause
user code to break.

\pnum
Real-world experience is needed to assess the seriousness of the breakage. From a correctness point
of view, the constraints are logical and valuable, but it's possible that for the sake of
compatibility we provide both constrained and unconstrained functional objects.

\rSec2[diff.cpp.defaultconstruct]{Iterators and Default-Constructibility}

\pnum
In today's STL, iterators need not be default-constructible. The \tcode{Iterator} concept described
in this document requires default-constructibility. This could potentially cause breakage in users'
code. Also, it makes the implementation of some types of iterators more complicated. Any iterator
that has members that are not default constructible (e.g., an iterator that contains a lambda that
has captured by reference) must take special steps to provide default-constructibility (e.g.,
by wrapping non-default-constructible types in something like \tcode{std::optional}, as specified
in the \Cpp17 Working Draft N4618 \S20.6). This can weaken class invariants.

\pnum
The guarantee of default-constructibility simplifies the implementation of much iterator- and
range-based code that would otherwise need to wrap iterators in \tcode{std::optional}. But the
needs of backward-compatibility, the extra complexity to iterator implementors, and the weakened
invariants may prove to be too great a burden.

\pnum
We may in fact go even farther and remove the requirement of default-constructibility from the
\tcode{Semi\-reg\-ul\-ar} concept. Time and experience will give us guidance here.

\rSec2[diff.cpp.iteratortraits]{iterator_traits cannot be specialized}

\pnum
In this STL design, \tcode{iterator_traits} changes from being a class template to being an
alias template. This is to intentionally break any code that tries to specialize it. In its place
are the three class templates \tcode{difference_type}, \tcode{value_type}, and
\tcode{iterator_category}. The need for this traits balkanization is because the associated types
belong to separate concepts: \tcode{difference_type} belongs to \tcode{WeaklyIncrementable};
\tcode{value_type} belongs to \tcode{Readable}; and \tcode{iterator_category} belongs to
\tcode{Input\-Iter\-at\-or}.

\pnum
This breakage is intentional and inherent in the decomposition of the iterator concepts established
by the Palo Alto report~(\cite{palo-alto}).

\rSec1[diff.n3351]{Ranges and the Palo Alto TR (N3351)}

\pnum
The Palo Alto report~(\cite{palo-alto}) presents a comprehensive design for the Standard Template
Library constrained with concepts. It served both as a basis for the Concepts Lite language feature
and for this document. However, this document diverges from the Palo Alto report in small ways. The
differences are in the interests of backwards compatability, to avoid confusing a large installed
base of programmers already familiar with the STL, and to keep the scope of this document as small
as possible. This section describes the ways in which the two suggested designs differ.

\rSec2[diff.n3351.sentinels]{Sentinels}

\pnum
In the design presented in this document, the type of a range's end delimiter may differ from the
iterator representing the range's start position. The reasons for this change are described in
N4128~(\cite{niebler2014}). This causes a number of differences from the Palo Alto report:

\begin{itemize}
\item The algorithms get an additional constraint for the sentinel.
\item The return types of the algorithms are changed as described above~(\ref{diff.cpp.algo_return}).
\item Some algorithms have operational semantics that require them to know the
physical end position (e.g., \tcode{reverse}). Those algorithms must make an \bigoh{N} probe for
the end position before proceeding. This does not change the operational semantics of any code that
is valid today (the probe is unnecessary when the types of the begin and end are the
same), and even when the probe is needed, in no cases does this change the compexity guarantee of
any algorithm.
\end{itemize}

\rSec2[diff.n3351.invok_proj]{Invocables and Projections}

\pnum
Adobe's Source Libraries~\cite{ASL} pioneered the use of \techterm{callables} and
\techterm{projections} in the standard algorithms. Invocables let users pass member pointers
where the algorithms expect callables, saving users the trouble of using a binder or a lambda.
Projections are extra optional arguments that give users a way to trivially transform input data
on the fly during the execution of the algorithms. Neither significantly changes the operational
semantics of the algorithms, but they do change the form of the algorithm constraints. To deal with
the extra complexity of the constraints, the design presented here adds higher-level composite
concepts for concisely expressing the necessary relationships between callables, iterators, and
projections.

\rSec2[diff.n3351.distance_type]{No Distinct DistanceType Associated Type}

\pnum
In the Palo Alto report, the \tcode{WeaklyIncrementable} concept has an associated type called
\tcode{DistanceType}, and the \tcode{RandomAccessIterator} concepts adds another associated type
called \tcode{DifferenceType}. The latter is required to be convertible to the former, but they are
not required to be the same type. (\tcode{DifferenceType} is required to be a signed integral type,
but \tcode{DistanceType} need not be signed.) Although sensible from a soundness point of view,
the author of this document feels this is potentially a rich source of confusion. This document hews
closer to the current standard by having only one associated type, \tcode{DifferenceType}, and
requiring it to be signed.

\rSec2[diff.n3351.distance_algo]{Distance Primitive is \bigoh{1} for Random Access Iterators}

\pnum
In the Palo Alto report, the \tcode{distance} iterator primitive for computing the distance from one
iterator position to another is not implemented in terms of \tcode{operator-} for random access
iterators. \tcode{distance}, according to the report, should always be \bigoh{N}. It reads:

\begin{quote}
The standard mandates a different definition for random access iterators:
\tcode{distance(i, j) == j - i}. We see this as a specification error; the guarantees of the
\tcode{distance} operation have been weakened for an iterator specialization.

In our design, we consider the two operations to be distinct.
\end{quote}

The design presented in this document keeps the specialization for random access iterators. To do
otherwise would be to silently break complexity guarantees in an unknown amount of working code.

To address the concern about weakened guarantees of the \tcode{distance} primitive, the design
presented here requires that random access iterators model
\tcode{SizedSentinel}~(\ref{iterators.sizedsentinel}). The \tcode{SizedSentinel}
concept requires that \tcode{b - a} return the number of times \tcode{a} would have to be
incremented to make it compare equal to \tcode{b}. Any type purporting to be a random access
iterator that fails to meet that requirement is by definition not a valid random access iterator.

\rSec2[diff.n3351.output_iters]{Output Iterators}

\pnum
The Palo Alto report does not define concepts for output iterators, making do with
\tcode{WeaklyIncrementable}, \tcode{Writable}, and (where needed) \tcode{EqualityComparable}. The
author of this document sees little downside to grouping these into the familiar
\tcode{OutputIterator} concept. Even if not strictly needed, its absence would be surprising.

\rSec2[diff.n3351.no_eop_algos]{No Algorithm Reformulations}

\pnum
Between the standardization of the Standard Library and the Palo Alto report, much new research was
done to further generalize the standard algorithms
(see ``Element of Programming'', Stepanov, McJones~\cite{Stepanov:2009:EP:1614221}). The algorithms
presented in The Palo Alto report reflect the results of that research in the algorithm constraints,
some of which (e.g., \tcode{sort}, \tcode{inplace_merge}) take iterators with weaker categories than
they do in the current standard. The design presented in this document does not reflect those
changes. Although those changes are desirable, generalizing the algorithms as described in The Palo
Alto report feels like it would be best done in a separate proposal.
